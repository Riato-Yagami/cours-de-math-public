\documentclass[aspectratio=169, usenames,dvipsnames,xcolor=table]{beamer}
\usepackage{amsmath}
\usepackage{amsthm}
\usepackage{amssymb}
\usepackage{graphicx}
\usepackage{dashundergaps}
\usepackage{array}
\usepackage{multicol}
\usepackage{wrapfig}
\usepackage{numprint}
\usepackage{ulem}
\usepackage{hyperref}
\usepackage{mathrsfs}
\usepackage{mathtools}
\usepackage[many]{tcolorbox}
\usepackage{xparse}
\usepackage{float}
\usepackage{lipsum}
\usepackage{pgf}
\usepackage{ifthen}
\usepackage{caption}
\usepackage{tikz}
\usepackage{tkz-tab}
\usepackage{xifthen}
\usepackage{listings}
\usepackage[linesnumbered,vlined,boxed]{algorithm2e}
\usepackage[squaren,Gray]{SIunits}
\usepackage{animate}
\usepackage{eurosym}
\usepackage{tkz-euclide}
\usepackage{etoolbox}
\usepackage{textcase}
\usepackage{adjustbox}
% \usepackage{mathabx}

\usepackage{tabularx}
\usepackage{scratch3}

\usepackage{multicol}
\usepackage{multirow}
% \usepackage{appendix}
% \usepackage[
%     backend=biber,        % compilateur par défaut pour biblatex
%     sorting=nyt,          % trier par nom, année, titre
%     citestyle=authoryear, % style de citation auteur-année
%     bibstyle=alphabetic,  % style de bibliographie alphabétique
% ]{biblatex}

% \usepackage[T1]{fontenc}

% \usepackage{pifont}

% \usepackage[squaren,Gray]{SIunits}

% BREVET

% \usepackage{makeidx}
% \usepackage{fancybox}
% \usepackage{tabularx}
% \usepackage[normalem]{ulem}
% \usepackage{pifont}
% \usepackage{lscape}
% \usepackage{diagbox}
% \usepackage{multido}
% \usepackage[dvipsnames]{pstricks}
% \usepackage{pst-plot,pst-text,pst-tree,pstricks-add}
% \usepackage{textcomp}
% \usepackage{scratch3}
% \usepackage[T1]{fontenc}
% \usepackage{fourier}
% \usepackage[french]{babel}
% \usepackage{pstricks}

% \usepackage[scaled=0.875]{helvet}
% \usepackage{pst-plot,pst-text,pst-tree,pstricks-add}

% hyperref

\hypersetup{
    colorlinks=true,       % false: boxed links; true: colored links
    linkcolor=red,          % color of internal links (change box color with linkbordercolor)
    citecolor=green,        % color of links to bibliography
    filecolor=magenta,      % color of file links
    urlcolor=blue,          % color of external links
    urlbordercolor=blue,    % borders of external links
    linkbordercolor=red,    % borders of internal links
    pdfborderstyle={/S/U/W 1}% border style will be underline of width 1pt
}

\frenchbsetup{StandardItemLabels=true}

% listings

\lstdefinestyle{pythonStyle}{
    language=Python,
    basicstyle=\ttfamily\small,  % Adjust the font and size
    commentstyle=\itshape\color{green!40!black},
    keywordstyle=\bfseries\color{violet},
    numbers=left,
    numberstyle=\tiny\color{gray},
    frame=single,
    breaklines=true,
    breakatwhitespace=true,
    tabsize=4,
    captionpos=b,
    identifierstyle=\color{Blue},
}

\lstset{style=pythonStyle}

% algorithm2e

\SetNlSty{}{}{.}
\SetKwInput{KwRes}{R\'esultat}%
\SetKwIF{Si}{SinonSi}{Sinon}{si}{alors}{sinon si}{sinon}{fin si}%
\SetKwFor{Tq}{tant que}{faire}{fin tq}%

% GLOBAL VARIABLES %%%
\graphicspath{{images}}
\def\cwidth{4cm}
\def\tspace{0.5cm}

% BOOLEAN %%%
\newboolean{anwser}
\newboolean{demonstration}
\newboolean{boxedProperties}
\newboolean{showID}
\newboolean{parenthisedID}
\newboolean{animated}
\newboolean{outline}

\setboolean{anwser}{false}
\setboolean{demonstration}{true}
\setboolean{parenthisedID}{true}
\setboolean{showID}{true}
\setboolean{boxedProperties}{false} % false = edge
\setboolean{outline}{false}

\def\DefinitionColor{PineGreen}
\def\PropertyColor{Blue}
\def\TheoremColor{Plum}

\def\SectionColor{Red}
\def\SubSectionColor{Green}

\setboolean{animated}{true}

% Switch implementation
\newboolean{default}
\newcommand{\case}{}
\newcommand{\default}{}

\newenvironment{switch}[1]{%
    \setboolean{default}{true}
    \renewcommand{\case}[2]{\ifthenelse{\equal{#1}{##1}}{%
        \setboolean{default}{false}##2}{}}%
    \renewcommand{\default}[1]{\ifthenelse{\boolean{default}}{##1}{}}
}{}

% SECTIONS
\input{header/command/sections.tex}

% ANSWERS
\newlength{\parline}
\newlength{\paroutindent}
\newlength{\lineheight}
\setlength{\lineheight}{\heightof{abcdefghijklmnoprstuvwxyz}}

\newcommand{\countlines}[1]{%
    \setlength{\paroutindent}{\expandafter\parindent}
    \setlength{\parline}{\heightof{\noindent\begin{minipage}{\linewidth}%
                \setlength{\parindent}{\paroutindent}#1\end{minipage}}}%
    \pgfmathparse{round(\parline / (0.9*\lineheight))}
    \newcount\linecount
    \pgfmathsetcount{\linecount}{\pgfmathresult}
}

\newcommand{\looptext}[2]{%
    \noindent
    \newcount\printcount
    \printcount=#2
    \loop
        #1
        \advance\printcount by -1
        \ifnum\printcount>0
    \repeat
}

\newcommand{\awsr}[1]{%
    \ifthenelse{\boolean{answer}}{
        \result{#1}
    }{
        \countlines{#1}
        \pgfmathsetcount{\linecount}{\linecount + 1}
        \noindent\hspace{-9pt}
        \looptext{
            \noindent\dotfill
    
        }{\the\linecount}
    }
}

\newcommand{\dottedLines}[1]{%
    \noindent\hspace{-9pt}%

    \looptext{%
        \noindent\dotfill%

    }{#1}
}

\newcommand{\result}[1]{\color{OrangeRed}#1 \color{black}%
}

% MATH
\input{header/command/math.tex}

% IMAGES
\input{header/command/image.tex}

% COMMANDS

\newcommand{\fsize}[1]{\fontsize{#1}{#1}\selectfont}

\NewDocumentCommand{\ifNotNull}{mmo}{
    \IfValueT{#1}{
        \ifx\relax#1\relax
            \IfValueT{#3}{
                #3
            }
        \else
            #2
        \fi
    }
}

\NewDocumentCommand{\ilink}{m g}{%
    \item
    \IfValueTF{#2}{\link{#1}{#2}}{\link{#1}}
}

\NewDocumentCommand{\link}{m g}{%
    \csn{#1}%
    \IfValueT{#2}{(#2)}%
}

\NewDocumentCommand{\TODO}{g}{%
    {\color{Red} $\rightarrow$ \textbf{TODO}
    \IfValueT{#1}{(#1)}}
    % \color{black}
}

\newcommand{\leconInfoBox}[2]{
    \textbf{#1 :}\vspace{-0.25cm}
        \begin{multicols}{2}
            \begin{itemize}[label=$\blacktriangleright$, font = \small \color{Red}]
                #2
            \end{itemize}
        \end{multicols}
        \vspace{-0.4cm}
}

% TCOLORBOX

\input{header/command/tcolorbox.tex}

\NewDocumentCommand{\leconInfo}{mooo}{
    \begin{infoBox}
        \leconInfoBox{Niveaux}{#1}
        \ifNotNull{#2}{
            \tcbline
            \leconInfoBox{Prérequis}{#2}
        }
        \ifNotNull{#3}{
            \tcbline
            \leconInfoBox{Thèmes}{#3}
        }
        \ifNotNull{#4}{
            \tcbline
            \textbf{Motivation :} 
            #4
        }
    \end{infoBox}
}

\NewDocumentCommand{\seanceInfo}{oooooooo}{
    \begin{infoBox}
        \vspace{-0.05cm}
        \begin{tcbitemize}[raster rows=1,raster columns=20,raster height=1.65cm,
            raster every box/.style={colframe=red!50!black,colback=red!10!white}]
            \tcbitem[raster multicolumn=6] \textbf{Date :} #1
            \tcbitem[raster multicolumn=10] \textbf{Séquence :} #2
            \tcbitem[raster multicolumn=4] \textbf{Séance :} #3
        \end{tcbitemize}
        \vspace{-0.25cm}
        \ifNotNull{#4}{\tcbline \textbf{Objectif :} #4}
        \ifNotNull{#5}{\tcbline \leconInfoBox{Classe(s)}{#5}}
        \ifNotNull{#6}{\tcbline \leconInfoBox{Prérequi(s)}{#6}}
        \ifNotNull{#7}{\tcbline \textbf{Séance précédente :} #7}
        \ifNotNull{#7}{\tcbline \leconInfoBox{Matériel(s)}{#8}}
    \end{infoBox}
}

\def\pDscr{\tcbitem[enhanced jigsaw, breakable,
    raster multicolumn=6]
}
\def\pMdlt{\tcbitem[enhanced jigsaw, breakable,
    raster multicolumn=11]
}
\def\pTime{\tcbitem[enhanced jigsaw, breakable,
    raster multicolumn=3, halign=center]
}

\newcommand{\prepRow}[3]{
    \tcbitem[raster multicolumn=20]
    \tcblower

    \pDscr #1
    \pMdlt #2
    \pTime #3
}

\newcommand{\prepTable}[1]{
    \begin{prepBox}
        \begin{tcbitemize}[enhanced jigsaw, breakable, raster rows=1,raster columns=20,raster height=1.1cm, halign=center,
            raster every box/.style={enhanced jigsaw, breakable, colframe=Blue!50!black,colback=Blue!10!white}]
            \pDscr \textbf{Descriptif}
            \pMdlt \textbf{Modalité}
            \pTime \textbf{Durée}
        \end{tcbitemize}
        \begin{tcbitemize}[enhanced jigsaw, breakable,
            raster equal height = rows, 
            raster columns=20, frame hidden,
            raster every box/.style={
                enhanced jigsaw, breakable,
                opacityback=0, valign=top, 
                size = tight
            }]
            #1
        \end{tcbitemize}
    \end{prepBox}
}

% TIKZ

\newcommand{\ctikz}[1]{
    \begin{center}
        \begin{tikzpicture}
            #1
        \end{tikzpicture}
    \end{center}
}

\newcommand{\axis}[1]{%Draw coordinate axes
    \draw[thin, -Stealth] (-0.5,0) -- (#1,0);% node[right] {$x$}; % x-axis
    \draw[thin, -Stealth] (0,-0.5) -- (0,#1);% node[above] {$y$}; % y-axis
}

\newcommand{\drawGrid}[3]{
    \foreach \n in {0,...,#1}
        \draw[line width = #3] (\n,0) -- (\n,#2);
    \foreach \n in {0,...,#2}
        \draw[line width = #3] (0,\n) -- (#1,\n);
}

\newcommand{\drawPoint}[4]{
    \node[shift={#4}, color = \pointColor] at (#2 - 0.5,#3 - 0.5) {#1};
    \draw[line width = \crossWidth, shift={#4}, color = \pointColor] (#2 - 0.25,#3) -- (#2 + 0.25,#3);
    \draw[line width = \crossWidth, shift={#4}, color = \pointColor] (#2,#3 - 0.25) -- (#2,#3 + 0.25);
}

% Tabular
\newcolumntype{C}[1]{>{\centering\arraybackslash}p{#1}}
\newcolumntype{M}[1]{>{\centering\arraybackslash}m{#1}}
\newcolumntype{K}{@{}m{0pt}@{}}

% GEOMETRY

% \newcommand{\restoregeometry}{def}

\newcommand{\multiColItemize}[2]{
    \begin{multicols}{#1}
        \begin{itemize}
            #2
        \end{itemize}
    \end{multicols}
}

\newcommand{\multiColEnumerate}[2]{
    \begin{multicols}{#1}
        \begin{enumerate}
            #2
        \end{enumerate}
    \end{multicols}
}

\makeatletter
\newcommand\pgfinvisible{\pgfsys@begininvisible}
\newcommand\pgfshown{\pgfsys@endinvisible}
\makeatother

\renewcommand*{\phantom}[1]{
    \pgfinvisible #1 \pgfshown
}

\newcounter{size}
\newcommand{\listSize}[1]{%
    \setcounter{size}{0}%
    \foreach \n in {#1}{\stepcounter{size}}%
    % \thesize
}

\newcounter{elemPos}
\newcommand{\listElement}[2]{
    \setcounter{elemPos}{0} % Start counting from 1
    \def\resultVal{0} % Default value
    \renewcommand*{\do}[1]{%
        \ifnumequal{\value{elemPos}}{#2}{%
            \def\resultVal{##1}%
            \listbreak% Break out of the loop
        }{}%
        \stepcounter{elemPos}%
    }
    % \docsvlist{#1}
    \expandafter\docsvlist\expandafter{#1} % Expand the list before passing it to \docsvlist
    \resultVal
}

% \NewDocumentCommand{\exoslide}{m O{10cm}}{
%     \slide{}{
%         \img{\imgf{#1}}[#2]
%     }
% }

\NewDocumentCommand{\exoSlide}{m O{10cm} O{1} O{} O{exo}}{%
    \slide{#5}{%
        \ifthenelse{\equal{#3}{1}}{\vspace{-0.5cm}}{\vspace{-1cm}}
        \def\exercices{\foreach \q in {#1}{\imgp{\q}[#2]\vspace{-0.5cm}}}
        \exo{#1}{\wideFrame[7em]{\bvspace{0.25cm}\avspace{-0.25cm}
            \ifthenelse{\equal{#3}{1}}{\exercices}
            {\begin{multicols}{#3}\exercices\end{multicols}}}
            \avspace{0.75cm}
        }[#4]
    }
}

\NewDocumentCommand{\exoList}{m O{} O{3}}{%
    \section*{Exercices}%
    \slide{EXERCICES}{
        \exo{#2}{
            \vspace{-0.25cm}
            \multiColEnumerate{#3}{
                \foreach \q in {#1}{
                    \item \q
                }
            }
        }
    }
}

\newcommand{\questions}[1]{
    \begin{enumerate}
        \foreach \q in {#1}{
            \item \q\\
            \vspace*{-0.45cm}
            \dottedLines{3}
        }
    \end{enumerate}
}

% Define a new boolean for checking if the section is starred
\newboolean{section@star}

\makeatletter
% Redefine \section and \section* to set the boolean
\let\old@section\section
\renewcommand{\section}{%
    \@ifstar
        {\setboolean{section@star}{true}\old@section*}
        {\setboolean{section@star}{false}\old@section}%
}
\makeatother

\newcommand{\qt}[1]{«\textit{#1}»}

\newcommand{\calc}[1]{\numexpr#1\relax}
\newcommand{\ncalc}[1]{\number\calc{#1}}
\newcommand{\pcalc}[1]{\numprint{\ncalc{#1}}}

\newcommand{\setgrade}[1]{
    \def\grade{#1}
    % \begin{switch}{#1}
    %     \case{6e}{\global\definecolor{gradeColor}{hex}{FA8072}}
    %     \default{
    %         Default
    %         \global\definecolor{gradeColor}{RGB}{200, 50, 50}
    %     }
    % \end{switch}
    \ifthenelse{\equal{#1}{6e}}{
        \definecolor{gradeColor}{HTML}{C6233D} % FA8072 in hex
    }{
    \ifthenelse{\equal{#1}{5e}}{
        \definecolor{gradeColor}{HTML}{088255}
    }{
    \ifthenelse{\equal{#1}{4e}}{
        \definecolor{gradeColor}{HTML}{1466A8}
    }{
    \ifthenelse{\equal{#1}{3e}}{
        \definecolor{gradeColor}{HTML}{844499}
    }{
        \definecolor{gradeColor}{RGB}{0, 0, 0}
    }}}}
}

\gdef\phase{}
\newcommand{\setPhase}[1]{%
    \begin{switch}{#1}
        \case{exo}{\gdef\phase{EXERCICES}}
        \case{cr}{\gdef\phase{COURS}}
        \case{qf}{\gdef\phase{QUESTIONS FLASH}}
        \case{dm}{\gdef\phase{DEVOIR MAISON}}
        \default{\gdef\phase{#1}}
    \end{switch}
}

\newcommand\csn[1]{\csname #1\endcsname}

\newcommand{\vect}[1]{\ensuremath{\overrightarrow{#1}}}
% \newcommand{\vect}[1]{\overrightarrow{\,\mathstrut#1\,}}
\newcommand{\m}[1]{\ensuremath{\mathbf{#1}}}
\newcommand\lm[2]{\lim_{#1\to#2}}

\def\eqv{\Leftrightarrow}
\def\ssi{si et seulement si }
\def\pt{pour tout }
\def\poly2{fonction polynôme du second degré }
\def\eq2{équation second degré }
\def\discr{b^2-4ac}

% MATH TEXT
\def\et{\textrm{ et }}
\def\si{\textrm{ si }}
\def\avec{\textrm{ avec }}
\def\car{\textrm{ car }}
\def\alors{\textrm{ alors }}
\def\ou{\textrm{ ou }}
\def\ona{\textrm{ on a }}

\def\iet{\shortintertext{et}}
\def\ialors{\shortintertext{alors}}
\def\idou{\shortintertext{d'où}}
\def\ior{\shortintertext{or}}
\def\iona{\shortintertext{on a}}

\def\studentinfo{
    \vspace*{-1cm}
    \begin{minipage}{0.35\linewidth}
        nom: \dotfill
    \end{minipage}
    \begin{minipage}{0.35\linewidth}
        prénom: \dotfill
    \end{minipage}
    \begin{minipage}{0.15\linewidth}
        classes: \dotfill
    \end{minipage}
    
    \noindent\hrulefill
}

% UNITS
\def\cm{\,\centi\meter}
\def\km{\,\kilo\meter}
\newcommand{\defl}[2]{%
    \expandafter\def\csname #1\endcsname{\href{#2}{#1}\space}%
}

% Page Eduscol
\defl{Eduscol Cycle 3}{https://eduscol.education.fr/251/mathematiques-cycle-3}
\defl{Eduscol Cycle 4}{https://eduscol.education.fr/280/mathematiques-cycle-4}
\defl{Eduscol Lycée Général et technologique}{https://eduscol.education.fr/1723/programmes-et-resources-en-mathematiques-voie-gt}
\defl{Eduscol Lycée Professionnel}{https://eduscol.education.fr/1793/programmes-et-resources-en-mathematiques-voie-professionnelle}

% Repères annuels
\defl{Cycle 2}{https://eduscol.education.fr/document/13972/download}
\defl{Cycle 3}{https://eduscol.education.fr/document/14026/download}
\defl{Cycle 4}{https://eduscol.education.fr/document/14080/download}

% Attendus de fin d'année
\defl{CM2}{https://eduscol.education.fr/document/14002/download}
\defl{6e}{https://eduscol.education.fr/document/14014/download}
\defl{5e}{https://eduscol.education.fr/document/14044/download}
\defl{4e}{https://eduscol.education.fr/document/14056/download}
\defl{3e}{https://eduscol.education.fr/document/14068/download}

% Programme de mathématiques
\defl{cycle 3}{https://eduscol.education.fr/document/50990/download}
\defl{cycle 4}{https://cache.media.education.gouv.fr/file/31/89/1/ensel714_annexe3_1312891.pdf}
\defl{2nd}{https://eduscol.education.fr/document/24553/download}
\defl{2nd STHR}{https://eduscol.education.fr/document/24556/download}
\defl{1re}{https://eduscol.education.fr/document/24565/download}
\defl{1re Technologique}{https://eduscol.education.fr/document/24559/download}
\defl{Terminale Option Spécialité}{https://eduscol.education.fr/document/24568/download}
\defl{Terminale Option Complémentaire}{https://eduscol.education.fr/document/24571/download}
\defl{Terminale Option Expertes}{https://eduscol.education.fr/document/24574/download}
\defl{Terminale Technologique}{https://eduscol.education.fr/document/23107/download}

% resources thématiques
\defl{Proportionnalité}{https://eduscol.education.fr/document/17281/download}
\defl{Probabilités}{https://eduscol.education.fr/document/17275/download}
\defl{Traitement des données}{https://eduscol.education.fr/document/17269/download}

\defl{Fonctions}{https://eduscol.education.fr/document/17287/download}
\defl{Fractions}{https://eduscol.education.fr/document/17239/download}
\defl{Nombres relatifs}{https://eduscol.education.fr/document/17245/download}
\defl{Puissances}{https://eduscol.education.fr/document/17251/download}
\defl{Divisibilité et nombres premiers}{https://eduscol.education.fr/document/17257/download}
\defl{Calcul littéral}{https://eduscol.education.fr/document/17263/download}

\defl{Grandeurs et mesures}{https://eduscol.education.fr/document/17293/download}
\defl{Algorithmique et programmation}{https://eduscol.education.fr/document/17311/download}

\defl{Suites}{https://eduscol.education.fr/document/24586/download}
\defl{Produit Scalaire}{https://eduscol.education.fr/document/24589/download}
\defl{Raisonnement et démonstration (seconde)}{https://eduscol.education.fr/document/24580/download}
\defl{Raisonnement et démonstrations (première)}{https://eduscol.education.fr/document/24583/download}

\def\jules{\href{https://juels.dev/}{Jules PESIN}}
\def\yuyu{\href{https://www.instagram.com/yuyuvrajav/}{@yuyuvraj}}

\defl{Utiliser les notions de géométrie planepour démontrer}{https://eduscol.education.fr/document/17305/download}

% Manuels
\def\dim{\href{https://www.editions-hatier.fr/livre/dimensions-mathematiques-6e-ed-2016-manuel-de-leleve-9782401020023}
    {Dimensions 6e (Ed. 2016)}
}

\definecolor{myriade}{HTML}{0F83B3} %#0F83B3
\def\my{\href{https://www.editions-bordas.fr/ouvrage/myriade-mathematiques-6e-manuel-de-leleve-ed-2021-9782047337752.html}
    {Myriade 6e (Ed. 2021)}
}

\def\mm{\href{https://www.editions-hatier.fr/livre/maths-monde-cycle-4-livre-1-volume-9782278083459}
    {Maths Monde cycle 4 (Ed. 2016)}
}

\def\mi{\href{https://www.enseignants.hachette-education.com/livres/mission-indigo-mathematiques-cycle-4-5e-4e-3e-livre-eleve-ed-2017-9782013953962}
    {Mission Indigo mathématiques cycle 4 éd. 2017}
}

% https://www.armitiere.com/livre/1833174-des-maths-ensemble-et-pour-chacun-5e-mise-en--jean-philippe-rouques-helene-stainer-canope-crdp-44
\def\dmeepcC{\href{https://publimath.univ-irem.fr/PCO10003}
    {Des maths ensemble et pour chacun 5e}
}

\def\dmeepcS{\href{https://www.reseau-canope.fr/notice/des-maths-ensemble-et-pour-chacun-6e.html}
    {Des maths ensemble et pour chacun 6e}
}

\NewDocumentCommand{\dmeepc}{m O{}}{%
    \href{https://www.reseau-canope.fr/notice/des-maths-ensemble-et-pour-chacun-6e.html}
    {Des maths ensemble et pour chacun #1e \ifNotNull{#2}{(p.#2)}}
}

\NewDocumentCommand{\sesa}{m m O{} O{}}{%
    \href{https://manuel.sesamath.net/numerique/index.php?ouvrage=cm#1_#2&page_gauche=#4}{%
    Sésamath #1e #2 \ifNotNull{#4}{(#3 p.#4)}
    }
}

\NewDocumentCommand{\iP}{m m O{} O{}}{%
    \href{https://www.iparcours.fr/ouvrages/ouvrages.php?ouvrage=Cahier#1#2}{%
    iParcours #1e #2 \ifNotNull{#4}{(#3 p.#4)}
    }
}

\NewDocumentCommand{\ching}{m m O{}}{%
    \href{https://chingmath.fr/#1eme/#2}{%
    Ching@Math #1e (\reverseKebabCase{#2}\ifNotNull{#3}{ E.#3})
    }
}

\NewDocumentCommand{\wiki}{m O{}}{%
    \def\ext{}%
    \ifNotNull{#2}{\def\ext{\##2}}%
    \href{https://fr.wikipedia.org/wiki/#1\ext}%
    {Wikipédia (\reverseSnakeCase{#1}%
    \ifNotNull{#2}{ {\scriptsize $\rightarrow$ \reverseSnakeCase{#2}}}%
    )}
}

\newcommand*{\prbltq}[1]{\href{https://www.problematheque-csen.fr/fiche-probleme/#1}{Problémathèque (\reverseKebabCase{#1})}}

\NewDocumentCommand{\rpmc}{O{}}{%
    \href{https://eduscol.education.fr/document/13132/download?attachment\#page=#1}{%
    La résolution de problèmes mathématiques au collège
    (p.%
    #1%
    % \directlua{tex.print((tonumber("#1") or 0) + 3)}
    )}%
}

% Attendus de fin d'année
\NewDocumentCommand{\afa}{m O{}}{
    \ifthenelse{\equal{#1}{CM2}}{
        \def\afalink{https://eduscol.education.fr/document/14002/download}
    }{
    \ifthenelse{\equal{#1}{6e}}{
        \def\afalink{https://eduscol.education.fr/document/14014/download}
    }{
    \ifthenelse{\equal{#1}{5e}}{
        \def\afalink{https://eduscol.education.fr/document/14044/download}
    }{
    \ifthenelse{\equal{#1}{4e}}{
        \def\afalink{https://eduscol.education.fr/document/14056/download}
    }{
    \ifthenelse{\equal{#1}{3e}}{
        \def\afalink{https://eduscol.education.fr/document/14068/download}
    }{
        \def\afalink{https://eduscol.education.fr/document/14014/download}
    }}}}}
    \def\page{}
    \ifNotNull{#2}{\def\page{(p.#2)}}
    \href{\afalink\#page=#2}{Attendus de fin d'année de #1 \page}
}

\def\ca{%
    \href{https://pedagogie.ac-strasbourg.fr/mathematiques/competitions/course-aux-nombres/}%
    {Course aux nombres}%
}

% Euclide https://www.pedagogie.ac-aix-marseille.fr/jcms/c_10743971/it/les-elements-d-euclide-traduction-par-oliver-byrne

\NewDocumentCommand{\eucl}{O{1804} O{}}{
    \ifthenelse{\equal{#1}{1632}}{ % 1632
        \def\trad{D. Henrion}
        \def\afalink{https://www.pedagogie.ac-aix-marseille.fr/upload/docs/application/pdf/2019-11/elements_euclide_-_denis_henrion.pdf}
    }{
    \ifthenelse{\equal{#1}{1804}}{ % 1804 traduction F. Peyrard
        \def\trad{F. Peyrard}
        \def\afalink{https://eduscol.education.fr/document/14014/download}
    }{}
    }
    \def\page{}
    \ifNotNull{#2}{\def\page{p.#2}}
    \href{\afalink\#page=#2}{Les Éléments d'Euclide (traduction de \trad \page)}
}
% 1632 https://www.pedagogie.ac-aix-marseille.fr/upload/docs/application/pdf/2019-11/elements_euclide_-_denis_henrion.pdf
% Logiciels
\newcommand{\defIconLink}[4]{% 1 text , 2 : color , 3 : icon , 4 : link
    \expandafter\def\csname #1\endcsname{%
        {\def\iconPath{}%
        \icon{#3} \textbf{\href{#4}{\color{#2}#1}}}
    }%
}

\newcommand{\cmdIconLink}[4]{% 1 text , 2 : color , 3 : icon , 4 : link
    \expandafter\NewDocumentCommand\csname cmd#1\endcsname{O{}}
    {%
        {\def\iconPath{}%
        \icon{#3} \textbf{\href{#4/##1}{\color{#2}#1}}}
    }%
}

\definecolor{capytale}{HTML}{1E293B} % #1E293B
\definecolor{capytale-2}{HTML}{F0F1F2} % #F0F1F2

\def\Capytale{%
    \href{https://capytale2.ac-paris.fr/~/my}{\shl{capytale}{capytale-2}{CAPYTALE}}%
}

\newcommand{\capytale}[1]{%
    \href{https://capytale2.ac-paris.fr/web/c/#1}{\shl{capytale}{capytale-2}{CAPYTALE \shl{capytale-2}{capytale}{#1}}}%\shl{capytale}{capytale-2}{CAPYTALE 
}

\definecolor{enc}{HTML}{1D3D6E} % #1D3D6E
\defIconLink{ENC}{enc}{ENC-Hauts-de-Seine}{https://enc.hauts-de-seine.fr/}

\definecolor{pronote}{HTML}{1A6E45} % #1A6E45
\defIconLink{Pronote}{pronote}{pronote}{https://0922247t.index-education.net/pronote/}

\definecolor{calc}{HTML}{00A500} % #00A500
\defIconLink{Calc}{calc}{libreOffice/calc/logo}{https://fr.libreoffice.org/discover/calc/}

\definecolor{geogebra}{HTML}{9693F7} % #9693F7
\defIconLink{Geogebra}{geogebra}{geogebra/logo}{https://www.geogebra.org/classic}
\cmdIconLink{Geogebra}{geogebra}{geogebra/logo}{https://www.geogebra.org/m}

\definecolor{scratch}{HTML}{FFAB19} % #FFAB19
\defIconLink{Scratch}{Orange}{scratch/logo}{https://scratch.mit.edu/projects/editor/}

% http://trucsmaths.free.fr/etymologie.htm

\newcommand{\dym}[1]{\def\ym{\href{#1}{Yvan Monka}}}

\captionsetup{labelformat=empty,labelsep=none}

% ANNE
\setboolean{boxedProperties}{true} % false = edge
\setboolean{parenthisedID}{false}
\setboolean{showID}{false}

\def\DefinitionColor{Red}
\def\PropertyColor{Red}
\def\TheoremColor{Red}

% TIKZ
\def\crossWidth{0.25mm}
\def\pointColor{blue}

\usepackage{bookmark}
% \usepackage{unicode-math}

% \usefonttheme[onlymath]{serif}

% \setmainfont{Libertinus Serif}
% \setmathfont{Libertinus Math}

\usetheme{Madrid}
% \usetheme{shadow}
% \usetheme{CambridgeUS}
% \usetheme{AnnArbor}
% \usecolortheme{spruce}
\usecolortheme{beaver}

\setbeamersize{
    text margin left=1.5cm,
    text margin right=1.5cm
}

% \setbeamertemplate{enumerate items}[square]
\setbeamertemplate{enumerate items}[default]
\def\authors{Jules PESIN}
\def\longTitle{long Title}
\def\shortTitle{short Title}
% \def\day{XX/XX/XX}

\title[\shortTitle]{\longTitle}
% \date{\day}

\newcommand{\slide}[2]{
    \begin{frame}
    \frametitle[#1]{#1}
        #2
    \end{frame}
}

\newcounter{sec}
% \stepcounter{sec}
\newcounter{subsec}
% \stepcounter{subsec}

\newcommand{\bchap}[1]{
    \color{Red} CHAPITRE : #1\color{black}\\
}

\newcommand{\bsec}[1]{
    \def\ssec{\color{Red} \Roman{sec}. #1\color{black}\\}
    \stepcounter{sec}
    \setcounter{subsec}{0}
}

\newcommand{\bsubsec}[1]{
    \def\ssubsec{\color{Green} \thesubsec) #1\color{black}\\}
    \stepcounter{subsec}
}

\newcommand{\palt}[2]{
    \alt<#1>{#2}{\phantom{#2}}
}

\newcounter{question}

\newcommand{\startQuestions}{
    \setcounter{question}{2}
}

\newcommand{\iquestion}[2]{
    \item $\question{#1}{#2}$
}

\newcommand{\question}[2]{
        #1 = \onslide<\thequestion->{#2}
        \stepcounter{question}
}

% \renewcommand{\question}[2]{
%         #1 = #2
% }


\newcommand{\disableAnimation}{
    \renewcommand{\question}[2]{
        ##1 = ##2
    }
    
    \renewcommand{\palt}[2]{
        ##2
    }
}

\newcommand{\shortAnimation}{
    \renewcommand{\question}[2]{
        ##1 = \onslide<2->{##2}
    }
}

\newcommand{\firstSlide}{
    \renewcommand{\question}[2]{
        ##1 =
    }

    \renewcommand{\palt}[2]{
        \phantom{##2}
    }
}


% \documentclass[a4paper, 12pt
% % , landscape
% ]{extarticle}
% \usepackage[top=1.5cm, bottom=2cm, left=2cm, right=2cm]{geometry}
\usepackage[dvipsnames, table]{xcolor}
\usepackage{lastpage}
\usepackage{fancyhdr}
\usepackage{titlesec}
\usepackage{enumitem}
\usepackage{longtable}
\usepackage{pdfpages}
% FANCYHDR

\setlength{\headheight}{18pt}
\fancyhead[C]{\normalsize \title}
\fancyhead[R]{}
\fancyhead[L]{}
\fancyfoot[L]{\authors}
\fancyfoot[C]{\textbf{Page \thepage/\pageref{LastPage}}}
\fancyfoot[R]{\date}

\fancypagestyle{firstpage}{
    \setlength{\headheight}{29pt}
    \fancyhead[C]{\LARGE \title}
}

\fancypagestyle{assignment}{
    \setlength{\headheight}{29pt}
    \fancyhead[C]{}
    \fancyhead[L]{\large \title}
    \fancyhead[R]{%
        \begin{tabular}{p{7.5cm}p{2.5cm}}%
            \normalsize nom:& \normalsize classe: \link{\grade}\_\\%
            \normalsize prénom:& \normalsize date:\\%
            % \normalsize date:\hspace*{3.5cm}%
        \end{tabular}%
    }
}

\fancypagestyle{empty}{
    \renewcommand{\headrulewidth}{0pt}
    \setlength{\headheight}{-10pt}
    \fancyhead[C]{}
    \fancyhead[R]{}
    \fancyhead[L]{}
    \fancyfoot[L]{}
    \fancyfoot[C]{}
    \fancyfoot[R]{}
}

\fancypagestyle{assignment-empty-foot}{
    \setlength{\headheight}{29pt}
    \fancyhead[C]{}
    \fancyhead[L]{\large \title}
    \fancyhead[R]{%
        \begin{tabular}{p{0.25\pdfpagewidth}p{0.15\pdfpagewidth}}%
            \normalsize nom:& \normalsize classe:\\%
            \normalsize prénom:& \normalsize date:\\%
            % \normalsize date:\hspace*{3.5cm}%
        \end{tabular}%
    }
    \fancyfoot[L]{}
    \fancyfoot[C]{}
    \fancyfoot[R]{}
}

\fancypagestyle{small}{
    \setlength{\headheight}{20pt}
    \fancyhead[C]{}
    \fancyhead[C]{\large \title}
    \fancyhead[L]{}
    \fancyhead[R]{}
    \fancyfoot[L]{}
    \fancyfoot[C]{}
    \fancyfoot[R]{}
}

\thispagestyle{firstpage}

% \fancyfoot[C]{\textbf{Page 1/1}}

\def\title{\theme}
\def\authors{Jules PESIN}

\pagestyle{fancy}

% \titleformat*{\section}{\small\bfseries}

\titleformat{\section}
{\normalfont\large\bfseries\color{\SectionColor}}{\thesection}{0.6em}{}

\titleformat{\subsection}
{\normalfont\normalsize\bfseries\color{\SubSectionColor}}{\thesubsection}{0.6em}{}

\titleformat{\subsubsection}
{\normalfont\small\bfseries}{\thesection}{0.6em}{}

\renewcommand{\theenumi}{\small\color{Blue}\arabic{enumi}}

\renewcommand{\labelenumii}{\scriptsize\color{RoyalBlue}\alph{enumii})}
% \renewcommand{\theenumii}{.\arabic{enumii}}
% \frenchbsetup{StandardItemLabels=true}
\renewcommand{\labelitemi}{$\color{Blue}.$}

% BEAMER CONVERSION

\newcommand{\bchap}[1]{\def\title{Chapitre: #1}}
\newcommand{\bsec}[1]{\section{#1}}
\newcommand{\bsubsec}[1]{\subsection{#1}}

\newcommand{\ssec}{}
\newcommand{\ssubsec}{}

\newcommand{\slide}[2]{#2}

\newcommand{\startQuestions}{}
\newcommand{\iquestion}[2]{\item $#1 = #2$}

\newcommand{\palt}[2]{#2}

\newcommand{\disableAnimation}{}
\newcommand{\shortAnimation}{}

\newcommand{\firstSlide}{
    \renewcommand{\iquestion}[2]{\item $##1 = \phantom{##2}$}
    \renewcommand{\palt}[2]{\phantom{##2}}
}



\usepackage{amsmath}
\usepackage{amsthm}
\usepackage{amssymb}
\usepackage{graphicx}
\usepackage{dashundergaps}
\usepackage{array}
\usepackage{multicol}
\usepackage{wrapfig}
\usepackage{numprint}
\usepackage{ulem}
\usepackage{hyperref}
\usepackage{mathrsfs}
\usepackage{mathtools}
\usepackage[many]{tcolorbox}
\usepackage{xparse}
\usepackage{float}
\usepackage{lipsum}
\usepackage{pgf}
\usepackage{ifthen}
\usepackage{caption}
\usepackage{tikz}
\usepackage{tkz-tab}
\usepackage{xifthen}
\usepackage{listings}
\usepackage[linesnumbered,vlined,boxed]{algorithm2e}
\usepackage[squaren,Gray]{SIunits}
\usepackage{animate}
\usepackage{eurosym}
\usepackage{tkz-euclide}
\usepackage{etoolbox}
\usepackage{textcase}
\usepackage{adjustbox}
% \usepackage{mathabx}

\usepackage{tabularx}
\usepackage{scratch3}

\usepackage{multicol}
\usepackage{multirow}
% \usepackage{appendix}
% \usepackage[
%     backend=biber,        % compilateur par défaut pour biblatex
%     sorting=nyt,          % trier par nom, année, titre
%     citestyle=authoryear, % style de citation auteur-année
%     bibstyle=alphabetic,  % style de bibliographie alphabétique
% ]{biblatex}

% \usepackage[T1]{fontenc}

% \usepackage{pifont}

% \usepackage[squaren,Gray]{SIunits}

% BREVET

% \usepackage{makeidx}
% \usepackage{fancybox}
% \usepackage{tabularx}
% \usepackage[normalem]{ulem}
% \usepackage{pifont}
% \usepackage{lscape}
% \usepackage{diagbox}
% \usepackage{multido}
% \usepackage[dvipsnames]{pstricks}
% \usepackage{pst-plot,pst-text,pst-tree,pstricks-add}
% \usepackage{textcomp}
% \usepackage{scratch3}
% \usepackage[T1]{fontenc}
% \usepackage{fourier}
% \usepackage[french]{babel}
% \usepackage{pstricks}

% \usepackage[scaled=0.875]{helvet}
% \usepackage{pst-plot,pst-text,pst-tree,pstricks-add}

% hyperref

\hypersetup{
    colorlinks=true,       % false: boxed links; true: colored links
    linkcolor=red,          % color of internal links (change box color with linkbordercolor)
    citecolor=green,        % color of links to bibliography
    filecolor=magenta,      % color of file links
    urlcolor=blue,          % color of external links
    urlbordercolor=blue,    % borders of external links
    linkbordercolor=red,    % borders of internal links
    pdfborderstyle={/S/U/W 1}% border style will be underline of width 1pt
}

\frenchbsetup{StandardItemLabels=true}

% listings

\lstdefinestyle{pythonStyle}{
    language=Python,
    basicstyle=\ttfamily\small,  % Adjust the font and size
    commentstyle=\itshape\color{green!40!black},
    keywordstyle=\bfseries\color{violet},
    numbers=left,
    numberstyle=\tiny\color{gray},
    frame=single,
    breaklines=true,
    breakatwhitespace=true,
    tabsize=4,
    captionpos=b,
    identifierstyle=\color{Blue},
}

\lstset{style=pythonStyle}

% algorithm2e

\SetNlSty{}{}{.}
\SetKwInput{KwRes}{R\'esultat}%
\SetKwIF{Si}{SinonSi}{Sinon}{si}{alors}{sinon si}{sinon}{fin si}%
\SetKwFor{Tq}{tant que}{faire}{fin tq}%

% GLOBAL VARIABLES %%%
\graphicspath{{images}}
\def\cwidth{4cm}
\def\tspace{0.5cm}

% BOOLEAN %%%
\newboolean{anwser}
\newboolean{demonstration}
\newboolean{boxedProperties}
\newboolean{showID}
\newboolean{parenthisedID}
\newboolean{animated}
\newboolean{outline}

\setboolean{anwser}{false}
\setboolean{demonstration}{true}
\setboolean{parenthisedID}{true}
\setboolean{showID}{true}
\setboolean{boxedProperties}{false} % false = edge
\setboolean{outline}{false}

\def\DefinitionColor{PineGreen}
\def\PropertyColor{Blue}
\def\TheoremColor{Plum}

\def\SectionColor{Red}
\def\SubSectionColor{Green}

\setboolean{animated}{true}

% Switch implementation
\newboolean{default}
\newcommand{\case}{}
\newcommand{\default}{}

\newenvironment{switch}[1]{%
    \setboolean{default}{true}
    \renewcommand{\case}[2]{\ifthenelse{\equal{#1}{##1}}{%
        \setboolean{default}{false}##2}{}}%
    \renewcommand{\default}[1]{\ifthenelse{\boolean{default}}{##1}{}}
}{}

% SECTIONS
\input{header/command/sections.tex}

% ANSWERS
\newlength{\parline}
\newlength{\paroutindent}
\newlength{\lineheight}
\setlength{\lineheight}{\heightof{abcdefghijklmnoprstuvwxyz}}

\newcommand{\countlines}[1]{%
    \setlength{\paroutindent}{\expandafter\parindent}
    \setlength{\parline}{\heightof{\noindent\begin{minipage}{\linewidth}%
                \setlength{\parindent}{\paroutindent}#1\end{minipage}}}%
    \pgfmathparse{round(\parline / (0.9*\lineheight))}
    \newcount\linecount
    \pgfmathsetcount{\linecount}{\pgfmathresult}
}

\newcommand{\looptext}[2]{%
    \noindent
    \newcount\printcount
    \printcount=#2
    \loop
        #1
        \advance\printcount by -1
        \ifnum\printcount>0
    \repeat
}

\newcommand{\awsr}[1]{%
    \ifthenelse{\boolean{answer}}{
        \result{#1}
    }{
        \countlines{#1}
        \pgfmathsetcount{\linecount}{\linecount + 1}
        \noindent\hspace{-9pt}
        \looptext{
            \noindent\dotfill
    
        }{\the\linecount}
    }
}

\newcommand{\dottedLines}[1]{%
    \noindent\hspace{-9pt}%

    \looptext{%
        \noindent\dotfill%

    }{#1}
}

\newcommand{\result}[1]{\color{OrangeRed}#1 \color{black}%
}

% MATH
\input{header/command/math.tex}

% IMAGES
\input{header/command/image.tex}

% COMMANDS

\newcommand{\fsize}[1]{\fontsize{#1}{#1}\selectfont}

\NewDocumentCommand{\ifNotNull}{mmo}{
    \IfValueT{#1}{
        \ifx\relax#1\relax
            \IfValueT{#3}{
                #3
            }
        \else
            #2
        \fi
    }
}

\NewDocumentCommand{\ilink}{m g}{%
    \item
    \IfValueTF{#2}{\link{#1}{#2}}{\link{#1}}
}

\NewDocumentCommand{\link}{m g}{%
    \csn{#1}%
    \IfValueT{#2}{(#2)}%
}

\NewDocumentCommand{\TODO}{g}{%
    {\color{Red} $\rightarrow$ \textbf{TODO}
    \IfValueT{#1}{(#1)}}
    % \color{black}
}

\newcommand{\leconInfoBox}[2]{
    \textbf{#1 :}\vspace{-0.25cm}
        \begin{multicols}{2}
            \begin{itemize}[label=$\blacktriangleright$, font = \small \color{Red}]
                #2
            \end{itemize}
        \end{multicols}
        \vspace{-0.4cm}
}

% TCOLORBOX

\input{header/command/tcolorbox.tex}

\NewDocumentCommand{\leconInfo}{mooo}{
    \begin{infoBox}
        \leconInfoBox{Niveaux}{#1}
        \ifNotNull{#2}{
            \tcbline
            \leconInfoBox{Prérequis}{#2}
        }
        \ifNotNull{#3}{
            \tcbline
            \leconInfoBox{Thèmes}{#3}
        }
        \ifNotNull{#4}{
            \tcbline
            \textbf{Motivation :} 
            #4
        }
    \end{infoBox}
}

\NewDocumentCommand{\seanceInfo}{oooooooo}{
    \begin{infoBox}
        \vspace{-0.05cm}
        \begin{tcbitemize}[raster rows=1,raster columns=20,raster height=1.65cm,
            raster every box/.style={colframe=red!50!black,colback=red!10!white}]
            \tcbitem[raster multicolumn=6] \textbf{Date :} #1
            \tcbitem[raster multicolumn=10] \textbf{Séquence :} #2
            \tcbitem[raster multicolumn=4] \textbf{Séance :} #3
        \end{tcbitemize}
        \vspace{-0.25cm}
        \ifNotNull{#4}{\tcbline \textbf{Objectif :} #4}
        \ifNotNull{#5}{\tcbline \leconInfoBox{Classe(s)}{#5}}
        \ifNotNull{#6}{\tcbline \leconInfoBox{Prérequi(s)}{#6}}
        \ifNotNull{#7}{\tcbline \textbf{Séance précédente :} #7}
        \ifNotNull{#7}{\tcbline \leconInfoBox{Matériel(s)}{#8}}
    \end{infoBox}
}

\def\pDscr{\tcbitem[enhanced jigsaw, breakable,
    raster multicolumn=6]
}
\def\pMdlt{\tcbitem[enhanced jigsaw, breakable,
    raster multicolumn=11]
}
\def\pTime{\tcbitem[enhanced jigsaw, breakable,
    raster multicolumn=3, halign=center]
}

\newcommand{\prepRow}[3]{
    \tcbitem[raster multicolumn=20]
    \tcblower

    \pDscr #1
    \pMdlt #2
    \pTime #3
}

\newcommand{\prepTable}[1]{
    \begin{prepBox}
        \begin{tcbitemize}[enhanced jigsaw, breakable, raster rows=1,raster columns=20,raster height=1.1cm, halign=center,
            raster every box/.style={enhanced jigsaw, breakable, colframe=Blue!50!black,colback=Blue!10!white}]
            \pDscr \textbf{Descriptif}
            \pMdlt \textbf{Modalité}
            \pTime \textbf{Durée}
        \end{tcbitemize}
        \begin{tcbitemize}[enhanced jigsaw, breakable,
            raster equal height = rows, 
            raster columns=20, frame hidden,
            raster every box/.style={
                enhanced jigsaw, breakable,
                opacityback=0, valign=top, 
                size = tight
            }]
            #1
        \end{tcbitemize}
    \end{prepBox}
}

% TIKZ

\newcommand{\ctikz}[1]{
    \begin{center}
        \begin{tikzpicture}
            #1
        \end{tikzpicture}
    \end{center}
}

\newcommand{\axis}[1]{%Draw coordinate axes
    \draw[thin, -Stealth] (-0.5,0) -- (#1,0);% node[right] {$x$}; % x-axis
    \draw[thin, -Stealth] (0,-0.5) -- (0,#1);% node[above] {$y$}; % y-axis
}

\newcommand{\drawGrid}[3]{
    \foreach \n in {0,...,#1}
        \draw[line width = #3] (\n,0) -- (\n,#2);
    \foreach \n in {0,...,#2}
        \draw[line width = #3] (0,\n) -- (#1,\n);
}

\newcommand{\drawPoint}[4]{
    \node[shift={#4}, color = \pointColor] at (#2 - 0.5,#3 - 0.5) {#1};
    \draw[line width = \crossWidth, shift={#4}, color = \pointColor] (#2 - 0.25,#3) -- (#2 + 0.25,#3);
    \draw[line width = \crossWidth, shift={#4}, color = \pointColor] (#2,#3 - 0.25) -- (#2,#3 + 0.25);
}

% Tabular
\newcolumntype{C}[1]{>{\centering\arraybackslash}p{#1}}
\newcolumntype{M}[1]{>{\centering\arraybackslash}m{#1}}
\newcolumntype{K}{@{}m{0pt}@{}}

% GEOMETRY

% \newcommand{\restoregeometry}{def}

\newcommand{\multiColItemize}[2]{
    \begin{multicols}{#1}
        \begin{itemize}
            #2
        \end{itemize}
    \end{multicols}
}

\newcommand{\multiColEnumerate}[2]{
    \begin{multicols}{#1}
        \begin{enumerate}
            #2
        \end{enumerate}
    \end{multicols}
}

\makeatletter
\newcommand\pgfinvisible{\pgfsys@begininvisible}
\newcommand\pgfshown{\pgfsys@endinvisible}
\makeatother

\renewcommand*{\phantom}[1]{
    \pgfinvisible #1 \pgfshown
}

\newcounter{size}
\newcommand{\listSize}[1]{%
    \setcounter{size}{0}%
    \foreach \n in {#1}{\stepcounter{size}}%
    % \thesize
}

\newcounter{elemPos}
\newcommand{\listElement}[2]{
    \setcounter{elemPos}{0} % Start counting from 1
    \def\resultVal{0} % Default value
    \renewcommand*{\do}[1]{%
        \ifnumequal{\value{elemPos}}{#2}{%
            \def\resultVal{##1}%
            \listbreak% Break out of the loop
        }{}%
        \stepcounter{elemPos}%
    }
    % \docsvlist{#1}
    \expandafter\docsvlist\expandafter{#1} % Expand the list before passing it to \docsvlist
    \resultVal
}

% \NewDocumentCommand{\exoslide}{m O{10cm}}{
%     \slide{}{
%         \img{\imgf{#1}}[#2]
%     }
% }

\NewDocumentCommand{\exoSlide}{m O{10cm} O{1} O{} O{exo}}{%
    \slide{#5}{%
        \ifthenelse{\equal{#3}{1}}{\vspace{-0.5cm}}{\vspace{-1cm}}
        \def\exercices{\foreach \q in {#1}{\imgp{\q}[#2]\vspace{-0.5cm}}}
        \exo{#1}{\wideFrame[7em]{\bvspace{0.25cm}\avspace{-0.25cm}
            \ifthenelse{\equal{#3}{1}}{\exercices}
            {\begin{multicols}{#3}\exercices\end{multicols}}}
            \avspace{0.75cm}
        }[#4]
    }
}

\NewDocumentCommand{\exoList}{m O{} O{3}}{%
    \section*{Exercices}%
    \slide{EXERCICES}{
        \exo{#2}{
            \vspace{-0.25cm}
            \multiColEnumerate{#3}{
                \foreach \q in {#1}{
                    \item \q
                }
            }
        }
    }
}

\newcommand{\questions}[1]{
    \begin{enumerate}
        \foreach \q in {#1}{
            \item \q\\
            \vspace*{-0.45cm}
            \dottedLines{3}
        }
    \end{enumerate}
}

% Define a new boolean for checking if the section is starred
\newboolean{section@star}

\makeatletter
% Redefine \section and \section* to set the boolean
\let\old@section\section
\renewcommand{\section}{%
    \@ifstar
        {\setboolean{section@star}{true}\old@section*}
        {\setboolean{section@star}{false}\old@section}%
}
\makeatother

\newcommand{\qt}[1]{«\textit{#1}»}

\newcommand{\calc}[1]{\numexpr#1\relax}
\newcommand{\ncalc}[1]{\number\calc{#1}}
\newcommand{\pcalc}[1]{\numprint{\ncalc{#1}}}

\newcommand{\setgrade}[1]{
    \def\grade{#1}
    % \begin{switch}{#1}
    %     \case{6e}{\global\definecolor{gradeColor}{hex}{FA8072}}
    %     \default{
    %         Default
    %         \global\definecolor{gradeColor}{RGB}{200, 50, 50}
    %     }
    % \end{switch}
    \ifthenelse{\equal{#1}{6e}}{
        \definecolor{gradeColor}{HTML}{C6233D} % FA8072 in hex
    }{
    \ifthenelse{\equal{#1}{5e}}{
        \definecolor{gradeColor}{HTML}{088255}
    }{
    \ifthenelse{\equal{#1}{4e}}{
        \definecolor{gradeColor}{HTML}{1466A8}
    }{
    \ifthenelse{\equal{#1}{3e}}{
        \definecolor{gradeColor}{HTML}{844499}
    }{
        \definecolor{gradeColor}{RGB}{0, 0, 0}
    }}}}
}

\gdef\phase{}
\newcommand{\setPhase}[1]{%
    \begin{switch}{#1}
        \case{exo}{\gdef\phase{EXERCICES}}
        \case{cr}{\gdef\phase{COURS}}
        \case{qf}{\gdef\phase{QUESTIONS FLASH}}
        \case{dm}{\gdef\phase{DEVOIR MAISON}}
        \default{\gdef\phase{#1}}
    \end{switch}
}

\newcommand\csn[1]{\csname #1\endcsname}

\newcommand{\vect}[1]{\ensuremath{\overrightarrow{#1}}}
% \newcommand{\vect}[1]{\overrightarrow{\,\mathstrut#1\,}}
\newcommand{\m}[1]{\ensuremath{\mathbf{#1}}}
\newcommand\lm[2]{\lim_{#1\to#2}}

\def\eqv{\Leftrightarrow}
\def\ssi{si et seulement si }
\def\pt{pour tout }
\def\poly2{fonction polynôme du second degré }
\def\eq2{équation second degré }
\def\discr{b^2-4ac}

% MATH TEXT
\def\et{\textrm{ et }}
\def\si{\textrm{ si }}
\def\avec{\textrm{ avec }}
\def\car{\textrm{ car }}
\def\alors{\textrm{ alors }}
\def\ou{\textrm{ ou }}
\def\ona{\textrm{ on a }}

\def\iet{\shortintertext{et}}
\def\ialors{\shortintertext{alors}}
\def\idou{\shortintertext{d'où}}
\def\ior{\shortintertext{or}}
\def\iona{\shortintertext{on a}}

\def\studentinfo{
    \vspace*{-1cm}
    \begin{minipage}{0.35\linewidth}
        nom: \dotfill
    \end{minipage}
    \begin{minipage}{0.35\linewidth}
        prénom: \dotfill
    \end{minipage}
    \begin{minipage}{0.15\linewidth}
        classes: \dotfill
    \end{minipage}
    
    \noindent\hrulefill
}

% UNITS
\def\cm{\,\centi\meter}
\def\km{\,\kilo\meter}
\newcommand{\defl}[2]{%
    \expandafter\def\csname #1\endcsname{\href{#2}{#1}\space}%
}

% Page Eduscol
\defl{Eduscol Cycle 3}{https://eduscol.education.fr/251/mathematiques-cycle-3}
\defl{Eduscol Cycle 4}{https://eduscol.education.fr/280/mathematiques-cycle-4}
\defl{Eduscol Lycée Général et technologique}{https://eduscol.education.fr/1723/programmes-et-resources-en-mathematiques-voie-gt}
\defl{Eduscol Lycée Professionnel}{https://eduscol.education.fr/1793/programmes-et-resources-en-mathematiques-voie-professionnelle}

% Repères annuels
\defl{Cycle 2}{https://eduscol.education.fr/document/13972/download}
\defl{Cycle 3}{https://eduscol.education.fr/document/14026/download}
\defl{Cycle 4}{https://eduscol.education.fr/document/14080/download}

% Attendus de fin d'année
\defl{CM2}{https://eduscol.education.fr/document/14002/download}
\defl{6e}{https://eduscol.education.fr/document/14014/download}
\defl{5e}{https://eduscol.education.fr/document/14044/download}
\defl{4e}{https://eduscol.education.fr/document/14056/download}
\defl{3e}{https://eduscol.education.fr/document/14068/download}

% Programme de mathématiques
\defl{cycle 3}{https://eduscol.education.fr/document/50990/download}
\defl{cycle 4}{https://cache.media.education.gouv.fr/file/31/89/1/ensel714_annexe3_1312891.pdf}
\defl{2nd}{https://eduscol.education.fr/document/24553/download}
\defl{2nd STHR}{https://eduscol.education.fr/document/24556/download}
\defl{1re}{https://eduscol.education.fr/document/24565/download}
\defl{1re Technologique}{https://eduscol.education.fr/document/24559/download}
\defl{Terminale Option Spécialité}{https://eduscol.education.fr/document/24568/download}
\defl{Terminale Option Complémentaire}{https://eduscol.education.fr/document/24571/download}
\defl{Terminale Option Expertes}{https://eduscol.education.fr/document/24574/download}
\defl{Terminale Technologique}{https://eduscol.education.fr/document/23107/download}

% resources thématiques
\defl{Proportionnalité}{https://eduscol.education.fr/document/17281/download}
\defl{Probabilités}{https://eduscol.education.fr/document/17275/download}
\defl{Traitement des données}{https://eduscol.education.fr/document/17269/download}

\defl{Fonctions}{https://eduscol.education.fr/document/17287/download}
\defl{Fractions}{https://eduscol.education.fr/document/17239/download}
\defl{Nombres relatifs}{https://eduscol.education.fr/document/17245/download}
\defl{Puissances}{https://eduscol.education.fr/document/17251/download}
\defl{Divisibilité et nombres premiers}{https://eduscol.education.fr/document/17257/download}
\defl{Calcul littéral}{https://eduscol.education.fr/document/17263/download}

\defl{Grandeurs et mesures}{https://eduscol.education.fr/document/17293/download}
\defl{Algorithmique et programmation}{https://eduscol.education.fr/document/17311/download}

\defl{Suites}{https://eduscol.education.fr/document/24586/download}
\defl{Produit Scalaire}{https://eduscol.education.fr/document/24589/download}
\defl{Raisonnement et démonstration (seconde)}{https://eduscol.education.fr/document/24580/download}
\defl{Raisonnement et démonstrations (première)}{https://eduscol.education.fr/document/24583/download}

\def\jules{\href{https://juels.dev/}{Jules PESIN}}
\def\yuyu{\href{https://www.instagram.com/yuyuvrajav/}{@yuyuvraj}}

\defl{Utiliser les notions de géométrie planepour démontrer}{https://eduscol.education.fr/document/17305/download}

% Manuels
\def\dim{\href{https://www.editions-hatier.fr/livre/dimensions-mathematiques-6e-ed-2016-manuel-de-leleve-9782401020023}
    {Dimensions 6e (Ed. 2016)}
}

\definecolor{myriade}{HTML}{0F83B3} %#0F83B3
\def\my{\href{https://www.editions-bordas.fr/ouvrage/myriade-mathematiques-6e-manuel-de-leleve-ed-2021-9782047337752.html}
    {Myriade 6e (Ed. 2021)}
}

\def\mm{\href{https://www.editions-hatier.fr/livre/maths-monde-cycle-4-livre-1-volume-9782278083459}
    {Maths Monde cycle 4 (Ed. 2016)}
}

\def\mi{\href{https://www.enseignants.hachette-education.com/livres/mission-indigo-mathematiques-cycle-4-5e-4e-3e-livre-eleve-ed-2017-9782013953962}
    {Mission Indigo mathématiques cycle 4 éd. 2017}
}

% https://www.armitiere.com/livre/1833174-des-maths-ensemble-et-pour-chacun-5e-mise-en--jean-philippe-rouques-helene-stainer-canope-crdp-44
\def\dmeepcC{\href{https://publimath.univ-irem.fr/PCO10003}
    {Des maths ensemble et pour chacun 5e}
}

\def\dmeepcS{\href{https://www.reseau-canope.fr/notice/des-maths-ensemble-et-pour-chacun-6e.html}
    {Des maths ensemble et pour chacun 6e}
}

\NewDocumentCommand{\dmeepc}{m O{}}{%
    \href{https://www.reseau-canope.fr/notice/des-maths-ensemble-et-pour-chacun-6e.html}
    {Des maths ensemble et pour chacun #1e \ifNotNull{#2}{(p.#2)}}
}

\NewDocumentCommand{\sesa}{m m O{} O{}}{%
    \href{https://manuel.sesamath.net/numerique/index.php?ouvrage=cm#1_#2&page_gauche=#4}{%
    Sésamath #1e #2 \ifNotNull{#4}{(#3 p.#4)}
    }
}

\NewDocumentCommand{\iP}{m m O{} O{}}{%
    \href{https://www.iparcours.fr/ouvrages/ouvrages.php?ouvrage=Cahier#1#2}{%
    iParcours #1e #2 \ifNotNull{#4}{(#3 p.#4)}
    }
}

\NewDocumentCommand{\ching}{m m O{}}{%
    \href{https://chingmath.fr/#1eme/#2}{%
    Ching@Math #1e (\reverseKebabCase{#2}\ifNotNull{#3}{ E.#3})
    }
}

\NewDocumentCommand{\wiki}{m O{}}{%
    \def\ext{}%
    \ifNotNull{#2}{\def\ext{\##2}}%
    \href{https://fr.wikipedia.org/wiki/#1\ext}%
    {Wikipédia (\reverseSnakeCase{#1}%
    \ifNotNull{#2}{ {\scriptsize $\rightarrow$ \reverseSnakeCase{#2}}}%
    )}
}

\newcommand*{\prbltq}[1]{\href{https://www.problematheque-csen.fr/fiche-probleme/#1}{Problémathèque (\reverseKebabCase{#1})}}

\NewDocumentCommand{\rpmc}{O{}}{%
    \href{https://eduscol.education.fr/document/13132/download?attachment\#page=#1}{%
    La résolution de problèmes mathématiques au collège
    (p.%
    #1%
    % \directlua{tex.print((tonumber("#1") or 0) + 3)}
    )}%
}

% Attendus de fin d'année
\NewDocumentCommand{\afa}{m O{}}{
    \ifthenelse{\equal{#1}{CM2}}{
        \def\afalink{https://eduscol.education.fr/document/14002/download}
    }{
    \ifthenelse{\equal{#1}{6e}}{
        \def\afalink{https://eduscol.education.fr/document/14014/download}
    }{
    \ifthenelse{\equal{#1}{5e}}{
        \def\afalink{https://eduscol.education.fr/document/14044/download}
    }{
    \ifthenelse{\equal{#1}{4e}}{
        \def\afalink{https://eduscol.education.fr/document/14056/download}
    }{
    \ifthenelse{\equal{#1}{3e}}{
        \def\afalink{https://eduscol.education.fr/document/14068/download}
    }{
        \def\afalink{https://eduscol.education.fr/document/14014/download}
    }}}}}
    \def\page{}
    \ifNotNull{#2}{\def\page{(p.#2)}}
    \href{\afalink\#page=#2}{Attendus de fin d'année de #1 \page}
}

\def\ca{%
    \href{https://pedagogie.ac-strasbourg.fr/mathematiques/competitions/course-aux-nombres/}%
    {Course aux nombres}%
}

% Euclide https://www.pedagogie.ac-aix-marseille.fr/jcms/c_10743971/it/les-elements-d-euclide-traduction-par-oliver-byrne

\NewDocumentCommand{\eucl}{O{1804} O{}}{
    \ifthenelse{\equal{#1}{1632}}{ % 1632
        \def\trad{D. Henrion}
        \def\afalink{https://www.pedagogie.ac-aix-marseille.fr/upload/docs/application/pdf/2019-11/elements_euclide_-_denis_henrion.pdf}
    }{
    \ifthenelse{\equal{#1}{1804}}{ % 1804 traduction F. Peyrard
        \def\trad{F. Peyrard}
        \def\afalink{https://eduscol.education.fr/document/14014/download}
    }{}
    }
    \def\page{}
    \ifNotNull{#2}{\def\page{p.#2}}
    \href{\afalink\#page=#2}{Les Éléments d'Euclide (traduction de \trad \page)}
}
% 1632 https://www.pedagogie.ac-aix-marseille.fr/upload/docs/application/pdf/2019-11/elements_euclide_-_denis_henrion.pdf
% Logiciels
\newcommand{\defIconLink}[4]{% 1 text , 2 : color , 3 : icon , 4 : link
    \expandafter\def\csname #1\endcsname{%
        {\def\iconPath{}%
        \icon{#3} \textbf{\href{#4}{\color{#2}#1}}}
    }%
}

\newcommand{\cmdIconLink}[4]{% 1 text , 2 : color , 3 : icon , 4 : link
    \expandafter\NewDocumentCommand\csname cmd#1\endcsname{O{}}
    {%
        {\def\iconPath{}%
        \icon{#3} \textbf{\href{#4/##1}{\color{#2}#1}}}
    }%
}

\definecolor{capytale}{HTML}{1E293B} % #1E293B
\definecolor{capytale-2}{HTML}{F0F1F2} % #F0F1F2

\def\Capytale{%
    \href{https://capytale2.ac-paris.fr/~/my}{\shl{capytale}{capytale-2}{CAPYTALE}}%
}

\newcommand{\capytale}[1]{%
    \href{https://capytale2.ac-paris.fr/web/c/#1}{\shl{capytale}{capytale-2}{CAPYTALE \shl{capytale-2}{capytale}{#1}}}%\shl{capytale}{capytale-2}{CAPYTALE 
}

\definecolor{enc}{HTML}{1D3D6E} % #1D3D6E
\defIconLink{ENC}{enc}{ENC-Hauts-de-Seine}{https://enc.hauts-de-seine.fr/}

\definecolor{pronote}{HTML}{1A6E45} % #1A6E45
\defIconLink{Pronote}{pronote}{pronote}{https://0922247t.index-education.net/pronote/}

\definecolor{calc}{HTML}{00A500} % #00A500
\defIconLink{Calc}{calc}{libreOffice/calc/logo}{https://fr.libreoffice.org/discover/calc/}

\definecolor{geogebra}{HTML}{9693F7} % #9693F7
\defIconLink{Geogebra}{geogebra}{geogebra/logo}{https://www.geogebra.org/classic}
\cmdIconLink{Geogebra}{geogebra}{geogebra/logo}{https://www.geogebra.org/m}

\definecolor{scratch}{HTML}{FFAB19} % #FFAB19
\defIconLink{Scratch}{Orange}{scratch/logo}{https://scratch.mit.edu/projects/editor/}

% http://trucsmaths.free.fr/etymologie.htm

\newcommand{\dym}[1]{\def\ym{\href{#1}{Yvan Monka}}}

\captionsetup{labelformat=empty,labelsep=none}

% ANNE
\setboolean{boxedProperties}{true} % false = edge
\setboolean{parenthisedID}{false}
\setboolean{showID}{false}

\def\DefinitionColor{Red}
\def\PropertyColor{Red}
\def\TheoremColor{Red}

% TIKZ
\def\crossWidth{0.25mm}
\def\pointColor{blue}

% EB Garamond
% \usepackage[cmintegrals,cmbraces]{newtxmath} 
% \usepackage{ebgaramond-maths}

% Linux Libertine
% \usepackage{libertine}
% \usepackage{libertinust1math}

% Source Serif Pro
% \usepackage[default,regular,black]{sourceserifpro}

% Source Sans Pro
% \usepackage[default]{sourcesanspro}

% TeX Gyre Pagella:
% \usepackage{tgpagella,eulervm}


%%% LuaLaTex

% Libertine
\setmainfont{Libertinus Serif}
\setmathfont{Libertinus Math}

% OpenDyslexic
% \defaultfontfeatures{ Ligatures=TeX,Scale=MatchUppercase }
% \setmainfont{OpenDyslexic}[Scale=1.0]
% \setmathfont{Fira Math} % Or maybe try KPMath-Sans?
% \setmathfont{OpenDyslexic Italic}[range=it/{Latin,latin}]
% \setmathfont{OpenDyslexic}[range=up/{Latin,latin,num}]

% Gyre
% \setromanfont{TeX Gyre Termes}
% \setsansfont{TeX Gyre Heros}
% \setmonofont{TeX Gyre Cursor}[Ligatures=NoCommon]
% \setmathfont{TeX Gyre Termes Math}

% STIX
% \setmainfont{STIX Two Text}
% \setmathfont{STIX Two Math}

% Fira
% \setmainfont{Fira Sans}
% \setmathfont{Fira Math}

\begin{document}

\ifBeamer{%
    \renewcommand*{\theenumii}{\alph{enumii}}

    \firstSlide
    \setboolean{showRef}{false}
}

\ifArticle{%
    \renewcommand*{\theenumii}{\alph{enumii}}
    
    \disableAnimation
}



% DOCUMENTS

% % VARIABLES %%%
\def\authors{\jules}
% \date{\today}
\def\longTitle{Géometrie plane - points et droites}
\def\shortTitle{\MakeUppercase{\longTitle}}
% \bseq{\longTitle}
% \def\theme{\longTitle}

\setboolean{showRef}{false}

\def\dim{Dimension 6e 2016}

\def\imgPath{enseignement/6e/geometrie-plane/points-et-droites/}
\def\imgExtension{.png}

%%

% Yvan Monka : https://www.maths-et-tiques.fr/telech/19Para_Perp.pdf

\ifArticle{\vspace*{0.1cm}}

\scn{Objet géométrique}{}

\bseq{\longTitle}
\bsec{Objet géométrique}
\bsubsec{Le point}

\slide{COURS}{
    \ssec\ssubsec
    \vc{}{
        On nomme \key{point} est le plus petit élément géométrique.
        Il est infiniment petit,
        tant qu'il n'a pas de dimension.
    }
}

\slide{EXERCICES}{
    \act{}{
        Représenter un point.
    }
}

\slide{COURS}{
    \rmk{}{
        On représente le point par une croix
    }

    \df{}{
        Deux points sont dits distincts s'ils ne sont pas confondus
    }

    \expl{}{
        Les points $A$ et $B$ sont confondus et les points $A$ et $C$ sont distincts.
    }
}

\bsubsec{La droite}
\slide{COURS}{
    \ssubsec

    \vc{}{On nomme \key{droite} un tracé rectiligne infini.}
}

\slide{EXERCICES}{
    \act{}{Représenter une droite passant par deux points $A$ et $B$}
}

\slide{COURS}{
    \rmk{}{On ne peut pas représenter une droite entièrement}

    \axio{}{
        Il existe une unique droite passant par deux points.
    }

    \expl{}{
        Pour deux points $D$ et $E$ disctincts.
        On peut noter la droite qui passe par $D$ et $E$, $(DE)$ ou $(ED)$.
    }
}

\bsubsec{Le segment et la demi-droite}

\slide{COURS}{
    \ssubsec

    \df{}{Un \key{segment} est une portion de droite,
    limité par deux \key{extrémités}
    }

    \expl{}{
        Pour deux points $D$ et $E$ disctincts.
        Le segment reliant $D$ et $E$ est le chemin le plus court entre ces deux points.
        On le note $[DE]$.
    }
}


\slide{COURS}{
    \ssubsec

    \df{}{Une \key{demi-droite} est une portion de droite,
    limité par une seule extrémité, son \key{origine}.
    }

    \expl{}{
        Pour deux points $D$ et $E$ disctincts.
        On peut noter, la demi-droite d'origine $D$ passant par $E$, $[DE)$.
    }
}

\scn{Alignement}{}

\slide{QUESTIONS FLASH}{%
    \sqf Comment peut-on nommer la droite ci-dessous?
    \vspace*{-0.5cm}
    \imgp{qf1}[5cm]
    \qfs $(AB)$ \hfill \qfs $[AB]$ \hfill \qfs $(BA)$ \hfill \qfs $(d)$ \hfill \qfs $AB$ \hfill
    \vspace*{0.5cm}
    \\ \hint{plusieurs réponses sont attendues}
}

\slide{}{
    \sqf Quels points semblent alignés?
    \vspace*{-0.5cm}
    \imgp{qf2}[5cm]
    \qfs $A$ et $F$ \hfill \qfs $A$,$D$ et $F$ \hfill \qfs $A$,$C$ et $D$ \hfill \qfs $C$,$A$ et $D$ \hfill
    \vspace*{0.5cm}
    \\ \hint{plusieurs réponses sont attendues}
}

\bsec{Alignement}

\slide{EXERCICES}{
    \vspace*{-0.5cm}
    \act{1 p208}{%
        \vspace*{-1cm}
        \imgp{dim-6e-act-1-p208}[10cm]
    }[\dim]
}

% \bsubsec{Alignement}
\slide{COURS}{
    \sseq\ssec%

    \df{}{
        Des points sont dit \key{alignés} s'il existe une droite passant par tous ces points.
    }
}

\slide{}{
    \vspace*{-0.45cm}
    \expl{}{
        \dividePage{
            \imgp{alignement}[4cm]
        }{
            Les points $A$, $C$ et $B$ sont alignés.\\
            Les points $A$, $D$ et $B$ ne sont pas alignés.
        }[0.35]
    }
    \vspace*{-1.2cm}
    \rmk{}{
        \begin{enumerate}
            \item Deux points peuvent toujours être reliés par une droite et sont donc toujours alignés.
            \item Dans l'exemple précédent ; C est placé sur la droite $(AB)$. On dit que $C$ appartient à la droite $(AB)$ et on note : $C\in(AB)$.
        \end{enumerate}
        
    }
}

% \exoList{5 p211,7 p211}[][3]

\slide{EXERCICES}{
    \exo{p211}{}
    \vspace*{-1cm}
    \imgp{dim-6e-exo-5-p211}[10cm]
    \imgp{dim-6e-exo-7-p211}[10cm]
}

\scn{Perpendicularité et parallèlisme}{}

\bsec{Droites}
\bsubsec{Definitions}

\setcounter{qf}{0}
\slide{QUESTIONS FLASH}{
    \sqf Les droites $(f)$ et $(g)$ semblent être:
    \vspace*{-0.5cm}
    \imgp{qf3}[4cm]
    \qfs Sécantes \hfill \qfs Parallèles \hfill \qfs Perpendiculaires \hfill
    \vspace*{0.5cm}
    \\ \hint{plusieurs réponses sont attendues}
}

\slide{}{
    \sqf Quelles droites semblent parallèles :
    \vspace*{-0.5cm}
    \imgp{qf4}[5cm]
    \qfs $(AB) \et (CE)$ \hfill \qfs $(AB) \et (DC)$ \hfill \qfs $(CD) \et (EB)$ \hfill
    \vspace*{0.5cm}
    % \\ \hint{plusieurs réponses sont attendus}
}

\slide{COURS}{
    \ssec\ssubsec%
    %
    \df{}{
        Deux droites sont dites \key{sécantes} si elles se coupent en un unique point.
    }
    \vspace*{-1cm}
    \expl{}{
        \dividePage{
            \imgp{secantes}[4cm]
        }{
            Les droites $(AE)$ et$(BD)$ sont sécantes.\\ $C$ est leur point d'intersection.
        }[0.45]
    }[\myl{https://biblio.manuel-numerique.com?openBook=9782047392935\%3FY29udGV4dGVSZXNvdXJjZT17InR5cGUiOiJhcnRpY2xlIiwiaWRyZWYiOiJpZF9DaGFwdGVyXzAxMl9ab29tX0dyYXBoaWNfNTUxX1NDUl94aHRtbCIsImFydGljbGVUeXBlIjoiem9vbSJ9}]
}

\slide{}{
    \df{}{
        Deux droites sont dites \key{perpendiculaires} si elles sont sécantes et leur intersection forme un angle droit.
    }
    \vspace*{-1cm}
    \expl{}{
        \dividePage{
            \imgp{perpendiculaires}[4cm]
        }{
            Les droites $(EF)$ et$(GF)$ sont perpendiculaires.\\ On note $(EF) \perp (GF)$.
        }[0.35]
    }[\myl{https://biblio.manuel-numerique.com?openBook=9782047392935\%3FY29udGV4dGVSZXNvdXJjZT17InR5cGUiOiJhcnRpY2xlIiwiaWRyZWYiOiJpZF9DaGFwdGVyXzAxMl9ab29tX0dyYXBoaWNfNTUyX1NDUl94aHRtbCIsImFydGljbGVUeXBlIjoiem9vbSJ9}]
}

\slide{}{
    \df{}{
        Deux droites sont dites \key{parallèles} si elles ne sont pas sécantes.
    }
    \vspace*{-1cm}
    \expl{}{
        \dividePage{
            \imgp{paralleles}[4cm]
        }{
            Les droites $(d)$ et$(d')$ sont parallèles.\\ On note $(d) \parallel (d')$.
            \rmk{}{Deux droites parallèles conservent le même écartement}
        }[0.3]
    }[\myl{https://biblio.manuel-numerique.com?openBook=9782047392935\%3FY29udGV4dGVSZXNvdXJjZT17InR5cGUiOiJhcnRpY2xlIiwiaWRyZWYiOiJpZF9DaGFwdGVyXzAxMl9ab29tX0dyYXBoaWNfNTUzX1NDUl94aHRtbCIsImFydGljbGVUeXBlIjoiem9vbSJ9}]
}

\slide{EXERCICES}{
    \vspace*{-0.5cm}
    \exo{11 p213}{
        \vspace*{-0.75cm}
        \imgp{dim-6e-exo-11-p213}[6.25cm]
    }
}

\scn{Construction}{}

\bsubsec{Construction}

\slide{}{
    \act{}{
        \begin{enumerate}
            \item Placer trois points $A,B$ et $C$.
            \item Tracer $(AB)$.
            \item Tracer une droite perpendiculaire à $(AB)$ passant par $C$.
        \end{enumerate}
    }
}

\slide{COURS}{
    \dividePage{
        \mthd{}{
            \imgp{construction-perpendiculaire}[5cm]
        }
    }{
        \expl{}{Construire la droite perpendiculaires à $(d)$ passant par $A$.
        \imgp{construction}[5cm]}
    }
}

\slide{EXERCICES}{
    \act{}{
        \begin{enumerate}
            \item Placer trois points $A,B$ et $C$.
            \item Tracer $(AB)$.
            \item Tracer une droite parallèle à $(AB)$ passant par $C$.
        \end{enumerate}
    }
}

\slide{COURS}{
    \dividePage{
        \mthd{}{
            \imgp{construction-parallele}[5cm]
        }
    }{
        \expl{}{Construire la droite parallèles à $(d)$ passant par $A$.
        \imgp{construction}[5cm]}
    }
}

\slide{EXERCICES}{
    \vspace*{-0.5cm}
    \exo{14 p213}{
        \vspace*{-0.75cm}
        \imgp{dim-6e-exo-14-p213}[10cm]
    }
}

\scn{Propriétés}{}

\bsec{Propriétés}
\slide{}{
    \ssec
    \pr{}{}
}

\slide{}{
    \pr{}{}
}

\slide{}{
    \pr{}{}
}
% % VARIABLES %%%
\def\authors{\jules}
% \date{\today}
\def\longTitle{Nombres Relatifs : Repérage et comparaison}
\def\shortTitle{\MakeUppercase{\longTitle}}
% \bseq{\longTitle}
% \def\theme{\longTitle}

\setboolean{showRef}{false}

% \def\my{Myriade 6e}
% \newcommand{\myl}[1]{\href{#1}{\my}}

\def\imgPath{enseignement/5e/nombres-relatifs/reperage-et-comparaison/}
\def\imgExtension{.png}
%%

% Yvan Monka : https://www.maths-et-tiques.fr/telech/19Nomb_rel1.pdf
% Euler : https://euler-ressources.ac-versailles.fr/wims/wims.cgi?module=help%2Fteacher%2Fprogram%2F&+cmd=new&+job=math.cycle4#chapitre000

% \disableAnimation
% \shortAnimation
% \firstSlide

\def\pv{\; ; \;}
\scn{Repérage}{}

\slide{QUESTIONS FLASH}{%
    \sqf Comparer à l'aide du signe $>$ ou $<$ ou $=$ les nombres :
    \begin{align*}
        &\qfs \; 10 \et 10,075\\
        &\qfs \; 0,5 \et \frac{1}{2}\\
        &\qfs \; \frac{6}{10} \et \frac{6}{9}
    \end{align*}
}

\slide{}{%
    \sqf Ranger les nombres suivant dans l'ordre croissant les nombres :
    \begin{align*}
        1,2 \pv 6 \pv 1,15 \pv 2 \pv 100 \pv 0,584
    \end{align*}
}

\slide{}{%
    \sqf Ranger les nombres suivant dans l'ordre croissant les nombres :
    \begin{align*}
        1,2 \pv 6 \pv 1,15 \pv 2 \pv 100 \pv 0,584
    \end{align*}
}


\slide{EXERCICES}{
    \act{}{
        \dividePage{
            \imgp{carte}[6cm]
        }{
            \begin{enumerate}
                \item Quelles informations nous apporte ce document?
                \item Classer les nombres en deux catégories et donner un nom à chaque catégorie.
                \item Ranger les températures par ordre croissant.
            \end{enumerate}
        }
    }
    % [\href{https://www.facebook.com/groups/994675223903586/search/?q=activite\%20m%C3\%A9t\%C3\%A9o\%20nombres\%20relatifs\&locale=fr\_FR}{Vanessa Cazier}]
}
% % VARIABLES %%%
\def\authors{\jules}
% \date{\today}
\def\longTitle{Divisibilité et nombres premiers}
\setcounter{seq}{1}
\bseq{\longTitle}

\setgrade{4e}
% \newcommand{\myl}[1]{\href{#1}{\my}}

\def\imgPath{enseignement/4e/divisibilite-et-nombres-premiers/}
\def\imgExtension{.png}
%%

% Yvan Monka : https://www.maths-et-tiques.fr/telech/19Divi-np.pdf
% Crible d'Ératosthène : https://fr.wikipedia.org/wiki/Crible_d%27%C3%89ratosth%C3%A8ne
% Juniper Green : https://fr.wikipedia.org/wiki/Juniper_Green_(jeu)

% \disableAnimation
% \shortAnimation
% \firstSlide

\avspace{0.1cm}

\obj{
    \item Déterminer la liste des nombres premiers inférieurs à 100.
    \item Décomposition d'un nombre entier en produit de facteurs premiers
    \item Modéliser et résoudre des problèmes simples mettant en jeu les notions de divisibilité et de nombre premier.
}

\scn{Divisibilité}{}

\qf{
    {$2$ divise : \choice{$4$} \choice{$5$} \choice{$6$}, \choicea{1} et \choicea{3}},
    {$30$ est divisible par : \choice{$3$} \choice{$10$} \choice{$5$} \choice{$4$}, \choicea{1}{,} \choicea{2} et \choicea{3}},
    {Donner le résultat de la division euclienne de $31$ par $4$, $31 = 4 \times 7 + 3$},
    {Quels nombres sont premiers ? : \choice{$6$} \choice{$13$} \choice{$2$} \choice{$1$}, \choicea{2} et \choicea{3}}%
}

\bsec{Divisibilité}
\slide{COURS}{
    \sseq\ssec
    \df{}{
        On dit qu'un entier $a$ est \key{divisible} par un entier $b$ s'il existe un entier $k$ tel que $a = \palt{2}{b \times k}  $.\\
        On dit alors que $a$ est un \key{multiple} de $b$, et que $b$ \key{divise} $a$ ou est un \key{diviseur} de $a$.}[\href{https://fr.wikipedia.org/wiki/Divisibilité}{Wikipédia}]
    \bvspace{-1cm}
}

\slide{}{
    \expl{}{\bvspace{-0.25cm}
        \begin{enumerate}
            \item $14 = \palt{3}{2} \times \palt{3}{7} \palt{4}{= 1 \times 14}$ \\
            alors 14 est divisible par $\palt{3}{2 ; 7} \palt{4}{; 1 \et 14}$
            \item $124 = \palt{5}{1 \times 124 = 2 \times 62 = 4 \times 31}$\\
            $\palt{5}{1;2;4;31;62;124$}\; sont les diviseurs de 124.
            \item $57 = 8 \times \palt{6}{7+1}$
            alors le reste de la division euclienne de $57$ par $8$ est $\palt{6}{1}$.
            57 est donc \palt{6}{non divisible} par $8$.
        \end{enumerate}
    }
}

\def\imgPrefix{mi-c4/exo-}

\scn{Critères de divisibilité}{}

\qf{
    {$6$ divise : \choice{$18$} \choice{$12$} \choice{$2$}, \choicea{1} et \choicea{2}},
    {Donner la liste des diviseurs de 12 : , 1;2;6;12},
    {Donner la liste des diviseurs de 39 : , 1;3;13;39}%
}

\exoSlide{21p17,22p17,23p17}[7cm][2][\mi]

\slide{COURS}{
    \pr{Critères de divisibilité}{
        Un nombre entier est divisible par :
        \begin{itemize}
            \setlength\itemsep{-0.1em}
            \item $2$ si son chiffre des unités est pair.
            \item $5$ si son chiffre des unités est $0$ ou $5$.
            \item $10$ si son chiffre des unités est $0$.
            \item $3$ si la somme de ces chiffres est un multiple de $3$.
            \item $9$ si la somme de ces chiffres est un multiple de $9$.
            \item $4$ si le nombre formé par ses deux derniers chiffres est multiple de $4$.
        \end{itemize}
    }
}

\slide{}{
    \expl{}{
        Donner les diviseurs de $1944$ inférieurs à $10$.
        \palt{2}{
            \begin{itemize}
                \item $1$ divise tous les entiers.
                \item $4$ est pair donc 1944 est divisible par $2$
                \item $44 = 4 \times 11$ donc 1944 est divisible par $4$.
                \item $1+9+4+4 = \pcalc{1+9+4+4}$ et $1+8 = 9$ donc $1944$ est divisible par $9$ et $3$.
                \item $1944 = 277 \times 7 + \ncalc{1944-277*7}$ donc $1944$ n'est pas divisible par $7$.
                \item $1944 \div 8 = \ncalc{1944/2} \div 4 = \ncalc{1944/4} \div 2 $
                donc $1944$ est divisible par $8$, car $1944$ est divisible par $2$, $3$ fois de suite.
            \end{itemize}
            Les diviseurs de $1944$ inférieurs à $10$ sont donc $1;2;3;4;8;9$.
        }
    }
}

\scn{Problème de divisibilité}{travail de groupes}

\def\imgPrefix{mi-c4/qf-}
\qfSlide{
    \imgp{2abcdp12}[9cm]
    \imgp{5p12}[9cm]
    \imgp{6p12}[9cm]
}

\slide{EXERCICES}{
    \exo{}{
        Établir la liste des diviseurs communs de $189$ et $126$.
    }
}

\def\imgPrefix{mi-c4/exo-}
\exoSlide{42p19,48p19,65p21}[7cm][2][\mi]

% \avspace{0.25cm}
% \ifArticle{\TODO{? Scéance critères de divisibilité}}

\scn{Nombres premiers}{}

\def\imgPrefix{mi-c4/qf-}
\qfSlide{
    \imgp{3p12}[9cm]
    \imgp{7p12}[9cm]
}

\bsec{Nombres premiers}
\bsubsec{Définition}

\def\imgPrefix{}
\slide{}{
    \bvspace{-0.4cm}
    \act{4 p13 - Crible d'Ératosthène}{
        \ifBeamer{\vspace{-0.5cm}\small}
        \dividePage{%
            \imgp{crible-d-eratosthene}[5cm]%
        }{%
            \begin{enumerate}
                \item Barrer le 1.
                \item Entourer 2,
                puis barrer tous les multiples de 2 autres que 2.
                \item Entourer le premier nombre ni entouré ni barré,
                puis barrer tous ses multiples autres que lui-même.
                \item Répéter la consigne jusqu'à atteindre le premier nombre premier plus grand que 10.
                \item Quelle particularité possèdent les vingt-cinq nombres entourés ?
            \end{enumerate}
        }[0.35]%
        % \avspace{0.25cm}
        % \bvspace{-0.25cm}
        % \begin{enumerate}
        %     \setcounter{enumi}{3}
        %     \item Quelle particularité possèdent les vingt-cinq nombres entourés ?
        % \end{enumerate}
    }[\mi]
}

\slide{COURS}{
    \ssec
    \df{}{
        Un nombre entier est dit \key{premier} s'il a exactement deux diviseurs différents: 1 et lui-même.
    }
    % \bvspace{-1cm}
    \ifBeamer{\small}
    \bvspace{-1cm}
    \rmk{}{%
    \bvspace{-0.25cm}
        \begin{itemize}
            \item $1$ n'est pas premier. Il n'a qu'un seul diviseur, lui-même.
            \item $2$, le plus petit nombre premier est le seul nombre premier pair.
        \end{itemize}
    }
    \pr{}{%
        Il existe une infinité de nombres premiers.
    }
}

% \slide{EXERCICES}{
%     \exo{27;28;29}
% }

\def\imgPrefix{mi-c4/exo-}
\exoSlide{27p17,28p17,29p17}[7cm][2][\mi]

\scn{Décomposition en facteurs premiers}{}

\def\imgPrefix{mi-c4/qf-}
\qfSlide{
    \imgp{5p17}[7cm]
}

\bsubsec{Décomposition en facteurs premiers}

\def\imgPrefix{}

\slide{EXERCICES}{
    \bvspace{-0.75cm}
    \act{Representation chromatique des nombres}{\bvspace{-1cm}
    \wideFrame[6em]{
        \dividePage{\imgp{representation-chromatique-des-nombres}[5.75cm]}{
            \begin{enumerate}
                \item Écrire sous la forme d'un produit de facteurs premiers.
                \begin{align*}
                    4 &= \hole \qquad 10 = \hole \qquad 14 = \hole \\
                    15 &= \hole \qquad 20 = \hole
                \end{align*}
                \item Expliquer comment les nombres sont représentés sur ce tableau.
            \end{enumerate}
        }[0.35]
    }
    }[\href{https://www.monclasseurdemaths.fr/tc/représentation-chromatique-des-nombres/}
    {Mon classeur de maths}]
}

\slide{COURS}{
    \ssubsec
    \pr{}{
        Tout nombre entier strictement supérieur à 1 peut se décomposer en produit de facteurs premiers.
    }
    \bvspace{-0.5cm}
    \expl{}{Donner la décomposition en facteurs premiers de :\bvspace{-1cm}%
        \begin{align*}%
            1100 &= \palt{2}{%
                2 \times 550\\
                &=2 \times 2 \times 275\\
                &=2 \times 2 \times 5 \times 275\\
                &=2 \times 2 \times 5 \times 55\\
                &=2 \times 2 \times 5 \times 5 \times 11
            }
        \end{align*}%
    }
}

\slide{cr}{
    \mthd{Décomposition en facteurs premiers}{
        On prend la liste des nombres premiers dans l'ordre,
        puis tester la division par chaque nombre premier,
        un par un.
    }
}

\slide{qf}{
    Donner la décomposition en facteurs premiers de :
    \multiColEnumerate{3}{
        \item 6 \item 28 \item 5 \item 24 \item 66 \item 52
    }
}

\def\imgPrefix{mi-c4/exo-}

\slide{EXERCICES}{
    \exo{}{
        \begin{enumerate}
            \item Trouver la décomposition en facteurs premiers de $70$ et $105$.
            \item En déduire les diviseurs communs de $70$ et $105$.
        \end{enumerate}
    }
    \bvspace{-0.5cm}
    \exo{}{
        Un fleuriste doit réaliser des bouquets tous identiques.
        Il dispose pour cela de 434 roses et 620 tulipes. Quelles sont toutes les compositions de bouquets possibles ?
    }[\href{https://eduscol.education.fr/document/14056/download}
    {Attendues de fin d'année de 4e}]
}

\slide{qf}{
    Trouver les diviseurs communs de $60$ et $42$
}

\exoSlide{20p17,45p19,56p20}[7cm][2][\mi][dm]

\scn{Simplification de fractions}{}

\slide{qf}{
    Ecrire sous forme de fractions irréductibles:
    \multiColEnumerate{2}{
        \item $\frac{2}{4}$
        \item $\frac{60}{50}$
        \item $\frac{33}{55}$
        \item $\frac{4}{6}$
    }
}

\slide{cr}{
    \app{Simplification de fractions}{}
    \bvspace{-1cm}
    \expl{}{
        Simplifier la fraction :
        \begin{align*}
            \frac{140}{294} &= \palt{2}{
                \frac{2 \times 2 \times 5 \times 7}{2 \times 3 \times 7 \times 7}\\
                &= \frac{2 \times 5}{3 \times 7}\\
                &= \frac{10}{21}
            }
        \end{align*}
    }[\href{https://math-coaching.com/fiche/simplifier-fraction-decomposition-facteurs-premiers-130}{Math Coaching}]
}

\slide{cr}{
    \mthd{Simplification de fractions}{
        \begin{enumerate}
            \item On décompose le numérateur et dénominateurs en facteurs premiers.
            \item On supprime les facteurs communs.
            \item On multiplie ensemble les facteurs restants au numérateur et dénominateurs.
        \end{enumerate}
    }
}

\exoSlide{58p54}[10cm][1][\mi]

% % VARIABLES %%%
% \date{\today}

\setSeq{2}{Organisation et gestion de données}

\setGrade{6e}
% \setImgPath
\def\imgPath{enseignement/6e/organisation-et-gestion-de-donnees/}
%%

\def\ym{\href{https://www.maths-et-tiques.fr/telech/19Tab_Graph.pdf}{Yvan Monka}}

\avspace{0.1cm}

\obj{
    \item Prélever des données numériques à partir de supports variés.
    \item Produire des tableaux, diagrammes et graphiques organisant des données numériques.
    % \item Exploiter et communiquer des résultats de mesures.
    \item Reconnaitre une situations de proportionnalité/situations qui ne sont pas de proportionnalité
}

\scn{Tableaux}{}

\bsec{Tableaux}
% \bsubsec{Tableaux de données}

\def\imgPrefix{dim-6e/qf-}
\exoSlide{1-2p96}[10cm][1][\dim][qf]
\def\imgPrefix{}

\def\imgPrefix{dim-6e/act-}
\slide{exo}{
    \bvspace{-0.75cm}
    \act{1p98}{\bvspace{-0.75cm}\imgp{1p98}[11.5cm]}[\dim]
}

\slide{cr}{
    \sseq\ssec
    \vc{}{
        Un \key{tableau} permet de rassembler et d'organiser des données pour les lire plus facilement.
    }
}

\slide{cr}{
    \expl{}{
        Au collège de la Paix,
        les enfants ont le choix entre 3 LV2 :
        italien, allemand ou espagnol.\\

        En 6eA, il y a 25 élèves. 12 ont choisi espagnol, 6 allemand et les autres italien.

        En 6eB, 13 élèves ont choisi espagnol et 5 élèves allemand.

        Dans ces deux classes, 12 élèves ont choisi italien.\\

        Présenter ces données dans un tableau à double entrée.
    }[\href{https://eduscol.education.fr/document/14014/download}{Attendues 6e}]
}

\newcommand{\last}[1]{\color{red}\textbf{#1}}

\def\imgPrefix{}
\slide{cr}{\bvspace{-0.5cm}
    \mthd{Construire un tableau}{\bvspace{-0.75cm}
        \begin{enumerate}
            \item On réalise un tableau à double entrée avec les données de l'énoncé et on ajoute une colonne et une ligne «total». 
            \item On le complète avec les données de l'énoncé.
            \item On finit de compléter le tableau en effectuant les calculs.
        \end{enumerate}
        \begin{center}
            \begin{tabular}{|c|c|c|c|c|}
                \hline
                & Espagnol & Allemand & Italien & Total\\
                \hline
                5eA    & 12       & 6        & \last{7}       & 25        \\
                \hline
                5eB    & 13       & 5        & \last{5}       & \last{23}   \\
                \hline
                Total  & \last{25}       & \last{11}       & 12    & \last{48}      \\
                \hline
            \end{tabular}
        \end{center}
        % \imgp{tableau-exemple}[8cm]
    }[\ym]
}

\def\imgPrefix{dim-6e/exo-}
\exoSlide{17p104,19p104}[6cm][2][\dim]

\scn{Diagrammes en bâton}{}

\bsec{Diagrammes}
\bsubsec{Diagrammes en bâton}

\def\imgPrefix{dim-6e/qf-}
\exoSlide{3-4p96}[8cm][1][\dim][qf]

\def\imgPrefix{dim-6e/exo-}
\exoSlide{6p101}[8cm][1][\dim]

\slide{}{
    \act{}{
        \ifBeamer{\imgp{6p101}[10cm]}
        \begin{itemize}
            \item Représente les données de l'exercice 6 page 101 sous la forme d'un diagramme en bâtons.
        \end{itemize}
        \def\imgPrefix{}
        \imgp{diagramme-baton-fond-exo-6p102}[6cm]
    }
}

\def\imgPrefix{}
\slide{cr}{
    \ssec\ssubsec

    \vc{}{
        Un \key{diagramme en bâton} (ou à barres) permet de comparer visuellement des données.
    }[\dim]
    
    \expl{}{
        \begin{itemize}
            \item Notes (sur 5) de la dernière évaluation des 6eA; sous forme de \key{liste} :
            \begin{align*}
                0;4;5;5;5;1;3;4;2;3;4;1;0;4;3;4;5;4;3;2;4;2;5;5;4
            \end{align*}
            \item On peut représenter ces données dans un \key{tableau} :
            \begin{center}
                \begin{tabular}{|c|c|c|c|c|c|c|c|}
                    \hline
                    Notes & 0 & 1 & 2 & 3 & 4 & 5 & Total \\
                    \hline
                    Effectifs & 2 & 2 & 3 & 4 & 7 & 6 & 24 \\
                    \hline
                \end{tabular}
            \end{center}
            \item Ou avec un \key{diagramme en bâton}.
            \imgp{diagramme-baton-exo-6p102}[8cm]
        \end{itemize}
    }
}
% \def\imgPrefix{dim-6e/act-}
% \slide{exo}{
%     \bvspace{-0.75cm}
%     \act{2p98}{\bvspace{-0.75cm}\imgp{2p98}[9cm]}[\dim]
% }

% \def\imgPrefix{}
% \slide{exo}{
%     \imgp{tableau-activite-2p98}[11cm]
% }

% \slide{cr}{
% }

\slide{exo}{
    \bvspace{-0.5cm}
    \exo{Vrai ou faux?}{
        Le nombre de tablettes vendues de la marque B est trois fois plus important que le nombre de tablettes vendues de la marque A.
        \imgp{diagramme-en-baton-attendus-6e}[7cm]
    }
}

\slide{cr}{
    \pr{}{
        Si un diagramme en bâtons a pour origine 0,
        alors la hauteur des barres est proportionnelle aux effectifs.
    }
}


% \newpage

\def\imgPrefix{dim-6e/exo-}
\exoSlide{27p105}[3.8cm][1][\dim]

\scn{Diagrammes circulaires}{}

\bsubsec{Diagrammes circulaires}

\def\imgPrefix{dim-6e/qf-}
\exoSlide{5-6p96}[8cm][1][\dim][qf]

\def\imgPrefix{dim-6e/act-}
\slide{exo}{
    \bvspace{-0.75cm}
    \act{4p99}{\bvspace{-0.75cm}\imgp{4p99}[11cm]}[\dim]
}

\slide{cr}{
    \ssubsec
    \vc{}{
        Un \key{diagramme circulaire} permet d'observer une répartition.
    }[\dim]
    \pr{}{
        La valeur de chaque angle au centre est proportionnelle à la portion représentée.
    }
}

\scn{Utiliser différents modes de représentation}
\slide{qf}{
    \multiColEnumerate{3}{
        \item $7 \times 2 + 2$
        \item $15 + 28 \div 7$
        \item $12 \times (15-13)$
        \item $2 \times 3 - 1$
        \item $30 \div (15-13)$
        \item $19 + 2 \times 6$
    }
}

\def\imgPrefix{}
\slide{exo}{
    \bvspace{-0.5cm}
    \exo{}{\ifBeamer{\small}
        \bvspace{-0.25cm}
        \begin{enumerate}
            \dividePage{%
                Dans un collège,
                112 élèves viennent en voiture,
                autant viennent à vélo,
                56 viennent en bus et 280 viennent à pied.
                \item Un seul de ces diagrammes circulaires représente le mode de déplacement des élèves de ce collège.
                Lequel?
            }{\bvspace{-1cm}\imgp{vers-de-mobilites-douces-1}[6.5cm]}[0.5]
            
            \ifArticle{
                \dividePage{\imgp{vers-de-mobilites-douces-2}[8cm]}{%
                \item Compléter le tableau ci-contre, puis choisir les nombres appropriés pour graduer
                le diagramme en bâtons qui représente ces données.
                }
            }
            \saveenumi
        \end{enumerate}
    }[\rpmc[34]]
}

\ifBeamer{
    \slide{exo}{\bvspace{-0.25cm}
        \begin{enumerate}
            \loadenumi
            \item Compléter le tableau ci-contre, puis choisir les nombres appropriés pour graduer
        le diagramme en bâtons qui représente ces données.
        \bvspace{-0.45cm}\imgp{vers-de-mobilites-douces-2}[8cm]
        \end{enumerate}
    }
}

\def\imgPrefix{dim-6e/exo-}

\scn{Situations de proportionnalité}{}

\exoSlide{36p107}[5.25cm][1][\dim][qf]

\bsec{Situations de proportionnalité}

\slide{exo}{
    \bvspace{-0.5cm}
    \act{}{
        \begin{itemize}
            \item Pour préparer une recette de dahl,
            Christopher a besoin de 4 gousses d'ail pour 6 personnes.
            Comme il déjeune avec ses amis Sarah et Jean.

            Peut-il prévoir le nombre de gousses d'ail nécessaires ?
            Pourquoi ? Si oui, combien ?

            \item Jean l'a félicité 2 fois pour sa cuisine en 20 minutes.

            Peut-il prévoir combien de félicitations il recevra de Jean en 40 minutes?
            Pourquoi ? Si oui, combien ?
        \end{itemize}
        
    }
}

\slide{cr}{
    \ssec
    % \df{}{
    %     Deux grandeurs sont dites \key{proportionnelles} si les valeurs de l'une s'obtiennent en multipliant les valeurs de l'autre par un même nombre non nul,
    %     appelé le \key{coefficient de proportionnalité}.
    % }
    \vc{}{
        Deux \key{grandeurs proportionnelles} sont deux grandeurs qui varient dans les mêmes proportions.
    }
}

\scn{Grandeurs proportionnelles}

\def\imgPrefix{}
\slide{qf}{%
    \begin{enumerate}
        \item \imgp{cn-CM2-juin-2023-exo-18}[8cm]
        \item \imgp{cn-CM2-juin-2023-exo-21-22}[8cm]
    \end{enumerate}
}

\slide{cr}{
    \expl{}{
        Les $2kg$ de lentilles coutent $5\EUR$.
        Combien en coutent $10kg$ de lentilles ? $12kg$ de lentilles?
    }
}

% \slide{cr}{
%     \rmk{}{
%         On peut présenter une situation de proportionnalité dans un \key{tableau de proportionnalite}.
%     }
% }


\def\imgPrefix{dim-6e/exo-}
\exoSlide{2p83}[6cm][1][\dim]

\scn{Decouvrir les graphiques cartésiens}{}

\def\imgPrefix{dim-6e/qf-}
\exoSlide{54-55-56p112}[10cm][1][\dim][qf]
\def\imgPrefix{}

\bsec{Graphiques cartésiens}

\def\imgPrefix{dim-6e/act-}
\slide{exo}{
    \bvspace{-0.70cm}
    \act{3p99}{\bvspace{-0.75cm}\imgp{3p99}[9cm]}[\dim]
}

\def\imgPrefix{}

\slide{exo}{
    \bvspace{-0.25cm}
    \begin{center}
        \begin{tikzpicture}[yscale = 0.75, xscale = 0.75]
            \tkzInit[xmin=0,xmax=10,ymin=0,ymax=10]
            \tkzGrid[sub,color=gradeColor!50!white,subxstep=5,subystep=0.1]        
            % \tkzLabelX[step=10]
            % \tkzLabelY[step=2]
            \tkzDrawY[label={}]
            \tkzDrawX[label= {}]
        \end{tikzpicture}
    \end{center}
    % \imgp{graph-bg-alcoolemie}[7cm]
}

\slide{cr}{
    \ssec
    \vc{}{
        Un \key{graphique cartésien} permet de présenter l'évolution d'une grandeur en fonction d'une autre.
    }
}

\scn{Dessiner un graphique cartésien}

\def\imgPrefix{dim-6e/qf-}
\exoSlide{57p112}[10cm][1][\dim][qf]
\def\imgPrefix{}

\def\graph{
    \begin{center}
        \begin{tikzpicture}[yscale = 0.35, xscale = 0.75]
            \tkzInit[xmin=0,xmax=12,ymin=0,ymax=20]
            \tkzGrid[sub,color=gradeColor!50!white,subxstep=5,subystep=0.2]        
            % \tkzLabelX[step=10]
            % \tkzLabelY[step=2]
            \tkzDrawY[label={}]
            \tkzDrawX[label= {}]
        \end{tikzpicture}
    \end{center}
}

\slide{cr}{
    \bvspace{-0.6cm}
    \ifBeamer{\small}
    \expl{}{%
    Le tableau ci-dessous représente les températures moyennes à Issy-les-Moulineaux pour chaque mois de 2023.
        
        \vspace{0.1cm}
        \wideFrame[9.2em]{
            \def\cW{0.55cm}
            \begin{tabular}{|c|C{\cW}|C{\cW}|C{\cW}|C{\cW}|C{\cW}|C{\cW}|C{\cW}|C{\cW}|C{\cW}|C{\cW}|C{\cW}|C{\cW}|C{\cW}|}
                \hline
                Mois & J & F & M & A & M & J & J & A & S & O & N & D \\
                \hline
                Température (\degres C) & 4,0 & 4,3 & 7,0 & 10,4 & 14,2 & 17,4 & 19,1 & 19,1 & 15,9 & 12,4 & 7,6 & 4,3 \\
                \hline
            \end{tabular}
        }

        \vspace{0.25cm}

        Représenter les données du tableau dans un graphique cartésien.
    }[\href{https://www.annuaire-mairie.fr/ensoleillement-issy-les-moulineaux.html}{https://www.annuaire-mairie.fr/}]
    \bvspace{-0.8cm}

    \ifArticle{\graph}
    % \imgp{graph-bg-meteo}[7cm]
}

\ifBeamer{\slide{cr}{\graph}}

\slide{exo}{
    \exo{}{
        Que pourrait représenter ce graphique à propos d'une salle de classe?
        Le décrire avec le plus de précision possible.
        Justifier et compléter le graphique ci-dessous.
        \imgp{l-allure-de-la-courbe}[10cm]
    }[\rpmc[31]]
}

\slide{cr}{
    \rmk{}{
        Les points sont :
        \begin{enumerate}
            \item représentés par des «+».
            \item reliés par des segments.
        \end{enumerate}
    }
}

% \bsec{Reconnaître une situation de proportionnalité}

\slide{exo}{%
    \bvspace{-0.6cm}
    \exo{}{
        On a demandé aux élèves des trois classes de 6e du collège Anatole France combien d'animaux de compagnie vivaient avec eux.
        On a représenté les résultats dans le diagramme suivant.
        \bvspace{-0.25cm}
        \imgp{nos-amis-les-betes}[9cm]
    }[\rpmc[28]]
}

% \exoSlide{41p109,25p105}[10cm][1][\dim][dm]

\slide{exo}{
    Les affirmations suivantes sont-elles vraies ou fausses? Justifier.
    \begin{enumerate}
        \setItemColor[exo]
        \item 21 élèves ont un seul animal de compagnie.
        \item Il y a 75 élèves en 6e au collège Anatole France.
        \item Les élèves qui ont deux animaux de compagnie sont trois fois plus nombreux que les élèves qui ont trois animaux de compagnie.
        \item 70 élèves ont moins de trois animaux de compagnie.
        \item Plus de la moitié des élèves ont au moins un animal de compagnie.
    \end{enumerate}
}
% VARIABLES %%%
\def\authors{\jules}
% \date{\today}
\setSeq{2}{Proportionnalité - Situation de proportionnalité et conversions}
% \def\theme{\longTitle}

\setGrade{5e}

\def\imgPath{enseignement/5e/proportionnalite/situation-de-proportionnalite-et-conversions/}
\def\imgExtension{.png}

% \firstSlide
%%

% Yvan Monka : https://www.maths-et-tiques.fr/telech/19Prop1.pdf
\ifArticle{\vspace*{0.1cm}}

\obj{
    \item Reconnaître une situation de proportionnalité ou de non proportionnalité entre deux grandeurs.
    \item Résoudre des problèmes de proportionnalité par passage à l'unité.
    \item Effectuer des calculs de durées et d'horaires.
    \item Effectuer des conversions d'unités de longueurs, et de durées.
    \item Partager une quantité en deux ou trois parts selon un ratio donné.
}

\scn{Passage à l'unité}{}

\qfSlide{
    Donner le resultat sous la forme décimale de :
    \multiColEnumerate{2}{
        \item $14 \times 3 = \palt{2}{\ncalc{14*3}} $
        \item $3 \div 2 = \palt{2}{1,5} $
        \item $6 \div 4 = \palt{2}{1,5} $
        \item $5,6 \times 2 = \palt{2}{11,2} $
        \item $\frac{1}{4} = \palt{2}{0.25}$
        \item $\frac{3}{4} = \palt{2}{0.75}$
    }
}

\bsec{Proportionnalité}
\bsubsec{Définition}

\slide{exo}{
    \bvspace{-0.5cm}
    \act{}{
        \begin{itemize}
            \item Pour une recette de dahl,
            Christopher a besoin de 8 gousses d'ail pour 4 personnes.
            Il déjeune avec ses amis Sarah et Jean.
            Peut-il prévoir combien de gousses d'ail il aura besoin ?\\
            Pourquoi ? et si oui, combien ?
            \item Jean l'a félicité 2 fois pour sa cuisine en 20 minutes.
            Peut-il prévoir combien de félicitations il recevra de Jean en 40 minutes?\\
            Pourquoi ? et si oui, combien ?
        \end{itemize}
        
    }
}

\slide{cr}{
    \sseq\ssec\ssubsec
    \df{}{
        Deux grandeurs sont dites \key{proportionnelles} si les valeurs de l'une s'obtiennent en multipliant les valeurs de l'autre par un même nombre non nul,
        appelé le \key{coefficient de proportionnalité}.
    }
}

\slide{cr}{
    \expl{}{
        Les $2kg$ de lentilles $5\EUR$.
        Combien en coutent $13kg$ de lentilles ?
    }

    \mthd{Passage à l'unité}{
        \palt{2}{
            \begin{tabular}{|c|c|c|c|}
                \hline
                Masse (en $\kilo\gram$)  & 2 & 1 & 13\\
                \hline
                Prix (en \EUR)    & 5 & 2,5 & 32,5\\
                \hline
            \end{tabular}
        }
        \palt{3}{Le coefficient de proportionnalité est 2,5.}
    }
}

\slide{cr}{
    \vc{}{
        Un \key{tableau de proportionnalité} permet de présenter une situation de proportionnalité.
        Sa deuxième ligne s'obtient en multipliant la première par le coefficient de proportionnalité.
    }
}

\def\imgPrefix{mm-c4/exo-}
\exoSlide{32p27,51p28,53p28}[7cm][2][\mm]

\scn{Reconnaître une situation de proportionnalité}{}

\def\imgPrefix{mm-c4/qf-}
\exoSlide{15p26,17p26}[7cm][2][\mm][qf]

\bsubsec{Reconnaître une situation de proportionnalité}

\slide{cr}{
    \bvspace{-0.25cm}
    \ssubsec
    \bvspace{-0.5cm}
    \expl{}{
        Une marque d'épices vend différentes tailles de pots de curcuma et de cumin.
        Présenter avec leur prix dans les deux tableaux ci-dessous.\\

        \begin{tabular}{|c|c|c|c|}
            \hline
            Masse de curcuma (en $\gram$)  & 45 & 60 & 90\\
            \hline
            Prix (en \EUR)    & 2.5 & 4 & 5\\
            \hline
        \end{tabular}\ifArticle{\quad}\ifBeamer{\\ \\}
        \begin{tabular}{|c|c|c|c|}
            \hline
            Masse de cumin (en $\gram$)  & 45 & 60 & 90\\
            \hline
            Prix (en \EUR)    & 2.7 & 4.05 & 5.4\\
            \hline
        \end{tabular}\\

        Les prix de ces deux épices sont-ils proportionnels à leur masse ?
    }
}

\slide{cr}{
    \mthd{Reconnaître une situation de proportionnalité}{
        \palt{2}{
            On divise chaque nombre de la première grandeur par ceux de la 2e grandeurs correspondant.
            Si on obtient le même résultat, il y a proportionnalité.
        }
    }
}

% \exoSlide{}[][][][]
\scn{Conversions}

\def\imgPrefix{}
\slide{qf}{
    \begin{enumerate}
        \item \imgp{qf-cn-3-5e-mars-2023}[6cm]
        \item \imgp{qf-cn-8-5e-mars-2023}[6cm]
    \end{enumerate}
}

\scn{Distinction grandeurs et mesures}

\bsec{Grandeurs}

\bsubsec{Definition}

% \newpage
\slide{exo}{
    \act{Trier ces mots en deux catégories que vous nommerez}{
        \avspace{-0.5cm}\multiColItemize{2}{
            \item volume
            \item durée
            \item centimètre
            \item prix
            \item euro
            \item gramme
            \item litre
            \item surface
        }
    }
}

\slide{cr}{
    \ssec\ssubsec
    \df{}{
        Une \key{grandeur} est une caractéristique d'un objet qui se mesure ou se calcule.
    }
}

\slide{VIDEO}{
    \bvspace{-1cm}
    \hist{}{
        \imgp{l-histoire-du-metre}[9cm]
        \begin{center}
            \hypersetup{urlcolor = gradeColor!90}
            \href{https://youtu.be/PvlsXcOzNd0?si=P0qko7XnxAFT-KCm}
            {L'histoire du mètre (J't'explique)}
        \end{center}
    }[\href{https://youtu.be/PvlsXcOzNd0?si=P0qko7XnxAFT-KCm}{J't'explique}]
}

\slide{}{
    \vc{}{
        On donne la valeur d'une grandeur en comparaison à une \key{unité de mesure} de référence.
    }
    
    \bvspace{-1cm}
    \expl{}{
        \ifBeamer{\\}
        \def\cW{1.8cm}
        \wideFrame[7em]{
            \begin{center}
                \begin{tabular}{|C{\cW}|C{\cW}|C{\cW}|C{\cW}|C{\cW}|C{\cW}|C{\cW}|C{\cW}|}
                    \hline
                    \color{Red}Grandeur & \palt{2}{Masse} & \palt{2}{Longueurs} & \palt{2}{Volume} & \palt{2}{Temps} & \palt{2}{Stockage}\\
                    \hline
                    \color{Red}Unité de mesure & \palt{2}{gramme} & \palt{2}{mètre} & \palt{2}{litre et mètre cube} & \palt{2}{seconde} & \palt{2}{octet}\\
                    \hline
                \end{tabular}
            \end{center}
        }
    }
}

\scn{Utiliser les préfixes de mesures}

\def\imgPrefix{mm-c4/}
\exoSlide{13p58,14p58}[6cm][2][\mm][qf]

\bsubsec{Conversions}

\slide{exo}{
    \act{Compléte avec les unités manquantes}{\avspace{-0.5cm}
        \multiColEnumerate{2}{
            \item Une bouteille de $1,5 \palt{2}{ \liter}$
            \item Une règle de $30 \palt{2}{ \centi\meter}$
            \item Une feuille d'une épaisseur de $0,2 \palt{2}{ \milli\meter}$
            \item Une bébé de $3 \palt{2}{ \kilo\gram}$
            \item Un verre de $24 \palt{2}{ \centi\liter}$
            \item Une carte mémoire de $64 \palt{2}{ \giga\octet}$
            \item Un pont de $1,5 \palt{2}{ \kilo\meter}$
        }
    }
}

\def\imgPrefix{mm-c4/}
\exoSlide{25p384,38p385}[7cm][2][\mm]

\slide{cr}{\def\cW{1cm}
    % \wideFrame[6em]{
    \bvspace{-0.5cm}
        \vc{}{
            \ifBeamer{\\ \\}
            \begin{tabular}{|C{3cm}|C{\cW}|C{\cW}|C{\cW}|C{\cW}|C{\cW}|C{\cW}|C{\cW}|}
                \hline
                \key{Préfixe} & kilo & hecto & deca & \_ & déci & centi & milli \\
                \hline
                \key{Symbole} & \kilo & \hecto & \deca & \_ & \deci & \centi & \milli \\
                \hline
                \key{Signification} & $1000$ & $100$ & $10$ & $1$ & $0,1$ & $0,01$ & $0,001$ \\
                \hline
            \end{tabular}
        }
        \bvspace{-1cm}
        \expl{}{
            \multiColEnumerate{2}{
                \item $1\deci\meter = \palt{2}{0,1} \meter$
                \item $3,6\hecto\liter = \palt{2}{360} \liter$
                \item $1\meter = \palt{2}{100} \centi\meter$
                \item $1\kilo\octet = \palt{2}{100} \deca\octet$
                \item $1\hecto\gram = \palt{2}{100 000} \milli\gram$
                \item $12\centi\meter = \palt{2}{0.012} \deca\meter$ 
            }
        }
    % }
}

\exoSlide{43p385}[9cm][1][\mm]

\scn{Convertir des durées}

\exoSlide{26p384,30p384}[9cm][1][\mm][qf]

\bsubsec{Durées}

\slide{exo}{
    \act{}{
        Un pilotte de kart parcours 90 km en trois-quarts d'heure.
        En imaginant qu'il soit capable de maintenir cette vitesse,
        combien de kilomètres aura-t-il parcourut en 1 heure ?
    }
}

\slide{cr}{
    \ssubsec
    \pr{}{
        \multiColItemize{3}{
            \item $1\hour = \palt{2}{60}\minute$
            \item $1\minute = \palt{2}{60}\second$
            \item $1\textrm{ jour} = \palt{2}{24}\hour$
        }
    }

    \expl{}{
        \begin{enumerate}
            \item Combien y'a-t-il de seconds dans une journée ?
            \item Convertir $78\min$ en heures.
            \item Convertir $7292\sec$ en heures, minutes, secondes.
        \end{enumerate}
    }
}

% \slide{exo}{
%     \exo{}{
%         On dispose de deux robinets.
%         Le premier est capable de remplir un réservoir d'eau de 24 L en 1 minute,
%         le second peut remplir ce même réservoir en 2 minutes.
%         Ouvrant les deux robinets au même moment,
%         combien de temps faudrait-il pour remplir un jacuzzi avec 1 080 L d'eau?
%         \multiColEnumerate{4}{
%             \item 15 min
%             \item 67,5 min
%             \item 135 min
%             \item 30 min
%         }
%     }[\rpmc[162]]
% }

% \bsubsec{Grandeurs proportionnelles}

\scn{Problème de proportionnalité}

\exoSlide{27p384}[9cm][1][\mm][qf]

\slide{exo}{
    \exo{}{
        \begin{enumerate}
            \item Le périmètre d'un cercle est-il proportionnel à son rayon?
            \item L'aire d'un carré est-elle proportionnelle à la longueur de ces côtés?
        \end{enumerate}
    }
}

\slide{exo}{
    \exo{}{
        On dispose de deux robinets.
        Le premier est capable de remplir un réservoir d'eau de 24 L en 1 minute,
        le second peut remplir ce même réservoir en 2 minutes.
        Ouvrant les deux robinets au même moment,
        combien de temps faudrait-il pour remplir un jacuzzi avec 1 080 L d'eau?
        \multiColEnumerate{4}{
            \item 15 min
            \item 67,5 min
            \item 135 min
            \item 30 min
        }
    }[\rpmc[162]]
}

\scn{Utiliser les ratios}

\bsec{Ratio}

\exoSlide{28p384}[9cm][1][\mm][qf]

\slide{exo}{
    \act{}{
        \begin{enumerate}
            \item Pour faire un dahl formidable,
            il faut mettre du lait de coco et de l'eau dans le ratio 3 pour 2 (noté $3 : 2$).
            Combien faut-il ajouter d'eau pour $150\milli\liter$ de lait de coco?
            \item Afin d'améliorer encore les saveurs,
            on peut ajouter un peu de sauces piquante.
            La recette indique le ratio $1:9:6$. A quel nombre du ratio peut correspondre la sauce piquante?
        \end{enumerate}
    }
}
\slide{cr}{
    \ssec

    \df{}{
        Un \key{ratio} exprime une comparaison entre plusieurs quantités.
    }[\href{https://pedagogie.ac-montpellier.fr/sites/default/files/ressources/Les\%20ratios\%20au\%20cycle\%204.pdf\#page=6}{Académie de Montpellier}]
}

\slide{}{
    \expl{}{
        Assan et Clara se partagent les 140 bonbons qu'ils ont obtenus à Halloween.
        Clara en récupère 60.
        \begin{enumerate}
            \item Dans quel ratio se sont-il partagé ces bonbons?
            \item Quel fraction des bonbons récupère Assan?
            \item Si Clara récupère 9 bonbons et que l'on reste dans le même ratio,
            combien Assan v'a-t-il en récupèrer?
        \end{enumerate}
    }
}

\slide{exo}{
    \bvspace{-0.75cm}
    \exo{}{
        \begin{enumerate}
            \item Comment partager 48 macarons entre Simon et Mandy dans le ratio $5:11$?
            \item Ahmed, Simon et Mandy se partagent des macarons dans le ratio $4:3:2$.
            Simon en a 9, combien en ont Ahmed et Mandy?
            \item Simon et Mandy ont réalisé un certain nombre de macarons dans le ratio $5:8$.
            Sachant que Mandy,
            plus expérimentée,
            a fait 66 macarons de plus que Simon,
            combien Mandy en a préparé?
        \end{enumerate}
    }[\rpmc[63]]
}
% % VARIABLES %%%
% \date{\today}
\def\longTitle{Théorème de Pythagore - Sens direct}
\def\shortTitle{\MakeUppercase{\longTitle}}

\setcounter{seq}{2}
\bseq{\longTitle}

\setgrade{4e}
\def\imgPath{enseignement/4e/theoreme-de-pythagore/sens-direct/}
%%

\def\ym{\href{https://www.maths-et-tiques.fr/telech/19Pyth1.pdf}{Yvan Monka}}

\avspace{0.1cm}

\obj{
    \item Utiliser la calculatrice pour déterminer une valeur approchée de la racine carrée d'un nombre positif.
    \item Utiliser la racine carrée d'un nombre positif en lien avec des situations géométriques.
    \item Utiliser le sens direct du Théorème de Pythagore.
}

\scn{Decouvrir l'égalité de Pythagore}
\bsec{Egalité de Pythagore}

\slide{qf}{
    \dividePage{
        \begin{enumerate}
            \item Peut-on construire un triangle $ABC$ avec:
            \begin{enumerate}
                \item $AB = 5cm \pv BC = 11cm \pv AC = 4cm$
                \item $AB = 8cm \pv BC = 9cm \pv AC = 6cm$
            \end{enumerate}
        \end{enumerate}
    }{
        \imgp{mi-c4/qf-5p426}[7cm]
    }
}

\slide{exo}{
    \bvspace{-0.5cm}
    \act{}{
        \bvspace{-0.5cm}
        \dividePage{%
            \imgp{activite-decouverte-fig}[5cm]
        }
        {
            \ifBeamer{\small}
            \begin{enumerate}
                \item Donner une expression mathématique permettant de calculer l'aire des carrés 1; 2 et 3.
                \item Découper les quatre triangles et le carré fournis en annexe.
                \item Disposer les quatre triangles dans le carré afin de former deux nouveaux carrés.
                \item Quelles sont les aires des deux carrés formés ?
            \end{enumerate}
        }[0.35]
    }[\href{https://drive.google.com/drive/folders/1ipPqxysYc8GHNOIT0u6HSi8cAnUiYfxx}{Audrey Belay}]
}

\slide{exo}{
    \begin{enumerate}\setcounter{enumi}{4}
        \item Disposer les quatre triangles dans le carré afin de former un seul carré.
        \item Quelle est l'aire de ce carré ?
        \item Quelle relation peut-on établir entre les aires des carrés 1, 2 et l'aire du carré 3 ?
        \item Quelle relation peut-on établir entre les longueurs $a$, $b$ et $c$ ?
    \end{enumerate}
    \ifArticle{Annexe:\imgp{activite-decouverte-print}[6cm]}
}

\slide{cr}{
    \sseq\ssec
    \df{}{
        Dans un triangle rectangle, le côté opposé à l'angle droit est appelé \key{hypoténuse}.
    }[\href{https://fr.wikipedia.org/wiki/Hypoténuse}{Wikipédia}]
}

\slide{cr}{
    \setboolean{showID}{false}
    \thm{de Pythagore}{
        \Si un triangle est rectangle,
        \Alors le carré de la longueur de l'hypoténuse est égal à la somme des carrés des longueurs des deux autres côtés.
        \color{black}
        \begin{center}
            \begin{tikzpicture}[scale=1]%,cap=round,>=latex]
                \coordinate (A) at (-1.5cm,-1.cm);
                \coordinate (C) at (1.5cm,-1.0cm);
                \coordinate (B) at (1.5cm,1.0cm);
                \draw (A) -- node[above] {$a$} (B) -- node[right] {$c$} (C) -- node[below] {$b$} (A);
                \draw[color = BlueViolet] (1.25cm,-1.0cm) rectangle (1.5cm,-0.75cm);
            \end{tikzpicture}\\
            \color{ForestGreen}$c^2=a^2+b^2$
        \end{center}
    }
    \setboolean{showID}{true}
}

\slide{cr}{
    \expl{}{
        Soit $ABC$ un triangle rectangle en $A$, tel que $AB = 6$, $AC = 8$, $BC=10$.
        Verifier si le triangle $ABC$ respecte bien l'égalité de Pythagore.
    }
}

\def\imgPrefix{mi-c4/exo-}
\exoSlide{3p430,4p430}[7cm][2][\mi]

\scn{Decouvrir la racine carré}

\def\imgPrefix{mi-c4/qf-}
\exoSlide{3p426}[10cm][1][\mi][qf]

\bsec{Racine carré}

\slide{exo}{
    \act{}{
        \begin{enumerate}
            \item Calculer l'aire d'un carré de côté:
            \multiColEnumerate{4}{
                \item $3\cm$ \item $12\cm$ \item $5,5\cm$ \item $x\cm$
            }
            \item Trouvé le côté d'un carré d'aire:
            \multiColEnumerate{4}{
                \item $4\cmd$ \item $25\cmd$ \item $1\cmd$ \item $169\cmd$
            }
            \item Encadrer par deux entiers le côté d'un carré d'aire $40\cmd$.
            \item Trouver le côté d'un carré d'aire $20.25\cmd$.
            \item Approcher au centième près le côté d'un carré d'aire $2\cmd$.
        \end{enumerate}
    }
}

\slide{cr}{
    \ssec
    \df{}{
        La \key{racine carrée} d'un nombre positif $x$ est l'unique nombre positif qui,
        lorsqu'il est multiplié par lui-même,
        donne $x$.
    }[\wiki{Racine_carrée}]

    \df{}{
        Un \key{carré parfait} est le carré d'un entier naturel.
    }[\wiki{Carré_parfait}]
}

\scn{Calculer une longueur grace au Théorème de Pythagore}

\def\imgPrefix{mi-c4/qf-}
\exoSlide{1p430}[8cm][1][\mi][qf]

\expl{}{
    Soit $ABC$ un triangle rectangle en $A$, tel que $AB = 3$ et $AC = 4$.
    Déterminer $BC$.
}

\slide{exo}{
    \exo{}{%
        Un professeur d'EPS trace un circuit de course à pied avec des plots :
        \begin{itemize}
            \item le plot n°2 est situé à 36 m au nord du plot n°1, qui est le plot de départ ;
            \item le plot n°3 est situé à 69 m à l'est du plot n°2 ;
            \item le plot n°4 est situé à 72 m au sud du plot n°3.
        \end{itemize}
        Chaque élève va d'un plot au suivant en ligne droite,
        et parcourt un certain nombre de fois le circuit 1-2-3-4-1.
        Il continue sur le même circuit jusqu'au plot d'arrivée,
        placé sur ce circuit de telle sorte le trajet total ait une longueur de 1,5 km.
        \\Où le professeur doit-il placer le plot d'arrivée ?
        \\Combien de fois un élève doit-il parcourir le circuit ?
    }[\href{https://eduscol.education.fr/document/17305/download\#page=8}
    {Utiliser les notions de géométrie plane pour démontrer}]
}

\slide{exo}{
    \exo{}{

    }[\href{https://blogdemaths.wordpress.com/2015/06/27/jusquou-peut-on-voir-a-lhorizon/}{Jusqu'où peut on voir à l'horizon?}]
}

% % VARIABLES %%%
\setTitle{Correction - Devoir Maison - Séquence 1}
\setGrade{6e}
\def\imgPath{enseignement/6e/geometrie-plane/points-et-droites/}
%%

\exo{44p217}{
    \imgp{dim-6e/exo-44p217}[5cm]
    \imgp{dm-44p217}[5cm]
    \color{BurntOrange} d. \color{black}
    La droite $(AC)$ semble passer par le point $P$.
}

\exo{71p221}{
    \imgp{dim-6e/exo-71p221}[5cm]
    \imgp{dm-71p221}[5cm]
    \color{BurntOrange} e. \color{black}
    $A,C$ et $D$ sont positionnés au milieu des segments qui composent les côtés du triangle $IJK$.
}

% % VARIABLES %%%
\setTitle{Correction - Devoir Maison - Séquence 1}
\setGrade{5e}
\def\imgPath{enseignement/5e/nombres-relatifs/reperage-et-comparaison/}
%%

\exo{70p198}{
    \imgp{mm-c4/exo-70p198}[6cm]
    Il y a \key{7 chemins} possible :
    \multiColEnumerate{2}{
        \item $-7,8 < -5,2 < -3 < 2$
        \item $-7,8 < -5,2 < -3 < 0 < 2$
        \item $-7,8 < -5,2 < -3 < -2 < 0 < 2$
        \item $-7,8 < -6,5 < -5,2 < -3 < 2$
        \item $-7,8 < -6,5 < -5,2 < -3 < 0 < 2$
        \item $-7,8 < -6,5 < -5,2 < -3 < -2 < 0 < 2$
        \item $-7,8 < -6,5 < -2 < 0 < 2$
    }
}

\exo{87p200}{
    \imgp{mm-c4/exo-87p200}[6cm]

    \begin{enumerate}
        \item On peut trouver l'altitude du fond du lac Ontario en enlevant sa profondeur à l'altitude de sa surface.\\
        74,2 - 244 = -169,8\\
        Le fond du lac Ontario se trouve à une altitude de -169,8 \meter.
        \item On peut trouver l'altitude du sommet des Chutes du Niagara en ajoutant leurs hauteur à l'altitude de la surface du lac Ontario.\\
        74,2 + 52 = 126.2\\
        Le sommet des Chutes du Niagara se trouve à une altitude de 126,2 \meter.
    \end{enumerate}
}
% % VARIABLES %%%
\setTitle{Correction - Devoir Maison - Séquence 1}
\setGrade{4e}
\def\imgPath{enseignement/4e/divisibilite-et-nombres-premiers/}
%%

\exo{20p17}{
    \imgp{mi-c4/exo-20p17}[6cm]

    \imgp{dm-20p17}[8cm]
}

\exo{45p19}{
    \imgp{mi-c4/exo-45p19}[6cm]
    \begin{enumerate}
        \item\begin{itemize}
            \item Pour trouver les diviseurs de $6$, on commence par le décomposer en facteurs premiers.
            \item $6 = 2 \times 3$
            \item Les diviseurs de $6$ sont donc $1;2;3$ et $6$.
            \item Or $1+2+3 = 6$.
            \item $6$ est donc bien un nombre parfait.
        \end{itemize}
        \item\begin{itemize}
            \item Pour trouver les diviseurs de $28$, on commence par le décomposer en facteurs premiers.
            \item $28 = 2 \times 2 \times 7$
            \item Les diviseurs de $6$ sont donc $1;2;7;2\times2 = 4;2\times7 = 14$ et $28$.
            \item Or $1+2+7+4+14 = 28$.
            \item $28$ est donc bien un nombre parfait.
        \end{itemize}
    \end{enumerate}
}

\exo{56p20}{
    \imgp{mi-c4/exo-56p20}[6cm]
    \begin{itemize}
        \item Le prénom Annabelle comporte $9$ lettres.
        \item Or la division euclidienne de $1000$ par $9$ donne : $1000 = 9 \times 111 + 1$.
        \item Alors à la $999$ lettre Annabelle aura écrit son prénom pour la $111^e$ fois.
        \item Et la $1000^e$ lettre sera donc un $A$.
    \end{itemize}
}

% % VARIABLES %%%
\setTitle{Interrogation - Séquence 1}
\setGrade{6e}
\thispagestyle{assignment}
%%

\def\Ona{\key{On a} }
\def\Or{\key{Or} }
\def\Donc{\key{Donc} }


\exo{}{
    La droite $(AB)$ est perpendiculaire à la droite $(CD)$,
    et la droite $(CD)$ est parallèle à la droite $(EF)$. 
    Quelle est la relation entre les droites $(AB)$ et $(EF)$ ?
}

\corr{}{
    \Ona $(CD) \prll (EF)$ et $(AB) \perp (CD)$.\\
    \Or \sialors{deux droites sont parallèles}{toute perpendiculaire à l'une
    est perpendiculaire à l'autre.} \\ 
    \Donc $(AB) \perp (EF)$.
}
% % VARIABLES %%%
\setTitle{Interrogation - Séquence 2}
\setGrade{5e}
\thispagestyle{assignment}
%%

% \hint{Calculatrice autorisée}
\calculator

\def\cW{1.5cm}
\exo{Les tableaux ci-dessous représentent-ils une situation de proportionnalité ?}{\vspace{-0.5cm}%
    \multiColEnumerate{2}{
        % \item \propTable{5}{8}{7,5}{12}
        \item \begin{tabular}{|C{\cW}|C{\cW}|C{\cW}|}
            \hline
            $36$ & $13,95$ & $40,5$\\
            \hline
            $8$ & $3,1$ & $9$\\
            \hline
        \end{tabular}
        \item \begin{tabular}{|C{\cW}|C{\cW}|}
            \hline
            $651,3$ & $128$\\
            \hline
            $100,2$ & $20$\\
            \hline
        \end{tabular}
    }
}

\corr{}{
    \begin{enumerate}
        \item \begin{align*}
            \frac{36}{8} = 4,5\qquad
            \frac{13,95}{3,1} &= 4,5\qquad
            \frac{40,5}{9} = 4,5\\
            \ialors \frac{36}{8} = \frac{13,95}{3,1} &= \frac{40,5}{9}
        \end{align*}
        L'\key{égalité} des quotients indique qu'\key{il s'agit bien} d'une situation de proportionnalité.
        \item \begin{align*}
            \frac{651,3}{100,2} = 6,5\qquad
            \frac{128}{20} &= 6,4\qquad
            \ialors \frac{651,3}{100,2} \neq \frac{128}{20}
        \end{align*}
        L'\key{inégalité} des quotients indique qu'\key{il ne sagit pas} d'une situation de proportionnalité.
    \end{enumerate}
}

% % VARIABLES %%%
\setTitle{Interrogation - Séquence 1}
\setGrade{4e}
\thispagestyle{assignment}
%%

\exo{}{
    \begin{enumerate}
        \item Donner la décomposition en facteurs premiers de : $1980$
        \item Écrire sous forme de fraction irréductible : $\dfrac{3150}{84}$
    \end{enumerate}
}

\corr{}{
    \begin{enumerate}
        \item \begin{align*}
            1980 &= 10 \times 198\\
            &= 2 \times 5 \times 2 \times 99\\
            &= 2 \times 2 \times 5 \times 9 \times 11\\
            &= 2 \times 2 \times 3 \times 3 \times 5 \times 11
        \end{align*}
        \item \begin{align*}
            3150 &= 2 \times 3 \times 3 \times 5 \times 5 \times 7\\
            \iet 84 &= 2 \times 2 \times 3 \times 7\\
            \ialors \dfrac{3150}{84} &= 
            \dfrac{\cancel{2} \times \cancel{3} \times 3 \times 5 \times 5 \times \cancel{7}}
            {\cancel{2} \times 2 \times \cancel{3} \times \cancel{7}}\\
            &= \dfrac{3 \times 5 \times 5}{2}\\
            & = \dfrac{75}{2}
        \end{align*}
    \end{enumerate}
}

% % VARIABLES %%%
\setTitle{Interrogation - Entrainement - Séquence 2}
\setGrade{5e}
%%

\calculator

\exo{Les tableaux ci-dessous représentent-ils une situation de proportionnalité ?}{%
\multiColEnumerate{2}{
\item\begin{tabular}{|C{1.5cm}|C{1.5cm}|C{1.5cm}|C{1.5cm}|}
    \hline
    89.08 & 147.39 & 18.53 & 23.46\\
    \hline
    52.4 & 86.7 & 10.9 & 13.8\\
    \hline
\end{tabular}

\item\begin{tabular}{|C{1.5cm}|C{1.5cm}|C{1.5cm}|}
    \hline
    277.5 & 7.83 & 139.2\\
    \hline
    92.5 & 2.7 & 46.4\\
    \hline
\end{tabular}

\item\begin{tabular}{|C{1.5cm}|C{1.5cm}|C{1.5cm}|}
    \hline
    124 & 19.26 & 172.44\\
    \hline
    77.5 & 10.7 & 95.8\\
    \hline
\end{tabular}

\item\begin{tabular}{|C{1.5cm}|C{1.5cm}|}
    \hline
    469.44 & 171.36\\
    \hline
    97.8 & 35.7\\
    \hline
\end{tabular}

\item\begin{tabular}{|C{1.5cm}|C{1.5cm}|C{1.5cm}|C{1.5cm}|}
    \hline
    18.16 & 6.32 & 64.48 & 38.56\\
    \hline
    22.7 & 7.9 & 80.6 & 48.2\\
    \hline
\end{tabular}

\item\begin{tabular}{|C{1.5cm}|C{1.5cm}|C{1.5cm}|C{1.5cm}|}
    \hline
    56.84 & 281.06 & 389.76 & 331.73\\
    \hline
    11.6 & 61.1 & 81.2 & 67.7\\
    \hline
\end{tabular}

\item\begin{tabular}{|C{1.5cm}|C{1.5cm}|C{1.5cm}|C{1.5cm}|}
    \hline
    101.83 & 29.07 & 80.07 & 109.65\\
    \hline
    59.9 & 17.1 & 47.1 & 64.5\\
    \hline
\end{tabular}

\item\begin{tabular}{|C{1.5cm}|C{1.5cm}|}
    \hline
    214.5 & 123.25\\
    \hline
    85.8 & 49.3\\
    \hline
\end{tabular}

\item\begin{tabular}{|C{1.5cm}|C{1.5cm}|C{1.5cm}|}
    \hline
    261.8 & 108.16 & 150.04\\
    \hline
    77 & 33.8 & 48.4\\
    \hline
\end{tabular}

\item\begin{tabular}{|C{1.5cm}|C{1.5cm}|C{1.5cm}|C{1.5cm}|}
    \hline
    58 & 101.4 & 33.2 & 118.2\\
    \hline
    29 & 50.7 & 16.6 & 59.1\\
    \hline
\end{tabular}

\item\begin{tabular}{|C{1.5cm}|C{1.5cm}|C{1.5cm}|}
    \hline
    72.1 & 197.28 & 336.6\\
    \hline
    20.6 & 54.8 & 93.5\\
    \hline
\end{tabular}

\item\begin{tabular}{|C{1.5cm}|C{1.5cm}|C{1.5cm}|}
    \hline
    29.25 & 66.13 & 127.36\\
    \hline
    19.5 & 38.9 & 79.6\\
    \hline
\end{tabular}

\item\begin{tabular}{|C{1.5cm}|C{1.5cm}|}
    \hline
    71.25 & 36.9\\
    \hline
    47.5 & 24.6\\
    \hline
\end{tabular}

\item\begin{tabular}{|C{1.5cm}|C{1.5cm}|C{1.5cm}|C{1.5cm}|}
    \hline
    4.86 & 8.67 & 20.28 & 20.55\\
    \hline
    8.1 & 28.9 & 67.6 & 41.1\\
    \hline
\end{tabular}

\item\begin{tabular}{|C{1.5cm}|C{1.5cm}|}
    \hline
    10.92 & 5.07\\
    \hline
    36.4 & 16.9\\
    \hline
\end{tabular}

\item\begin{tabular}{|C{1.5cm}|C{1.5cm}|C{1.5cm}|C{1.5cm}|}
    \hline
    123.28 & 152.55 & 222.18 & 280.14\\
    \hline
    26.8 & 33.9 & 48.3 & 60.9\\
    \hline
\end{tabular}

\item\begin{tabular}{|C{1.5cm}|C{1.5cm}|}
    \hline
    35.28 & 58.14\\
    \hline
    19.6 & 32.3\\
    \hline
\end{tabular}

\item\begin{tabular}{|C{1.5cm}|C{1.5cm}|C{1.5cm}|C{1.5cm}|}
    \hline
    51 & 121.75 & 178.36 & 125.58\\
    \hline
    20.4 & 48.7 & 68.6 & 54.6\\
    \hline
\end{tabular}

\item\begin{tabular}{|C{1.5cm}|C{1.5cm}|}
    \hline
    209.96 & 40.23\\
    \hline
    72.4 & 14.9\\
    \hline
\end{tabular}

\item\begin{tabular}{|C{1.5cm}|C{1.5cm}|C{1.5cm}|C{1.5cm}|}
    \hline
    94.64 & 13.77 & 279 & 378.93\\
    \hline
    18.2 & 2.7 & 55.8 & 74.3\\
    \hline
\end{tabular}

\item\begin{tabular}{|C{1.5cm}|C{1.5cm}|C{1.5cm}|}
    \hline
    21.76 & 7.48 & 17.6\\
    \hline
    12.8 & 4.4 & 11\\
    \hline
\end{tabular}

\item\begin{tabular}{|C{1.5cm}|C{1.5cm}|}
    \hline
    369 & 43\\
    \hline
    90 & 10\\
    \hline
\end{tabular}

\item\begin{tabular}{|C{1.5cm}|C{1.5cm}|C{1.5cm}|}
    \hline
    267.43 & 339.02 & 401.58\\
    \hline
    56.9 & 73.7 & 87.3\\
    \hline
\end{tabular}

\item\begin{tabular}{|C{1.5cm}|C{1.5cm}|}
    \hline
    16.53 & 178.6\\
    \hline
    8.7 & 94\\
    \hline
\end{tabular}

\item\begin{tabular}{|C{1.5cm}|C{1.5cm}|C{1.5cm}|C{1.5cm}|}
    \hline
    36.96 & 40.59 & 26.4 & 22.77\\
    \hline
    11.2 & 12.3 & 8 & 6.9\\
    \hline
\end{tabular}

\item\begin{tabular}{|C{1.5cm}|C{1.5cm}|C{1.5cm}|}
    \hline
    167.67 & 86.02 & 24.61\\
    \hline
    72.9 & 37.4 & 10.7\\
    \hline
\end{tabular}

\item\begin{tabular}{|C{1.5cm}|C{1.5cm}|C{1.5cm}|C{1.5cm}|}
    \hline
    159.16 & 95.76 & 220.56 & 128.64\\
    \hline
    69.2 & 39.9 & 91.9 & 53.6\\
    \hline
\end{tabular}

\item\begin{tabular}{|C{1.5cm}|C{1.5cm}|C{1.5cm}|}
    \hline
    327.32 & 328.8 & 210.21\\
    \hline
    66.8 & 68.5 & 42.9\\
    \hline
\end{tabular}

\item\begin{tabular}{|C{1.5cm}|C{1.5cm}|C{1.5cm}|C{1.5cm}|}
    \hline
    252.48 & 92.8 & 205.76 & 48\\
    \hline
    78.9 & 29 & 64.3 & 15\\
    \hline
\end{tabular}

\item\begin{tabular}{|C{1.5cm}|C{1.5cm}|C{1.5cm}|}
    \hline
    130.41 & 20.46 & 169.4\\
    \hline
    62.1 & 9.3 & 77\\
    \hline
\end{tabular}

}}

\newpage

\corr{}{%
\begin{enumerate}\item\begin{align*}
\dfrac{89.08}{52.4} = 1.7\qquad \dfrac{147.39}{86.7} = 1.7\qquad \dfrac{18.53}{10.9} = 1.7\qquad \dfrac{23.46}{13.8} = 1.7\qquad 
\end{align*}
L'\key{égalité} des quotients indique qu'\key{il s'agit bien} d'une situation de proportionnalité.

\item\begin{align*}
\dfrac{277.5}{92.5} = 3\qquad \dfrac{7.83}{2.7} = 2.9\qquad \dfrac{139.2}{46.4} = 3\qquad 
\end{align*}
L'\key{inégalité} des quotients indique qu'\key{il ne sagit pas} d'une situation de proportionnalité.

\item\begin{align*}
\dfrac{124}{77.5} = 1.6\qquad \dfrac{19.26}{10.7} = 1.8\qquad \dfrac{172.44}{95.8} = 1.8\qquad 
\end{align*}
L'\key{inégalité} des quotients indique qu'\key{il ne sagit pas} d'une situation de proportionnalité.

\item\begin{align*}
\dfrac{469.44}{97.8} = 4.8\qquad \dfrac{171.36}{35.7} = 4.8\qquad 
\end{align*}
L'\key{égalité} des quotients indique qu'\key{il s'agit bien} d'une situation de proportionnalité.

\item\begin{align*}
\dfrac{18.16}{22.7} = 0.8\qquad \dfrac{6.32}{7.9} = 0.8\qquad \dfrac{64.48}{80.6} = 0.8\qquad \dfrac{38.56}{48.2} = 0.8\qquad 
\end{align*}
L'\key{égalité} des quotients indique qu'\key{il s'agit bien} d'une situation de proportionnalité.

\item\begin{align*}
\dfrac{56.84}{11.6} = 4.9\qquad \dfrac{281.06}{61.1} = 4.6\qquad \dfrac{389.76}{81.2} = 4.8\qquad \dfrac{331.73}{67.7} = 4.9\qquad 
\end{align*}
L'\key{inégalité} des quotients indique qu'\key{il ne sagit pas} d'une situation de proportionnalité.

\item\begin{align*}
\dfrac{101.83}{59.9} = 1.7\qquad \dfrac{29.07}{17.1} = 1.7\qquad \dfrac{80.07}{47.1} = 1.7\qquad \dfrac{109.65}{64.5} = 1.7\qquad 
\end{align*}
L'\key{égalité} des quotients indique qu'\key{il s'agit bien} d'une situation de proportionnalité.

\item\begin{align*}
\dfrac{214.5}{85.8} = 2.5\qquad \dfrac{123.25}{49.3} = 2.5\qquad 
\end{align*}
L'\key{égalité} des quotients indique qu'\key{il s'agit bien} d'une situation de proportionnalité.

\item\begin{align*}
\dfrac{261.8}{77} = 3.4\qquad \dfrac{108.16}{33.8} = 3.2\qquad \dfrac{150.04}{48.4} = 3.1\qquad 
\end{align*}
L'\key{inégalité} des quotients indique qu'\key{il ne sagit pas} d'une situation de proportionnalité.

\item\begin{align*}
\dfrac{58}{29} = 2\qquad \dfrac{101.4}{50.7} = 2\qquad \dfrac{33.2}{16.6} = 2\qquad \dfrac{118.2}{59.1} = 2\qquad 
\end{align*}
L'\key{égalité} des quotients indique qu'\key{il s'agit bien} d'une situation de proportionnalité.

\item\begin{align*}
\dfrac{72.1}{20.6} = 3.5\qquad \dfrac{197.28}{54.8} = 3.6\qquad \dfrac{336.6}{93.5} = 3.6\qquad 
\end{align*}
L'\key{inégalité} des quotients indique qu'\key{il ne sagit pas} d'une situation de proportionnalité.

\item\begin{align*}
\dfrac{29.25}{19.5} = 1.5\qquad \dfrac{66.13}{38.9} = 1.7\qquad \dfrac{127.36}{79.6} = 1.6\qquad 
\end{align*}
L'\key{inégalité} des quotients indique qu'\key{il ne sagit pas} d'une situation de proportionnalité.

\item\begin{align*}
\dfrac{71.25}{47.5} = 1.5\qquad \dfrac{36.9}{24.6} = 1.5\qquad 
\end{align*}
L'\key{inégalité} des quotients indique qu'\key{il ne sagit pas} d'une situation de proportionnalité.

\item\begin{align*}
\dfrac{4.86}{8.1} = 0.6\qquad \dfrac{8.67}{28.9} = 0.3\qquad \dfrac{20.28}{67.6} = 0.3\qquad \dfrac{20.55}{41.1} = 0.5\qquad 
\end{align*}
L'\key{inégalité} des quotients indique qu'\key{il ne sagit pas} d'une situation de proportionnalité.

\item\begin{align*}
\dfrac{10.92}{36.4} = 0.3\qquad \dfrac{5.07}{16.9} = 0.3\qquad 
\end{align*}
L'\key{égalité} des quotients indique qu'\key{il s'agit bien} d'une situation de proportionnalité.

\item\begin{align*}
\dfrac{123.28}{26.8} = 4.6\qquad \dfrac{152.55}{33.9} = 4.5\qquad \dfrac{222.18}{48.3} = 4.6\qquad \dfrac{280.14}{60.9} = 4.6\qquad 
\end{align*}
L'\key{inégalité} des quotients indique qu'\key{il ne sagit pas} d'une situation de proportionnalité.

\item\begin{align*}
\dfrac{35.28}{19.6} = 1.8\qquad \dfrac{58.14}{32.3} = 1.8\qquad 
\end{align*}
L'\key{égalité} des quotients indique qu'\key{il s'agit bien} d'une situation de proportionnalité.

\item\begin{align*}
\dfrac{51}{20.4} = 2.5\qquad \dfrac{121.75}{48.7} = 2.5\qquad \dfrac{178.36}{68.6} = 2.6\qquad \dfrac{125.58}{54.6} = 2.3\qquad 
\end{align*}
L'\key{inégalité} des quotients indique qu'\key{il ne sagit pas} d'une situation de proportionnalité.

\item\begin{align*}
\dfrac{209.96}{72.4} = 2.9\qquad \dfrac{40.23}{14.9} = 2.7\qquad 
\end{align*}
L'\key{inégalité} des quotients indique qu'\key{il ne sagit pas} d'une situation de proportionnalité.

\item\begin{align*}
\dfrac{94.64}{18.2} = 5.2\qquad \dfrac{13.77}{2.7} = 5.1\qquad \dfrac{279}{55.8} = 5\qquad \dfrac{378.93}{74.3} = 5.1\qquad 
\end{align*}
L'\key{inégalité} des quotients indique qu'\key{il ne sagit pas} d'une situation de proportionnalité.

\item\begin{align*}
\dfrac{21.76}{12.8} = 1.7\qquad \dfrac{7.48}{4.4} = 1.7\qquad \dfrac{17.6}{11} = 1.6\qquad 
\end{align*}
L'\key{inégalité} des quotients indique qu'\key{il ne sagit pas} d'une situation de proportionnalité.

\item\begin{align*}
\dfrac{369}{90} = 4.1\qquad \dfrac{43}{10} = 4.3\qquad 
\end{align*}
L'\key{inégalité} des quotients indique qu'\key{il ne sagit pas} d'une situation de proportionnalité.

\item\begin{align*}
\dfrac{267.43}{56.9} = 4.7\qquad \dfrac{339.02}{73.7} = 4.6\qquad \dfrac{401.58}{87.3} = 4.6\qquad 
\end{align*}
L'\key{inégalité} des quotients indique qu'\key{il ne sagit pas} d'une situation de proportionnalité.

\item\begin{align*}
\dfrac{16.53}{8.7} = 1.9\qquad \dfrac{178.6}{94} = 1.9\qquad 
\end{align*}
L'\key{égalité} des quotients indique qu'\key{il s'agit bien} d'une situation de proportionnalité.

\item\begin{align*}
\dfrac{36.96}{11.2} = 3.3\qquad \dfrac{40.59}{12.3} = 3.3\qquad \dfrac{26.4}{8} = 3.3\qquad \dfrac{22.77}{6.9} = 3.3\qquad 
\end{align*}
L'\key{égalité} des quotients indique qu'\key{il s'agit bien} d'une situation de proportionnalité.

\item\begin{align*}
\dfrac{167.67}{72.9} = 2.3\qquad \dfrac{86.02}{37.4} = 2.3\qquad \dfrac{24.61}{10.7} = 2.3\qquad 
\end{align*}
L'\key{égalité} des quotients indique qu'\key{il s'agit bien} d'une situation de proportionnalité.

\item\begin{align*}
\dfrac{159.16}{69.2} = 2.3\qquad \dfrac{95.76}{39.9} = 2.4\qquad \dfrac{220.56}{91.9} = 2.4\qquad \dfrac{128.64}{53.6} = 2.4\qquad 
\end{align*}
L'\key{inégalité} des quotients indique qu'\key{il ne sagit pas} d'une situation de proportionnalité.

\item\begin{align*}
\dfrac{327.32}{66.8} = 4.9\qquad \dfrac{328.8}{68.5} = 4.8\qquad \dfrac{210.21}{42.9} = 4.9\qquad 
\end{align*}
L'\key{inégalité} des quotients indique qu'\key{il ne sagit pas} d'une situation de proportionnalité.

\item\begin{align*}
\dfrac{252.48}{78.9} = 3.2\qquad \dfrac{92.8}{29} = 3.2\qquad \dfrac{205.76}{64.3} = 3.2\qquad \dfrac{48}{15} = 3.2\qquad 
\end{align*}
L'\key{égalité} des quotients indique qu'\key{il s'agit bien} d'une situation de proportionnalité.

\item\begin{align*}
\dfrac{130.41}{62.1} = 2.1\qquad \dfrac{20.46}{9.3} = 2.2\qquad \dfrac{169.4}{77} = 2.2\qquad 
\end{align*}
L'\key{inégalité} des quotients indique qu'\key{il ne sagit pas} d'une situation de proportionnalité.

\end{enumerate}}
% % VARIABLES %%%
\setTitle{Interrogation - Entrainement - Séquence 1}
\setGrade{4e}
%%

\exo{Écrire sous forme de fractions irréductibles :}{%
\multiColEnumerate{6}{
\item$\dfrac{5500}{275}$

\item$\dfrac{33}{72}$

\item$\dfrac{2548}{8112}$

\item$\dfrac{2184}{21}$

\item$\dfrac{99372}{4840}$

\item$\dfrac{4620}{130}$

\item$\dfrac{770}{572}$

\item$\dfrac{66}{6}$

\item$\dfrac{70}{91}$

\item$\dfrac{1618617}{78}$

\item$\dfrac{5544}{4158}$

\item$\dfrac{12}{622545}$

\item$\dfrac{25}{42}$

\item$\dfrac{784}{33}$

\item$\dfrac{248430}{14014}$

\item$\dfrac{4914}{220}$

\item$\dfrac{660}{8250}$

\item$\dfrac{7280}{1540}$

\item$\dfrac{76050}{3780}$

\item$\dfrac{507}{1087515}$

\item$\dfrac{10}{588}$

\item$\dfrac{3675}{6}$

\item$\dfrac{91}{572}$

\item$\dfrac{1375}{5005}$

\item$\dfrac{429}{104}$

\item$\dfrac{362505}{9438}$

\item$\dfrac{15}{1386}$

\item$\dfrac{275275}{1001}$

\item$\dfrac{26}{350350}$

\item$\dfrac{2556125}{248430}$

\item$\dfrac{102245}{231}$

\item$\dfrac{5250}{2860}$

\item$\dfrac{108}{22}$

\item$\dfrac{330}{33957}$

\item$\dfrac{726}{420}$

\item$\dfrac{210}{27}$

\item$\dfrac{308}{8316}$

\item$\dfrac{10}{484}$

\item$\dfrac{78}{372645}$

\item$\dfrac{5148}{5460}$

\item$\dfrac{1274}{7872865}$

\item$\dfrac{195}{2640}$

\item$\dfrac{65}{520}$

\item$\dfrac{168}{22}$

\item$\dfrac{84084}{1144}$

\item$\dfrac{280}{13552}$

\item$\dfrac{11550}{1386}$

\item$\dfrac{15}{153790}$

\item$\dfrac{1470}{1617}$

\item$\dfrac{546}{504}$

\item$\dfrac{429}{2600}$

\item$\dfrac{8}{42588}$

\item$\dfrac{2366}{3300}$

\item$\dfrac{20}{5775}$

\item$\dfrac{819}{390}$

\item$\dfrac{175}{6}$

\item$\dfrac{650}{10010}$

\item$\dfrac{3120}{168}$

\item$\dfrac{13860}{1680}$

\item$\dfrac{50820}{10}$

\item$\dfrac{98}{11011}$

\item$\dfrac{308}{1014}$

\item$\dfrac{51975}{48}$

\item$\dfrac{35}{676}$

\item$\dfrac{4}{2860}$

\item$\dfrac{84}{280}$

\item$\dfrac{3432}{484}$

\item$\dfrac{5005}{5915}$

\item$\dfrac{1014}{14}$

\item$\dfrac{12}{630}$

\item$\dfrac{1848}{364}$

\item$\dfrac{4}{1680}$

\item$\dfrac{15288}{296450}$

\item$\dfrac{14520}{286}$

\item$\dfrac{195}{22}$

\item$\dfrac{286}{6}$

\item$\dfrac{352}{330}$

\item$\dfrac{2310}{6468}$

\item$\dfrac{650650}{80}$

\item$\dfrac{2184}{27027}$

\item$\dfrac{252}{10164}$

\item$\dfrac{14014}{150}$

\item$\dfrac{10}{14}$

\item$\dfrac{4620}{15}$

\item$\dfrac{1287}{16}$

\item$\dfrac{85995}{936}$

\item$\dfrac{14742}{385}$

\item$\dfrac{3003}{1274}$

\item$\dfrac{37180}{5460}$

\item$\dfrac{975975}{4}$

\item$\dfrac{70}{131820}$

\item$\dfrac{9800}{140}$

\item$\dfrac{10}{6468}$

\item$\dfrac{15400}{700}$

\item$\dfrac{210}{30}$

\item$\dfrac{33}{780}$

\item$\dfrac{32340}{217503}$

\item$\dfrac{792}{165}$

\item$\dfrac{126}{264}$

\item$\dfrac{936}{1320}$

}}

\newpage

\corr{}{%
\multiColEnumerate{2}{
\item\begin{align*}
    \dfrac{5500}{275} &=
    \dfrac{2 \times 2 \times \cancel{5} \times \cancel{5} \times 5 \times \cancel{11}}
    {\cancel{5} \times \cancel{5} \times \cancel{11}}\\ &=
    \dfrac{2 \times 2 \times 5}
    {1} =
    \dfrac{20}{1}
    \end{align*}

\item\begin{align*}
    \dfrac{33}{72} &=
    \dfrac{\cancel{3} \times 11}
    {2 \times 2 \times 2 \times \cancel{3} \times 3}\\ &=
    \dfrac{11}
    {2 \times 2 \times 2 \times 3} =
    \dfrac{11}{24}
    \end{align*}

\item\begin{align*}
    \dfrac{2548}{8112} &=
    \dfrac{\cancel{2} \times \cancel{2} \times 7 \times 7 \times \cancel{13}}
    {\cancel{2} \times \cancel{2} \times 2 \times 2 \times 3 \times \cancel{13} \times 13}\\ &=
    \dfrac{7 \times 7}
    {2 \times 2 \times 3 \times 13} =
    \dfrac{49}{156}
    \end{align*}

\item\begin{align*}
    \dfrac{2184}{21} &=
    \dfrac{2 \times 2 \times 2 \times \cancel{3} \times \cancel{7} \times 13}
    {\cancel{3} \times \cancel{7}}\\ &=
    \dfrac{2 \times 2 \times 2 \times 13}
    {1} =
    \dfrac{104}{1}
    \end{align*}

\item\begin{align*}
    \dfrac{99372}{4840} &=
    \dfrac{\cancel{2} \times \cancel{2} \times 3 \times 7 \times 7 \times 13 \times 13}
    {\cancel{2} \times \cancel{2} \times 2 \times 5 \times 11 \times 11}\\ &=
    \dfrac{3 \times 7 \times 7 \times 13 \times 13}
    {2 \times 5 \times 11 \times 11} =
    \dfrac{24843}{1210}
    \end{align*}

\item\begin{align*}
    \dfrac{4620}{130} &=
    \dfrac{\cancel{2} \times 2 \times 3 \times \cancel{5} \times 7 \times 11}
    {\cancel{2} \times \cancel{5} \times 13}\\ &=
    \dfrac{2 \times 3 \times 7 \times 11}
    {13} =
    \dfrac{462}{13}
    \end{align*}

\item\begin{align*}
    \dfrac{770}{572} &=
    \dfrac{\cancel{2} \times 5 \times 7 \times \cancel{11}}
    {\cancel{2} \times 2 \times \cancel{11} \times 13}\\ &=
    \dfrac{5 \times 7}
    {2 \times 13} =
    \dfrac{35}{26}
    \end{align*}

\item\begin{align*}
    \dfrac{66}{6} &=
    \dfrac{\cancel{2} \times \cancel{3} \times 11}
    {\cancel{2} \times \cancel{3}}\\ &=
    \dfrac{11}
    {1} =
    \dfrac{11}{1}
    \end{align*}

\item\begin{align*}
    \dfrac{70}{91} &=
    \dfrac{2 \times 5 \times \cancel{7}}
    {\cancel{7} \times 13}\\ &=
    \dfrac{2 \times 5}
    {13} =
    \dfrac{10}{13}
    \end{align*}

\item\begin{align*}
    \dfrac{1618617}{78} &=
    \dfrac{\cancel{3} \times 7 \times 7 \times 7 \times 11 \times 11 \times \cancel{13}}
    {2 \times \cancel{3} \times \cancel{13}}\\ &=
    \dfrac{7 \times 7 \times 7 \times 11 \times 11}
    {2} =
    \dfrac{41503}{2}
    \end{align*}

\item\begin{align*}
    \dfrac{5544}{4158} &=
    \dfrac{\cancel{2} \times 2 \times 2 \times \cancel{3} \times \cancel{3} \times \cancel{7} \times \cancel{11}}
    {\cancel{2} \times \cancel{3} \times \cancel{3} \times 3 \times \cancel{7} \times \cancel{11}}\\ &=
    \dfrac{2 \times 2}
    {3} =
    \dfrac{4}{3}
    \end{align*}

\item\begin{align*}
    \dfrac{12}{622545} &=
    \dfrac{2 \times 2 \times \cancel{3}}
    {\cancel{3} \times 5 \times 7 \times 7 \times 7 \times 11 \times 11}\\ &=
    \dfrac{2 \times 2}
    {5 \times 7 \times 7 \times 7 \times 11 \times 11} =
    \dfrac{4}{207515}
    \end{align*}

\item\begin{align*}
    \dfrac{25}{42} &=
    \dfrac{5 \times 5}
    {2 \times 3 \times 7}\\ &=
    \dfrac{5 \times 5}
    {2 \times 3 \times 7} =
    \dfrac{25}{42}
    \end{align*}

\item\begin{align*}
    \dfrac{784}{33} &=
    \dfrac{2 \times 2 \times 2 \times 2 \times 7 \times 7}
    {3 \times 11}\\ &=
    \dfrac{2 \times 2 \times 2 \times 2 \times 7 \times 7}
    {3 \times 11} =
    \dfrac{784}{33}
    \end{align*}

\item\begin{align*}
    \dfrac{248430}{14014} &=
    \dfrac{\cancel{2} \times 3 \times 5 \times \cancel{7} \times \cancel{7} \times \cancel{13} \times 13}
    {\cancel{2} \times \cancel{7} \times \cancel{7} \times 11 \times \cancel{13}}\\ &=
    \dfrac{3 \times 5 \times 13}
    {11} =
    \dfrac{195}{11}
    \end{align*}

\item\begin{align*}
    \dfrac{4914}{220} &=
    \dfrac{\cancel{2} \times 3 \times 3 \times 3 \times 7 \times 13}
    {\cancel{2} \times 2 \times 5 \times 11}\\ &=
    \dfrac{3 \times 3 \times 3 \times 7 \times 13}
    {2 \times 5 \times 11} =
    \dfrac{2457}{110}
    \end{align*}

\item\begin{align*}
    \dfrac{660}{8250} &=
    \dfrac{\cancel{2} \times 2 \times \cancel{3} \times \cancel{5} \times \cancel{11}}
    {\cancel{2} \times \cancel{3} \times \cancel{5} \times 5 \times 5 \times \cancel{11}}\\ &=
    \dfrac{2}
    {5 \times 5} =
    \dfrac{2}{25}
    \end{align*}

\item\begin{align*}
    \dfrac{7280}{1540} &=
    \dfrac{\cancel{2} \times \cancel{2} \times 2 \times 2 \times \cancel{5} \times \cancel{7} \times 13}
    {\cancel{2} \times \cancel{2} \times \cancel{5} \times \cancel{7} \times 11}\\ &=
    \dfrac{2 \times 2 \times 13}
    {11} =
    \dfrac{52}{11}
    \end{align*}

\item\begin{align*}
    \dfrac{76050}{3780} &=
    \dfrac{\cancel{2} \times \cancel{3} \times \cancel{3} \times \cancel{5} \times 5 \times 13 \times 13}
    {\cancel{2} \times 2 \times \cancel{3} \times \cancel{3} \times 3 \times \cancel{5} \times 7}\\ &=
    \dfrac{5 \times 13 \times 13}
    {2 \times 3 \times 7} =
    \dfrac{845}{42}
    \end{align*}

\item\begin{align*}
    \dfrac{507}{1087515} &=
    \dfrac{\cancel{3} \times \cancel{13} \times \cancel{13}}
    {\cancel{3} \times 3 \times 5 \times 11 \times \cancel{13} \times \cancel{13} \times 13}\\ &=
    \dfrac{1}
    {3 \times 5 \times 11 \times 13} =
    \dfrac{1}{2145}
    \end{align*}

\item\begin{align*}
    \dfrac{10}{588} &=
    \dfrac{\cancel{2} \times 5}
    {\cancel{2} \times 2 \times 3 \times 7 \times 7}\\ &=
    \dfrac{5}
    {2 \times 3 \times 7 \times 7} =
    \dfrac{5}{294}
    \end{align*}

\item\begin{align*}
    \dfrac{3675}{6} &=
    \dfrac{\cancel{3} \times 5 \times 5 \times 7 \times 7}
    {2 \times \cancel{3}}\\ &=
    \dfrac{5 \times 5 \times 7 \times 7}
    {2} =
    \dfrac{1225}{2}
    \end{align*}

\item\begin{align*}
    \dfrac{91}{572} &=
    \dfrac{7 \times \cancel{13}}
    {2 \times 2 \times 11 \times \cancel{13}}\\ &=
    \dfrac{7}
    {2 \times 2 \times 11} =
    \dfrac{7}{44}
    \end{align*}

\item\begin{align*}
    \dfrac{1375}{5005} &=
    \dfrac{\cancel{5} \times 5 \times 5 \times \cancel{11}}
    {\cancel{5} \times 7 \times \cancel{11} \times 13}\\ &=
    \dfrac{5 \times 5}
    {7 \times 13} =
    \dfrac{25}{91}
    \end{align*}

\item\begin{align*}
    \dfrac{429}{104} &=
    \dfrac{3 \times 11 \times \cancel{13}}
    {2 \times 2 \times 2 \times \cancel{13}}\\ &=
    \dfrac{3 \times 11}
    {2 \times 2 \times 2} =
    \dfrac{33}{8}
    \end{align*}

\item\begin{align*}
    \dfrac{362505}{9438} &=
    \dfrac{\cancel{3} \times 5 \times \cancel{11} \times \cancel{13} \times 13 \times 13}
    {2 \times \cancel{3} \times \cancel{11} \times 11 \times \cancel{13}}\\ &=
    \dfrac{5 \times 13 \times 13}
    {2 \times 11} =
    \dfrac{845}{22}
    \end{align*}

\item\begin{align*}
    \dfrac{15}{1386} &=
    \dfrac{\cancel{3} \times 5}
    {2 \times \cancel{3} \times 3 \times 7 \times 11}\\ &=
    \dfrac{5}
    {2 \times 3 \times 7 \times 11} =
    \dfrac{5}{462}
    \end{align*}

\item\begin{align*}
    \dfrac{275275}{1001} &=
    \dfrac{5 \times 5 \times \cancel{7} \times \cancel{11} \times 11 \times \cancel{13}}
    {\cancel{7} \times \cancel{11} \times \cancel{13}}\\ &=
    \dfrac{5 \times 5 \times 11}
    {1} =
    \dfrac{275}{1}
    \end{align*}

\item\begin{align*}
    \dfrac{26}{350350} &=
    \dfrac{\cancel{2} \times \cancel{13}}
    {\cancel{2} \times 5 \times 5 \times 7 \times 7 \times 11 \times \cancel{13}}\\ &=
    \dfrac{1}
    {5 \times 5 \times 7 \times 7 \times 11} =
    \dfrac{1}{13475}
    \end{align*}

\item\begin{align*}
    \dfrac{2556125}{248430} &=
    \dfrac{\cancel{5} \times 5 \times 5 \times 11 \times 11 \times \cancel{13} \times \cancel{13}}
    {2 \times 3 \times \cancel{5} \times 7 \times 7 \times \cancel{13} \times \cancel{13}}\\ &=
    \dfrac{5 \times 5 \times 11 \times 11}
    {2 \times 3 \times 7 \times 7} =
    \dfrac{3025}{294}
    \end{align*}

\item\begin{align*}
    \dfrac{102245}{231} &=
    \dfrac{5 \times \cancel{11} \times 11 \times 13 \times 13}
    {3 \times 7 \times \cancel{11}}\\ &=
    \dfrac{5 \times 11 \times 13 \times 13}
    {3 \times 7} =
    \dfrac{9295}{21}
    \end{align*}

\item\begin{align*}
    \dfrac{5250}{2860} &=
    \dfrac{\cancel{2} \times 3 \times \cancel{5} \times 5 \times 5 \times 7}
    {\cancel{2} \times 2 \times \cancel{5} \times 11 \times 13}\\ &=
    \dfrac{3 \times 5 \times 5 \times 7}
    {2 \times 11 \times 13} =
    \dfrac{525}{286}
    \end{align*}

\item\begin{align*}
    \dfrac{108}{22} &=
    \dfrac{\cancel{2} \times 2 \times 3 \times 3 \times 3}
    {\cancel{2} \times 11}\\ &=
    \dfrac{2 \times 3 \times 3 \times 3}
    {11} =
    \dfrac{54}{11}
    \end{align*}

\item\begin{align*}
    \dfrac{330}{33957} &=
    \dfrac{2 \times \cancel{3} \times 5 \times \cancel{11}}
    {\cancel{3} \times 3 \times 7 \times 7 \times 7 \times \cancel{11}}\\ &=
    \dfrac{2 \times 5}
    {3 \times 7 \times 7 \times 7} =
    \dfrac{10}{1029}
    \end{align*}

\item\begin{align*}
    \dfrac{726}{420} &=
    \dfrac{\cancel{2} \times \cancel{3} \times 11 \times 11}
    {\cancel{2} \times 2 \times \cancel{3} \times 5 \times 7}\\ &=
    \dfrac{11 \times 11}
    {2 \times 5 \times 7} =
    \dfrac{121}{70}
    \end{align*}

\item\begin{align*}
    \dfrac{210}{27} &=
    \dfrac{2 \times \cancel{3} \times 5 \times 7}
    {\cancel{3} \times 3 \times 3}\\ &=
    \dfrac{2 \times 5 \times 7}
    {3 \times 3} =
    \dfrac{70}{9}
    \end{align*}

\item\begin{align*}
    \dfrac{308}{8316} &=
    \dfrac{\cancel{2} \times \cancel{2} \times \cancel{7} \times \cancel{11}}
    {\cancel{2} \times \cancel{2} \times 3 \times 3 \times 3 \times \cancel{7} \times \cancel{11}}\\ &=
    \dfrac{1}
    {3 \times 3 \times 3} =
    \dfrac{1}{27}
    \end{align*}

\item\begin{align*}
    \dfrac{10}{484} &=
    \dfrac{\cancel{2} \times 5}
    {\cancel{2} \times 2 \times 11 \times 11}\\ &=
    \dfrac{5}
    {2 \times 11 \times 11} =
    \dfrac{5}{242}
    \end{align*}

\item\begin{align*}
    \dfrac{78}{372645} &=
    \dfrac{2 \times \cancel{3} \times \cancel{13}}
    {\cancel{3} \times 3 \times 5 \times 7 \times 7 \times \cancel{13} \times 13}\\ &=
    \dfrac{2}
    {3 \times 5 \times 7 \times 7 \times 13} =
    \dfrac{2}{9555}
    \end{align*}

\item\begin{align*}
    \dfrac{5148}{5460} &=
    \dfrac{\cancel{2} \times \cancel{2} \times \cancel{3} \times 3 \times 11 \times \cancel{13}}
    {\cancel{2} \times \cancel{2} \times \cancel{3} \times 5 \times 7 \times \cancel{13}}\\ &=
    \dfrac{3 \times 11}
    {5 \times 7} =
    \dfrac{33}{35}
    \end{align*}

\item\begin{align*}
    \dfrac{1274}{7872865} &=
    \dfrac{2 \times \cancel{7} \times 7 \times \cancel{13}}
    {5 \times \cancel{7} \times 11 \times 11 \times 11 \times \cancel{13} \times 13}\\ &=
    \dfrac{2 \times 7}
    {5 \times 11 \times 11 \times 11 \times 13} =
    \dfrac{14}{86515}
    \end{align*}

\item\begin{align*}
    \dfrac{195}{2640} &=
    \dfrac{\cancel{3} \times \cancel{5} \times 13}
    {2 \times 2 \times 2 \times 2 \times \cancel{3} \times \cancel{5} \times 11}\\ &=
    \dfrac{13}
    {2 \times 2 \times 2 \times 2 \times 11} =
    \dfrac{13}{176}
    \end{align*}

\item\begin{align*}
    \dfrac{65}{520} &=
    \dfrac{\cancel{5} \times \cancel{13}}
    {2 \times 2 \times 2 \times \cancel{5} \times \cancel{13}}\\ &=
    \dfrac{1}
    {2 \times 2 \times 2} =
    \dfrac{1}{8}
    \end{align*}

\item\begin{align*}
    \dfrac{168}{22} &=
    \dfrac{\cancel{2} \times 2 \times 2 \times 3 \times 7}
    {\cancel{2} \times 11}\\ &=
    \dfrac{2 \times 2 \times 3 \times 7}
    {11} =
    \dfrac{84}{11}
    \end{align*}

\item\begin{align*}
    \dfrac{84084}{1144} &=
    \dfrac{\cancel{2} \times \cancel{2} \times 3 \times 7 \times 7 \times \cancel{11} \times \cancel{13}}
    {\cancel{2} \times \cancel{2} \times 2 \times \cancel{11} \times \cancel{13}}\\ &=
    \dfrac{3 \times 7 \times 7}
    {2} =
    \dfrac{147}{2}
    \end{align*}

\item\begin{align*}
    \dfrac{280}{13552} &=
    \dfrac{\cancel{2} \times \cancel{2} \times \cancel{2} \times 5 \times \cancel{7}}
    {\cancel{2} \times \cancel{2} \times \cancel{2} \times 2 \times \cancel{7} \times 11 \times 11}\\ &=
    \dfrac{5}
    {2 \times 11 \times 11} =
    \dfrac{5}{242}
    \end{align*}

\item\begin{align*}
    \dfrac{11550}{1386} &=
    \dfrac{\cancel{2} \times \cancel{3} \times 5 \times 5 \times \cancel{7} \times \cancel{11}}
    {\cancel{2} \times \cancel{3} \times 3 \times \cancel{7} \times \cancel{11}}\\ &=
    \dfrac{5 \times 5}
    {3} =
    \dfrac{25}{3}
    \end{align*}

\item\begin{align*}
    \dfrac{15}{153790} &=
    \dfrac{3 \times \cancel{5}}
    {2 \times \cancel{5} \times 7 \times 13 \times 13 \times 13}\\ &=
    \dfrac{3}
    {2 \times 7 \times 13 \times 13 \times 13} =
    \dfrac{3}{30758}
    \end{align*}

\item\begin{align*}
    \dfrac{1470}{1617} &=
    \dfrac{2 \times \cancel{3} \times 5 \times \cancel{7} \times \cancel{7}}
    {\cancel{3} \times \cancel{7} \times \cancel{7} \times 11}\\ &=
    \dfrac{2 \times 5}
    {11} =
    \dfrac{10}{11}
    \end{align*}

\item\begin{align*}
    \dfrac{546}{504} &=
    \dfrac{\cancel{2} \times \cancel{3} \times \cancel{7} \times 13}
    {\cancel{2} \times 2 \times 2 \times \cancel{3} \times 3 \times \cancel{7}}\\ &=
    \dfrac{13}
    {2 \times 2 \times 3} =
    \dfrac{13}{12}
    \end{align*}

\item\begin{align*}
    \dfrac{429}{2600} &=
    \dfrac{3 \times 11 \times \cancel{13}}
    {2 \times 2 \times 2 \times 5 \times 5 \times \cancel{13}}\\ &=
    \dfrac{3 \times 11}
    {2 \times 2 \times 2 \times 5 \times 5} =
    \dfrac{33}{200}
    \end{align*}

\item\begin{align*}
    \dfrac{8}{42588} &=
    \dfrac{\cancel{2} \times \cancel{2} \times 2}
    {\cancel{2} \times \cancel{2} \times 3 \times 3 \times 7 \times 13 \times 13}\\ &=
    \dfrac{2}
    {3 \times 3 \times 7 \times 13 \times 13} =
    \dfrac{2}{10647}
    \end{align*}

\item\begin{align*}
    \dfrac{2366}{3300} &=
    \dfrac{\cancel{2} \times 7 \times 13 \times 13}
    {\cancel{2} \times 2 \times 3 \times 5 \times 5 \times 11}\\ &=
    \dfrac{7 \times 13 \times 13}
    {2 \times 3 \times 5 \times 5 \times 11} =
    \dfrac{1183}{1650}
    \end{align*}

\item\begin{align*}
    \dfrac{20}{5775} &=
    \dfrac{2 \times 2 \times \cancel{5}}
    {3 \times \cancel{5} \times 5 \times 7 \times 11}\\ &=
    \dfrac{2 \times 2}
    {3 \times 5 \times 7 \times 11} =
    \dfrac{4}{1155}
    \end{align*}

\item\begin{align*}
    \dfrac{819}{390} &=
    \dfrac{\cancel{3} \times 3 \times 7 \times \cancel{13}}
    {2 \times \cancel{3} \times 5 \times \cancel{13}}\\ &=
    \dfrac{3 \times 7}
    {2 \times 5} =
    \dfrac{21}{10}
    \end{align*}

\item\begin{align*}
    \dfrac{175}{6} &=
    \dfrac{5 \times 5 \times 7}
    {2 \times 3}\\ &=
    \dfrac{5 \times 5 \times 7}
    {2 \times 3} =
    \dfrac{175}{6}
    \end{align*}

\item\begin{align*}
    \dfrac{650}{10010} &=
    \dfrac{\cancel{2} \times \cancel{5} \times 5 \times \cancel{13}}
    {\cancel{2} \times \cancel{5} \times 7 \times 11 \times \cancel{13}}\\ &=
    \dfrac{5}
    {7 \times 11} =
    \dfrac{5}{77}
    \end{align*}

\item\begin{align*}
    \dfrac{3120}{168} &=
    \dfrac{\cancel{2} \times \cancel{2} \times \cancel{2} \times 2 \times \cancel{3} \times 5 \times 13}
    {\cancel{2} \times \cancel{2} \times \cancel{2} \times \cancel{3} \times 7}\\ &=
    \dfrac{2 \times 5 \times 13}
    {7} =
    \dfrac{130}{7}
    \end{align*}

\item\begin{align*}
    \dfrac{13860}{1680} &=
    \dfrac{\cancel{2} \times \cancel{2} \times \cancel{3} \times 3 \times \cancel{5} \times \cancel{7} \times 11}
    {\cancel{2} \times \cancel{2} \times 2 \times 2 \times \cancel{3} \times \cancel{5} \times \cancel{7}}\\ &=
    \dfrac{3 \times 11}
    {2 \times 2} =
    \dfrac{33}{4}
    \end{align*}

\item\begin{align*}
    \dfrac{50820}{10} &=
    \dfrac{\cancel{2} \times 2 \times 3 \times \cancel{5} \times 7 \times 11 \times 11}
    {\cancel{2} \times \cancel{5}}\\ &=
    \dfrac{2 \times 3 \times 7 \times 11 \times 11}
    {1} =
    \dfrac{5082}{1}
    \end{align*}

\item\begin{align*}
    \dfrac{98}{11011} &=
    \dfrac{2 \times \cancel{7} \times 7}
    {\cancel{7} \times 11 \times 11 \times 13}\\ &=
    \dfrac{2 \times 7}
    {11 \times 11 \times 13} =
    \dfrac{14}{1573}
    \end{align*}

\item\begin{align*}
    \dfrac{308}{1014} &=
    \dfrac{\cancel{2} \times 2 \times 7 \times 11}
    {\cancel{2} \times 3 \times 13 \times 13}\\ &=
    \dfrac{2 \times 7 \times 11}
    {3 \times 13 \times 13} =
    \dfrac{154}{507}
    \end{align*}

\item\begin{align*}
    \dfrac{51975}{48} &=
    \dfrac{\cancel{3} \times 3 \times 3 \times 5 \times 5 \times 7 \times 11}
    {2 \times 2 \times 2 \times 2 \times \cancel{3}}\\ &=
    \dfrac{3 \times 3 \times 5 \times 5 \times 7 \times 11}
    {2 \times 2 \times 2 \times 2} =
    \dfrac{17325}{16}
    \end{align*}

\item\begin{align*}
    \dfrac{35}{676} &=
    \dfrac{5 \times 7}
    {2 \times 2 \times 13 \times 13}\\ &=
    \dfrac{5 \times 7}
    {2 \times 2 \times 13 \times 13} =
    \dfrac{35}{676}
    \end{align*}

\item\begin{align*}
    \dfrac{4}{2860} &=
    \dfrac{\cancel{2} \times \cancel{2}}
    {\cancel{2} \times \cancel{2} \times 5 \times 11 \times 13}\\ &=
    \dfrac{1}
    {5 \times 11 \times 13} =
    \dfrac{1}{715}
    \end{align*}

\item\begin{align*}
    \dfrac{84}{280} &=
    \dfrac{\cancel{2} \times \cancel{2} \times 3 \times \cancel{7}}
    {\cancel{2} \times \cancel{2} \times 2 \times 5 \times \cancel{7}}\\ &=
    \dfrac{3}
    {2 \times 5} =
    \dfrac{3}{10}
    \end{align*}

\item\begin{align*}
    \dfrac{3432}{484} &=
    \dfrac{\cancel{2} \times \cancel{2} \times 2 \times 3 \times \cancel{11} \times 13}
    {\cancel{2} \times \cancel{2} \times \cancel{11} \times 11}\\ &=
    \dfrac{2 \times 3 \times 13}
    {11} =
    \dfrac{78}{11}
    \end{align*}

\item\begin{align*}
    \dfrac{5005}{5915} &=
    \dfrac{\cancel{5} \times \cancel{7} \times 11 \times \cancel{13}}
    {\cancel{5} \times \cancel{7} \times \cancel{13} \times 13}\\ &=
    \dfrac{11}
    {13} =
    \dfrac{11}{13}
    \end{align*}

\item\begin{align*}
    \dfrac{1014}{14} &=
    \dfrac{\cancel{2} \times 3 \times 13 \times 13}
    {\cancel{2} \times 7}\\ &=
    \dfrac{3 \times 13 \times 13}
    {7} =
    \dfrac{507}{7}
    \end{align*}

\item\begin{align*}
    \dfrac{12}{630} &=
    \dfrac{\cancel{2} \times 2 \times \cancel{3}}
    {\cancel{2} \times \cancel{3} \times 3 \times 5 \times 7}\\ &=
    \dfrac{2}
    {3 \times 5 \times 7} =
    \dfrac{2}{105}
    \end{align*}

\item\begin{align*}
    \dfrac{1848}{364} &=
    \dfrac{\cancel{2} \times \cancel{2} \times 2 \times 3 \times \cancel{7} \times 11}
    {\cancel{2} \times \cancel{2} \times \cancel{7} \times 13}\\ &=
    \dfrac{2 \times 3 \times 11}
    {13} =
    \dfrac{66}{13}
    \end{align*}

\item\begin{align*}
    \dfrac{4}{1680} &=
    \dfrac{\cancel{2} \times \cancel{2}}
    {\cancel{2} \times \cancel{2} \times 2 \times 2 \times 3 \times 5 \times 7}\\ &=
    \dfrac{1}
    {2 \times 2 \times 3 \times 5 \times 7} =
    \dfrac{1}{420}
    \end{align*}

\item\begin{align*}
    \dfrac{15288}{296450} &=
    \dfrac{\cancel{2} \times 2 \times 2 \times 3 \times \cancel{7} \times \cancel{7} \times 13}
    {\cancel{2} \times 5 \times 5 \times \cancel{7} \times \cancel{7} \times 11 \times 11}\\ &=
    \dfrac{2 \times 2 \times 3 \times 13}
    {5 \times 5 \times 11 \times 11} =
    \dfrac{156}{3025}
    \end{align*}

\item\begin{align*}
    \dfrac{14520}{286} &=
    \dfrac{\cancel{2} \times 2 \times 2 \times 3 \times 5 \times \cancel{11} \times 11}
    {\cancel{2} \times \cancel{11} \times 13}\\ &=
    \dfrac{2 \times 2 \times 3 \times 5 \times 11}
    {13} =
    \dfrac{660}{13}
    \end{align*}

\item\begin{align*}
    \dfrac{195}{22} &=
    \dfrac{3 \times 5 \times 13}
    {2 \times 11}\\ &=
    \dfrac{3 \times 5 \times 13}
    {2 \times 11} =
    \dfrac{195}{22}
    \end{align*}

\item\begin{align*}
    \dfrac{286}{6} &=
    \dfrac{\cancel{2} \times 11 \times 13}
    {\cancel{2} \times 3}\\ &=
    \dfrac{11 \times 13}
    {3} =
    \dfrac{143}{3}
    \end{align*}

\item\begin{align*}
    \dfrac{352}{330} &=
    \dfrac{\cancel{2} \times 2 \times 2 \times 2 \times 2 \times \cancel{11}}
    {\cancel{2} \times 3 \times 5 \times \cancel{11}}\\ &=
    \dfrac{2 \times 2 \times 2 \times 2}
    {3 \times 5} =
    \dfrac{16}{15}
    \end{align*}

\item\begin{align*}
    \dfrac{2310}{6468} &=
    \dfrac{\cancel{2} \times \cancel{3} \times 5 \times \cancel{7} \times \cancel{11}}
    {\cancel{2} \times 2 \times \cancel{3} \times \cancel{7} \times 7 \times \cancel{11}}\\ &=
    \dfrac{5}
    {2 \times 7} =
    \dfrac{5}{14}
    \end{align*}

\item\begin{align*}
    \dfrac{650650}{80} &=
    \dfrac{\cancel{2} \times \cancel{5} \times 5 \times 7 \times 11 \times 13 \times 13}
    {\cancel{2} \times 2 \times 2 \times 2 \times \cancel{5}}\\ &=
    \dfrac{5 \times 7 \times 11 \times 13 \times 13}
    {2 \times 2 \times 2} =
    \dfrac{65065}{8}
    \end{align*}

\item\begin{align*}
    \dfrac{2184}{27027} &=
    \dfrac{2 \times 2 \times 2 \times \cancel{3} \times \cancel{7} \times \cancel{13}}
    {\cancel{3} \times 3 \times 3 \times \cancel{7} \times 11 \times \cancel{13}}\\ &=
    \dfrac{2 \times 2 \times 2}
    {3 \times 3 \times 11} =
    \dfrac{8}{99}
    \end{align*}

\item\begin{align*}
    \dfrac{252}{10164} &=
    \dfrac{\cancel{2} \times \cancel{2} \times \cancel{3} \times 3 \times \cancel{7}}
    {\cancel{2} \times \cancel{2} \times \cancel{3} \times \cancel{7} \times 11 \times 11}\\ &=
    \dfrac{3}
    {11 \times 11} =
    \dfrac{3}{121}
    \end{align*}

\item\begin{align*}
    \dfrac{14014}{150} &=
    \dfrac{\cancel{2} \times 7 \times 7 \times 11 \times 13}
    {\cancel{2} \times 3 \times 5 \times 5}\\ &=
    \dfrac{7 \times 7 \times 11 \times 13}
    {3 \times 5 \times 5} =
    \dfrac{7007}{75}
    \end{align*}

\item\begin{align*}
    \dfrac{10}{14} &=
    \dfrac{\cancel{2} \times 5}
    {\cancel{2} \times 7}\\ &=
    \dfrac{5}
    {7} =
    \dfrac{5}{7}
    \end{align*}

\item\begin{align*}
    \dfrac{4620}{15} &=
    \dfrac{2 \times 2 \times \cancel{3} \times \cancel{5} \times 7 \times 11}
    {\cancel{3} \times \cancel{5}}\\ &=
    \dfrac{2 \times 2 \times 7 \times 11}
    {1} =
    \dfrac{308}{1}
    \end{align*}

\item\begin{align*}
    \dfrac{1287}{16} &=
    \dfrac{3 \times 3 \times 11 \times 13}
    {2 \times 2 \times 2 \times 2}\\ &=
    \dfrac{3 \times 3 \times 11 \times 13}
    {2 \times 2 \times 2 \times 2} =
    \dfrac{1287}{16}
    \end{align*}

\item\begin{align*}
    \dfrac{85995}{936} &=
    \dfrac{\cancel{3} \times \cancel{3} \times 3 \times 5 \times 7 \times 7 \times \cancel{13}}
    {2 \times 2 \times 2 \times \cancel{3} \times \cancel{3} \times \cancel{13}}\\ &=
    \dfrac{3 \times 5 \times 7 \times 7}
    {2 \times 2 \times 2} =
    \dfrac{735}{8}
    \end{align*}

\item\begin{align*}
    \dfrac{14742}{385} &=
    \dfrac{2 \times 3 \times 3 \times 3 \times 3 \times \cancel{7} \times 13}
    {5 \times \cancel{7} \times 11}\\ &=
    \dfrac{2 \times 3 \times 3 \times 3 \times 3 \times 13}
    {5 \times 11} =
    \dfrac{2106}{55}
    \end{align*}

\item\begin{align*}
    \dfrac{3003}{1274} &=
    \dfrac{3 \times \cancel{7} \times 11 \times \cancel{13}}
    {2 \times \cancel{7} \times 7 \times \cancel{13}}\\ &=
    \dfrac{3 \times 11}
    {2 \times 7} =
    \dfrac{33}{14}
    \end{align*}

\item\begin{align*}
    \dfrac{37180}{5460} &=
    \dfrac{\cancel{2} \times \cancel{2} \times \cancel{5} \times 11 \times \cancel{13} \times 13}
    {\cancel{2} \times \cancel{2} \times 3 \times \cancel{5} \times 7 \times \cancel{13}}\\ &=
    \dfrac{11 \times 13}
    {3 \times 7} =
    \dfrac{143}{21}
    \end{align*}

\item\begin{align*}
    \dfrac{975975}{4} &=
    \dfrac{3 \times 5 \times 5 \times 7 \times 11 \times 13 \times 13}
    {2 \times 2}\\ &=
    \dfrac{3 \times 5 \times 5 \times 7 \times 11 \times 13 \times 13}
    {2 \times 2} =
    \dfrac{975975}{4}
    \end{align*}

\item\begin{align*}
    \dfrac{70}{131820} &=
    \dfrac{\cancel{2} \times \cancel{5} \times 7}
    {\cancel{2} \times 2 \times 3 \times \cancel{5} \times 13 \times 13 \times 13}\\ &=
    \dfrac{7}
    {2 \times 3 \times 13 \times 13 \times 13} =
    \dfrac{7}{13182}
    \end{align*}

\item\begin{align*}
    \dfrac{9800}{140} &=
    \dfrac{\cancel{2} \times \cancel{2} \times 2 \times \cancel{5} \times 5 \times \cancel{7} \times 7}
    {\cancel{2} \times \cancel{2} \times \cancel{5} \times \cancel{7}}\\ &=
    \dfrac{2 \times 5 \times 7}
    {1} =
    \dfrac{70}{1}
    \end{align*}

\item\begin{align*}
    \dfrac{10}{6468} &=
    \dfrac{\cancel{2} \times 5}
    {\cancel{2} \times 2 \times 3 \times 7 \times 7 \times 11}\\ &=
    \dfrac{5}
    {2 \times 3 \times 7 \times 7 \times 11} =
    \dfrac{5}{3234}
    \end{align*}

\item\begin{align*}
    \dfrac{15400}{700} &=
    \dfrac{\cancel{2} \times \cancel{2} \times 2 \times \cancel{5} \times \cancel{5} \times \cancel{7} \times 11}
    {\cancel{2} \times \cancel{2} \times \cancel{5} \times \cancel{5} \times \cancel{7}}\\ &=
    \dfrac{2 \times 11}
    {1} =
    \dfrac{22}{1}
    \end{align*}

\item\begin{align*}
    \dfrac{210}{30} &=
    \dfrac{\cancel{2} \times \cancel{3} \times \cancel{5} \times 7}
    {\cancel{2} \times \cancel{3} \times \cancel{5}}\\ &=
    \dfrac{7}
    {1} =
    \dfrac{7}{1}
    \end{align*}

\item\begin{align*}
    \dfrac{33}{780} &=
    \dfrac{\cancel{3} \times 11}
    {2 \times 2 \times \cancel{3} \times 5 \times 13}\\ &=
    \dfrac{11}
    {2 \times 2 \times 5 \times 13} =
    \dfrac{11}{260}
    \end{align*}

\item\begin{align*}
    \dfrac{32340}{217503} &=
    \dfrac{2 \times 2 \times \cancel{3} \times 5 \times 7 \times 7 \times \cancel{11}}
    {\cancel{3} \times 3 \times \cancel{11} \times 13 \times 13 \times 13}\\ &=
    \dfrac{2 \times 2 \times 5 \times 7 \times 7}
    {3 \times 13 \times 13 \times 13} =
    \dfrac{980}{6591}
    \end{align*}

\item\begin{align*}
    \dfrac{792}{165} &=
    \dfrac{2 \times 2 \times 2 \times \cancel{3} \times 3 \times \cancel{11}}
    {\cancel{3} \times 5 \times \cancel{11}}\\ &=
    \dfrac{2 \times 2 \times 2 \times 3}
    {5} =
    \dfrac{24}{5}
    \end{align*}

\item\begin{align*}
    \dfrac{126}{264} &=
    \dfrac{\cancel{2} \times \cancel{3} \times 3 \times 7}
    {\cancel{2} \times 2 \times 2 \times \cancel{3} \times 11}\\ &=
    \dfrac{3 \times 7}
    {2 \times 2 \times 11} =
    \dfrac{21}{44}
    \end{align*}

\item\begin{align*}
    \dfrac{936}{1320} &=
    \dfrac{\cancel{2} \times \cancel{2} \times \cancel{2} \times \cancel{3} \times 3 \times 13}
    {\cancel{2} \times \cancel{2} \times \cancel{2} \times \cancel{3} \times 5 \times 11}\\ &=
    \dfrac{3 \times 13}
    {5 \times 11} =
    \dfrac{39}{55}
    \end{align*}

}}
% % VARIABLES %%%
\setTitle{Fiche de Prep - Séance 3: Lire et construire un diagrammes en bâton}
\date{27/09/2024}
\setgrade{6e}
\def\imgPath{enseignement/6e/organisation-et-gestion-de-donnees/}
%%

\def\imgPrefix{dim-6e/}
\prepTable{
    \prepRow{
        \slide{SEANCES PRECEDENTES}{}
    }{
        \begin{itemize}
            \item Lire un tableau
            \item Construire un tableau
        \end{itemize}
    }{-}
    \prepRow{
        \slide{qf}{}
        \imgp{qf-3-4p96}
    }{
        \begin{itemize}[wide=0pt, leftmargin=*]
            \item 3p96: 
            \begin{itemize}[wide=0pt, leftmargin=*]
                \item Tâche: Lire un diagramme en bâton.
                \item Modalité de correction : correction d'un élève au tableau qui entourera les données recherché dans le document.
            \end{itemize}
        \end{itemize}
    }{
        $5\min$
    }
    \prepRow{
        \slide{CORRECTION}{}
        \imgp{exo-6p101.png}
    }{
        \begin{itemize}[wide=0pt, leftmargin=*]
            \item 6p101: 
            \begin{itemize}[wide=0pt, leftmargin=*]
                \item Tâche: Constrution de tableau a partir de série statique.
                \item Difficultés attendues : 
                identification des données à organiser,
                classement des données,
                différence entre modalité et effectif,
                utilisation de la ligne ou de la colonne
                \item Modalité de correction : création d'un tableau sur libreOffice Calc.
            \end{itemize}
        \end{itemize}
    }{
        $7\min$
    }
    \prepRow{
        \slide{ACTIVITE}{}
        \imgp{exo-6p101.png}
    }{
        \begin{itemize}[wide=0pt, leftmargin=*]
            \item A partir des données de l'exercice corrigé contruire un diagramme en bâton: 
            \begin{itemize}[wide=0pt, leftmargin=*]
                \item Tâche: Constrution d'un diagramme en bâton a partir d'un tableau.
                \item Difficultés attendues :
                gestion des axes,
                présentation graphique,
                respect de la proportionnalité,
                compréhension de la consigne,
                choix des unités et des échelles
                \item Modalité de correction :
                Sur un fond de graphique,
                un élève dessine un diagramme en bâton.
                Un éleve dessine au tableau le diagramme en bâton.
            \end{itemize}
        \end{itemize}
    }{
        $15\min$
    }
    \prepRow{
        \slide{cr}{}
        \setcounter{section}{1}
        \section{Diagrammes}
        \subsection{Diagrammes en bâton}
    }{
        \begin{itemize}
            \item \vc{Diagrammes en bâton}{à recopier}
            \item \expl{Differentes représentation}{à coller (correction de l'exercice/activité)}
        \end{itemize}
    }{
        $5\min$
    }

    \def\imgPrefix{}
    \prepRow{
        \slide{exo}
        \def\imgPrefix{}
        \imgp{diagramme-en-baton-attendus-6e}
    }{
        \begin{itemize}
            \item Vrai ou Faux?
            \begin{itemize}
                \item Tâche : Porter un regard critique sur une représentation graphique.
                \item Difficultés attendues:
                confusion entre les nombres réels et la représentation visuelle,
                compréhension de l'échelle
            \end{itemize}
        \end{itemize}
    }{10min}

    \prepRow{
        \slide{COURS (si le temps)}{}
    }{
        \begin{itemize}
            \item \pr{bâtons proportionnels}{à copier}
        \end{itemize}
    }{6min}

    \prepRow{
        \slide{EXERCICES A LA MAISON}{}
        \begin{itemize}
            \item 27p105
        \end{itemize}
    }{\imgp{exo-27p105}[5cm]}{3min}
    \prepRow{
        \slide{SEANCES SUIVANTES}{}
    }{
        \begin{itemize}
            \item Diagrammes circulaires
            \item Situations de proportionnalité
            \item Graphiques cartésiens
        \end{itemize}
    }{-}
}
% % VARIABLES %%%
\setTitle{\jules}
\def\imgPath{enseignement/4e/divisibilite-et-nombres-premiers/}
%%

\definecolor{gradeColor}{HTML}{E46C4B} %#E46C4B
\def\currentColor{gradeColor}

\newcommand{\adress}[1]{{\small #1}}
\newcommand{\yr}[1]{{\color{Gray} #1}}
\def\SU{\adress{Sorbonne Université, 4 Place Jussieu, 75005 Paris }}

\slide{Occupation}{
    \begin{itemize}
        \item \key{Professeur stagiaire à temps complet} -
        \adress{Collège de la Paix, 76 Rue du Fort, 92130 Issy-les-Moulineaux} -
        \yr{2024-2025}
    \end{itemize}
}

\slide{Formation}{
    \begin{itemize}
        \item \key{Master 2 MEEF mathématiques} -
        \SU -
        \yr{2023-2024}
        \item \key{Master 1 MEEF mathématiques} -
        \SU -
        \yr{2022-2023}
        \item \key{Licence 3 Mono-Math} -
        \SU -
        \yr{2020-2022}
        \item \key{Prépa Math Physique} -
        \adress{lycée Fénelon, 2 rue de l'Eperon, 75006 Paris} -
        \yr{2018-2020}
    \end{itemize}
}

\slide{Experiences professionnelles}{
    \begin{itemize}
        \item \key{Stage SOPA} -
        \adress{Collège Janson de Sailly, 106 Rue de la Pompe, 75016 Paris} -
        \yr{2023-2024 , 324 heures}
        \item \key{Stage SOPA} -
        \adress{Collège Jean-Moulin, 75 rue d'Alésia, 75014 Paris} -
        \yr{2022-2023 , 6 semaines}
        \item \key{Stage SOPA} -
        \adress{Collège Rodin, 19 Rue Corvisart, 75013 Paris} -
        \yr{2023 - 5 demi-journées}
        \item \key{Vente / Animation} -
        Emploi étudiant -
        \adress{Le Paysans Urbain, 14 Rue Stendhal, 75020 Paris} -
        \yr{2021-2022 , 6 mois à mi-temps}
        \item \key{Plaidoyer - écologie et alimentation} -
        Association étudiante -
        \adress{LUPA, Sorbonne Université} -
        \yr{2020-2023}
        \item \key{Projectionniste} -
        Association étudiante -
        \adress{SUper8, Sorbonne Université} -
        \yr{2020-2022}
        \item \key{Professeur particulier} - mathématiques -
        \yr{2021-2024 , 2h/semaines}
    \end{itemize}
}

\slide{Coordonnées}{
    \begin{itemize}
        \item 38 rue Fessart, 92100 Boulogne-Billancourt
        \item jules.pesin@ac-versailles.fr
        \item 0695010324
    \end{itemize}
}

\end{document}