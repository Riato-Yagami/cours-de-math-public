%% BEAMER %%
% \documentclass[aspectratio=169, usenames,dvipsnames,xcolor=table]{beamer} \usepackage{amsmath}
\usepackage{amsthm}
\usepackage{amssymb}
\usepackage{graphicx}
\usepackage{dashundergaps}
\usepackage{array}
\usepackage{multicol}
\usepackage{wrapfig}
\usepackage{numprint}
\usepackage{ulem}
\usepackage{hyperref}
\usepackage{mathrsfs}
\usepackage{mathtools}
\usepackage[many]{tcolorbox}
\usepackage{xparse}
\usepackage{float}
\usepackage{lipsum}
\usepackage{pgf}
\usepackage{ifthen}
\usepackage{caption}
\usepackage{tikz}
\usepackage{tkz-tab}
\usepackage{xifthen}
\usepackage{listings}
\usepackage[linesnumbered,vlined,boxed]{algorithm2e}
\usepackage[squaren,Gray]{SIunits}
\usepackage{animate}
\usepackage{eurosym}
\usepackage{tkz-euclide}
\usepackage{etoolbox}
\usepackage{textcase}
\usepackage{adjustbox}
% \usepackage{mathabx}

\usepackage{tabularx}
\usepackage{scratch3}

\usepackage{multicol}
\usepackage{multirow}
% \usepackage{appendix}
% \usepackage[
%     backend=biber,        % compilateur par défaut pour biblatex
%     sorting=nyt,          % trier par nom, année, titre
%     citestyle=authoryear, % style de citation auteur-année
%     bibstyle=alphabetic,  % style de bibliographie alphabétique
% ]{biblatex}

% \usepackage[T1]{fontenc}

% \usepackage{pifont}

% \usepackage[squaren,Gray]{SIunits}

% BREVET

% \usepackage{makeidx}
% \usepackage{fancybox}
% \usepackage{tabularx}
% \usepackage[normalem]{ulem}
% \usepackage{pifont}
% \usepackage{lscape}
% \usepackage{diagbox}
% \usepackage{multido}
% \usepackage[dvipsnames]{pstricks}
% \usepackage{pst-plot,pst-text,pst-tree,pstricks-add}
% \usepackage{textcomp}
% \usepackage{scratch3}
% \usepackage[T1]{fontenc}
% \usepackage{fourier}
% \usepackage[french]{babel}
% \usepackage{pstricks}

% \usepackage[scaled=0.875]{helvet}
% \usepackage{pst-plot,pst-text,pst-tree,pstricks-add}

% hyperref

\hypersetup{
    colorlinks=true,       % false: boxed links; true: colored links
    linkcolor=red,          % color of internal links (change box color with linkbordercolor)
    citecolor=green,        % color of links to bibliography
    filecolor=magenta,      % color of file links
    urlcolor=blue,          % color of external links
    urlbordercolor=blue,    % borders of external links
    linkbordercolor=red,    % borders of internal links
    pdfborderstyle={/S/U/W 1}% border style will be underline of width 1pt
}

\frenchbsetup{StandardItemLabels=true}

% listings

\lstdefinestyle{pythonStyle}{
    language=Python,
    basicstyle=\ttfamily\small,  % Adjust the font and size
    commentstyle=\itshape\color{green!40!black},
    keywordstyle=\bfseries\color{violet},
    numbers=left,
    numberstyle=\tiny\color{gray},
    frame=single,
    breaklines=true,
    breakatwhitespace=true,
    tabsize=4,
    captionpos=b,
    identifierstyle=\color{Blue},
}

\lstset{style=pythonStyle}

% algorithm2e

\SetNlSty{}{}{.}
\SetKwInput{KwRes}{R\'esultat}%
\SetKwIF{Si}{SinonSi}{Sinon}{si}{alors}{sinon si}{sinon}{fin si}%
\SetKwFor{Tq}{tant que}{faire}{fin tq}%

% GLOBAL VARIABLES %%%
\graphicspath{{images}}
\def\cwidth{4cm}
\def\tspace{0.5cm}

% BOOLEAN %%%
\newboolean{anwser}
\newboolean{demonstration}
\newboolean{boxedProperties}
\newboolean{showID}
\newboolean{parenthisedID}
\newboolean{animated}
\newboolean{outline}

\setboolean{anwser}{false}
\setboolean{demonstration}{true}
\setboolean{parenthisedID}{true}
\setboolean{showID}{true}
\setboolean{boxedProperties}{false} % false = edge
\setboolean{outline}{false}

\def\DefinitionColor{PineGreen}
\def\PropertyColor{Blue}
\def\TheoremColor{Plum}

\def\SectionColor{Red}
\def\SubSectionColor{Green}

\setboolean{animated}{true}

% Switch implementation
\newboolean{default}
\newcommand{\case}{}
\newcommand{\default}{}

\newenvironment{switch}[1]{%
    \setboolean{default}{true}
    \renewcommand{\case}[2]{\ifthenelse{\equal{#1}{##1}}{%
        \setboolean{default}{false}##2}{}}%
    \renewcommand{\default}[1]{\ifthenelse{\boolean{default}}{##1}{}}
}{}

% SECTIONS
\input{header/command/sections.tex}

% ANSWERS
\newlength{\parline}
\newlength{\paroutindent}
\newlength{\lineheight}
\setlength{\lineheight}{\heightof{abcdefghijklmnoprstuvwxyz}}

\newcommand{\countlines}[1]{%
    \setlength{\paroutindent}{\expandafter\parindent}
    \setlength{\parline}{\heightof{\noindent\begin{minipage}{\linewidth}%
                \setlength{\parindent}{\paroutindent}#1\end{minipage}}}%
    \pgfmathparse{round(\parline / (0.9*\lineheight))}
    \newcount\linecount
    \pgfmathsetcount{\linecount}{\pgfmathresult}
}

\newcommand{\looptext}[2]{%
    \noindent
    \newcount\printcount
    \printcount=#2
    \loop
        #1
        \advance\printcount by -1
        \ifnum\printcount>0
    \repeat
}

\newcommand{\awsr}[1]{%
    \ifthenelse{\boolean{answer}}{
        \result{#1}
    }{
        \countlines{#1}
        \pgfmathsetcount{\linecount}{\linecount + 1}
        \noindent\hspace{-9pt}
        \looptext{
            \noindent\dotfill
    
        }{\the\linecount}
    }
}

\newcommand{\dottedLines}[1]{%
    \noindent\hspace{-9pt}%

    \looptext{%
        \noindent\dotfill%

    }{#1}
}

\newcommand{\result}[1]{\color{OrangeRed}#1 \color{black}%
}

% MATH
\input{header/command/math.tex}

% IMAGES
\input{header/command/image.tex}

% COMMANDS

\newcommand{\fsize}[1]{\fontsize{#1}{#1}\selectfont}

\NewDocumentCommand{\ifNotNull}{mmo}{
    \IfValueT{#1}{
        \ifx\relax#1\relax
            \IfValueT{#3}{
                #3
            }
        \else
            #2
        \fi
    }
}

\NewDocumentCommand{\ilink}{m g}{%
    \item
    \IfValueTF{#2}{\link{#1}{#2}}{\link{#1}}
}

\NewDocumentCommand{\link}{m g}{%
    \csn{#1}%
    \IfValueT{#2}{(#2)}%
}

\NewDocumentCommand{\TODO}{g}{%
    {\color{Red} $\rightarrow$ \textbf{TODO}
    \IfValueT{#1}{(#1)}}
    % \color{black}
}

\newcommand{\leconInfoBox}[2]{
    \textbf{#1 :}\vspace{-0.25cm}
        \begin{multicols}{2}
            \begin{itemize}[label=$\blacktriangleright$, font = \small \color{Red}]
                #2
            \end{itemize}
        \end{multicols}
        \vspace{-0.4cm}
}

% TCOLORBOX

\input{header/command/tcolorbox.tex}

\NewDocumentCommand{\leconInfo}{mooo}{
    \begin{infoBox}
        \leconInfoBox{Niveaux}{#1}
        \ifNotNull{#2}{
            \tcbline
            \leconInfoBox{Prérequis}{#2}
        }
        \ifNotNull{#3}{
            \tcbline
            \leconInfoBox{Thèmes}{#3}
        }
        \ifNotNull{#4}{
            \tcbline
            \textbf{Motivation :} 
            #4
        }
    \end{infoBox}
}

\NewDocumentCommand{\seanceInfo}{oooooooo}{
    \begin{infoBox}
        \vspace{-0.05cm}
        \begin{tcbitemize}[raster rows=1,raster columns=20,raster height=1.65cm,
            raster every box/.style={colframe=red!50!black,colback=red!10!white}]
            \tcbitem[raster multicolumn=6] \textbf{Date :} #1
            \tcbitem[raster multicolumn=10] \textbf{Séquence :} #2
            \tcbitem[raster multicolumn=4] \textbf{Séance :} #3
        \end{tcbitemize}
        \vspace{-0.25cm}
        \ifNotNull{#4}{\tcbline \textbf{Objectif :} #4}
        \ifNotNull{#5}{\tcbline \leconInfoBox{Classe(s)}{#5}}
        \ifNotNull{#6}{\tcbline \leconInfoBox{Prérequi(s)}{#6}}
        \ifNotNull{#7}{\tcbline \textbf{Séance précédente :} #7}
        \ifNotNull{#7}{\tcbline \leconInfoBox{Matériel(s)}{#8}}
    \end{infoBox}
}

\def\pDscr{\tcbitem[enhanced jigsaw, breakable,
    raster multicolumn=6]
}
\def\pMdlt{\tcbitem[enhanced jigsaw, breakable,
    raster multicolumn=11]
}
\def\pTime{\tcbitem[enhanced jigsaw, breakable,
    raster multicolumn=3, halign=center]
}

\newcommand{\prepRow}[3]{
    \tcbitem[raster multicolumn=20]
    \tcblower

    \pDscr #1
    \pMdlt #2
    \pTime #3
}

\newcommand{\prepTable}[1]{
    \begin{prepBox}
        \begin{tcbitemize}[enhanced jigsaw, breakable, raster rows=1,raster columns=20,raster height=1.1cm, halign=center,
            raster every box/.style={enhanced jigsaw, breakable, colframe=Blue!50!black,colback=Blue!10!white}]
            \pDscr \textbf{Descriptif}
            \pMdlt \textbf{Modalité}
            \pTime \textbf{Durée}
        \end{tcbitemize}
        \begin{tcbitemize}[enhanced jigsaw, breakable,
            raster equal height = rows, 
            raster columns=20, frame hidden,
            raster every box/.style={
                enhanced jigsaw, breakable,
                opacityback=0, valign=top, 
                size = tight
            }]
            #1
        \end{tcbitemize}
    \end{prepBox}
}

% TIKZ

\newcommand{\ctikz}[1]{
    \begin{center}
        \begin{tikzpicture}
            #1
        \end{tikzpicture}
    \end{center}
}

\newcommand{\axis}[1]{%Draw coordinate axes
    \draw[thin, -Stealth] (-0.5,0) -- (#1,0);% node[right] {$x$}; % x-axis
    \draw[thin, -Stealth] (0,-0.5) -- (0,#1);% node[above] {$y$}; % y-axis
}

\newcommand{\drawGrid}[3]{
    \foreach \n in {0,...,#1}
        \draw[line width = #3] (\n,0) -- (\n,#2);
    \foreach \n in {0,...,#2}
        \draw[line width = #3] (0,\n) -- (#1,\n);
}

\newcommand{\drawPoint}[4]{
    \node[shift={#4}, color = \pointColor] at (#2 - 0.5,#3 - 0.5) {#1};
    \draw[line width = \crossWidth, shift={#4}, color = \pointColor] (#2 - 0.25,#3) -- (#2 + 0.25,#3);
    \draw[line width = \crossWidth, shift={#4}, color = \pointColor] (#2,#3 - 0.25) -- (#2,#3 + 0.25);
}

% Tabular
\newcolumntype{C}[1]{>{\centering\arraybackslash}p{#1}}
\newcolumntype{M}[1]{>{\centering\arraybackslash}m{#1}}
\newcolumntype{K}{@{}m{0pt}@{}}

% GEOMETRY

% \newcommand{\restoregeometry}{def}

\newcommand{\multiColItemize}[2]{
    \begin{multicols}{#1}
        \begin{itemize}
            #2
        \end{itemize}
    \end{multicols}
}

\newcommand{\multiColEnumerate}[2]{
    \begin{multicols}{#1}
        \begin{enumerate}
            #2
        \end{enumerate}
    \end{multicols}
}

\makeatletter
\newcommand\pgfinvisible{\pgfsys@begininvisible}
\newcommand\pgfshown{\pgfsys@endinvisible}
\makeatother

\renewcommand*{\phantom}[1]{
    \pgfinvisible #1 \pgfshown
}

\newcounter{size}
\newcommand{\listSize}[1]{%
    \setcounter{size}{0}%
    \foreach \n in {#1}{\stepcounter{size}}%
    % \thesize
}

\newcounter{elemPos}
\newcommand{\listElement}[2]{
    \setcounter{elemPos}{0} % Start counting from 1
    \def\resultVal{0} % Default value
    \renewcommand*{\do}[1]{%
        \ifnumequal{\value{elemPos}}{#2}{%
            \def\resultVal{##1}%
            \listbreak% Break out of the loop
        }{}%
        \stepcounter{elemPos}%
    }
    % \docsvlist{#1}
    \expandafter\docsvlist\expandafter{#1} % Expand the list before passing it to \docsvlist
    \resultVal
}

% \NewDocumentCommand{\exoslide}{m O{10cm}}{
%     \slide{}{
%         \img{\imgf{#1}}[#2]
%     }
% }

\NewDocumentCommand{\exoSlide}{m O{10cm} O{1} O{} O{exo}}{%
    \slide{#5}{%
        \ifthenelse{\equal{#3}{1}}{\vspace{-0.5cm}}{\vspace{-1cm}}
        \def\exercices{\foreach \q in {#1}{\imgp{\q}[#2]\vspace{-0.5cm}}}
        \exo{#1}{\wideFrame[7em]{\bvspace{0.25cm}\avspace{-0.25cm}
            \ifthenelse{\equal{#3}{1}}{\exercices}
            {\begin{multicols}{#3}\exercices\end{multicols}}}
            \avspace{0.75cm}
        }[#4]
    }
}

\NewDocumentCommand{\exoList}{m O{} O{3}}{%
    \section*{Exercices}%
    \slide{EXERCICES}{
        \exo{#2}{
            \vspace{-0.25cm}
            \multiColEnumerate{#3}{
                \foreach \q in {#1}{
                    \item \q
                }
            }
        }
    }
}

\newcommand{\questions}[1]{
    \begin{enumerate}
        \foreach \q in {#1}{
            \item \q\\
            \vspace*{-0.45cm}
            \dottedLines{3}
        }
    \end{enumerate}
}

% Define a new boolean for checking if the section is starred
\newboolean{section@star}

\makeatletter
% Redefine \section and \section* to set the boolean
\let\old@section\section
\renewcommand{\section}{%
    \@ifstar
        {\setboolean{section@star}{true}\old@section*}
        {\setboolean{section@star}{false}\old@section}%
}
\makeatother

\newcommand{\qt}[1]{«\textit{#1}»}

\newcommand{\calc}[1]{\numexpr#1\relax}
\newcommand{\ncalc}[1]{\number\calc{#1}}
\newcommand{\pcalc}[1]{\numprint{\ncalc{#1}}}

\newcommand{\setgrade}[1]{
    \def\grade{#1}
    % \begin{switch}{#1}
    %     \case{6e}{\global\definecolor{gradeColor}{hex}{FA8072}}
    %     \default{
    %         Default
    %         \global\definecolor{gradeColor}{RGB}{200, 50, 50}
    %     }
    % \end{switch}
    \ifthenelse{\equal{#1}{6e}}{
        \definecolor{gradeColor}{HTML}{C6233D} % FA8072 in hex
    }{
    \ifthenelse{\equal{#1}{5e}}{
        \definecolor{gradeColor}{HTML}{088255}
    }{
    \ifthenelse{\equal{#1}{4e}}{
        \definecolor{gradeColor}{HTML}{1466A8}
    }{
    \ifthenelse{\equal{#1}{3e}}{
        \definecolor{gradeColor}{HTML}{844499}
    }{
        \definecolor{gradeColor}{RGB}{0, 0, 0}
    }}}}
}

\gdef\phase{}
\newcommand{\setPhase}[1]{%
    \begin{switch}{#1}
        \case{exo}{\gdef\phase{EXERCICES}}
        \case{cr}{\gdef\phase{COURS}}
        \case{qf}{\gdef\phase{QUESTIONS FLASH}}
        \case{dm}{\gdef\phase{DEVOIR MAISON}}
        \default{\gdef\phase{#1}}
    \end{switch}
}

\newcommand\csn[1]{\csname #1\endcsname}

\newcommand{\vect}[1]{\ensuremath{\overrightarrow{#1}}}
% \newcommand{\vect}[1]{\overrightarrow{\,\mathstrut#1\,}}
\newcommand{\m}[1]{\ensuremath{\mathbf{#1}}}
\newcommand\lm[2]{\lim_{#1\to#2}}

\def\eqv{\Leftrightarrow}
\def\ssi{si et seulement si }
\def\pt{pour tout }
\def\poly2{fonction polynôme du second degré }
\def\eq2{équation second degré }
\def\discr{b^2-4ac}

% MATH TEXT
\def\et{\textrm{ et }}
\def\si{\textrm{ si }}
\def\avec{\textrm{ avec }}
\def\car{\textrm{ car }}
\def\alors{\textrm{ alors }}
\def\ou{\textrm{ ou }}
\def\ona{\textrm{ on a }}

\def\iet{\shortintertext{et}}
\def\ialors{\shortintertext{alors}}
\def\idou{\shortintertext{d'où}}
\def\ior{\shortintertext{or}}
\def\iona{\shortintertext{on a}}

\def\studentinfo{
    \vspace*{-1cm}
    \begin{minipage}{0.35\linewidth}
        nom: \dotfill
    \end{minipage}
    \begin{minipage}{0.35\linewidth}
        prénom: \dotfill
    \end{minipage}
    \begin{minipage}{0.15\linewidth}
        classes: \dotfill
    \end{minipage}
    
    \noindent\hrulefill
}

% UNITS
\def\cm{\,\centi\meter}
\def\km{\,\kilo\meter}
\newcommand{\defl}[2]{%
    \expandafter\def\csname #1\endcsname{\href{#2}{#1}\space}%
}

% Page Eduscol
\defl{Eduscol Cycle 3}{https://eduscol.education.fr/251/mathematiques-cycle-3}
\defl{Eduscol Cycle 4}{https://eduscol.education.fr/280/mathematiques-cycle-4}
\defl{Eduscol Lycée Général et technologique}{https://eduscol.education.fr/1723/programmes-et-resources-en-mathematiques-voie-gt}
\defl{Eduscol Lycée Professionnel}{https://eduscol.education.fr/1793/programmes-et-resources-en-mathematiques-voie-professionnelle}

% Repères annuels
\defl{Cycle 2}{https://eduscol.education.fr/document/13972/download}
\defl{Cycle 3}{https://eduscol.education.fr/document/14026/download}
\defl{Cycle 4}{https://eduscol.education.fr/document/14080/download}

% Attendus de fin d'année
\defl{CM2}{https://eduscol.education.fr/document/14002/download}
\defl{6e}{https://eduscol.education.fr/document/14014/download}
\defl{5e}{https://eduscol.education.fr/document/14044/download}
\defl{4e}{https://eduscol.education.fr/document/14056/download}
\defl{3e}{https://eduscol.education.fr/document/14068/download}

% Programme de mathématiques
\defl{cycle 3}{https://eduscol.education.fr/document/50990/download}
\defl{cycle 4}{https://cache.media.education.gouv.fr/file/31/89/1/ensel714_annexe3_1312891.pdf}
\defl{2nd}{https://eduscol.education.fr/document/24553/download}
\defl{2nd STHR}{https://eduscol.education.fr/document/24556/download}
\defl{1re}{https://eduscol.education.fr/document/24565/download}
\defl{1re Technologique}{https://eduscol.education.fr/document/24559/download}
\defl{Terminale Option Spécialité}{https://eduscol.education.fr/document/24568/download}
\defl{Terminale Option Complémentaire}{https://eduscol.education.fr/document/24571/download}
\defl{Terminale Option Expertes}{https://eduscol.education.fr/document/24574/download}
\defl{Terminale Technologique}{https://eduscol.education.fr/document/23107/download}

% resources thématiques
\defl{Proportionnalité}{https://eduscol.education.fr/document/17281/download}
\defl{Probabilités}{https://eduscol.education.fr/document/17275/download}
\defl{Traitement des données}{https://eduscol.education.fr/document/17269/download}

\defl{Fonctions}{https://eduscol.education.fr/document/17287/download}
\defl{Fractions}{https://eduscol.education.fr/document/17239/download}
\defl{Nombres relatifs}{https://eduscol.education.fr/document/17245/download}
\defl{Puissances}{https://eduscol.education.fr/document/17251/download}
\defl{Divisibilité et nombres premiers}{https://eduscol.education.fr/document/17257/download}
\defl{Calcul littéral}{https://eduscol.education.fr/document/17263/download}

\defl{Grandeurs et mesures}{https://eduscol.education.fr/document/17293/download}
\defl{Algorithmique et programmation}{https://eduscol.education.fr/document/17311/download}

\defl{Suites}{https://eduscol.education.fr/document/24586/download}
\defl{Produit Scalaire}{https://eduscol.education.fr/document/24589/download}
\defl{Raisonnement et démonstration (seconde)}{https://eduscol.education.fr/document/24580/download}
\defl{Raisonnement et démonstrations (première)}{https://eduscol.education.fr/document/24583/download}

\def\jules{\href{https://juels.dev/}{Jules PESIN}}
\def\yuyu{\href{https://www.instagram.com/yuyuvrajav/}{@yuyuvraj}}

\defl{Utiliser les notions de géométrie planepour démontrer}{https://eduscol.education.fr/document/17305/download}

% Manuels
\def\dim{\href{https://www.editions-hatier.fr/livre/dimensions-mathematiques-6e-ed-2016-manuel-de-leleve-9782401020023}
    {Dimensions 6e (Ed. 2016)}
}

\definecolor{myriade}{HTML}{0F83B3} %#0F83B3
\def\my{\href{https://www.editions-bordas.fr/ouvrage/myriade-mathematiques-6e-manuel-de-leleve-ed-2021-9782047337752.html}
    {Myriade 6e (Ed. 2021)}
}

\def\mm{\href{https://www.editions-hatier.fr/livre/maths-monde-cycle-4-livre-1-volume-9782278083459}
    {Maths Monde cycle 4 (Ed. 2016)}
}

\def\mi{\href{https://www.enseignants.hachette-education.com/livres/mission-indigo-mathematiques-cycle-4-5e-4e-3e-livre-eleve-ed-2017-9782013953962}
    {Mission Indigo mathématiques cycle 4 éd. 2017}
}

% https://www.armitiere.com/livre/1833174-des-maths-ensemble-et-pour-chacun-5e-mise-en--jean-philippe-rouques-helene-stainer-canope-crdp-44
\def\dmeepcC{\href{https://publimath.univ-irem.fr/PCO10003}
    {Des maths ensemble et pour chacun 5e}
}

\def\dmeepcS{\href{https://www.reseau-canope.fr/notice/des-maths-ensemble-et-pour-chacun-6e.html}
    {Des maths ensemble et pour chacun 6e}
}

\NewDocumentCommand{\dmeepc}{m O{}}{%
    \href{https://www.reseau-canope.fr/notice/des-maths-ensemble-et-pour-chacun-6e.html}
    {Des maths ensemble et pour chacun #1e \ifNotNull{#2}{(p.#2)}}
}

\NewDocumentCommand{\sesa}{m m O{} O{}}{%
    \href{https://manuel.sesamath.net/numerique/index.php?ouvrage=cm#1_#2&page_gauche=#4}{%
    Sésamath #1e #2 \ifNotNull{#4}{(#3 p.#4)}
    }
}

\NewDocumentCommand{\iP}{m m O{} O{}}{%
    \href{https://www.iparcours.fr/ouvrages/ouvrages.php?ouvrage=Cahier#1#2}{%
    iParcours #1e #2 \ifNotNull{#4}{(#3 p.#4)}
    }
}

\NewDocumentCommand{\ching}{m m O{}}{%
    \href{https://chingmath.fr/#1eme/#2}{%
    Ching@Math #1e (\reverseKebabCase{#2}\ifNotNull{#3}{ E.#3})
    }
}

\NewDocumentCommand{\wiki}{m O{}}{%
    \def\ext{}%
    \ifNotNull{#2}{\def\ext{\##2}}%
    \href{https://fr.wikipedia.org/wiki/#1\ext}%
    {Wikipédia (\reverseSnakeCase{#1}%
    \ifNotNull{#2}{ {\scriptsize $\rightarrow$ \reverseSnakeCase{#2}}}%
    )}
}

\newcommand*{\prbltq}[1]{\href{https://www.problematheque-csen.fr/fiche-probleme/#1}{Problémathèque (\reverseKebabCase{#1})}}

\NewDocumentCommand{\rpmc}{O{}}{%
    \href{https://eduscol.education.fr/document/13132/download?attachment\#page=#1}{%
    La résolution de problèmes mathématiques au collège
    (p.%
    #1%
    % \directlua{tex.print((tonumber("#1") or 0) + 3)}
    )}%
}

% Attendus de fin d'année
\NewDocumentCommand{\afa}{m O{}}{
    \ifthenelse{\equal{#1}{CM2}}{
        \def\afalink{https://eduscol.education.fr/document/14002/download}
    }{
    \ifthenelse{\equal{#1}{6e}}{
        \def\afalink{https://eduscol.education.fr/document/14014/download}
    }{
    \ifthenelse{\equal{#1}{5e}}{
        \def\afalink{https://eduscol.education.fr/document/14044/download}
    }{
    \ifthenelse{\equal{#1}{4e}}{
        \def\afalink{https://eduscol.education.fr/document/14056/download}
    }{
    \ifthenelse{\equal{#1}{3e}}{
        \def\afalink{https://eduscol.education.fr/document/14068/download}
    }{
        \def\afalink{https://eduscol.education.fr/document/14014/download}
    }}}}}
    \def\page{}
    \ifNotNull{#2}{\def\page{(p.#2)}}
    \href{\afalink\#page=#2}{Attendus de fin d'année de #1 \page}
}

\def\ca{%
    \href{https://pedagogie.ac-strasbourg.fr/mathematiques/competitions/course-aux-nombres/}%
    {Course aux nombres}%
}

% Euclide https://www.pedagogie.ac-aix-marseille.fr/jcms/c_10743971/it/les-elements-d-euclide-traduction-par-oliver-byrne

\NewDocumentCommand{\eucl}{O{1804} O{}}{
    \ifthenelse{\equal{#1}{1632}}{ % 1632
        \def\trad{D. Henrion}
        \def\afalink{https://www.pedagogie.ac-aix-marseille.fr/upload/docs/application/pdf/2019-11/elements_euclide_-_denis_henrion.pdf}
    }{
    \ifthenelse{\equal{#1}{1804}}{ % 1804 traduction F. Peyrard
        \def\trad{F. Peyrard}
        \def\afalink{https://eduscol.education.fr/document/14014/download}
    }{}
    }
    \def\page{}
    \ifNotNull{#2}{\def\page{p.#2}}
    \href{\afalink\#page=#2}{Les Éléments d'Euclide (traduction de \trad \page)}
}
% 1632 https://www.pedagogie.ac-aix-marseille.fr/upload/docs/application/pdf/2019-11/elements_euclide_-_denis_henrion.pdf
% Logiciels
\newcommand{\defIconLink}[4]{% 1 text , 2 : color , 3 : icon , 4 : link
    \expandafter\def\csname #1\endcsname{%
        {\def\iconPath{}%
        \icon{#3} \textbf{\href{#4}{\color{#2}#1}}}
    }%
}

\newcommand{\cmdIconLink}[4]{% 1 text , 2 : color , 3 : icon , 4 : link
    \expandafter\NewDocumentCommand\csname cmd#1\endcsname{O{}}
    {%
        {\def\iconPath{}%
        \icon{#3} \textbf{\href{#4/##1}{\color{#2}#1}}}
    }%
}

\definecolor{capytale}{HTML}{1E293B} % #1E293B
\definecolor{capytale-2}{HTML}{F0F1F2} % #F0F1F2

\def\Capytale{%
    \href{https://capytale2.ac-paris.fr/~/my}{\shl{capytale}{capytale-2}{CAPYTALE}}%
}

\newcommand{\capytale}[1]{%
    \href{https://capytale2.ac-paris.fr/web/c/#1}{\shl{capytale}{capytale-2}{CAPYTALE \shl{capytale-2}{capytale}{#1}}}%\shl{capytale}{capytale-2}{CAPYTALE 
}

\definecolor{enc}{HTML}{1D3D6E} % #1D3D6E
\defIconLink{ENC}{enc}{ENC-Hauts-de-Seine}{https://enc.hauts-de-seine.fr/}

\definecolor{pronote}{HTML}{1A6E45} % #1A6E45
\defIconLink{Pronote}{pronote}{pronote}{https://0922247t.index-education.net/pronote/}

\definecolor{calc}{HTML}{00A500} % #00A500
\defIconLink{Calc}{calc}{libreOffice/calc/logo}{https://fr.libreoffice.org/discover/calc/}

\definecolor{geogebra}{HTML}{9693F7} % #9693F7
\defIconLink{Geogebra}{geogebra}{geogebra/logo}{https://www.geogebra.org/classic}
\cmdIconLink{Geogebra}{geogebra}{geogebra/logo}{https://www.geogebra.org/m}

\definecolor{scratch}{HTML}{FFAB19} % #FFAB19
\defIconLink{Scratch}{Orange}{scratch/logo}{https://scratch.mit.edu/projects/editor/}

% http://trucsmaths.free.fr/etymologie.htm

\newcommand{\dym}[1]{\def\ym{\href{#1}{Yvan Monka}}}

\captionsetup{labelformat=empty,labelsep=none}

% ANNE
\setboolean{boxedProperties}{true} % false = edge
\setboolean{parenthisedID}{false}
\setboolean{showID}{false}

\def\DefinitionColor{Red}
\def\PropertyColor{Red}
\def\TheoremColor{Red}

% TIKZ
\def\crossWidth{0.25mm}
\def\pointColor{blue}

\usepackage{bookmark}
% \usepackage{unicode-math}

% \usefonttheme[onlymath]{serif}

% \setmainfont{Libertinus Serif}
% \setmathfont{Libertinus Math}

\usetheme{Madrid}
% \usetheme{shadow}
% \usetheme{CambridgeUS}
% \usetheme{AnnArbor}
% \usecolortheme{spruce}
\usecolortheme{beaver}

\setbeamersize{
    text margin left=1.5cm,
    text margin right=1.5cm
}

% \setbeamertemplate{enumerate items}[square]
\setbeamertemplate{enumerate items}[default]
\def\authors{Jules PESIN}
\def\longTitle{long Title}
\def\shortTitle{short Title}
% \def\day{XX/XX/XX}

\title[\shortTitle]{\longTitle}
% \date{\day}

\newcommand{\slide}[2]{
    \begin{frame}
    \frametitle[#1]{#1}
        #2
    \end{frame}
}

\newcounter{sec}
% \stepcounter{sec}
\newcounter{subsec}
% \stepcounter{subsec}

\newcommand{\bchap}[1]{
    \color{Red} CHAPITRE : #1\color{black}\\
}

\newcommand{\bsec}[1]{
    \def\ssec{\color{Red} \Roman{sec}. #1\color{black}\\}
    \stepcounter{sec}
    \setcounter{subsec}{0}
}

\newcommand{\bsubsec}[1]{
    \def\ssubsec{\color{Green} \thesubsec) #1\color{black}\\}
    \stepcounter{subsec}
}

\newcommand{\palt}[2]{
    \alt<#1>{#2}{\phantom{#2}}
}

\newcounter{question}

\newcommand{\startQuestions}{
    \setcounter{question}{2}
}

\newcommand{\iquestion}[2]{
    \item $\question{#1}{#2}$
}

\newcommand{\question}[2]{
        #1 = \onslide<\thequestion->{#2}
        \stepcounter{question}
}

% \renewcommand{\question}[2]{
%         #1 = #2
% }


\newcommand{\disableAnimation}{
    \renewcommand{\question}[2]{
        ##1 = ##2
    }
    
    \renewcommand{\palt}[2]{
        ##2
    }
}

\newcommand{\shortAnimation}{
    \renewcommand{\question}[2]{
        ##1 = \onslide<2->{##2}
    }
}

\newcommand{\firstSlide}{
    \renewcommand{\question}[2]{
        ##1 =
    }

    \renewcommand{\palt}[2]{
        \phantom{##2}
    }
}


%% Article %%
\documentclass[a4paper, 12pt, 
% landscape
]{extarticle} \usepackage[top=1.5cm, bottom=2cm, left=2cm, right=2cm]{geometry}
\usepackage[dvipsnames, table]{xcolor}
\usepackage{lastpage}
\usepackage{fancyhdr}
\usepackage{titlesec}
\usepackage{enumitem}
\usepackage{longtable}
\usepackage{pdfpages}
% FANCYHDR

\setlength{\headheight}{18pt}
\fancyhead[C]{\normalsize \title}
\fancyhead[R]{}
\fancyhead[L]{}
\fancyfoot[L]{\authors}
\fancyfoot[C]{\textbf{Page \thepage/\pageref{LastPage}}}
\fancyfoot[R]{\date}

\fancypagestyle{firstpage}{
    \setlength{\headheight}{29pt}
    \fancyhead[C]{\LARGE \title}
}

\fancypagestyle{assignment}{
    \setlength{\headheight}{29pt}
    \fancyhead[C]{}
    \fancyhead[L]{\large \title}
    \fancyhead[R]{%
        \begin{tabular}{p{7.5cm}p{2.5cm}}%
            \normalsize nom:& \normalsize classe: \link{\grade}\_\\%
            \normalsize prénom:& \normalsize date:\\%
            % \normalsize date:\hspace*{3.5cm}%
        \end{tabular}%
    }
}

\fancypagestyle{empty}{
    \renewcommand{\headrulewidth}{0pt}
    \setlength{\headheight}{-10pt}
    \fancyhead[C]{}
    \fancyhead[R]{}
    \fancyhead[L]{}
    \fancyfoot[L]{}
    \fancyfoot[C]{}
    \fancyfoot[R]{}
}

\fancypagestyle{assignment-empty-foot}{
    \setlength{\headheight}{29pt}
    \fancyhead[C]{}
    \fancyhead[L]{\large \title}
    \fancyhead[R]{%
        \begin{tabular}{p{0.25\pdfpagewidth}p{0.15\pdfpagewidth}}%
            \normalsize nom:& \normalsize classe:\\%
            \normalsize prénom:& \normalsize date:\\%
            % \normalsize date:\hspace*{3.5cm}%
        \end{tabular}%
    }
    \fancyfoot[L]{}
    \fancyfoot[C]{}
    \fancyfoot[R]{}
}

\fancypagestyle{small}{
    \setlength{\headheight}{20pt}
    \fancyhead[C]{}
    \fancyhead[C]{\large \title}
    \fancyhead[L]{}
    \fancyhead[R]{}
    \fancyfoot[L]{}
    \fancyfoot[C]{}
    \fancyfoot[R]{}
}

\thispagestyle{firstpage}

% \fancyfoot[C]{\textbf{Page 1/1}}

\def\title{\theme}
\def\authors{Jules PESIN}

\pagestyle{fancy}

% \titleformat*{\section}{\small\bfseries}

\titleformat{\section}
{\normalfont\large\bfseries\color{\SectionColor}}{\thesection}{0.6em}{}

\titleformat{\subsection}
{\normalfont\normalsize\bfseries\color{\SubSectionColor}}{\thesubsection}{0.6em}{}

\titleformat{\subsubsection}
{\normalfont\small\bfseries}{\thesection}{0.6em}{}

\renewcommand{\theenumi}{\small\color{Blue}\arabic{enumi}}

\renewcommand{\labelenumii}{\scriptsize\color{RoyalBlue}\alph{enumii})}
% \renewcommand{\theenumii}{.\arabic{enumii}}
% \frenchbsetup{StandardItemLabels=true}
\renewcommand{\labelitemi}{$\color{Blue}.$}

% BEAMER CONVERSION

\newcommand{\bchap}[1]{\def\title{Chapitre: #1}}
\newcommand{\bsec}[1]{\section{#1}}
\newcommand{\bsubsec}[1]{\subsection{#1}}

\newcommand{\ssec}{}
\newcommand{\ssubsec}{}

\newcommand{\slide}[2]{#2}

\newcommand{\startQuestions}{}
\newcommand{\iquestion}[2]{\item $#1 = #2$}

\newcommand{\palt}[2]{#2}

\newcommand{\disableAnimation}{}
\newcommand{\shortAnimation}{}

\newcommand{\firstSlide}{
    \renewcommand{\iquestion}[2]{\item $##1 = \phantom{##2}$}
    \renewcommand{\palt}[2]{\phantom{##2}}
}



\usepackage{amsmath}
\usepackage{amsthm}
\usepackage{amssymb}
\usepackage{graphicx}
\usepackage{dashundergaps}
\usepackage{array}
\usepackage{multicol}
\usepackage{wrapfig}
\usepackage{numprint}
\usepackage{ulem}
\usepackage{hyperref}
\usepackage{mathrsfs}
\usepackage{mathtools}
\usepackage[many]{tcolorbox}
\usepackage{xparse}
\usepackage{float}
\usepackage{lipsum}
\usepackage{pgf}
\usepackage{ifthen}
\usepackage{caption}
\usepackage{tikz}
\usepackage{tkz-tab}
\usepackage{xifthen}
\usepackage{listings}
\usepackage[linesnumbered,vlined,boxed]{algorithm2e}
\usepackage[squaren,Gray]{SIunits}
\usepackage{animate}
\usepackage{eurosym}
\usepackage{tkz-euclide}
\usepackage{etoolbox}
\usepackage{textcase}
\usepackage{adjustbox}
% \usepackage{mathabx}

\usepackage{tabularx}
\usepackage{scratch3}

\usepackage{multicol}
\usepackage{multirow}
% \usepackage{appendix}
% \usepackage[
%     backend=biber,        % compilateur par défaut pour biblatex
%     sorting=nyt,          % trier par nom, année, titre
%     citestyle=authoryear, % style de citation auteur-année
%     bibstyle=alphabetic,  % style de bibliographie alphabétique
% ]{biblatex}

% \usepackage[T1]{fontenc}

% \usepackage{pifont}

% \usepackage[squaren,Gray]{SIunits}

% BREVET

% \usepackage{makeidx}
% \usepackage{fancybox}
% \usepackage{tabularx}
% \usepackage[normalem]{ulem}
% \usepackage{pifont}
% \usepackage{lscape}
% \usepackage{diagbox}
% \usepackage{multido}
% \usepackage[dvipsnames]{pstricks}
% \usepackage{pst-plot,pst-text,pst-tree,pstricks-add}
% \usepackage{textcomp}
% \usepackage{scratch3}
% \usepackage[T1]{fontenc}
% \usepackage{fourier}
% \usepackage[french]{babel}
% \usepackage{pstricks}

% \usepackage[scaled=0.875]{helvet}
% \usepackage{pst-plot,pst-text,pst-tree,pstricks-add}

% hyperref

\hypersetup{
    colorlinks=true,       % false: boxed links; true: colored links
    linkcolor=red,          % color of internal links (change box color with linkbordercolor)
    citecolor=green,        % color of links to bibliography
    filecolor=magenta,      % color of file links
    urlcolor=blue,          % color of external links
    urlbordercolor=blue,    % borders of external links
    linkbordercolor=red,    % borders of internal links
    pdfborderstyle={/S/U/W 1}% border style will be underline of width 1pt
}

\frenchbsetup{StandardItemLabels=true}

% listings

\lstdefinestyle{pythonStyle}{
    language=Python,
    basicstyle=\ttfamily\small,  % Adjust the font and size
    commentstyle=\itshape\color{green!40!black},
    keywordstyle=\bfseries\color{violet},
    numbers=left,
    numberstyle=\tiny\color{gray},
    frame=single,
    breaklines=true,
    breakatwhitespace=true,
    tabsize=4,
    captionpos=b,
    identifierstyle=\color{Blue},
}

\lstset{style=pythonStyle}

% algorithm2e

\SetNlSty{}{}{.}
\SetKwInput{KwRes}{R\'esultat}%
\SetKwIF{Si}{SinonSi}{Sinon}{si}{alors}{sinon si}{sinon}{fin si}%
\SetKwFor{Tq}{tant que}{faire}{fin tq}%

% GLOBAL VARIABLES %%%
\graphicspath{{images}}
\def\cwidth{4cm}
\def\tspace{0.5cm}

% BOOLEAN %%%
\newboolean{anwser}
\newboolean{demonstration}
\newboolean{boxedProperties}
\newboolean{showID}
\newboolean{parenthisedID}
\newboolean{animated}
\newboolean{outline}

\setboolean{anwser}{false}
\setboolean{demonstration}{true}
\setboolean{parenthisedID}{true}
\setboolean{showID}{true}
\setboolean{boxedProperties}{false} % false = edge
\setboolean{outline}{false}

\def\DefinitionColor{PineGreen}
\def\PropertyColor{Blue}
\def\TheoremColor{Plum}

\def\SectionColor{Red}
\def\SubSectionColor{Green}

\setboolean{animated}{true}

% Switch implementation
\newboolean{default}
\newcommand{\case}{}
\newcommand{\default}{}

\newenvironment{switch}[1]{%
    \setboolean{default}{true}
    \renewcommand{\case}[2]{\ifthenelse{\equal{#1}{##1}}{%
        \setboolean{default}{false}##2}{}}%
    \renewcommand{\default}[1]{\ifthenelse{\boolean{default}}{##1}{}}
}{}

% SECTIONS
\input{header/command/sections.tex}

% ANSWERS
\newlength{\parline}
\newlength{\paroutindent}
\newlength{\lineheight}
\setlength{\lineheight}{\heightof{abcdefghijklmnoprstuvwxyz}}

\newcommand{\countlines}[1]{%
    \setlength{\paroutindent}{\expandafter\parindent}
    \setlength{\parline}{\heightof{\noindent\begin{minipage}{\linewidth}%
                \setlength{\parindent}{\paroutindent}#1\end{minipage}}}%
    \pgfmathparse{round(\parline / (0.9*\lineheight))}
    \newcount\linecount
    \pgfmathsetcount{\linecount}{\pgfmathresult}
}

\newcommand{\looptext}[2]{%
    \noindent
    \newcount\printcount
    \printcount=#2
    \loop
        #1
        \advance\printcount by -1
        \ifnum\printcount>0
    \repeat
}

\newcommand{\awsr}[1]{%
    \ifthenelse{\boolean{answer}}{
        \result{#1}
    }{
        \countlines{#1}
        \pgfmathsetcount{\linecount}{\linecount + 1}
        \noindent\hspace{-9pt}
        \looptext{
            \noindent\dotfill
    
        }{\the\linecount}
    }
}

\newcommand{\dottedLines}[1]{%
    \noindent\hspace{-9pt}%

    \looptext{%
        \noindent\dotfill%

    }{#1}
}

\newcommand{\result}[1]{\color{OrangeRed}#1 \color{black}%
}

% MATH
\input{header/command/math.tex}

% IMAGES
\input{header/command/image.tex}

% COMMANDS

\newcommand{\fsize}[1]{\fontsize{#1}{#1}\selectfont}

\NewDocumentCommand{\ifNotNull}{mmo}{
    \IfValueT{#1}{
        \ifx\relax#1\relax
            \IfValueT{#3}{
                #3
            }
        \else
            #2
        \fi
    }
}

\NewDocumentCommand{\ilink}{m g}{%
    \item
    \IfValueTF{#2}{\link{#1}{#2}}{\link{#1}}
}

\NewDocumentCommand{\link}{m g}{%
    \csn{#1}%
    \IfValueT{#2}{(#2)}%
}

\NewDocumentCommand{\TODO}{g}{%
    {\color{Red} $\rightarrow$ \textbf{TODO}
    \IfValueT{#1}{(#1)}}
    % \color{black}
}

\newcommand{\leconInfoBox}[2]{
    \textbf{#1 :}\vspace{-0.25cm}
        \begin{multicols}{2}
            \begin{itemize}[label=$\blacktriangleright$, font = \small \color{Red}]
                #2
            \end{itemize}
        \end{multicols}
        \vspace{-0.4cm}
}

% TCOLORBOX

\input{header/command/tcolorbox.tex}

\NewDocumentCommand{\leconInfo}{mooo}{
    \begin{infoBox}
        \leconInfoBox{Niveaux}{#1}
        \ifNotNull{#2}{
            \tcbline
            \leconInfoBox{Prérequis}{#2}
        }
        \ifNotNull{#3}{
            \tcbline
            \leconInfoBox{Thèmes}{#3}
        }
        \ifNotNull{#4}{
            \tcbline
            \textbf{Motivation :} 
            #4
        }
    \end{infoBox}
}

\NewDocumentCommand{\seanceInfo}{oooooooo}{
    \begin{infoBox}
        \vspace{-0.05cm}
        \begin{tcbitemize}[raster rows=1,raster columns=20,raster height=1.65cm,
            raster every box/.style={colframe=red!50!black,colback=red!10!white}]
            \tcbitem[raster multicolumn=6] \textbf{Date :} #1
            \tcbitem[raster multicolumn=10] \textbf{Séquence :} #2
            \tcbitem[raster multicolumn=4] \textbf{Séance :} #3
        \end{tcbitemize}
        \vspace{-0.25cm}
        \ifNotNull{#4}{\tcbline \textbf{Objectif :} #4}
        \ifNotNull{#5}{\tcbline \leconInfoBox{Classe(s)}{#5}}
        \ifNotNull{#6}{\tcbline \leconInfoBox{Prérequi(s)}{#6}}
        \ifNotNull{#7}{\tcbline \textbf{Séance précédente :} #7}
        \ifNotNull{#7}{\tcbline \leconInfoBox{Matériel(s)}{#8}}
    \end{infoBox}
}

\def\pDscr{\tcbitem[enhanced jigsaw, breakable,
    raster multicolumn=6]
}
\def\pMdlt{\tcbitem[enhanced jigsaw, breakable,
    raster multicolumn=11]
}
\def\pTime{\tcbitem[enhanced jigsaw, breakable,
    raster multicolumn=3, halign=center]
}

\newcommand{\prepRow}[3]{
    \tcbitem[raster multicolumn=20]
    \tcblower

    \pDscr #1
    \pMdlt #2
    \pTime #3
}

\newcommand{\prepTable}[1]{
    \begin{prepBox}
        \begin{tcbitemize}[enhanced jigsaw, breakable, raster rows=1,raster columns=20,raster height=1.1cm, halign=center,
            raster every box/.style={enhanced jigsaw, breakable, colframe=Blue!50!black,colback=Blue!10!white}]
            \pDscr \textbf{Descriptif}
            \pMdlt \textbf{Modalité}
            \pTime \textbf{Durée}
        \end{tcbitemize}
        \begin{tcbitemize}[enhanced jigsaw, breakable,
            raster equal height = rows, 
            raster columns=20, frame hidden,
            raster every box/.style={
                enhanced jigsaw, breakable,
                opacityback=0, valign=top, 
                size = tight
            }]
            #1
        \end{tcbitemize}
    \end{prepBox}
}

% TIKZ

\newcommand{\ctikz}[1]{
    \begin{center}
        \begin{tikzpicture}
            #1
        \end{tikzpicture}
    \end{center}
}

\newcommand{\axis}[1]{%Draw coordinate axes
    \draw[thin, -Stealth] (-0.5,0) -- (#1,0);% node[right] {$x$}; % x-axis
    \draw[thin, -Stealth] (0,-0.5) -- (0,#1);% node[above] {$y$}; % y-axis
}

\newcommand{\drawGrid}[3]{
    \foreach \n in {0,...,#1}
        \draw[line width = #3] (\n,0) -- (\n,#2);
    \foreach \n in {0,...,#2}
        \draw[line width = #3] (0,\n) -- (#1,\n);
}

\newcommand{\drawPoint}[4]{
    \node[shift={#4}, color = \pointColor] at (#2 - 0.5,#3 - 0.5) {#1};
    \draw[line width = \crossWidth, shift={#4}, color = \pointColor] (#2 - 0.25,#3) -- (#2 + 0.25,#3);
    \draw[line width = \crossWidth, shift={#4}, color = \pointColor] (#2,#3 - 0.25) -- (#2,#3 + 0.25);
}

% Tabular
\newcolumntype{C}[1]{>{\centering\arraybackslash}p{#1}}
\newcolumntype{M}[1]{>{\centering\arraybackslash}m{#1}}
\newcolumntype{K}{@{}m{0pt}@{}}

% GEOMETRY

% \newcommand{\restoregeometry}{def}

\newcommand{\multiColItemize}[2]{
    \begin{multicols}{#1}
        \begin{itemize}
            #2
        \end{itemize}
    \end{multicols}
}

\newcommand{\multiColEnumerate}[2]{
    \begin{multicols}{#1}
        \begin{enumerate}
            #2
        \end{enumerate}
    \end{multicols}
}

\makeatletter
\newcommand\pgfinvisible{\pgfsys@begininvisible}
\newcommand\pgfshown{\pgfsys@endinvisible}
\makeatother

\renewcommand*{\phantom}[1]{
    \pgfinvisible #1 \pgfshown
}

\newcounter{size}
\newcommand{\listSize}[1]{%
    \setcounter{size}{0}%
    \foreach \n in {#1}{\stepcounter{size}}%
    % \thesize
}

\newcounter{elemPos}
\newcommand{\listElement}[2]{
    \setcounter{elemPos}{0} % Start counting from 1
    \def\resultVal{0} % Default value
    \renewcommand*{\do}[1]{%
        \ifnumequal{\value{elemPos}}{#2}{%
            \def\resultVal{##1}%
            \listbreak% Break out of the loop
        }{}%
        \stepcounter{elemPos}%
    }
    % \docsvlist{#1}
    \expandafter\docsvlist\expandafter{#1} % Expand the list before passing it to \docsvlist
    \resultVal
}

% \NewDocumentCommand{\exoslide}{m O{10cm}}{
%     \slide{}{
%         \img{\imgf{#1}}[#2]
%     }
% }

\NewDocumentCommand{\exoSlide}{m O{10cm} O{1} O{} O{exo}}{%
    \slide{#5}{%
        \ifthenelse{\equal{#3}{1}}{\vspace{-0.5cm}}{\vspace{-1cm}}
        \def\exercices{\foreach \q in {#1}{\imgp{\q}[#2]\vspace{-0.5cm}}}
        \exo{#1}{\wideFrame[7em]{\bvspace{0.25cm}\avspace{-0.25cm}
            \ifthenelse{\equal{#3}{1}}{\exercices}
            {\begin{multicols}{#3}\exercices\end{multicols}}}
            \avspace{0.75cm}
        }[#4]
    }
}

\NewDocumentCommand{\exoList}{m O{} O{3}}{%
    \section*{Exercices}%
    \slide{EXERCICES}{
        \exo{#2}{
            \vspace{-0.25cm}
            \multiColEnumerate{#3}{
                \foreach \q in {#1}{
                    \item \q
                }
            }
        }
    }
}

\newcommand{\questions}[1]{
    \begin{enumerate}
        \foreach \q in {#1}{
            \item \q\\
            \vspace*{-0.45cm}
            \dottedLines{3}
        }
    \end{enumerate}
}

% Define a new boolean for checking if the section is starred
\newboolean{section@star}

\makeatletter
% Redefine \section and \section* to set the boolean
\let\old@section\section
\renewcommand{\section}{%
    \@ifstar
        {\setboolean{section@star}{true}\old@section*}
        {\setboolean{section@star}{false}\old@section}%
}
\makeatother

\newcommand{\qt}[1]{«\textit{#1}»}

\newcommand{\calc}[1]{\numexpr#1\relax}
\newcommand{\ncalc}[1]{\number\calc{#1}}
\newcommand{\pcalc}[1]{\numprint{\ncalc{#1}}}

\newcommand{\setgrade}[1]{
    \def\grade{#1}
    % \begin{switch}{#1}
    %     \case{6e}{\global\definecolor{gradeColor}{hex}{FA8072}}
    %     \default{
    %         Default
    %         \global\definecolor{gradeColor}{RGB}{200, 50, 50}
    %     }
    % \end{switch}
    \ifthenelse{\equal{#1}{6e}}{
        \definecolor{gradeColor}{HTML}{C6233D} % FA8072 in hex
    }{
    \ifthenelse{\equal{#1}{5e}}{
        \definecolor{gradeColor}{HTML}{088255}
    }{
    \ifthenelse{\equal{#1}{4e}}{
        \definecolor{gradeColor}{HTML}{1466A8}
    }{
    \ifthenelse{\equal{#1}{3e}}{
        \definecolor{gradeColor}{HTML}{844499}
    }{
        \definecolor{gradeColor}{RGB}{0, 0, 0}
    }}}}
}

\gdef\phase{}
\newcommand{\setPhase}[1]{%
    \begin{switch}{#1}
        \case{exo}{\gdef\phase{EXERCICES}}
        \case{cr}{\gdef\phase{COURS}}
        \case{qf}{\gdef\phase{QUESTIONS FLASH}}
        \case{dm}{\gdef\phase{DEVOIR MAISON}}
        \default{\gdef\phase{#1}}
    \end{switch}
}

\newcommand\csn[1]{\csname #1\endcsname}

\newcommand{\vect}[1]{\ensuremath{\overrightarrow{#1}}}
% \newcommand{\vect}[1]{\overrightarrow{\,\mathstrut#1\,}}
\newcommand{\m}[1]{\ensuremath{\mathbf{#1}}}
\newcommand\lm[2]{\lim_{#1\to#2}}

\def\eqv{\Leftrightarrow}
\def\ssi{si et seulement si }
\def\pt{pour tout }
\def\poly2{fonction polynôme du second degré }
\def\eq2{équation second degré }
\def\discr{b^2-4ac}

% MATH TEXT
\def\et{\textrm{ et }}
\def\si{\textrm{ si }}
\def\avec{\textrm{ avec }}
\def\car{\textrm{ car }}
\def\alors{\textrm{ alors }}
\def\ou{\textrm{ ou }}
\def\ona{\textrm{ on a }}

\def\iet{\shortintertext{et}}
\def\ialors{\shortintertext{alors}}
\def\idou{\shortintertext{d'où}}
\def\ior{\shortintertext{or}}
\def\iona{\shortintertext{on a}}

\def\studentinfo{
    \vspace*{-1cm}
    \begin{minipage}{0.35\linewidth}
        nom: \dotfill
    \end{minipage}
    \begin{minipage}{0.35\linewidth}
        prénom: \dotfill
    \end{minipage}
    \begin{minipage}{0.15\linewidth}
        classes: \dotfill
    \end{minipage}
    
    \noindent\hrulefill
}

% UNITS
\def\cm{\,\centi\meter}
\def\km{\,\kilo\meter}
\newcommand{\defl}[2]{%
    \expandafter\def\csname #1\endcsname{\href{#2}{#1}\space}%
}

% Page Eduscol
\defl{Eduscol Cycle 3}{https://eduscol.education.fr/251/mathematiques-cycle-3}
\defl{Eduscol Cycle 4}{https://eduscol.education.fr/280/mathematiques-cycle-4}
\defl{Eduscol Lycée Général et technologique}{https://eduscol.education.fr/1723/programmes-et-resources-en-mathematiques-voie-gt}
\defl{Eduscol Lycée Professionnel}{https://eduscol.education.fr/1793/programmes-et-resources-en-mathematiques-voie-professionnelle}

% Repères annuels
\defl{Cycle 2}{https://eduscol.education.fr/document/13972/download}
\defl{Cycle 3}{https://eduscol.education.fr/document/14026/download}
\defl{Cycle 4}{https://eduscol.education.fr/document/14080/download}

% Attendus de fin d'année
\defl{CM2}{https://eduscol.education.fr/document/14002/download}
\defl{6e}{https://eduscol.education.fr/document/14014/download}
\defl{5e}{https://eduscol.education.fr/document/14044/download}
\defl{4e}{https://eduscol.education.fr/document/14056/download}
\defl{3e}{https://eduscol.education.fr/document/14068/download}

% Programme de mathématiques
\defl{cycle 3}{https://eduscol.education.fr/document/50990/download}
\defl{cycle 4}{https://cache.media.education.gouv.fr/file/31/89/1/ensel714_annexe3_1312891.pdf}
\defl{2nd}{https://eduscol.education.fr/document/24553/download}
\defl{2nd STHR}{https://eduscol.education.fr/document/24556/download}
\defl{1re}{https://eduscol.education.fr/document/24565/download}
\defl{1re Technologique}{https://eduscol.education.fr/document/24559/download}
\defl{Terminale Option Spécialité}{https://eduscol.education.fr/document/24568/download}
\defl{Terminale Option Complémentaire}{https://eduscol.education.fr/document/24571/download}
\defl{Terminale Option Expertes}{https://eduscol.education.fr/document/24574/download}
\defl{Terminale Technologique}{https://eduscol.education.fr/document/23107/download}

% resources thématiques
\defl{Proportionnalité}{https://eduscol.education.fr/document/17281/download}
\defl{Probabilités}{https://eduscol.education.fr/document/17275/download}
\defl{Traitement des données}{https://eduscol.education.fr/document/17269/download}

\defl{Fonctions}{https://eduscol.education.fr/document/17287/download}
\defl{Fractions}{https://eduscol.education.fr/document/17239/download}
\defl{Nombres relatifs}{https://eduscol.education.fr/document/17245/download}
\defl{Puissances}{https://eduscol.education.fr/document/17251/download}
\defl{Divisibilité et nombres premiers}{https://eduscol.education.fr/document/17257/download}
\defl{Calcul littéral}{https://eduscol.education.fr/document/17263/download}

\defl{Grandeurs et mesures}{https://eduscol.education.fr/document/17293/download}
\defl{Algorithmique et programmation}{https://eduscol.education.fr/document/17311/download}

\defl{Suites}{https://eduscol.education.fr/document/24586/download}
\defl{Produit Scalaire}{https://eduscol.education.fr/document/24589/download}
\defl{Raisonnement et démonstration (seconde)}{https://eduscol.education.fr/document/24580/download}
\defl{Raisonnement et démonstrations (première)}{https://eduscol.education.fr/document/24583/download}

\def\jules{\href{https://juels.dev/}{Jules PESIN}}
\def\yuyu{\href{https://www.instagram.com/yuyuvrajav/}{@yuyuvraj}}

\defl{Utiliser les notions de géométrie planepour démontrer}{https://eduscol.education.fr/document/17305/download}

% Manuels
\def\dim{\href{https://www.editions-hatier.fr/livre/dimensions-mathematiques-6e-ed-2016-manuel-de-leleve-9782401020023}
    {Dimensions 6e (Ed. 2016)}
}

\definecolor{myriade}{HTML}{0F83B3} %#0F83B3
\def\my{\href{https://www.editions-bordas.fr/ouvrage/myriade-mathematiques-6e-manuel-de-leleve-ed-2021-9782047337752.html}
    {Myriade 6e (Ed. 2021)}
}

\def\mm{\href{https://www.editions-hatier.fr/livre/maths-monde-cycle-4-livre-1-volume-9782278083459}
    {Maths Monde cycle 4 (Ed. 2016)}
}

\def\mi{\href{https://www.enseignants.hachette-education.com/livres/mission-indigo-mathematiques-cycle-4-5e-4e-3e-livre-eleve-ed-2017-9782013953962}
    {Mission Indigo mathématiques cycle 4 éd. 2017}
}

% https://www.armitiere.com/livre/1833174-des-maths-ensemble-et-pour-chacun-5e-mise-en--jean-philippe-rouques-helene-stainer-canope-crdp-44
\def\dmeepcC{\href{https://publimath.univ-irem.fr/PCO10003}
    {Des maths ensemble et pour chacun 5e}
}

\def\dmeepcS{\href{https://www.reseau-canope.fr/notice/des-maths-ensemble-et-pour-chacun-6e.html}
    {Des maths ensemble et pour chacun 6e}
}

\NewDocumentCommand{\dmeepc}{m O{}}{%
    \href{https://www.reseau-canope.fr/notice/des-maths-ensemble-et-pour-chacun-6e.html}
    {Des maths ensemble et pour chacun #1e \ifNotNull{#2}{(p.#2)}}
}

\NewDocumentCommand{\sesa}{m m O{} O{}}{%
    \href{https://manuel.sesamath.net/numerique/index.php?ouvrage=cm#1_#2&page_gauche=#4}{%
    Sésamath #1e #2 \ifNotNull{#4}{(#3 p.#4)}
    }
}

\NewDocumentCommand{\iP}{m m O{} O{}}{%
    \href{https://www.iparcours.fr/ouvrages/ouvrages.php?ouvrage=Cahier#1#2}{%
    iParcours #1e #2 \ifNotNull{#4}{(#3 p.#4)}
    }
}

\NewDocumentCommand{\ching}{m m O{}}{%
    \href{https://chingmath.fr/#1eme/#2}{%
    Ching@Math #1e (\reverseKebabCase{#2}\ifNotNull{#3}{ E.#3})
    }
}

\NewDocumentCommand{\wiki}{m O{}}{%
    \def\ext{}%
    \ifNotNull{#2}{\def\ext{\##2}}%
    \href{https://fr.wikipedia.org/wiki/#1\ext}%
    {Wikipédia (\reverseSnakeCase{#1}%
    \ifNotNull{#2}{ {\scriptsize $\rightarrow$ \reverseSnakeCase{#2}}}%
    )}
}

\newcommand*{\prbltq}[1]{\href{https://www.problematheque-csen.fr/fiche-probleme/#1}{Problémathèque (\reverseKebabCase{#1})}}

\NewDocumentCommand{\rpmc}{O{}}{%
    \href{https://eduscol.education.fr/document/13132/download?attachment\#page=#1}{%
    La résolution de problèmes mathématiques au collège
    (p.%
    #1%
    % \directlua{tex.print((tonumber("#1") or 0) + 3)}
    )}%
}

% Attendus de fin d'année
\NewDocumentCommand{\afa}{m O{}}{
    \ifthenelse{\equal{#1}{CM2}}{
        \def\afalink{https://eduscol.education.fr/document/14002/download}
    }{
    \ifthenelse{\equal{#1}{6e}}{
        \def\afalink{https://eduscol.education.fr/document/14014/download}
    }{
    \ifthenelse{\equal{#1}{5e}}{
        \def\afalink{https://eduscol.education.fr/document/14044/download}
    }{
    \ifthenelse{\equal{#1}{4e}}{
        \def\afalink{https://eduscol.education.fr/document/14056/download}
    }{
    \ifthenelse{\equal{#1}{3e}}{
        \def\afalink{https://eduscol.education.fr/document/14068/download}
    }{
        \def\afalink{https://eduscol.education.fr/document/14014/download}
    }}}}}
    \def\page{}
    \ifNotNull{#2}{\def\page{(p.#2)}}
    \href{\afalink\#page=#2}{Attendus de fin d'année de #1 \page}
}

\def\ca{%
    \href{https://pedagogie.ac-strasbourg.fr/mathematiques/competitions/course-aux-nombres/}%
    {Course aux nombres}%
}

% Euclide https://www.pedagogie.ac-aix-marseille.fr/jcms/c_10743971/it/les-elements-d-euclide-traduction-par-oliver-byrne

\NewDocumentCommand{\eucl}{O{1804} O{}}{
    \ifthenelse{\equal{#1}{1632}}{ % 1632
        \def\trad{D. Henrion}
        \def\afalink{https://www.pedagogie.ac-aix-marseille.fr/upload/docs/application/pdf/2019-11/elements_euclide_-_denis_henrion.pdf}
    }{
    \ifthenelse{\equal{#1}{1804}}{ % 1804 traduction F. Peyrard
        \def\trad{F. Peyrard}
        \def\afalink{https://eduscol.education.fr/document/14014/download}
    }{}
    }
    \def\page{}
    \ifNotNull{#2}{\def\page{p.#2}}
    \href{\afalink\#page=#2}{Les Éléments d'Euclide (traduction de \trad \page)}
}
% 1632 https://www.pedagogie.ac-aix-marseille.fr/upload/docs/application/pdf/2019-11/elements_euclide_-_denis_henrion.pdf
% Logiciels
\newcommand{\defIconLink}[4]{% 1 text , 2 : color , 3 : icon , 4 : link
    \expandafter\def\csname #1\endcsname{%
        {\def\iconPath{}%
        \icon{#3} \textbf{\href{#4}{\color{#2}#1}}}
    }%
}

\newcommand{\cmdIconLink}[4]{% 1 text , 2 : color , 3 : icon , 4 : link
    \expandafter\NewDocumentCommand\csname cmd#1\endcsname{O{}}
    {%
        {\def\iconPath{}%
        \icon{#3} \textbf{\href{#4/##1}{\color{#2}#1}}}
    }%
}

\definecolor{capytale}{HTML}{1E293B} % #1E293B
\definecolor{capytale-2}{HTML}{F0F1F2} % #F0F1F2

\def\Capytale{%
    \href{https://capytale2.ac-paris.fr/~/my}{\shl{capytale}{capytale-2}{CAPYTALE}}%
}

\newcommand{\capytale}[1]{%
    \href{https://capytale2.ac-paris.fr/web/c/#1}{\shl{capytale}{capytale-2}{CAPYTALE \shl{capytale-2}{capytale}{#1}}}%\shl{capytale}{capytale-2}{CAPYTALE 
}

\definecolor{enc}{HTML}{1D3D6E} % #1D3D6E
\defIconLink{ENC}{enc}{ENC-Hauts-de-Seine}{https://enc.hauts-de-seine.fr/}

\definecolor{pronote}{HTML}{1A6E45} % #1A6E45
\defIconLink{Pronote}{pronote}{pronote}{https://0922247t.index-education.net/pronote/}

\definecolor{calc}{HTML}{00A500} % #00A500
\defIconLink{Calc}{calc}{libreOffice/calc/logo}{https://fr.libreoffice.org/discover/calc/}

\definecolor{geogebra}{HTML}{9693F7} % #9693F7
\defIconLink{Geogebra}{geogebra}{geogebra/logo}{https://www.geogebra.org/classic}
\cmdIconLink{Geogebra}{geogebra}{geogebra/logo}{https://www.geogebra.org/m}

\definecolor{scratch}{HTML}{FFAB19} % #FFAB19
\defIconLink{Scratch}{Orange}{scratch/logo}{https://scratch.mit.edu/projects/editor/}

% http://trucsmaths.free.fr/etymologie.htm

\newcommand{\dym}[1]{\def\ym{\href{#1}{Yvan Monka}}}

\captionsetup{labelformat=empty,labelsep=none}

% ANNE
\setboolean{boxedProperties}{true} % false = edge
\setboolean{parenthisedID}{false}
\setboolean{showID}{false}

\def\DefinitionColor{Red}
\def\PropertyColor{Red}
\def\TheoremColor{Red}

% TIKZ
\def\crossWidth{0.25mm}
\def\pointColor{blue}

% EB Garamond
% \usepackage[cmintegrals,cmbraces]{newtxmath} 
% \usepackage{ebgaramond-maths}

% Linux Libertine
% \usepackage{libertine}
% \usepackage{libertinust1math}

% Source Serif Pro
% \usepackage[default,regular,black]{sourceserifpro}

% Source Sans Pro
% \usepackage[default]{sourcesanspro}

% TeX Gyre Pagella:
% \usepackage{tgpagella,eulervm}


%%% LuaLaTex

% Libertine
\setmainfont{Libertinus Serif}
\setmathfont{Libertinus Math}

% OpenDyslexic
% \defaultfontfeatures{ Ligatures=TeX,Scale=MatchUppercase }
% \setmainfont{OpenDyslexic}[Scale=1.0]
% \setmathfont{Fira Math} % Or maybe try KPMath-Sans?
% \setmathfont{OpenDyslexic Italic}[range=it/{Latin,latin}]
% \setmathfont{OpenDyslexic}[range=up/{Latin,latin,num}]

% Gyre
% \setromanfont{TeX Gyre Termes}
% \setsansfont{TeX Gyre Heros}
% \setmonofont{TeX Gyre Cursor}[Ligatures=NoCommon]
% \setmathfont{TeX Gyre Termes Math}

% STIX
% \setmainfont{STIX Two Text}
% \setmathfont{STIX Two Math}

% Fira
% \setmainfont{Fira Sans}
% \setmathfont{Fira Math}

\begin{document}

\hfuzz=30pt

\ifBeamer{%
    \renewcommand*{\theenumii}{\alph{enumii}}

    \firstSlide
    \setboolean{showRef}{false}
}

\ifArticle{%
    \renewcommand*{\theenumii}{\alph{enumii}}
    
    \disableAnimation
}



% DOCUMENTS

% % VARIABLES %%%
\setSeq{4}{Nombres - Entiers}
\setGrade{6e}

\def\imgPath{enseignement/6e/nombres/entiers/}

\def\ym{\href{https://www.maths-et-tiques.fr/telech/19Nombres1.pdf}{Yvan Monka}}

% https://www.maths-et-tiques.fr/telech/19Nombres2.pdf

\forPrint
%%

\obj{
    \item Decomposer un nombre dans plusieurs bases.
    \item Conversion de durée.
    \item Utiliser et représenter les grands nombres entiers (en chiffres et en lettres).
    \item Utiliser la division euclidienne.
    \item Organiser un calcul en une seule ligne, utilisant si nécessaire des parenthèses.
    \item Savoir ce qu'est un ordre de grandeur et savoir l'utiliser pour prévoir un résultat.
    \item Résoudre des problèmes relevant des structures additives et multiplicatives en mobilisant une ou plusieurs étapes de raisonnement.
}

\scn{Découvrir une numération préhistorique}

\qfSlide{
    \begin{enumerate}
        \item $2 + 3 \time 5 = $
        \item $6 - 2 + 3 - 1 =$
        \item $4 \times (6 + 3) =$
    \end{enumerate}
}

\bsec{Compter}
\bsubsec{Base 12}

\slide{exo}{\bvspace{-0.75cm}
    \act{Numération préhistorique}{
        Certains hommes préhistorique utilisaient leurs main pour communiquer sur des nombres.
        \imgp{historic/numbers}[6.5cm]
    }[\dmeepc{6}[108]]
}

% \endinput

\slide{exo}{
    \begin{enumerate}\setItemColor{act}
        \item \multiColEnumerate{3}{
            \item $\Prehistoric{6} = \awsr{6}$
            \item $\Prehistoric{10} = \awsr{10}$
            \item $\icon{prehistoric-numbers/u/0}[60pt] = 7$
        }
        \item Quel est le nombre le plus grand pouvant etre communiquer de cette manière ?
        \item Comment pourait-on communiquer des nombres plus grands ?
        \saveenumi
    \end{enumerate}
}

\slide{exo}{
    \begin{enumerate}\loadenumi[act]
        \item \multiColEnumerate{2}{
            \item $\Prehistoric{12} = \awsr{12}$
            \item $\Prehistoric{15} = \awsr{15}$
            \item $\Prehistoric{50} = \awsr{50}$
            \item $\Prehistoric{0} = 58$
            \item $\Prehistoric{0} = 100$
            \item $\Prehistoric{135} = \awsr{135}$
        }
        \saveenumi
    \end{enumerate}
}

\slide{exo}{
    \begin{enumerate}\loadenumi[act]
        \item Quel est le plus grand nombre pouvant etre communiqué avec cette méthode ?
        \item Certains hommes préhistoriques comptaient donc par 12 car ils avaient 12 phalanges?
        De la même manière, par combien comptons-nous ? Pourquoi selon vous ?
    \end{enumerate}
}

\scn{Utiliser une numération en base 12}

\slide{exo}{\cdp{Table de 12}{\Table{12}{12}}}

\slide{cr}{
    \sseq\ssec\ssubsec \bvspace{-0.5cm}
    \vc{}{
        Le \key{système duodécimal},
        ou de \key{base $12$}.
        Est une méthode de \key{comptage par douzaines}.
    }[\wiki{Système_duodécimal}]
}

\slide{cr}{
    \expl{}{
        \multiColEnumerate{2}{
            \item $15 =$ \baseDecomposition{15}{12}["hide"]
            \item $56 =$ \baseDecomposition{56}{12}["hide"] 
            \item $\awsr{27} =$ \baseDecomposition{27}{12}
            \item $\awsr{112} =$ \baseDecomposition{112}{12} \saveenumi
        }\multiColEnumerate{1}{\loadenumi
            \item $145 =$ \awsr{\baseDecomposition{145}{12}}
        }
    }
}

\def\scale{1.15}
\slide{exo}{\small
    \exo{Justifiez, en détaillant vos calculs, le nombre de cubes présents dans chacune des figures.}{\calculator
        \multiColEnumerate{2}{
            \item \RepresenterEntier[Base=12,Echelle=\scale]{28}
            \item \RepresenterEntier[Base=12,Echelle=\scale]{100} \saveenumi
        }
    }
}

\slide{exo}{
    \multiColEnumerate{1}{ \loadenumi[exo]
        \item \RepresenterEntier[Base=12,Echelle=\scale]{332}
        \item \RepresenterEntier[Base=12,Echelle=\scale]{1942}
    }
}


\scn{Découvrir la numération Babylonienne}

\qfSlide{
    \multiColEnumerate{2}{
        \item \awsr{6425} = \baseDecomposition{6425}{10}
        \item \awsr{1237} = \baseDecomposition{1237}{12}
        \item 9233 = \baseDecomposition{9233}{10}["hide"]
        \item 232 = \baseDecomposition{232}{12}["hide"]
    }
}

\bsubsec{Base 60}

\def\aspc{\ifbool{answer}{}{\vspace{1cm}}}

\slide{exo}{\bshrink
    \act{Numération Babylonienne}{
        \begin{enumerate}\bvspace{-1cm}
            \item \multiColEnumerate{3}{
                \item \Babylone{24} = 24
                \item \Babylone{6} = 6
                \item \Babylone{50} = 50
            }
            Que représentent le clou \Babylone{1} et le chevron \Babylone{10} ?
            \item \multiColEnumerate{2}{
                \item \Babylone{12} = \awsr{12}
                \item \Babylone{34} = \awsr{34}
                \item \aspc \awsr{\Babylone{23}} = 23
            } \saveenumi
        \end{enumerate}
    }[\wiki{Numération_mésopotamienne}[Numération_sexagésimale_de_position]]
}

\slide{exo}{
    \begin{enumerate} \loadenumi[act]
        \item \multiColEnumerate{2}{
            \item \Babylone{70} = 70
            \item \Babylone{61} = 61
            \item \Babylone{190} = 190
            \item \Babylone{1380} = 1380
            \item \Babylone{3865} = 3865
        }
        Comment ces nombres sont-ils composés ?
        \saveenumi
    \end{enumerate}
}

\slide{exo}{
    \begin{enumerate} \loadenumi[act]
        \item \multiColEnumerate{2}{
            \item \Babylone{86} = \awsr{86}
            \item \aspc \awsr{\Babylone{132}} = 132
            \item \Babylone{325} = \awsr{325}
            \item \Babylone{7271} = \awsr{7271}
            \item \aspc \awsr{\Babylone{10872}} = 10872
        } \saveenumi
    \end{enumerate}
}

\slide{exo}{
    \begin{enumerate} \loadenumi[act]
        \item Les chiffres babyloniens changent de signification selon leur position : on parle donc de \key{numération de position}.
        Notre système de numération actuel est-il aussi un système de numération de position ?
        \item Existe-t-il des éléments que nous comptons encore aujourd'hui de manière similaire aux Babyloniens ?
    \end{enumerate}
}

\scn{Utiliser une numération en base 60}

\qfSlide{
    \exo{Donner l'heure}{
        \multiColEnumerate{3}{
            \item \Horloge[Secondes=false]{7:30}
            \item \Horloge[Secondes=false]{15:24}
            \item \Horloge[Secondes=false]{12:46}
        }
    }
}

\slide{cr}{
    \ssubsec
    \vc{}{
        Le \key{système sexagésimal},
        ou de \key{base $60$}.
        Est une méthode de \key{comptage par soixantaines}.
    }[\wiki{Système_sexagésimal}]
}

\slide{cr}{
    \expl{}{
        \multiColEnumerate{2}{
            \item \awsr{186} = \baseDecomposition{186}{60}
            \item \awsr{1200} = \baseDecomposition{1200}{60}
            \item 720 = \baseDecomposition{75}{60}["hide"]
            \item 1842 = \baseDecomposition{500}{60}["hide"] \saveenumi
        }\multiColEnumerate{1}{\loadenumi
            \item 8350 = \awsr{\baseDecomposition{8350}{60}}
        }
    }
}

\slide{cr}{
    \rmk{}{
        L'usage moderne du sexagésimal est assez proche de celui de la mesure du temps.
        \multiColItemize{2}{
            \item $\Horaire{1} = \awsr{60}\textrm{min}$
            \item $\Horaire{;1} = \awsr{60}\textrm{s}$
        }
    }

    \expl{}{
        \multiColItemize{1}{
            \item $\Horaire{1} = \awsr{3600}\sec$
            \item $\Horaire{;;602} = \parseSeconds{602}["hide"]$
            \item $\Horaire{;;7623} = \parseSeconds{7623}["hide"]$
        }
    }
}

\bookSlide{36p149,38p149,39p149}[7cm][2]

\def\scale{2}

\scn{Utiliser une numération en base 10}

\slide{qf}{\bsmall
    \nullsubsec{}{
        \begin{itemize}
            \item La Lune est à \Ecriture{384 000} de kilomètre de la Terre.
            \item Jupiter est à \Ecriture{91 000 000} de kilomètre de la Terre.
            \item Pluton est à \Ecriture{4 297 000 000} de kilomètre de la Terre.
        \end{itemize}

        Complète le tableau ci-dessous avec ces nombres écrits en chiffres.

        \begin{center}
            \Tableau[%
            DoubleEntree,
            Couleur=gradeColor!15,
            LegendesH={Lune,Jupiter,Pluton},
            LegendesV={Distance à la Terre (km)},
            Largeur=135pt
            ]{\awsr{384 000},\awsr{91 000 000},\awsr{4 297 000 000}}
        \end{center}
    }[\dmeepcS]
}

\bsec{Numération décimales}

\bsubsec{Nombres entiers}

\slide{cr}{
    \ssec\ssubsec
    \vc{}{
        Le \key{système décimal},
        ou de \key{base $10$}.
        Est une méthode de \key{comptage par dizaines}.
    }[\wiki{Système_décimal}]

    \hist{}{
        L'utilisation actuelle des \key{chiffres arabes} repose sur un système de numération \key{décimal et positionnel}.
        Leur diffusion au Moyen-Orient et en Europe serait due à un ouvrage du mathématicien persan d'\key{Al-Khwârizmî} (780-850 ap. J.-C.).
    }[\wiki{Al-Khwârizmî} \wiki{Système_de_numération_indo-arabe}]
}

\slide{cr}{
    \mthd{}{
        Pour écrire un nombre on utilise \key{10 symboles} appelé \key{chiffres}.
        La \key{position} des chiffres dans l'écriture d'un nombre détermine sa valeure.
    }

    \expl{}{
        \begin{enumerate}
            \item $\np{110} =$ \baseDecomposition{110}{10}["hide"] 
            \item $\np{5841} =$ \baseDecomposition{5841}{10}["hide"]
            \item $\awsr{{\np{1010}}} =$ \baseDecomposition{1010}{10}
        \end{enumerate}
    }
}

\slide{cr}{
    \rmk{}{
        On regroupe les chiffres de nombres par groupes de trois afin d'en améliorer la lisibilité.
    }
}

\slide{cr}{\bshrink
    \expl{}{
        Classer les chiffres des nombres suivants,
        puis les réécrire correctement :
        \multiColEnumerate{3}{
            \item $54 454$
            \item $36 119 312$
            \item $3 300 001 200$
        }
        \bvspace{-0.5cm}
        \decimalTable{{54454,36119312,3300001200}}["hide"]
        \multiColEnumerate{3}{
            \item $\awsr{\np{54454}}$
            \item $\awsr{\np{36 119 312}}$
            \item $\awsr{\np{3 300 001 200}}$
        }
    }
}

\scn{Manipuler des nombres entiers dans le système décimal}

\slide{qf}{
    Donner le chiffres des :
    \multiColEnumerate{2}{
        \item milliers de $\np{56165453}$
        \item milliards de $\np{546160006546521}$
        \item dizaines de milliers de $\np{346805235}$
        \item centaines de milliards de $\np{340045235}$
        \item centaines de $\np{65465,654654}$
        \item dizaines de millions de $\np{211010100,001}$
    }
}

\slide{exo}{
    \exo{Nombres mystères}{
        \begin{enumerate}
            \item Donne un exemple de nombre inférieur à $400$ pour lequel :
            \begin{itemize}
                \item le chiffre des dizaines est la moitié du chiffre des centaines.
                \item la somme des chiffres est $11$.
            \end{itemize}\saveenumi
        \end{enumerate}
    }[\dmeepcS]
}

\slide{exo}{
    \begin{enumerate}\loadenumi[exo]
        \item Donne un exemple de nombre à quatre chiffres tel que :
        \begin{itemize}
            \item le chiffre des dizaines est la moitié du chiffre des centaines.
            \item la somme des chiffres est $11$.
        \end{itemize}
        \item Donne un exemple de nombres à trois chiffres pour lequel :
        \begin{itemize}
            \item il est inférieur à $\np{2000}$ ;
            \item il a trois chiffres identiques ;
            \item la somme de ses chiffres est 10.
        \end{itemize}
    \end{enumerate}
}

\scn{Formaliser la notion de la division euclidienne}

\slide{qf}{Compléter les divisions suivantes
    \multiColEnumerate{2}{
        \item \longDivision{155}{5}
        \item \longDivision{700}{49}
    }
}

\bsubsec{Division euclidienne}

\slide{exo}{\bshrink
    \act{}{%
    Le roi de Divisia possède $27$ pièces d'or et souhaite les partager équitablement entre $4$ chevaliers.
    Les pièces restantes seront données à son écuyer.
    
    \begin{enumerate}
        \item Combien de pièces chaque chevalier recevra-t-il ?
        \item Combien de pièces resteront pour l'écuyer ?
        \item Supposons maintenant que le roi possède $100$ pièces et qu'il les partage entre $6$ chevaliers :
        \begin{enumerate}
            \item Combien de pièces chaque chevalier recevra-t-il ?
            \item Combien de pièces resteront pour l'écuyer ?
        \end{enumerate}\saveenumi
    \end{enumerate}
    }
}

\slide{exo}{
    \begin{enumerate}\loadenumi[act]
        \item Toujours avec $100$ pièces,
        existe-t-il un nombre de chevaliers tel que; l'écuyer reçoive:
        \multiColEnumerate{1}{
            \item $0$ pièce ?
            \item plus de pièces qu'un chevalier ?
            \item autant de pièces qu'un chevalier ?
            \item autant de pièces qu'il y a de chevalier ?
        }
    \end{enumerate}
}

\slide{cr}{\bsmall
    \ssubsec

    \df{}{
        Pour deux entiers $a$ et $b$,
        on appelle \key{division euclidienne}
        du \textcolor{Green}{\key{dividende} $a$} par le \textcolor{Red}{\key{diviseur} $b$}
        l'expression :
        \begin{align*}
            \textcolor{Green}{a}
            = \textcolor{Red}{b}
            \times \textcolor{Blue}{q}
            + \textcolor{Violet}{r}
        \end{align*}
        Où le \textcolor{Blue}{\key{quotient} $q$} et \textcolor{Violet}{\key{reste} $r$} sont deux entiers avec
        $\textcolor{Violet}{r} < \textcolor{Red}{b}$.
    }

    \expl{}{
        \multiColEnumerate{3}{
            \item $100 = 6 \times \awsr{16} + \awsr{4}$
            \item $246 = 3 \times \awsr{82} + \awsr{0}$
            \item $360 = 23 \times \awsr{15} + \awsr{15}$
        }
    }
}

\scn{Notion de divisibilité}

\slide{qf}{
    \exo{}{
        Quel est le nombre de:
        \multiColEnumerate{2}{
            \item dizaines dans $750$.
            \item milliers dans $\np{665454}$.
            \item millions dans $\np{9876502300}$.
            \item dizaines de milliers dans $\np{121321}$.
            \item centaines de millions dans $\np{2313251}$.
            \item centaines dans $\np{352154.16}$.
        }
    }
}

\slide{cr}{\bsmall
    \df{}{
        Pour $a$ et $b$ deux entiers,
        on dit que $b$ \key{divise} $a$ si le reste de la division euclidienne de $a$ par $b$ est $0$.
    }

    \pr{}{
        \Sialors{$b$ \key{divise} $a$}{$a$ est un \key{multiple} de $b$}
    }

    \begin{center}
        \expl{}{
            \Tableau[%
                DoubleEntree,
                Stretch=1.5,
                Couleur=gradeColor!15,
                LegendesH={\qquad$4$\qquad,\qquad$6$\qquad,\qquad$12$\qquad,\qquad$31$\qquad},
                LegendesV={Divise 62 ?, Multiple de 4 ?},
                Largeur=2cm
            ]{
                \awsr{Oui},\awsr{Non},\awsr{Oui},\awsr{Oui},
                \awsr{Oui},\awsr{Non},\awsr{Oui},\awsr{Non}
            }
        }
    \end{center}
}

\slide{exo}{
    \exo{des fleurs}{
        Une fleuriste dispose de 1815 fleurs.
        Doit-elle réaliser des bouquets de 16 fleurs ou de 17 fleurs pour en utiliser le plus possible ?
    }[\iP{6}{2021}[2][16]]
}

\slide{exo}{
    \exo{des crêpes}{
        Dans une fabrique, on emballe les crêpes par paquets de 12.
        En fin de journé, les employés mangent les crêpes qui restent après emballage.
        Voici le nombre de crêpes fabriquées cette semaine:
        \begin{center}
            \begin{tabular}{|>{\bfseries}c|*{5}{c|}} % Colonne en gras pour la première colonne
                \hline
                \rowcolor{gray!15} 
                Jour & lundi & mardi & mercredi & jeudi & vendredi \\ \hline
                Nombre de crêpes  & 384 & 1004 & 519 & 1069 & 1102 \\ \hline
            \end{tabular}
        \end{center}
        \begin{enumerate}
            \item Le lundi, combien de douzaine de crêpes sont emballées et combien de crêpes sont mangées ?
            \item Sur l'ensemble de la semaine, chaque employé à mangé le même nombre de crêpes.
            Combien peut-il y avoir d'employés dans cette fabrique?
        \end{enumerate}
    }[\dmeepc{6}[166]]
}

\bookSlide{21p45,23p45,25p45}[7cm][2]

\scn{Priorités opératoires}

\slide{qf}{\bvspace{-0.5cm}
    \exo{}{
        \multiColEnumerate{1}{
            \item $\np{44420} \times 100 = $
            \item $981 \times \np{100000} = $
            \item $\np{685540000} \div \np{10} = $
            \item $\np{230020000000} \div \np{10000} = $
            \item $\np{10001} \times \np{20000} = $
            \item $\np{3090300} \div \np{300} = $
        }
    }
}

\bsec{Opérations}
\bsubsec{Règles opératoires}

\slide{cr}{
    \ssec\ssubsec

    \rl{}{
        Les calculs se font dans l'ordre des priorités suivant:%
        \begin{enumerate}
            \item La multiplication et la division
            \item L'addition et la soustraction
        \end{enumerate}
    }
}

\slide{cr}{
    \rl{}{
        En cas d'opérations de mêmes priorités, on effectue les opérations de gauche à droite.
    }

    \expl{}{
        \multiColEnumerate{1}{
            \item $3 - 2 + 3 = \awsr{1 + 3 = 4}$ 
            \item $19 - 6\times 3 = \awsr{19 - 18 =  1}$
            \item $3 + \np{3.2} \times 2 - 4 = \awsr{3 + \np{6.4} -4 = \np{9.4} - 4 = \np{5.4}}$
        }
    }
}

\slide{cr}{
    \rl{}{
        On commence par effectuer les calculs entre parenthèses.
    }

    \expl{}{
        \multiColEnumerate{1}{
            \item $(1 + 2) \times 21 = \awsr{3 \times 21 = 63}$
            \item $(11 \times 3) + (15 \div 2) = \awsr{33 + 7.5 = 40.5}$
            \item $(34 - 2 + 8 ) \div 4 = \awsr{(32 + 8) \div 4 = 40 \div 4 = 10}$
            \item $((13 - (3 - 2)) + 2) = \awsr{(13 - 1) + 2 = 12 + 2 = 14}$
        }
    }
}

\scn{Formaliser le vocabulaire opératoire}

\slide{qf}{
    \multiColEnumerate{1}{
        \item $7 + 5 - 4 \times 2 = \awsr{7 + 5 - 8 = 12 - 8 = 4}$
        \item $7 + (5 - 4) \times 2 = \awsr{7 + 1 \times 2 = 7 + 2 = 9}$
        \item $(7 + 5 - 4) \times 2 = \awsr{(13 - 4) \times 2 = 9 \times 2 = 18}$
    }
}

\bsubsec{Vocabulaire opératoires}

\slide{cr}{\bsmall
    \ssubsec
    \bvspace{-0.5cm}
    \vc{}{
        On connait quatres types d'opérations :
        \begin{itemize}
            \item L'\key{addition} permet de calculer la \key{somme} de deux \key{termes}.
            \item La \key{soustraction}  permet de calculer la \key{différence} entre deux \key{termes}.
            \item La \key{multiplication} permet de calculer le \key{produit} de deux \key{facteurs}.
            \item La \key{division} permet de calculer le \key{quotient} de deux \key{nombres}.
        \end{itemize}
    }
}

\slide{cr}{\bshrink
    \vc{}{Dans un calcul,
    le type de la dernière opération effectuée détermine le nom donné au calcul dans son ensemble.}
    \bvspace{-0.5cm}
    \expl{Nommer les calculs suivants}{
        \bvspace{-0.5cm}
        \multiColEnumerate{1}{
            \item $1,6 + 4$ est \awsr{la somme} de \awsr{$1,6$ et $4$}.
            \item $(\frac{2}{6} + 3) \times 9$ est \awsr{le produit } de \awsr{$\frac{2}{6} + 3$ et $9$}.
            \item $6,6 + 1 \times 8$ est \awsr{la somme de $6,6$ par $1 \times 8$}.
            \item $\frac{2}{6} + 3 - 9$ est \awsr{la différence entre $\frac{2}{6}$ et $9$}.
            \item $\pi \div (3 - 9)$ est \awsr{le quotient de $\pi$ par $(3 - 9)$}.
        }
    }
}

\scn{Résoudre des problèmes relevant des structures additives et multiplicatives en mobilisant une ou plusieurs étapes de raisonnement}

\slide{qf}{
    Quel est le périmètre d'un quadrilatère dont les côtés ont pour mesures :
    \multiColItemize{4}{
        \item \Lg{2} \item \Lg{3.9} \item \Lg[mm]{81} \item \Lg{10}
    }
    \awsr[0]{
        \multiColItemize{1}{
            \item $\Lg[mm]{81} = \Lg{8.1}$ \item $\Lg{2} + \Lg{3.9} + \Lg{8.1} + \Lg{10} = \Lg{24}$
        }
    }
}

\slide{qf}{\bvspace{-0.5cm}
    \exo{}{
        \multiColEnumerate{1}{
            \item $10 \times 3 + 4 \div 2 = \awsr{30 + 2 = 32}$
            \item $5 \times 4 \div 2 = \awsr{20 \div 2 = 10}$
            \item $4 \times (1 + 16 - 10) = \awsr{4 \times (17 - 10) = 4 \times 7 = 28}$
            \item $100 \div (10 \times (2+3)) = \awsr{100 \div (10 \times 5) = 100 \div 50 = 2}$
        }
    }
}

\slide{exo}{\bshrink
    \exo{Les bons comptes}{
        Pour résoudre les problèmes suivants :  
        \begin{itemize}
            \item Présenter chaque calcul séparément, en précisant ce que représente la valeur obtenue à chaque étape.  
            \item Présenter tous les calculs en une seule expression permettant d'obtenir le résultat final.  
        \end{itemize}
        
        \begin{enumerate}
        \item Xavier possède \Prix{28} dans sa tirelire.  
                Son grand-père lui donne \Prix{75}. Il a désormais \Prix{53} de plus que sa sœur Christine.  
                Quelle somme d'argent possède Christine ?\saveenumi
        \end{enumerate}
    }[\dmeepc{6}[161]]
}

\slide{exo}{
    \begin{enumerate}\loadenumi[exo]
        \item Un commerçant achète sept rouleaux de \Lg[m]{50} de tissu.  
        Chaque rouleau coûte \Prix{392}.  
        Il revend le tissu au prix de \Prix{12} par mètre.  
        Quel bénéfice réalise-t-il après avoir revendu la totalité du tissu ?
        \item Julien et Georges possèdent à eux deux un total de \Prix{47}.  
        Julien dépense \Prix{12} et Georges dépense \Prix{7}.  
        Après ces dépenses, ils ont chacun la même somme.  
        Quelle somme Julien possédait-il avant sa dépense ?
    \end{enumerate}
    \awsr[0]{
        \begin{enumerate}
            \item 
            \begin{itemize} 
                \item \begin{itemize}
                    \item $28 + 75 = 103$ : Xavier possède maintenant \Prix{103}.
                    \item $103 - 53 = 50$ : Christine possède \Prix{50}.
                \end{itemize}
                \item $28 + 75 - 53 = 50$
            \end{itemize}
            \item Le bénéfice est la différence entre le prix de vente total et le prix d'achat total.
            \begin{itemize}
                \item \begin{itemize}
                    \item $392 \times 7 = 2744$ : prix d'achat total des 7 rouleaux, soit \Prix{2744}.
                    \item $7 \times 50 = 350$ : le commerçant dispose de \Lg[m]{350} de tissu.
                    \item $350 \times 12 = 4200$ : prix de vente total du tissu, soit \Prix{4200}.
                    \item $4200 - 2744 = 1456$ : bénéfice réalisé après la revente de la totalité du tissu, soit \Prix{1456}.
                \end{itemize}
                \item $7 \times 50 \times 12 - 392 \times 7 = 1456$
            \end{itemize}
            \item 
            \begin{itemize}
                \item \begin{itemize}
                    \item $47 - 12 = 35$ : après la dépense de Julien, il reste \Prix{35}.
                    \item $35 - 7 = 28$ : après la dépense de Georges, il reste \Prix{28}.
                    \item $28 \div 2 = 14$ : chacun possède maintenant \Prix{14}.
                    \item $14 + 12 = 26$ : Julien possédait donc \Prix{26} avant sa dépense.
                \end{itemize}
                \item $(47 - 12 - 7) \div 2 + 12 = 26$
            \end{itemize}
        \end{enumerate}
    }
}
% % VARIABLES %%%
\setSeq{3}{Symétries}
\setGrade{5e}

% \setboolean{answer}{false}
% \setboolean{newPageOnSlide}{true}

\def\imgPath{enseignement/5e/symetries/}

\def\ym{\href{https://www.maths-et-tiques.fr/telech/19Sym.pdf}{Yvan Monka}}
\def\jerome{\href{https://drive.google.com/drive/folders/1Itzq0ZPj1sHwSIv9SgbUlIIuHdG28A5I}{Jérôme Potel}}
%%

\obj{
    \item Transformer une figure par une symétrie centrale.
    \item Identifier des symétries dans des frises, des pavages, des rosaces.
    \item Comprendre l'effet des symétries (axiale et centrale) :
    conservation du parallélisme, des longueurs et des angles.
    \item Mobiliser les connaissances des figures,
    des configurations et des symétries pour déterminer des grandeurs géométriques.
    \item Mener des raisonnements en utilisant des propriétés des figures,
    des configurations et des symétries.
}

\def\grid{\draw[gray!40] (0,0) grid (14,8);}
\def\arrow{\draw [ultra thick] (1,4)--(4,4)--(4,3)--(6,5)--(4,7)--(4,6)--(1,6)--cycle;}

\NewDocumentCommand{\mushroom}{O{(0,0)}}{
    \draw[thick, shift={#1}]
    (0,2) -- (2,4) -- (4,4) -- (6,2) -- (6,0) -- (0,0) -- cycle;
    
    % Spot
    % \draw[ultra thick, shift={#1}] 
    %     (1.5,3.5) -- (2,2) -- (4,2) -- (4.5,3.5);  % Center spot
        
    % Mushroom Stem
    \draw[thick, shift={#1}] 
        (1,0) -- (2,-2) -- (4,-2) -- (5,0) -- cycle; % Stem (rectangle)

    % Eyes
    \draw[thick, shift={#1}] 
        (2.5,-1) -- (2.5,0);

    \draw[thick, shift={#1}] 
        (3.5,-1) -- (3.5,0);
}

\def\figInit{
    \grid
    \arrow
    \drawPoint{$O$}{7}{4}
}

\scn{Rappels sur la symétrie axiales}

\slide{qf}{\bvspace{-1cm}
    \exo{}{
        Est-ce que deux dinosaures identiques peuvent être superposés l'un à l'autre grâce à une symétrie axiale ?
        Si ce n'est pas possible, pourquoi ?
        \imgp{qf/dinos-axiales}[9cm]
    }[\yuyu]
}

\bsec{Les symétries}
\bsubsec{Symétrie Axiale}

\slide{exo}{
    \bvspace{-0.5cm}
    \act{}{
        Construire l'image de la figure par la symétrie d'axe $(d)$.
        \bvspace{-0.75cm}
        \ctikz{
            \draw[gray!40] (0,-3) grid (14,8);
            \arrow
            \draw[thick, gradeColor] (0,-3)--(11,8);
            \node[gradeColor, above left] at (2,-1) {$(d)$};
        }
    }
}

\slide{cr}{
    \sseq\ssec\ssubsec
    \df{}{
        Deux figures sont dites \key{symétriques par rapport à une droite} si elles se \key{superposent par pliage} le long de cette droite.
    }[\wiki{Symétrie_axiale}]
}

\slide{cr}{\bvspace{-0.6cm}
    \expl{}{\bvspace{-0.5cm}
        \imgp{expl-symetrie-axiale}[7.5cm]
    }
}

\slide{cr}{
    \pr{}{
        \Sialors{le point $M'$ est l'image du point $M$ par la symétrie d'axe $(d)$}{
            \begin{enumerate}
                \item la droite $(MM')$ est \bawsr{perpendiculaire} à la droite $(d)$
                \item le milieu du segment $[MM']$ est sur la droite $(d)$.
            \end{enumerate}
        }
        \rmk{}{%
            La droite $(d)$ est alors la \bawsr{\key{médiatrice} } du segment $[MM']$.
        }
    }
}

\slide{exo}{
    \bvspace{-0.6cm}
    \exo{}{
        Construire l'image de la figure par la symétrie d'axe $(d)$.
        \bvspace{-0.75cm}
        \ctikz{
            \draw[gray!40] (0,-4) rectangle (17,9);
            \mushroom[(10,-1)]
            \draw[thick, gradeColor] (7,-3)--(10,8);
            \node[gradeColor, above left] at (7,-3) {$(d)$};
        }
    }
}

\scn{Découvrir la symétrie centrale}

\slide{qf}{
    \exo{}{
        Combien existe-t-il d'axes de symétrie pour chacun de ces panneaux?
        
        \imgp{panneaux-de-signalisation}[8cm]
    }[\href{https://clairelommeblog.fr/wp-content/uploads/2020/03/panneaux_routiers.pdf}
    {Claire Lommé}]
}

\bsubsec{Symétrie Centrale}

\def\one{%
    On va construire l'image de la figure par la symétrie de centre $O$.

    \ctikz{\figInit}
}

\def\two{%
    On regarde le «\textit{chemin}» du point $A$ au point $O$.
    
    \ctikz{
        \figInit
        \drawPoint{$A$}{4}{3}
        \draw[dashed, thick, Red] (4,3)--(7,4); 
    }
}

\def\three{%
    On exécute le même «\textit{chemin}», cette fois en partant du point $O$. On trouve alors le point $A'$,
    image du point $A$ par la symétrie de centre $O$.

    \ctikz{
        \figInit
        \drawPoint{$A$}{4}{3}
        \draw[dashed, thick, Red] (7,4)--(10,5);
        \drawPoint{$A'$}{10}{5}
    }
}

\def\four{%
    Fait de même avec les autres points de la figure, puis les reliers, de façon à obtenir l'image de la flèche par la symétrie de centre $O$.

    \ctikz{
        \figInit
        \drawPoint{$A$}{4}{3}
        \drawPoint{$B$}{4}{4}
        \drawPoint{$C$}{1}{4}
        \drawPoint{$D$}{1}{6}
        \drawPoint{$E$}{4}{6}
        \drawPoint{$F$}{4}{7}
        \drawPoint{$G$}{6}{5}
        \drawPoint{$A'$}{10}{5}
    }
}

\ifArticle{%
    \slide{exo}{
        \act{}{\vspace{-0.5cm}
            \multiColEnumerate{2}{
                \item[] \one
                \item \two
                \item \three
                \item \four
            }
        }[\jerome]
    }
}


\ifBeamer{%
    \slide{exo}{\act{}{\one}}
    \slide{exo}{\two}
    \slide{exo}{\three}
    \slide{exo}{\four}
}

\slide{cr}{
    \ssubsec
    
    \df{}{
        Deux figures sont dites \key{symétriques par rapport à un point}
        si l'on peut obtenir l'une en effectuant un \key{demi-tour} de l'autre \key{autour de ce point}.
    }[\wiki{Symétrie_centrale}]
}

\slide{cr}{\bvspace{-0.6cm}
    \expl{}{\bvspace{-0.5cm}
        \imgp{expl-symetrie-centrale}[9cm]
    }
}

\slide{cr}{
    \pr{}{
        \Sialors{le point $M'$ est le symétrique du point $M$ par la symétrie de centre $O$}{
            le point $O$ est \bawsr{le milieu du segment } $[MM']$.
        }
    }
}

\scn{Manipuler la symétrie centrale}

\slide{qf}{\bvspace{-1cm}
    \exo{}{
        Est-ce que deux dinosaures identiques peuvent être superposés l'un à l'autre grâce à une symétrie centrale ?
        Si ce n'est pas possible, pourquoi ?
        \bvspace{-0.5cm}
        \imgp{qf/dinos-centrales}[9cm]
    }[\yuyu]
}

\def\konoha{%
    % \draw [thick] (9,3)-- (8,2)-- (7,3)-- (8,4)-- (10,3)-- (9,1)-- (7,1)-- (5,1)-- (6,3)-- (7,5)-- (9,5)-- (10,4)-- (11,5);
    % \draw [thick] (7,1)-- (6,3);
    \draw [thick] (3,5)-- (4,6)-- (5,5)-- (4,4)-- (2,5)-- (3,7)-- (5,7)-- (7,7)-- (6,5)-- (5,3)-- (3,3)-- (2,4)-- (1,3);
    \draw [thick] (5,7)-- (6,5);
    \drawPoint{$O$}{6}{4}[gradeColor]
}

\slide{exo}{\bvspace{-0.75cm}
    \exo{}{
        Construis l'image de la figure par la symétrie de centre $O$.
        \ctikz[0.6]{
            \draw[gray!40] (0,-1) grid (13,8);
            \konoha
        }
    }
}

% Define a command to place numbered nodes at specific coordinates
\def\hexNode#1#2{
    \draw [color=gradeColor!75, fill opacity=1] #1 node[anchor=center, scale=1] {$#2$};
}

% Define a command for drawing quadrilateral shapes based on four coordinates
\def\quadShape#1#2#3#4{
    \draw[color=black] #1--#2--#3--#4--cycle;
}

% Coordinates for nodes and quadrilateral shapes
\newcommand{\hexagones}{
    % Place nodes with loop (coordinates, label)
    \foreach \coord/\label in {
        (2.75,-1.3)/1, (5,0)/2, (2.75,1.3)/3, (9.5,2.6)/4,
        (7.25,1.3)/5, (7.25,3.9)/6, (7.25,-3.9)/7, (9.5,-2.6)/8,
        (7.25,-1.3)/9, (11.75,-1.3)/10, (14,0)/11, (11.75,1.3)/12,
        (18.5,2.6)/13, (16.25,1.3)/14, (16.25,3.9)/15, (16.25,-3.9)/16,
        (18.5,-2.6)/17, (16.25,-1.3)/18, (7.25,-9.09)/19, (9.5,-7.79)/20,
        (7.25,-6.5)/21, (14,-5.2)/22, (11.75,-6.5)/23, (11.75,-3.9)/24,
        (11.75,-11.69)/25, (14,-10.39)/26, (11.75,-9.09)/27, (16.25,-9.09)/28,
        (18.5,-7.79)/29, (16.25,-6.5)/30, (23,-5.2)/31, (20.75,-6.5)/32,
        (20.75,-3.9)/33, (20.75,-11.69)/34, (23,-10.39)/35, (20.75,-9.09)/36
    }{%
        \hexNode{\coord}{\label}
    }
    % Draw quadrilateral shapes
    \quadShape{(0,0)}{(3,0)}{(4.5,-2.6)}{(1.5,-2.6)} \quadShape{(4.5,2.6)}{(6,0)}{(4.5,-2.6)}{(3,0)}
    \quadShape{(0,0)}{(1.5,2.6)}{(4.5,2.6)}{(3,0)} \quadShape{(9,5.2)}{(10.5,2.6)}{(9,0)}{(7.5,2.6)}
    \quadShape{(4.5,2.6)}{(7.5,2.6)}{(9,0)}{(6,0)} \quadShape{(4.5,2.6)}{(6,5.2)}{(9,5.2)}{(7.5,2.6)}
    \quadShape{(4.5,-2.6)}{(7.5,-2.6)}{(9,-5.2)}{(6,-5.2)} \quadShape{(9,0)}{(10.5,-2.6)}{(9,-5.2)}{(7.5,-2.6)}
    \quadShape{(4.5,-2.6)}{(6,0)}{(9,0)}{(7.5,-2.6)} \quadShape{(9,0)}{(12,0)}{(13.5,-2.6)}{(10.5,-2.6)}
    \quadShape{(13.5,2.6)}{(15,0)}{(13.5,-2.6)}{(12,0)} \quadShape{(9,0)}{(10.5,2.6)}{(13.5,2.6)}{(12,0)}
    \quadShape{(18,5.2)}{(19.5,2.6)}{(18,0)}{(16.5,2.6)} \quadShape{(13.5,2.6)}{(16.5,2.6)}{(18,0)}{(15,0)}
    \quadShape{(13.5,2.6)}{(15,5.2)}{(18,5.2)}{(16.5,2.6)} \quadShape{(13.5,-2.6)}{(16.5,-2.6)}{(18,-5.2)}{(15,-5.2)}
    \quadShape{(18,0)}{(19.5,-2.6)}{(18,-5.2)}{(16.5,-2.6)} \quadShape{(13.5,-2.6)}{(15,0)}{(18,0)}{(16.5,-2.6)}
    \quadShape{(4.5,-7.79)}{(7.5,-7.79)}{(9,-10.39)}{(6,-10.39)} \quadShape{(9,-5.2)}{(10.5,-7.79)}{(9,-10.39)}{(7.5,-7.79)}
    \quadShape{(4.5,-7.79)}{(6,-5.2)}{(9,-5.2)}{(7.5,-7.79)} \quadShape{(13.5,-2.6)}{(15,-5.2)}{(13.5,-7.79)}{(12,-5.2)}
    \quadShape{(9,-5.2)}{(12,-5.2)}{(13.5,-7.79)}{(10.5,-7.79)} \quadShape{(9,-5.2)}{(10.5,-2.6)}{(13.5,-2.6)}{(12,-5.2)}
    \quadShape{(9,-10.39)}{(12,-10.39)}{(13.5,-12.99)}{(10.5,-12.99)} \quadShape{(13.5,-7.79)}{(15,-10.39)}{(13.5,-12.99)}{(12,-10.39)}
    \quadShape{(9,-10.39)}{(10.5,-7.79)}{(13.5,-7.79)}{(12,-10.39)} \quadShape{(13.5,-7.79)}{(16.5,-7.79)}{(18,-10.39)}{(15,-10.39)}
    \quadShape{(18,-5.2)}{(19.5,-7.79)}{(18,-10.39)}{(16.5,-7.79)} \quadShape{(13.5,-7.79)}{(15,-5.2)}{(18,-5.2)}{(16.5,-7.79)}
    \quadShape{(22.5,-2.6)}{(24,-5.2)}{(22.5,-7.79)}{(21,-5.2)} \quadShape{(18,-5.2)}{(21,-5.2)}{(22.5,-7.79)}{(19.5,-7.79)}
    \quadShape{(18,-5.2)}{(19.5,-2.6)}{(22.5,-2.6)}{(21,-5.2)} \quadShape{(18,-10.39)}{(21,-10.39)}{(22.5,-12.99)}{(19.5,-12.99)}
    \quadShape{(22.5,-7.79)}{(24,-10.39)}{(22.5,-12.99)}{(21,-10.39)} \quadShape{(18,-10.39)}{(19.5,-7.79)}{(22.5,-7.79)}{(21,-10.39)}    
}

\slide{exo}{\bvspace{-0.75cm}
    \exo{Pavage}{\def\crossWidth{0.5mm}\def\crossSize{0.5}
        Le pavage ci-dessous est réalisé à l'aide de 36 pièces identiques.
        \bvspace{-0.5cm}
        \ctikz[0.35]{
            \hexagones
            \drawPoint{A}{13.5}{-2.6}
            \drawPoint{B}{12}{-5.2}
            \drawPoint{C}{13.5}{-7.79}
            \drawPoint{D}{18}{-5.2}
            \drawPoint{E}{9}{-2.6}
        }
    }[\jerome{} et \href{https://www.iparcours.fr/ouvrages/ouvrages.php?ouvrage=Cahier52022}{iParcours}]
}

\slide{exo}{\bsmall
    Par la symetrie de centre $A$, quelle est l'image de la figure :
    \begin{enumerate}\setItemColor{Gray}
        \item Observe le pavage, puis complete le tableau
        \begin{center}
            \def\cW{1cm}\renewcommand{\arraystretch}{1.75}%
            \begin{tabular}{|C{5cm}|C{\cW}|C{\cW}|C{\cW}|C{\cW}|}
                \hline La pièce &
                16 & \bawsr{8}  & 36 & 8\\
                \hline est l'image de la pièce&
                \bawsr{10} & 17 & 18 & \bawsr{34}\\
                \hline par rapport au point &
                A & A & \bawsr{D} & C\\
                \hline
            \end{tabular}
        \end{center}
        \item La pièce 6 et 24 sont symétrique par rapport au point $F$.
        Place le point $F$ sur la figure. \saveenumi{2}
    \end{enumerate}
}

\slide{exo}{
    \begin{enumerate}\loadenumi\setItemColor{Gray}
        \item Ahmed dit :
        \guillemetleft J'ai transformé la pièce 11 par symétrie de centre $A$ puis par symétrie d'axe $(AB)$.\guillemetright
        Quelle pièce à t-il trouvé.
        \item Comme Ahmed rédige un programme de construction qui permet de transformer: 
        \begin{enumerate}
            \item la pièce 32 en la pièce 20.
            \item la pièce 10 en la pièce 28.
        \end{enumerate}
    \end{enumerate}
}

\scn{Construire une image par symétrie centrale sur papier blanc}

\def\caPrefix{5e-mars-2023-}
\caSlide{23-24-25}

\slide{cr}{
    \mthd{Construire l'image d'un point par une symétrie centrale}{
        Pour construire l'image du point $A$ par une symétrie de centre $O$ :
        \begin{enumerate}
            \item Tracez la demi-droite $[AO)$.
            \item Reportez la mesure $AO$ de l'autre côté du point $O$ sur la droite $(AO)$.
            \item Vous obtenez l'image du point $A$, que l'on peut nommer $A'$.
        \end{enumerate}
    }
}

\slide{cr}{
    \expl{}{Construisez l'image du point $A$ par la symétrie de centre $O$.
    \ctikz{
        \draw[gray!40] (-5,-2) rectangle (10,9);
        \drawPoint{$A$}{-3}{0}
        \drawPoint{$O$}{2}{3}[gradeColor]
    }
    }
}

\slide{cr}{
    \mthd{Construire l'image d'une figure par une symétrie centrale}{
        Pour construire l'image d'une figure par une symétrie de centre $O$ :
        \begin{enumerate}
            \item Construisez l'image de chacun des points de cette figure par la symétrie de centre $O$.
            \item Reliez chacun de ces points comme sur la figure d'origine.
        \end{enumerate}
    }
}

\slide{cr}{\bvspace{-0.5cm}
    \expl{}{Construisez l'image de la figure par la symétrie de centre $O$.
    \ctikz[0.4]{
        \draw[gray!40] (-10,-8) rectangle (11,9);
        \draw [line width=1pt] (0,5)-- (10,6)-- (5,1) -- cycle;
        \drawPoint{$O$}{1}{1}[gradeColor]
    }
}
}

\scn{Découvrir les propriétés de la symétrie centrale}

\def\caPrefix{5e-mars-2022-}
\caSlide{14-15}

\bsec{Propriétés des symétries}

\slide{exo}{\bvspace{-0.75cm}\bsmall
    \act{}{
        \begin{enumerate}
            \item \begin{enumerate}
                \item Sur feuille blanche, construire un rectangle $ABCD$.
                \item Placer un point $P$ sur la droite $(BD)$.
                \item Placer un point $E$ n'importe où sur votre feuille.
                \item Construire les points $A'$, $B'$, $C'$, $D'$ et $P'$, images des points $A$, $B$, $C$ et $D$ par la symétrie de centre $E$.
            \end{enumerate}
            \item \begin{enumerate}
                \item Comment semblent les distances $BD$ et $B'D'$ ?
                \item Comment semblent les aires des triangles $BCD$ et $B'C'D'$ ?
                \item Comment semblent les angles $\widehat{ADB}$ et $\widehat{A'D'B'}$ ?
                \item Sur quelle droite semble être située le point $P$.
                \item Comment semblent les droites $CD$ et $C'D'$ ?
            \end{enumerate}\saveenumi{2}
        \end{enumerate}
    }
}

\slide{exo}{
    \begin{enumerate}\loadenumi\setItemColor{RoyalBlue}
        \item \begin{enumerate}
            \item A partir des questions précédentes;
            formulez des conjectures sur les propriétés de la symétrie centrale.
            \item En déduire la nature du quadrilatère $A'B'C'D'$.
        \end{enumerate}
    \end{enumerate}
}

\slide{cr}{
    \ssec
    \pr{}{
        Les symétries conservent les mesures:
        \multiColItemize{2}{
            \item de distances.
            \item d'angles.
        }
    }

    \cor{}{
        Les symétries conservent:
        \multiColItemize{2}{
            \item les aires.
            \item les alignements.
        }
    }
}

\slide{cr}{
    \pr{}{
        \Sialors{$(d_1)$ est l'image de $(d_2)$ par une symétrie centrale}
        {$(d_1)$ est parallèle à $(d_2)$.}
    }

    \expl{}{On considère un triangle $OAB$.
    \begin{enumerate}
        \item Le point $A'$ est le symétrique de $A$ par rapport à $O$.
        \item Le point $B'$ est le symétrique de $B$ par rapport à $O$.
    \end{enumerate}
        Prouver que les droites $(AB)$ et $(A'B')$ sont parallèles. 
    }[\ym]
}

\scn{Utiliser les propriétés de la symétrie centrale}

\def\caPrefix{5e-juin-2022-}
\caSlide{14-15-16}

\slide{exo}{\bvspace{-0.75cm}\bsmall
    \exo{Sans le centre}{
        $[A'B']$ est le symétrique du segment $[AB]$ par rapport à un point $O$.
        En utilisant uniquement une règle non graduée et un rapporteur, construis la figure symétrique par rapport à $O$ de la ligne brisée $ABCD$.
        Comment as-tu raisonné ?\bvspace{-0.25cm}
        \ctikz[0.35]{
            \draw[gray!40] (-7,-10) rectangle (12,9);
            \draw [thick] (-6.06,1.06)-- (-2.3,2.88);
            \draw [thick] (-2.3,2.88)-- (-2.42,-1.68)-- (1.34,-3.52);
            \draw [thick] (7.34,-1.38)-- (3.58,-3.2);
            \drawPoint{A}{-6.06}{1.06}
            \drawPoint{B}{-2.3}{2.88}
            \drawPoint{C}{-2.42}{-1.68}
            \drawPoint{D}{1.34}{-3.52}
            % \drawPoint{O}{0.64}{-0.16}
            \drawPoint{A'}{7.34}{-1.38}
            \drawPoint{B'}{3.58}{-3.2}
        }
    }[\dmeepcC]
}

\slide{exo}{\bvspace{-0.7cm}\bsmall
    \exo{Figure incomplète}{
        $ABC$ est un triangle, mais le point $C$ est en dehors de la feuille.
        Construis le symétrique du triangle $ABC$ par rapport au point $O$.
        Comment as-tu raisonné ?\bvspace{-0.25cm}
        \ctikz[0.45]{
            \draw[gray!40] (-5,-10) rectangle (10,3);
            \draw [thick] (6.18,-5.08)-- (5,-1.14);
            \draw [thick] (5,-1.14)-- (9.86,-2.845909090909091);
            \draw [thick] (6.18,-5.08)-- (9.86,-4.4466468842729965);
            \drawPoint{B}{5}{-1.14}
            \drawPoint{A}{6.18}{-5.08}
            \drawPoint{O}{4.36}{-4.14}
        }
}[\dmeepcC]
}

\scn{Découvrir la notion de centre de symétrie dans une figure}

\def\caPrefix{5e-entrainement.4-}
\caSlide{29-30}

\bsec{Centre de symétrie}

\NewDocumentCommand{\mimg}{m O{1}}{%
    \listSize{#1}%
    \ifBeamer{\def\imgswidht{0.9\linewidth*#2/\thesize}}
    \ifArticle{\def\imgswidht{\linewidth*#2/\thesize}}
    \foreach \sport in {#1}{%
        \includegraphics[width=\imgswidht]{\imgf{\sport}}
    }
}

\slide{cr}{
    \vc{}{
        Un point est un \key{centre de symétrie} d'une figure,
        lorsqu'en effectuant un \key{demi-tour}
        \key{autour du point},
        la figure se \key{superposent avec elle-même}.
    }[\ym]

    \expl{Pictogrammes JO 2024}{
        \def\imgPrefix{jo/}
        \mimg{athletisme,aviron,basketball,VTT,escalade,natation}[1]
        Lesquels de ces pictogrammes possèdent un centre de symétrie?
    }
}

% \slide{exo}{}
% % VARIABLES %%%
% \date{\today}
\setSeq{3}{Nombres Relatifs}
\setGrade{4e}
\def\imgPath{enseignement/4e/nombres-relatifs}

\def\ym{\href{https://www.maths-et-tiques.fr/telech/19Nomb_rela.pdf}{Yvan Monka}}
%%

\obj{
    \item Produits et des quotients de nombres relatifs
}

\scn{Découvrir les produits de nombres relatifs}

\slide{qf}{
    \begin{enumerate}
        \item Développer les expressions suivantes :
        \multiColEnumerate{2}{
            \item $(x + 2) \times 6 =$
            \item $y \times (a + (-6)) =$
            \item $k \times (a + b) =$
            \item $8,1 \times (10 + 2) =$
        }
        \item Réduire les expressions suivantes :
        \multiColEnumerate{2}{
            \item $4 \times x =$
            \item $y \times 3 =$
            \item $a \times b =$
            \item $- q \times l =$
        }
    \end{enumerate}
}

\slide{exo}{
    \act{}{
        Dans cette activité on cherche à trouvé,
        pour deux nombres positifs $x$ et $y$,
        à quoi est égale $x \times  (- y)$ et $ (- x) \times (- y)$.
        \begin{enumerate}
            \item On commence par s'interesser au résultat de $3 \times (-5)$.
            \begin{enumerate}
                \item Combien donne $3 \times (5 + (-5))$
                \item Développer le produit $3 \times (5 + (-5))$
                \item D'après le ${\color{\currentColor}a)}$, à combien est égale la forme développer de l'expression précédente ?
                \item Combien donne $3 \times 5$ ?
                \item Combien donne alors $3 \times (-5)$?
            \end{enumerate}
            \item On s'interesse maintenant au résultat de $x \times (-y)$.\\
            Reproduit le raisonnement de ${\color{\currentColor}a)}$ à ${\color{\currentColor}e)}$ en substituant $3$ par $x$ et $5$ par $y$.
            \item Que donne le produit d'un nombre positif par un nombre négatif?
            \item On s'interesse maintenant au résultat de $-3 \times (-5)$.\\
            Reproduit le raisonnement de ${\color{\currentColor}a)}$ à ${\color{\currentColor}e)}$ en substituant $3$ par $-3$.
            \item On s'interesse maintenant au résultat de $-x \times (-y)$.\\
            Reproduit le raisonnement de ${\color{\currentColor}a)}$ à ${\color{\currentColor}e)}$ en substituant $3$ par $-x$ et $5$ par $y$.
            \item Que donne le produit d'un nombre négatif par un nombre négatif?
        \end{enumerate}
    }
}

\slide{cr}{
    \pr{}{
        Le \key{produit} de deux nombres :
        \begin{itemize}
            \item de \key{même signes} est \palt{2}{\key{positif}}.
            \item de \key{signe opposés} est \palt{2}{\key{négatifs}}.
        \end{itemize}
    }

    \expl{}{
        \multiColEnumerate{2}{
            \item $2 \times 7 = \palt{2}{14}$
            \item $2 \times (-7) = \palt{2}{-14}$
            \item $-2 \times 7 = \palt{2}{-14}$
            \item $-2 \times (-7) = \palt{2}{14}$
        }
    }
}

% % VARIABLES %%%
\setSeq{4}{Théorème de Pythagore - Contraposée et réciproque}
\setGrade{4e}
\def\imgPath{enseignement/4e/theoreme-de-pythagore/contraposee-et-reciproque/}

% \forPrint
% \setboolean{answer}{true}
%%

\obj{
    \item Comprendre les notions de réciproque et de contraposée.
    \item Utiliser la contraposée du Théorème de Pythagore pour montrer qu'un triangle n'est pas rectangle.
    \item Utiliser la réciproque du Théorème de Pythagore pour montrer qu'un triangle est rectangle.
    \item Déterminer si un triangle est rectangle ou non.
}

\obj{
    \item Reconnaitre sur un graphique une situation de proportionnalité ou de non proportionnalité.
    \item Calcule d'une quatrième proportionnelle.
    \item Utiliser une formule liant deux grandeurs dans une situation de proportionnalité.
    \item Résoudre des problèmes en utilisant la proportionnalité dans le cadre de la géométrie.
}[Flash]

\scn{Découvrir des notions de logiques; réciproque et contraposée}

\slide{qf}{\bvspace{-0.35cm}
    \begin{enumerate}
        \item Les tableaux suivants représentent-ils des situations de proportionnalité ?
        Utilisez une calculatrice pour vérifier vos hypothèses.
        \multiColEnumerate{1}{
            \item \Propor[Simple]{1/2,6/12,3/5,10/20}
            \item \Propor[Simple]{2/3,6/9,30/45}
        }
        \item Essayez ensuite de justifier vos réponses sans calculatrice en expliquant votre raisonnement.
    \end{enumerate}
}

\bsec{Logique}
\bsubsec{Réciproque}

\slide{exo}{\bshrink
    \act{}{
        \begin{enumerate}
            \item « \Sialors{c'est un triangle}{il a trois côtés} » est une \key{proposition}.  
            Est-elle vraie ? Justifie ta réponse.  
            
            \item « \Sialors{c'est un triangle}{il a quatre côtés} » est une autre proposition.  
            Est-elle vraie ? Justifie ta réponse.  
            
            \item La première proposition est composée de deux parties :  
            \multiColItemize{1}{
                \item l'\key{antécédent} : « c'est un triangle »,
                \item le \key{conséquent} : « il a trois côtés ».
            }  
            On appelle \key{réciproque} d'une proposition la phrase qu'on obtient en inversant l'antécédent et le conséquent.  
            Écris la réciproque de la première proposition.
            Est-elle vraie ? \saveenumi
        \end{enumerate}
    }[\href{http://www.mathsaharry.com/aw/52.pdf}{Math à Harry}]
}

\slide{exo}{
    \begin{enumerate} \loadenumi[act]
        \item Les propositions suivantes sont-elles vraies ?
        Écris leurs réciproques et détermine si elles sont vraies.
        \multiColEnumerate{1}{
            \item \Sialors{c'est un carré}{c'est un rectangle avec tous ses côtés égaux}
            \item \Sialors{il pond des œufs}{c'est un oiseau}
            \item \Sialors{c'est un rectangle}{c'est un quadrilatère dont tous les côtés sont parallèles} 
            \item \Sialors{c'est un rectangle}{c'est un carré}
            \item \Sialors{$AB = BC$}{$B$ est le milieu de $[AC]$}
        } 
    \end{enumerate}
}

\slide{cr}{\bsmall
    \ssec
    \ssubsec

    \df{}{
        On appelle \key{réciproque}
        d'une proposition :
        {« \Sialors{$A$}{$B$} »}
        ; la proposition :
        {« \Sialors{$B$}{$A$} »}.
    }
    
    \rmk{}{
        Une proposition peut être vraie sans que sa réciproque le soit, et inversement.
    }

    \expl{}{
        La proposition « \Sialors{$[AB]$ et $[CD]$ ne se coupent pas}{$[AB]$ et $[CD]$ sont parallèles}»
        est \bawsr{fausse}.\\
        Sa réciproque: \bawsr{« \Sialors{$[AB]$ et $[CD]$ sont parallèle}{$[AB]$ et $[CD]$ ne se coupent pas}»}
        est \bawsr{vrai}.
    }
}

\bsubsec{Contraposée}

\slide{exo}{
    \ssubsec
    \act{}{
        On appelle \key{contraposée} d'une proposition
        {« \Sialors{$A$}{$B$} »}
        la proposition obtenue en écrivant :  
        {« \Sialors{non $B$}{non $A$} »}.
        \begin{enumerate}
            \item La contraposée de la proposition : «\Sialors{c'est un triangle}{il a trois côtés}».
            est donc «\Sialors{il n'a pas trois cotés}{ce n'est pas un triangle}» Est-elle vraie ? \saveenumi
        \end{enumerate}  
    }
}

\slide{exo}{
    \begin{enumerate} \loadenumi[act]
        \item Les propositions suivantes sont-elles vraies ? Écris leurs contraposées et dis si elles sont vraies :  
        \multiColEnumerate{1}{ 
            \item \Sialors{$x=7$}{$x$ est un nombre premier}
            \item \Sialors{c'est un nombre positif}{il est strictement inférieur à zéro}
            \item \Sialors{il est à Issy-les-Moulineaux}{il n'est pas en Espagne}  
            \item \Sialors{$AB=BC$}{$B$ est le milieu de $[AC]$} 
        }
        \item Que remarques-tu ?
    \end{enumerate}
}

\slide{cr}{
    \df{}{
        On appelle \key{contraposée}
        d'une proposition ;
        {« \Sialors{$A$}{$B$} »} ;
        la proposition : 
        {« \Sialors{non $B$}{non $A$} »}.  
    }

    \rmk{}{
        Une proposition et sa contraposée sont toujours soit toutes les deux vraies,
        soit toutes les deux fausses.
    }

    \expl{}{
        La proposition « \Sialors{c'est un triangle est équilatéral}{ses trois côtés sont égaux} »  
        est \bawsr{vraie}.  
        Sa contraposée: \bawsr{« \Sialors{ses trois côtés ne sont pas égaux}{ce n'est pas un triangle équilatéral} »},
        est \bawsr{également vraie}.
    }
}

\bsec{Contraposée du Théorème de Pythagore}

\slide{cr}{
    \ssec
    \ctr{du théorème de Pythagore}{}
}

\bsec{Réciproque du Théorème de Pythagore}

\slide{cr}{
    \ssec
    \rcp{du théorème de Pythagore}{}
}

% \setSeq{6}{Géométrie dans l'espace - Solides}
\setGrade{6e}
\def\imgPath{enseignement/6e/geometrie-dans-l-espace/solides/}

\obj{
    \item Reconnaître des solides (pavé droit, cube, cône et cylindre).
    \item Identifier les caractéristiques de différents solides :
    sommets, faces et arêtes.
    \item Représenter un cube et un pavé droit.
}


\slide{}{}
% % VARIABLES %%%
\setSeq{5}{Nombres - Decimaux}
\setGrade{6e}

\def\imgPath{enseignement/6e/nombres/decimaux/}

\def\ym{\href{https://www.maths-et-tiques.fr/telech/19Nombres1.pdf}{Yvan Monka}}

% https://www.maths-et-tiques.fr/telech/19Nombres2.pdf

\obj{
    \item Utiliser une fraction et en donner progressivement le statut de nombre.
    \item Utiliser et représenter les nombres décimaux jusqu'à trois décimales.
    \item Ajouter, soustraire et multiplier des nombres décimaux.
    \item Résoudre des problèmes relevant des structures additives et multiplicatives en mobilisant une ou plusieurs étapes de raisonnement.
}

\scn{Découvrir les fractions décimales}

\bsubsec{Nombres décimaux}

\slide{exo}{
    \act{}{
        \multiColEnumerate{1}{
            \item $\pow{10}{2} = \bawsr{\num{\powTenPositive{2}}}$
            \item $\pow{10}{3} = \bawsr{\num{\powTenPositive{3}}}$
            \item $\pow{10}{6} = \bawsr{\num{\powTenPositive{6}}}$
            \item $\powBrace{10}{15} = \bawsr{\num{\powTenPositive{15}}}$
            \item $\powBrace{10}{100} = \bawsr{\powTenBrace{100}}$
        }
    }
}

\slide{cr}{
    \ssubsec

    \vc{}{
        Une \key{puissance de 10} est le résultat d'un produit dont tous les facteurs sont $10$.
    }

    \expl{}{
        \Tableau[%
            DoubleEntree,
            Stretch=1.5,
            Couleur=gradeColor!15,
            LegendesH={$100$,$2$,$\num{1001}$,$\num{100000}$,$\num{200}$,$\num{10}$},
            LegendesV={Puissance de 10 ?},
            Largeur=2cm
        ]{\bawsr{Oui},\bawsr{Non},\bawsr{Non},\bawsr{Oui},\bawsr{Non},\bawsr{Oui}}
    }
}

\slide{cr}{
    \pr{}{
        Pour $n$ un nombre entier, on a :
        \begin{align*}
            \powBrace{10}{n} = \bawsr{\powTenBrace{n}}
        \end{align*}
    }

    \expl{}{
        \multiColEnumerate{2}{
            \item $\pow{10}{3} = \powTenPositive{3}$
            \item $\pow{10}{6} = \powTenPositive{6}$
        }
    }
}

\slide{cr}{
    \df{}{
        On appelle \key{fraction décimale} une fraction dont le \key{dénominateur est une puissance de $10$}.
    }

    \expl{}{
        \Tableau[%
            DoubleEntree,
            Stretch=1.5,
            Couleur=gradeColor!15,
            LegendesH={$\frac{1}{10}$,$\frac{1}{2}$,$\frac{5}{10}$,$\frac{1}{\num{1000}}$,$\frac{1}{\num{30000}}$,$\frac{546985}{\num{10000000}}$},
            LegendesV={Fraction décimale ?},
            Largeur=2cm
        ]{\bawsr{Oui},\bawsr{Oui},\bawsr{Non},\bawsr{Oui},\bawsr{Non},\bawsr{Oui}}
    }
}

\slide{exo}{
    \act{}{
        Ecrire les nombres suivants sous forme de fractions décimales.
        \multiColEnumerate{2}{
            \item $\num{3.2} = \bawsr{\frac{32}{10}}$
            \item $\num{10.2} = \bawsr{\frac{102}{10}}$
            \item $\num{0.03} = \bawsr{\frac{3}{100}}$
            \item $\num{0.0001} = \bawsr{\frac{1}{1000}}$
            \item $6 \div 10 \div 10 = \bawsr{\frac{6}{100}}$
            \item $32 \div 10 \div 10 \div 10 \div 10 = \bawsr{\frac{32}{10000}}$
        }
    }
}

\slide{cr}{
    \df{}{
        On appelle \key{nombre décimal}, un nombre pouvant s'écrire sous forme de fraction décimale.
    }

    \expl{}{
        \Tableau[%
            DoubleEntree,
            Stretch=1.5,
            Couleur=gradeColor!15,
            LegendesH={$\num{0.6}$,$\num{13.2}$,$\frac{1}{10}$,$\frac{1}{2}$,$60$,$\frac{1}{3}$,$\frac{30}{3}$,$\pi$,$0$},
            LegendesV={Nombre décimal ?},
            Largeur=2cm
        ]{\bawsr{Oui},\bawsr{Oui},\bawsr{Oui},\bawsr{Oui},\bawsr{Oui},\bawsr{Non},\bawsr{Oui},\bawsr{Non},\bawsr{Oui}}
    }
}

\slide{cr}{
    \pr{}{Pour $n$ un nombre entier, on a :
    \begin{align*}
        1 \repeatBrace{\div 10}{n}[quotients]
        = \frac{1}{\powTenBrace{n}}
        = \underbrace{0, 0 ... 0}_{n \textrm{ zéros}} 1
    \end{align*}
    }

    \expl{}{
        \multiColEnumerate{1}{
            \item $1 \div 10 \div 10 = \frac{1}{\bawsr{100}} = \bawsr{\num{0.01}}$
            \item $\bawsr{1 \div 10 \div 10 \div 10 \div 10} = \frac{1}{10000} = \bawsr{\num{0.0001}}$
            \item $1 \bawsr{\div 10 \div 10 \div 10} = \frac{1}{\bawsr{1000}} = \bawsr{\num{0.001}}$
        }
    }
}

\scn{Décomposer un nombre décimal en fractions décimales}

% % VARIABLES %%%
\setSeq{5}{Proportionnalité - Tableaux et graphiques}
\setGrade{4e}
\def\imgPath{enseignement/4e/theoreme-de-pythagore/contraposé-et-reciproque/}
% \setboolean{answer}{true}
\def\ym{\href{https://www.maths-et-tiques.fr/telech/19Proport1.pdf}{Yvan Monka}}
% \forStudents
% \setboolean{demonstration}{false}
%%

\def\cp{coefficient de proportionnalité}
% \obj{
%     \item Reconnaitre sur un graphique une situation de proportionnalité ou de non proportionnalité.
%     \item Calcule d'une quatrième proportionnelle.
%     \item Utiliser une formule liant deux grandeurs dans une situation de proportionnalité.
%     \item Résoudre des problèmes en utilisant la proportionnalité dans le cadre de la géométrie.
% }

\renewcommand{\arraystretch}{1.5}

\avspace{0.1cm}

\bsec{Grandeurs proportionnelles}

\df{Grandeurs proportionnelles}{%
    Deux grandeurs sont dites \key{proportionnelles}
    lorsque les valeurs de l'une sont obtenues en multipliant les valeurs de l'autre par un même nombre non nul,
    appelé \key{coefficient de proportionnalité}.
}

\expl{}{
    \begin{tabular}{|>{\bfseries}c|*{4}{c|}} % Colonne en gras pour la première colonne
        \hline
        \rowcolor{gray!15} 
        Grandeur 1 & coté & coté & rayon & tension \\ \hline
        Grandeur 2  & périmètre du carré & aire du carré & périmètre du cercle & intensité \\ \hline
        coefficient?  & \nswr{$4$} & \nswr{non} & \nswr{$2\pi$} & \nswr{$R$} \\ \hline
    \end{tabular}
    % \begin{itemize}
    %     \item La longueur du coté d'un carré et sont périmètre avec \cp{} : $4$ car $\mathcal{P} = 4 \times c$.
    %     \item 
    % \end{itemize}
}

\bsec{Tableau de proportionnalité}

% \pr{Coefficient de proportionnalité}{
%     \Sialors{on est dans un tableau de proportionnalité}
%     {on peu passer d'une ligne à l'autre en multipliant par un \key{\cp}}
% }

Pour un tableau de proportionnalité : \propTable{a}{c}{b}{d} avec $a,b,c,d$ des nombres.

\pr{}{
    On peu passer d'une ligne à l'autre en multipliant par un \cp.
}

\expl{}{\vspace{-0.75cm}
    \multiColEnumerate{2}{
        \item \Propor[Stretch=1.5, Simple]{1/2.5,2/5,5/12.5}
        \FlechesPD{1}{2}{$\times\nswr{2}$}
        \FlechesPG{2}{1}{$\div\nswr{2}$}
        \item \Propor[Stretch=1.5, Simple]{120/12,3/0.3}
        \FlechesPD{1}{2}{$\times\nswr{\frac{1}{10}}$}
        \FlechesPG{2}{1}{$\div\nswr{\frac{1}{10}}$}
    }
}

\pr{Egalité des produits en croix}{
    On a l'égalité: $a \times d = b \times c$.
}

\newpage

\expl{}{
    Utiliser l'égalité des produits en croix pour vérifier si on a bien proportionnalité.
    \vspace{-0.75cm}\multiColEnumerate{2}{
        \item \Propor[Stretch=1.5, Simple]{15/25,1.2/2}
        \item \Propor[Stretch=1.5, Simple]{6/1.5,3/0.5}
    }\vspace{-0.75cm}
    \nswr[5]{
        \begin{enumerate}
            \item $15 \times 2 = 30$ et $25 \times \np{1.2} = 30$
            alors on égalité des produits en croix : $15 \times 2 = 25 \times \np{1.2}$,
            il sagit donc d'un tableau de proportionnalité.
            \item $6 \times \np{0.5} = 3$ et $\np{1.5} \times 3 = 4.5$
            alors on n'a pas égalité des produits en croix : $6 \times \np{0.5} \neq \np{1.5} \times 3 = 4.5$,
            il ne sagit donc pas d'un tableau de proportionnalité.
        \end{enumerate}
    }
}

\cor{Egalité des quotients}{
    On a aussi : $\frac{a}{b} = \frac{c}{d}$
}

\demo{}{
    On a : $\frac{a}{b} = \frac{ a\times d}{b \times d} = \frac{ \nswr{b \times c} }{b \times d}$ d'après l'égalité des produits en croix.\\
    Or $\frac{ b \times c }{b \times d} = \nswr{\frac{c}{d}}$.
    Alors $\frac{a}{b} = \nswr{\frac{c}{d}}.$
}[\href{https://pedagogie.ac-toulouse.fr/mathematiques/system/files/2023-03/demonstration_produits_en_croix.pdf}{Académie de Toulouse}]

\expl{}{
    Utiliser l'égalité des quotients pour vérifier si on a bien proportionnalité.
    \vspace{-0.75cm}\multiColEnumerate{2}{
        \item \Propor[Stretch=1.5, Simple]{10/12,5/6,20/23}
        \item \Propor[Stretch=1.5, Simple]{2/3,4/6,6/9}
    }\vspace{-0.75cm}
    \nswr[5]{
        \begin{enumerate}
            \item $\frac{10}{12} = \frac{20}{24} \neq \frac{20}{23}$
            alors on n'a pas égalité des quotients,
            il ne sagit donc pas d'un tableau de proportionnalité.
            \item $\frac{2}{3} = \frac{4}{6} = \frac{6}{9}$
            alors on égalité des quotients,
            il sagit donc d'un tableau de proportionnalité.
        \end{enumerate}
    }
}

% \ctr{}{
%     \Sialors{$\frac{a}{b} \neq \frac{c}{d}$}{l'égalité des quotients n'est pas respécté et on a pas proportionnalité}
% }

\mthd{Calcul 4e proportionnelle}{
    Si l'on connait 3 valeurs par exemple $b,c,d$.\\
    On peu calculer $a$ avec l'égalité $a = \nswr{\frac{b \times c}{d}}$.
}

\demo{}{
    En partant de l'égalité des produits en croix : $a \times d = b \times c$.\\
    Alors $a$ est le nombre qui multiplié par \nswr{$d$} done \nswr{$b \times c$}.\\
    D'après la définition du quotient : $a = \nswr{\frac{b \times c}{d}}$.
}

\expl{Compléter les tableaux de proportionnalité suivants}{
    \def\cW{2.5cm}
    \multiColEnumerate{2}{
        \item \begin{tabular}{|C{\cW}|C{1cm}|}
            \hline
            \np{9.6} & 3 \\ \hline
            \nswr{$\frac{\np{9.6}\times2}{3} = \np{6.4}$} & 2 \\ \hline
        \end{tabular}
        \item \begin{tabular}{|C{1cm}|*{2}{C{\cW}|}}
            \hline
            3 & 5 & \nswr{$\frac{\np{21.7}\times5}{7} = \np{15.5}$} \\ \hline
            \np{4.2} & \nswr{$\frac{\np{4.2}\times6}{3} = 7$} & \np{21.7} \\ \hline
        \end{tabular}
    }
}

\bsec{Représentation graphique}

\pr{Représentation graphique}{
    Sur un graphique, une situation de proportionnalité est représentée par des points alignés
    avec l'origine.
}[\ym]

\expl{}{
    Chaque graphique suivant représente-t-il une situation de proportionnalité ?
    \def\repere{%
        \tkzInit[xmin=0,xmax=8,ymin=0,ymax=8]
        \tkzGrid[sub,color=gradeColor!50!white,subxstep=1,subystep=1]        
        \tkzLabelX[step=2]
        \tkzLabelY[step=2]
        \tkzDrawY[step=1]
        \tkzDrawX[step=1]
    }
    \def\size{0.55}\def\crossWidth{0.25mm}
    \vspace{-0.75cm}
    \multiColItemize{3}{
        \item[]\ctikz[\size]{
            \repere
            \node at (4,9) {\cir[gradeColor]{1}};
            \drawPoint{}{2}{2.4}[Red]
            \drawPoint{}{4}{4.8}[Red]
            \drawPoint{}{6.5}{7.8}[Red]
            \ifthenelse{\boolean{answer}}{\draw[answer] (-1,-1.2) -- (7.5,9);}{}
        }
        \item[]\ctikz[\size]{
            \node at (4,9) {\cir[gradeColor]{2}};
            \repere
            \drawPoint{}{1}{1.3}[Red]
            \drawPoint{}{3}{3}[Red]
            \drawPoint{}{6}{7.8}[Red]
            \ifthenelse{\boolean{answer}}{\draw[answer] (-1,-1.3) -- (9/1.3,9);}{}
        }
        \item[]\ctikz[\size]{
            \repere
            \node at (4,9) {\cir[gradeColor]{3}};
            \drawPoint{}{1}{2}[Red]
            \drawPoint{}{4}{5}[Red]
            \drawPoint{}{6}{7}[Red]
            \drawPoint{}{7}{8}[Red]
            \ifthenelse{\boolean{answer}}{\draw[answer] (-1,0) -- (8,9);}{}
        }
    }
    \nswr[6]{Seuls les points du graphique \cir[gradeColor]{1} sont alignés avec l'origine.
    Ainsi, parmi les trois graphiques,
    c'est le seul qui représente une situation de proportionnalité.}
}



% \slide{qf}{
%     Les situations présentées dans ces tableaux sont-elles proportionnelles ?
%     \multiColEnumerate{2}{
%         \item \begin{center}
%             \Propor[Simple,
%             Math,
%             Stretch=1.25,%
%             ]{12/3,16/4,40/10}
%         \end{center}
%         \item \begin{center}
%             \Propor[Simple,
%             Math,
%             Stretch=1.25,%
%             ]{15/5,9/3,20/6}
%         \end{center}
%     }
% }

% \slide{qf}{\calculator \\ Completer les tableaux suivants :
%     \multiColEnumerate{3}{
%         \item \begin{center}
%             \Propor[Simple,
%             Math,
%             Stretch=1.25,%
%             ]{6/5,\nswr{\np{2.4}}/2}
%         \end{center}
%         \item \begin{center}
%             \Propor[Simple,
%             Math,
%             Stretch=1.25,%
%             ]{\np{237.6}/\nswr{66},\np{46.8}/13}
%         \end{center}
%         \item \begin{center}
%             \Propor[Simple,
%             Math,
%             Stretch=1.25,%
%             ]{\nswr{12}/18,-3/-4.5}
%         \end{center}
%     }
% }

% \slide{qf}{
%     \nullsubsec{}{
%         Sachant que huit briques de masse identique pèsent 13,6 kg, calcule la masse de six de ces
%         briques.
%     }[\afa{4e}[6]]
% }

% \slide{qf}{
%     \nullsubsec{}{
%         \begin{enumerate}
%             \item Sachant que la longueur $\mathcal{P}$ d'un cercle
%             est proportionnelle à son rayon $r$
%             avec un \cp $2\pi$.
%             Donnez la formule permettant de calculer $\mathcal{P}$ en fonction de $r$.
%             \item Sachant que la tension $U$ aux bornes d'une résistance
%             est proportionnelle à l'intensité $I$ du courant qui la traverse
%             avec un \cp égal à la valeur de la résistance $R$.
%             Donnez la formule permettant de calculer $U$ en fonction de $I$.
%         \end{enumerate}
%     }
% }

% % VARIABLES %%%
\setSeq{3}{Nombres Relatifs - Sommes et différences}
\setGrade{5e}

% \setboolean{answer}{false}
% \setboolean{newPageOnSlide}{true}

\def\imgPath{enseignement/5e/nombres-relatifs/sommes-et-differences/}
\def\ym{\href{https://www.maths-et-tiques.fr/telech/19Nomb_rel2.pdf}{Yvan Monka}}
% Yvan Monka RÈGLES DE CALCUL : https://www.maths-et-tiques.fr/telech/19Calcul_num.pdf
%%

\obj{
    \item Traduire un enchaînement d'opérations à l'aide d'une expression avec des parenthèses.
    \item Effectuer mentalement, à la main ou l'aide d'une calculatrice un enchaînement.
    \item Additionner et soustraire des nombres décimaux relatifs.
    \item Résoudre des problèmes faisant intervenir des nombres décimaux relatifs et des fractions simples.
}

\qfSlide{
    \multiColEnumerate{2}{
        \item $8 + 9 - 7 = \bawsr{10}$
        \item $8 \times (2 + 10) = \bawsr{96}$
        \item $12 + 11 \times 10 = \bawsr{122}$
        \item $7 \div (7 - 3 + 6) = \bawsr{\frac{7}{10} = 0,7}$
        \item $9 - 8 + 1 = \bawsr{2}$
    }
}

\bsec{Opérations}
\bsubsec{Règles opératoires}

\slide{cr}{
    \sseq\ssec\ssubsec

    \rl{}{
        Les calculs se font dans l'ordre des priorités suivant:%
        \begin{enumerate}
            \item La multiplication et la division
            \item L'addition et la soustraction
        \end{enumerate}
    }
}

\slide{}{
    \rl{}{
        En cas d'opérations de mêmes priorités, on effectue les opérations de gauche à droite.
    }
}

\slide{}{
    \rl{}{
        On commence par effectuer les calculs entre parenthèses.
    }
}

\bsubsec{Vocabulaire opératoires}

\slide{}{
    \ssubsec
    \vc{}{
        On connait quatres types d'opérations:
        \begin{itemize}
            \item L'\key{addition} permet de calculer la \key{somme} de deux \key{termes}.
            \item La \key{soustraction}  permet de calculer la \key{différence} entre deux \key{termes}.
            \item La \key{multiplication} permet de calculer la \key{produit} de deux \key{facteurs}.
            \item La \key{division} permet de calculer la \key{quotient} de deux \key{nombres}.
        \end{itemize}
    }
}

\slide{}{
    \pr{}{Dans un calculs,
    la dernière opérations détermine le nom pour désigner le calcul en entier.}
}

\slide{}{
    \expl{Donner le type d'opérations des expression suivantes}{
        \multiColEnumerate{2}{
            \item $1,6 + 3 + 4$ est \bawsr{une somme}.
            \item $(\frac{2}{6} + 3) \times 9$ est \bawsr{un produit}.
            \item $6,6 + 1 \times 8$ est \bawsr{une somme}.
            \item $\frac{2}{6} + 3 - 9$ est \bawsr{une différence}.
            \item $\pi \div (3 - 9)$ est \bawsr{un quotient}.
        }
    }
}

\bsec{Sommes et différences de nombres relatifs}

\slide{exo}{
    \act{}{

    }
}

% % https://clg-monnet-briis.ac-versailles.fr/IMG/pdf/tp_programme_de_calcul_1_1_.pdf

\setGrade{5e}
\tp{\Scratch - Calcul littéral}
[corr]
% DOC https://ctan.math.illinois.edu/macros/latex/contrib/scratch3/scratch3-fr.pdf

\setscratch{scale=.75}

\def\block{{\setscratch{scale=.5}\begin{scratch}\blockmove{\Large bloc}\end{scratch} }}

\newcommand{\scr}[1]{\begin{scratch}#1\end{scratch}}

\definecolor{smotion}{HTML}{4C97FF} % #4C97FF
\definecolor{slooks}{HTML}{9966FF} % #9966FF
\definecolor{ssound}{HTML}{D65CD6} % #D65CD6
\definecolor{sevents}{HTML}{FFD500} % #FFD500
\definecolor{scontrol}{HTML}{FFAB19} % #FFAB19
\definecolor{ssensing}{HTML}{4CBFE6} % #4CBFE6
\definecolor{soperators}{HTML}{6DB26E} % #6DB26E
\definecolor{svariables}{HTML}{F28011} % #F28011
\definecolor{smyblocks}{HTML}{FF6680} % #FF6680

\def\smotion{\textcolor{smotion}{\faCircle\,Mouvement}} % Déplacement du lutin
\def\slooks{\textcolor{slooks}{\faCircle\,Apparence}} % Modifier l'apparence du lutin ou de la scène
\def\ssound{\textcolor{ssound}{\faCircle\,Son}} % Jouer des sons ou de la musique
\def\sevents{\textcolor{sevents}{\faCircle\,Événement}} % Déclencher des scripts en réponse à des actions
\def\scontrol{\textcolor{scontrol}{\faCircle\,Contrôle}} % Boucles, conditions, et contrôle du flux
\def\ssensing{\textcolor{ssensing}{\faCircle\,Capteur}} % Réagir à des informations extérieures ou internes
\def\soperators{\textcolor{soperators}{\faCircle\,Opérateur}} % Calculs mathématiques et logiques
\def\svariables{\textcolor{svariables}{\faCircle\,Variable}} % Stockage et manipulation de données
\def\smyblocks{\textcolor{smyblocks}{\faCircle\,Mes blocs}} % Création de blocs personnalisés

\def\spen{{\icon{scratch/pen} Stylo}}
\def\spenExtension{{\icon{scratch/pen-extension} Stylo}}
\def\sextensions{{\icon{scratch/extensions} $\lbrack$ Ajouter une extensions $\rbrack$}}
\def\sflag{{\icon{scratch/flag}%
%  Drapeau
}}

% \setscratch{scale=.75}
% \setscratch{print=true}
% \setscratch{fill blocks=true}

\exo{Découvrire les variables}{
    \begin{enumerate}
        \item Utilisez les blocs des onglets \slooks{} et \sevents{} pour créer un programme \Scratch qui fait dire « Salut » au lutin pendant 2 secondes lorsque l'on clique sur le \sflag.
        \item Depuis l'onglet \svariables{} :
        \begin{itemize}
            \item Créez une variable \ovalvariable{nombre}.
            \item Ajoutez \scr{\blockvariable{mettre \selectmenu{nombre} à \ovalnum{$1$}}} dans votre programme pour initialiser \ovalvariable{nombre} à la valeur $1$.
            \item Faites dire au lutin la valeur de \ovalvariable{nombre} pendant 2 secondes lorsque l'on clique sur le \sflag.
            \item Testez le programme pour plusieurs valeurs de \ovalvariable{nombre} que vous pouvez modifier dans \scr{\blockvariable{mettre \selectmenu{nombre} à \ovalnum{$1$}}}.
        \end{itemize}
        \item \begin{itemize}
            \item Utilisez à nouveau \scr{\blockvariable{mettre \selectmenu{nombre} à \ovalnum{ }}} et \ovaloperator{\ovalnum{ } + \ovalnum{ }} pour ajouter $5$ à \ovalvariable{nombre}.
            \item Faites dire au lutin la valeur de \ovalvariable{nombre} pendant 2 secondes après cette opération.
            \item Testez le programme pour plusieurs valeurs de \ovalvariable{nombre}.
        \end{itemize}
        \item Depuis l'onglet \smyblocks{} :
        \begin{itemize}
            \item Créez un bloc \scr{\blockmoreblocks{Exercice 1}}.
            \item Sous \scr{\initmoreblocks{définir \namemoreblocks{Exercice 1}}}, placez les blocs que vous avez utilisés dans cet exercice.
            \item Placez \scr{\blockmoreblocks{Exercice 1}} de sorte que l'exercice s'exécute lorsque vous appuyez sur le \sflag.
        \end{itemize}
    \end{enumerate}
}
% https://clg-monnet-briis.ac-versailles.fr/IMG/pdf/tp_programme_de_calcul_1_1_.pdf

\exo{Implémenter un programme de calcul dans Scratch}{
    \begin{enumerate}
        \item Créez un bloc \scr{\blockmoreblocks{Exercice 2}} pour composer le nouveau programme.
        \item Créez une variable \ovalvariable{résultat} pour stocker le résultat des calculs.
        \item Initialisez la variable \ovalvariable{nombre} avec un nombre de votre choix.
        \item Utilisez des blocs de l'onglet \soperators{} pour créer un programme correspondant au programme de calcul suivant :
        \begin{itemize}
            \item Choisissez un nombre.
            \item Ajoutez-lui $3$.
            \item Multipliez le résultat par $5$.
        \end{itemize}
        \item Faites dire au lutin le résultat de ce calcul.
        \item Testez le programme avec plusieurs valeurs de \ovalvariable{nombre} et vérifiez que le résultat affiché par le lutin est correct.
    \end{enumerate}
}

% % VARIABLES %%%
% \def\authors{\jules \ et \href{http://www.cellulegeometrie.eu/documents/pub/pub_14.pdf}{la Haute École en Hainaut}}
\setGrade{5e}
\def\assignmentNameWidth{6cm}
\tp{Scratch - Triangles}
\def\imgPath{enseignement/6e/geometrie-plane/frises/}
%%

% DOC https://ctan.math.illinois.edu/macros/latex/contrib/scratch3/scratch3-fr.pdf

\setscratch{scale=.75}

\def\block{{\setscratch{scale=.5}\begin{scratch}\blockmove{\Large bloc}\end{scratch} }}

\newcommand{\scr}[1]{\begin{scratch}#1\end{scratch}}

\definecolor{smotion}{HTML}{4C97FF} % #4C97FF
\definecolor{slooks}{HTML}{9966FF} % #9966FF
\definecolor{ssound}{HTML}{D65CD6} % #D65CD6
\definecolor{sevents}{HTML}{FFD500} % #FFD500
\definecolor{scontrol}{HTML}{FFAB19} % #FFAB19
\definecolor{ssensing}{HTML}{4CBFE6} % #4CBFE6
\definecolor{soperators}{HTML}{6DB26E} % #6DB26E
\definecolor{svariables}{HTML}{F28011} % #F28011
\definecolor{smyblocks}{HTML}{FF6680} % #FF6680

\def\smotion{\textcolor{smotion}{\faCircle\,Mouvement}} % Déplacement du lutin
\def\slooks{\textcolor{slooks}{\faCircle\,Apparence}} % Modifier l'apparence du lutin ou de la scène
\def\ssound{\textcolor{ssound}{\faCircle\,Son}} % Jouer des sons ou de la musique
\def\sevents{\textcolor{sevents}{\faCircle\,Événement}} % Déclencher des scripts en réponse à des actions
\def\scontrol{\textcolor{scontrol}{\faCircle\,Contrôle}} % Boucles, conditions, et contrôle du flux
\def\ssensing{\textcolor{ssensing}{\faCircle\,Capteur}} % Réagir à des informations extérieures ou internes
\def\soperators{\textcolor{soperators}{\faCircle\,Opérateur}} % Calculs mathématiques et logiques
\def\svariables{\textcolor{svariables}{\faCircle\,Variable}} % Stockage et manipulation de données
\def\smyblocks{\textcolor{smyblocks}{\faCircle\,Mes blocs}} % Création de blocs personnalisés

\def\spen{{\icon{scratch/pen} Stylo}}
\def\spenExtension{{\icon{scratch/pen-extension} Stylo}}
\def\sextensions{{\icon{scratch/extensions} $\lbrack$ Ajouter une extensions $\rbrack$}}
\def\sflag{{\icon{scratch/flag}%
%  Drapeau
}}

% \setscratch{scale=.75}
% \setscratch{print=true}
% \setscratch{fill blocks=true}

\hint{
    \begin{itemize}
        \item Bien lire les indications.
        \item Répondre aux questions sur son cahier d'exercices.
        \item Appeler M. Pesin à la fin de chaque partie.
        \item Enregistre tes productions avec le nom : "NOM.S-tp-frises-partie-X.ggb"
    \end{itemize}
}

\section{Tracer un triangle équilatéral à l'aide de Scratch} 

\begin{enumerate}
    \item Dessine un triangle équilatéral à main levée.
    \item Note les propriétés d'un triangle équilatéral.
    \item Trace un triangle équilatéral à l'aide de ta règle et ton compas.
    \item Mesure ses angles avec un rapporteur et vérifie si tes mesures confirment les propriétés d'un triangle équilatéral.
    \item Assemble un script dans \Scratch{} pour que le lutin dessine un triangle équilatéral.
    \item Assemble un script dans \Scratch{} pour que le lutin dessine un triangle équilatéral. 
    \hint{\begin{enumerate} 
        \item Utilise une boucle pour répéter les instructions nécessaires au tracé. 
        \item Réfléchis attentivement à l'angle de rotation. Imagine que tu es à la place du lutin et que tu suis les instructions de ton programme : comment devrais-tu te déplacer et tourner pour revenir à ta position de départ ? 
    \end{enumerate}} 
    
\end{enumerate}

\section{Frises}

\begin{enumerate}
    \item Ecrit un programme permettant de dessiner cette frise :
    \ctikz[0.85]{
        % Données des triangles
        \def\side{2} % Longueur du côté du triangle
        \def\spacing{1} % Espacement entre les triangles
        % Calcul des coordonnées
        \foreach \i in {0, 1, 2, 3} {
            % Position de la base gauche du triangle
            \pgfmathsetmacro{\xBase}{\i * (\side + \spacing)}
            \pgfmathsetmacro{\yBase}{0}
            % Dessin du triangle équilatéral
            \draw (\xBase, \yBase) -- 
                ({\xBase + \side}, \yBase) -- 
                ({\xBase + \side / 2}, {\yBase + \side * sqrt(3) / 2}) -- 
                cycle;
        }
    }
    \item Et cette frise :
    \ctikz[0.55]{
        \draw[] (0,0) grid (5,1);
    }
    \item Et ce pavage :
    \ctikz[0.5]{
        \draw[] (0,0) grid (5,8);
    }
\end{enumerate}



% % VARIABLES %%%
% \def\authors{\jules \ et \href{http://www.cellulegeometrie.eu/documents/pub/pub_14.pdf}{la Haute École en Hainaut}}
% \setGrade{6e}
\setTitle{Scratch - Prise en main}
% DOC https://ctan.math.illinois.edu/macros/latex/contrib/scratch3/scratch3-fr.pdf

\setscratch{scale=.75}

\def\block{{\setscratch{scale=.5}\begin{scratch}\blockmove{\Large bloc}\end{scratch} }}

\newcommand{\scr}[1]{\begin{scratch}#1\end{scratch}}

\definecolor{smotion}{HTML}{4C97FF} % #4C97FF
\definecolor{slooks}{HTML}{9966FF} % #9966FF
\definecolor{ssound}{HTML}{D65CD6} % #D65CD6
\definecolor{sevents}{HTML}{FFD500} % #FFD500
\definecolor{scontrol}{HTML}{FFAB19} % #FFAB19
\definecolor{ssensing}{HTML}{4CBFE6} % #4CBFE6
\definecolor{soperators}{HTML}{6DB26E} % #6DB26E
\definecolor{svariables}{HTML}{F28011} % #F28011
\definecolor{smyblocks}{HTML}{FF6680} % #FF6680

\def\smotion{\textcolor{smotion}{\faCircle\,Mouvement}} % Déplacement du lutin
\def\slooks{\textcolor{slooks}{\faCircle\,Apparence}} % Modifier l'apparence du lutin ou de la scène
\def\ssound{\textcolor{ssound}{\faCircle\,Son}} % Jouer des sons ou de la musique
\def\sevents{\textcolor{sevents}{\faCircle\,Événement}} % Déclencher des scripts en réponse à des actions
\def\scontrol{\textcolor{scontrol}{\faCircle\,Contrôle}} % Boucles, conditions, et contrôle du flux
\def\ssensing{\textcolor{ssensing}{\faCircle\,Capteur}} % Réagir à des informations extérieures ou internes
\def\soperators{\textcolor{soperators}{\faCircle\,Opérateur}} % Calculs mathématiques et logiques
\def\svariables{\textcolor{svariables}{\faCircle\,Variable}} % Stockage et manipulation de données
\def\smyblocks{\textcolor{smyblocks}{\faCircle\,Mes blocs}} % Création de blocs personnalisés

\def\spen{{\icon{scratch/pen} Stylo}}
\def\spenExtension{{\icon{scratch/pen-extension} Stylo}}
\def\sextensions{{\icon{scratch/extensions} $\lbrack$ Ajouter une extensions $\rbrack$}}
\def\sflag{{\icon{scratch/flag}%
%  Drapeau
}}

% \setscratch{scale=.75}
% \setscratch{print=true}
% \setscratch{fill blocks=true}
\colorlet{gradeColor}{scratch}
\emptyBackground
%%



% \hint{
%     \begin{itemize}
%         \item Appeler M. Pesin à la fin de chaque partie.
%         \item Enregistrer les productions avec le nom : "NOM.S-tp-polygones-partie-X.ggb"
%     \end{itemize}
% }

\Scratch est un logiciel de programmation par \block, il va t'aider à découvrire l'algorithmique.

\section{Découvrir l'interface}

\begin{enumerate}
    \item Ouvre le logiciel \Scratch.
    \item Observe l'interface :
    \begin{enumerate}
        \item A gauche il y a des \block
        organisés par catégories : \smouvement, \sapparance, \ssound, etc.
        \item Au centre, il y a la \key{zone de scripts}, où tu pourras assembler les \block pour créer des algorithmes.
        \item À droite, tu vois la \key{scène} , où se déplacera ton personnage (appelé \key{lutin} ou sprite).  
    \end{enumerate}
    \item Teste le fonctionnement de  \Scratch : 
    \begin{itemize} 
        \item Glisse un bloc de la catégorie \smouvement
        (par exemple :
        \begin{scratch}\blockmove{avancer de \ovalnum{10}}\end{scratch}
        ) dans la \key{zone de scripts}. 
        \item Clique dessus pour voir le lutin bouger. 
        \item Change la valeur dans le bloc
        (par exemple :
        \begin{scratch}\blockmove{avancer de \ovalnum{100}}\end{scratch}
        ) et clique à nouveau. 
    \end{itemize}
\end{enumerate}

\hint{Si tu fais une erreur, tu peux supprimer un \block en le glissant vers la liste des \block.}

\section{Imbriquer des blocs}

\begin{enumerate}
    \item Places les \block suivants en les connectant dans cet ordre :
        \begin{scratch}
            \blockmove{avancer de \ovalnum{40}}
            \blocklook{dit \ovalnum{Bonjour !}}
            \blockcontrol{attendre \ovalnum{1} secondes}
            \blocklook{dit \ovalnum{Au revoir !}}
            \blockmove{avancer de \ovalnum{60}}
        \end{scratch}
        \hint{La couleur des \block correspond à leur catégorie.}
    \item Clique sur n'importe quel \block placer pour executer ton programme.
\end{enumerate}


\section{Utiliser l'extension Crayon pour tracer des formes}

\begin{enumerate}
    \item Clique sur l'icône \sextensions{} en bas à gauche pour ajouter une \key{extension}.
    \item Sélectionne l'extension \spenExtension.
    De nouveaux blocs, comme
    \begin{scratch}\blockpen{stylo en position d'écriture}\end{scratch}
    et
    \begin{scratch}\blockpen{relever le stylo}\end{scratch}
    seront ajoutés dans la nouvelle catégorie \spen .
    \item Teste les blocs suivants :
    \begin{scratch}
        \blockpen{stylo en position d'écriture}
        \blockmove{avancer de \ovalnum{50}}
        \blockmove{tourner \turnright{} de \ovalnum{45} degrés}
        \blockmove{avancer de \ovalnum{50}}
    \end{scratch}
    % \item Observe ce qui se passe lorsque le crayon est baissé et que le lutin avance. Relève ensuite le crayon pour qu'il cesse de tracer.
\end{enumerate}

\hint{Pour mieux visualiser tes tracés :
tu peux ajuster la taille de ton lutin en utilisant l'option "Taille" située dans l'onglet "Sprite" sous la scène.
}

\section{Automatiser l'exécution}

\begin{enumerate}
    \item Glisse le bloc
    \begin{scratch}\blockinit{quand \greenflag est cliqué}\end{scratch}
    de la catégorie \sevent.
    % \begin{scratch}\blockevent{quand \flag est cliqué}\end{scratch}) dans la \key{zone de scripts}.
    \item Connecte ce bloc à une série d'actions, comme :
    % \begin{center}
        \begin{scratch}
            % \blockcustom{effacer tout}
            \blockmove{aller à x: \ovalnum{0} y: \ovalnum{0}}
            \blockpen{effacer tout}
            \blockpen{stylo en position d'écriture}
            \blockmove{avancer de \ovalnum{100}}
            \blockmove{tourner \turnright{} de \ovalnum{90} degrés}
            \blockmove{avancer de \ovalnum{100}}
        \end{scratch}
    % \end{center}
    \item Clique sur le \sflag{} pour exécuter l'algorithme.
\end{enumerate}

%%%
\setGrade{6e}
\evaluation{2}
%%%
% \def\imgPath{enseignement/6e/}

\seqEvaluation{3}{Géométrie plane - distance}{
    Tracer un segment de longueur donnée.
    /2,
    Placer le milieu d'un segment de longueur donnée.
    /4,
    Reconnaître{,} nommer{,} décrire{,} reproduire un cercle.
    /4,
    Déterminer le plus court chemin entre un point et une droite.
    /0%
}

\seqEvaluation{4}{Nombres}{
    Utiliser et représenter les grands nombres entiers.
    /3,
    Utiliser et représenter les nombres décimaux jusqu'à trois décimales.
    /4,
    Utiliser la division euclidienne.
    / 2,
    Résoudre des problèmes relevant des structures additives et multiplicatives en mobilisant une ou plusieurs étapes de raisonnement.
    / 2,
    Organiser un calcul en une seule ligne{,} utilisant si nécessaire des parenthèses.
    / 2%
}

\seqEvaluation{}{Compétences générales}{
    Écrire ses calculs
    /1,
    Rédiger des phrases réponses
    /2
}
% %%%
\setGrade{5e}
\evaluation{2}
[corr]
%%%
% \def\imgPath{enseignement/6e/}

\seqEvaluation{3}{Symétrie}{
    Comprendre l'effet des symétries (axiale et centrale) :
    conservation du parallélisme{,} des longueurs et des angles.
    /2,
    Identifier des symétries dans des frises{,} des pavages{,} des rosaces.
    /2%
}

\seqEvaluation{4}{Nombres Relatifs - Sommes et differences}{
    Traduire un enchaînement d'opérations à l'aide d'une expression avec des parenthèses.
    /3,
    Additionner et soustraire des nombres décimaux relatifs.
    /4,
    Résoudre des problèmes faisant intervenir des nombres décimaux relatifs et des fractions simples.
    / 2,
}

\seqEvaluation{}{Compétences générales}{
    Écrire ses calculs
    /1,
    Rédiger des phrases réponses
    /2
}

\exo{\tiersTemps Reconnaître des symétries}{
    \def\crossWidth{0.3mm}\def\crossSize{0.15}\def\nodeShift{0.25}
    \ctikz[1.25]{
        \draw[gray!40] (0,0) grid (6,6);
        % Nested loops to generate points with names
        \newcounter{i}\setcounter{i}{1}%
        \foreach \x in {1,...,5} {%
            \foreach \y in {1,...,5} {%
                % Generate names programmatically
                \pgfmathtruncatemacro{\charCode}{64+\thei} % Convert x to letter (A=1, B=2, etc.)
                \edef\pointName{\char\charCode} % Get the letter name
                \drawPoint{\pointName}{\x}{\y}%
                \stepcounter{i}%
            }
        }
    }
    \begin{enumerate}
        \item L'image du segment $[HR]$ par la symétrie de centre $N$ est : \awsr{le segment $[JT]$}.
        \item Le triangle $QUV$ est l'image de du triangle $SON$ par \awsr{la symetries de centre $R$}.
        \item Le point \awsr{$D$} est l'image de point P par la symetrie d'axe $(AG)$.
        \item L'image du quadrilatère $NXQL$ par la symétrie de centre $M$ est : \awsr{le quadrilatère $LBIN$}.
    \end{enumerate}
}[\sesa{5}{2024}[3][89]]

\exo{Utiliser les propriétés de la symétrie centrale}{
    On considère quatre points $P$, $O$, $K$ et $E$ tels que :  
    \begin{align*}
        KE = \Lg{10.1}, \quad PO = \Lg{3.6}, \quad EP = \Lg{2.6}, \quad PK = \Lg{5}.
    \end{align*}
    Les points $A$,$S$ et $H$ sont respectivement les images des points $P$,$E$ et $K$ par la symétrie centrale de centre $O$.
    Calculez le périmètre du triangle $ASH$.
}

% \begin{Geometrie}[TypeTrace="Schema"]
%     pair A,B,C;
%     A=u*(1,1);
%     B-A=u*(4,1);
%     C=rotation(A,B,-70);
%     trace polygone(A,B,C);
% \end{Geometrie}


\ctikz[0.75]{
    \draw[gray!40] (-1,-10) rectangle (18,2);
    \node at (3,1) {Schéma à main levée :};
    \ifthenelse{\boolean{answer}}{%
        \draw [penciline,thick] (0.48,-7.04)-- (7.4,-8.06);
        \draw [penciline,thick] (7.4,-8.06)-- (5.18,-4.68);
        \draw [penciline,thick] (5.18,-4.68)-- (0.48,-7.04);
        \draw [penciline,thick] (11.06,-3.16)-- (15.683121999999992,1.1780524000000023);
        \draw [penciline,thick] (15.683121999999992,1.1780524000000023)-- (9.047820799999997,0.5338484000000019);
        \draw [penciline,thick] (9.047820799999997,0.5338484000000019)-- (11.06,-3.16);
        \draw (6.825316999999991,-6.262503800000003) node[anchor=north west] {\Lg{2.6}};
        \draw (3.2821949999999944,-8.098485200000004) node[anchor=north west] {\Lg{10.1}};
        \draw (2.090417599999996,-5.038516200000002) node[anchor=north west] {\Lg{5}};
        \drawPoint{P}{5.18}{-4.68}
        \drawPoint{O}{8.16}{-4.22}
        \drawPoint{K}{0.48}{-7.04}
        \drawPoint{E}{7.40}{-8.06}
        \drawPoint{A}{11.06}{-3.16}
        \drawPoint{S}{9.05}{0.53}
        \drawPoint{H}{15.68}{1.18};
    }{}%
}


\answerFill[Réponse][
    \begin{itemize}
        \item On a : $PE + KE + KP = \Lg{2,6} + \Lg{10,1} + \Lg{5} = \Lg{17.7}$.
        \item Alors, le périmètre du triangle $PEK$ est de \Lg{17.7}.
        \item On sait que les points $A$, $S$ et $H$ sont respectivement les images des points $P$, $E$ et $K$ par la symétrie centrale de centre $O$.
        \item Alors, le triangle $ASH$ est l'image du triangle $PEK$ par cette symétrie.
        \item Or, la symétrie centrale conserve les longueurs.
        \item Alors, le périmètre du triangle $ASH$ est égal à celui du triangle $PEK$.
        \item Donc le périmètre du triangle $ASH$ est de \Lg{17.7}.
    \end{itemize}
]

\exo{Calculs}{
    \multiColEnumerate{1}{
        \item $\frac{81}{9} \times 5 -1
        = \awsr[2]{9\times 5 - 1 = 45 -1 = 44}$
        \item $\frac{45,5}{2\times3-1}
        = \awsr[2]{\frac{45,5}{6-1} = \frac{45,5}{5} = 9,1}$
        % \item $\frac{27}{2\times3}-1
        % = \awsr[3]{\frac{27}{6} -1 = 4,5 - 1}$
        % \item $\frac{17-5}{3}+2
        % = \awsr[3]{\frac{12}{3}+2 = 4 + 2 = 6}$
        \item $7\times\frac{15\times4}{3-2}+2\times8
        = \awsr[2]{7 \times \frac{60}{6}+ 16 = 7 \times 10 + 16 = 70 + 16 = 86}$
        \item $7 - (-\np{12}) = \awsr[2]{19}$
        \item $-9 + 6 - 78 = \awsr[2]{-3 - 78 = -81}$
        % \item $12 + (-\np{22,6}) = \awsr[2]{-10,6}$
        \item $39 + (7 - 18 + (-1)) = \awsr[2]{39 + (-11 + (-1)) = 39 + (-12) = 27}$
    }
}[\sesa{5}{2024}[10][41]]

% \exo{}{
%     \begin{enumerate}

%     \end{enumerate}
% }

\exo{La randonnée d'Élodie}{
    Élodie effectue une randonnée en montagne qui se déroule en plusieurs étapes :  
    \begin{itemize}
        \item Depuis son point de départ situé à \Lg{1800} d'altitude, elle grimpe de \Lg{560} pour atteindre un premier refuge.  
        \item Après une pause, elle descend de \Lg[km]{1,3} pour visiter un lac de montagne.  
        \item Ensuite, elle remonte de \Lg{230} pour reprendre le chemin principal.  
        \item Elle rejoint un second refuge situé à \Lg[km]{1,1} plus haut.  
        \item Enfin, le dernier jour, elle redescend de \Lg{860} jusqu'à un arrêt de bus.  
    \end{itemize}

    \begin{enumerate}
        \item Quelle est l'altitude maximale atteinte par Élodie pendant sa randonnée ? Et l'altitude minimale ?  
        \item Quel est le dénivelé total de sa randonnée, c'est-à-dire la différence d'altitude entre son point de départ et son point d’arrivée ?  
    \end{enumerate}
}

\answerFill[Réponse][]
% %%%
\setGrade{4e}
\evaluation{2}
% [corr]
%%%
% \def\imgPath{enseignement/6e/}

\seqEvaluation{3}{Nombres Relatifs}{
    Déterminer le produit et le quotient de nombres relatifs.
    /2,
    Déterminer la somme et la différence de nombres relatifs.
    /2,
    Déterminer le signe d'un produit comportant plusieurs facteurs relatifs.
    /1,
    Trouver les antécédents du carré d'un nombre donné.
    /1%
}

\seqEvaluation{4}{Théorème de pythagore - Contraposé et réciproque}{
    Montrer qu'un triangle n'est pas rectangle.
    /2,
    Montrer qu'un triangle est rectangle.
    /2,
    Déterminer si un triangle est rectangle.
    /1,
    Maîtriser les notions de contraposée et de réciproque. /0%
    }

\seqEvaluation{5}{Proportionnalité - Tableaux et graphiques}{
    Reconnaître un tableau de proportionnalité.
    /2,
    Compléter un graphique cartésien.
    /1,
    Reconnaître une situation de proportionnalité sur un graphique.
    /1%
}

\evalutionEnd[3][2]

% \newpage

% \exo{Calcul de 4e proportionnelle avec des nombres relatifs \tt}{
%     Trouver les valeurs manquantes des tableaux de proportionnalité ci-dessous :
%     \multiColEnumerate{3}{
%         \item \Propor[Simple,Math,Stretch=1.25,%
%         ]{-18/30,-6/\awsr{\np{10}}}
%         % \item \Propor[Simple,Math,Stretch=1.25,%
%         % ]{\awsr{\np{3}}/22,\np{1.5}/11}
%         \item \Propor[Simple,Math,Stretch=1.25,%
%         ]{\np{2.5}/-5,-60/\awsr{\np{120}},\awsr{\np{-360}}/720}
%     }
% }
    
% \answerFill[Calculs][%
%     \begin{enumerate}
%         \item On peut utiliser l'égalité des produits en croix pour calculer la quatrième proportionnelle :
%         $\frac{22\times\np{1.5}}{11} = \frac{33}{11} = 3$
%         % \item $\frac{30\times(-6)}{-18} = \frac{-180}{-18} = 10$
%         \item On a $\np{2,5} \times (-2) = -5$ alors $-2$ est le coefficient qui permet de passer de la première à la deuxième ligne.
%         On calcule alors : $-60 \times (-2) = 120$ et $720 \div (-2) = -360$
%     \end{enumerate}
% ]

\exo{Manipulation de relatifs}{
    \multiColEnumerate{1}{
        \item $[-3-(-7+5)] \times (-\np{0.5}) = \awsr[4]{
            [-3-2] \times (-\np{1.5})
            = -5 \times (-\np{1.5})
            = \np{7.5}
        }$
        \item Quel est le résultat d'un produit de $\np{4815162342}$ fracteurs égales à $-1$.\\
        \awsr[4]{
            $\np{4815162342}$ est pair donc le résultat est positif
            et multiplié $1$ par lui même donne $1$ donc le résultat est $1$.
        }
        \item Donner deux nombres dont le carré vaut $18$.
        \awsr[4]{
            $\sqrt{18}^2 = 18$ et $(-\sqrt{18})^2 = 18$.
            $18$ et $-18$ sont deux nombres dont le carré vaut $18$.
        }
        % \item $\frac{2-[5-3\times(2-4)]}{2-15\div5} = \awsr[3]{
        %     \frac{2-[5-3\times(-2)]}{2-3}
        %     = \frac{2-[5-6]}{-1}
        %     = \frac{2-[-1]}{-1}
        %     = \frac{3}{-1}
        %     = -3
        % }$
    }
}[\ching{4}{nombres-relatifs-operations}[$26$a et E.$35$a]]

\newpage

\exo{Déterminer si un triangle est rectangle}{
    Déterminer la nature de chacun des triangles ci-dessous.
    \ctikz[0.6]{
        \draw[gray!40] (-6,-4) rectangle (8,5);
        \draw [penciline, thick] (-3.2,3.86)-- (-0.84,-0.1);
        \draw [penciline,thick] (-0.84,-0.1)-- (-4.5,-1.86);
        \draw [penciline,thick] (-4.5,-1.86)-- (-3.2,3.86);
        \draw [penciline,thick] (-0.28,-2.28)-- (2.28,1.94);
        \draw [penciline,thick] (6.14,-2.72)-- (2.28,1.94);
        \draw [penciline,thick] (-0.28,-2.28)-- (6.14,-2.72);
        \draw (-4.86,1.04) node[anchor=north west] {6cm};
        \draw (-2.8,-1.22) node[anchor=north west] {5cm};
        \draw (-1.8,2.04) node[anchor=north west] {3cm};
        \draw (0.26,0.32) node[anchor=north west] {6dm};
        \draw (2.26,-2.78) node[anchor=north west] {8dm};
        \draw (4.46,0.14) node[anchor=north west] {10dm};
        \drawPoint{A}{-3.20}{3.86}
        \drawPoint{B}{-4.50}{-1.86}
        \drawPoint{C}{-0.84}{-0.10}
        \drawPoint{D}{-0.28}{-2.28}
        \drawPoint{E}{2.28}{1.94}
        \drawPoint{F}{6.14}{-2.72}
    }
}[\ching{4}{reciproque-pythagore}[$8$]]


\answerSec{14}[Triangle ABC][
    \Pythagore[Reciproque,Unite=cm]{ABC}{3}{5}{6}
]

\answerFill[Triangle EDF][
    \Pythagore[Reciproque,Unite=dm]{EDF}{10}{8}{6}
]

\exo{}{
    Chez Zoro, des tee-shirts sont en vente. Les prix normaux ainsi que les prix en période de soldes sont indiqués dans le tableau ci-dessous.
    \begin{enumerate}
        \begin{table}[h!]
            \centering
            \renewcommand{\arraystretch}{1.5} % Ajuste la hauteur des lignes
            \setlength{\tabcolsep}{8pt} % Ajuste l'espacement des colonnes
            \begin{tabular}{|>{\bfseries}c|*{7}{c|}} % Colonne en gras pour la première colonne
                \hline
                \rowcolor{gray!15} 
                Tee-shirts vendus & 1 & 2 & 3 & 4 & 5 & 6 & 7 \\ \hline
                Prix normal (en \euro) & 5 & 10 & 15 & 20 & 25 & 30 & 35 \\ \hline
                Prix soldé (en \euro)  & 5 & 10 & 12 & 17 & 22 & 24 & 29 \\ \hline
            \end{tabular}
        \end{table}
        \item Complétez le graphique cartésien ci-dessous en plaçant les points correspondant :
        \begin{itemize}
            \item en \textcolor{Blue}{bleu}, les points représentant les prix normaux ;
            \item en \textcolor{Red}{rouge}, les points représentant les prix en période de soldes.
        \end{itemize}
        \vspace{-1cm}
        \begin{center}
            \begin{tikzpicture}[yscale = 0.2, xscale = 1.5]
                \tkzInit[xmin=0,xmax=7.5,ymin=0,ymax=37]
                \tkzGrid[sub,color=gradeColor!50!white,subxstep=1,subystep=1]        
                \tkzLabelX[step=1]
                \tkzLabelY[step=5]
                \tkzDrawY[label={Prix (en \euro)}, above , step=5]
                \tkzDrawX[label={Tee-shirts vendus}, right, step=1]
                % Tracer les points
                \ifthenelse{\boolean{answer}}{
                    \foreach \x/\y/\z in {1/5/5, 2/10/10, 3/15/12, 4/20/17, 5/25/22, 6/30/24, 7/35/29}{
                        \drawPoint{}{\x}{\y}[Blue];
                        \drawPoint{}{\x}{\z}[Red];
                    }
                    \draw[Blue] (0,0) -- (7,35);
                    \draw[Red] (0,0) -- (2,10) -- (3,12) -- (4,17) -- (5,22) -- (6,24) -- (7,29);
                }{}
            \end{tikzpicture}
        \end{center}
        \item Les prix normaux et soldés sont-ils proportionnels au nombre de tee-shirts vendus ?
        Justifiez votre réponse à l'aide d'un argument graphique pour chaque cas.
    \end{enumerate}
}[\sesa{4}{2021}[5][61]]

\answerFill[Réponse][
    \begin{enumerate}\loadenumi[exo][1]
        \item \begin{itemize}
            \item Les points des prix des t-shirts non soldés sont alignés avec l'origine,
            il y a donc bien proportionnalité en période normale.
            \item Dans le cas des prix soldés,
            les points ne sont pas alignés,
            il n'y a donc pas proportionnalité.
        \end{itemize}
    \end{enumerate}
]

\exo{\tiersTemps Reconnaître un tableau de proportionnalité}{
    Les tableaux suivants présentent-ils des situations de proportionnalités ?
    \vspace{-1cm}
    \multiColEnumerate{2}{
        \item \Propor[Simple,Math,Stretch=1.25,%
        ]{-18/30,-6/10}
        % \item \Propor[Simple,Math,Stretch=1.25,%
        % ]{\awsr{\np{3}}/22,\np{1.5}/11}
        \item \Propor[Simple,Math,Stretch=1.25,%
        ]{\np{2.5}/-5,-60/120,-360/710}
    }
}

\answerSec{11}[Réponse][%
    \begin{enumerate}
        \item $-6 \times 30 = 180 = -18 \times 10$\\
        L'égalité des produits en croix étant vérifier,
        on a bien proportionnalité.
        \item $\frac{22\times\np{1.5}}{11} = \frac{33}{11} = 3$
        % \item $\frac{30\times(-6)}{-18} = \frac{-180}{-18} = 10$
        \item $\frac{-5}{\np{2.5}} = \frac{1}{2} = \frac{-360}{\np{720}} \neq \frac{-360}{\np{710}}$\\
        L'égalité des quotients n'étant pas vérifier, il n'y a pas proportionnalité. 
    \end{enumerate}
]

\exo{\bonus Contraposée et réciproque}{
    On s'intéresse à un quadrilatère.
    \begin{enumerate}  
        \item La proposition suivante est-elle vraie ?  
        «\Sialors{c'est un losange}{ses diagonales sont perpendiculaires.}»
        \item Écrivez la contraposée de cette proposition. Cette contraposée est-elle vraie ?
        \item Écrivez la réciproque de cette proposition. Cette réciproque est-elle vraie ? Justifiez votre réponse.  
    \end{enumerate}  
    
}

\answerFill[Réponse][%
    \begin{enumerate}
        \item La proposition est vraie.
        \item La contraposée de cette proposition est :
        «\Sialors{les diagonales ne sont pas perpendiculaires}{ce n'est pas un losange}».
        Cette contraposée est également vraie.
        \item La réciproque de cette proposition est :
        «\Sialors{les diagonales sont perpendiculaires}{c'est un losange}».
        Cette réciproque est fausse,
        car un cerf-volant est un quadrilatère dont les diagonales sont perpendiculaires sans être nécessairement un losange.
    \end{enumerate}
]

% % VARIABLES %%%
\setTitle{TEST}
\definecolor{gradeColor}{HTML}{29B65A} %#29B65A
%%%%%%%%%%%%%%%


\exo{\bonus Patates de solides}{
    De la même manière que ce qui a été fait en cours pour les polygones et les nombres :
    constituez des « patates » (ou ensembles) avec les solides vus en cours.
    Pensez à inclure des schémas illustrant vos différentes « patates ».
    \nswr[0]{
        \ctikz[0.6]{
    % \boundingBox[18][14][0.5pt][1][(-6,-5)]
    \draw [rotate around={0:(-2.60,-0.00)},thick] (-2.60,-0.00) ellipse (0.50cm and 0.29cm);
    \draw [rotate around={0:(-0.64,-0.21)},thick] (-0.64,-0.21) ellipse (0.50cm and 0.29cm);
    \draw [rotate around={0:(-0.64,0.79)},thick] (-0.64,0.79) ellipse (0.50cm and 0.29cm);
    \draw (-2.79,-0.25) node[anchor=north west] {cône};
    \draw (-1.90,-0.59) node[anchor=north west] {cylindre de révolution};
    \draw (-0.29,-2.31) node[anchor=north west] {sphère};
    \draw [thick] (-0.07,-1.82) circle (0.50cm);
    \draw [rotate around={0:(-0.08,-1.82)},thick] (-0.08,-1.82) ellipse (0.50cm and 0.29cm);
    \draw [rotate around={122.67:(-1.61,0.45)},thick] (-1.61,0.45) ellipse (4.48cm and 1.78cm);
    \draw (-4.47,3.61) node[anchor=north west] {Solides de révolution};
    \draw (6.49,2.85) node[anchor=north west] {pavés droits};
    \draw (6.00,-0.39) node[anchor=north west] {cube};
    \draw [rotate around={-115.63:(6.87,1.09)},thick] (6.87,1.09) ellipse (2.33cm and 1.78cm);
    \draw (6.97,-2.57) node[anchor=north west] {pyramide};
    \draw (5.43,7.02) node[anchor=north west] {prismes droits};
    \draw [rotate around={98.24:(6.77,2.86)},thick] (6.77,2.86) ellipse (4.53cm and 2.96cm);
    \draw [rotate around={141.45:(5.51,2.77)},thick] (5.51,2.77) ellipse (7.11cm and 5.02cm);
    \draw (1.99,8.28) node[anchor=north west] {polyèdres};
    \draw [rotate around={164.57:(3.28,2.24)},thick] (3.28,2.24) ellipse (9.23cm and 6.91cm);
    \draw [shift={(-2.60,6.50)},thick]  plot[domain=0:3.14,variable=\t]({1*0.50*cos(\t r)+0*0.50*sin(\t r)},{0*0.50*cos(\t r)+1*0.50*sin(\t r)});
    \draw [rotate around={2.60:(-2.60,6.50)},thick] (-2.60,6.50) ellipse (0.50cm and 0.25cm);
    \draw (-1.48,8.45) node[anchor=north west] {Solides};
    \draw [rotate around={0:(-2.96,2)},thick] (-2.96,2) ellipse (0.50cm and 0.29cm);
    \draw [rotate around={0.27:(-2.96,3.03)},thick] (-2.96,3.03) ellipse (0.25cm and 0.10cm);
    \draw [thick] (-1.14,0.79) -- (-1.14,-0.21);
    \draw [thick] (-0.14,0.79) -- (-0.14,-0.21);
    \draw [thick] (5.63,-0.25) -- (6.06,-0.50) -- (6.50,-0.25) -- (6.50,0.25) -- (6.06,0) -- (6.06,-0.50);
    \draw [thick] (5.63,-0.25) -- (5.63,0.25) -- (6.06,0);
    \draw [thick] (5.63,0.25) -- (6.06,0.50) -- (6.50,0.25);
    \draw [thick] (6.05,1.28) -- (7.35,0.53) -- (7.35,1.53) -- (6.05,2.28) -- (6.05,1.28);
    \draw [thick] (7.35,1.53) -- (7.78,1.78) -- (7.78,0.78) -- (7.35,0.53);
    \draw [thick] (7.78,1.78) -- (6.48,2.53) -- (6.05,2.28);
    \draw [thick] (7.99,-2.31) -- (8.42,-1.56) -- (8.86,-2.31) -- (8.86,-2.81) -- (8.42,-1.56);
    \draw [thick] (7.99,-2.31) -- (8.86,-2.81);
    \draw [thick] (6.48,5.45) -- (7.35,3.95) -- (8.21,3.45) -- (8.65,4.70) -- (6.48,5.45) -- (6.52,6.36) -- (7.35,4.95) -- (7.35,3.95);
    \draw [thick] (7.35,4.95) -- (8.21,4.45) -- (8.65,5.70) -- (6.52,6.36);
    \draw [thick] (8.21,4.45) -- (8.21,3.45);
    \draw [thick] (8.65,5.70) -- (8.65,4.70);
    \draw [thick] (1.21,4.75) -- (1.21,5.75) -- (1.65,5.50) -- (1.65,4.50) -- (3.38,4.50) -- (3.38,5.50) -- (1.65,5.50);
    \draw [thick] (2.08,6.25) -- (3.38,5.50) -- (2.08,5.75) -- (2.08,6.25) -- (1.65,6.50) -- (1.65,6.00) -- (2.08,5.75);
    \draw [thick] (2.08,6.25) -- (2.51,6.50) -- (2.08,6.75) -- (1.65,6.50);
    \draw [thick] (3.38,5.50) -- (2.51,6.50);
    \draw [thick] (1.65,4.50) -- (1.21,4.75);
    \draw [thick] (1.21,5.75) -- (1.65,6.00);
    \draw [thick] (-2.60,1) -- (-3.08,0.08);
    \draw [thick] (-2.60,1) -- (-2.12,0.08);
    \draw [thick] (-2.60,5) -- (-1.73,5.50) -- (-1.73,6.50) -- (-2.60,6) -- (-3.46,6.50) -- (-3.46,6) -- (-3.90,5.25) -- (-3.46,5) -- (-3.03,5.75) -- (-3.46,6);
    \draw [thick] (-2.60,6) -- (-2.60,5) -- (-3.03,5.25) -- (-3.46,5) -- (-3.46,5);
    \draw [thick] (-3.03,5.25) -- (-3.03,5.75);
    \draw [thick] (-3.46,6.50) -- (-3.03,6.75);
    \draw [thick] (-1.73,6.50) -- (-2.17,6.75);
    \draw [thick] (-3.46,2) -- (-3.21,3.03);
    \draw [thick] (-2.72,3.01) -- (-2.46,2);
}
    }
}
\end{document}