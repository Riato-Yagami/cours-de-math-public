%% BEAMER %%
% \documentclass[aspectratio=169, usenames,dvipsnames,xcolor=table]{beamer} \usepackage[fontsize=14pt]{fontsize}

\usepackage[T1]{fontenc}
\usepackage[french]{babel}

\usepackage[utf8]{inputenc}
\usepackage{amsmath}
\usepackage{amsthm}
\usepackage{amssymb}
\usepackage{graphicx}
\usepackage{dashundergaps}
\usepackage{array}
\usepackage{multicol}
\usepackage{wrapfig}
\usepackage{numprint}
\usepackage{ulem}
\usepackage{hyperref}
\usepackage{mathrsfs}
\usepackage{mathtools}
\usepackage[many]{tcolorbox}
\usepackage{xparse}
\usepackage{float}
\usepackage{lipsum}
\usepackage{pgf}
\usepackage{ifthen}
\usepackage{caption}
\usepackage{tikz}
\usepackage{xifthen}

% \usepackage[squaren,Gray]{SIunits}

% BREVET

% \usepackage{makeidx}
% \usepackage{fancybox}
% \usepackage{tabularx}
% \usepackage[normalem]{ulem}
% \usepackage{pifont}
% \usepackage{lscape}
% \usepackage{diagbox}
% \usepackage{multirow} 
% \usepackage{textcomp}
% \usepackage{scratch3}
% \usepackage[T1]{fontenc}
% \usepackage{fourier}
% \usepackage[french]{babel}
% \usepackage{pstricks}

% \usepackage[scaled=0.875]{helvet}
% \usepackage{pst-plot,pst-text,pst-tree,pstricks-add}

% fancyhdr

\setlength{\headheight}{18pt}
\fancyhead[C]{\normalsize \title}
% \renewcommand{\headrulewidth}{0pt} % Remove header line
\fancyhead[R]{}
\fancyfoot[L]{\author}
\fancyfoot[C]{\textbf{Page \thepage/\pageref{LastPage}}}
\fancyfoot[R]{\date}

\fancypagestyle{firstpage}{
    \setlength{\headheight}{29pt}
    \fancyhead[C]{\LARGE \title}
    \fancyhead[R]{}
    \fancyfoot[L]{\author}
    \fancyfoot[C]{\textbf{Page \thepage/\pageref{LastPage}}}
    \fancyfoot[R]{\date}
}

\thispagestyle{firstpage}

% \fancyfoot[C]{\textbf{Page 1/1}}

% HYPERREF

\hypersetup{
    colorlinks=true,       % false: boxed links; true: colored links
    linkcolor=red,          % color of internal links (change box color with linkbordercolor)
    citecolor=green,        % color of links to bibliography
    filecolor=magenta,      % color of file links
    urlcolor=blue,          % color of external links
    urlbordercolor=blue,    % borders of external links
    linkbordercolor=red,    % borders of internal links
    pdfborderstyle={/S/U/W 1}% border style will be underline of width 1pt
}

\usepackage[fontsize=14pt]{fontsize}

\usepackage[T1]{fontenc}
\usepackage[french]{babel}
\usepackage[utf8]{inputenc}

\frenchbsetup{StandardItemLabels=true}

% GLOBAL VARIABLES %%%
\graphicspath{{images}}
\def\cwidth{4cm}
\def\tspace{0.5cm}

% BOOLEAN %%%
\newboolean{anwser}
\newboolean{demonstration}
\newboolean{boxedProperties}
\newboolean{showID}
\newboolean{parenthisedID}
\newboolean{animated}
\newboolean{outline}

\setboolean{anwser}{false}
\setboolean{demonstration}{true}
\setboolean{parenthisedID}{true}
\setboolean{showID}{true}
\setboolean{boxedProperties}{false} % false = edge
\setboolean{outline}{false}

\def\DefinitionColor{PineGreen}
\def\PropertyColor{Blue}
\def\TheoremColor{Plum}

\def\SectionColor{Red}
\def\SubSectionColor{Green}

\setboolean{animated}{true}

% \DeclareMathOperator{\PGCD}{PGCD}
% \DeclareMathOperator{\PPCM}{PPCM}

\DeclareMathOperator{\sh}{sh}
\DeclareMathOperator{\ch}{ch}
% \DeclareMathOperator{\th}{th}

\DeclareMathOperator{\argsh}{argsh}
\DeclareMathOperator{\argch}{argch}
\DeclareMathOperator{\argth}{argth}
\DeclareMathOperator{\I}{I}
\DeclareMathOperator{\Id}{Id}
\DeclareMathOperator{\Ker}{Ker}
% \DeclareMathOperator{\dl}{o}
\newcommand{\dl}[1]{
    \operatorname*{o}_{#1}
}

\def\deg{\ensuremath{^\circ}}
\def\prll{\mathbin{\!/\mkern-5mu/\!}}
\renewcommand{\parallel}{\mathbin{\!/\mkern-5mu/\!}}

\def\octet{\textrm{o}}
\def\byte{\textrm{B}}

\def\hour{\textrm{h}}
\def\minute{\textrm{min}}
\def\second{\textrm{s}}
% ENVIRONMENT
\newenvironment{mysection}[1][gray!20]{%
    \begin{sectionBox}[#1]
}{%
    \end{sectionBox}
}

\newenvironment{mysubsection}[1][gray!20]{%
    \begin{subsectionBox}[#1]
}{%
    \end{subsectionBox}
}

% Switch implementation
\newboolean{default}
\newcommand{\case}{}
\newcommand{\default}{}

\newenvironment{switch}[1]{%
    \setboolean{default}{true}
    \renewcommand{\case}[2]{\ifthenelse{\equal{#1}{##1}}{%
        \setboolean{default}{false}##2}{}}%
    \renewcommand{\default}[1]{\ifthenelse{\boolean{default}}{##1}{}}
}{}

% SECTIONS
\input{header/command/sections.tex}

% ANSWERS
\newlength{\parline}
\newlength{\paroutindent}
\newlength{\lineheight}
\setlength{\lineheight}{\heightof{abcdefghijklmnoprstuvwxyz}}

\newcommand{\countlines}[1]{%
    \setlength{\paroutindent}{\expandafter\parindent}
    \setlength{\parline}{\heightof{\noindent\begin{minipage}{\linewidth}%
                \setlength{\parindent}{\paroutindent}#1\end{minipage}}}%
    \pgfmathparse{round(\parline / (0.9*\lineheight))}
    \newcount\linecount
    \pgfmathsetcount{\linecount}{\pgfmathresult}
}

\newcommand{\looptext}[2]{%
    \noindent
    \newcount\printcount
    \printcount=#2
    \loop
        #1
        \advance\printcount by -1
        \ifnum\printcount>0
    \repeat
}

\newcommand{\awsr}[1]{%
    \ifthenelse{\boolean{answer}}{
        \result{#1}
    }{
        \countlines{#1}
        \pgfmathsetcount{\linecount}{\linecount + 1}
        \noindent\hspace{-9pt}
        \looptext{
            \noindent\dotfill
    
        }{\the\linecount}
    }
}

\newcommand{\dottedLines}[1]{%
    \noindent\hspace{-9pt}%

    \looptext{%
        \noindent\dotfill%

    }{#1}
}

\newcommand{\result}[1]{\color{OrangeRed}#1\color{black}}

% MATH
\input{header/command/math.tex}

% IMAGES
\input{header/command/image.tex}

% COMMANDS

\newcommand{\fsize}[1]{\fontsize{#1}{#1}\selectfont}

\NewDocumentCommand{\ifNotNull}{mmo}{
    \IfValueT{#1}{
        \ifx\relax#1\relax
            \IfValueT{#3}{
                #3
            }
        \else
            #2
        \fi
    }
}

\NewDocumentCommand{\ilink}{m g}{%
    \item
    \IfValueTF{#2}{\link{#1}{#2}}{\link{#1}}
}

\NewDocumentCommand{\link}{m g}{%
    \csn{#1}%
    \IfValueT{#2}{(#2)}%
}

\NewDocumentCommand{\TODO}{g}{%
    {\color{Red} $\rightarrow$ \textbf{TODO}
    \IfValueT{#1}{(#1)}}
    % \color{black}
}

\newcommand{\leconInfoBox}[2]{
    \textbf{#1 :}\vspace{-0.25cm}
        \begin{multicols}{2}
            \begin{itemize}[label=$\blacktriangleright$, font = \small \color{Red}]
                #2
            \end{itemize}
        \end{multicols}
        \vspace{-0.4cm}
}

% TCOLORBOX

\input{header/command/tcolorbox.tex}

\NewDocumentCommand{\leconInfo}{mooo}{
    \begin{infoBox}
        \leconInfoBox{Niveaux}{#1}
        \ifNotNull{#2}{
            \tcbline
            \leconInfoBox{Prérequis}{#2}
        }
        \ifNotNull{#3}{
            \tcbline
            \leconInfoBox{Thèmes}{#3}
        }
        \ifNotNull{#4}{
            \tcbline
            \textbf{Motivation :} 
            #4
        }
    \end{infoBox}
}

\NewDocumentCommand{\seanceInfo}{oooooooo}{
    \begin{infoBox}
        \vspace{-0.05cm}
        \begin{tcbitemize}[raster rows=1,raster columns=20,raster height=1.65cm,
            raster every box/.style={colframe=red!50!black,colback=red!10!white}]
            \tcbitem[raster multicolumn=6] \textbf{Date :} #1
            \tcbitem[raster multicolumn=10] \textbf{Séquence :} #2
            \tcbitem[raster multicolumn=4] \textbf{Séance :} #3
        \end{tcbitemize}
        \vspace{-0.25cm}
        \ifNotNull{#4}{\tcbline \textbf{Objectif :} #4}
        \ifNotNull{#5}{\tcbline \leconInfoBox{Classe(s)}{#5}}
        \ifNotNull{#6}{\tcbline \leconInfoBox{Prérequi(s)}{#6}}
        \ifNotNull{#7}{\tcbline \textbf{Séance précédente :} #7}
        \ifNotNull{#7}{\tcbline \leconInfoBox{Matériel(s)}{#8}}
    \end{infoBox}
}

\def\pDscr{\tcbitem[enhanced jigsaw, breakable,
    raster multicolumn=6]
}
\def\pMdlt{\tcbitem[enhanced jigsaw, breakable,
    raster multicolumn=11]
}
\def\pTime{\tcbitem[enhanced jigsaw, breakable,
    raster multicolumn=3, halign=center]
}

\newcommand{\prepRow}[3]{
    \tcbitem[raster multicolumn=20]
    \tcblower

    \pDscr #1
    \pMdlt #2
    \pTime #3
}

\newcommand{\prepTable}[1]{
    \begin{prepBox}
        \begin{tcbitemize}[enhanced jigsaw, breakable, raster rows=1,raster columns=20,raster height=1.1cm, halign=center,
            raster every box/.style={enhanced jigsaw, breakable, colframe=Blue!50!black,colback=Blue!10!white}]
            \pDscr \textbf{Descriptif}
            \pMdlt \textbf{Modalité}
            \pTime \textbf{Durée}
        \end{tcbitemize}
        \begin{tcbitemize}[enhanced jigsaw, breakable,
            raster equal height = rows, 
            raster columns=20, frame hidden,
            raster every box/.style={
                enhanced jigsaw, breakable,
                opacityback=0, valign=top, 
                size = tight
            }]
            #1
        \end{tcbitemize}
    \end{prepBox}
}

% TIKZ

\newcommand{\ctikz}[1]{
    \begin{center}
        \begin{tikzpicture}
            #1
        \end{tikzpicture}
    \end{center}
}

\newcommand{\axis}[1]{%Draw coordinate axes
    \draw[thin, -Stealth] (-0.5,0) -- (#1,0);% node[right] {$x$}; % x-axis
    \draw[thin, -Stealth] (0,-0.5) -- (0,#1);% node[above] {$y$}; % y-axis
}

\newcommand{\drawGrid}[3]{
    \foreach \n in {0,...,#1}
        \draw[line width = #3] (\n,0) -- (\n,#2);
    \foreach \n in {0,...,#2}
        \draw[line width = #3] (0,\n) -- (#1,\n);
}

\newcommand{\drawPoint}[4]{
    \node[shift={#4}, color = \pointColor] at (#2 - 0.5,#3 - 0.5) {#1};
    \draw[line width = \crossWidth, shift={#4}, color = \pointColor] (#2 - 0.25,#3) -- (#2 + 0.25,#3);
    \draw[line width = \crossWidth, shift={#4}, color = \pointColor] (#2,#3 - 0.25) -- (#2,#3 + 0.25);
}

% Tabular
\newcolumntype{C}[1]{>{\centering\arraybackslash}p{#1}}
\newcolumntype{M}[1]{>{\centering\arraybackslash}m{#1}}
\newcolumntype{K}{@{}m{0pt}@{}}

% GEOMETRY

% \newcommand{\restoregeometry}{def}

\newcommand{\multiColItemize}[2]{
    \begin{multicols}{#1}
        \begin{itemize}
            #2
        \end{itemize}
    \end{multicols}
}

\newcommand{\multiColEnumerate}[2]{
    \ifthenelse{\isequivalentto{#1}{1}}{
        \begin{enumerate}
            #2
        \end{enumerate}
    }{
        \begin{multicols}{#1}
            \begin{enumerate}
                #2
            \end{enumerate}
        \end{multicols}
    }
}

\makeatletter
\newcommand\pgfinvisible{\pgfsys@begininvisible}
\newcommand\pgfshown{\pgfsys@endinvisible}
\makeatother

\renewcommand*{\phantom}[1]{
    \pgfinvisible #1 \pgfshown
}

\newcounter{size}
\newcommand{\listSize}[1]{%
    \setcounter{size}{0}%
    \foreach \n in {#1}{\stepcounter{size}}%
    % \thesize
}

\newcounter{elemPos}
\newcommand{\listElement}[2]{
    \setcounter{elemPos}{0} % Start counting from 1
    \def\resultVal{0} % Default value
    \renewcommand*{\do}[1]{%
        \ifnumequal{\value{elemPos}}{#2}{%
            \def\resultVal{##1}%
            \listbreak% Break out of the loop
        }{}%
        \stepcounter{elemPos}%
    }
    % \docsvlist{#1}
    \expandafter\docsvlist\expandafter{#1} % Expand the list before passing it to \docsvlist
    \resultVal
}

% \NewDocumentCommand{\exoslide}{m O{10cm}}{
%     \slide{}{
%         \img{\imgf{#1}}[#2]
%     }
% }

\NewDocumentCommand{\exoSlide}{m O{10cm} O{1} O{} O{exo}}{%
    \slide{#5}{%
        \ifthenelse{\equal{#3}{1}}{\vspace{-0.5cm}}{\vspace{-1cm}}
        \def\exercices{\foreach \q in {#1}{\imgp{\q}[#2]\vspace{-0.5cm}}}
        \exo{#1}{\wideFrame[7em]{\bvspace{0.25cm}\avspace{-0.25cm}
            \ifthenelse{\equal{#3}{1}}{\exercices}
            {\begin{multicols}{#3}\exercices\end{multicols}}}
            \avspace{0.75cm}
        }[#4]
    }
}

\NewDocumentCommand{\exoList}{m O{} O{3}}{%
    \section*{Exercices}%
    \slide{EXERCICES}{
        \exo{#2}{
            \vspace{-0.25cm}
            \multiColEnumerate{#3}{
                \foreach \q in {#1}{
                    \item \q
                }
            }
        }
    }
}

\newcommand{\questions}[1]{
    \begin{enumerate}
        \foreach \q in {#1}{
            \item \q\\
            \vspace*{-0.45cm}
            \dottedLines{3}
        }
    \end{enumerate}
}

% Define a new boolean for checking if the section is starred
\newboolean{section@star}

\makeatletter
% Redefine \section and \section* to set the boolean
\let\old@section\section
\renewcommand{\section}{%
    \@ifstar
        {\setboolean{section@star}{true}\old@section*}
        {\setboolean{section@star}{false}\old@section}%
}
\makeatother

\newcommand{\qt}[1]{«\textit{#1}»}

\newcommand{\calc}[1]{\numexpr#1\relax}
\newcommand{\ncalc}[1]{\number\calc{#1}}
\newcommand{\pcalc}[1]{\numprint{\ncalc{#1}}}

\newcommand{\setgrade}[1]{
    \def\grade{#1}
    % \begin{switch}{#1}
    %     \case{6e}{\global\definecolor{gradeColor}{hex}{FA8072}}
    %     \default{
    %         Default
    %         \global\definecolor{gradeColor}{RGB}{200, 50, 50}
    %     }
    % \end{switch}
    \ifthenelse{\equal{#1}{6e}}{
        \definecolor{gradeColor}{HTML}{C6233D} % FA8072 in hex
    }{
    \ifthenelse{\equal{#1}{5e}}{
        \definecolor{gradeColor}{HTML}{088255}
    }{
    \ifthenelse{\equal{#1}{4e}}{
        \definecolor{gradeColor}{HTML}{1466A8}
    }{
    \ifthenelse{\equal{#1}{3e}}{
        \definecolor{gradeColor}{HTML}{844499}
    }{
        \definecolor{gradeColor}{RGB}{0, 0, 0}
    }}}}
}

\gdef\phase{}
\newcommand{\setPhase}[1]{%
    \begin{switch}{#1}
        \case{exo}{\gdef\phase{EXERCICES}}
        \case{cr}{\gdef\phase{COURS}}
        \case{qf}{\gdef\phase{QUESTIONS FLASH}}
        \case{dm}{\gdef\phase{DEVOIR MAISON}}
        \default{\gdef\phase{#1}}
    \end{switch}
}

\newcounter{savedenumi}
\setcounter{savedenumi}{0}
\xdef\savedenumi{0}
% \newcommand{\saveenumi}{
%     % \xdef\savedenumi{\calc{\theenumi-1}}
%     \setcounter{savedenumi}{0}
% }

\newcommand{\saveenumi}[1]{
    \setcounter{savedenumi}{#1}
}

\newcommand{\loadenumi}{
    \setItemColor{\currentColor}
    \setcounter{enumi}{\thesavedenumi}
}

\newcommand\csn[1]{\csname #1\endcsname}

\newcommand{\vect}[1]{\ensuremath{\overrightarrow{#1}}}
% \newcommand{\vect}[1]{\overrightarrow{\,\mathstrut#1\,}}
\newcommand{\m}[1]{\ensuremath{\mathbf{#1}}}
\newcommand\lm[2]{\lim_{#1\to#2}}

\def\eqv{\Leftrightarrow}
\def\ssi{si et seulement si }
\def\pt{pour tout }
\def\poly2{fonction polynôme du second degré }
\def\eq2{équation second degré }
\def\discr{b^2-4ac}

% MATH TEXT
\def\et{\textrm{ et }}
\def\si{\textrm{ si }}
\def\avec{\textrm{ avec }}
\def\car{\textrm{ car }}
\def\alors{\textrm{ alors }}
\def\ou{\textrm{ ou }}
\def\ona{\textrm{ on a }}

\def\iet{\shortintertext{et}}
\def\ialors{\shortintertext{alors}}
\def\idou{\shortintertext{d'où}}
\def\ior{\shortintertext{or}}
\def\iona{\shortintertext{on a}}

\def\studentinfo{
    \vspace*{-1cm}
    \begin{minipage}{0.35\linewidth}
        nom: \dotfill
    \end{minipage}
    \begin{minipage}{0.35\linewidth}
        prénom: \dotfill
    \end{minipage}
    \begin{minipage}{0.15\linewidth}
        classes: \dotfill
    \end{minipage}
    
    \noindent\hrulefill
}

% UNITS
\def\cm{\,\centi\meter}
\def\km{\,\kilo\meter}
\newcommand{\defl}[2]{%
    \expandafter\def\csname #1\endcsname{\href{#2}{#1}\space}%
}

% Page Eduscol
\defl{Eduscol Cycle 3}{https://eduscol.education.fr/251/mathematiques-cycle-3}
\defl{Eduscol Cycle 4}{https://eduscol.education.fr/280/mathematiques-cycle-4}
\defl{Eduscol Lycée Général et technologique}{https://eduscol.education.fr/1723/programmes-et-ressources-en-mathematiques-voie-gt}
\defl{Eduscol Lycée Professionnel}{https://eduscol.education.fr/1793/programmes-et-ressources-en-mathematiques-voie-professionnelle}

% Repères annuels
\defl{Cycle 3}{https://eduscol.education.fr/document/14026/download}
\defl{Cycle 4}{https://eduscol.education.fr/document/14080/download}

% Attendus de fin d'année
\defl{5e}{https://eduscol.education.fr/document/14044/download}
\defl{4e}{https://eduscol.education.fr/document/14056/download}
\defl{3e}{https://eduscol.education.fr/document/14068/download}

% Programme de mathématiques
\defl{cycle 3}{https://eduscol.education.fr/document/50990/download}
\defl{cycle 4}{https://cache.media.education.gouv.fr/file/31/89/1/ensel714_annexe3_1312891.pdf}
\defl{2nd}{https://eduscol.education.fr/document/24553/download}
\defl{1re}{https://eduscol.education.fr/document/24565/download}
\defl{1re STL}{https://eduscol.education.fr/document/23098/download}
\defl{1re STI2D}{https://eduscol.education.fr/document/24919/download}
\defl{Terminale Option Spécialité}{https://eduscol.education.fr/document/24568/download}
\defl{Terminale Option Complémentaire}{https://eduscol.education.fr/document/24571/download}
\defl{Terminale Option Expertes}{https://eduscol.education.fr/document/24574/download}
\defl{Terminale STL}{https://eduscol.education.fr/document/23107/download}
\defl{Terminale STI2D}{https://eduscol.education.fr/document/24922/download}
% Ressources thématiques
\defl{Proportionnalité}{https://eduscol.education.fr/document/17281/download}
\defl{Probabilités}{https://eduscol.education.fr/document/17275/download}
\defl{Fonctions}{https://eduscol.education.fr/document/17287/download}
\defl{Traitement des données}{https://eduscol.education.fr/document/17269/download}

\defl{Fonctions}{https://eduscol.education.fr/document/17287/download}
\defl{Fractions}{https://eduscol.education.fr/document/17239/download}
\defl{Nombres relatifs}{https://eduscol.education.fr/document/17245/download}
\defl{Puissances}{https://eduscol.education.fr/document/17251/download}
\defl{Divisibilité et nombres premiers}{https://eduscol.education.fr/document/17257/download}
\defl{Calcul littéral}{https://eduscol.education.fr/document/17263/download}

\defl{Grandeurs et mesures}{https://eduscol.education.fr/document/17293/download}
\defl{Algorithmique et programmation}{https://eduscol.education.fr/document/17311/download}

\defl{Suites}{https://eduscol.education.fr/document/24586/download}
\defl{Produit Scalaire}{https://eduscol.education.fr/document/24589/download}
\defl{Raisonnement et démonstration (seconde)}{https://eduscol.education.fr/document/24580/download}
\defl{Raisonnement et démonstrations (première)}{https://eduscol.education.fr/document/24583/download}

\captionsetup{labelformat=empty,labelsep=none}

% \setboolean{boxedProperties}{true} % false = edge
% \setboolean{parenthisedID}{false}
% \setboolean{showID}{false}

% \def\DefinitionColor{Red}
\def\PropertyColor{Red}
\def\TheoremColor{Red}

% TIKZ
\def\crossWidth{0.25mm}
\def\pointColor{blue}

\usepackage{bookmark}
% THEMES
% http://mcclinews.free.fr/latex/beamergalerie/completsgalerie.html
% default
% \usetheme{Madrid}
% \usetheme{CambridgeUS}

% tree
% \usetheme{Montpellier}
% \usetheme{Juanlespins}

\newcommand{\headerBox}[2]{
    \begin{beamercolorbox}[wd=\paperwidth,ht=2.125ex,dp=1.125ex,leftskip=.3cm,rightskip=.3cm plus1fil]{#1}%
        \usebeamerfont{#1}#2%
    \end{beamercolorbox}
}

\usetheme{Antibes}
\setbeamertemplate{headline}
{%  
    \headerBox{title in head/foot}{
        \insertshorttitle
        \hfill
        \color{links}\insertauthor
    }
    \ifx\insertsectionhead\empty\else
    \headerBox{section in head/foot}{
        \hskip6pt \ifthenelse{\boolean{section@star}}{$\rightarrow$}{\Roman{sec}.} \insertsectionhead
        \hfill
        \insertframenumber{} / \inserttotalframenumber
    }
    \fi
    \ifx\insertsubsectionhead\empty\else
    \headerBox{subsection in head/foot}{
        \hskip12pt \thesubsection{} \insertsubsectionhead
    }
    \fi
    \ifx\insertsubsubsectionhead\empty\else
    \headerBox{subsubsection in head/foot}{
        \hskip18pt \thesubsubsection{} \insertsubsubsectionhead
    }
    \fi
}

\setbeamerfont{frametitle}{size=\small,series=\bfseries}

\setbeamertemplate{frametitle}
{
    \vspace{-1.5pt} % Adjust the vertical space before the title
    \begin{beamercolorbox}[ht=2.5ex,dp=1.0ex,wd=\paperwidth,leftskip=.3cm,rightskip=.3cm]{frametitle}
        \usebeamerfont{frametitle}\insertframetitle
    \end{beamercolorbox}
}


% \setbeamertemplate{footline}
% {%
%     \leavevmode%
%     \hbox{%
%     \begin{beamercolorbox}[wd=.5\paperwidth,ht=2.25ex,dp=1ex,left]{author in head/foot}%
%         \usebeamerfont{author in head/foot}\hspace*{2ex}\insertauthor
%     \end{beamercolorbox}%
%     \begin{beamercolorbox}[wd=.5\paperwidth,ht=2.25ex,dp=1ex,right]{title in head/foot}%
%         \usebeamerfont{title in head/foot}\insertframenumber{} / \inserttotalframenumber\hspace*{2ex}
%     \end{beamercolorbox}}%
%     \vskip0pt%
% }

% lateral
% \usetheme{Hannover}

% navigation
% \usetheme{Frankfurt}

% sections and sub
% \usetheme{Warsaw}

% \usetheme{shadow}
% \usetheme{AnnArbor}

% color
% \usecolortheme{beaver}
\usecolortheme{spruce}
% \definecolor{UBCblue}{rgb}{0.04706, 0.13725, 0.26667} % UBC Blue (primary)
% \definecolor{UBCgrey}{rgb}{0.3686, 0.5255, 0.6235} % UBC Grey (secondary)
\setbeamercolor{palette primary}{bg=gradeColor!20,fg=gradeColor}
\setbeamercolor{palette secondary}{bg=gradeColor!95,fg=white}
\setbeamercolor{palette tertiary}{bg=gradeColor!70,fg=white}
\setbeamercolor{palette quaternary}{bg=gradeColor!60,fg=white}

% \setbeamercolor{structure}{fg=gradeColor} % itemize, enumerate, etc
% \setbeamercolor{section in toc}{fg=gradeColor} % TOC sections


% \usecolortheme[named=gradeColor]{structure}

\definecolor{links}{HTML}{e6ffe6}
\definecolor{hyperlinks}{HTML}{e6ffe6}
\hypersetup{
    colorlinks = true,
    linkcolor = links, % Apply the color to internal links
    urlcolor = hyperlinks   % Apply the color to URLs
}

% \setbeamercolor{item}{fg=ForestGreen}
\setbeamercolor{item}{fg=MidnightBlue}

\setbeamersize{
    text margin left=1.5cm,
    text margin right=1.5cm
}

\setbeamercovered{transparent = 25}
\setbeamertemplate{navigation symbols}{}

% \setbeamertemplate{enumerate subitem}{(\alph{enumii})}

% \setbeamertemplate{enumerate subitem}[square]
% \setbeamertemplate{enumerate items}[default]

\newcommand*{\setItemColor}[1]{
    \setbeamercolor{item}{fg=#1}
}
\def\authors{Jules PESIN}
\def\longTitle{long Title}
\def\shortTitle{short Title}
% \def\day{XX/XX/XX}

\title[\shortTitle]{\longTitle}
% \date{\day}

\newcommand{\slide}[2]{
    \begin{frame}
    \frametitle[#1]{#1}
        #2
    \end{frame}
}

\newcounter{sec}
% \stepcounter{sec}
\newcounter{subsec}
% \stepcounter{subsec}

% \newcommand{\bchap}[1]{
%     \color{Red} CHAPITRE : #1\color{black}\\
% }

\newcommand{\bseq}[1]{
    \def\sseq{\color{Red} CHAPITRE \theseq{} : #1\color{black}\\}
    \def\shortTitle{\MakeUppercase{#1}}
    \def\theme{#1}
    \setcounter{sec}{0}
}

\newcommand{\bsec}[1]{
    \section{#1}
    \def\ssec{\color{Red} \Roman{sec}. #1\color{black}\\}
    \stepcounter{sec}
    \setcounter{subsec}{0}
}

\newcommand{\bsubsec}[1]{
    \subsection{#1}
    \def\ssubsec{\color{Green} \thesubsec) #1\color{black}\\}
    \stepcounter{subsec}
}

\newcommand{\palt}[2]{
    \alt<#1->{\result{#2}}{\phantom{#2}}
    % \alt<#1->{\result{#2}}{\pgfinvisible #2 \pgfshown}
    % \alt<#1->{\result{#2}}{\textcolor{white}{#2}}
    % \alt<#1->{#2}{\awsr{#2}}
}

\NewDocumentCommand{\aalt}{O{2} m m}{%
    \alt<#1>{#2}{#3}
}

\newcounter{question}

\newcommand{\startQuestions}{
    \setcounter{question}{2}
}

\newcommand{\iquestion}[2]{
    \item $\question{#1}{#2}$
}

\newcommand{\question}[2]{
    #1 = \palt{\thequestion}{#2}
    \stepcounter{question}
}

% \renewcommand{\question}[2]{
%         #1 = #2
% }


\newcommand{\disableAnimation}{
    % \renewcommand{\question}[2]{
    %     ##1 = ##2
    % }
    
    \renewcommand{\palt}[2]{
        \result{##2}
    }

    \renewcommand{\pause}{}
}

\newcommand{\shortAnimation}{
    \renewcommand{\palt}[2]{
        % \alt<2->{\result{#2}}{\phantom{#2}}
    }
}

\newcommand{\firstSlide}{
    % \renewcommand{\question}[2]{
    %     ##1 =
    % }

    \renewcommand{\palt}[2]{
        \phantom{##2}
        % \pgfinvisible ##2 \pgfshown
    }
}

\newcounter{timer}
\NewDocumentCommand{\qf}{m O{15}}{
    \setcounter{qf}{0}
    \slide{EXERCICES}{\qfSUB{}{
        \begin{itemize}
            \item \large $#2\sec$ par question
            \item \listSize{#1}\thesize{} questions
        \end{itemize}
    }}
    \foreach \q in {#1}{
        \stepcounter{qf}
        \setcounter{choice}{1}
        % Timer slides
        \setcounter{timer}{#2}
        \whiledo{\thetimer>0}{
            \addtocounter{timer}{-1}
            \slide{QUESTIONS FLASH}{
                % \hspace{0.25cm}
                \large \color{Blue}\theqf.\color{black}
                \hspace*{-1cm} \huge \listElement{\q}{0}\\
                \ifthenelse{\boolean{qftimer}}{
                    \vspace{1cm}
                    \transduration{1}
                    \centering
                    \normalsize \color{CadetBlue}$\thetimer\sec$
                }{
                    \transduration{#2}
                    \setcounter{timer}{0}
                }
            }
        }
    }

    \slide{QUESTIONS FLASH}{
        \qfRes{#1}
    }
}

\NewDocumentCommand{\dividePage}{mm O{0.5}}{
    \pgfmathparse{1-#3}
    \begin{columns}[T]
        \begin{column}{#3\textwidth}
            #1
        \end{column}
        \begin{column}{\pgfmathresult\textwidth}
            #2
        \end{column}
    \end{columns}
}

%% Article %%
\documentclass[a4paper, 12pt, 
% landscape
]{extarticle} \usepackage[top=1.5cm, bottom=2cm, left=2cm, right=2cm]{geometry}
\usepackage[dvipsnames, table]{xcolor}
\usepackage{lastpage}
\usepackage{fancyhdr}
\usepackage{titlesec}
\usepackage{enumitem}
\usepackage{longtable}
\usepackage{pdfpages}
% FANCYHDR

\def\background{
    \fancyhead[L]{
        \begin{tikzpicture}[overlay]
            \fill[gradeColor!65] (-3cm,-\paperheight) rectangle (-1cm,2cm);
        \end{tikzpicture}%
    }
    \fancyhead[R]{
        \begin{tikzpicture}[overlay]
            \node[anchor=north east, font=\fontsize{40}{36}\selectfont] at (1.81cm,0.1cm + 0.55\headheight)
            {\hypersetup{urlcolor=gradeColor!65}\link{\grade}};
        \end{tikzpicture}%
    }
}

\def\emptyBackground{
    \fancyhead[L]{}
    \fancyhead[R]{}
}

\setlength{\headheight}{18pt}
\fancyhead[C]{\normalsize \title}
\background
\fancyfoot[L]{\authors}
\fancyfoot[C]{$\textbf{Page}\;\mathbf{\thepage / {\hypersetup{linkcolor=black}\pageref{LastPage}}}$}
\fancyfoot[R]{\date}

\fancypagestyle{firstpage}{
    \setlength{\headheight}{29pt}
    \fancyhead[C]{\LARGE \title}
}

\def\assignmentNameWidth{7.5cm}
\fancypagestyle{assignment}{
    \setlength{\headheight}{29pt}
    \fancyhead[C]{}
    \fancyhead[L]{\large \title}
    \fancyhead[R]{%
        \begin{tabular}{p{\assignmentNameWidth}p{2.5cm}}%
            \normalsize nom:& \normalsize classe: \link{\grade}\_\\%
            \normalsize prénom:& \normalsize date:\\%
            % \normalsize date:\hspace*{3.5cm}%
        \end{tabular}%
    }
}

\fancypagestyle{empty}{
    \renewcommand{\headrulewidth}{0pt}
    \setlength{\headheight}{-10pt}
    \fancyhead[C]{}
    \fancyhead[R]{}
    \fancyhead[L]{}
    \fancyfoot[L]{}
    \fancyfoot[C]{}
    \fancyfoot[R]{}
}

\fancypagestyle{empty-head}{
    \renewcommand{\headrulewidth}{0pt}
    \setlength{\headheight}{-10pt}
    \fancyhead[C]{}
    \fancyhead[R]{}
    \fancyhead[L]{}
}

\fancypagestyle{assignment-empty-foot}{
    \setlength{\headheight}{29pt}
    \fancyhead[C]{}
    \fancyhead[L]{\large \title}
    \fancyhead[R]{%
        \begin{tabular}{p{\assignmentNameWidth}p{0.15\pdfpagewidth}}%
            \normalsize nom:& \normalsize classe:\\%
            \normalsize prénom:& \normalsize date:\\%
            % \normalsize date:\hspace*{3.5cm}%
        \end{tabular}%
    }
    \fancyfoot[L]{}
    \fancyfoot[C]{}
    \fancyfoot[R]{}
}

\fancypagestyle{small}{
    \setlength{\headheight}{20pt}
    \fancyhead[C]{}
    \fancyhead[C]{\large \title}
    \fancyhead[L]{}
    \fancyhead[R]{}
    \fancyfoot[L]{}
    \fancyfoot[C]{}
    \fancyfoot[R]{}
}

\fancypagestyle{screenread}{
    \fancyhead[C]{}
    \fancyfoot[L]{}
    \fancyfoot[C]{}
    \fancyfoot[R]{}
    % \background
}

\thispagestyle{firstpage}

% \fancyfoot[C]{\textbf{Page 1/1}}

\def\title{\theme}
\def\authors{Jules PESIN}

\pagestyle{fancy}

% \titleformat*{\section}{\small\bfseries}

\titleformat{\section}
{\normalfont\large\bfseries\color{\SectionColor}}{\thesection}{0.6em}{}

\titleformat{\subsection}
{\normalfont\normalsize\bfseries\color{\SubSectionColor}}{\thesubsection}{0.6em}{}

\titleformat{\subsubsection}
{\normalfont\small\bfseries\color{\SubSubSectionColor}}{\thesubsubsection}{0.6em}{}


% \renewcommand{\theenumii}{.\arabic{enumii}}
% \frenchbsetup{StandardItemLabels=true}
\renewcommand{\theenumi}{\small\color{Blue}\arabic{enumi}}
\renewcommand{\labelenumii}{\scriptsize\color{RoyalBlue}\alph{enumii})}
% \renewcommand{\labelitemi}{$\color{Blue}.$}

\newcommand*{\setItemColor}[1]{
    \renewcommand{\theenumi}{\small\color{#1}\arabic{enumi}}
    \renewcommand{\labelenumii}{\scriptsize\color{#1}\alph{enumii})}
    \renewcommand{\labelitemi}{$\color{#1}\blacksquare$}
    \renewcommand{\labelitemii}{$\color{#1}\blacktriangleright$}
    \renewcommand{\labelitemiii}{$\color{#1}\bullet$}
}

\definecolor{gradeColor}{RGB}{200, 50, 50}

\backgroundsetup{
    scale=1,
    color=gradeColor,
    opacity=0.65,
    position=current page.south west,
    angle = 0,
    contents={%
        % Crée une bande colorée sur la gauche
        \begin{tikzpicture}[remember picture, overlay]
            \fill[gradeColor] (0,0) rectangle (1cm, \paperheight);
            % Texte en haut à droite
            \node[shift={(-0.32,-0.25)},anchor=north east, color=gradeColor, font=\fontsize{40}{36}\selectfont] 
                at (current page.north east) {6e};
        \end{tikzpicture}%
    }
}

\newcommand{\emptyBackground}{\backgroundsetup{
    position=current page.south west,
    angle=0,
    scale=1,
    color=gradeColor,
    hshift=0cm,  % No shift
    vshift=0cm,
    contents={}
}}

% BEAMER CONVERSION

% \newcommand{\bchap}[1]{\def\title{Chapitre: #1}}
\newcommand{\bseq}[1]{\def\title{Séquence: #1}}
\newcommand{\bsec}[1]{\section{#1}}
\newcommand{\bsubsec}[1]{\subsection{#1}}

\newcommand{\ssec}{}
\newcommand{\ssubsec}{}

\newcommand{\slide}[2]{#2}

\newcommand{\startQuestions}{}
\newcommand{\iquestion}[2]{\item $#1 = \result{#2}$}

\newcommand{\palt}[2]{\result{#2}}
\NewDocumentCommand{\aalt}{o m m}{%
    \noindent #2\\#3%
}

\newcommand{\disableAnimation}{}
\newcommand{\shortAnimation}{}

\newcommand{\firstSlide}{
    \renewcommand{\iquestion}[2]{\item $##1 = \phantom{##2}$}
    \renewcommand{\palt}[2]{
        \phantom{##2}
    }
}

\newenvironment{columns}[1][T]{}{}
\newenvironment{column}[1]{\begin{minipage}{#1}}{\end{minipage}}

\newcounter{qf}
\NewDocumentCommand{\qf}{m O{15}}{
    \qfSUB{}{
        \qfRes{#1}
    }
}

\newcounter{annex}
\renewcommand{\theannex}{\Alph{annex}} % Define how the annex counter will be displayed

\newcommand{\annex}[1]{%
    \changelocaltocdepth{0}
    \setcounter{section}{0}%
    \setcounter{subsection}{0}%
    \setcounter{subsubsection}{0}%
    \newpage%
    \fancyhead[L]{\color{Red} ANNEXE \theannex}
    \refstepcounter{annex}
    \label{annex:\theannex}
    \input{#1}
    \changelocaltocdepth{2}
}

\newcommand*{\rannex}[1]{
    (\hyperref[annex:#1]{Annexe #1})
}

\newcommand{\changelocaltocdepth}[1]{%
    \addtocontents{toc}{\protect\setcounter{tocdepth}{#1}}%
    \setcounter{tocdepth}{#1}%
}

\usepackage[fontsize=14pt]{fontsize}

\usepackage[T1]{fontenc}
\usepackage[french]{babel}

\usepackage[utf8]{inputenc}
\usepackage{amsmath}
\usepackage{amsthm}
\usepackage{amssymb}
\usepackage{graphicx}
\usepackage{dashundergaps}
\usepackage{array}
\usepackage{multicol}
\usepackage{wrapfig}
\usepackage{numprint}
\usepackage{ulem}
\usepackage{hyperref}
\usepackage{mathrsfs}
\usepackage{mathtools}
\usepackage[many]{tcolorbox}
\usepackage{xparse}
\usepackage{float}
\usepackage{lipsum}
\usepackage{pgf}
\usepackage{ifthen}
\usepackage{caption}
\usepackage{tikz}
\usepackage{xifthen}

% \usepackage[squaren,Gray]{SIunits}

% BREVET

% \usepackage{makeidx}
% \usepackage{fancybox}
% \usepackage{tabularx}
% \usepackage[normalem]{ulem}
% \usepackage{pifont}
% \usepackage{lscape}
% \usepackage{diagbox}
% \usepackage{multirow} 
% \usepackage{textcomp}
% \usepackage{scratch3}
% \usepackage[T1]{fontenc}
% \usepackage{fourier}
% \usepackage[french]{babel}
% \usepackage{pstricks}

% \usepackage[scaled=0.875]{helvet}
% \usepackage{pst-plot,pst-text,pst-tree,pstricks-add}

% fancyhdr

\setlength{\headheight}{18pt}
\fancyhead[C]{\normalsize \title}
% \renewcommand{\headrulewidth}{0pt} % Remove header line
\fancyhead[R]{}
\fancyfoot[L]{\author}
\fancyfoot[C]{\textbf{Page \thepage/\pageref{LastPage}}}
\fancyfoot[R]{\date}

\fancypagestyle{firstpage}{
    \setlength{\headheight}{29pt}
    \fancyhead[C]{\LARGE \title}
    \fancyhead[R]{}
    \fancyfoot[L]{\author}
    \fancyfoot[C]{\textbf{Page \thepage/\pageref{LastPage}}}
    \fancyfoot[R]{\date}
}

\thispagestyle{firstpage}

% \fancyfoot[C]{\textbf{Page 1/1}}

% HYPERREF

\hypersetup{
    colorlinks=true,       % false: boxed links; true: colored links
    linkcolor=red,          % color of internal links (change box color with linkbordercolor)
    citecolor=green,        % color of links to bibliography
    filecolor=magenta,      % color of file links
    urlcolor=blue,          % color of external links
    urlbordercolor=blue,    % borders of external links
    linkbordercolor=red,    % borders of internal links
    pdfborderstyle={/S/U/W 1}% border style will be underline of width 1pt
}

\usepackage[fontsize=14pt]{fontsize}

\usepackage[T1]{fontenc}
\usepackage[french]{babel}
\usepackage[utf8]{inputenc}

\frenchbsetup{StandardItemLabels=true}

% GLOBAL VARIABLES %%%
\graphicspath{{images}}
\def\cwidth{4cm}
\def\tspace{0.5cm}

% BOOLEAN %%%
\newboolean{anwser}
\newboolean{demonstration}
\newboolean{boxedProperties}
\newboolean{showID}
\newboolean{parenthisedID}
\newboolean{animated}
\newboolean{outline}

\setboolean{anwser}{false}
\setboolean{demonstration}{true}
\setboolean{parenthisedID}{true}
\setboolean{showID}{true}
\setboolean{boxedProperties}{false} % false = edge
\setboolean{outline}{false}

\def\DefinitionColor{PineGreen}
\def\PropertyColor{Blue}
\def\TheoremColor{Plum}

\def\SectionColor{Red}
\def\SubSectionColor{Green}

\setboolean{animated}{true}

% \DeclareMathOperator{\PGCD}{PGCD}
% \DeclareMathOperator{\PPCM}{PPCM}

\DeclareMathOperator{\sh}{sh}
\DeclareMathOperator{\ch}{ch}
% \DeclareMathOperator{\th}{th}

\DeclareMathOperator{\argsh}{argsh}
\DeclareMathOperator{\argch}{argch}
\DeclareMathOperator{\argth}{argth}
\DeclareMathOperator{\I}{I}
\DeclareMathOperator{\Id}{Id}
\DeclareMathOperator{\Ker}{Ker}
% \DeclareMathOperator{\dl}{o}
\newcommand{\dl}[1]{
    \operatorname*{o}_{#1}
}

\def\deg{\ensuremath{^\circ}}
\def\prll{\mathbin{\!/\mkern-5mu/\!}}
\renewcommand{\parallel}{\mathbin{\!/\mkern-5mu/\!}}

\def\octet{\textrm{o}}
\def\byte{\textrm{B}}

\def\hour{\textrm{h}}
\def\minute{\textrm{min}}
\def\second{\textrm{s}}
% ENVIRONMENT
\newenvironment{mysection}[1][gray!20]{%
    \begin{sectionBox}[#1]
}{%
    \end{sectionBox}
}

\newenvironment{mysubsection}[1][gray!20]{%
    \begin{subsectionBox}[#1]
}{%
    \end{subsectionBox}
}

% Switch implementation
\newboolean{default}
\newcommand{\case}{}
\newcommand{\default}{}

\newenvironment{switch}[1]{%
    \setboolean{default}{true}
    \renewcommand{\case}[2]{\ifthenelse{\equal{#1}{##1}}{%
        \setboolean{default}{false}##2}{}}%
    \renewcommand{\default}[1]{\ifthenelse{\boolean{default}}{##1}{}}
}{}

% SECTIONS
\input{header/command/sections.tex}

% ANSWERS
\newlength{\parline}
\newlength{\paroutindent}
\newlength{\lineheight}
\setlength{\lineheight}{\heightof{abcdefghijklmnoprstuvwxyz}}

\newcommand{\countlines}[1]{%
    \setlength{\paroutindent}{\expandafter\parindent}
    \setlength{\parline}{\heightof{\noindent\begin{minipage}{\linewidth}%
                \setlength{\parindent}{\paroutindent}#1\end{minipage}}}%
    \pgfmathparse{round(\parline / (0.9*\lineheight))}
    \newcount\linecount
    \pgfmathsetcount{\linecount}{\pgfmathresult}
}

\newcommand{\looptext}[2]{%
    \noindent
    \newcount\printcount
    \printcount=#2
    \loop
        #1
        \advance\printcount by -1
        \ifnum\printcount>0
    \repeat
}

\newcommand{\awsr}[1]{%
    \ifthenelse{\boolean{answer}}{
        \result{#1}
    }{
        \countlines{#1}
        \pgfmathsetcount{\linecount}{\linecount + 1}
        \noindent\hspace{-9pt}
        \looptext{
            \noindent\dotfill
    
        }{\the\linecount}
    }
}

\newcommand{\dottedLines}[1]{%
    \noindent\hspace{-9pt}%

    \looptext{%
        \noindent\dotfill%

    }{#1}
}

\newcommand{\result}[1]{\color{OrangeRed}#1\color{black}}

% MATH
\input{header/command/math.tex}

% IMAGES
\input{header/command/image.tex}

% COMMANDS

\newcommand{\fsize}[1]{\fontsize{#1}{#1}\selectfont}

\NewDocumentCommand{\ifNotNull}{mmo}{
    \IfValueT{#1}{
        \ifx\relax#1\relax
            \IfValueT{#3}{
                #3
            }
        \else
            #2
        \fi
    }
}

\NewDocumentCommand{\ilink}{m g}{%
    \item
    \IfValueTF{#2}{\link{#1}{#2}}{\link{#1}}
}

\NewDocumentCommand{\link}{m g}{%
    \csn{#1}%
    \IfValueT{#2}{(#2)}%
}

\NewDocumentCommand{\TODO}{g}{%
    {\color{Red} $\rightarrow$ \textbf{TODO}
    \IfValueT{#1}{(#1)}}
    % \color{black}
}

\newcommand{\leconInfoBox}[2]{
    \textbf{#1 :}\vspace{-0.25cm}
        \begin{multicols}{2}
            \begin{itemize}[label=$\blacktriangleright$, font = \small \color{Red}]
                #2
            \end{itemize}
        \end{multicols}
        \vspace{-0.4cm}
}

% TCOLORBOX

\input{header/command/tcolorbox.tex}

\NewDocumentCommand{\leconInfo}{mooo}{
    \begin{infoBox}
        \leconInfoBox{Niveaux}{#1}
        \ifNotNull{#2}{
            \tcbline
            \leconInfoBox{Prérequis}{#2}
        }
        \ifNotNull{#3}{
            \tcbline
            \leconInfoBox{Thèmes}{#3}
        }
        \ifNotNull{#4}{
            \tcbline
            \textbf{Motivation :} 
            #4
        }
    \end{infoBox}
}

\NewDocumentCommand{\seanceInfo}{oooooooo}{
    \begin{infoBox}
        \vspace{-0.05cm}
        \begin{tcbitemize}[raster rows=1,raster columns=20,raster height=1.65cm,
            raster every box/.style={colframe=red!50!black,colback=red!10!white}]
            \tcbitem[raster multicolumn=6] \textbf{Date :} #1
            \tcbitem[raster multicolumn=10] \textbf{Séquence :} #2
            \tcbitem[raster multicolumn=4] \textbf{Séance :} #3
        \end{tcbitemize}
        \vspace{-0.25cm}
        \ifNotNull{#4}{\tcbline \textbf{Objectif :} #4}
        \ifNotNull{#5}{\tcbline \leconInfoBox{Classe(s)}{#5}}
        \ifNotNull{#6}{\tcbline \leconInfoBox{Prérequi(s)}{#6}}
        \ifNotNull{#7}{\tcbline \textbf{Séance précédente :} #7}
        \ifNotNull{#7}{\tcbline \leconInfoBox{Matériel(s)}{#8}}
    \end{infoBox}
}

\def\pDscr{\tcbitem[enhanced jigsaw, breakable,
    raster multicolumn=6]
}
\def\pMdlt{\tcbitem[enhanced jigsaw, breakable,
    raster multicolumn=11]
}
\def\pTime{\tcbitem[enhanced jigsaw, breakable,
    raster multicolumn=3, halign=center]
}

\newcommand{\prepRow}[3]{
    \tcbitem[raster multicolumn=20]
    \tcblower

    \pDscr #1
    \pMdlt #2
    \pTime #3
}

\newcommand{\prepTable}[1]{
    \begin{prepBox}
        \begin{tcbitemize}[enhanced jigsaw, breakable, raster rows=1,raster columns=20,raster height=1.1cm, halign=center,
            raster every box/.style={enhanced jigsaw, breakable, colframe=Blue!50!black,colback=Blue!10!white}]
            \pDscr \textbf{Descriptif}
            \pMdlt \textbf{Modalité}
            \pTime \textbf{Durée}
        \end{tcbitemize}
        \begin{tcbitemize}[enhanced jigsaw, breakable,
            raster equal height = rows, 
            raster columns=20, frame hidden,
            raster every box/.style={
                enhanced jigsaw, breakable,
                opacityback=0, valign=top, 
                size = tight
            }]
            #1
        \end{tcbitemize}
    \end{prepBox}
}

% TIKZ

\newcommand{\ctikz}[1]{
    \begin{center}
        \begin{tikzpicture}
            #1
        \end{tikzpicture}
    \end{center}
}

\newcommand{\axis}[1]{%Draw coordinate axes
    \draw[thin, -Stealth] (-0.5,0) -- (#1,0);% node[right] {$x$}; % x-axis
    \draw[thin, -Stealth] (0,-0.5) -- (0,#1);% node[above] {$y$}; % y-axis
}

\newcommand{\drawGrid}[3]{
    \foreach \n in {0,...,#1}
        \draw[line width = #3] (\n,0) -- (\n,#2);
    \foreach \n in {0,...,#2}
        \draw[line width = #3] (0,\n) -- (#1,\n);
}

\newcommand{\drawPoint}[4]{
    \node[shift={#4}, color = \pointColor] at (#2 - 0.5,#3 - 0.5) {#1};
    \draw[line width = \crossWidth, shift={#4}, color = \pointColor] (#2 - 0.25,#3) -- (#2 + 0.25,#3);
    \draw[line width = \crossWidth, shift={#4}, color = \pointColor] (#2,#3 - 0.25) -- (#2,#3 + 0.25);
}

% Tabular
\newcolumntype{C}[1]{>{\centering\arraybackslash}p{#1}}
\newcolumntype{M}[1]{>{\centering\arraybackslash}m{#1}}
\newcolumntype{K}{@{}m{0pt}@{}}

% GEOMETRY

% \newcommand{\restoregeometry}{def}

\newcommand{\multiColItemize}[2]{
    \begin{multicols}{#1}
        \begin{itemize}
            #2
        \end{itemize}
    \end{multicols}
}

\newcommand{\multiColEnumerate}[2]{
    \ifthenelse{\isequivalentto{#1}{1}}{
        \begin{enumerate}
            #2
        \end{enumerate}
    }{
        \begin{multicols}{#1}
            \begin{enumerate}
                #2
            \end{enumerate}
        \end{multicols}
    }
}

\makeatletter
\newcommand\pgfinvisible{\pgfsys@begininvisible}
\newcommand\pgfshown{\pgfsys@endinvisible}
\makeatother

\renewcommand*{\phantom}[1]{
    \pgfinvisible #1 \pgfshown
}

\newcounter{size}
\newcommand{\listSize}[1]{%
    \setcounter{size}{0}%
    \foreach \n in {#1}{\stepcounter{size}}%
    % \thesize
}

\newcounter{elemPos}
\newcommand{\listElement}[2]{
    \setcounter{elemPos}{0} % Start counting from 1
    \def\resultVal{0} % Default value
    \renewcommand*{\do}[1]{%
        \ifnumequal{\value{elemPos}}{#2}{%
            \def\resultVal{##1}%
            \listbreak% Break out of the loop
        }{}%
        \stepcounter{elemPos}%
    }
    % \docsvlist{#1}
    \expandafter\docsvlist\expandafter{#1} % Expand the list before passing it to \docsvlist
    \resultVal
}

% \NewDocumentCommand{\exoslide}{m O{10cm}}{
%     \slide{}{
%         \img{\imgf{#1}}[#2]
%     }
% }

\NewDocumentCommand{\exoSlide}{m O{10cm} O{1} O{} O{exo}}{%
    \slide{#5}{%
        \ifthenelse{\equal{#3}{1}}{\vspace{-0.5cm}}{\vspace{-1cm}}
        \def\exercices{\foreach \q in {#1}{\imgp{\q}[#2]\vspace{-0.5cm}}}
        \exo{#1}{\wideFrame[7em]{\bvspace{0.25cm}\avspace{-0.25cm}
            \ifthenelse{\equal{#3}{1}}{\exercices}
            {\begin{multicols}{#3}\exercices\end{multicols}}}
            \avspace{0.75cm}
        }[#4]
    }
}

\NewDocumentCommand{\exoList}{m O{} O{3}}{%
    \section*{Exercices}%
    \slide{EXERCICES}{
        \exo{#2}{
            \vspace{-0.25cm}
            \multiColEnumerate{#3}{
                \foreach \q in {#1}{
                    \item \q
                }
            }
        }
    }
}

\newcommand{\questions}[1]{
    \begin{enumerate}
        \foreach \q in {#1}{
            \item \q\\
            \vspace*{-0.45cm}
            \dottedLines{3}
        }
    \end{enumerate}
}

% Define a new boolean for checking if the section is starred
\newboolean{section@star}

\makeatletter
% Redefine \section and \section* to set the boolean
\let\old@section\section
\renewcommand{\section}{%
    \@ifstar
        {\setboolean{section@star}{true}\old@section*}
        {\setboolean{section@star}{false}\old@section}%
}
\makeatother

\newcommand{\qt}[1]{«\textit{#1}»}

\newcommand{\calc}[1]{\numexpr#1\relax}
\newcommand{\ncalc}[1]{\number\calc{#1}}
\newcommand{\pcalc}[1]{\numprint{\ncalc{#1}}}

\newcommand{\setgrade}[1]{
    \def\grade{#1}
    % \begin{switch}{#1}
    %     \case{6e}{\global\definecolor{gradeColor}{hex}{FA8072}}
    %     \default{
    %         Default
    %         \global\definecolor{gradeColor}{RGB}{200, 50, 50}
    %     }
    % \end{switch}
    \ifthenelse{\equal{#1}{6e}}{
        \definecolor{gradeColor}{HTML}{C6233D} % FA8072 in hex
    }{
    \ifthenelse{\equal{#1}{5e}}{
        \definecolor{gradeColor}{HTML}{088255}
    }{
    \ifthenelse{\equal{#1}{4e}}{
        \definecolor{gradeColor}{HTML}{1466A8}
    }{
    \ifthenelse{\equal{#1}{3e}}{
        \definecolor{gradeColor}{HTML}{844499}
    }{
        \definecolor{gradeColor}{RGB}{0, 0, 0}
    }}}}
}

\gdef\phase{}
\newcommand{\setPhase}[1]{%
    \begin{switch}{#1}
        \case{exo}{\gdef\phase{EXERCICES}}
        \case{cr}{\gdef\phase{COURS}}
        \case{qf}{\gdef\phase{QUESTIONS FLASH}}
        \case{dm}{\gdef\phase{DEVOIR MAISON}}
        \default{\gdef\phase{#1}}
    \end{switch}
}

\newcounter{savedenumi}
\setcounter{savedenumi}{0}
\xdef\savedenumi{0}
% \newcommand{\saveenumi}{
%     % \xdef\savedenumi{\calc{\theenumi-1}}
%     \setcounter{savedenumi}{0}
% }

\newcommand{\saveenumi}[1]{
    \setcounter{savedenumi}{#1}
}

\newcommand{\loadenumi}{
    \setItemColor{\currentColor}
    \setcounter{enumi}{\thesavedenumi}
}

\newcommand\csn[1]{\csname #1\endcsname}

\newcommand{\vect}[1]{\ensuremath{\overrightarrow{#1}}}
% \newcommand{\vect}[1]{\overrightarrow{\,\mathstrut#1\,}}
\newcommand{\m}[1]{\ensuremath{\mathbf{#1}}}
\newcommand\lm[2]{\lim_{#1\to#2}}

\def\eqv{\Leftrightarrow}
\def\ssi{si et seulement si }
\def\pt{pour tout }
\def\poly2{fonction polynôme du second degré }
\def\eq2{équation second degré }
\def\discr{b^2-4ac}

% MATH TEXT
\def\et{\textrm{ et }}
\def\si{\textrm{ si }}
\def\avec{\textrm{ avec }}
\def\car{\textrm{ car }}
\def\alors{\textrm{ alors }}
\def\ou{\textrm{ ou }}
\def\ona{\textrm{ on a }}

\def\iet{\shortintertext{et}}
\def\ialors{\shortintertext{alors}}
\def\idou{\shortintertext{d'où}}
\def\ior{\shortintertext{or}}
\def\iona{\shortintertext{on a}}

\def\studentinfo{
    \vspace*{-1cm}
    \begin{minipage}{0.35\linewidth}
        nom: \dotfill
    \end{minipage}
    \begin{minipage}{0.35\linewidth}
        prénom: \dotfill
    \end{minipage}
    \begin{minipage}{0.15\linewidth}
        classes: \dotfill
    \end{minipage}
    
    \noindent\hrulefill
}

% UNITS
\def\cm{\,\centi\meter}
\def\km{\,\kilo\meter}
\newcommand{\defl}[2]{%
    \expandafter\def\csname #1\endcsname{\href{#2}{#1}\space}%
}

% Page Eduscol
\defl{Eduscol Cycle 3}{https://eduscol.education.fr/251/mathematiques-cycle-3}
\defl{Eduscol Cycle 4}{https://eduscol.education.fr/280/mathematiques-cycle-4}
\defl{Eduscol Lycée Général et technologique}{https://eduscol.education.fr/1723/programmes-et-ressources-en-mathematiques-voie-gt}
\defl{Eduscol Lycée Professionnel}{https://eduscol.education.fr/1793/programmes-et-ressources-en-mathematiques-voie-professionnelle}

% Repères annuels
\defl{Cycle 3}{https://eduscol.education.fr/document/14026/download}
\defl{Cycle 4}{https://eduscol.education.fr/document/14080/download}

% Attendus de fin d'année
\defl{5e}{https://eduscol.education.fr/document/14044/download}
\defl{4e}{https://eduscol.education.fr/document/14056/download}
\defl{3e}{https://eduscol.education.fr/document/14068/download}

% Programme de mathématiques
\defl{cycle 3}{https://eduscol.education.fr/document/50990/download}
\defl{cycle 4}{https://cache.media.education.gouv.fr/file/31/89/1/ensel714_annexe3_1312891.pdf}
\defl{2nd}{https://eduscol.education.fr/document/24553/download}
\defl{1re}{https://eduscol.education.fr/document/24565/download}
\defl{1re STL}{https://eduscol.education.fr/document/23098/download}
\defl{1re STI2D}{https://eduscol.education.fr/document/24919/download}
\defl{Terminale Option Spécialité}{https://eduscol.education.fr/document/24568/download}
\defl{Terminale Option Complémentaire}{https://eduscol.education.fr/document/24571/download}
\defl{Terminale Option Expertes}{https://eduscol.education.fr/document/24574/download}
\defl{Terminale STL}{https://eduscol.education.fr/document/23107/download}
\defl{Terminale STI2D}{https://eduscol.education.fr/document/24922/download}
% Ressources thématiques
\defl{Proportionnalité}{https://eduscol.education.fr/document/17281/download}
\defl{Probabilités}{https://eduscol.education.fr/document/17275/download}
\defl{Fonctions}{https://eduscol.education.fr/document/17287/download}
\defl{Traitement des données}{https://eduscol.education.fr/document/17269/download}

\defl{Fonctions}{https://eduscol.education.fr/document/17287/download}
\defl{Fractions}{https://eduscol.education.fr/document/17239/download}
\defl{Nombres relatifs}{https://eduscol.education.fr/document/17245/download}
\defl{Puissances}{https://eduscol.education.fr/document/17251/download}
\defl{Divisibilité et nombres premiers}{https://eduscol.education.fr/document/17257/download}
\defl{Calcul littéral}{https://eduscol.education.fr/document/17263/download}

\defl{Grandeurs et mesures}{https://eduscol.education.fr/document/17293/download}
\defl{Algorithmique et programmation}{https://eduscol.education.fr/document/17311/download}

\defl{Suites}{https://eduscol.education.fr/document/24586/download}
\defl{Produit Scalaire}{https://eduscol.education.fr/document/24589/download}
\defl{Raisonnement et démonstration (seconde)}{https://eduscol.education.fr/document/24580/download}
\defl{Raisonnement et démonstrations (première)}{https://eduscol.education.fr/document/24583/download}

\captionsetup{labelformat=empty,labelsep=none}

% \setboolean{boxedProperties}{true} % false = edge
% \setboolean{parenthisedID}{false}
% \setboolean{showID}{false}

% \def\DefinitionColor{Red}
\def\PropertyColor{Red}
\def\TheoremColor{Red}

% TIKZ
\def\crossWidth{0.25mm}
\def\pointColor{blue}

\begin{document}

\hfuzz=30pt

\ifBeamer{%
    \renewcommand*{\theenumii}{\alph{enumii}}

    \firstSlide
    \setboolean{showRef}{false}
}

\ifArticle{%
    \renewcommand*{\theenumii}{\alph{enumii}}
    
    \disableAnimation
}



% DOCUMENTS

% % VARIABLES %%%
\setSeq{4}{Nombres - Entiers}
\setGrade{6e}

\def\imgPath{enseignement/6e/nombres/entiers/}

\def\ym{\href{https://www.maths-et-tiques.fr/telech/19Nombres1.pdf}{Yvan Monka}}

% https://www.maths-et-tiques.fr/telech/19Nombres2.pdf

% \setboolean{newPageOnSlide}{true}
% \setboolean{answer}{false}
% \setboolean{debugMode}{true}
%%

\obj{
    \item Decomposer un nombre dans plusieurs bases.
    \item Conversion de durée.
    \item Utiliser et représenter les grands nombres entiers (en chiffres et en lettres).
    \item Utiliser la division euclidienne.
    \item Organiser un calcul en une seule ligne, utilisant si nécessaire des parenthèses.
    \item Savoir ce qu'est un ordre de grandeur et savoir l'utiliser pour prévoir un résultat.
}

\scn{Découvrir une numération préhistorique}

\qfSlide{
    \begin{enumerate}
        \item $2 + 3 \time 5 = $
        \item $6 - 2 + 3 - 1 =$
        \item $4 \times (6 + 3) =$
    \end{enumerate}
}

\bsec{Compter}
\bsubsec{Base 12}

\slide{exo}{\bvspace{-0.75cm}
    \act{Numération préhistorique}{
        Certains hommes préhistorique utilisaient leurs main pour communiquer sur des nombres.
        \imgp{historic/numbers}[6.5cm]
    }[\dmeepcS]
}

% \endinput

\slide{exo}{
    \begin{enumerate}\setItemColor{act}
        \item \multiColEnumerate{3}{
            \item $\Prehistoric{6} = \awsr{6}$
            \item $\Prehistoric{10} = \awsr{10}$
            \item $\icon{prehistoric-numbers/u/0}[60pt] = 7$
        }
        \item Quel est le nombre le plus grand pouvant etre communiquer de cette manière ?
        \item Comment pourait-on communiquer des nombres plus grands ?
        \saveenumi
    \end{enumerate}
}

\slide{exo}{
    \begin{enumerate}\loadenumi[act]
        \item \multiColEnumerate{2}{
            \item $\Prehistoric{12} = \awsr{12}$
            \item $\Prehistoric{15} = \awsr{15}$
            \item $\Prehistoric{50} = \awsr{50}$
            \item $\Prehistoric{0} = 58$
            \item $\Prehistoric{0} = 100$
            \item $\Prehistoric{135} = \awsr{135}$
        }
        \saveenumi
    \end{enumerate}
}

\slide{exo}{
    \begin{enumerate}\loadenumi[act]
        \item Quel est le plus grand nombre pouvant etre communiqué avec cette méthode ?
        \item Certains hommes préhistoriques comptaient donc par 12 car ils avaient 12 phalanges?
        De la même manière, par combien comptons-nous ? Pourquoi selon vous ?
    \end{enumerate}
}

\scn{Utiliser une numération en base 12}

\slide{exo}{\cdp{Table de 12}{\Table{12}{12}}}

\slide{cr}{
    \sseq\ssec\ssubsec \bvspace{-0.5cm}
    \vc{}{
        Le \key{système duodécimal},
        ou de \key{base $12$}.
        Est une méthode de \key{comptage par douzaines}.
    }[\wiki{Système_duodécimal}]
}

\slide{cr}{
    \expl{}{
        \multiColEnumerate{2}{
            \item $15 =$ \baseDecomposition{15}{12}["hide"]
            \item $56 =$ \baseDecomposition{56}{12}["hide"] 
            \item $\awsr{27} =$ \baseDecomposition{27}{12}
            \item $\awsr{112} =$ \baseDecomposition{112}{12} \saveenumi
        }\multiColEnumerate{1}{\loadenumi
            \item $145 =$ \awsr{\baseDecomposition{145}{12}}
        }
    }
}

\def\scale{1.15}
\slide{exo}{\small
    \exo{Justifiez, en détaillant vos calculs, le nombre de cubes présents dans chacune des figures.}{\calculator
        \multiColEnumerate{2}{
            \item \RepresenterEntier[Base=12,Echelle=\scale]{28}
            \item \RepresenterEntier[Base=12,Echelle=\scale]{100} \saveenumi
        }
    }
}

\slide{exo}{
    \multiColEnumerate{1}{ \loadenumi[exo]
        \item \RepresenterEntier[Base=12,Echelle=\scale]{332}
        \item \RepresenterEntier[Base=12,Echelle=\scale]{1942}
    }
}


\scn{Découvrir la numération Babylonienne}

\qfSlide{
    \multiColEnumerate{2}{
        \item \awsr{6425} = \baseDecomposition{6425}{10}
        \item \awsr{1237} = \baseDecomposition{1237}{12}
        \item 9233 = \baseDecomposition{9233}{10}["hide"]
        \item 232 = \baseDecomposition{232}{12}["hide"]
    }
}

\bsubsec{Base 60}

\def\aspc{\ifbool{answer}{}{\vspace{1cm}}}

\slide{exo}{\bshrink
    \act{Numération Babylonienne}{
        \begin{enumerate}\bvspace{-1cm}
            \item \multiColEnumerate{3}{
                \item \Babylone{24} = 24
                \item \Babylone{6} = 6
                \item \Babylone{50} = 50
            }
            Que représentent le clou \Babylone{1} et le chevron \Babylone{10} ?
            \item \multiColEnumerate{2}{
                \item \Babylone{12} = \awsr{12}
                \item \Babylone{34} = \awsr{34}
                \item \aspc \awsr{\Babylone{23}} = 23
            } \saveenumi
        \end{enumerate}
    }[\wiki{Numération_mésopotamienne}[Numération_sexagésimale_de_position]]
}

\slide{exo}{
    \begin{enumerate} \loadenumi[act]
        \item \multiColEnumerate{2}{
            \item \Babylone{70} = 70
            \item \Babylone{61} = 61
            \item \Babylone{190} = 190
            \item \Babylone{1380} = 1380
            \item \Babylone{3865} = 3865
        }
        Comment ces nombres sont-ils composés ?
        \saveenumi
    \end{enumerate}
}

\slide{exo}{
    \begin{enumerate} \loadenumi[act]
        \item \multiColEnumerate{2}{
            \item \Babylone{86} = \awsr{86}
            \item \aspc \awsr{\Babylone{132}} = 132
            \item \Babylone{325} = \awsr{325}
            \item \Babylone{7271} = \awsr{7271}
            \item \aspc \awsr{\Babylone{10872}} = 10872
        } \saveenumi
    \end{enumerate}
}

\slide{exo}{
    \begin{enumerate} \loadenumi[act]
        \item Les chiffres babyloniens changent de signification selon leur position : on parle donc de \key{numération de position}.
        Notre système de numération actuel est-il aussi un système de numération de position ?
        \item Existe-t-il des éléments que nous comptons encore aujourd'hui de manière similaire aux Babyloniens ?
    \end{enumerate}
}

\scn{Utiliser une numération en base 60}

\qfSlide{
    \exo{Donner l'heure}{
        \multiColEnumerate{3}{
            \item \Horloge[Secondes=false]{7:30}
            \item \Horloge[Secondes=false]{15:24}
            \item \Horloge[Secondes=false]{12:46}
        }
    }
}

\slide{cr}{
    \ssubsec
    \vc{}{
        Le \key{système sexagésimal},
        ou de \key{base $60$}.
        Est une méthode de \key{comptage par soixantaines}.
    }[\wiki{Système_sexagésimal}]
}

\slide{cr}{
    \expl{}{
        \multiColEnumerate{2}{
            \item \awsr{186} = \baseDecomposition{186}{60}
            \item \awsr{1200} = \baseDecomposition{1200}{60}
            \item 720 = \baseDecomposition{75}{60}["hide"]
            \item 1842 = \baseDecomposition{500}{60}["hide"] \saveenumi
        }\multiColEnumerate{1}{\loadenumi
            \item 8350 = \awsr{\baseDecomposition{8350}{60}}
        }
    }
}

\slide{cr}{
    \rmk{}{
        L'usage moderne du sexagésimal est assez proche de celui de la mesure du temps.
        \multiColItemize{2}{
            \item $\Horaire{1} = \awsr{60}\textrm{min}$
            \item $\Horaire{;1} = \awsr{60}\textrm{s}$
        }
    }

    \expl{}{
        \multiColItemize{1}{
            \item $\Horaire{1} = \awsr{3600}\sec$
            \item $\Horaire{;;602} = \parseSeconds{602}["hide"]$
            \item $\Horaire{;;7623} = \parseSeconds{7623}["hide"]$
        }
    }
}

\bookSlide{36p149,38p149,39p149}[7cm][2]

\def\scale{2}

\scn{Utiliser une numération en base 10}

\slide{qf}{\bsmall
    \nullsubsec{}{
        \begin{itemize}
            \item La Lune est à \Ecriture{384 000} de kilomètre de la Terre.
            \item Jupiter est à \Ecriture{91 000 000} de kilomètre de la Terre.
            \item Pluton est à \Ecriture{4 297 000 000} de kilomètre de la Terre.
        \end{itemize}

        Complète le tableau ci-dessous avec ces nombres écrits en chiffres.

        \begin{center}
            \Tableau[%
            DoubleEntree,
            Couleur=gradeColor!15,
            LegendesH={Lune,Jupiter,Pluton},
            LegendesV={Distance à la Terre (km)},
            Largeur=135pt
            ]{\awsr{384 000},\awsr{91 000 000},\awsr{4 297 000 000}}
        \end{center}
    }[\dmeepcS]
}

\bsec{Numération décimales}

\bsubsec{Nombres entiers}

\slide{cr}{
    \ssec\ssubsec
    \vc{}{
        Le \key{système décimal},
        ou de \key{base $10$}.
        Est une méthode de \key{comptage par dizaines}.
    }[\wiki{Système_décimal}]

    \hist{}{
        L'utilisation actuelle des \key{chiffres arabes} repose sur un système de numération \key{décimal et positionnel}.
        Leur diffusion au Moyen-Orient et en Europe serait due à un ouvrage du mathématicien persan d'\key{Al-Khwârizmî} (780-850 ap. J.-C.).
    }[\wiki{Al-Khwârizmî} \wiki{Système_de_numération_indo-arabe}]
}

\slide{cr}{
    \mthd{}{
        Pour écrire un nombre on utilise \key{10 symboles} appelé \key{chiffres}.
        La \key{position} des chiffres dans l'écriture d'un nombre détermine sa valeure.
    }

    \expl{}{
        \begin{enumerate}
            \item $\np{110} =$ \baseDecomposition{110}{10}["hide"] 
            \item $\np{5841} =$ \baseDecomposition{5841}{10}["hide"]
            \item $\awsr{{\np{1010}}} =$ \baseDecomposition{1010}{10}
        \end{enumerate}
    }
}

\slide{cr}{
    \rmk{}{
        On regroupe les chiffres de nombres par groupes de trois afin d'en améliorer la lisibilité.
    }
}

\slide{cr}{\bshrink
    \expl{}{
        Classer les chiffres des nombres suivants,
        puis les réécrire correctement :
        \multiColEnumerate{3}{
            \item $54 454$
            \item $36 119 312$
            \item $3 300 001 200$
        }
        \bvspace{-0.5cm}
        \decimalTable{{54454,36119312,3300001200}}["hide"]
        \multiColEnumerate{3}{
            \item $\awsr{\np{54454}}$
            \item $\awsr{\np{36 119 312}}$
            \item $\awsr{\np{3 300 001 200}}$
        }
    }
}

\scn{Manipuler des nombres entiers dans le système décimal}

\slide{qf}{
    Donner le chiffres des :
    \multiColEnumerate{2}{
        \item milliers de $\np{56165453}$
        \item milliards de $\np{546160006546521}$
        \item dizaines de milliers de $\np{346805235}$
        \item centaines de milliards de $\np{340045235}$
        \item centaines de $\np{65465,654654}$
        \item dizaines de millions de $\np{211010100,001}$
    }
}

\slide{exo}{
    \exo{Nombres mystères}{
        \begin{enumerate}
            \item Donne un exemple de nombre inférieur à $400$ pour lequel :
            \begin{itemize}
                \item le chiffre des dizaines est la moitié du chiffre des centaines.
                \item la somme des chiffres est $11$.
            \end{itemize}\saveenumi
        \end{enumerate}
    }[\dmeepcS]
}

\slide{exo}{
    \begin{enumerate}\loadenumi[exo]
        \item Donne un exemple de nombre à quatre chiffres tel que :
        \begin{itemize}
            \item le chiffre des dizaines est la moitié du chiffre des centaines.
            \item la somme des chiffres est $11$.
        \end{itemize}
        \item Donne un exemple de nombres à trois chiffres pour lequel :
        \begin{itemize}
            \item il est inférieur à $\np{2000}$ ;
            \item il a trois chiffres identiques ;
            \item la somme de ses chiffres est 10.
        \end{itemize}
    \end{enumerate}
}

\scn{Formaliser la notion de la division euclidienne}

\slide{qf}{Compléter les divisions suivantes
    \multiColEnumerate{2}{
        \item \longDivision{155}{5}
        \item \longDivision{700}{49}
    }
}

\bsubsec{Division euclidienne}

\slide{exo}{\bshrink
    \act{}{%
    Le roi de Divisia possède $27$ pièces d'or et souhaite les partager équitablement entre $4$ chevaliers.
    Les pièces restantes seront données à son écuyer.
    
    \begin{enumerate}
        \item Combien de pièces chaque chevalier recevra-t-il ?
        \item Combien de pièces resteront pour l'écuyer ?
        \item Supposons maintenant que le roi possède $100$ pièces et qu'il les partage entre $6$ chevaliers :
        \begin{enumerate}
            \item Combien de pièces chaque chevalier recevra-t-il ?
            \item Combien de pièces resteront pour l'écuyer ?
        \end{enumerate}\saveenumi
    \end{enumerate}
    }
}

\slide{exo}{
    \begin{enumerate}\loadenumi[act]
        \item Toujours avec $100$ pièces,
        existe-t-il un nombre de chevaliers tel que; l'écuyer reçoive:
        \multiColEnumerate{1}{
            \item $0$ pièce ?
            \item plus de pièces qu'un chevalier ?
            \item autant de pièces qu'un chevalier ?
            \item autant de pièces qu'il y a de chevalier ?
        }
    \end{enumerate}
}

\slide{cr}{\bsmall
    \ssubsec

    \df{}{
        Pour deux entiers $a$ et $b$,
        on appelle \key{division euclidienne}
        du \textcolor{Green}{\key{dividende} $a$} par le \textcolor{Red}{\key{diviseur} $b$}
        l'expression :
        \begin{align*}
            \textcolor{Green}{a}
            = \textcolor{Red}{b}
            \times \textcolor{Blue}{q}
            + \textcolor{Violet}{r}
        \end{align*}
        Où le \textcolor{Blue}{\key{quotient} $q$} et \textcolor{Violet}{\key{reste} $r$} sont deux entiers avec
        $\textcolor{Violet}{r} < \textcolor{Red}{b}$.
    }

    \expl{}{
        \multiColEnumerate{3}{
            \item $100 = 6 \times \awsr{16} + \awsr{4}$
            \item $246 = 3 \times \awsr{82} + \awsr{0}$
            \item $360 = 23 \times \awsr{15} + \awsr{15}$
        }
    }
}

\scn{Notion de divisibilité}

\slide{qf}{
    \exo{}{
        Quel est le nombre de:
        \multiColEnumerate{2}{
            \item dizaines dans $750$.
            \item milliers dans $\np{665454}$.
            \item millions dans $\np{9876502300}$.
            \item dizaines de milliers dans $\np{121321}$.
            \item centaines de millions dans $\np{2313251}$.
            \item centaines dans $\np{352154.16}$.
        }
    }
}

\slide{cr}{\bsmall
    \df{}{
        Pour $a$ et $b$ deux entiers,
        on dit que $b$ \key{divise} $a$ si le reste de la division euclidienne de $a$ par $b$ est $0$.
    }

    \pr{}{
        \Sialors{$b$ \key{divise} $a$}{$a$ est un \key{multiple} de $b$}
    }

    \begin{center}
        \expl{}{
            \Tableau[%
                DoubleEntree,
                Stretch=1.5,
                Couleur=gradeColor!15,
                LegendesH={\qquad$4$\qquad,\qquad$6$\qquad,\qquad$12$\qquad,\qquad$31$\qquad},
                LegendesV={Divise 62 ?, Multiple de 4 ?},
                Largeur=2cm
            ]{
                \awsr{Oui},\awsr{Non},\awsr{Oui},\awsr{Oui},
                \awsr{Oui},\awsr{Non},\awsr{Oui},\awsr{Non}
            }
        }
    \end{center}
}

\slide{exo}{
    \exo{}{
        Une fleuriste dispose de 1815 fleurs.
        Doit-elle réaliser des bouquets de 16 fleurs ou de 17 fleurs pour en utiliser le plus possible ?
    }[\iP{6}{2021}[2][16]]
}
\bookSlide{21p45,23p45,25p45}[7cm][2]

\scn{Priorités opératoires}

\bsec{Opérations}
\bsubsec{Règles opératoires}

\slide{qf}{\bvspace{-0.5cm}
    \exo{}{
        \multiColEnumerate{1}{
            \item $\np{44420} \times 100 = $
            \item $981 \times \np{100000} = $
            \item $\np{685540000} \div \np{10} = $
            \item $\np{230020000000} \div \np{10000} = $
            \item $\np{10001} \times \np{20000} = $
            \item $\np{3090300} \div \np{300} = $
        }
    }
}

\slide{cr}{
    \ssec\ssubsec

    \rl{}{
        Les calculs se font dans l'ordre des priorités suivant:%
        \begin{enumerate}
            \item La multiplication et la division
            \item L'addition et la soustraction
        \end{enumerate}
    }
}

\slide{cr}{
    \rl{}{
        En cas d'opérations de mêmes priorités, on effectue les opérations de gauche à droite.
    }

    \expl{}{
        \multiColEnumerate{1}{
            \item $3 - 2 + 3 = \awsr{1 + 3 = 4}$ 
            \item $19 - 6\times 3 = \awsr{19 - 18 =  1}$
            \item $3 + \np{3.2} \times 2 - 4 = \awsr{3 + \np{6.4} -4 = \np{9.4} - 4 = \np{5.4}}$
        }
    }
}

\slide{cr}{
    \rl{}{
        On commence par effectuer les calculs entre parenthèses.
    }

    \expl{}{
        \multiColEnumerate{1}{
            \item $(1 + 2) \times 21 = \awsr{3 \times 21 = 63}$
            \item $(11 \times 3) + (15 \div 2) = \awsr{33 + 7.5 = 40.5}$
            \item $((13 - (3 - 2)) + 2) = \awsr{(13 - 1) + 2 = 12 + 2 = 14}$
        }
    }
}

\bsubsec{Vocabulaire opératoires}

% \newpage

\slide{cr}{
    \ssubsec
    \bvspace{-0.5cm}
    \vc{}{
        On connait quatres types d'opérations :
        \begin{itemize}
            \item L'\key{addition} permet de calculer la \key{somme} de deux \key{termes}.
            \item La \key{soustraction}  permet de calculer la \key{différence} entre deux \key{termes}.
            \item La \key{multiplication} permet de calculer la \key{produit} de deux \key{facteurs}.
            \item La \key{division} permet de calculer la \key{quotient} de deux \key{nombres}.
        \end{itemize}
    }
}

\slide{cr}{
    \vc{}{Dans un calcul,
    le type de la dernière opération effectuée détermine le nom donné au calcul dans son ensemble.}
    \bvspace{-0.5cm}
    \expl{Nommer les calculs suivants}{
        \bvspace{-0.5cm}
        \multiColEnumerate{1}{
            \item $1,6 + 4$ est \awsr{la somme} de \awsr{$1,6$ et $4$}.
            \item $(\frac{2}{6} + 3) \times 9$ est \awsr{le produit } de \awsr{$\frac{2}{6} + 3$ et $9$}.
            \item $6,6 + 1 \times 8$ est \awsr{la somme de $6,6$ par $1 \times 8$}.
            \item $\frac{2}{6} + 3 - 9$ est \awsr{la différence entre $\frac{2}{6}$ et $9$}.
            \item $\pi \div (3 - 9)$ est \awsr{le quotient de $\pi$ par $(3 - 9)$}.
        }
    }
}

\slide{exo}{\bshrink
    \exo{Les bons comptes}{
        Pour résoudre les problèmes suivants :  
        \begin{itemize}
            \item Présenter chaque calcul séparément, en précisant ce que représente la valeur obtenue à chaque étape.  
            \item Présenter tous les calculs en une seule expression permettant d'obtenir le résultat final.  
        \end{itemize}
        
        \begin{enumerate}
        \item Xavier possède \Prix{28} dans sa tirelire.  
                Son grand-père lui donne \Prix{75}. Il a désormais \Prix{53} de plus que sa sœur Christine.  
                Quelle somme d'argent possède Christine ?\saveenumi
        \end{enumerate}
    }
}

\slide{exo}{
    \begin{enumerate}\loadenumi[exo]
        \item Un commerçant achète sept rouleaux de \Lg[m]{50} de tissu.  
        Chaque rouleau coûte \Prix{392}.  
        Il revend le tissu au prix de \Prix{12} par mètre.  
        Quel bénéfice réalise-t-il après avoir revendu la totalité du tissu ?
        \item Julien et Georges possèdent à eux deux un total de \Prix{47}.  
        Julien dépense \Prix{12} et Georges dépense \Prix{7}.  
        Après ces dépenses, ils ont chacun la même somme.  
        Quelle somme Julien possédait-il avant sa dépense ?
    \end{enumerate}
    \awsr[0]{
        \begin{enumerate}
            \item 
            \begin{itemize} 
                \item \begin{itemize}
                    \item $28 + 75 = 103$ : Xavier possède maintenant \Prix{103}.
                    \item $103 - 53 = 50$ : Christine possède \Prix{50}.
                \end{itemize}
                \item $28 + 75 - 53 = 50$
            \end{itemize}
            \item Le bénéfice est la différence entre le prix de vente total et le prix d'achat total.
            \begin{itemize}
                \item \begin{itemize}
                    \item $392 \times 7 = 2744$ : prix d'achat total des 7 rouleaux, soit \Prix{2744}.
                    \item $7 \times 50 = 350$ : le commerçant dispose de \Lg[m]{350} de tissu.
                    \item $350 \times 12 = 4200$ : prix de vente total du tissu, soit \Prix{4200}.
                    \item $4200 - 2744 = 1456$ : bénéfice réalisé après la revente de la totalité du tissu, soit \Prix{1456}.
                \end{itemize}
                \item $7 \times 50 \times 12 - 392 \times 7 = 1456$
            \end{itemize}
            \item 
            \begin{itemize}
                \item \begin{itemize}
                    \item $47 - 12 = 35$ : après la dépense de Julien, il reste \Prix{35}.
                    \item $35 - 7 = 28$ : après la dépense de Georges, il reste \Prix{28}.
                    \item $28 \div 2 = 14$ : chacun possède maintenant \Prix{14}.
                    \item $14 + 12 = 26$ : Julien possédait donc \Prix{26} avant sa dépense.
                \end{itemize}
                \item $(47 - 12 - 7) \div 2 + 12 = 26$
            \end{itemize}
        \end{enumerate}        
    }
}
% % VARIABLES %%%
\setSeq{3}{Symétries}
\setGrade{5e}

\def\imgPath{enseignement/5e/symetries/}

\def\ym{\href{https://www.maths-et-tiques.fr/telech/19Sym.pdf}{Yvan Monka}}
%%

\obj{
    \item Transformer une figure par une symétrie centrale.
    \item Identifier des symétries dans des frises, des pavages, des rosaces.
    \item Comprendre l'effet des symétries (axiale et centrale) :
    conservation du parallélisme, des longueurs et des angles.
    \item Mobiliser les connaissances des figures,
    des configurations et des symétries pour déterminer des grandeurs géométriques.
    \item Mener des raisonnements en utilisant des propriétés des figures,
    des configurations et des symétries.
}

\def\grid{\draw[gray!40] (0,0) grid (14,8);}
\def\arrow{\draw [ultra thick] (1,4)--(4,4)--(4,3)--(6,5)--(4,7)--(4,6)--(1,6)--cycle;}

\NewDocumentCommand{\mushroom}{O{(0,0)}}{
    \draw[thick, shift={#1}]
    (0,2) -- (2,4) -- (4,4) -- (6,2) -- (6,0) -- (0,0) -- cycle;
    
    % Spot
    % \draw[ultra thick, shift={#1}] 
    %     (1.5,3.5) -- (2,2) -- (4,2) -- (4.5,3.5);  % Center spot
        
    % Mushroom Stem
    \draw[thick, shift={#1}] 
        (1,0) -- (2,-2) -- (4,-2) -- (5,0) -- cycle; % Stem (rectangle)

    % Eyes
    \draw[thick, shift={#1}] 
        (2.5,-1) -- (2.5,0);

    \draw[thick, shift={#1}] 
        (3.5,-1) -- (3.5,0);
}

\def\figInit{
    \grid
    \arrow
    \drawPoint{$O$}{7}{4}
}

\newcommand{\tikzc}[1]{
    \begin{center}
        \begin{tikzpicture}[scale=0.5]
            #1
        \end{tikzpicture}
    \end{center}
}

\scn{Rappels sur la symétrie axiale}

\slide{qf}{
    \exo{}{
        Combien existe-t-il d'axes de symétrie pour chacun de ces panneaux?
        
        \imgp{panneaux-de-signalisation}[8cm]
    }[\href{https://clairelommeblog.fr/wp-content/uploads/2020/03/panneaux_routiers.pdf}
    {Claire Lommé}]
}

\bsec{Les symétries}
\bsubsec{Symétries Axiale}

% \newpage
\slide{exo}{
    \bvspace{-0.5cm}
    \act{}{
        Construire l'image de la figure par la symétrie d'axe $(d)$.
        \bvspace{-0.75cm}
        \tikzc{
            \draw[gray!40] (0,-3) grid (14,8);
            \arrow
            \draw[thick, gradeColor] (0,-3)--(11,8);
            \node[gradeColor, above left] at (2,-1) {$(d)$};
        }
    }
}

\slide{cr}{
    \sseq\ssec\ssubsec
    \df{}{
        Deux figures sont dites \key{symétriques par rapport à une droite} si elles se \key{superposent par pliage} le long de cette droite.
    }[\wiki{Symétrie_axiale}]
}

\slide{cr}{\bvspace{-0.6cm}
    \expl{}{\bvspace{-0.5cm}
        \imgp{expl-symetrie-axiale}[7.5cm]
    }
}

\slide{cr}{
    \pr{}{
        \Sialors{le point $M'$ est l'image du point $M$ par la symétrie d'axe $(d)$}{
            \begin{enumerate}
                \item la droite $(MM')$ est \bawsr{perpendiculaire} à la droite $(d)$
                \item le milieu du segment $[MM']$ est sur la droite $(d)$.
            \end{enumerate}
        }
        \rmk{}{%
            La droite $(d)$ est alors la \bawsr{\key{médiatrice} } du segment $[MM']$.
        }
    }
}

\slide{exo}{
    \bvspace{-0.6cm}
    \exo{}{
        Construire l'image de la figure par la symétrie d'axe $(d)$.
        \bvspace{-0.75cm}
        \tikzc{
            \draw[gray!40] (0,-4) rectangle (17,9);
            \mushroom[(10,-1)]
            \draw[thick, gradeColor] (7,-3)--(10,8);
            \node[gradeColor, above left] at (7,-3) {$(d)$};
        }
    }
}

\bsubsec{Symétries Centrale}

\def\one{%
    On va construire l'image de la figure par la symétrie de centre $O$.

    \tikzc{\figInit}
}

\def\two{%
    On regarde le «\textit{chemin}» du point $A$ au point $O$.
    
    \tikzc{
        \figInit
        \drawPoint{$A$}{4}{3}
        \draw[dashed, thick, Red] (4,3)--(7,4); 
    }
}

\def\three{%
    On exécute le même «\textit{chemin}», cette fois en partant du point $O$. On trouve alors le point $A'$,
    image du point $A$ par la symétrie de centre $O$.

    \tikzc{
        \figInit
        \drawPoint{$A$}{4}{3}
        \draw[dashed, thick, Red] (7,4)--(10,5);
        \drawPoint{$A'$}{10}{5}
    }
}

\def\four{%
    Fait de même avec les autres points de la figure, puis les reliers, de façon à obtenir l'image de la flèche par la symétrie de centre $O$.

    \tikzc{
        \figInit
        \drawPoint{$A$}{4}{3}
        \drawPoint{$B$}{4}{4}
        \drawPoint{$C$}{1}{4}
        \drawPoint{$D$}{1}{6}
        \drawPoint{$E$}{4}{6}
        \drawPoint{$F$}{4}{7}
        \drawPoint{$G$}{6}{5}
        \drawPoint{$A'$}{10}{5}
    }
}

\scn{Découvrir la symétrie centrale}

\ifArticle{%
    \slide{exo}{
        \act{}{\vspace{-0.5cm}
            \multiColEnumerate{2}{
                \item[] \one
                \item \two
                \item \three
                \item \four
            }
        }[\href{https://drive.google.com/drive/folders/1Itzq0ZPj1sHwSIv9SgbUlIIuHdG28A5I}{Jérôme Potel}]
    }
}


\ifBeamer{%
    \slide{exo}{\act{}{\one}}
    \slide{exo}{\two}
    \slide{exo}{\three}
    \slide{exo}{\four}
}

\slide{cr}{
    \ssubsec
    
    \df{}{
        Deux figures sont dites \key{symétriques par rapport à un point}
        si l'on peut obtenir l'une en effectuant un \key{demi-tour} de l'autre \key{autour de ce point}.
    }[\wiki{Symétrie_centrale}]
}

\slide{cr}{\bvspace{-0.6cm}
    \expl{}{\bvspace{-0.5cm}
        \imgp{expl-symetrie-centrale}[9cm]
    }
}

\slide{cr}{
    \pr{}{
        \Sialors{le point $M'$ est le symétrique du point $M$ par la symétrie de centre $O$}{
            le point $O$ est \bawsr{le milieu du segment} $[MM']$.
        }
    }
}

% % VARIABLES %%%
% \date{\today}
\setSeq{3}{Nombres Relatifs}
\setGrade{4e}
\def\imgPath{enseignement/4e/nombres-relatifs/}
% \setboolean{answer}{false}
% \print

\def\ym{\href{https://www.maths-et-tiques.fr/telech/19Nomb_rela.pdf}{Yvan Monka}}
\def\caPrefix{4e-entrainement.2-}
%%

\obj{
    \item Appliquer la règle des signes pour les produits et quotients de nombres relatifs.
    \item Déterminer le signe d'un produit de plusieurs facteurs relatifs.
    \item Déterminer le carré et le cube d'un nombre relatif.
    \item Trouver les antécédents du carré d'un nombre donné.
}

\scn{Tournois \icon{RELATIvs/logo}}

\bsec{Produits et quotients de nombres relatifs}
\bsubsec{Règle des signes}

\scn{Démontration des propriétés sur les produits de nombres relatifs}

\slide{qf}{
    \begin{enumerate}
        \item Développer les expressions suivantes :
        \multiColEnumerate{2}{
            \item $(x + 2) \times 6 =$
            \item $y \times (a + (-6)) =$
            \item $k \times (a + b) =$
            \item $8,1 \times (10 + 2) =$
        }
        \item Réduire les expressions suivantes :
        \multiColEnumerate{2}{
            \item $4 \times x =$
            \item $y \times 3 =$
            \item $a \times b =$
            \item $- q \times l =$
        }
    \end{enumerate}
}

\slide{exo}{\bsmall\bvspace{-0.7cm}
    \act{}{
        Dans cette activité on cherche à trouvé,
        pour deux nombres positifs $x$ et $y$,
        à quoi est égale $x \times  (- y)$ et $ (- x) \times (- y)$.
        \begin{enumerate}
            \item On commence par s'interesser au résultat de $3 \times (-5)$.
            \begin{enumerate}
                \item Combien donne $3 \times (5 + (-5))$
                \item Développer le produit $3 \times (5 + (-5))$
                \item D'après le ${\color{\currentColor}a)}$, à combien est égale la forme développer de l'expression précédente ?
                \item Combien donne $3 \times 5$ ?
                \item Combien donne alors $3 \times (-5)$?
            \end{enumerate}\saveenumi
        \end{enumerate}
    }
}

\slide{exo}{\bsmall
    \begin{enumerate}\loadenumi[act]
        \item On s'interesse maintenant au résultat de $x \times (-y)$.\\
        Reproduit le raisonnement de ${\color{\currentColor}a)}$ à ${\color{\currentColor}e)}$ en substituant $3$ par $x$ et $5$ par $y$.
        \item Que donne le produit d'un nombre positif par un nombre négatif?
        \item On s'interesse maintenant au résultat de $-3 \times (-5)$.\\
        Reproduit le raisonnement de ${\color{\currentColor}a)}$ à ${\color{\currentColor}e)}$ en substituant $3$ par $-3$.
        \item On s'interesse maintenant au résultat de $-x \times (-y)$.\\
        Reproduit le raisonnement de ${\color{\currentColor}a)}$ à ${\color{\currentColor}e)}$ en substituant $3$ par $-x$ et $5$ par $y$.
        \item Que donne le produit d'un nombre négatif par un nombre négatif?
    \end{enumerate}
    
}

\slide{cr}{
    \sseq\ssec\ssubsec
    \pr{}{
        Le \key{produit} ou \key{quotient} de deux nombres :
        \begin{itemize}
            \item de \key{même signes} est \bawsr{\key{positif}}.
            \item de \key{signe opposés} est \bawsr{\key{négatifs}}.
        \end{itemize}
    }
}

\slide{cr}{
    \expl{}{
        \multiColEnumerate{2}{
            \item $-3 \times (-8) = \bawsr{24}$
            \item $-8 \div (-6) = \bawsr{\frac{-8}{-6} = \frac{8}{6} = \frac{4}{3} \approx \num{1.3333}}$
            \item $-3 \times 5 = \bawsr{-15}$
            \item $5 \times (-4) = \bawsr{-20}$
            \item $-4 \div 2 = \bawsr{\frac{-4}{2} = -\frac{4}{2} = -\frac{2}{1} = -2}$
            \item $4 \div (-5) = \bawsr{-\frac{4}{5} = -0.8}$
        }
    }

    \bvspace{-0.5cm}

    \rmk{}{
        On a pour $a$ est $b$ deux nombres.
        \begin{enumerate}
            \item $a \times (-b) = (-a) \times b = - (a \times b)$
            \item $\frac{-a}{b} = \frac{a}{-b} = -\frac{a}{b}$
        \end{enumerate}
    }
}

\bookSlide{29p87,27p87,35p88,38p88}[6.25cm][2]

\scn{Déterminer le signe d'un produits de relatifs à plusieurs facteurs}

\slide{qf}{
    \multiColEnumerate{1}{
        % \item $\num{1.5} \times (-2) = \bawsr{-3}$
        \item $1 + (-6) - (-2) = \bawsr{13}$
        \item $(-30) + \num{50.9} = \bawsr{20.9}$
        \item $1 + (-6) \times (-2) = \bawsr{13}$
        \item $3 \div (-1) = \bawsr{-3}$
        \item $-7 \times (1 - (-2)) = \bawsr{-21}$
    }
}

% \def\caPrefix{4e-entrainement.2-}
% \caSlide{22-23-24}

% \slide{qf}{
%     \renewcommand{\arraystretch}{1.75}
%     \begin{center}
%         \begin{tabular}{| M{0.1\linewidth} | M{0.5\linewidth} | M{0.4\linewidth} |}
%             \hline
%             22 & Le périmètre d'un rectangle de longueur $14,6$ et de largeur $5,5$ est : &
%             \\\hline
%             23 & $7-9-(-4) =$ &
%             \\\hline
%             24 & $-13+10\times0,5 =$ &
%             \\\hline
%         \end{tabular}
%     \end{center}
% }

\bsubsec{Corollaires}

\slide{exo}{\bshrink
    \act{}{\noCalculator
        \begin{enumerate}
            \item Calculer le \key{résultat} des produits suivants :
            \multiColEnumerate{2}{
                \item $3 \times (-6) \times (-2)$
                \item $-1 \times (-3) \times (-1)$
                \item $-2 \times 1 \times (-1) \times (-6)$
                \item $-1 \times (-1) \times (-1) \times (-1) \times (-1)$
            }
            \item Déterminer le \key{signe} des produits suivants :
            \multiColEnumerate{1}{
                \item $-15198 \times 1231 \times (-1,6)$
                \item $-1,465884 \times (-3) \times (-1) \times (-3) \times (-\frac{1}{654})$
                \item $51,65465 \times (-1) \times (-\pi) \times (-6)$
            }\saveenumi
        \end{enumerate}
    }
}

\slide{exo}{
    \begin{enumerate}\loadenumi[act]
        \item Déterminer le signe d'un produit de :
        \multiColEnumerate{1}{
            \item $10$ facteurs égaux à $-1$.
            \item $11$ facteurs égaux à $-1$.
            \item $1541$ facteurs égaux à $-1$.
            \item $8$ facteurs négatifs.
            \item $5$ facteurs négatifs et $31$ facteurs positifs.
            \item $5$ facteurs négatifs et $32$ facteurs positifs.
        }
        \item Émettre une conjecture sur le signe d'un produit de nombres relatifs.
    \end{enumerate}
}

\slide{cr}{
    \ssubsec
    \pr{}{
        Le résultat d'un \key{produit de relatifs} est :
        \begin{enumerate}
            \item \key{positif} s'il y a un nombre \key{pair} de facteurs négatifs dans le produit.
            \item \key{négatif} s'il y a un nombre \key{impair} de facteurs négatifs dans le produit.
        \end{enumerate}
    }
}

\slide{cr}{
    \expl{}{
        Déterminer le signe des produits suivants :
        \multiColEnumerate{1}{
            \item $-1 \times (-1,2) \times 2 \times (-3,36954) \times 2$
            \item $2 \times \frac{1}{2}  \times (-\pi) \times 1221 \times (-2)$
            \item $(-1,58)^2$
            \item $(-1654)^3$
            \item Un produit de $2024$ fracteurs égales à $-1$.
        }
    }
}

\slide{cr}{
    \rmk{}{
    \begin{enumerate}
        \item Le \key{carré} d'un nombre est toujours \key{positif}.
        \item Le \key{cube} d'un nombre \key{conserve le signe} de ce nombre.
    \end{enumerate}
}
}

\bookSlide{20p87}[6.25cm][1]

\scn{Résoudre une équation du second degré sans terme linéaire}

\caSlide{28-29-30}

\bsec{Antécédents du carré}

\slide{exo}{\bsmall
    \act{}{%
        \begin{enumerate}
            \item Trouver deux nombres distincts dont le carré est égal à $4$.
            \item Trouver deux nombres distincts dont le carré est égal à $1$.
            \item Trouver deux nombres distincts dont le carré est égal à $2$.
            \item Trouver deux nombres distincts dont le carré est égal à $x$.
        \end{enumerate}
    }
}

\slide{cr}{
    \pr{}{
        Pour $x$ un nombre positif.
        Les nombres dont le carré est égal à $x$ sont $\sqrt{x}$ ou $-\sqrt{x}$.
    }
}

\slide{cr}{
    \expl{}{
        \begin{enumerate}
            \item \sialors{$a^2 = 16$}{$a = \bawsr{4}$ ou $a = \bawsr{-4}$}
            \item \sialors{$b^2 = 1$}{$b = \bawsr{1}$ ou $b = \bawsr{-1}$}
            \item \sialors{$c^2 = 21$}{$c = \bawsr{\sqrt{21}}$ ou $c = \bawsr{-\sqrt{21}}$}
            \item Soit $x$ un nombre positif, \sialors{$d^2 = x$}{$d = \bawsr{\sqrt{x}}$ ou $d = \bawsr{-\sqrt{x}}$}
        \end{enumerate}
    }
}

\bookSlide{50p91,47p90}[5.5cm][2]

\slide{qf}{
    \nullsubsec{}{
        Soit $y$ un nombre négatif différent de $0$.\\
        Indiquer le signe de chacune de ces expressions:
        \multiColEnumerate{2}{
            \item $A = -4 - y$
            \item $B = y \times y$
            \item $C = y + y + y$
            \item $D = y ^ 3$
            \item $E = -3 \times y$
            \item $F = -3 + y$
        }
    }[\rpmc[184]]
}


% % VARIABLES %%%
\setSeq{4}{Théorème de Pythagore - Contraposée et réciproque}
\setGrade{4e}
\def\imgPath{enseignement/4e/theoreme-de-pythagore/contraposee-et-reciproque/}

\forPrint
% \setboolean{answer}{true}

\def\ym{https://www.maths-et-tiques.fr/telech/19Pyth2.pdf}
%%

\obj{
    \item Comprendre les notions de réciproque et de contraposée.
    \item Utiliser la contraposée du Théorème de Pythagore pour montrer qu'un triangle n'est pas rectangle.
    \item Utiliser la réciproque du Théorème de Pythagore pour montrer qu'un triangle est rectangle.
    \item Déterminer si un triangle est rectangle ou non.
}

\obj{
    \item Reconnaitre sur un graphique une situation de proportionnalité ou de non proportionnalité.
    \item Calcule d'une quatrième proportionnelle.
    \item Utiliser une formule liant deux grandeurs dans une situation de proportionnalité.
    \item Résoudre des problèmes en utilisant la proportionnalité dans le cadre de la géométrie.
}[Flash]

\scn{Découvrir des notions de logiques ; la réciproque}

\slide{qf}{\bvspace{-0.35cm}
    \begin{enumerate}
        \item Les tableaux suivants représentent-ils des situations de proportionnalité ?
        Utilisez une calculatrice pour vérifier vos hypothèses.
        \multiColEnumerate{1}{
            \item \Propor[Simple]{1/2,6/12,3/5,10/20}
            \item \Propor[Simple]{2/3,6/9,30/45}
        }
        \item Essayez ensuite de justifier vos réponses sans calculatrice en expliquant votre raisonnement.
    \end{enumerate}
}

\bsec{Logique}
\bsubsec{Réciproque}

\slide{exo}{\bshrink
    \act{}{
        \begin{enumerate}
            \item « \Sialors{c'est un triangle}{c'est un polygones à trois sommets} » est une \key{proposition}.  
            Est-elle vraie ? Justifie ta réponse.  
            
            \item « \Sialors{c'est un triangle}{c'est un polygones à quatre sommets} » est une autre proposition.  
            Est-elle vraie ? Justifie ta réponse.  
            
            \item La première proposition est composée de deux parties :  
            \multiColItemize{1}{
                \item l'\key{antécédent} : « c'est un triangle »,
                \item le \key{conséquent} : « c'est un polygones à trois sommets ».
            }  
            On appelle \key{réciproque} d'une proposition la phrase qu'on obtient en inversant l'antécédent et le conséquent.  
            Écris la réciproque de la première proposition.
            Est-elle vraie ? \saveenumi
        \end{enumerate}
    }[\href{http://www.mathsaharry.com/aw/52.pdf}{Math à Harry}]
}

\slide{exo}{
    \begin{enumerate} \loadenumi[act]
        \item Les propositions suivantes sont-elles vraies ?
        Écris leurs réciproques et détermine si elles sont vraies.
        \multiColEnumerate{1}{
            \item \Sialors{c'est un carré}{c'est un rectangle avec tous ses côtés égaux}
            \item \Sialors{il peut pondre des œufs}{c'est un oiseau}
            \item \Sialors{c'est un rectangle}{c'est un quadrilatère dont tous les opposés sont parallèles} 
            \item \Sialors{c'est un rectangle}{c'est un carré}
            \item \Sialors{$AB = BC$}{$B$ est le milieu de $[AC]$}
        } 
    \end{enumerate}
}

\slide{cr}{\bsmall
    \ssec
    \ssubsec

    \df{}{
        On appelle \key{réciproque}
        d'une proposition :
        {« \Sialors{$A$}{$B$} »}
        ; la proposition :
        {« \Sialors{$B$}{$A$} »}.
    }
    
    \rmk{}{
        Une proposition peut être vraie sans que sa réciproque le soit, et inversement.
    }

    \expl{}{
        La proposition « \Sialors{$[AB]$ et $[CD]$ ne se coupent pas}{$[AB]$ et $[CD]$ sont parallèles}»
        est \awsr{fausse}.\\
        Sa réciproque: \awsr{« \Sialors{$[AB]$ et $[CD]$ sont parallèle}{$[AB]$ et $[CD]$ ne se coupent pas}»}
        est \awsr{vrai}.
    }
}

\scn{Découvrir des notions de logiques ; la contraposée}

\slide{qf}{\calculator \\ Completer les tableaux suivants :
    \multiColEnumerate{3}{
        \item \begin{center}
            \Propor[Simple,
            Math,
            Stretch=1.25,%
            ]{6/5,\awsr{\np{2.4}}/2}
        \end{center}
        \item \begin{center}
            \Propor[Simple,
            Math,
            Stretch=1.25,%
            ]{\np{237.6}/\awsr{66},\np{46.8}/13}
        \end{center}
        \item \begin{center}
            \Propor[Simple,
            Math,
            Stretch=1.25,%
            ]{\awsr{12}/18,-3/-4.5}
        \end{center}
    }
}

\bsubsec{Contraposée}

\slide{exo}{
    \ssubsec
    \act{}{
        On appelle \key{contraposée} d'une proposition
        {« \Sialors{$A$}{$B$} »}
        la proposition obtenue en écrivant :  
        {« \Sialors{non $B$}{non $A$} »}.
        \begin{enumerate}
            \item La contraposée de la proposition : «\Sialors{c'est un triangle}{il a trois côtés}».
            est donc «\Sialors{il n'a pas trois cotés}{ce n'est pas un triangle}» Est-elle vraie ? \saveenumi
        \end{enumerate}  
    }
}

\slide{exo}{
    \begin{enumerate} \loadenumi[act]
        \item Les propositions suivantes sont-elles vraies ? Écris leurs contraposées et dis si elles sont vraies :  
        \multiColEnumerate{1}{ 
            \item \Sialors{$x=7$}{$x$ est un nombre premier}
            \item \Sialors{c'est un nombre positif}{il est strictement inférieur à zéro}
            \item \Sialors{il est à Issy-les-Moulineaux}{il n'est pas en Espagne}  
            \item \Sialors{$AB=BC$}{$B$ est le milieu de $[AC]$} 
        }
        \item Que remarques-tu ?
    \end{enumerate}
}

\slide{cr}{
    \df{}{
        On appelle \key{contraposée}
        d'une proposition ;
        {« \Sialors{$A$}{$B$} »} ;
        la proposition : 
        {« \Sialors{non $B$}{non $A$} »}.  
    }

    \rmk{}{
        Une proposition et sa contraposée sont toujours soit toutes les deux vraies,
        soit toutes les deux fausses.
    }

    \expl{}{
        La proposition « \Sialors{c'est un triangle est équilatéral}{ses trois côtés sont égaux} »  
        est \awsr{vraie}.  
        Sa contraposée: \awsr{« \Sialors{ses trois côtés ne sont pas égaux}{ce n'est pas un triangle équilatéral} »},
        est \awsr{également vraie}.
    }
}

\scn{Appliquer ses connaissances sur la contraposée au Théorème de Pythagore}

\slide{qf}{
    \calculator
    \exo{}{
        Sachant que huit briques de masse identique pèsent \Masse{13.6},
        calcule la masse de six de ces briques.
    }[\afa{4e}[6]]
}

\bsec{Contraposée du Théorème de Pythagore}

\slide{exo}{\bshrink
    \act{}{\bvspace{-1cm}
        \def\crossSize{0.15}
        \ctikz[1]{
            \draw[gray!40] (-1,-4) rectangle (11,3);
            \draw [penciline,thick] (8.2,1.76)-- (5.28,-1.84);
            \draw [penciline,thick] (5.28,-1.84)-- (9.5,-0.3);
            \draw [penciline,thick] (9.5,-0.3)-- (8.2,1.76);
            \draw [penciline,thick] (2.76,-2.24)-- (3.84,-0.56);
            \draw [penciline,thick] (3.84,-0.56)-- (0.34,1.22);
            \draw [penciline,thick] (0.34,1.22)-- (2.76,-2.24);
            \drawPoint{D}{3.84}{-0.56}
            \drawPoint{E}{0.34}{1.22}
            \drawPoint{F}{2.76}{-2.24}
            \drawPoint{G}{8.20}{1.76}
            \drawPoint{H}{9.50}{-0.30}
            \drawPoint{I}{5.28}{-1.84}
            \draw (3.48,-1.38) node[anchor=north west] {4cm};
            \draw (0.92,-0.74) node[anchor=north west] {5cm};
            \draw (2.06,0.92) node[anchor=north west] {6cm};
            \draw (5.72,0.4) node[anchor=north west] {5cm};
            \draw (7.62,-1.22) node[anchor=north west] {12cm};
            \draw (9,1.3) node[anchor=north west] {13cm};
        }
    }
}

\slide{exo}{
    Pour les triangles $EDF$ et $GHI$, répondre aux questions suivantes :
    \begin{enumerate}\setItemColor{act}
        \item Ce triangle respect-ils l'égalité de Pythagore ?
        \item En utilisant vos connaissances sur le théorème de Pythagore,
        pouvez-vous conclure si le triangle est rectangle ou non ?
        Justifiez votre réponse en précisant l'outil de logique utilisé.
    \end{enumerate}
}

\slide{cr}{
    \ssec
    \bvspace{-0.5cm}
    \ctr{du théorème de Pythagore}{
        Dans un triangle $ABC$.
        \Sialors{$AB^2 \neq AC^2 + BC^2$}{le triangle $ABC$ n'est pas rectangle en $C$}
    }
    \bvspace{-0.75cm}
    \expl{}{
        Soit $NEZ$ est un triangle tel que : $NE = \Lg{8}, EZ = \Lg{16}\et ZN = \Lg{14}$.\\
        Démontrer que le triangle n'est pas rectangle.\\
        \awsr[5]{
            \begin{itemize}
                \item $NE$ est le plus long coté, il sagirait donc de l'hypothénus si le triangle était rectangle.
                \item D'une part :$EZ^2 = 16^2 = 256$
                \item D'autre part : $NE^2 + ZN^2 = 8^2 + 14^2 = 64 + 196= 260$
                \item Alors : $EZ^2 \neq NE^2 + ZN^2$
                \item D'après la contraposée du théorème de Pythagore le triangle $NEZ$ n'est pas rectangle.
            \end{itemize}
        }
    }
}

\bookSlide{29p431}[12cm]

\bsec{Réciproque du théorème de Pythagore}

\slide{exo}{
    \act{}{
        On va démontrer pour un exemple que la réciproque du théorème de Pythagore est vraie.
        Soit $ABC$ un triangle tel que : $AB = \Lg{5}; AC = \Lg{4}; BC = \Lg{5}$.
        \begin{enumerate}
            \item Tracer le triangle $ABC$.
            \item Verifier si $AB^2 = AC^2 + BC^2$. Est-ce que le triangle $ABC$ peut être rectangle?
            \item Contruire une droite perpendiculaire à la droite $(BC)$.
            \item Placer un point $D$ sur cette perpendiculaire tel que $DC = AC$
            et $D$ soit placé à «l'opposé» de $A$. \saveenumi
        \end{enumerate}
    }[\href{https://clg-monnet-briis.ac-versailles.fr/La-reciproque-du-theoreme-de-Pythagore}{Collège Jean Monnet}]
}

\slide{exo}{
    \begin{enumerate} \loadenumi[act]
        \item Quelle est la nature du triangle $BCD$?
        \item Calculer la longueur $BD$.
        \item Comparer les triangles $ABC$ et $BCD$.
        \item Conclure sur la nature du triangle $ABC$.
    \end{enumerate}
    De manière similaire on pourait prouver que la réciproque du théorème de Pythagore est toujours vraie.
}

\slide{cr}{
    \ssec

    \bvspace{-0.5cm}

    \rcp{du théorème de Pythagore}{
        Dans un triangle ABC.
        \Sialors{$AB^2 = AC^2 + BC^2$}
        {le triangle $ABC$ est rectangle en $C$}
    }

    \bvspace{-0.75cm}

    \expl{}{
        Soit $CGT$ est un triangle tel que : $CG = \Lg{45}, GT = \Lg{28}\et TC = \Lg{53}$.\\
        Démontrer que le triangle $CGT$ est pas rectangle.\\
        \awsr[6]{
            \begin{itemize}
                \item $TC$ est le plus long coté, il sagirait donc de l'hypothénus si le triangle était rectangle.
                \item D'une part :$TC^2 = 53^2 = 256$
                \item D'autre part : $NE^2 + ZN^2 = 8^2 + 14^2 = 64 + 196= 260$
                \item Alors : $EZ^2 \neq NE^2 + ZN^2$
                \item D'après la réciproque du théorème de Pythagore le triangle $CGT$ est rectangle en $G$.
            \end{itemize}
        }
    }
}

\bookSlide{27p431,26p431,36p432}[7cm][2]

\bookSlide{35p432,52p435}[6cm][2]

% \setSeq{6}{Géométrie dans l'espace - Solides}
\setGrade{6e}
\def\imgPath{enseignement/6e/geometrie-dans-l-espace/solides/}

\obj{
    \item Reconnaître des solides (pavé droit, cube, cône et cylindre).
    \item Identifier les caractéristiques de différents solides :
    sommets, faces et arêtes.
    \item Représenter un cube et un pavé droit.
}


\slide{}{}
% % VARIABLES %%%
\setSeq{5}{Nombres - Decimaux}
\setGrade{6e}

\def\imgPath{enseignement/6e/nombres/decimaux/}

\def\ym{\href{https://www.maths-et-tiques.fr/telech/19Nombres1.pdf}{Yvan Monka}}

% https://www.maths-et-tiques.fr/telech/19Nombres2.pdf

\obj{
    \item Utiliser une fraction et en donner progressivement le statut de nombre.
    \item Utiliser et représenter les nombres décimaux jusqu'à trois décimales.
    \item Ajouter, soustraire et multiplier des nombres décimaux.
    \item Résoudre des problèmes relevant des structures additives et multiplicatives en mobilisant une ou plusieurs étapes de raisonnement.
}

\scn{Découvrir les fractions décimales}

\bsubsec{Nombres décimaux}

\slide{exo}{
    \act{}{
        \multiColEnumerate{1}{
            \item $\pow{10}{2} = \awsr{\np{\powTenPositive{2}}}$
            \item $\pow{10}{3} = \awsr{\np{\powTenPositive{3}}}$
            \item $\pow{10}{6} = \awsr{\np{\powTenPositive{6}}}$
            \item $\powBrace{10}{15} = \awsr{\np{\powTenPositive{15}}}$
            \item $\powBrace{10}{100} = \awsr{\powTenBrace{100}}$
        }
    }
}

\slide{cr}{
    \ssubsec

    \vc{}{
        Une \key{puissance de 10} est le résultat d'un produit dont tous les facteurs sont $10$.
    }

    \expl{}{
        \Tableau[%
            DoubleEntree,
            Stretch=1.5,
            Couleur=gradeColor!15,
            LegendesH={$100$,$2$,$\np{1001}$,$\np{100000}$,$\np{200}$,$\np{10}$},
            LegendesV={Puissance de 10 ?},
            Largeur=2cm
        ]{\awsr{Oui},\awsr{Non},\awsr{Non},\awsr{Oui},\awsr{Non},\awsr{Oui}}
    }
}

\slide{cr}{
    \pr{}{
        Pour $n$ un nombre entier, on a :
        \begin{align*}
            \powBrace{10}{n} = \awsr{\powTenBrace{n}}
        \end{align*}
    }

    \expl{}{
        \multiColEnumerate{2}{
            \item $\pow{10}{3} = \powTenPositive{3}$
            \item $\pow{10}{6} = \powTenPositive{6}$
        }
    }
}

\slide{cr}{
    \df{}{
        On appelle \key{fraction décimale} une fraction dont le \key{dénominateur est une puissance de $10$}.
    }

    \expl{}{
        \Tableau[%
            DoubleEntree,
            Stretch=1.5,
            Couleur=gradeColor!15,
            LegendesH={$\frac{1}{10}$,$\frac{1}{2}$,$\frac{5}{10}$,$\frac{1}{\np{1000}}$,$\frac{1}{\np{30000}}$,$\frac{546985}{\np{10000000}}$},
            LegendesV={Fraction décimale ?},
            Largeur=2cm
        ]{\awsr{Oui},\awsr{Oui},\awsr{Non},\awsr{Oui},\awsr{Non},\awsr{Oui}}
    }
}

\slide{exo}{
    \act{}{
        Ecrire les nombres suivants sous forme de fractions décimales.
        \multiColEnumerate{2}{
            \item $\np{3.2} = \awsr{\frac{32}{10}}$
            \item $\np{10.2} = \awsr{\frac{102}{10}}$
            \item $\np{0.03} = \awsr{\frac{3}{100}}$
            \item $\np{0.0001} = \awsr{\frac{1}{1000}}$
            \item $6 \div 10 \div 10 = \awsr{\frac{6}{100}}$
            \item $32 \div 10 \div 10 \div 10 \div 10 = \awsr{\frac{32}{10000}}$
        }
    }
}

\slide{cr}{
    \df{}{
        On appelle \key{nombre décimal}, un nombre pouvant s'écrire sous forme de fraction décimale.
    }

    \expl{}{
        \Tableau[%
            DoubleEntree,
            Stretch=1.5,
            Couleur=gradeColor!15,
            LegendesH={$\np{0.6}$,$\np{13.2}$,$\frac{1}{10}$,$\frac{1}{2}$,$60$,$\frac{1}{3}$,$\frac{30}{3}$,$\pi$,$0$},
            LegendesV={Nombre décimal ?},
            Largeur=2cm
        ]{\awsr{Oui},\awsr{Oui},\awsr{Oui},\awsr{Oui},\awsr{Oui},\awsr{Non},\awsr{Oui},\awsr{Non},\awsr{Oui}}
    }
}

\slide{cr}{
    \pr{}{Pour $n$ un nombre entier, on a :
    \begin{align*}
        1 \repeatBrace{\div 10}{n}[quotients]
        = \frac{1}{\powTenBrace{n}}
        = \underbrace{0, 0 ... 0}_{n \textrm{ zéros}} 1
    \end{align*}
    }

    \expl{}{
        \multiColEnumerate{1}{
            \item $1 \div 10 \div 10 = \frac{1}{\awsr{100}} = \awsr{\np{0.01}}$
            \item $\awsr{1 \div 10 \div 10 \div 10 \div 10} = \frac{1}{10000} = \awsr{\np{0.0001}}$
            \item $1 \awsr{\div 10 \div 10 \div 10} = \frac{1}{\awsr{1000}} = \awsr{\np{0.001}}$
        }
    }
}

\scn{Décomposer un nombre décimal en fractions décimales}

% % VARIABLES %%%
\setSeq{5}{Proportionnalité - Tableaux et graphiques}
\setGrade{4e}
\def\imgPath{enseignement/4e/theoreme-de-pythagore/contraposé-et-reciproque/}
% \setboolean{answer}{true}
\def\ym{\href{https://www.maths-et-tiques.fr/telech/19Proport1.pdf}{Yvan Monka}}
% \forStudent
% \setboolean{demonstration}{false}
%%

\def\cp{coefficient de proportionnalité}
% \obj{
%     \item Reconnaitre sur un graphique une situation de proportionnalité ou de non proportionnalité.
%     \item Calcule d'une quatrième proportionnelle.
%     \item Utiliser une formule liant deux grandeurs dans une situation de proportionnalité.
%     \item Résoudre des problèmes en utilisant la proportionnalité dans le cadre de la géométrie.
% }

\renewcommand{\arraystretch}{1.5}

\avspace{0.1cm}

\bsec{Grandeurs proportionnelles}

\df{Grandeurs proportionnelles}{%
    Deux grandeurs sont dites \key{proportionnelles}
    lorsque les valeurs de l'une sont obtenues en multipliant les valeurs de l'autre par un même nombre non nul,
    appelé \key{coefficient de proportionnalité}.
}

\expl{}{
    \begin{tabular}{|>{\bfseries}c|*{4}{c|}} % Colonne en gras pour la première colonne
        \hline
        \rowcolor{gray!15} 
        Grandeur 1 & coté & coté & rayon & tension \\ \hline
        Grandeur 2  & périmètre du carré & aire du carré & périmètre du cercle & intensité \\ \hline
        coefficient?  & \awsr{$4$} & \awsr{non} & \awsr{$2\pi$} & \awsr{$R$} \\ \hline
    \end{tabular}
    % \begin{itemize}
    %     \item La longueur du coté d'un carré et sont périmètre avec \cp{} : $4$ car $\mathcal{P} = 4 \times c$.
    %     \item 
    % \end{itemize}
}

\bsec{Tableau de proportionnalité}

% \pr{Coefficient de proportionnalité}{
%     \Sialors{on est dans un tableau de proportionnalité}
%     {on peu passer d'une ligne à l'autre en multipliant par un \key{\cp}}
% }

Pour un tableau de proportionnalité : \propTable{a}{c}{b}{d} avec $a,b,c,d$ des nombres.

\pr{}{
    On peu passer d'une ligne à l'autre en multipliant par un \cp.
}

\expl{}{\vspace{-0.75cm}
    \multiColEnumerate{2}{
        \item \Propor[Stretch=1.5, Simple]{1/2.5,2/5,5/12.5}
        \FlechesPD{1}{2}{$\times\awsr{2}$}
        \FlechesPG{2}{1}{$\div\awsr{2}$}
        \item \Propor[Stretch=1.5, Simple]{120/12,3/0.3}
        \FlechesPD{1}{2}{$\times\awsr{\frac{1}{10}}$}
        \FlechesPG{2}{1}{$\div\awsr{\frac{1}{10}}$}
    }
}

\pr{Egalité des produits en croix}{
    On a l'égalité: $a \times d = b \times c$.
}

\newpage

\expl{}{
    Utiliser l'égalité des produits en croix pour vérifier si on a bien proportionnalité.
    \vspace{-0.75cm}\multiColEnumerate{2}{
        \item \Propor[Stretch=1.5, Simple]{15/36,1.2/2}
        \item \Propor[Stretch=1.5, Simple]{6/1.5,3/0.5}
    }\vspace{-0.75cm}
    \awsr[5]{
        \begin{enumerate}
            \item $15 \times 2 = 30$ et $36 \times \np{1.2} = 30$
            alors on égalité des produits en croix : $15 \times 2 = 36 \times \np{1.2}$,
            il sagit donc d'un tableau de proportionnalité.
            \item $6 \times \np{0.5} = 3$ et $\np{1.5} \times 3 = 4.5$
            alors on n'a pas égalité des produits en croix : $6 \times \np{0.5} \neq \np{1.5} \times 3 = 4.5$,
            il ne sagit donc pas d'un tableau de proportionnalité.
        \end{enumerate}
    }
}

\cor{Egalité des quotients}{
    On a aussi : $\frac{a}{b} = \frac{c}{d}$
}

\demo{}{
    On a : $\frac{a}{b} = \frac{ a\times d}{b \times d} = \frac{ \awsr{b \times c} }{b \times d}$ d'après l'égalité des produits en croix.\\
    Or $\frac{ b \times c }{b \times d} = \awsr{\frac{c}{d}}$.
    Alors $\frac{a}{b} = \awsr{\frac{c}{d}}.$
}[\href{https://pedagogie.ac-toulouse.fr/mathematiques/system/files/2023-03/demonstration_produits_en_croix.pdf}{Académie de Toulouse}]

\expl{}{
    Utiliser l'égalité des quotients pour vérifier si on a bien proportionnalité.
    \vspace{-0.75cm}\multiColEnumerate{2}{
        \item \Propor[Stretch=1.5, Simple]{10/12,5/6,20/23}
        \item \Propor[Stretch=1.5, Simple]{2/3,4/6,6/9}
    }\vspace{-0.75cm}
    \awsr[5]{
        \begin{enumerate}
            \item $\frac{10}{12} = \frac{20}{24} \neq \frac{20}{23}$
            alors on n'a pas égalité des quotients,
            il ne sagit donc pas d'un tableau de proportionnalité.
            \item $\frac{2}{3} = \frac{4}{6} = \frac{6}{9}$
            alors on égalité des quotients,
            il sagit donc d'un tableau de proportionnalité.
        \end{enumerate}
    }
}

% \ctr{}{
%     \Sialors{$\frac{a}{b} \neq \frac{c}{d}$}{l'égalité des quotients n'est pas respécté et on a pas proportionnalité}
% }

\mthd{Calcul 4e proportionnelle}{
    Si l'on connait 3 valeurs par exemple $b,c,d$.\\
    On peu calculer $a$ avec l'égalité $a = \awsr{\frac{b \times c}{d}}$.
}

\demo{}{
    En partant de l'égalité des produits en croix : $a \times d = b \times c$.\\
    Alors $a$ est le nombre qui multiplié par \awsr{$d$} done \awsr{$b \times c$}.\\
    D'après la définition du quotient : $a = \awsr{\frac{b \times c}{d}}$.
}

\expl{Compléter les tableaux de proportionnalité suivants}{
    \def\cW{2.5cm}
    \multiColEnumerate{2}{
        \item \begin{tabular}{|C{\cW}|C{1cm}|}
            \hline
            \np{9.6} & 3 \\ \hline
            \awsr{$\frac{\np{9.6}\times2}{3} = \np{6.4}$} & 2 \\ \hline
        \end{tabular}
        \item \begin{tabular}{|C{1cm}|*{2}{C{\cW}|}}
            \hline
            3 & 5 & \awsr{$\frac{\np{21.7}\times5}{7} = \np{15.5}$} \\ \hline
            \np{4.2} & \awsr{$\frac{\np{4.2}\times6}{3} = 7$} & \np{21.7} \\ \hline
        \end{tabular}
    }
}

\bsec{Représentation graphique}

\pr{Représentation graphique}{
    Sur un graphique, une situation de proportionnalité est représentée par des points alignés
    avec l'origine.
}[\ym]

\expl{}{
    Chaque graphique suivant représente-t-il une situation de proportionnalité ?
    \def\repere{%
        \tkzInit[xmin=0,xmax=8,ymin=0,ymax=8]
        \tkzGrid[sub,color=gradeColor!50!white,subxstep=1,subystep=1]        
        \tkzLabelX[step=2]
        \tkzLabelY[step=2]
        \tkzDrawY[step=1]
        \tkzDrawX[step=1]
    }
    \def\size{0.55}\def\crossWidth{0.25mm}
    \vspace{-0.75cm}
    \multiColItemize{3}{
        \item[]\ctikz[\size]{
            \repere
            \node at (4,9) {\cir[gradeColor]{1}};
            \drawPoint{}{2}{2.4}[Red]
            \drawPoint{}{4}{4.8}[Red]
            \drawPoint{}{6.5}{7.8}[Red]
            \ifthenelse{\boolean{answer}}{\draw[answer] (-1,-1.2) -- (7.5,9);}{}
        }
        \item[]\ctikz[\size]{
            \node at (4,9) {\cir[gradeColor]{2}};
            \repere
            \drawPoint{}{1}{1.3}[Red]
            \drawPoint{}{3}{3}[Red]
            \drawPoint{}{6}{7.8}[Red]
            \ifthenelse{\boolean{answer}}{\draw[answer] (-1,-1.3) -- (9/1.3,9);}{}
        }
        \item[]\ctikz[\size]{
            \repere
            \node at (4,9) {\cir[gradeColor]{3}};
            \drawPoint{}{1}{2}[Red]
            \drawPoint{}{4}{5}[Red]
            \drawPoint{}{6}{7}[Red]
            \drawPoint{}{7}{8}[Red]
            \ifthenelse{\boolean{answer}}{\draw[answer] (-1,0) -- (8,9);}{}
        }
    }
    \awsr[6]{Seuls les points du graphique \cir[gradeColor]{1} sont alignés avec l'origine.
    Ainsi, parmi les trois graphiques,
    c'est le seul qui représente une situation de proportionnalité.}
}



% \slide{qf}{
%     Les situations présentées dans ces tableaux sont-elles proportionnelles ?
%     \multiColEnumerate{2}{
%         \item \begin{center}
%             \Propor[Simple,
%             Math,
%             Stretch=1.25,%
%             ]{12/3,16/4,40/10}
%         \end{center}
%         \item \begin{center}
%             \Propor[Simple,
%             Math,
%             Stretch=1.25,%
%             ]{15/5,9/3,20/6}
%         \end{center}
%     }
% }

% \slide{qf}{\calculator \\ Completer les tableaux suivants :
%     \multiColEnumerate{3}{
%         \item \begin{center}
%             \Propor[Simple,
%             Math,
%             Stretch=1.25,%
%             ]{6/5,\awsr{\np{2.4}}/2}
%         \end{center}
%         \item \begin{center}
%             \Propor[Simple,
%             Math,
%             Stretch=1.25,%
%             ]{\np{237.6}/\awsr{66},\np{46.8}/13}
%         \end{center}
%         \item \begin{center}
%             \Propor[Simple,
%             Math,
%             Stretch=1.25,%
%             ]{\awsr{12}/18,-3/-4.5}
%         \end{center}
%     }
% }

% \slide{qf}{
%     \nullsubsec{}{
%         Sachant que huit briques de masse identique pèsent 13,6 kg, calcule la masse de six de ces
%         briques.
%     }[\afa{4e}[6]]
% }

% \slide{qf}{
%     \nullsubsec{}{
%         \begin{enumerate}
%             \item Sachant que la longueur $\mathcal{P}$ d'un cercle
%             est proportionnelle à son rayon $r$
%             avec un \cp $2\pi$.
%             Donnez la formule permettant de calculer $\mathcal{P}$ en fonction de $r$.
%             \item Sachant que la tension $U$ aux bornes d'une résistance
%             est proportionnelle à l'intensité $I$ du courant qui la traverse
%             avec un \cp égal à la valeur de la résistance $R$.
%             Donnez la formule permettant de calculer $U$ en fonction de $I$.
%         \end{enumerate}
%     }
% }

% % VARIABLES %%%
\setSeq{4}{Nombres Relatifs - Sommes et différences}
\setGrade{5e}

% \setboolean{answer}{false}
% \setboolean{newPageOnSlide}{true}

% \forPrint

\def\imgPath{enseignement/5e/nombres-relatifs/sommes-et-differences/}
\def\ym{\href{https://www.maths-et-tiques.fr/telech/19Nomb_rel2.pdf}{Yvan Monka}}
% Yvan Monka RÈGLES DE CALCUL : https://www.maths-et-tiques.fr/telech/19Calcul_num.pdf
%%

\obj{
    \item Traduire un enchaînement d'opérations à l'aide d'une expression avec des parenthèses.
    \item Effectuer mentalement, à la main ou l'aide d'une calculatrice un enchaînement d'opérations.
    \item Additionner et soustraire des nombres décimaux relatifs.
    \item Résoudre des problèmes faisant intervenir des nombres décimaux relatifs et des fractions simples.
    \item Utiliser la notion de fractions quotients dans des calculs.
}

\scn{Rappel sur les règles opératoires}

\slide{qf}{
    \multiColEnumerate{2}{
        \item $8 + 9 - 7 = \awsr{10}$
        \item $8 \times (2 + 10) = \awsr{96}$
        \item $12 + 11 \times 10 = \awsr{122}$
        \item $7 \div (7 - 3 + 6) = \awsr{\frac{7}{10} = 0,7}$
        \item $9 - 8 + 1 = \awsr{2}$
    }
}

\bsec{Opérations}
\bsubsec{Règles opératoires}

\slide{cr}{
    \sseq\ssec\ssubsec

    \rl{}{
        Les calculs se font dans l'ordre des priorités suivant:%
        \begin{enumerate}
            \item La multiplication et la division
            \item L'addition et la soustraction
        \end{enumerate}
    }
}

\slide{cr}{
    \rl{}{
        En cas d'opérations de mêmes priorités, on effectue les opérations de gauche à droite.
    }

    \expl{}{
        \multiColEnumerate{1}{
            \item $3 - 2 + 3 = \awsr{1 + 3 = 4}$ 
            \item $19 - 6\times 3 = \awsr{19 - 18 =  1}$
            \item $3 + \np{3.2} \times 2 - 4 = \awsr{3 + \np{6.4} -4 = \np{9.4} - 4 = \np{5.4}}$
        }
    }
}

\slide{cr}{
    \rl{}{
        On commence par effectuer les calculs entre parenthèses.
    }

    \expl{}{
        \multiColEnumerate{1}{
            \item $(1 + 2) \times 21 = \awsr{3 \times 21 = 63}$
            \item $(11 \times 3) + (15 \div 2) = \awsr{33 + 7.5 = 40.5}$
            \item $((13 - (3 - 2)) + 2) = \awsr{(13 - 1) + 2 = 12 + 2 = 14}$
        }
    }
}

\scn{Rappel sur le vocabulaire opératoire}

\bsubsec{Vocabulaire opératoire}

% \newpage

\slide{cr}{
    \ssubsec
    \bvspace{-0.5cm}
    \vc{}{
        On connait quatres types d'opérations :
        \begin{itemize}
            \item L'\key{addition} permet de calculer la \key{somme} de deux \key{termes}.
            \item La \key{soustraction}  permet de calculer la \key{différence} entre deux \key{termes}.
            \item La \key{multiplication} permet de calculer la \key{produit} de deux \key{facteurs}.
            \item La \key{division} permet de calculer la \key{quotient} de deux \key{nombres}.
        \end{itemize}
    }
}

\slide{cr}{
    \vc{}{Dans un calcul,
    le type de la dernière opération effectuée détermine le nom donné au calcul dans son ensemble.}
    \bvspace{-0.5cm}
    \expl{Nommer les calculs suivants}{
        \multiColEnumerate{1}{
            \item $1,6 + 4$ est \awsr{la somme} de \awsr{$1,6$ et $4$}.
            \item $(\frac{2}{6} + 3) \times 9$ est \awsr{le produit } de \awsr{$\frac{2}{6} + 3$ et $9$}.
            \item $6,6 + 1 \times 8$ est \awsr{la somme de $6,6$ par $1 \times 8$}.
            \item $\frac{2}{6} + 3 - 9$ est \awsr{la différence entre $\frac{2}{6}$ et $9$}.
            \item $\pi \div (3 - 9)$ est \awsr{le quotient de $\pi$ par $(3 - 9)$}.
        }
    }
}

\scn{Tournois \icon{RELATIvs/logo}}

\scn{Cours sur la somme et différence de nombres relatifs}

\bsec{Opérations sur les nombres relatifs}
\bsubsec{Somme}

\def\daz{distance à zéro}

\slide{cr}{
    \ssec\ssubsec
    \pr{}{
        Pour la somme deux nombres relatifs :
        \begin{itemize}
            \item S'ils ont le \key{même signe} :
            \begin{itemize}
                \item Le signe du résultat est identique à celui des deux nombres.
                \item La \daz{} du résultat est obtenue en additionnant les \daz{} des deux nombres.
            \end{itemize} 
            \item S'ils ont des \key{signes opposés} :
            \begin{itemize}
                \item Le signe du résultat est celui du nombre ayant la plus grande \daz.
                \item La \daz{} du résultat est obtenue en soustrayant la plus petite \daz{} de la plus grande.
            \end{itemize}
        \end{itemize}
    }
}

\slide{exo}{
    \exo{Effectuer les calculs suivants}{
        \multiColEnumerate{2}{
            \item $12 + (-11) + 25 + (-17)$
            \item $14 + (-7) + 2 + (-3,75) + (-5,25)$
            \item $(-2,1) + (-9) + 6,4 + (-8,3)$
            \item $3,6 + 30 + (-6,4) + (-49)$
        }
    }[\iP{5}{2022}[1][56]]
}

\slide{exo}{\bshrink
    \exo{Pyramides de nombres}{
        Complète sachant que chaque nombre est la somme des nombres se trouvant dans les deux cases juste en dessous.
        \multiColEnumerate{2}{
            \item \PyramideNombre[Largeur=1.5cm,Etages=4]{%
            -1,3,-5,-10,%
            ~,~,~,%
            ~,~,%
            ~}
            \item \PyramideNombre[Largeur=1.5cm,Etages=4]{%
            6,~,~,-15,%
            -3,~,-5,%
            ~,~,%
            ~}\saveenumi
        }
    }
}

\slide{exo}{
    \begin{enumerate}\loadenumi[exo]
        \item \PyramideNombre[Largeur=1.5cm,]{%
        -3,~,~,~,~,%
        -8,~,~,~,%
        16,~,~,%
        -33,~,%
        71
        }
    \end{enumerate}
}

\scn{Fraction quotient}

\scn{Moyenne de nombres relatifs}

\slide{exo}{

}

% \slide{exo}{
%     \act{}{

%     }
% }

% % VARIABLES %%%
% \def\authors{\jules \ et \href{http://www.cellulegeometrie.eu/documents/pub/pub_14.pdf}{la Haute École en Hainaut}}
\setGrade{6e}
\def\assignmentNameWidth{6cm}
\tp{Scratch - Polygones}
\def\imgPath{enseignement/6e/geometrie-plane/polygones/}
%%

% DOC https://ctan.math.illinois.edu/macros/latex/contrib/scratch3/scratch3-fr.pdf

\setscratch{scale=.75}

\def\block{{\setscratch{scale=.5}\begin{scratch}\blockmove{\Large bloc}\end{scratch} }}

\newcommand{\scr}[1]{\begin{scratch}#1\end{scratch}}

\definecolor{smotion}{HTML}{4C97FF} % #4C97FF
\definecolor{slooks}{HTML}{9966FF} % #9966FF
\definecolor{ssound}{HTML}{D65CD6} % #D65CD6
\definecolor{sevents}{HTML}{FFD500} % #FFD500
\definecolor{scontrol}{HTML}{FFAB19} % #FFAB19
\definecolor{ssensing}{HTML}{4CBFE6} % #4CBFE6
\definecolor{soperators}{HTML}{6DB26E} % #6DB26E
\definecolor{svariables}{HTML}{F28011} % #F28011
\definecolor{smyblocks}{HTML}{FF6680} % #FF6680

\def\smotion{\textcolor{smotion}{\faCircle\,Mouvement}} % Déplacement du lutin
\def\slooks{\textcolor{slooks}{\faCircle\,Apparence}} % Modifier l'apparence du lutin ou de la scène
\def\ssound{\textcolor{ssound}{\faCircle\,Son}} % Jouer des sons ou de la musique
\def\sevents{\textcolor{sevents}{\faCircle\,Événement}} % Déclencher des scripts en réponse à des actions
\def\scontrol{\textcolor{scontrol}{\faCircle\,Contrôle}} % Boucles, conditions, et contrôle du flux
\def\ssensing{\textcolor{ssensing}{\faCircle\,Capteur}} % Réagir à des informations extérieures ou internes
\def\soperators{\textcolor{soperators}{\faCircle\,Opérateur}} % Calculs mathématiques et logiques
\def\svariables{\textcolor{svariables}{\faCircle\,Variable}} % Stockage et manipulation de données
\def\smyblocks{\textcolor{smyblocks}{\faCircle\,Mes blocs}} % Création de blocs personnalisés

\def\spen{{\icon{scratch/pen} Stylo}}
\def\spenExtension{{\icon{scratch/pen-extension} Stylo}}
\def\sextensions{{\icon{scratch/extensions} $\lbrack$ Ajouter une extensions $\rbrack$}}
\def\sflag{{\icon{scratch/flag}%
%  Drapeau
}}

% \setscratch{scale=.75}
% \setscratch{print=true}
% \setscratch{fill blocks=true}

\hint{
    \begin{itemize}
        \item Bien lire les indications.
        \item Répondre aux questions sur son cahier d'exercices.
        \item Appeler M. Pesin à la fin de chaques parties.
        \item Enregistre tes productions avec le nom : "NOM.S-tp-polygones-partie-X.ggb"
    \end{itemize}
}

\section{Tracer un carré à l'aide de Scratch} 

\subsection{Propriétés d'un carré} 

\begin{enumerate} 
    \item Quelles sont les propriétés d'un carré ? Note-les. Assure-toi de mentionner : 
    \begin{itemize} 
        \item La longueur des côtés. 
        \item Les angles entre les côtés. 
    \end{itemize} 
    % \item Pourquoi ces propriétés sont-elles importantes pour programmer le tracé d'un carré ? 
\end{enumerate} 

\subsection{Création du programme} 

\begin{enumerate} 
    \item Assemble un script sur \Scratch pour que le lutin dessine un carré : 
    \begin{itemize} 
        \item Utilise le bloc
        \begin{scratch} \blockmove{avancer de \ovalnum{}} \end{scratch}
        pour tracer un côté du carré. 
        \item Utilise le bloc
        \begin{scratch} \blockmove{tourner \turnright{} de \ovalnum{}} \end{scratch}
        pour changer de direction. 
    \end{itemize}

    \hint{Utilise l'extension \spenExtension{} pour faire apparaitre le tracé.}

    \item Combien de fois faudra-t-il répéter ces instructions pour dessiner un carré ? Teste ton hypothèse. 
    \item Afin d'améliore ton programme utilise un bloc
    \begin{scratch} \blockrepeat{répéter \ovalnum{} fois}{} \end{scratch}
    dans lequel tu peux placer les blocs que tu souhaite executer plusieurs fois.
\end{enumerate}

\section{Tracer un triangle équilatéral à l'aide de Scratch} 

\begin{enumerate}
    \item Dessine un triangle équilatéral à main levée.
    \item Note les propriétés d'un triangle équilatéral.
    \item Trace un triangle équilatéral à l'aide de ta règle et ton compas.
    \item Mesure ses angles avec un rapporteur et vérifie si tes mesures confirment les propriétés d'un triangle équilatéral.
    \item Assemble un script dans \Scratch{} pour que le lutin dessine un triangle équilatéral.
    \hint{\begin{enumerate} 
        \item Utilise une boucle pour répéter les instructions nécessaires au tracé. 
        \item Réfléchis attentivement à l'angle de rotation. Imagine que tu es à la place du lutin et que tu suis les instructions de ton programme : comment devrais-tu te déplacer et tourner pour revenir à ta position de départ ? 
    \end{enumerate}} 
\end{enumerate}

% % VARIABLES %%%
% \def\authors{\jules \ et \href{http://www.cellulegeometrie.eu/documents/pub/pub_14.pdf}{la Haute École en Hainaut}}
\setGrade{5e}
\def\assignmentNameWidth{6cm}
\tp{Scratch - Triangles}
\def\imgPath{enseignement/6e/geometrie-plane/frises/}
%%

% DOC https://ctan.math.illinois.edu/macros/latex/contrib/scratch3/scratch3-fr.pdf

\setscratch{scale=.75}

\def\block{{\setscratch{scale=.5}\begin{scratch}\blockmove{\Large bloc}\end{scratch} }}

\newcommand{\scr}[1]{\begin{scratch}#1\end{scratch}}

\definecolor{smotion}{HTML}{4C97FF} % #4C97FF
\definecolor{slooks}{HTML}{9966FF} % #9966FF
\definecolor{ssound}{HTML}{D65CD6} % #D65CD6
\definecolor{sevents}{HTML}{FFD500} % #FFD500
\definecolor{scontrol}{HTML}{FFAB19} % #FFAB19
\definecolor{ssensing}{HTML}{4CBFE6} % #4CBFE6
\definecolor{soperators}{HTML}{6DB26E} % #6DB26E
\definecolor{svariables}{HTML}{F28011} % #F28011
\definecolor{smyblocks}{HTML}{FF6680} % #FF6680

\def\smotion{\textcolor{smotion}{\faCircle\,Mouvement}} % Déplacement du lutin
\def\slooks{\textcolor{slooks}{\faCircle\,Apparence}} % Modifier l'apparence du lutin ou de la scène
\def\ssound{\textcolor{ssound}{\faCircle\,Son}} % Jouer des sons ou de la musique
\def\sevents{\textcolor{sevents}{\faCircle\,Événement}} % Déclencher des scripts en réponse à des actions
\def\scontrol{\textcolor{scontrol}{\faCircle\,Contrôle}} % Boucles, conditions, et contrôle du flux
\def\ssensing{\textcolor{ssensing}{\faCircle\,Capteur}} % Réagir à des informations extérieures ou internes
\def\soperators{\textcolor{soperators}{\faCircle\,Opérateur}} % Calculs mathématiques et logiques
\def\svariables{\textcolor{svariables}{\faCircle\,Variable}} % Stockage et manipulation de données
\def\smyblocks{\textcolor{smyblocks}{\faCircle\,Mes blocs}} % Création de blocs personnalisés

\def\spen{{\icon{scratch/pen} Stylo}}
\def\spenExtension{{\icon{scratch/pen-extension} Stylo}}
\def\sextensions{{\icon{scratch/extensions} $\lbrack$ Ajouter une extensions $\rbrack$}}
\def\sflag{{\icon{scratch/flag}%
%  Drapeau
}}

% \setscratch{scale=.75}
% \setscratch{print=true}
% \setscratch{fill blocks=true}

\hint{
    \begin{itemize}
        \item Bien lire les indications.
        \item Répondre aux questions sur son cahier d'exercices.
        \item Appeler M. Pesin à la fin de chaque partie.
        \item Enregistre tes productions avec le nom : "NOM.S-tp-frises-partie-X.ggb"
    \end{itemize}
}

\section{Tracer un triangle équilatéral à l'aide de Scratch} 

\begin{enumerate}
    \item Dessine un triangle équilatéral à main levée.
    \item Note les propriétés d'un triangle équilatéral.
    \item Trace un triangle équilatéral à l'aide de ta règle et ton compas.
    \item Mesure ses angles avec un rapporteur et vérifie si tes mesures confirment les propriétés d'un triangle équilatéral.
    \item Assemble un script dans \Scratch{} pour que le lutin dessine un triangle équilatéral.
    \item Assemble un script dans \Scratch{} pour que le lutin dessine un triangle équilatéral. 
    \hint{\begin{enumerate} 
        \item Utilise une boucle pour répéter les instructions nécessaires au tracé. 
        \item Réfléchis attentivement à l'angle de rotation. Imagine que tu es à la place du lutin et que tu suis les instructions de ton programme : comment devrais-tu te déplacer et tourner pour revenir à ta position de départ ? 
    \end{enumerate}} 
    
\end{enumerate}

\section{Frises}

\begin{enumerate}
    \item Ecrit un programme permettant de dessiner cette frise :
    \ctikz[0.85]{
        % Données des triangles
        \def\side{2} % Longueur du côté du triangle
        \def\spacing{1} % Espacement entre les triangles
        % Calcul des coordonnées
        \foreach \i in {0, 1, 2, 3} {
            % Position de la base gauche du triangle
            \pgfmathsetmacro{\xBase}{\i * (\side + \spacing)}
            \pgfmathsetmacro{\yBase}{0}
            % Dessin du triangle équilatéral
            \draw (\xBase, \yBase) -- 
                ({\xBase + \side}, \yBase) -- 
                ({\xBase + \side / 2}, {\yBase + \side * sqrt(3) / 2}) -- 
                cycle;
        }
    }
    \item Et cette frise :
    \ctikz[0.55]{
        \draw[] (0,0) grid (5,1);
    }
    \item Et ce pavage :
    \ctikz[0.5]{
        \draw[] (0,0) grid (5,8);
    }
\end{enumerate}



% % VARIABLES %%%
% \def\authors{\jules \ et \href{http://www.cellulegeometrie.eu/documents/pub/pub_14.pdf}{la Haute École en Hainaut}}
% \setGrade{6e}
\setTitle{Scratch - Prise en main}
% DOC https://ctan.math.illinois.edu/macros/latex/contrib/scratch3/scratch3-fr.pdf

\setscratch{scale=.75}

\def\block{{\setscratch{scale=.5}\begin{scratch}\blockmove{\Large bloc}\end{scratch} }}

\newcommand{\scr}[1]{\begin{scratch}#1\end{scratch}}

\definecolor{smotion}{HTML}{4C97FF} % #4C97FF
\definecolor{slooks}{HTML}{9966FF} % #9966FF
\definecolor{ssound}{HTML}{D65CD6} % #D65CD6
\definecolor{sevents}{HTML}{FFD500} % #FFD500
\definecolor{scontrol}{HTML}{FFAB19} % #FFAB19
\definecolor{ssensing}{HTML}{4CBFE6} % #4CBFE6
\definecolor{soperators}{HTML}{6DB26E} % #6DB26E
\definecolor{svariables}{HTML}{F28011} % #F28011
\definecolor{smyblocks}{HTML}{FF6680} % #FF6680

\def\smotion{\textcolor{smotion}{\faCircle\,Mouvement}} % Déplacement du lutin
\def\slooks{\textcolor{slooks}{\faCircle\,Apparence}} % Modifier l'apparence du lutin ou de la scène
\def\ssound{\textcolor{ssound}{\faCircle\,Son}} % Jouer des sons ou de la musique
\def\sevents{\textcolor{sevents}{\faCircle\,Événement}} % Déclencher des scripts en réponse à des actions
\def\scontrol{\textcolor{scontrol}{\faCircle\,Contrôle}} % Boucles, conditions, et contrôle du flux
\def\ssensing{\textcolor{ssensing}{\faCircle\,Capteur}} % Réagir à des informations extérieures ou internes
\def\soperators{\textcolor{soperators}{\faCircle\,Opérateur}} % Calculs mathématiques et logiques
\def\svariables{\textcolor{svariables}{\faCircle\,Variable}} % Stockage et manipulation de données
\def\smyblocks{\textcolor{smyblocks}{\faCircle\,Mes blocs}} % Création de blocs personnalisés

\def\spen{{\icon{scratch/pen} Stylo}}
\def\spenExtension{{\icon{scratch/pen-extension} Stylo}}
\def\sextensions{{\icon{scratch/extensions} $\lbrack$ Ajouter une extensions $\rbrack$}}
\def\sflag{{\icon{scratch/flag}%
%  Drapeau
}}

% \setscratch{scale=.75}
% \setscratch{print=true}
% \setscratch{fill blocks=true}
\colorlet{gradeColor}{scratch}
\emptyBackground
%%



% \hint{
%     \begin{itemize}
%         \item Appeler M. Pesin à la fin de chaque partie.
%         \item Enregistrer les productions avec le nom : "NOM.S-tp-polygones-partie-X.ggb"
%     \end{itemize}
% }

\Scratch est un logiciel de programmation par \block, il va t'aider à découvrire l'algorithmique.

\section{Découvrir l'interface}

\begin{enumerate}
    \item Ouvre le logiciel \Scratch.
    \item Observe l'interface :
    \begin{enumerate}
        \item A gauche il y a des \block
        organisés par catégories : \smotion, \slooks, \ssound, etc.
        \item Au centre, il y a la \key{zone de scripts}, où tu pourras assembler les \block pour créer des algorithmes.
        \item À droite, tu vois la \key{scène} , où se déplacera ton personnage (appelé \key{lutin} ou sprite).  
    \end{enumerate}
    \item Teste le fonctionnement de  \Scratch : 
    \begin{itemize} 
        \item Glisse un bloc de la catégorie \smotion
        (par exemple :
        \begin{scratch}\blockmove{avancer de \ovalnum{10}}\end{scratch}
        ) dans la \key{zone de scripts}. 
        \item Clique dessus pour voir le lutin bouger. 
        \item Change la valeur dans le bloc
        (par exemple :
        \begin{scratch}\blockmove{avancer de \ovalnum{100}}\end{scratch}
        ) et clique à nouveau. 
    \end{itemize}
\end{enumerate}

\hint{Si tu fais une erreur, tu peux supprimer un \block en le glissant vers la liste des \block.}

\section{Imbriquer des blocs}

\begin{enumerate}
    \item Places les \block suivants en les connectant dans cet ordre :
        \begin{scratch}
            \blockmove{avancer de \ovalnum{40}}
            \blocklook{dit \ovalnum{Bonjour !}}
            \blockcontrol{attendre \ovalnum{1} secondes}
            \blocklook{dit \ovalnum{Au revoir !}}
            \blockmove{avancer de \ovalnum{60}}
        \end{scratch}
        \hint{La couleur des \block correspond à leur catégorie.}
    \item Clique sur n'importe quel \block placer pour executer ton programme.
\end{enumerate}


\section{Utiliser l'extension Crayon pour tracer des formes}

\begin{enumerate}
    \item Clique sur l'icône \sextensions{} en bas à gauche pour ajouter une \key{extension}.
    \item Sélectionne l'extension \spenExtension.
    De nouveaux blocs, comme
    \begin{scratch}\blockpen{stylo en position d'écriture}\end{scratch}
    et
    \begin{scratch}\blockpen{relever le stylo}\end{scratch}
    seront ajoutés dans la nouvelle catégorie \spen .
    \item Teste les blocs suivants :
    \begin{scratch}
        \blockpen{stylo en position d'écriture}
        \blockmove{avancer de \ovalnum{50}}
        \blockmove{tourner \turnright{} de \ovalnum{45} degrés}
        \blockmove{avancer de \ovalnum{50}}
    \end{scratch}
    % \item Observe ce qui se passe lorsque le crayon est baissé et que le lutin avance. Relève ensuite le crayon pour qu'il cesse de tracer.
\end{enumerate}

\hint{Pour mieux visualiser tes tracés :
tu peux ajuster la taille de ton lutin en utilisant l'option "Taille" située dans l'onglet "Sprite" sous la scène.
}

\section{Automatiser l'exécution}

\begin{enumerate}
    \item Glisse le bloc
    \begin{scratch}\blockinit{quand \greenflag est cliqué}\end{scratch}
    de la catégorie \sevents.
    % \begin{scratch}\blockevent{quand \flag est cliqué}\end{scratch}) dans la \key{zone de scripts}.
    \item Connecte ce bloc à une série d'actions, comme :
    % \begin{center}
        \begin{scratch}
            % \blockcustom{effacer tout}
            \blockmove{aller à x: \ovalnum{0} y: \ovalnum{0}}
            \blockpen{effacer tout}
            \blockpen{stylo en position d'écriture}
            \blockmove{avancer de \ovalnum{100}}
            \blockmove{tourner \turnright{} de \ovalnum{90} degrés}
            \blockmove{avancer de \ovalnum{100}}
        \end{scratch}
    % \end{center}
    \item Clique sur le \sflag{} pour exécuter l'algorithme.
\end{enumerate}

% %%%
\setGrade{6e}
\evaluation{2}[corr]

\def\cwr{\href{https://college-willy-ronis.fr/maths/wp-content/uploads/2020/11/Chap-1-exercices-corriges-6eme.pdf}{College Willy Ronis}}
%%%
% \def\imgPath{enseignement/6e/}

\seqEvaluation{3}{Géométrie plane - distance}{
    Tracer un segment de longueur donnée.
    /2,
    Placer le milieu d'un segment de longueur donnée.
    /1,
    Construire un cercle.
    /3%
    % Déterminer le plus court chemin entre un point et une droite.
    % /0%
}

\seqEvaluation{4}{Nombres - Entiers}{
    Utiliser et représenter les grands nombres entiers.
    / 1,
    Utiliser la division euclidienne.
    / 3,
    Résoudre des problèmes relevant des structures additives et multiplicatives en mobilisant une ou plusieurs étapes de raisonnement.
    / 3,
    % Organiser un calcul en une seule ligne{,} utilisant si nécessaire des parenthèses.
    % / 2,
    Résoudre des calculs en lignes en respectant les priorités opératoires /2%
}

\evalutionEnd[3][2]

\exo{}{Écrire en chiffres :
    \multiColEnumerate{1}{
        \item Neuf-mille-quatre-vingt-quinze : \awsr{$\np{9095}$}
        \item $8\times\np{1000}+2\times10+1\times\np{10000}+7$ : \awsr{$\np{10827}$}
        \item $15$ milliers et $\np{3 234}$ unités : \awsr{$\np{18234}$}
        % \item \Ecriture{4 000 780}
        \item Quatre-milliards-sept-cent-quatre-vingts :
        \awsr{$\np{4 000 000 780}$}
        \item $2$ millions et $2$ soixantaines : \awsr{$\np{2 000 120}$}
    }
}[\cwr]

\exo{Calculs en lignes}{
    Résoudre les opérations suivantes :
    \begin{enumerate}
        \item $211 \times 2 + 12 \times 4 = \awsr[2]{422 + 48 = 470}$ 
        \item $2 \times \np{10 000} + (14 - 9) = \awsr[2]{\np{20 000} + 5 = \np{20 005}}$
        \item $(24 + 2 \times 6) \div 2 = \awsr[2]{(24 + 12) \div 2 = 36 \div 2 = 18}$
    \end{enumerate}
}

\newpage

\def\length{10}
\def\crossWidth{0.3mm}
\def\crossSize{0.15}
\exo{Triplet de cercles}{
    \begin{enumerate}
        \item Trace la demi droite $[AC)$.
        \item Place $B$ sur la demi-droite $[AC)$ tel que $AB = \Lg{\length}$.
        \item Marque le point $O$, milieu du segment $[AB]$.
        \item Trace le cercle de centre $O$ et de rayon $\Lg{\directlua{tex.print(math.floor(\length / 2))}}$.
        \item Trace les cercles de diamètres $[AO]$ et $[OB]$.
    \end{enumerate}
    \ctikz[1]{
        \draw[gradeColor!40] (-7,-3) rectangle (10,12);
        \ifthenelse{\boolean{answer}}{%
            \draw [thick,domain=-3.84:9] plot(\x,{(--7.6992--0.52*\x)/1.76});
            \draw [thick] (0.9550885402945024,4.6567307050870115) circle (5cm);
            \draw [thick] (-1.4424557298527487,3.948365352543506) circle (2.5cm);
            \draw [thick] (3.3526328104417535,5.365096057630517) circle (2.5cm);
            \drawPoint{B}{5.75}{6.07}
            \drawPoint{D}{0.96}{4.66}
            \drawPoint{E}{3.35}{5.37}
            \drawPoint{F}{-1.44}{3.95}
        }{}%
        \drawPoint{A}{-3.84}{3.24}
        \drawPoint{C}{-2.08}{3.76}
    }
}[\sesa{6}{2021}[9][85]]

\newpage

\exo{}{
    Salma range ses 8 000 timbres dans un classeur de 52 pages.
    Combien de timbres contient la dernière page non remplie?
}[\iP{6}{2021}[1][16]]

\answerSec{10}[Réponses][
    Pour trouver la réponse on peut calculer la division euclidienne de 8000 par 52.
    \begin{center}
        \longDivision{8000}{52}
    \end{center}
    La dernière page non remplie contient alors 44 timbres.
]

% \exo{BONUS}{
%     \begin{enumerate}
%         \item Le 17 juin 2345 sera une date très particulière car elle s'écrire : $17\;06\;2345$, c'est à dire avec huit chiffres tous
%         différents. Quelle a été la dernière date à posséder cette propriété, c'est à dire à s'écrire sous la forme d'un nombre à
%         huit chiffres tous différents ?
%     \end{enumerate}
% }[\cwr]

% \answerFill[Réponses][
%     \begin{enumerate}
%         \item le 26 août 1987 qui donne : $26\;07\;1987$
%     \end{enumerate}
% ]

\exo{}{
    Un boulanger achète 5 sacs de \Masse[kg]{25} de farine.  
    Chaque sac lui coûte \Prix{50}.  
    Pour chaque kilo de farine, il peut faire 8 baguettes qu'il vend à \Prix{2} l'unité.
    Si on considère qu'il ne dépense rien pour les autres ingrédients de sa baguette quel bénéfice réalisera-t-il?
}

\answerFill[Réponse][
    \begin{itemize}
        \item $5 \times 25 = 125$ \Masse[kg]{125}
        \item $5 \times 50 = 250$ \Prix{250}
        \item $125 \times 8 = 1250$ 1250 baguettes
        \item $1250 \times 2 = 2500$ \Prix{2500}
        \item $2500 - 250 = 2250$ \Prix{2250}
    \end{itemize}
]

\exo{\bonus Numération Maya}{
    On s'intéresse dans cet exercices à une numération pratiquée dans la civilisation mésoaméricaine maya.
    \begin{enumerate}
        \item \multiColItemize{3}{
            \item \Maya{4} = 4
            \item \Maya{8} = 8	
            \item \Maya{17} = 17
        }
        Que réprésente les symboles \Maya{1} et \Maya{5} ?
        \item Compléter les égalités suivantes :
        \multiColEnumerate{3}{
            \item \Maya{3} = \awsr{3}
            \item \Maya{12} = \awsr{12}
            \item \awsr{\Maya{19}} = 19
        }
        \item Observe les nombres ci-dessous et explique comment sont composé ces nombres.
        \multiColItemize{3}{
            \item \Maya{21} = 21
            \item \Maya{46} = 46	
            \item \Maya{60} = 60
        }
        \item Quelle est la base utilisé dans cette numération ?
        \item Compléter les égalités suivantes :
        \multiColEnumerate{3}{
            \item \Maya{33} = \awsr{33}
            \item \awsr{\Maya{156}} = 156
            \item \Maya{453} = \awsr{453}
        } 
    \end{enumerate}
}
%%%
\setGrade{5e}
\evaluation{2}

\setboolean{answer}{false}
%%%
% \def\imgPath{enseignement/6e/}

\seqEvaluation{3}{Symétrie}{
    Comprendre l'effet des symétries (axiale et centrale) :
    conservation du parallélisme{,} des longueurs et des angles.
    /2,
    Identifier des symétries dans des frises{,} des pavages{,} des rosaces.
    /2%
}

\seqEvaluation{4}{Nombres Relatifs - Sommes et differences}{
    Traduire un enchaînement d'opérations à l'aide d'une expression avec des parenthèses.
    /3,
    Additionner et soustraire des nombres décimaux relatifs.
    /4,
    Résoudre des problèmes faisant intervenir des nombres décimaux relatifs et des fractions simples.
    / 2,
}

\seqEvaluation{}{Compétences générales}{
    Écrire ses calculs
    /1,
    Rédiger des phrases réponses
    /2
}

\exo{Reconnaître des symétries}{
    \def\crossWidth{0.3mm}\def\crossSize{0.15}\def\nodeShift{0.25}
    \ctikz[1.25]{
        \draw[gray!40] (0,0) grid (6,6);
        % Nested loops to generate points with names
        \newcounter{i}\setcounter{i}{1}%
        \foreach \x in {1,...,5} {%
            \foreach \y in {1,...,5} {%
                % Generate names programmatically
                \pgfmathtruncatemacro{\charCode}{64+\thei} % Convert x to letter (A=1, B=2, etc.)
                \edef\pointName{\char\charCode} % Get the letter name
                \drawPoint{\pointName}{\x}{\y}%
                \stepcounter{i}%
            }
        }
    }
    \begin{enumerate}
        \item L'image du segment $[HR]$ par la symétrie de centre $N$ est : \bawsr{le segment $[JT]$}.
        \item Le triangle $QUV$ est l'image de du triangle $SON$ par \bawsr{la symetries de centre $R$}.
        \item Le point \bawsr{$D$} est l'image de point P par la symetrie d'axe $(AG)$.
        \item L'image du quadrilatère $NXQL$ par la symétrie de centre $M$ est : \bawsr{le quadrilatère $LBIN$}.
    \end{enumerate}
}[\sesa{5}{2024}[89]]

\exo{Utiliser les propriétés de la symétrie centrale}{
    On considère quatre points $P$, $O$, $K$ et $E$ tels que :  
    \begin{align*}
        KE = \Lg{10.1}, \quad PO = \Lg{3.6}, \quad EP = \Lg{2.6}, \quad PK = \Lg{5}.
    \end{align*}
    Les points $A$,$S$ et $H$ sont respectivement les images des points $P$,$E$ et $K$ par la symétrie centrale de centre $O$.
    Calculez le périmètre du triangle $ASH$.
}

% \begin{Geometrie}[TypeTrace="Schema"]
%     pair A,B,C;
%     A=u*(1,1);
%     B-A=u*(4,1);
%     C=rotation(A,B,-70);
%     trace polygone(A,B,C);
% \end{Geometrie}


\ctikz[0.75]{
    \draw[gray!40] (-1,-10) rectangle (18,2);
    \ifthenelse{\boolean{answer}}{%
        \draw [penciline,thick] (0.48,-7.04)-- (7.4,-8.06);
        \draw [penciline,thick] (7.4,-8.06)-- (5.18,-4.68);
        \draw [penciline,thick] (5.18,-4.68)-- (0.48,-7.04);
        \draw [penciline,thick] (11.06,-3.16)-- (15.683121999999992,1.1780524000000023);
        \draw [penciline,thick] (15.683121999999992,1.1780524000000023)-- (9.047820799999997,0.5338484000000019);
        \draw [penciline,thick] (9.047820799999997,0.5338484000000019)-- (11.06,-3.16);
        \draw (6.825316999999991,-6.262503800000003) node[anchor=north west] {\Lg{2.6}};
        \draw (3.2821949999999944,-8.098485200000004) node[anchor=north west] {\Lg{10.1}};
        \draw (2.090417599999996,-5.038516200000002) node[anchor=north west] {\Lg{5}};
        \drawPoint{P}{5.18}{-4.68}
        \drawPoint{O}{8.16}{-4.22}
        \drawPoint{K}{0.48}{-7.04}
        \drawPoint{E}{7.40}{-8.06}
        \drawPoint{A}{11.06}{-3.16}
        \drawPoint{S}{9.05}{0.53}
        \drawPoint{H}{15.68}{1.18};
    }{}%
}


\answerFill[Réponse][
    \begin{itemize}
        \item On a : $PE + KE + KP = \Lg{2,6} + \Lg{10,1} + \Lg{5} = \Lg{17.7}$.
        \item Alors, le périmètre du triangle $PEK$ est de \Lg{17.7}.
        \item On sait que les points $A$, $S$ et $H$ sont respectivement les images des points $P$, $E$ et $K$ par la symétrie centrale de centre $O$.
        \item Alors, le triangle $ASH$ est l'image du triangle $PEK$ par cette symétrie.
        \item Or, la symétrie centrale conserve les longueurs.
        \item Alors, le périmètre du triangle $ASH$ est égal à celui du triangle $PEK$.
        \item Donc le périmètre du triangle $ASH$ est de \Lg{17.7}.
    \end{itemize}
]

% %%%
\setGrade{5e}
\evaluation{2}
% [corr]
%%%
% \def\imgPath{enseignement/6e/}

\seqEvaluation{3}{Symétrie}{
    Comprendre l'effet des symétries (axiale et centrale) :
    conservation du parallélisme{,} des longueurs et des angles.
    /3,
    Reconnaître deux figures image l'une de l'autre par symetrie.
    /3,
    % Identifier des symétries dans des frises{,} des pavages{,} des rosaces.
    % /2%
}

\seqEvaluation{4}{Nombres Relatifs - Sommes et differences}{
    Résoudre un calcul en ligne.
    /3,
    Additionner et soustraire des nombres décimaux relatifs.
    /3,
    Résoudre des problèmes faisant intervenir des nombres relatifs.
    /2,
    Respecter l'ordre des priorités opératoires.
    /2,
    Raisonner avec des nombres relatifs.
    /0,
}

% \seqEvaluation{}{Compétences générales}{
%     Écrire ses calculs
%     /1,
%     Rédiger des phrases réponses
%     /2
% }

\evalutionEnd[2][2]

\noCalculator

\exo{Reconnaître des symétries}{
    \def\crossWidth{0.3mm}\def\crossSize{0.15}\def\nodeShift{0.25}
    \ctikz[1.1]{
        \draw[gray!40] (0,0) grid (6,6);
        % Nested loops to generate points with names
        \newcounter{i}\setcounter{i}{1}%
        \foreach \x in {1,...,5} {%
            \foreach \y in {1,...,5} {%
                % Generate names programmatically
                \pgfmathtruncatemacro{\charCode}{64+\thei} % Convert x to letter (A=1, B=2, etc.)
                \edef\pointName{\char\charCode} % Get the letter name
                \drawPoint{\pointName}{\x}{\y}%
                \stepcounter{i}%
            }
        }
    }
    \begin{enumerate}
        \item L'image du segment $[HR]$ par la symétrie de centre $N$ est : \nswr[2]{le segment $[JT]$}.
        \item Le triangle $QUV$ est l'image du triangle $SON$ par \nswr[2]{la symetries de centre $R$}.
        \item Le point \nswr{$D$} est l'image de point P par la symetrie d'axe $(AG)$.
        \item L'image du quadrilatère $NXQL$ par la symétrie de centre $M$ est : \nswr[2]{le quadrilatère $LBIN$}.
    \end{enumerate}
}[\sesa{5}{2024}[3][89]]

\exo{Utiliser les propriétés de la symétrie centrale}{
    On considère quatre points $P$, $O$, $K$ et $E$ tels que :  
    \begin{align*}
        KE = \Lg{10.1}, \quad PO = \Lg{3.6}, \quad EP = \Lg{2.6}, \quad PK = \Lg{5}.
    \end{align*}
    Les points $A$,$S$ et $H$ sont respectivement les images des points $P$,$E$ et $K$ par la symétrie centrale de centre $O$.
    Calculez le périmètre du triangle $ASH$.
}

% \begin{Geometrie}[TypeTrace="Schema"]
%     pair A,B,C;
%     A=u*(1,1);
%     B-A=u*(4,1);
%     C=rotation(A,B,-70);
%     trace polygone(A,B,C);
% \end{Geometrie}


\ctikz[0.75]{
    \draw[gray!40] (-1,-10) rectangle (18,2);
    \node at (3,1) {Schéma à main levée :};
    \ifthenelse{\boolean{answer}}{%
        \draw [penciline,thick] (0.48,-7.04)-- (7.4,-8.06);
        \draw [penciline,thick] (7.4,-8.06)-- (5.18,-4.68);
        \draw [penciline,thick] (5.18,-4.68)-- (0.48,-7.04);
        \draw [penciline,thick] (11.06,-3.16)-- (15.683121999999992,1.1780524000000023);
        \draw [penciline,thick] (15.683121999999992,1.1780524000000023)-- (9.047820799999997,0.5338484000000019);
        \draw [penciline,thick] (9.047820799999997,0.5338484000000019)-- (11.06,-3.16);
        \draw (6.825316999999991,-6.262503800000003) node[anchor=north west] {\Lg{2.6}};
        \draw (3.2821949999999944,-8.098485200000004) node[anchor=north west] {\Lg{10.1}};
        \draw (2.090417599999996,-5.038516200000002) node[anchor=north west] {\Lg{5}};
        \drawPoint{P}{5.18}{-4.68}
        \drawPoint{O}{8.16}{-4.22}
        \drawPoint{K}{0.48}{-7.04}
        \drawPoint{E}{7.40}{-8.06}
        \drawPoint{A}{11.06}{-3.16}
        \drawPoint{S}{9.05}{0.53}
        \drawPoint{H}{15.68}{1.18};
    }{}%
}


\answerFill[Réponse][
    \begin{itemize}
        \item On a : $PE + KE + KP = \Lg{2,6} + \Lg{10,1} + \Lg{5} = \Lg{17.7}$.
        \item Alors, le périmètre du triangle $PEK$ est de \Lg{17.7}.
        \item On sait que les points $A$, $S$ et $H$ sont respectivement les images des points $P$, $E$ et $K$ par la symétrie centrale de centre $O$.
        \item Alors, le triangle $ASH$ est l'image du triangle $PEK$ par cette symétrie.
        \item Or, la symétrie centrale conserve les longueurs.
        \item Alors, le périmètre du triangle $ASH$ est égal à celui du triangle $PEK$.
        \item Donc le périmètre du triangle $ASH$ est de \Lg{17.7}.
    \end{itemize}
]

\exo{Calculs}{
    \multiColEnumerate{1}{
        \item $\dfrac{81}{9} \times 5 -1
        = \nswr[3]{9\times 5 - 1 = 45 -1 = 44}$
        % \item $\dfrac{45,5}{2\times3-1}
        % = \nswr[2]{\dfrac{45,5}{6-1} = \dfrac{45,5}{5} = 9,1}$
        % \item $\dfrac{27}{2\times3}-1
        % = \nswr[3]{\dfrac{27}{6} -1 = 4,5 - 1}$
        % \item $\dfrac{17-5}{3}+2
        % = \nswr[3]{\dfrac{12}{3}+2 = 4 + 2 = 6}$
        \item $7\times\dfrac{15\times4}{8-2}+2\times8
        = \nswr[2]{7 \times \dfrac{60}{6}+ 16 = 7 \times 10 + 16 = 70 + 16 = 86}$
        \item $7 - (-\np{12}) = \nswr[2]{19}$
        \item $-9 + \np{6.3} - 78 = \nswr[2]{- \np{3.7} - 78 = - \np{81.7}}$
        % \item $12 + (-\np{22,6}) = \nswr[2]{-10,6}$
        \item $39 + (7 - 18 + (-1)) = \nswr[3]{39 + (-11 + (-1)) = 39 + (-12) = 27}$
    }
}[\sesa{5}{2024}[10][41]]

% \exo{}{
%     \begin{enumerate}

%     \end{enumerate}
% }

\exo{\ttps La randonnée d'Élodie}{
    Élodie effectue une randonnée en montagne qui se déroule en plusieurs étapes :  
    \begin{itemize}
        \item Depuis son point de départ situé à \Lg[m]{1800} d'altitude (hauteur par rapport au niveau de la mer),
        elle grimpe de \Lg[m]{340} pour atteindre un premier refuge.  
        \item Après une pause, elle descend de \Lg[km]{1,3} pour visiter un lac de montagne.  
        \item Ensuite, elle remonte de \Lg[m]{230} pour reprendre le chemin principal.  
        \item Elle rejoint un second refuge situé à \Lg[km]{0,9} plus haut.  
        \item Enfin, le dernier jour, elle redescend de \Lg[m]{860} jusqu'à un arrêt de bus.  
    \end{itemize}
    Quel est le dénivelé total de sa randonnée, c'est-à-dire la différence d'altitude entre son point de point d'arrivée et son point de départ ?
    \\\hint{Un dénivelé peut être positif ou négatif.}
    % \begin{enumerate}
    %     % \item Quelle est l'altitude maximale atteinte par Élodie pendant sa randonnée ? Et l'altitude minimale ?  
    %     \item 
    % \end{enumerate}
}

\answerFill[Réponse][
    \multiColItemize{1}{
        \item $1800 + 340 - 1300 + 230 + 900 - 860 = 1110$
        \item $1110 - 1800 = -690$
        Le dénivelé total de sa randonnée est de \Lg[m]{-690}
    }
]

\def\a{\cir[Red]{a}} \def\b{\cir[Green]{b}} \def\c{\cir[Blue]{c}} \def\d{\cir[violet]{d}} \def\e{\cir[Orange]{e}}
\def\f{\cir[violet]{5}} \def\t{\cir[Orange]{-3}}
\exo{\bonus Jeu de jetons}{
    Un sac contient des jetons,
    chacun portant une lettre sur une face
    (\a, \b, \c, \d ou \e)
    et un nombre relatif sur l'autre face.
    Chaque lettre correspond à un unique nombre relatif.
    Le jeu consiste à tirer plusieurs jetons,
    et ajouter les nombres associés,
    Le joueur ayant le score le plus élevé remporte la partie.
    Les tirages des joueurs sont donnés ci-dessous avec les valeurs des jetons \d et \e déjà visibles :
    \multiColItemize{2}{
        \item Lili : \a \b \c
        \item Mattéo : \a \f \b \f \c
        \item Loane : \b \t \c \a \b
        \item Antoine :\a \c \a \b \c \b
        \item Nina : \a \f \c \t \b \t \a \c
    }
    On sait que le score de Lili est -6.

    \begin{enumerate}
        \item Pour chaque joueur, déterminez si leur score est calculable.
        Si oui, donnez leur score.
        Sinon, justifiez pourquoi leur score reste indéterminable.
        \item Sachant que les valeurs des lettres sont comprises entre -9 et 9,
        est-il possible d'identifier le vainqueur avec certitude ?
        \item Quelle est la place la plus haute qu'Antoine pourrait atteindre ?
    \end{enumerate}
}[Inspiré de \mi]

\nswr[0]{\vspace{-0.75cm}}

\answerFill[Réponse][\small
    \begin{enumerate}
        \item L'adition est commutative (c'est à dire $\a+\b=\b+\a$),
        on peut alors ajouter les points des jetons dans n'importe quel ordre et on obtiendra toujours la meme chose.
        On va s'en servire pour essayer de determiner les scores de chacuns des joueurs.
        \multiColItemize{1}{
            \item Lili : $\a + \b + \c
            = -6$
            \item Mattéo : $\a + \f + \b + \f + \c
            = (\a  + \b + \c) + \f + \f
            = -6 + 10
            = 4$
            \item Loane : $\b + \t + \c + \a + \b
            = (\a  + \b + \c) + \t + \b
            = -6 + \t + \b
            = -9 + \b$
            \item Antoine : $\a + \c + \a + \b + \c + \b
            = (\a  + \b + \c) + (\a  + \b + \c)
            = -6 + -6 = -12$
            \item Nina : $\a + \f + \c + \t + \b + \t + \a + \c
            = (\a  + \b + \c) + \f + \t + \t + \a + \c
            = -6 + \f + \t + \t + \a + \c
            = -7 + \a + \c$
        }
        On ne connait pas les valeurs de \a et \c et on ne peut donc pas connaitre le score de Loane et Nina.
        \item \multiColItemize{1}{
            \item Max(Loane) $= -9 + 9 = 0$ avec $\b = 9$
            \item Max(Nina) $= -7 + -6 + 9 = -4$ avec $\b = -9$, $\a = 9$ et $\c = -6$ (on a bien $9 + (-9) + (-6) = -6$)
        }
        Mattéo est forcément le vainceur car aucun autre joueur ne peut atteindre un score de $4$.
        \item Avec $\a =-6, \b = -4 \et \c = 4$ (on a bien $-6 + (-4) + 4 = -6$) \vspace{-0.25cm}\multiColItemize{2}{
            \item Score(Loane) $= -9 + (-4) = -13$
            \item Score(Nina) $= -7 + (-6) + 4 = -9$
        }\vspace{-0.25cm}
        Antoine pourrait être 4e au maximum,
        car Loane et Nina ne peuvent pas toutes deux avoir un score inférieur à $-12$.
        En effet, dans le cas limite présenté précédemment,
        si on diminue le score de Nina, celui de Loane augmente nécessairement,
        ce qui la ramène d'abord à égalité avec Antoine, puis au-dessus de lui.
    \end{enumerate}
]



% % VARIABLES %%%
\setTitle{test}
%%%%%%%%%%%%%%%

\setSeq{2}{TEST SEQ}

\setGrade{6e}

\bsec{test}

\def\aspc{\ifbool{answer}{}{\vspace{1cm}}}

\begin{frame}
    \begin{enumerate}
        \item \multiColEnumerate{2}{
            \item zvgew 
            \item zvgew 
            \item zvgew 
            \item zvgew
            \item zvgew 
            \item zvgew
        }
    \end{enumerate}
    \framebreak
    eqgrpoiesqjgr
\end{frame}

\slide{exo}{
    \begin{enumerate} \loadenumi \setItemColor{RoyalBlue}
        \item \multiColEnumerate{2}{
            \item zvgew 
            \item zvgew 
            \item zvgew 
            \item zvgew
            \item zvgew 
            \item zvgew
        }
        \item zvgew 
        \item zvgew \framebreak
        \item zvgew 
        \item zvgew
        \item zvgew 
        \item zvgew 
    \end{enumerate}
}

\slide{exo}{
    efhuzeiugfh
}

\slide{cr}{
    z"gqrzg
}
\end{document}