%% BEAMER %%
% \documentclass[aspectratio=169, usenames,dvipsnames,xcolor=table]{beamer} \usepackage[fontsize=14pt]{fontsize}

\usepackage[T1]{fontenc}
\usepackage[french]{babel}

\usepackage[utf8]{inputenc}
\usepackage{amsmath}
\usepackage{amsthm}
\usepackage{amssymb}
\usepackage{graphicx}
\usepackage{dashundergaps}
\usepackage{array}
\usepackage{multicol}
\usepackage{wrapfig}
\usepackage{numprint}
\usepackage{ulem}
\usepackage{hyperref}
\usepackage{mathrsfs}
\usepackage{mathtools}
\usepackage[many]{tcolorbox}
\usepackage{xparse}
\usepackage{float}
\usepackage{lipsum}
\usepackage{pgf}
\usepackage{ifthen}
\usepackage{caption}
\usepackage{tikz}
\usepackage{xifthen}

% \usepackage[squaren,Gray]{SIunits}

% BREVET

% \usepackage{makeidx}
% \usepackage{fancybox}
% \usepackage{tabularx}
% \usepackage[normalem]{ulem}
% \usepackage{pifont}
% \usepackage{lscape}
% \usepackage{diagbox}
% \usepackage{multirow} 
% \usepackage{textcomp}
% \usepackage{scratch3}
% \usepackage[T1]{fontenc}
% \usepackage{fourier}
% \usepackage[french]{babel}
% \usepackage{pstricks}

% \usepackage[scaled=0.875]{helvet}
% \usepackage{pst-plot,pst-text,pst-tree,pstricks-add}

% fancyhdr

\setlength{\headheight}{18pt}
\fancyhead[C]{\normalsize \title}
% \renewcommand{\headrulewidth}{0pt} % Remove header line
\fancyhead[R]{}
\fancyfoot[L]{\author}
\fancyfoot[C]{\textbf{Page \thepage/\pageref{LastPage}}}
\fancyfoot[R]{\date}

\fancypagestyle{firstpage}{
    \setlength{\headheight}{29pt}
    \fancyhead[C]{\LARGE \title}
    \fancyhead[R]{}
    \fancyfoot[L]{\author}
    \fancyfoot[C]{\textbf{Page \thepage/\pageref{LastPage}}}
    \fancyfoot[R]{\date}
}

\thispagestyle{firstpage}

% \fancyfoot[C]{\textbf{Page 1/1}}

% HYPERREF

\hypersetup{
    colorlinks=true,       % false: boxed links; true: colored links
    linkcolor=red,          % color of internal links (change box color with linkbordercolor)
    citecolor=green,        % color of links to bibliography
    filecolor=magenta,      % color of file links
    urlcolor=blue,          % color of external links
    urlbordercolor=blue,    % borders of external links
    linkbordercolor=red,    % borders of internal links
    pdfborderstyle={/S/U/W 1}% border style will be underline of width 1pt
}

\usepackage[fontsize=14pt]{fontsize}

\usepackage[T1]{fontenc}
\usepackage[french]{babel}
\usepackage[utf8]{inputenc}

\frenchbsetup{StandardItemLabels=true}

% GLOBAL VARIABLES %%%
\graphicspath{{images}}
\def\cwidth{4cm}
\def\tspace{0.5cm}

% BOOLEAN %%%
\newboolean{anwser}
\newboolean{demonstration}
\newboolean{boxedProperties}
\newboolean{showID}
\newboolean{parenthisedID}
\newboolean{animated}
\newboolean{outline}

\setboolean{anwser}{false}
\setboolean{demonstration}{true}
\setboolean{parenthisedID}{true}
\setboolean{showID}{true}
\setboolean{boxedProperties}{false} % false = edge
\setboolean{outline}{false}

\def\DefinitionColor{PineGreen}
\def\PropertyColor{Blue}
\def\TheoremColor{Plum}

\def\SectionColor{Red}
\def\SubSectionColor{Green}

\setboolean{animated}{true}

% \DeclareMathOperator{\PGCD}{PGCD}
% \DeclareMathOperator{\PPCM}{PPCM}

\DeclareMathOperator{\sh}{sh}
\DeclareMathOperator{\ch}{ch}
% \DeclareMathOperator{\th}{th}

\DeclareMathOperator{\argsh}{argsh}
\DeclareMathOperator{\argch}{argch}
\DeclareMathOperator{\argth}{argth}
\DeclareMathOperator{\I}{I}
\DeclareMathOperator{\Id}{Id}
\DeclareMathOperator{\Ker}{Ker}
% \DeclareMathOperator{\dl}{o}
\newcommand{\dl}[1]{
    \operatorname*{o}_{#1}
}

\def\deg{\ensuremath{^\circ}}
\def\prll{\mathbin{\!/\mkern-5mu/\!}}
\renewcommand{\parallel}{\mathbin{\!/\mkern-5mu/\!}}

\def\octet{\textrm{o}}
\def\byte{\textrm{B}}

\def\hour{\textrm{h}}
\def\minute{\textrm{min}}
\def\second{\textrm{s}}
% ENVIRONMENT
\newenvironment{mysection}[1][gray!20]{%
    \begin{sectionBox}[#1]
}{%
    \end{sectionBox}
}

\newenvironment{mysubsection}[1][gray!20]{%
    \begin{subsectionBox}[#1]
}{%
    \end{subsectionBox}
}

% Switch implementation
\newboolean{default}
\newcommand{\case}{}
\newcommand{\default}{}

\newenvironment{switch}[1]{%
    \setboolean{default}{true}
    \renewcommand{\case}[2]{\ifthenelse{\equal{#1}{##1}}{%
        \setboolean{default}{false}##2}{}}%
    \renewcommand{\default}[1]{\ifthenelse{\boolean{default}}{##1}{}}
}{}

% SECTIONS
\input{header/command/sections.tex}

% ANSWERS
\newlength{\parline}
\newlength{\paroutindent}
\newlength{\lineheight}
\setlength{\lineheight}{\heightof{abcdefghijklmnoprstuvwxyz}}

\newcommand{\countlines}[1]{%
    \setlength{\paroutindent}{\expandafter\parindent}
    \setlength{\parline}{\heightof{\noindent\begin{minipage}{\linewidth}%
                \setlength{\parindent}{\paroutindent}#1\end{minipage}}}%
    \pgfmathparse{round(\parline / (0.9*\lineheight))}
    \newcount\linecount
    \pgfmathsetcount{\linecount}{\pgfmathresult}
}

\newcommand{\looptext}[2]{%
    \noindent
    \newcount\printcount
    \printcount=#2
    \loop
        #1
        \advance\printcount by -1
        \ifnum\printcount>0
    \repeat
}

\newcommand{\awsr}[1]{%
    \ifthenelse{\boolean{answer}}{
        \result{#1}
    }{
        \countlines{#1}
        \pgfmathsetcount{\linecount}{\linecount + 1}
        \noindent\hspace{-9pt}
        \looptext{
            \noindent\dotfill
    
        }{\the\linecount}
    }
}

\newcommand{\dottedLines}[1]{%
    \noindent\hspace{-9pt}%

    \looptext{%
        \noindent\dotfill%

    }{#1}
}

\newcommand{\result}[1]{\color{OrangeRed}#1\color{black}}

% MATH
\input{header/command/math.tex}

% IMAGES
\input{header/command/image.tex}

% COMMANDS

\newcommand{\fsize}[1]{\fontsize{#1}{#1}\selectfont}

\NewDocumentCommand{\ifNotNull}{mmo}{
    \IfValueT{#1}{
        \ifx\relax#1\relax
            \IfValueT{#3}{
                #3
            }
        \else
            #2
        \fi
    }
}

\NewDocumentCommand{\ilink}{m g}{%
    \item
    \IfValueTF{#2}{\link{#1}{#2}}{\link{#1}}
}

\NewDocumentCommand{\link}{m g}{%
    \csn{#1}%
    \IfValueT{#2}{(#2)}%
}

\NewDocumentCommand{\TODO}{g}{%
    {\color{Red} $\rightarrow$ \textbf{TODO}
    \IfValueT{#1}{(#1)}}
    % \color{black}
}

\newcommand{\leconInfoBox}[2]{
    \textbf{#1 :}\vspace{-0.25cm}
        \begin{multicols}{2}
            \begin{itemize}[label=$\blacktriangleright$, font = \small \color{Red}]
                #2
            \end{itemize}
        \end{multicols}
        \vspace{-0.4cm}
}

% TCOLORBOX

\input{header/command/tcolorbox.tex}

\NewDocumentCommand{\leconInfo}{mooo}{
    \begin{infoBox}
        \leconInfoBox{Niveaux}{#1}
        \ifNotNull{#2}{
            \tcbline
            \leconInfoBox{Prérequis}{#2}
        }
        \ifNotNull{#3}{
            \tcbline
            \leconInfoBox{Thèmes}{#3}
        }
        \ifNotNull{#4}{
            \tcbline
            \textbf{Motivation :} 
            #4
        }
    \end{infoBox}
}

\NewDocumentCommand{\seanceInfo}{oooooooo}{
    \begin{infoBox}
        \vspace{-0.05cm}
        \begin{tcbitemize}[raster rows=1,raster columns=20,raster height=1.65cm,
            raster every box/.style={colframe=red!50!black,colback=red!10!white}]
            \tcbitem[raster multicolumn=6] \textbf{Date :} #1
            \tcbitem[raster multicolumn=10] \textbf{Séquence :} #2
            \tcbitem[raster multicolumn=4] \textbf{Séance :} #3
        \end{tcbitemize}
        \vspace{-0.25cm}
        \ifNotNull{#4}{\tcbline \textbf{Objectif :} #4}
        \ifNotNull{#5}{\tcbline \leconInfoBox{Classe(s)}{#5}}
        \ifNotNull{#6}{\tcbline \leconInfoBox{Prérequi(s)}{#6}}
        \ifNotNull{#7}{\tcbline \textbf{Séance précédente :} #7}
        \ifNotNull{#7}{\tcbline \leconInfoBox{Matériel(s)}{#8}}
    \end{infoBox}
}

\def\pDscr{\tcbitem[enhanced jigsaw, breakable,
    raster multicolumn=6]
}
\def\pMdlt{\tcbitem[enhanced jigsaw, breakable,
    raster multicolumn=11]
}
\def\pTime{\tcbitem[enhanced jigsaw, breakable,
    raster multicolumn=3, halign=center]
}

\newcommand{\prepRow}[3]{
    \tcbitem[raster multicolumn=20]
    \tcblower

    \pDscr #1
    \pMdlt #2
    \pTime #3
}

\newcommand{\prepTable}[1]{
    \begin{prepBox}
        \begin{tcbitemize}[enhanced jigsaw, breakable, raster rows=1,raster columns=20,raster height=1.1cm, halign=center,
            raster every box/.style={enhanced jigsaw, breakable, colframe=Blue!50!black,colback=Blue!10!white}]
            \pDscr \textbf{Descriptif}
            \pMdlt \textbf{Modalité}
            \pTime \textbf{Durée}
        \end{tcbitemize}
        \begin{tcbitemize}[enhanced jigsaw, breakable,
            raster equal height = rows, 
            raster columns=20, frame hidden,
            raster every box/.style={
                enhanced jigsaw, breakable,
                opacityback=0, valign=top, 
                size = tight
            }]
            #1
        \end{tcbitemize}
    \end{prepBox}
}

% TIKZ

\newcommand{\ctikz}[1]{
    \begin{center}
        \begin{tikzpicture}
            #1
        \end{tikzpicture}
    \end{center}
}

\newcommand{\axis}[1]{%Draw coordinate axes
    \draw[thin, -Stealth] (-0.5,0) -- (#1,0);% node[right] {$x$}; % x-axis
    \draw[thin, -Stealth] (0,-0.5) -- (0,#1);% node[above] {$y$}; % y-axis
}

\newcommand{\drawGrid}[3]{
    \foreach \n in {0,...,#1}
        \draw[line width = #3] (\n,0) -- (\n,#2);
    \foreach \n in {0,...,#2}
        \draw[line width = #3] (0,\n) -- (#1,\n);
}

\newcommand{\drawPoint}[4]{
    \node[shift={#4}, color = \pointColor] at (#2 - 0.5,#3 - 0.5) {#1};
    \draw[line width = \crossWidth, shift={#4}, color = \pointColor] (#2 - 0.25,#3) -- (#2 + 0.25,#3);
    \draw[line width = \crossWidth, shift={#4}, color = \pointColor] (#2,#3 - 0.25) -- (#2,#3 + 0.25);
}

% Tabular
\newcolumntype{C}[1]{>{\centering\arraybackslash}p{#1}}
\newcolumntype{M}[1]{>{\centering\arraybackslash}m{#1}}
\newcolumntype{K}{@{}m{0pt}@{}}

% GEOMETRY

% \newcommand{\restoregeometry}{def}

\newcommand{\multiColItemize}[2]{
    \begin{multicols}{#1}
        \begin{itemize}
            #2
        \end{itemize}
    \end{multicols}
}

\newcommand{\multiColEnumerate}[2]{
    \ifthenelse{\isequivalentto{#1}{1}}{
        \begin{enumerate}
            #2
        \end{enumerate}
    }{
        \begin{multicols}{#1}
            \begin{enumerate}
                #2
            \end{enumerate}
        \end{multicols}
    }
}

\makeatletter
\newcommand\pgfinvisible{\pgfsys@begininvisible}
\newcommand\pgfshown{\pgfsys@endinvisible}
\makeatother

\renewcommand*{\phantom}[1]{
    \pgfinvisible #1 \pgfshown
}

\newcounter{size}
\newcommand{\listSize}[1]{%
    \setcounter{size}{0}%
    \foreach \n in {#1}{\stepcounter{size}}%
    % \thesize
}

\newcounter{elemPos}
\newcommand{\listElement}[2]{
    \setcounter{elemPos}{0} % Start counting from 1
    \def\resultVal{0} % Default value
    \renewcommand*{\do}[1]{%
        \ifnumequal{\value{elemPos}}{#2}{%
            \def\resultVal{##1}%
            \listbreak% Break out of the loop
        }{}%
        \stepcounter{elemPos}%
    }
    % \docsvlist{#1}
    \expandafter\docsvlist\expandafter{#1} % Expand the list before passing it to \docsvlist
    \resultVal
}

% \NewDocumentCommand{\exoslide}{m O{10cm}}{
%     \slide{}{
%         \img{\imgf{#1}}[#2]
%     }
% }

\NewDocumentCommand{\exoSlide}{m O{10cm} O{1} O{} O{exo}}{%
    \slide{#5}{%
        \ifthenelse{\equal{#3}{1}}{\vspace{-0.5cm}}{\vspace{-1cm}}
        \def\exercices{\foreach \q in {#1}{\imgp{\q}[#2]\vspace{-0.5cm}}}
        \exo{#1}{\wideFrame[7em]{\bvspace{0.25cm}\avspace{-0.25cm}
            \ifthenelse{\equal{#3}{1}}{\exercices}
            {\begin{multicols}{#3}\exercices\end{multicols}}}
            \avspace{0.75cm}
        }[#4]
    }
}

\NewDocumentCommand{\exoList}{m O{} O{3}}{%
    \section*{Exercices}%
    \slide{EXERCICES}{
        \exo{#2}{
            \vspace{-0.25cm}
            \multiColEnumerate{#3}{
                \foreach \q in {#1}{
                    \item \q
                }
            }
        }
    }
}

\newcommand{\questions}[1]{
    \begin{enumerate}
        \foreach \q in {#1}{
            \item \q\\
            \vspace*{-0.45cm}
            \dottedLines{3}
        }
    \end{enumerate}
}

% Define a new boolean for checking if the section is starred
\newboolean{section@star}

\makeatletter
% Redefine \section and \section* to set the boolean
\let\old@section\section
\renewcommand{\section}{%
    \@ifstar
        {\setboolean{section@star}{true}\old@section*}
        {\setboolean{section@star}{false}\old@section}%
}
\makeatother

\newcommand{\qt}[1]{«\textit{#1}»}

\newcommand{\calc}[1]{\numexpr#1\relax}
\newcommand{\ncalc}[1]{\number\calc{#1}}
\newcommand{\pcalc}[1]{\numprint{\ncalc{#1}}}

\newcommand{\setgrade}[1]{
    \def\grade{#1}
    % \begin{switch}{#1}
    %     \case{6e}{\global\definecolor{gradeColor}{hex}{FA8072}}
    %     \default{
    %         Default
    %         \global\definecolor{gradeColor}{RGB}{200, 50, 50}
    %     }
    % \end{switch}
    \ifthenelse{\equal{#1}{6e}}{
        \definecolor{gradeColor}{HTML}{C6233D} % FA8072 in hex
    }{
    \ifthenelse{\equal{#1}{5e}}{
        \definecolor{gradeColor}{HTML}{088255}
    }{
    \ifthenelse{\equal{#1}{4e}}{
        \definecolor{gradeColor}{HTML}{1466A8}
    }{
    \ifthenelse{\equal{#1}{3e}}{
        \definecolor{gradeColor}{HTML}{844499}
    }{
        \definecolor{gradeColor}{RGB}{0, 0, 0}
    }}}}
}

\gdef\phase{}
\newcommand{\setPhase}[1]{%
    \begin{switch}{#1}
        \case{exo}{\gdef\phase{EXERCICES}}
        \case{cr}{\gdef\phase{COURS}}
        \case{qf}{\gdef\phase{QUESTIONS FLASH}}
        \case{dm}{\gdef\phase{DEVOIR MAISON}}
        \default{\gdef\phase{#1}}
    \end{switch}
}

\newcounter{savedenumi}
\setcounter{savedenumi}{0}
\xdef\savedenumi{0}
% \newcommand{\saveenumi}{
%     % \xdef\savedenumi{\calc{\theenumi-1}}
%     \setcounter{savedenumi}{0}
% }

\newcommand{\saveenumi}[1]{
    \setcounter{savedenumi}{#1}
}

\newcommand{\loadenumi}{
    \setItemColor{\currentColor}
    \setcounter{enumi}{\thesavedenumi}
}

\newcommand\csn[1]{\csname #1\endcsname}

\newcommand{\vect}[1]{\ensuremath{\overrightarrow{#1}}}
% \newcommand{\vect}[1]{\overrightarrow{\,\mathstrut#1\,}}
\newcommand{\m}[1]{\ensuremath{\mathbf{#1}}}
\newcommand\lm[2]{\lim_{#1\to#2}}

\def\eqv{\Leftrightarrow}
\def\ssi{si et seulement si }
\def\pt{pour tout }
\def\poly2{fonction polynôme du second degré }
\def\eq2{équation second degré }
\def\discr{b^2-4ac}

% MATH TEXT
\def\et{\textrm{ et }}
\def\si{\textrm{ si }}
\def\avec{\textrm{ avec }}
\def\car{\textrm{ car }}
\def\alors{\textrm{ alors }}
\def\ou{\textrm{ ou }}
\def\ona{\textrm{ on a }}

\def\iet{\shortintertext{et}}
\def\ialors{\shortintertext{alors}}
\def\idou{\shortintertext{d'où}}
\def\ior{\shortintertext{or}}
\def\iona{\shortintertext{on a}}

\def\studentinfo{
    \vspace*{-1cm}
    \begin{minipage}{0.35\linewidth}
        nom: \dotfill
    \end{minipage}
    \begin{minipage}{0.35\linewidth}
        prénom: \dotfill
    \end{minipage}
    \begin{minipage}{0.15\linewidth}
        classes: \dotfill
    \end{minipage}
    
    \noindent\hrulefill
}

% UNITS
\def\cm{\,\centi\meter}
\def\km{\,\kilo\meter}
\newcommand{\defl}[2]{%
    \expandafter\def\csname #1\endcsname{\href{#2}{#1}\space}%
}

% Page Eduscol
\defl{Eduscol Cycle 3}{https://eduscol.education.fr/251/mathematiques-cycle-3}
\defl{Eduscol Cycle 4}{https://eduscol.education.fr/280/mathematiques-cycle-4}
\defl{Eduscol Lycée Général et technologique}{https://eduscol.education.fr/1723/programmes-et-ressources-en-mathematiques-voie-gt}
\defl{Eduscol Lycée Professionnel}{https://eduscol.education.fr/1793/programmes-et-ressources-en-mathematiques-voie-professionnelle}

% Repères annuels
\defl{Cycle 3}{https://eduscol.education.fr/document/14026/download}
\defl{Cycle 4}{https://eduscol.education.fr/document/14080/download}

% Attendus de fin d'année
\defl{5e}{https://eduscol.education.fr/document/14044/download}
\defl{4e}{https://eduscol.education.fr/document/14056/download}
\defl{3e}{https://eduscol.education.fr/document/14068/download}

% Programme de mathématiques
\defl{cycle 3}{https://eduscol.education.fr/document/50990/download}
\defl{cycle 4}{https://cache.media.education.gouv.fr/file/31/89/1/ensel714_annexe3_1312891.pdf}
\defl{2nd}{https://eduscol.education.fr/document/24553/download}
\defl{1re}{https://eduscol.education.fr/document/24565/download}
\defl{1re STL}{https://eduscol.education.fr/document/23098/download}
\defl{1re STI2D}{https://eduscol.education.fr/document/24919/download}
\defl{Terminale Option Spécialité}{https://eduscol.education.fr/document/24568/download}
\defl{Terminale Option Complémentaire}{https://eduscol.education.fr/document/24571/download}
\defl{Terminale Option Expertes}{https://eduscol.education.fr/document/24574/download}
\defl{Terminale STL}{https://eduscol.education.fr/document/23107/download}
\defl{Terminale STI2D}{https://eduscol.education.fr/document/24922/download}
% Ressources thématiques
\defl{Proportionnalité}{https://eduscol.education.fr/document/17281/download}
\defl{Probabilités}{https://eduscol.education.fr/document/17275/download}
\defl{Fonctions}{https://eduscol.education.fr/document/17287/download}
\defl{Traitement des données}{https://eduscol.education.fr/document/17269/download}

\defl{Fonctions}{https://eduscol.education.fr/document/17287/download}
\defl{Fractions}{https://eduscol.education.fr/document/17239/download}
\defl{Nombres relatifs}{https://eduscol.education.fr/document/17245/download}
\defl{Puissances}{https://eduscol.education.fr/document/17251/download}
\defl{Divisibilité et nombres premiers}{https://eduscol.education.fr/document/17257/download}
\defl{Calcul littéral}{https://eduscol.education.fr/document/17263/download}

\defl{Grandeurs et mesures}{https://eduscol.education.fr/document/17293/download}
\defl{Algorithmique et programmation}{https://eduscol.education.fr/document/17311/download}

\defl{Suites}{https://eduscol.education.fr/document/24586/download}
\defl{Produit Scalaire}{https://eduscol.education.fr/document/24589/download}
\defl{Raisonnement et démonstration (seconde)}{https://eduscol.education.fr/document/24580/download}
\defl{Raisonnement et démonstrations (première)}{https://eduscol.education.fr/document/24583/download}

\captionsetup{labelformat=empty,labelsep=none}

% \setboolean{boxedProperties}{true} % false = edge
% \setboolean{parenthisedID}{false}
% \setboolean{showID}{false}

% \def\DefinitionColor{Red}
\def\PropertyColor{Red}
\def\TheoremColor{Red}

% TIKZ
\def\crossWidth{0.25mm}
\def\pointColor{blue}

\usepackage{bookmark}
% THEMES
% http://mcclinews.free.fr/latex/beamergalerie/completsgalerie.html
% default
% \usetheme{Madrid}
% \usetheme{CambridgeUS}

% tree
% \usetheme{Montpellier}
% \usetheme{Juanlespins}

\newcommand{\headerBox}[2]{
    \begin{beamercolorbox}[wd=\paperwidth,ht=2.125ex,dp=1.125ex,leftskip=.3cm,rightskip=.3cm plus1fil]{#1}%
        \usebeamerfont{#1}#2%
    \end{beamercolorbox}
}

\usetheme{Antibes}
\setbeamertemplate{headline}
{%  
    \headerBox{title in head/foot}{
        \insertshorttitle
        \hfill
        \color{links}\insertauthor
    }
    \ifx\insertsectionhead\empty\else
    \headerBox{section in head/foot}{
        \hskip6pt \ifthenelse{\boolean{section@star}}{$\rightarrow$}{\Roman{sec}.} \insertsectionhead
        \hfill
        \insertframenumber{} / \inserttotalframenumber
    }
    \fi
    \ifx\insertsubsectionhead\empty\else
    \headerBox{subsection in head/foot}{
        \hskip12pt \thesubsection{} \insertsubsectionhead
    }
    \fi
    \ifx\insertsubsubsectionhead\empty\else
    \headerBox{subsubsection in head/foot}{
        \hskip18pt \thesubsubsection{} \insertsubsubsectionhead
    }
    \fi
}

\setbeamerfont{frametitle}{size=\small,series=\bfseries}

\setbeamertemplate{frametitle}
{
    \vspace{-1.5pt} % Adjust the vertical space before the title
    \begin{beamercolorbox}[ht=2.5ex,dp=1.0ex,wd=\paperwidth,leftskip=.3cm,rightskip=.3cm]{frametitle}
        \usebeamerfont{frametitle}\insertframetitle
    \end{beamercolorbox}
}


% \setbeamertemplate{footline}
% {%
%     \leavevmode%
%     \hbox{%
%     \begin{beamercolorbox}[wd=.5\paperwidth,ht=2.25ex,dp=1ex,left]{author in head/foot}%
%         \usebeamerfont{author in head/foot}\hspace*{2ex}\insertauthor
%     \end{beamercolorbox}%
%     \begin{beamercolorbox}[wd=.5\paperwidth,ht=2.25ex,dp=1ex,right]{title in head/foot}%
%         \usebeamerfont{title in head/foot}\insertframenumber{} / \inserttotalframenumber\hspace*{2ex}
%     \end{beamercolorbox}}%
%     \vskip0pt%
% }

% lateral
% \usetheme{Hannover}

% navigation
% \usetheme{Frankfurt}

% sections and sub
% \usetheme{Warsaw}

% \usetheme{shadow}
% \usetheme{AnnArbor}

% color
% \usecolortheme{beaver}
\usecolortheme{spruce}
% \definecolor{UBCblue}{rgb}{0.04706, 0.13725, 0.26667} % UBC Blue (primary)
% \definecolor{UBCgrey}{rgb}{0.3686, 0.5255, 0.6235} % UBC Grey (secondary)
\setbeamercolor{palette primary}{bg=gradeColor!20,fg=gradeColor}
\setbeamercolor{palette secondary}{bg=gradeColor!95,fg=white}
\setbeamercolor{palette tertiary}{bg=gradeColor!70,fg=white}
\setbeamercolor{palette quaternary}{bg=gradeColor!60,fg=white}

% \setbeamercolor{structure}{fg=gradeColor} % itemize, enumerate, etc
% \setbeamercolor{section in toc}{fg=gradeColor} % TOC sections


% \usecolortheme[named=gradeColor]{structure}

\definecolor{links}{HTML}{e6ffe6}
\definecolor{hyperlinks}{HTML}{e6ffe6}
\hypersetup{
    colorlinks = true,
    linkcolor = links, % Apply the color to internal links
    urlcolor = hyperlinks   % Apply the color to URLs
}

% \setbeamercolor{item}{fg=ForestGreen}
\setbeamercolor{item}{fg=MidnightBlue}

\setbeamersize{
    text margin left=1.5cm,
    text margin right=1.5cm
}

\setbeamercovered{transparent = 25}
\setbeamertemplate{navigation symbols}{}

% \setbeamertemplate{enumerate subitem}{(\alph{enumii})}

% \setbeamertemplate{enumerate subitem}[square]
% \setbeamertemplate{enumerate items}[default]

\newcommand*{\setItemColor}[1]{
    \setbeamercolor{item}{fg=#1}
}
\def\authors{Jules PESIN}
\def\longTitle{long Title}
\def\shortTitle{short Title}
% \def\day{XX/XX/XX}

\title[\shortTitle]{\longTitle}
% \date{\day}

\newcommand{\slide}[2]{
    \begin{frame}
    \frametitle[#1]{#1}
        #2
    \end{frame}
}

\newcounter{sec}
% \stepcounter{sec}
\newcounter{subsec}
% \stepcounter{subsec}

% \newcommand{\bchap}[1]{
%     \color{Red} CHAPITRE : #1\color{black}\\
% }

\newcommand{\bseq}[1]{
    \def\sseq{\color{Red} CHAPITRE \theseq{} : #1\color{black}\\}
    \def\shortTitle{\MakeUppercase{#1}}
    \def\theme{#1}
    \setcounter{sec}{0}
}

\newcommand{\bsec}[1]{
    \section{#1}
    \def\ssec{\color{Red} \Roman{sec}. #1\color{black}\\}
    \stepcounter{sec}
    \setcounter{subsec}{0}
}

\newcommand{\bsubsec}[1]{
    \subsection{#1}
    \def\ssubsec{\color{Green} \thesubsec) #1\color{black}\\}
    \stepcounter{subsec}
}

\newcommand{\palt}[2]{
    \alt<#1->{\result{#2}}{\phantom{#2}}
    % \alt<#1->{\result{#2}}{\pgfinvisible #2 \pgfshown}
    % \alt<#1->{\result{#2}}{\textcolor{white}{#2}}
    % \alt<#1->{#2}{\awsr{#2}}
}

\NewDocumentCommand{\aalt}{O{2} m m}{%
    \alt<#1>{#2}{#3}
}

\newcounter{question}

\newcommand{\startQuestions}{
    \setcounter{question}{2}
}

\newcommand{\iquestion}[2]{
    \item $\question{#1}{#2}$
}

\newcommand{\question}[2]{
    #1 = \palt{\thequestion}{#2}
    \stepcounter{question}
}

% \renewcommand{\question}[2]{
%         #1 = #2
% }


\newcommand{\disableAnimation}{
    % \renewcommand{\question}[2]{
    %     ##1 = ##2
    % }
    
    \renewcommand{\palt}[2]{
        \result{##2}
    }

    \renewcommand{\pause}{}
}

\newcommand{\shortAnimation}{
    \renewcommand{\palt}[2]{
        % \alt<2->{\result{#2}}{\phantom{#2}}
    }
}

\newcommand{\firstSlide}{
    % \renewcommand{\question}[2]{
    %     ##1 =
    % }

    \renewcommand{\palt}[2]{
        \phantom{##2}
        % \pgfinvisible ##2 \pgfshown
    }
}

\newcounter{timer}
\NewDocumentCommand{\qf}{m O{15}}{
    \setcounter{qf}{0}
    \slide{EXERCICES}{\qfSUB{}{
        \begin{itemize}
            \item \large $#2\sec$ par question
            \item \listSize{#1}\thesize{} questions
        \end{itemize}
    }}
    \foreach \q in {#1}{
        \stepcounter{qf}
        \setcounter{choice}{1}
        % Timer slides
        \setcounter{timer}{#2}
        \whiledo{\thetimer>0}{
            \addtocounter{timer}{-1}
            \slide{QUESTIONS FLASH}{
                % \hspace{0.25cm}
                \large \color{Blue}\theqf.\color{black}
                \hspace*{-1cm} \huge \listElement{\q}{0}\\
                \ifthenelse{\boolean{qftimer}}{
                    \vspace{1cm}
                    \transduration{1}
                    \centering
                    \normalsize \color{CadetBlue}$\thetimer\sec$
                }{
                    \transduration{#2}
                    \setcounter{timer}{0}
                }
            }
        }
    }

    \slide{QUESTIONS FLASH}{
        \qfRes{#1}
    }
}

\NewDocumentCommand{\dividePage}{mm O{0.5}}{
    \pgfmathparse{1-#3}
    \begin{columns}[T]
        \begin{column}{#3\textwidth}
            #1
        \end{column}
        \begin{column}{\pgfmathresult\textwidth}
            #2
        \end{column}
    \end{columns}
}

%% Article %%
\documentclass[a4paper, 12pt, 
% landscape
]{extarticle} \usepackage[top=1.5cm, bottom=2cm, left=2cm, right=2cm]{geometry}
\usepackage[dvipsnames, table]{xcolor}
\usepackage{lastpage}
\usepackage{fancyhdr}
\usepackage{titlesec}
\usepackage{enumitem}
\usepackage{longtable}
\usepackage{pdfpages}
% FANCYHDR

\def\background{
    \fancyhead[L]{
        \begin{tikzpicture}[overlay]
            \fill[gradeColor!65] (-3cm,-\paperheight) rectangle (-1cm,2cm);
        \end{tikzpicture}%
    }
    \fancyhead[R]{
        \begin{tikzpicture}[overlay]
            \node[anchor=north east, font=\fontsize{40}{36}\selectfont] at (1.81cm,0.1cm + 0.55\headheight)
            {\hypersetup{urlcolor=gradeColor!65}\link{\grade}};
        \end{tikzpicture}%
    }
}

\def\emptyBackground{
    \fancyhead[L]{}
    \fancyhead[R]{}
}

\setlength{\headheight}{18pt}
\fancyhead[C]{\normalsize \title}
\background
\fancyfoot[L]{\authors}
\fancyfoot[C]{$\textbf{Page}\;\mathbf{\thepage / {\hypersetup{linkcolor=black}\pageref{LastPage}}}$}
\fancyfoot[R]{\date}

\fancypagestyle{firstpage}{
    \setlength{\headheight}{29pt}
    \fancyhead[C]{\LARGE \title}
}

\def\assignmentNameWidth{7.5cm}
\fancypagestyle{assignment}{
    \setlength{\headheight}{29pt}
    \fancyhead[C]{}
    \fancyhead[L]{\large \title}
    \fancyhead[R]{%
        \begin{tabular}{p{\assignmentNameWidth}p{2.5cm}}%
            \normalsize nom:& \normalsize classe: \link{\grade}\_\\%
            \normalsize prénom:& \normalsize date:\\%
            % \normalsize date:\hspace*{3.5cm}%
        \end{tabular}%
    }
}

\fancypagestyle{empty}{
    \renewcommand{\headrulewidth}{0pt}
    \setlength{\headheight}{-10pt}
    \fancyhead[C]{}
    \fancyhead[R]{}
    \fancyhead[L]{}
    \fancyfoot[L]{}
    \fancyfoot[C]{}
    \fancyfoot[R]{}
}

\fancypagestyle{empty-head}{
    \renewcommand{\headrulewidth}{0pt}
    \setlength{\headheight}{-10pt}
    \fancyhead[C]{}
    \fancyhead[R]{}
    \fancyhead[L]{}
}

\fancypagestyle{assignment-empty-foot}{
    \setlength{\headheight}{29pt}
    \fancyhead[C]{}
    \fancyhead[L]{\large \title}
    \fancyhead[R]{%
        \begin{tabular}{p{\assignmentNameWidth}p{0.15\pdfpagewidth}}%
            \normalsize nom:& \normalsize classe:\\%
            \normalsize prénom:& \normalsize date:\\%
            % \normalsize date:\hspace*{3.5cm}%
        \end{tabular}%
    }
    \fancyfoot[L]{}
    \fancyfoot[C]{}
    \fancyfoot[R]{}
}

\fancypagestyle{small}{
    \setlength{\headheight}{20pt}
    \fancyhead[C]{}
    \fancyhead[C]{\large \title}
    \fancyhead[L]{}
    \fancyhead[R]{}
    \fancyfoot[L]{}
    \fancyfoot[C]{}
    \fancyfoot[R]{}
}

\fancypagestyle{screenread}{
    \fancyhead[C]{}
    \fancyfoot[L]{}
    \fancyfoot[C]{}
    \fancyfoot[R]{}
    % \background
}

\thispagestyle{firstpage}

% \fancyfoot[C]{\textbf{Page 1/1}}

\def\title{\theme}
\def\authors{Jules PESIN}

\pagestyle{fancy}

% \titleformat*{\section}{\small\bfseries}

\titleformat{\section}
{\normalfont\large\bfseries\color{\SectionColor}}{\thesection}{0.6em}{}

\titleformat{\subsection}
{\normalfont\normalsize\bfseries\color{\SubSectionColor}}{\thesubsection}{0.6em}{}

\titleformat{\subsubsection}
{\normalfont\small\bfseries\color{\SubSubSectionColor}}{\thesubsubsection}{0.6em}{}


% \renewcommand{\theenumii}{.\arabic{enumii}}
% \frenchbsetup{StandardItemLabels=true}
\renewcommand{\theenumi}{\small\color{Blue}\arabic{enumi}}
\renewcommand{\labelenumii}{\scriptsize\color{RoyalBlue}\alph{enumii})}
% \renewcommand{\labelitemi}{$\color{Blue}.$}

\newcommand*{\setItemColor}[1]{
    \renewcommand{\theenumi}{\small\color{#1}\arabic{enumi}}
    \renewcommand{\labelenumii}{\scriptsize\color{#1}\alph{enumii})}
    \renewcommand{\labelitemi}{$\color{#1}\blacksquare$}
    \renewcommand{\labelitemii}{$\color{#1}\blacktriangleright$}
    \renewcommand{\labelitemiii}{$\color{#1}\bullet$}
}

\definecolor{gradeColor}{RGB}{200, 50, 50}

\backgroundsetup{
    scale=1,
    color=gradeColor,
    opacity=0.65,
    position=current page.south west,
    angle = 0,
    contents={%
        % Crée une bande colorée sur la gauche
        \begin{tikzpicture}[remember picture, overlay]
            \fill[gradeColor] (0,0) rectangle (1cm, \paperheight);
            % Texte en haut à droite
            \node[shift={(-0.32,-0.25)},anchor=north east, color=gradeColor, font=\fontsize{40}{36}\selectfont] 
                at (current page.north east) {6e};
        \end{tikzpicture}%
    }
}

\newcommand{\emptyBackground}{\backgroundsetup{
    position=current page.south west,
    angle=0,
    scale=1,
    color=gradeColor,
    hshift=0cm,  % No shift
    vshift=0cm,
    contents={}
}}

% BEAMER CONVERSION

% \newcommand{\bchap}[1]{\def\title{Chapitre: #1}}
\newcommand{\bseq}[1]{\def\title{Séquence: #1}}
\newcommand{\bsec}[1]{\section{#1}}
\newcommand{\bsubsec}[1]{\subsection{#1}}

\newcommand{\ssec}{}
\newcommand{\ssubsec}{}

\newcommand{\slide}[2]{#2}

\newcommand{\startQuestions}{}
\newcommand{\iquestion}[2]{\item $#1 = \result{#2}$}

\newcommand{\palt}[2]{\result{#2}}
\NewDocumentCommand{\aalt}{o m m}{%
    \noindent #2\\#3%
}

\newcommand{\disableAnimation}{}
\newcommand{\shortAnimation}{}

\newcommand{\firstSlide}{
    \renewcommand{\iquestion}[2]{\item $##1 = \phantom{##2}$}
    \renewcommand{\palt}[2]{
        \phantom{##2}
    }
}

\newenvironment{columns}[1][T]{}{}
\newenvironment{column}[1]{\begin{minipage}{#1}}{\end{minipage}}

\newcounter{qf}
\NewDocumentCommand{\qf}{m O{15}}{
    \qfSUB{}{
        \qfRes{#1}
    }
}

\newcounter{annex}
\renewcommand{\theannex}{\Alph{annex}} % Define how the annex counter will be displayed

\newcommand{\annex}[1]{%
    \changelocaltocdepth{0}
    \setcounter{section}{0}%
    \setcounter{subsection}{0}%
    \setcounter{subsubsection}{0}%
    \newpage%
    \fancyhead[L]{\color{Red} ANNEXE \theannex}
    \refstepcounter{annex}
    \label{annex:\theannex}
    \input{#1}
    \changelocaltocdepth{2}
}

\newcommand*{\rannex}[1]{
    (\hyperref[annex:#1]{Annexe #1})
}

\newcommand{\changelocaltocdepth}[1]{%
    \addtocontents{toc}{\protect\setcounter{tocdepth}{#1}}%
    \setcounter{tocdepth}{#1}%
}

\usepackage[fontsize=14pt]{fontsize}

\usepackage[T1]{fontenc}
\usepackage[french]{babel}

\usepackage[utf8]{inputenc}
\usepackage{amsmath}
\usepackage{amsthm}
\usepackage{amssymb}
\usepackage{graphicx}
\usepackage{dashundergaps}
\usepackage{array}
\usepackage{multicol}
\usepackage{wrapfig}
\usepackage{numprint}
\usepackage{ulem}
\usepackage{hyperref}
\usepackage{mathrsfs}
\usepackage{mathtools}
\usepackage[many]{tcolorbox}
\usepackage{xparse}
\usepackage{float}
\usepackage{lipsum}
\usepackage{pgf}
\usepackage{ifthen}
\usepackage{caption}
\usepackage{tikz}
\usepackage{xifthen}

% \usepackage[squaren,Gray]{SIunits}

% BREVET

% \usepackage{makeidx}
% \usepackage{fancybox}
% \usepackage{tabularx}
% \usepackage[normalem]{ulem}
% \usepackage{pifont}
% \usepackage{lscape}
% \usepackage{diagbox}
% \usepackage{multirow} 
% \usepackage{textcomp}
% \usepackage{scratch3}
% \usepackage[T1]{fontenc}
% \usepackage{fourier}
% \usepackage[french]{babel}
% \usepackage{pstricks}

% \usepackage[scaled=0.875]{helvet}
% \usepackage{pst-plot,pst-text,pst-tree,pstricks-add}

% fancyhdr

\setlength{\headheight}{18pt}
\fancyhead[C]{\normalsize \title}
% \renewcommand{\headrulewidth}{0pt} % Remove header line
\fancyhead[R]{}
\fancyfoot[L]{\author}
\fancyfoot[C]{\textbf{Page \thepage/\pageref{LastPage}}}
\fancyfoot[R]{\date}

\fancypagestyle{firstpage}{
    \setlength{\headheight}{29pt}
    \fancyhead[C]{\LARGE \title}
    \fancyhead[R]{}
    \fancyfoot[L]{\author}
    \fancyfoot[C]{\textbf{Page \thepage/\pageref{LastPage}}}
    \fancyfoot[R]{\date}
}

\thispagestyle{firstpage}

% \fancyfoot[C]{\textbf{Page 1/1}}

% HYPERREF

\hypersetup{
    colorlinks=true,       % false: boxed links; true: colored links
    linkcolor=red,          % color of internal links (change box color with linkbordercolor)
    citecolor=green,        % color of links to bibliography
    filecolor=magenta,      % color of file links
    urlcolor=blue,          % color of external links
    urlbordercolor=blue,    % borders of external links
    linkbordercolor=red,    % borders of internal links
    pdfborderstyle={/S/U/W 1}% border style will be underline of width 1pt
}

\usepackage[fontsize=14pt]{fontsize}

\usepackage[T1]{fontenc}
\usepackage[french]{babel}
\usepackage[utf8]{inputenc}

\frenchbsetup{StandardItemLabels=true}

% GLOBAL VARIABLES %%%
\graphicspath{{images}}
\def\cwidth{4cm}
\def\tspace{0.5cm}

% BOOLEAN %%%
\newboolean{anwser}
\newboolean{demonstration}
\newboolean{boxedProperties}
\newboolean{showID}
\newboolean{parenthisedID}
\newboolean{animated}
\newboolean{outline}

\setboolean{anwser}{false}
\setboolean{demonstration}{true}
\setboolean{parenthisedID}{true}
\setboolean{showID}{true}
\setboolean{boxedProperties}{false} % false = edge
\setboolean{outline}{false}

\def\DefinitionColor{PineGreen}
\def\PropertyColor{Blue}
\def\TheoremColor{Plum}

\def\SectionColor{Red}
\def\SubSectionColor{Green}

\setboolean{animated}{true}

% \DeclareMathOperator{\PGCD}{PGCD}
% \DeclareMathOperator{\PPCM}{PPCM}

\DeclareMathOperator{\sh}{sh}
\DeclareMathOperator{\ch}{ch}
% \DeclareMathOperator{\th}{th}

\DeclareMathOperator{\argsh}{argsh}
\DeclareMathOperator{\argch}{argch}
\DeclareMathOperator{\argth}{argth}
\DeclareMathOperator{\I}{I}
\DeclareMathOperator{\Id}{Id}
\DeclareMathOperator{\Ker}{Ker}
% \DeclareMathOperator{\dl}{o}
\newcommand{\dl}[1]{
    \operatorname*{o}_{#1}
}

\def\deg{\ensuremath{^\circ}}
\def\prll{\mathbin{\!/\mkern-5mu/\!}}
\renewcommand{\parallel}{\mathbin{\!/\mkern-5mu/\!}}

\def\octet{\textrm{o}}
\def\byte{\textrm{B}}

\def\hour{\textrm{h}}
\def\minute{\textrm{min}}
\def\second{\textrm{s}}
% ENVIRONMENT
\newenvironment{mysection}[1][gray!20]{%
    \begin{sectionBox}[#1]
}{%
    \end{sectionBox}
}

\newenvironment{mysubsection}[1][gray!20]{%
    \begin{subsectionBox}[#1]
}{%
    \end{subsectionBox}
}

% Switch implementation
\newboolean{default}
\newcommand{\case}{}
\newcommand{\default}{}

\newenvironment{switch}[1]{%
    \setboolean{default}{true}
    \renewcommand{\case}[2]{\ifthenelse{\equal{#1}{##1}}{%
        \setboolean{default}{false}##2}{}}%
    \renewcommand{\default}[1]{\ifthenelse{\boolean{default}}{##1}{}}
}{}

% SECTIONS
\input{header/command/sections.tex}

% ANSWERS
\newlength{\parline}
\newlength{\paroutindent}
\newlength{\lineheight}
\setlength{\lineheight}{\heightof{abcdefghijklmnoprstuvwxyz}}

\newcommand{\countlines}[1]{%
    \setlength{\paroutindent}{\expandafter\parindent}
    \setlength{\parline}{\heightof{\noindent\begin{minipage}{\linewidth}%
                \setlength{\parindent}{\paroutindent}#1\end{minipage}}}%
    \pgfmathparse{round(\parline / (0.9*\lineheight))}
    \newcount\linecount
    \pgfmathsetcount{\linecount}{\pgfmathresult}
}

\newcommand{\looptext}[2]{%
    \noindent
    \newcount\printcount
    \printcount=#2
    \loop
        #1
        \advance\printcount by -1
        \ifnum\printcount>0
    \repeat
}

\newcommand{\awsr}[1]{%
    \ifthenelse{\boolean{answer}}{
        \result{#1}
    }{
        \countlines{#1}
        \pgfmathsetcount{\linecount}{\linecount + 1}
        \noindent\hspace{-9pt}
        \looptext{
            \noindent\dotfill
    
        }{\the\linecount}
    }
}

\newcommand{\dottedLines}[1]{%
    \noindent\hspace{-9pt}%

    \looptext{%
        \noindent\dotfill%

    }{#1}
}

\newcommand{\result}[1]{\color{OrangeRed}#1\color{black}}

% MATH
\input{header/command/math.tex}

% IMAGES
\input{header/command/image.tex}

% COMMANDS

\newcommand{\fsize}[1]{\fontsize{#1}{#1}\selectfont}

\NewDocumentCommand{\ifNotNull}{mmo}{
    \IfValueT{#1}{
        \ifx\relax#1\relax
            \IfValueT{#3}{
                #3
            }
        \else
            #2
        \fi
    }
}

\NewDocumentCommand{\ilink}{m g}{%
    \item
    \IfValueTF{#2}{\link{#1}{#2}}{\link{#1}}
}

\NewDocumentCommand{\link}{m g}{%
    \csn{#1}%
    \IfValueT{#2}{(#2)}%
}

\NewDocumentCommand{\TODO}{g}{%
    {\color{Red} $\rightarrow$ \textbf{TODO}
    \IfValueT{#1}{(#1)}}
    % \color{black}
}

\newcommand{\leconInfoBox}[2]{
    \textbf{#1 :}\vspace{-0.25cm}
        \begin{multicols}{2}
            \begin{itemize}[label=$\blacktriangleright$, font = \small \color{Red}]
                #2
            \end{itemize}
        \end{multicols}
        \vspace{-0.4cm}
}

% TCOLORBOX

\input{header/command/tcolorbox.tex}

\NewDocumentCommand{\leconInfo}{mooo}{
    \begin{infoBox}
        \leconInfoBox{Niveaux}{#1}
        \ifNotNull{#2}{
            \tcbline
            \leconInfoBox{Prérequis}{#2}
        }
        \ifNotNull{#3}{
            \tcbline
            \leconInfoBox{Thèmes}{#3}
        }
        \ifNotNull{#4}{
            \tcbline
            \textbf{Motivation :} 
            #4
        }
    \end{infoBox}
}

\NewDocumentCommand{\seanceInfo}{oooooooo}{
    \begin{infoBox}
        \vspace{-0.05cm}
        \begin{tcbitemize}[raster rows=1,raster columns=20,raster height=1.65cm,
            raster every box/.style={colframe=red!50!black,colback=red!10!white}]
            \tcbitem[raster multicolumn=6] \textbf{Date :} #1
            \tcbitem[raster multicolumn=10] \textbf{Séquence :} #2
            \tcbitem[raster multicolumn=4] \textbf{Séance :} #3
        \end{tcbitemize}
        \vspace{-0.25cm}
        \ifNotNull{#4}{\tcbline \textbf{Objectif :} #4}
        \ifNotNull{#5}{\tcbline \leconInfoBox{Classe(s)}{#5}}
        \ifNotNull{#6}{\tcbline \leconInfoBox{Prérequi(s)}{#6}}
        \ifNotNull{#7}{\tcbline \textbf{Séance précédente :} #7}
        \ifNotNull{#7}{\tcbline \leconInfoBox{Matériel(s)}{#8}}
    \end{infoBox}
}

\def\pDscr{\tcbitem[enhanced jigsaw, breakable,
    raster multicolumn=6]
}
\def\pMdlt{\tcbitem[enhanced jigsaw, breakable,
    raster multicolumn=11]
}
\def\pTime{\tcbitem[enhanced jigsaw, breakable,
    raster multicolumn=3, halign=center]
}

\newcommand{\prepRow}[3]{
    \tcbitem[raster multicolumn=20]
    \tcblower

    \pDscr #1
    \pMdlt #2
    \pTime #3
}

\newcommand{\prepTable}[1]{
    \begin{prepBox}
        \begin{tcbitemize}[enhanced jigsaw, breakable, raster rows=1,raster columns=20,raster height=1.1cm, halign=center,
            raster every box/.style={enhanced jigsaw, breakable, colframe=Blue!50!black,colback=Blue!10!white}]
            \pDscr \textbf{Descriptif}
            \pMdlt \textbf{Modalité}
            \pTime \textbf{Durée}
        \end{tcbitemize}
        \begin{tcbitemize}[enhanced jigsaw, breakable,
            raster equal height = rows, 
            raster columns=20, frame hidden,
            raster every box/.style={
                enhanced jigsaw, breakable,
                opacityback=0, valign=top, 
                size = tight
            }]
            #1
        \end{tcbitemize}
    \end{prepBox}
}

% TIKZ

\newcommand{\ctikz}[1]{
    \begin{center}
        \begin{tikzpicture}
            #1
        \end{tikzpicture}
    \end{center}
}

\newcommand{\axis}[1]{%Draw coordinate axes
    \draw[thin, -Stealth] (-0.5,0) -- (#1,0);% node[right] {$x$}; % x-axis
    \draw[thin, -Stealth] (0,-0.5) -- (0,#1);% node[above] {$y$}; % y-axis
}

\newcommand{\drawGrid}[3]{
    \foreach \n in {0,...,#1}
        \draw[line width = #3] (\n,0) -- (\n,#2);
    \foreach \n in {0,...,#2}
        \draw[line width = #3] (0,\n) -- (#1,\n);
}

\newcommand{\drawPoint}[4]{
    \node[shift={#4}, color = \pointColor] at (#2 - 0.5,#3 - 0.5) {#1};
    \draw[line width = \crossWidth, shift={#4}, color = \pointColor] (#2 - 0.25,#3) -- (#2 + 0.25,#3);
    \draw[line width = \crossWidth, shift={#4}, color = \pointColor] (#2,#3 - 0.25) -- (#2,#3 + 0.25);
}

% Tabular
\newcolumntype{C}[1]{>{\centering\arraybackslash}p{#1}}
\newcolumntype{M}[1]{>{\centering\arraybackslash}m{#1}}
\newcolumntype{K}{@{}m{0pt}@{}}

% GEOMETRY

% \newcommand{\restoregeometry}{def}

\newcommand{\multiColItemize}[2]{
    \begin{multicols}{#1}
        \begin{itemize}
            #2
        \end{itemize}
    \end{multicols}
}

\newcommand{\multiColEnumerate}[2]{
    \ifthenelse{\isequivalentto{#1}{1}}{
        \begin{enumerate}
            #2
        \end{enumerate}
    }{
        \begin{multicols}{#1}
            \begin{enumerate}
                #2
            \end{enumerate}
        \end{multicols}
    }
}

\makeatletter
\newcommand\pgfinvisible{\pgfsys@begininvisible}
\newcommand\pgfshown{\pgfsys@endinvisible}
\makeatother

\renewcommand*{\phantom}[1]{
    \pgfinvisible #1 \pgfshown
}

\newcounter{size}
\newcommand{\listSize}[1]{%
    \setcounter{size}{0}%
    \foreach \n in {#1}{\stepcounter{size}}%
    % \thesize
}

\newcounter{elemPos}
\newcommand{\listElement}[2]{
    \setcounter{elemPos}{0} % Start counting from 1
    \def\resultVal{0} % Default value
    \renewcommand*{\do}[1]{%
        \ifnumequal{\value{elemPos}}{#2}{%
            \def\resultVal{##1}%
            \listbreak% Break out of the loop
        }{}%
        \stepcounter{elemPos}%
    }
    % \docsvlist{#1}
    \expandafter\docsvlist\expandafter{#1} % Expand the list before passing it to \docsvlist
    \resultVal
}

% \NewDocumentCommand{\exoslide}{m O{10cm}}{
%     \slide{}{
%         \img{\imgf{#1}}[#2]
%     }
% }

\NewDocumentCommand{\exoSlide}{m O{10cm} O{1} O{} O{exo}}{%
    \slide{#5}{%
        \ifthenelse{\equal{#3}{1}}{\vspace{-0.5cm}}{\vspace{-1cm}}
        \def\exercices{\foreach \q in {#1}{\imgp{\q}[#2]\vspace{-0.5cm}}}
        \exo{#1}{\wideFrame[7em]{\bvspace{0.25cm}\avspace{-0.25cm}
            \ifthenelse{\equal{#3}{1}}{\exercices}
            {\begin{multicols}{#3}\exercices\end{multicols}}}
            \avspace{0.75cm}
        }[#4]
    }
}

\NewDocumentCommand{\exoList}{m O{} O{3}}{%
    \section*{Exercices}%
    \slide{EXERCICES}{
        \exo{#2}{
            \vspace{-0.25cm}
            \multiColEnumerate{#3}{
                \foreach \q in {#1}{
                    \item \q
                }
            }
        }
    }
}

\newcommand{\questions}[1]{
    \begin{enumerate}
        \foreach \q in {#1}{
            \item \q\\
            \vspace*{-0.45cm}
            \dottedLines{3}
        }
    \end{enumerate}
}

% Define a new boolean for checking if the section is starred
\newboolean{section@star}

\makeatletter
% Redefine \section and \section* to set the boolean
\let\old@section\section
\renewcommand{\section}{%
    \@ifstar
        {\setboolean{section@star}{true}\old@section*}
        {\setboolean{section@star}{false}\old@section}%
}
\makeatother

\newcommand{\qt}[1]{«\textit{#1}»}

\newcommand{\calc}[1]{\numexpr#1\relax}
\newcommand{\ncalc}[1]{\number\calc{#1}}
\newcommand{\pcalc}[1]{\numprint{\ncalc{#1}}}

\newcommand{\setgrade}[1]{
    \def\grade{#1}
    % \begin{switch}{#1}
    %     \case{6e}{\global\definecolor{gradeColor}{hex}{FA8072}}
    %     \default{
    %         Default
    %         \global\definecolor{gradeColor}{RGB}{200, 50, 50}
    %     }
    % \end{switch}
    \ifthenelse{\equal{#1}{6e}}{
        \definecolor{gradeColor}{HTML}{C6233D} % FA8072 in hex
    }{
    \ifthenelse{\equal{#1}{5e}}{
        \definecolor{gradeColor}{HTML}{088255}
    }{
    \ifthenelse{\equal{#1}{4e}}{
        \definecolor{gradeColor}{HTML}{1466A8}
    }{
    \ifthenelse{\equal{#1}{3e}}{
        \definecolor{gradeColor}{HTML}{844499}
    }{
        \definecolor{gradeColor}{RGB}{0, 0, 0}
    }}}}
}

\gdef\phase{}
\newcommand{\setPhase}[1]{%
    \begin{switch}{#1}
        \case{exo}{\gdef\phase{EXERCICES}}
        \case{cr}{\gdef\phase{COURS}}
        \case{qf}{\gdef\phase{QUESTIONS FLASH}}
        \case{dm}{\gdef\phase{DEVOIR MAISON}}
        \default{\gdef\phase{#1}}
    \end{switch}
}

\newcounter{savedenumi}
\setcounter{savedenumi}{0}
\xdef\savedenumi{0}
% \newcommand{\saveenumi}{
%     % \xdef\savedenumi{\calc{\theenumi-1}}
%     \setcounter{savedenumi}{0}
% }

\newcommand{\saveenumi}[1]{
    \setcounter{savedenumi}{#1}
}

\newcommand{\loadenumi}{
    \setItemColor{\currentColor}
    \setcounter{enumi}{\thesavedenumi}
}

\newcommand\csn[1]{\csname #1\endcsname}

\newcommand{\vect}[1]{\ensuremath{\overrightarrow{#1}}}
% \newcommand{\vect}[1]{\overrightarrow{\,\mathstrut#1\,}}
\newcommand{\m}[1]{\ensuremath{\mathbf{#1}}}
\newcommand\lm[2]{\lim_{#1\to#2}}

\def\eqv{\Leftrightarrow}
\def\ssi{si et seulement si }
\def\pt{pour tout }
\def\poly2{fonction polynôme du second degré }
\def\eq2{équation second degré }
\def\discr{b^2-4ac}

% MATH TEXT
\def\et{\textrm{ et }}
\def\si{\textrm{ si }}
\def\avec{\textrm{ avec }}
\def\car{\textrm{ car }}
\def\alors{\textrm{ alors }}
\def\ou{\textrm{ ou }}
\def\ona{\textrm{ on a }}

\def\iet{\shortintertext{et}}
\def\ialors{\shortintertext{alors}}
\def\idou{\shortintertext{d'où}}
\def\ior{\shortintertext{or}}
\def\iona{\shortintertext{on a}}

\def\studentinfo{
    \vspace*{-1cm}
    \begin{minipage}{0.35\linewidth}
        nom: \dotfill
    \end{minipage}
    \begin{minipage}{0.35\linewidth}
        prénom: \dotfill
    \end{minipage}
    \begin{minipage}{0.15\linewidth}
        classes: \dotfill
    \end{minipage}
    
    \noindent\hrulefill
}

% UNITS
\def\cm{\,\centi\meter}
\def\km{\,\kilo\meter}
\newcommand{\defl}[2]{%
    \expandafter\def\csname #1\endcsname{\href{#2}{#1}\space}%
}

% Page Eduscol
\defl{Eduscol Cycle 3}{https://eduscol.education.fr/251/mathematiques-cycle-3}
\defl{Eduscol Cycle 4}{https://eduscol.education.fr/280/mathematiques-cycle-4}
\defl{Eduscol Lycée Général et technologique}{https://eduscol.education.fr/1723/programmes-et-ressources-en-mathematiques-voie-gt}
\defl{Eduscol Lycée Professionnel}{https://eduscol.education.fr/1793/programmes-et-ressources-en-mathematiques-voie-professionnelle}

% Repères annuels
\defl{Cycle 3}{https://eduscol.education.fr/document/14026/download}
\defl{Cycle 4}{https://eduscol.education.fr/document/14080/download}

% Attendus de fin d'année
\defl{5e}{https://eduscol.education.fr/document/14044/download}
\defl{4e}{https://eduscol.education.fr/document/14056/download}
\defl{3e}{https://eduscol.education.fr/document/14068/download}

% Programme de mathématiques
\defl{cycle 3}{https://eduscol.education.fr/document/50990/download}
\defl{cycle 4}{https://cache.media.education.gouv.fr/file/31/89/1/ensel714_annexe3_1312891.pdf}
\defl{2nd}{https://eduscol.education.fr/document/24553/download}
\defl{1re}{https://eduscol.education.fr/document/24565/download}
\defl{1re STL}{https://eduscol.education.fr/document/23098/download}
\defl{1re STI2D}{https://eduscol.education.fr/document/24919/download}
\defl{Terminale Option Spécialité}{https://eduscol.education.fr/document/24568/download}
\defl{Terminale Option Complémentaire}{https://eduscol.education.fr/document/24571/download}
\defl{Terminale Option Expertes}{https://eduscol.education.fr/document/24574/download}
\defl{Terminale STL}{https://eduscol.education.fr/document/23107/download}
\defl{Terminale STI2D}{https://eduscol.education.fr/document/24922/download}
% Ressources thématiques
\defl{Proportionnalité}{https://eduscol.education.fr/document/17281/download}
\defl{Probabilités}{https://eduscol.education.fr/document/17275/download}
\defl{Fonctions}{https://eduscol.education.fr/document/17287/download}
\defl{Traitement des données}{https://eduscol.education.fr/document/17269/download}

\defl{Fonctions}{https://eduscol.education.fr/document/17287/download}
\defl{Fractions}{https://eduscol.education.fr/document/17239/download}
\defl{Nombres relatifs}{https://eduscol.education.fr/document/17245/download}
\defl{Puissances}{https://eduscol.education.fr/document/17251/download}
\defl{Divisibilité et nombres premiers}{https://eduscol.education.fr/document/17257/download}
\defl{Calcul littéral}{https://eduscol.education.fr/document/17263/download}

\defl{Grandeurs et mesures}{https://eduscol.education.fr/document/17293/download}
\defl{Algorithmique et programmation}{https://eduscol.education.fr/document/17311/download}

\defl{Suites}{https://eduscol.education.fr/document/24586/download}
\defl{Produit Scalaire}{https://eduscol.education.fr/document/24589/download}
\defl{Raisonnement et démonstration (seconde)}{https://eduscol.education.fr/document/24580/download}
\defl{Raisonnement et démonstrations (première)}{https://eduscol.education.fr/document/24583/download}

\captionsetup{labelformat=empty,labelsep=none}

% \setboolean{boxedProperties}{true} % false = edge
% \setboolean{parenthisedID}{false}
% \setboolean{showID}{false}

% \def\DefinitionColor{Red}
\def\PropertyColor{Red}
\def\TheoremColor{Red}

% TIKZ
\def\crossWidth{0.25mm}
\def\pointColor{blue}

% \documentclass[tikz,border={0.23cm 0.25cm}]{standalone} \newcommand{\ifArticle}[1]{}
\newcommand{\ifBeamer}[1]{}
\newcommand{\setItemColor}[1]{}

\usepackage[fontsize=14pt]{fontsize}

\usepackage[T1]{fontenc}
\usepackage[french]{babel}

\usepackage[utf8]{inputenc}
\usepackage{amsmath}
\usepackage{amsthm}
\usepackage{amssymb}
\usepackage{graphicx}
\usepackage{dashundergaps}
\usepackage{array}
\usepackage{multicol}
\usepackage{wrapfig}
\usepackage{numprint}
\usepackage{ulem}
\usepackage{hyperref}
\usepackage{mathrsfs}
\usepackage{mathtools}
\usepackage[many]{tcolorbox}
\usepackage{xparse}
\usepackage{float}
\usepackage{lipsum}
\usepackage{pgf}
\usepackage{ifthen}
\usepackage{caption}
\usepackage{tikz}
\usepackage{xifthen}

% \usepackage[squaren,Gray]{SIunits}

% BREVET

% \usepackage{makeidx}
% \usepackage{fancybox}
% \usepackage{tabularx}
% \usepackage[normalem]{ulem}
% \usepackage{pifont}
% \usepackage{lscape}
% \usepackage{diagbox}
% \usepackage{multirow} 
% \usepackage{textcomp}
% \usepackage{scratch3}
% \usepackage[T1]{fontenc}
% \usepackage{fourier}
% \usepackage[french]{babel}
% \usepackage{pstricks}

% \usepackage[scaled=0.875]{helvet}
% \usepackage{pst-plot,pst-text,pst-tree,pstricks-add}

% fancyhdr

\setlength{\headheight}{18pt}
\fancyhead[C]{\normalsize \title}
% \renewcommand{\headrulewidth}{0pt} % Remove header line
\fancyhead[R]{}
\fancyfoot[L]{\author}
\fancyfoot[C]{\textbf{Page \thepage/\pageref{LastPage}}}
\fancyfoot[R]{\date}

\fancypagestyle{firstpage}{
    \setlength{\headheight}{29pt}
    \fancyhead[C]{\LARGE \title}
    \fancyhead[R]{}
    \fancyfoot[L]{\author}
    \fancyfoot[C]{\textbf{Page \thepage/\pageref{LastPage}}}
    \fancyfoot[R]{\date}
}

\thispagestyle{firstpage}

% \fancyfoot[C]{\textbf{Page 1/1}}

% HYPERREF

\hypersetup{
    colorlinks=true,       % false: boxed links; true: colored links
    linkcolor=red,          % color of internal links (change box color with linkbordercolor)
    citecolor=green,        % color of links to bibliography
    filecolor=magenta,      % color of file links
    urlcolor=blue,          % color of external links
    urlbordercolor=blue,    % borders of external links
    linkbordercolor=red,    % borders of internal links
    pdfborderstyle={/S/U/W 1}% border style will be underline of width 1pt
}

\usepackage[fontsize=14pt]{fontsize}

\usepackage[T1]{fontenc}
\usepackage[french]{babel}
\usepackage[utf8]{inputenc}

\frenchbsetup{StandardItemLabels=true}

% GLOBAL VARIABLES %%%
\graphicspath{{images}}
\def\cwidth{4cm}
\def\tspace{0.5cm}

% BOOLEAN %%%
\newboolean{anwser}
\newboolean{demonstration}
\newboolean{boxedProperties}
\newboolean{showID}
\newboolean{parenthisedID}
\newboolean{animated}
\newboolean{outline}

\setboolean{anwser}{false}
\setboolean{demonstration}{true}
\setboolean{parenthisedID}{true}
\setboolean{showID}{true}
\setboolean{boxedProperties}{false} % false = edge
\setboolean{outline}{false}

\def\DefinitionColor{PineGreen}
\def\PropertyColor{Blue}
\def\TheoremColor{Plum}

\def\SectionColor{Red}
\def\SubSectionColor{Green}

\setboolean{animated}{true}

% \DeclareMathOperator{\PGCD}{PGCD}
% \DeclareMathOperator{\PPCM}{PPCM}

\DeclareMathOperator{\sh}{sh}
\DeclareMathOperator{\ch}{ch}
% \DeclareMathOperator{\th}{th}

\DeclareMathOperator{\argsh}{argsh}
\DeclareMathOperator{\argch}{argch}
\DeclareMathOperator{\argth}{argth}
\DeclareMathOperator{\I}{I}
\DeclareMathOperator{\Id}{Id}
\DeclareMathOperator{\Ker}{Ker}
% \DeclareMathOperator{\dl}{o}
\newcommand{\dl}[1]{
    \operatorname*{o}_{#1}
}

\def\deg{\ensuremath{^\circ}}
\def\prll{\mathbin{\!/\mkern-5mu/\!}}
\renewcommand{\parallel}{\mathbin{\!/\mkern-5mu/\!}}

\def\octet{\textrm{o}}
\def\byte{\textrm{B}}

\def\hour{\textrm{h}}
\def\minute{\textrm{min}}
\def\second{\textrm{s}}
% ENVIRONMENT
\newenvironment{mysection}[1][gray!20]{%
    \begin{sectionBox}[#1]
}{%
    \end{sectionBox}
}

\newenvironment{mysubsection}[1][gray!20]{%
    \begin{subsectionBox}[#1]
}{%
    \end{subsectionBox}
}

% Switch implementation
\newboolean{default}
\newcommand{\case}{}
\newcommand{\default}{}

\newenvironment{switch}[1]{%
    \setboolean{default}{true}
    \renewcommand{\case}[2]{\ifthenelse{\equal{#1}{##1}}{%
        \setboolean{default}{false}##2}{}}%
    \renewcommand{\default}[1]{\ifthenelse{\boolean{default}}{##1}{}}
}{}

% SECTIONS
\input{header/command/sections.tex}

% ANSWERS
\newlength{\parline}
\newlength{\paroutindent}
\newlength{\lineheight}
\setlength{\lineheight}{\heightof{abcdefghijklmnoprstuvwxyz}}

\newcommand{\countlines}[1]{%
    \setlength{\paroutindent}{\expandafter\parindent}
    \setlength{\parline}{\heightof{\noindent\begin{minipage}{\linewidth}%
                \setlength{\parindent}{\paroutindent}#1\end{minipage}}}%
    \pgfmathparse{round(\parline / (0.9*\lineheight))}
    \newcount\linecount
    \pgfmathsetcount{\linecount}{\pgfmathresult}
}

\newcommand{\looptext}[2]{%
    \noindent
    \newcount\printcount
    \printcount=#2
    \loop
        #1
        \advance\printcount by -1
        \ifnum\printcount>0
    \repeat
}

\newcommand{\awsr}[1]{%
    \ifthenelse{\boolean{answer}}{
        \result{#1}
    }{
        \countlines{#1}
        \pgfmathsetcount{\linecount}{\linecount + 1}
        \noindent\hspace{-9pt}
        \looptext{
            \noindent\dotfill
    
        }{\the\linecount}
    }
}

\newcommand{\dottedLines}[1]{%
    \noindent\hspace{-9pt}%

    \looptext{%
        \noindent\dotfill%

    }{#1}
}

\newcommand{\result}[1]{\color{OrangeRed}#1\color{black}}

% MATH
\input{header/command/math.tex}

% IMAGES
\input{header/command/image.tex}

% COMMANDS

\newcommand{\fsize}[1]{\fontsize{#1}{#1}\selectfont}

\NewDocumentCommand{\ifNotNull}{mmo}{
    \IfValueT{#1}{
        \ifx\relax#1\relax
            \IfValueT{#3}{
                #3
            }
        \else
            #2
        \fi
    }
}

\NewDocumentCommand{\ilink}{m g}{%
    \item
    \IfValueTF{#2}{\link{#1}{#2}}{\link{#1}}
}

\NewDocumentCommand{\link}{m g}{%
    \csn{#1}%
    \IfValueT{#2}{(#2)}%
}

\NewDocumentCommand{\TODO}{g}{%
    {\color{Red} $\rightarrow$ \textbf{TODO}
    \IfValueT{#1}{(#1)}}
    % \color{black}
}

\newcommand{\leconInfoBox}[2]{
    \textbf{#1 :}\vspace{-0.25cm}
        \begin{multicols}{2}
            \begin{itemize}[label=$\blacktriangleright$, font = \small \color{Red}]
                #2
            \end{itemize}
        \end{multicols}
        \vspace{-0.4cm}
}

% TCOLORBOX

\input{header/command/tcolorbox.tex}

\NewDocumentCommand{\leconInfo}{mooo}{
    \begin{infoBox}
        \leconInfoBox{Niveaux}{#1}
        \ifNotNull{#2}{
            \tcbline
            \leconInfoBox{Prérequis}{#2}
        }
        \ifNotNull{#3}{
            \tcbline
            \leconInfoBox{Thèmes}{#3}
        }
        \ifNotNull{#4}{
            \tcbline
            \textbf{Motivation :} 
            #4
        }
    \end{infoBox}
}

\NewDocumentCommand{\seanceInfo}{oooooooo}{
    \begin{infoBox}
        \vspace{-0.05cm}
        \begin{tcbitemize}[raster rows=1,raster columns=20,raster height=1.65cm,
            raster every box/.style={colframe=red!50!black,colback=red!10!white}]
            \tcbitem[raster multicolumn=6] \textbf{Date :} #1
            \tcbitem[raster multicolumn=10] \textbf{Séquence :} #2
            \tcbitem[raster multicolumn=4] \textbf{Séance :} #3
        \end{tcbitemize}
        \vspace{-0.25cm}
        \ifNotNull{#4}{\tcbline \textbf{Objectif :} #4}
        \ifNotNull{#5}{\tcbline \leconInfoBox{Classe(s)}{#5}}
        \ifNotNull{#6}{\tcbline \leconInfoBox{Prérequi(s)}{#6}}
        \ifNotNull{#7}{\tcbline \textbf{Séance précédente :} #7}
        \ifNotNull{#7}{\tcbline \leconInfoBox{Matériel(s)}{#8}}
    \end{infoBox}
}

\def\pDscr{\tcbitem[enhanced jigsaw, breakable,
    raster multicolumn=6]
}
\def\pMdlt{\tcbitem[enhanced jigsaw, breakable,
    raster multicolumn=11]
}
\def\pTime{\tcbitem[enhanced jigsaw, breakable,
    raster multicolumn=3, halign=center]
}

\newcommand{\prepRow}[3]{
    \tcbitem[raster multicolumn=20]
    \tcblower

    \pDscr #1
    \pMdlt #2
    \pTime #3
}

\newcommand{\prepTable}[1]{
    \begin{prepBox}
        \begin{tcbitemize}[enhanced jigsaw, breakable, raster rows=1,raster columns=20,raster height=1.1cm, halign=center,
            raster every box/.style={enhanced jigsaw, breakable, colframe=Blue!50!black,colback=Blue!10!white}]
            \pDscr \textbf{Descriptif}
            \pMdlt \textbf{Modalité}
            \pTime \textbf{Durée}
        \end{tcbitemize}
        \begin{tcbitemize}[enhanced jigsaw, breakable,
            raster equal height = rows, 
            raster columns=20, frame hidden,
            raster every box/.style={
                enhanced jigsaw, breakable,
                opacityback=0, valign=top, 
                size = tight
            }]
            #1
        \end{tcbitemize}
    \end{prepBox}
}

% TIKZ

\newcommand{\ctikz}[1]{
    \begin{center}
        \begin{tikzpicture}
            #1
        \end{tikzpicture}
    \end{center}
}

\newcommand{\axis}[1]{%Draw coordinate axes
    \draw[thin, -Stealth] (-0.5,0) -- (#1,0);% node[right] {$x$}; % x-axis
    \draw[thin, -Stealth] (0,-0.5) -- (0,#1);% node[above] {$y$}; % y-axis
}

\newcommand{\drawGrid}[3]{
    \foreach \n in {0,...,#1}
        \draw[line width = #3] (\n,0) -- (\n,#2);
    \foreach \n in {0,...,#2}
        \draw[line width = #3] (0,\n) -- (#1,\n);
}

\newcommand{\drawPoint}[4]{
    \node[shift={#4}, color = \pointColor] at (#2 - 0.5,#3 - 0.5) {#1};
    \draw[line width = \crossWidth, shift={#4}, color = \pointColor] (#2 - 0.25,#3) -- (#2 + 0.25,#3);
    \draw[line width = \crossWidth, shift={#4}, color = \pointColor] (#2,#3 - 0.25) -- (#2,#3 + 0.25);
}

% Tabular
\newcolumntype{C}[1]{>{\centering\arraybackslash}p{#1}}
\newcolumntype{M}[1]{>{\centering\arraybackslash}m{#1}}
\newcolumntype{K}{@{}m{0pt}@{}}

% GEOMETRY

% \newcommand{\restoregeometry}{def}

\newcommand{\multiColItemize}[2]{
    \begin{multicols}{#1}
        \begin{itemize}
            #2
        \end{itemize}
    \end{multicols}
}

\newcommand{\multiColEnumerate}[2]{
    \ifthenelse{\isequivalentto{#1}{1}}{
        \begin{enumerate}
            #2
        \end{enumerate}
    }{
        \begin{multicols}{#1}
            \begin{enumerate}
                #2
            \end{enumerate}
        \end{multicols}
    }
}

\makeatletter
\newcommand\pgfinvisible{\pgfsys@begininvisible}
\newcommand\pgfshown{\pgfsys@endinvisible}
\makeatother

\renewcommand*{\phantom}[1]{
    \pgfinvisible #1 \pgfshown
}

\newcounter{size}
\newcommand{\listSize}[1]{%
    \setcounter{size}{0}%
    \foreach \n in {#1}{\stepcounter{size}}%
    % \thesize
}

\newcounter{elemPos}
\newcommand{\listElement}[2]{
    \setcounter{elemPos}{0} % Start counting from 1
    \def\resultVal{0} % Default value
    \renewcommand*{\do}[1]{%
        \ifnumequal{\value{elemPos}}{#2}{%
            \def\resultVal{##1}%
            \listbreak% Break out of the loop
        }{}%
        \stepcounter{elemPos}%
    }
    % \docsvlist{#1}
    \expandafter\docsvlist\expandafter{#1} % Expand the list before passing it to \docsvlist
    \resultVal
}

% \NewDocumentCommand{\exoslide}{m O{10cm}}{
%     \slide{}{
%         \img{\imgf{#1}}[#2]
%     }
% }

\NewDocumentCommand{\exoSlide}{m O{10cm} O{1} O{} O{exo}}{%
    \slide{#5}{%
        \ifthenelse{\equal{#3}{1}}{\vspace{-0.5cm}}{\vspace{-1cm}}
        \def\exercices{\foreach \q in {#1}{\imgp{\q}[#2]\vspace{-0.5cm}}}
        \exo{#1}{\wideFrame[7em]{\bvspace{0.25cm}\avspace{-0.25cm}
            \ifthenelse{\equal{#3}{1}}{\exercices}
            {\begin{multicols}{#3}\exercices\end{multicols}}}
            \avspace{0.75cm}
        }[#4]
    }
}

\NewDocumentCommand{\exoList}{m O{} O{3}}{%
    \section*{Exercices}%
    \slide{EXERCICES}{
        \exo{#2}{
            \vspace{-0.25cm}
            \multiColEnumerate{#3}{
                \foreach \q in {#1}{
                    \item \q
                }
            }
        }
    }
}

\newcommand{\questions}[1]{
    \begin{enumerate}
        \foreach \q in {#1}{
            \item \q\\
            \vspace*{-0.45cm}
            \dottedLines{3}
        }
    \end{enumerate}
}

% Define a new boolean for checking if the section is starred
\newboolean{section@star}

\makeatletter
% Redefine \section and \section* to set the boolean
\let\old@section\section
\renewcommand{\section}{%
    \@ifstar
        {\setboolean{section@star}{true}\old@section*}
        {\setboolean{section@star}{false}\old@section}%
}
\makeatother

\newcommand{\qt}[1]{«\textit{#1}»}

\newcommand{\calc}[1]{\numexpr#1\relax}
\newcommand{\ncalc}[1]{\number\calc{#1}}
\newcommand{\pcalc}[1]{\numprint{\ncalc{#1}}}

\newcommand{\setgrade}[1]{
    \def\grade{#1}
    % \begin{switch}{#1}
    %     \case{6e}{\global\definecolor{gradeColor}{hex}{FA8072}}
    %     \default{
    %         Default
    %         \global\definecolor{gradeColor}{RGB}{200, 50, 50}
    %     }
    % \end{switch}
    \ifthenelse{\equal{#1}{6e}}{
        \definecolor{gradeColor}{HTML}{C6233D} % FA8072 in hex
    }{
    \ifthenelse{\equal{#1}{5e}}{
        \definecolor{gradeColor}{HTML}{088255}
    }{
    \ifthenelse{\equal{#1}{4e}}{
        \definecolor{gradeColor}{HTML}{1466A8}
    }{
    \ifthenelse{\equal{#1}{3e}}{
        \definecolor{gradeColor}{HTML}{844499}
    }{
        \definecolor{gradeColor}{RGB}{0, 0, 0}
    }}}}
}

\gdef\phase{}
\newcommand{\setPhase}[1]{%
    \begin{switch}{#1}
        \case{exo}{\gdef\phase{EXERCICES}}
        \case{cr}{\gdef\phase{COURS}}
        \case{qf}{\gdef\phase{QUESTIONS FLASH}}
        \case{dm}{\gdef\phase{DEVOIR MAISON}}
        \default{\gdef\phase{#1}}
    \end{switch}
}

\newcounter{savedenumi}
\setcounter{savedenumi}{0}
\xdef\savedenumi{0}
% \newcommand{\saveenumi}{
%     % \xdef\savedenumi{\calc{\theenumi-1}}
%     \setcounter{savedenumi}{0}
% }

\newcommand{\saveenumi}[1]{
    \setcounter{savedenumi}{#1}
}

\newcommand{\loadenumi}{
    \setItemColor{\currentColor}
    \setcounter{enumi}{\thesavedenumi}
}

\newcommand\csn[1]{\csname #1\endcsname}

\newcommand{\vect}[1]{\ensuremath{\overrightarrow{#1}}}
% \newcommand{\vect}[1]{\overrightarrow{\,\mathstrut#1\,}}
\newcommand{\m}[1]{\ensuremath{\mathbf{#1}}}
\newcommand\lm[2]{\lim_{#1\to#2}}

\def\eqv{\Leftrightarrow}
\def\ssi{si et seulement si }
\def\pt{pour tout }
\def\poly2{fonction polynôme du second degré }
\def\eq2{équation second degré }
\def\discr{b^2-4ac}

% MATH TEXT
\def\et{\textrm{ et }}
\def\si{\textrm{ si }}
\def\avec{\textrm{ avec }}
\def\car{\textrm{ car }}
\def\alors{\textrm{ alors }}
\def\ou{\textrm{ ou }}
\def\ona{\textrm{ on a }}

\def\iet{\shortintertext{et}}
\def\ialors{\shortintertext{alors}}
\def\idou{\shortintertext{d'où}}
\def\ior{\shortintertext{or}}
\def\iona{\shortintertext{on a}}

\def\studentinfo{
    \vspace*{-1cm}
    \begin{minipage}{0.35\linewidth}
        nom: \dotfill
    \end{minipage}
    \begin{minipage}{0.35\linewidth}
        prénom: \dotfill
    \end{minipage}
    \begin{minipage}{0.15\linewidth}
        classes: \dotfill
    \end{minipage}
    
    \noindent\hrulefill
}

% UNITS
\def\cm{\,\centi\meter}
\def\km{\,\kilo\meter}
\newcommand{\defl}[2]{%
    \expandafter\def\csname #1\endcsname{\href{#2}{#1}\space}%
}

% Page Eduscol
\defl{Eduscol Cycle 3}{https://eduscol.education.fr/251/mathematiques-cycle-3}
\defl{Eduscol Cycle 4}{https://eduscol.education.fr/280/mathematiques-cycle-4}
\defl{Eduscol Lycée Général et technologique}{https://eduscol.education.fr/1723/programmes-et-ressources-en-mathematiques-voie-gt}
\defl{Eduscol Lycée Professionnel}{https://eduscol.education.fr/1793/programmes-et-ressources-en-mathematiques-voie-professionnelle}

% Repères annuels
\defl{Cycle 3}{https://eduscol.education.fr/document/14026/download}
\defl{Cycle 4}{https://eduscol.education.fr/document/14080/download}

% Attendus de fin d'année
\defl{5e}{https://eduscol.education.fr/document/14044/download}
\defl{4e}{https://eduscol.education.fr/document/14056/download}
\defl{3e}{https://eduscol.education.fr/document/14068/download}

% Programme de mathématiques
\defl{cycle 3}{https://eduscol.education.fr/document/50990/download}
\defl{cycle 4}{https://cache.media.education.gouv.fr/file/31/89/1/ensel714_annexe3_1312891.pdf}
\defl{2nd}{https://eduscol.education.fr/document/24553/download}
\defl{1re}{https://eduscol.education.fr/document/24565/download}
\defl{1re STL}{https://eduscol.education.fr/document/23098/download}
\defl{1re STI2D}{https://eduscol.education.fr/document/24919/download}
\defl{Terminale Option Spécialité}{https://eduscol.education.fr/document/24568/download}
\defl{Terminale Option Complémentaire}{https://eduscol.education.fr/document/24571/download}
\defl{Terminale Option Expertes}{https://eduscol.education.fr/document/24574/download}
\defl{Terminale STL}{https://eduscol.education.fr/document/23107/download}
\defl{Terminale STI2D}{https://eduscol.education.fr/document/24922/download}
% Ressources thématiques
\defl{Proportionnalité}{https://eduscol.education.fr/document/17281/download}
\defl{Probabilités}{https://eduscol.education.fr/document/17275/download}
\defl{Fonctions}{https://eduscol.education.fr/document/17287/download}
\defl{Traitement des données}{https://eduscol.education.fr/document/17269/download}

\defl{Fonctions}{https://eduscol.education.fr/document/17287/download}
\defl{Fractions}{https://eduscol.education.fr/document/17239/download}
\defl{Nombres relatifs}{https://eduscol.education.fr/document/17245/download}
\defl{Puissances}{https://eduscol.education.fr/document/17251/download}
\defl{Divisibilité et nombres premiers}{https://eduscol.education.fr/document/17257/download}
\defl{Calcul littéral}{https://eduscol.education.fr/document/17263/download}

\defl{Grandeurs et mesures}{https://eduscol.education.fr/document/17293/download}
\defl{Algorithmique et programmation}{https://eduscol.education.fr/document/17311/download}

\defl{Suites}{https://eduscol.education.fr/document/24586/download}
\defl{Produit Scalaire}{https://eduscol.education.fr/document/24589/download}
\defl{Raisonnement et démonstration (seconde)}{https://eduscol.education.fr/document/24580/download}
\defl{Raisonnement et démonstrations (première)}{https://eduscol.education.fr/document/24583/download}

\captionsetup{labelformat=empty,labelsep=none}

% \setboolean{boxedProperties}{true} % false = edge
% \setboolean{parenthisedID}{false}
% \setboolean{showID}{false}

% \def\DefinitionColor{Red}
\def\PropertyColor{Red}
\def\TheoremColor{Red}

% TIKZ
\def\crossWidth{0.25mm}
\def\pointColor{blue}

% \geometry{paperheight=9cm}

\begin{document}

% \hfuzz=30pt

\ifBeamer{%
    \renewcommand*{\theenumii}{\alph{enumii}}

    \firstSlide
    \setboolean{showRef}{false}
}

\ifArticle{%
    \renewcommand*{\theenumii}{\alph{enumii}}
    
    \disableAnimation
}



% DOCUMENTS

% %%% VARIABLES %%%
\setSeq{5}{Géométrie plane - Polygones}
\setGrade{6e}
\def\imgPath{enseignement/6e/geometrie-plane/polygones/}
\def\ym{\href{https://www.maths-et-tiques.fr/telech/19Nombres1.pdf}{Yvan Monka}}

%\forPrint
%%


\slide{exo}{
    \exo{}{
        Un triangle equilatéral a le même périmètre q'un carré de \Lg{6} de côté.
        Quelle est la longueur d'un coté de ce triangle ?
    }[\dmeepc{6}[161]]
}
% %%% VARIABLES %%%
\setSeq{5}{Angles et parallélisme}
\setGrade{5e}
\def\imgPath{enseignement/5e/angles-et-parallelisme/}
\def\ym{\href{https://www.maths-et-tiques.fr/telech/19Angles5e.pdf}{Yvan Monka}}
% \forPrint
% \def\caPrefix{6e-juin-2022-}
%%
% [\href{https://myriade.editions-bordas.fr/Myriade4e/assets/cherchons-ensemble-chapitre-10-angles-et-parallelisme-triangles-semblables/preview}{Myriade 4e 2016}]

\obj{
    \item À partir des connaissances suivantes :
    codage des figures, caractérisations angulaires du parallélisme (angles alternes internes, angles correspondants),
    mener des raisonnements en utilisant des propriétés des figures, des configurations et des symétries.
}

\bsec{Angles alternes-internes et angles correspondants}

\slide{exo}{\bshrink
    \act{}{
        \ctikz[\ifBeamer{0.3}\ifArticle{0.5}]{
    \draw[gray!40] (-8,-15) rectangle (21,10);
    \draw [shift={(-2.34,3.99)},thick,color=gradeColor,fill=gradeColor,fill opacity=0.10] (0,0) -- (-75.47:1.06) arc (-75.47:19.12:1.06) -- cycle;
    \draw [shift={(-1.90,2.28)},thick,color=gradeColor,fill=gradeColor,fill opacity=0.10] (0,0) -- (-160.36:1.06) arc (-160.36:-75.47:1.06) -- cycle;
    \draw [shift={(10.34,0.90)},thick,color=gradeColor,fill=gradeColor,fill opacity=0.10] (0,0) -- (58.10:1.06) arc (58.10:173.22:1.06) -- cycle;
    \draw [shift={(10.34,0.90)},thick,color=gradeColor,fill=gradeColor,fill opacity=0.10] (0,0) -- (-121.90:1.06) arc (-121.90:-6.78:1.06) -- cycle;
    \draw [shift={(17.90,-0.95)},thick,color=gradeColor,fill=gradeColor,fill opacity=0.10] (0,0) -- (149.79:1.06) arc (149.79:240.46:1.06) -- cycle;
    \draw [shift={(16.45,-3.50)},thick,color=gradeColor,fill=gradeColor,fill opacity=0.10] (0,0) -- (-29.07:1.06) arc (-29.07:60.46:1.06) -- cycle;
    \draw [shift={(4.38,-5.55)},thick,color=gradeColor,fill=gradeColor,fill opacity=0.10] (0,0) -- (11.92:1.06) arc (11.92:80.48:1.06) -- cycle;
    \draw [shift={(4.93,-2.30)},thick,color=gradeColor,fill=gradeColor,fill opacity=0.10] (0,0) -- (140.82:1.06) arc (140.82:260.48:1.06) -- cycle;
    \draw [shift={(12.47,-9.50)},thick,color=gradeColor,fill=gradeColor,fill opacity=0.10] (0,0) -- (23.28:1.06) arc (23.28:160.72:1.06) -- cycle;
    \draw [shift={(15.60,-10.60)},thick,color=gradeColor,fill=gradeColor,fill opacity=0.10] (0,0) -- (-107.08:1.06) arc (-107.08:-19.28:1.06) -- cycle;
    \draw [shift={(-4.32,-1.70)},thick,color=gradeColor,fill=gradeColor,fill opacity=0.10] (0,0) -- (0.82:1.06) arc (0.82:111.40:1.06) -- cycle;
    \draw [shift={(-4.32,-1.70)},thick,color=gradeColor,fill=gradeColor,fill opacity=0.10] (0,0) -- (-179.18:1.06) arc (-179.18:-68.60:1.06) -- cycle;
    \draw [shift={(8.20,6.30)},thick,color=gradeColor,fill=gradeColor,fill opacity=0.10] (0,0) -- (51.41:1.06) arc (51.41:167.24:1.06) -- cycle;
    \draw [shift={(11.74,5.50)},thick,color=gradeColor,fill=gradeColor,fill opacity=0.10] (0,0) -- (27.08:1.06) arc (27.08:167.24:1.06) -- cycle;
    \draw [shift={(6.72,-8.78)},thick,color=gradeColor,fill=gradeColor,fill opacity=0.10] (0,0) -- (-156.16:1.06) arc (-156.16:-91.33:1.06) -- cycle;
    \draw [shift={(6.64,-12.11)},thick,color=gradeColor,fill=gradeColor,fill opacity=0.10] (0,0) -- (88.67:1.06) arc (88.67:182.92:1.06) -- cycle;
    \draw [shift={(-3.51,-8.93)},thick,color=gradeColor,fill=gradeColor,fill opacity=0.10] (0,0) -- (48.98:1.06) arc (48.98:133.40:1.06) -- cycle;
    \draw [shift={(-2.43,-10.08)},thick,color=gradeColor,fill=gradeColor,fill opacity=0.10] (0,0) -- (-46.60:1.06) arc (-46.60:0.28:1.06) -- cycle;
    \draw [thick] (1.19,0.74)-- (9.24,-5.81);
    \draw [thick] (0.45,-6.38)-- (12.03,-3.93);
    \draw [thick] (5.66,2.08)-- (4.03,-7.65);
    \draw [thick] (15.65,0.36)-- (19.47,-1.87);
    \draw [thick] (14.67,-2.51)-- (18.31,-4.53);
    \draw [thick] (19.28,1.48)-- (15.79,-4.68);
    \draw [thick] (6.78,1.32)-- (13.29,0.55);
    \draw [thick] (11.87,3.35)-- (8.84,-1.52);
    \draw [thick] (-3.01,6.57)-- (-1.43,0.48);
    \draw [thick] (-4.75,3.15)-- (3.33,5.96);
    \draw [thick] (-3.43,1.74)-- (3.43,4.18);
    \draw [thick] (10.36,-8.76)-- (17.54,-11.28);
    \draw [thick] (18.31,-6.99)-- (9.45,-10.81);
    \draw [thick] (-5.29,0.78)-- (-2.92,-5.28);
    \draw [thick] (-6.71,-1.74)-- (-1.78,-1.66);
    \draw [thick] (-5.61,-4.92)-- (-1.11,-3.47);
    \draw [thick] (14.80,-13.21)-- (17.06,-5.86);
    \draw [thick] (5.34,6.95)-- (14.73,4.82);
    \draw [thick] (9.73,8.22)-- (7.07,4.89);
    \draw [thick] (9.66,4.43)-- (15.90,7.62);
    \node at (0.17, 4.47) {\cir[gradeColor]{1}};
    \node at (8.95, 5.63) {\cir[gradeColor]{2}};
    \node at (11.96, 2.13) {\cir[gradeColor]{3}};
    \node at (15.47, -1.17) {\cir[gradeColor]{4}};
    \node at (-3.13, -2.76) {\cir[gradeColor]{5}};
    \node at (5.37, -3.79) {\cir[gradeColor]{6}};
    \node at (14.48, -9.10) {\cir[gradeColor]{7}};
    \draw [thick] (2.93,-10.45)-- (9.34,-7.62);
    \draw [thick] (3.74,-12.26)-- (8.60,-12.01);
    \draw [thick] (6.76,-7.09)-- (6.61,-13.18);
    \node at (5.73, -10.17) {\cir[gradeColor]{8}};
    \draw [thick] (-5.26,-7.09)-- (-0.93,-11.66);
    \draw [thick] (-5.86,-11.62)-- (-1.36,-6.45);
    \draw [thick] (-6.32,-10.10)-- (0.84,-10.06);
    \node at (-2.32, -8.43) {\cir[gradeColor]{9}};
}
    }
}

\slide{exo}{\bsmall
    \begin{enumerate}\setItemColor{act}
        \item Dans les figures \cir[gradeColor]{4} et \cir[gradeColor]{6}, les angles représentés sont dits {\key{alternes-internes}}.
        Ce n'est pas le cas pour les autres figures. 
        À partir de ces observations :
        \begin{itemize}
            \item Dessine à main levée deux couples d'angles alternes-internes et deux couples d'angles qui ne sont pas alternes-internes.
            \item Propose une définition pour expliquer dans quelles conditions deux angles sont alternes-internes.
        \end{itemize}
        \item De même, seuls les angles de la figure \cir[gradeColor]{2} sont dits {\key{correspondants}}. 
        \begin{itemize}
            \item Dessine à main levée deux couples d'angles correspondants et deux couples d'angles qui ne sont pas correspondants.
            \item Propose une définition pour expliquer dans quelles conditions deux angles sont correspondants.
        \end{itemize}
    \end{enumerate}
}

\slide{cr}{\bsmall
    \sseq\ssec
    \df{}{
        Deux angles formés par deux droites coupées par une sécante sont dits \key{alternes-internes} si :
        \begin{itemize}
            \item ils sont situés de part et d'autre de la sécante (\key{alternes});
            \item ils sont situés entre les deux droites (\key{internes});
            \item ils ne sont pas sur le même sommet.
        \end{itemize}
    }[\wiki{Angles_alternes-internes}]

    \expl{}{}
}

\slide{cr}{
    \df{}{
        Deux angles formés par deux droites coupées par une sécante sont dits \key{correspondants} si :
        \begin{itemize}
            \item ils sont du même côté de la sécante;
            \item l'un est situé entre les deux droites et l'autre hors des deux droites;
            \item ils ne sont pas sur le même sommet.
        \end{itemize}
    }[\wiki{Angles_alternes-internes}]

    \expl{}{}
}

\bsec{Propriétés de parallélisme}

\slide{exo}{\bvspace{-0.5cm}
    \act{Eléments d'Euclide : proposition 27 et 28}{
        \begin{enumerate}
            \item \imgp{thm-18-pr-27}[10cm]
            \begin{enumerate}
                \item Réaliser un schéma de la situation qui illustre la proposition.
                \item Colorier de la même couleur les angles « oppofez alternatiuement » qui seraient égaux.
                \item Comment nommes-t-on ces angles aujourd'hui?
            \end{enumerate}\saveenumi
        \end{enumerate}
    }[\eucl[1632][58] et \href{https://mathix.org/linux/archives/19990}{Mathix}]
}

\slide{exo}{
    \begin{enumerate}\loadenumi[act]
        \item \imgp{thm-19-pr-28-alt}[10cm]
        \begin{enumerate}
            \item Réaliser un schéma de la situation qui illustre la proposition.
            \item Colorier de la même couleur « l'angle extérieur et son opposé intérieur du même côté » qui seraient égaux.
            \item Comment nommes-t-on ces angles aujourd'hui?
        \end{enumerate}
    \end{enumerate}
}
% % VARIABLES %%%
\setSeq{4}{Théorème de Pythagore - Contraposée et réciproque}
\setGrade{4e}
\def\imgPath{enseignement/4e/theoreme-de-pythagore/contraposee-et-reciproque/}

\forPrint
% \setboolean{answer}{true}

\def\ym{https://www.maths-et-tiques.fr/telech/19Pyth2.pdf}
%%

\obj{
    \item Comprendre les notions de réciproque et de contraposée.
    \item Utiliser la contraposée du Théorème de Pythagore pour montrer qu'un triangle n'est pas rectangle.
    \item Utiliser la réciproque du Théorème de Pythagore pour montrer qu'un triangle est rectangle.
    \item Déterminer si un triangle est rectangle ou non.
}

\obj{
    \item Reconnaitre sur un graphique une situation de proportionnalité ou de non proportionnalité.
    \item Calcule d'une quatrième proportionnelle.
    \item Utiliser une formule liant deux grandeurs dans une situation de proportionnalité.
    \item Résoudre des problèmes en utilisant la proportionnalité dans le cadre de la géométrie.
}[Flash]

\scn{Découvrir des notions de logiques ; la réciproque}

\slide{qf}{\bvspace{-0.35cm}
    \begin{enumerate}
        \item Les tableaux suivants représentent-ils des situations de proportionnalité ?
        Utilisez une calculatrice pour vérifier vos hypothèses.
        \multiColEnumerate{1}{
            \item \Propor[Simple]{1/2,6/12,3/5,10/20}
            \item \Propor[Simple]{2/3,6/9,30/45}
        }
        \item Essayez ensuite de justifier vos réponses sans calculatrice en expliquant votre raisonnement.
    \end{enumerate}
}

\bsec{Logique}
\bsubsec{Réciproque}

\slide{exo}{\bshrink
    \act{}{
        \begin{enumerate}
            \item « \Sialors{c'est un triangle}{c'est un polygones à trois sommets} » est une \key{proposition}.  
            Est-elle vraie ? Justifie ta réponse.  
            
            \item « \Sialors{c'est un triangle}{c'est un polygones à quatre sommets} » est une autre proposition.  
            Est-elle vraie ? Justifie ta réponse.  
            
            \item La première proposition est composée de deux parties :  
            \multiColItemize{1}{
                \item l'\key{antécédent} : « c'est un triangle »,
                \item le \key{conséquent} : « c'est un polygones à trois sommets ».
            }  
            On appelle \key{réciproque} d'une proposition la phrase qu'on obtient en inversant l'antécédent et le conséquent.  
            Écris la réciproque de la première proposition.
            Est-elle vraie ? \saveenumi
        \end{enumerate}
    }[\href{http://www.mathsaharry.com/aw/52.pdf}{Math à Harry}]
}

\slide{exo}{
    \begin{enumerate} \loadenumi[act]
        \item Les propositions suivantes sont-elles vraies ?
        Écris leurs réciproques et détermine si elles sont vraies.
        \multiColEnumerate{1}{
            \item \Sialors{c'est un carré}{c'est un rectangle avec tous ses côtés égaux}
            \item \Sialors{il peut pondre des œufs}{c'est un oiseau}
            \item \Sialors{c'est un rectangle}{c'est un quadrilatère dont tous les opposés sont parallèles} 
            \item \Sialors{c'est un rectangle}{c'est un carré}
            \item \Sialors{$AB = BC$}{$B$ est le milieu de $[AC]$}
        } 
    \end{enumerate}
}

\slide{cr}{\bsmall
    \ssec
    \ssubsec

    \df{}{
        On appelle \key{réciproque}
        d'une proposition :
        {« \Sialors{$A$}{$B$} »}
        ; la proposition :
        {« \Sialors{$B$}{$A$} »}.
    }
    
    \rmk{}{
        Une proposition peut être vraie sans que sa réciproque le soit, et inversement.
    }

    \expl{}{
        La proposition « \Sialors{$[AB]$ et $[CD]$ ne se coupent pas}{$[AB]$ et $[CD]$ sont parallèles}»
        est \awsr{fausse}.\\
        Sa réciproque: \awsr{« \Sialors{$[AB]$ et $[CD]$ sont parallèle}{$[AB]$ et $[CD]$ ne se coupent pas}»}
        est \awsr{vrai}.
    }
}

\scn{Découvrir des notions de logiques ; la contraposée}

\slide{qf}{\calculator \\ Completer les tableaux suivants :
    \multiColEnumerate{3}{
        \item \begin{center}
            \Propor[Simple,
            Math,
            Stretch=1.25,%
            ]{6/5,\awsr{\np{2.4}}/2}
        \end{center}
        \item \begin{center}
            \Propor[Simple,
            Math,
            Stretch=1.25,%
            ]{\np{237.6}/\awsr{66},\np{46.8}/13}
        \end{center}
        \item \begin{center}
            \Propor[Simple,
            Math,
            Stretch=1.25,%
            ]{\awsr{12}/18,-3/-4.5}
        \end{center}
    }
}

\bsubsec{Contraposée}

\slide{exo}{
    \ssubsec
    \act{}{
        On appelle \key{contraposée} d'une proposition
        {« \Sialors{$A$}{$B$} »}
        la proposition obtenue en écrivant :  
        {« \Sialors{non $B$}{non $A$} »}.
        \begin{enumerate}
            \item La contraposée de la proposition : «\Sialors{c'est un triangle}{il a trois côtés}».
            est donc «\Sialors{il n'a pas trois cotés}{ce n'est pas un triangle}» Est-elle vraie ? \saveenumi
        \end{enumerate}  
    }
}

\slide{exo}{
    \begin{enumerate} \loadenumi[act]
        \item Les propositions suivantes sont-elles vraies ? Écris leurs contraposées et dis si elles sont vraies :  
        \multiColEnumerate{1}{ 
            \item \Sialors{$x=7$}{$x$ est un nombre premier}
            \item \Sialors{c'est un nombre positif}{il est strictement inférieur à zéro}
            \item \Sialors{il est à Issy-les-Moulineaux}{il n'est pas en Espagne}  
            \item \Sialors{$AB=BC$}{$B$ est le milieu de $[AC]$} 
        }
        \item Que remarques-tu ?
    \end{enumerate}
}

\slide{cr}{
    \df{}{
        On appelle \key{contraposée}
        d'une proposition ;
        {« \Sialors{$A$}{$B$} »} ;
        la proposition : 
        {« \Sialors{non $B$}{non $A$} »}.  
    }

    \rmk{}{
        Une proposition et sa contraposée sont toujours soit toutes les deux vraies,
        soit toutes les deux fausses.
    }

    \expl{}{
        La proposition « \Sialors{c'est un triangle est équilatéral}{ses trois côtés sont égaux} »  
        est \awsr{vraie}.  
        Sa contraposée: \awsr{« \Sialors{ses trois côtés ne sont pas égaux}{ce n'est pas un triangle équilatéral} »},
        est \awsr{également vraie}.
    }
}

\scn{Appliquer ses connaissances sur la contraposée au Théorème de Pythagore}

\slide{qf}{
    \calculator
    \exo{}{
        Sachant que huit briques de masse identique pèsent \Masse{13.6},
        calcule la masse de six de ces briques.
    }[\afa{4e}[6]]
}

\bsec{Contraposée du Théorème de Pythagore}

\slide{exo}{\bshrink
    \act{}{\bvspace{-1cm}
        \def\crossSize{0.15}
        \ctikz[1]{
            \draw[gray!40] (-1,-4) rectangle (11,3);
            \draw [penciline,thick] (8.2,1.76)-- (5.28,-1.84);
            \draw [penciline,thick] (5.28,-1.84)-- (9.5,-0.3);
            \draw [penciline,thick] (9.5,-0.3)-- (8.2,1.76);
            \draw [penciline,thick] (2.76,-2.24)-- (3.84,-0.56);
            \draw [penciline,thick] (3.84,-0.56)-- (0.34,1.22);
            \draw [penciline,thick] (0.34,1.22)-- (2.76,-2.24);
            \drawPoint{D}{3.84}{-0.56}
            \drawPoint{E}{0.34}{1.22}
            \drawPoint{F}{2.76}{-2.24}
            \drawPoint{G}{8.20}{1.76}
            \drawPoint{H}{9.50}{-0.30}
            \drawPoint{I}{5.28}{-1.84}
            \draw (3.48,-1.38) node[anchor=north west] {4cm};
            \draw (0.92,-0.74) node[anchor=north west] {5cm};
            \draw (2.06,0.92) node[anchor=north west] {6cm};
            \draw (5.72,0.4) node[anchor=north west] {5cm};
            \draw (7.62,-1.22) node[anchor=north west] {12cm};
            \draw (9,1.3) node[anchor=north west] {13cm};
        }
    }
}

\slide{exo}{
    Pour les triangles $EDF$ et $GHI$, répondre aux questions suivantes :
    \begin{enumerate}\setItemColor{act}
        \item Ce triangle respect-ils l'égalité de Pythagore ?
        \item En utilisant vos connaissances sur le théorème de Pythagore,
        pouvez-vous conclure si le triangle est rectangle ou non ?
        Justifiez votre réponse en précisant l'outil de logique utilisé.
    \end{enumerate}
}

\slide{cr}{
    \ssec
    \bvspace{-0.5cm}
    \ctr{du théorème de Pythagore}{
        Dans un triangle $ABC$.
        \Sialors{$AB^2 \neq AC^2 + BC^2$}{le triangle $ABC$ n'est pas rectangle en $C$}
    }
    \bvspace{-0.75cm}
    \expl{}{
        Soit $NEZ$ est un triangle tel que : $NE = \Lg{8}, EZ = \Lg{16}\et ZN = \Lg{14}$.\\
        Démontrer que le triangle n'est pas rectangle.\\
        \awsr[5]{
            \begin{itemize}
                \item $NE$ est le plus long coté, il sagirait donc de l'hypothénus si le triangle était rectangle.
                \item D'une part :$EZ^2 = 16^2 = 256$
                \item D'autre part : $NE^2 + ZN^2 = 8^2 + 14^2 = 64 + 196= 260$
                \item Alors : $EZ^2 \neq NE^2 + ZN^2$
                \item D'après la contraposée du théorème de Pythagore le triangle $NEZ$ n'est pas rectangle.
            \end{itemize}
        }
    }
}

\bookSlide{29p431}[12cm]

\bsec{Réciproque du théorème de Pythagore}

\slide{exo}{
    \act{}{
        On va démontrer pour un exemple que la réciproque du théorème de Pythagore est vraie.
        Soit $ABC$ un triangle tel que : $AB = \Lg{5}; AC = \Lg{4}; BC = \Lg{5}$.
        \begin{enumerate}
            \item Tracer le triangle $ABC$.
            \item Verifier si $AB^2 = AC^2 + BC^2$. Est-ce que le triangle $ABC$ peut être rectangle?
            \item Contruire une droite perpendiculaire à la droite $(BC)$.
            \item Placer un point $D$ sur cette perpendiculaire tel que $DC = AC$
            et $D$ soit placé à «l'opposé» de $A$. \saveenumi
        \end{enumerate}
    }[\href{https://clg-monnet-briis.ac-versailles.fr/La-reciproque-du-theoreme-de-Pythagore}{Collège Jean Monnet}]
}

\slide{exo}{
    \begin{enumerate} \loadenumi[act]
        \item Quelle est la nature du triangle $BCD$?
        \item Calculer la longueur $BD$.
        \item Comparer les triangles $ABC$ et $BCD$.
        \item Conclure sur la nature du triangle $ABC$.
    \end{enumerate}
    De manière similaire on pourait prouver que la réciproque du théorème de Pythagore est toujours vraie.
}

\slide{cr}{
    \ssec

    \bvspace{-0.5cm}

    \rcp{du théorème de Pythagore}{
        Dans un triangle ABC.
        \Sialors{$AB^2 = AC^2 + BC^2$}
        {le triangle $ABC$ est rectangle en $C$}
    }

    \bvspace{-0.75cm}

    \expl{}{
        Soit $CGT$ est un triangle tel que : $CG = \Lg{45}, GT = \Lg{28}\et TC = \Lg{53}$.\\
        Démontrer que le triangle $CGT$ est pas rectangle.\\
        \awsr[6]{
            \begin{itemize}
                \item $TC$ est le plus long coté, il sagirait donc de l'hypothénus si le triangle était rectangle.
                \item D'une part :$TC^2 = 53^2 = 256$
                \item D'autre part : $NE^2 + ZN^2 = 8^2 + 14^2 = 64 + 196= 260$
                \item Alors : $EZ^2 \neq NE^2 + ZN^2$
                \item D'après la réciproque du théorème de Pythagore le triangle $CGT$ est rectangle en $G$.
            \end{itemize}
        }
    }
}

\bookSlide{27p431,26p431,36p432}[7cm][2]

\bookSlide{35p432,52p435}[6cm][2]

% \setSeq{6}{Géométrie dans l'espace - Solides}
\setGrade{6e}
\def\imgPath{enseignement/6e/geometrie-dans-l-espace/solides/}

\obj{
    \item Reconnaître des solides (pavé droit, cube, cône et cylindre).
    \item Identifier les caractéristiques de différents solides :
    sommets, faces et arêtes.
    \item Représenter un cube et un pavé droit.
}


\slide{}{}
% % VARIABLES %%%
\setSeq{5}{Nombres - Decimaux}
\setGrade{6e}

\def\imgPath{enseignement/6e/nombres/decimaux/}

\def\ym{\href{https://www.maths-et-tiques.fr/telech/19Nombres1.pdf}{Yvan Monka}}

% https://www.maths-et-tiques.fr/telech/19Nombres2.pdf

\obj{
    \item Utiliser une fraction et en donner progressivement le statut de nombre.
    \item Utiliser et représenter les nombres décimaux jusqu'à trois décimales.
    \item Ajouter, soustraire et multiplier des nombres décimaux.
    \item Résoudre des problèmes relevant des structures additives et multiplicatives en mobilisant une ou plusieurs étapes de raisonnement.
}

\scn{Découvrir les fractions décimales}

\bsubsec{Nombres décimaux}

\slide{exo}{
    \act{}{
        \multiColEnumerate{1}{
            \item $\pow{10}{2} = \awsr{\np{\powTenPositive{2}}}$
            \item $\pow{10}{3} = \awsr{\np{\powTenPositive{3}}}$
            \item $\pow{10}{6} = \awsr{\np{\powTenPositive{6}}}$
            \item $\powBrace{10}{15} = \awsr{\np{\powTenPositive{15}}}$
            \item $\powBrace{10}{100} = \awsr{\powTenBrace{100}}$
        }
    }
}

\slide{cr}{
    \ssubsec

    \vc{}{
        Une \key{puissance de 10} est le résultat d'un produit dont tous les facteurs sont $10$.
    }

    \expl{}{
        \Tableau[%
            DoubleEntree,
            Stretch=1.5,
            Couleur=gradeColor!15,
            LegendesH={$100$,$2$,$\np{1001}$,$\np{100000}$,$\np{200}$,$\np{10}$},
            LegendesV={Puissance de 10 ?},
            Largeur=2cm
        ]{\awsr{Oui},\awsr{Non},\awsr{Non},\awsr{Oui},\awsr{Non},\awsr{Oui}}
    }
}

\slide{cr}{
    \pr{}{
        Pour $n$ un nombre entier, on a :
        \begin{align*}
            \powBrace{10}{n} = \awsr{\powTenBrace{n}}
        \end{align*}
    }

    \expl{}{
        \multiColEnumerate{2}{
            \item $\pow{10}{3} = \powTenPositive{3}$
            \item $\pow{10}{6} = \powTenPositive{6}$
        }
    }
}

\slide{cr}{
    \df{}{
        On appelle \key{fraction décimale} une fraction dont le \key{dénominateur est une puissance de $10$}.
    }

    \expl{}{
        \Tableau[%
            DoubleEntree,
            Stretch=1.5,
            Couleur=gradeColor!15,
            LegendesH={$\frac{1}{10}$,$\frac{1}{2}$,$\frac{5}{10}$,$\frac{1}{\np{1000}}$,$\frac{1}{\np{30000}}$,$\frac{546985}{\np{10000000}}$},
            LegendesV={Fraction décimale ?},
            Largeur=2cm
        ]{\awsr{Oui},\awsr{Oui},\awsr{Non},\awsr{Oui},\awsr{Non},\awsr{Oui}}
    }
}

\slide{exo}{
    \act{}{
        Ecrire les nombres suivants sous forme de fractions décimales.
        \multiColEnumerate{2}{
            \item $\np{3.2} = \awsr{\frac{32}{10}}$
            \item $\np{10.2} = \awsr{\frac{102}{10}}$
            \item $\np{0.03} = \awsr{\frac{3}{100}}$
            \item $\np{0.0001} = \awsr{\frac{1}{1000}}$
            \item $6 \div 10 \div 10 = \awsr{\frac{6}{100}}$
            \item $32 \div 10 \div 10 \div 10 \div 10 = \awsr{\frac{32}{10000}}$
        }
    }
}

\slide{cr}{
    \df{}{
        On appelle \key{nombre décimal}, un nombre pouvant s'écrire sous forme de fraction décimale.
    }

    \expl{}{
        \Tableau[%
            DoubleEntree,
            Stretch=1.5,
            Couleur=gradeColor!15,
            LegendesH={$\np{0.6}$,$\np{13.2}$,$\frac{1}{10}$,$\frac{1}{2}$,$60$,$\frac{1}{3}$,$\frac{30}{3}$,$\pi$,$0$},
            LegendesV={Nombre décimal ?},
            Largeur=2cm
        ]{\awsr{Oui},\awsr{Oui},\awsr{Oui},\awsr{Oui},\awsr{Oui},\awsr{Non},\awsr{Oui},\awsr{Non},\awsr{Oui}}
    }
}

\slide{cr}{
    \pr{}{Pour $n$ un nombre entier, on a :
    \begin{align*}
        1 \repeatBrace{\div 10}{n}[quotients]
        = \frac{1}{\powTenBrace{n}}
        = \underbrace{0, 0 ... 0}_{n \textrm{ zéros}} 1
    \end{align*}
    }

    \expl{}{
        \multiColEnumerate{1}{
            \item $1 \div 10 \div 10 = \frac{1}{\awsr{100}} = \awsr{\np{0.01}}$
            \item $\awsr{1 \div 10 \div 10 \div 10 \div 10} = \frac{1}{10000} = \awsr{\np{0.0001}}$
            \item $1 \awsr{\div 10 \div 10 \div 10} = \frac{1}{\awsr{1000}} = \awsr{\np{0.001}}$
        }
    }
}

\scn{Décomposer un nombre décimal en fractions décimales}

% % VARIABLES %%%
\setSeq{5}{Proportionnalité - Tableaux et graphiques}
\setGrade{4e}
\def\imgPath{enseignement/4e/theoreme-de-pythagore/contraposé-et-reciproque/}
% \setboolean{answer}{true}
\def\ym{\href{https://www.maths-et-tiques.fr/telech/19Proport1.pdf}{Yvan Monka}}
% \forStudent
% \setboolean{demonstration}{false}
%%

\def\cp{coefficient de proportionnalité}
% \obj{
%     \item Reconnaitre sur un graphique une situation de proportionnalité ou de non proportionnalité.
%     \item Calcule d'une quatrième proportionnelle.
%     \item Utiliser une formule liant deux grandeurs dans une situation de proportionnalité.
%     \item Résoudre des problèmes en utilisant la proportionnalité dans le cadre de la géométrie.
% }

\renewcommand{\arraystretch}{1.5}

\avspace{0.1cm}

\bsec{Grandeurs proportionnelles}

\df{Grandeurs proportionnelles}{%
    Deux grandeurs sont dites \key{proportionnelles}
    lorsque les valeurs de l'une sont obtenues en multipliant les valeurs de l'autre par un même nombre non nul,
    appelé \key{coefficient de proportionnalité}.
}

\expl{}{
    \begin{tabular}{|>{\bfseries}c|*{4}{c|}} % Colonne en gras pour la première colonne
        \hline
        \rowcolor{gray!15} 
        Grandeur 1 & coté & coté & rayon & tension \\ \hline
        Grandeur 2  & périmètre du carré & aire du carré & périmètre du cercle & intensité \\ \hline
        coefficient?  & \awsr{$4$} & \awsr{non} & \awsr{$2\pi$} & \awsr{$R$} \\ \hline
    \end{tabular}
    % \begin{itemize}
    %     \item La longueur du coté d'un carré et sont périmètre avec \cp{} : $4$ car $\mathcal{P} = 4 \times c$.
    %     \item 
    % \end{itemize}
}

\bsec{Tableau de proportionnalité}

% \pr{Coefficient de proportionnalité}{
%     \Sialors{on est dans un tableau de proportionnalité}
%     {on peu passer d'une ligne à l'autre en multipliant par un \key{\cp}}
% }

Pour un tableau de proportionnalité : \propTable{a}{c}{b}{d} avec $a,b,c,d$ des nombres.

\pr{}{
    On peu passer d'une ligne à l'autre en multipliant par un \cp.
}

\expl{}{\vspace{-0.75cm}
    \multiColEnumerate{2}{
        \item \Propor[Stretch=1.5, Simple]{1/2.5,2/5,5/12.5}
        \FlechesPD{1}{2}{$\times\awsr{2}$}
        \FlechesPG{2}{1}{$\div\awsr{2}$}
        \item \Propor[Stretch=1.5, Simple]{120/12,3/0.3}
        \FlechesPD{1}{2}{$\times\awsr{\frac{1}{10}}$}
        \FlechesPG{2}{1}{$\div\awsr{\frac{1}{10}}$}
    }
}

\pr{Egalité des produits en croix}{
    On a l'égalité: $a \times d = b \times c$.
}

\newpage

\expl{}{
    Utiliser l'égalité des produits en croix pour vérifier si on a bien proportionnalité.
    \vspace{-0.75cm}\multiColEnumerate{2}{
        \item \Propor[Stretch=1.5, Simple]{15/36,1.2/2}
        \item \Propor[Stretch=1.5, Simple]{6/1.5,3/0.5}
    }\vspace{-0.75cm}
    \awsr[5]{
        \begin{enumerate}
            \item $15 \times 2 = 30$ et $36 \times \np{1.2} = 30$
            alors on égalité des produits en croix : $15 \times 2 = 36 \times \np{1.2}$,
            il sagit donc d'un tableau de proportionnalité.
            \item $6 \times \np{0.5} = 3$ et $\np{1.5} \times 3 = 4.5$
            alors on n'a pas égalité des produits en croix : $6 \times \np{0.5} \neq \np{1.5} \times 3 = 4.5$,
            il ne sagit donc pas d'un tableau de proportionnalité.
        \end{enumerate}
    }
}

\cor{Egalité des quotients}{
    On a aussi : $\frac{a}{b} = \frac{c}{d}$
}

\demo{}{
    On a : $\frac{a}{b} = \frac{ a\times d}{b \times d} = \frac{ \awsr{b \times c} }{b \times d}$ d'après l'égalité des produits en croix.\\
    Or $\frac{ b \times c }{b \times d} = \awsr{\frac{c}{d}}$.
    Alors $\frac{a}{b} = \awsr{\frac{c}{d}}.$
}[\href{https://pedagogie.ac-toulouse.fr/mathematiques/system/files/2023-03/demonstration_produits_en_croix.pdf}{Académie de Toulouse}]

\expl{}{
    Utiliser l'égalité des quotients pour vérifier si on a bien proportionnalité.
    \vspace{-0.75cm}\multiColEnumerate{2}{
        \item \Propor[Stretch=1.5, Simple]{10/12,5/6,20/23}
        \item \Propor[Stretch=1.5, Simple]{2/3,4/6,6/9}
    }\vspace{-0.75cm}
    \awsr[5]{
        \begin{enumerate}
            \item $\frac{10}{12} = \frac{20}{24} \neq \frac{20}{23}$
            alors on n'a pas égalité des quotients,
            il ne sagit donc pas d'un tableau de proportionnalité.
            \item $\frac{2}{3} = \frac{4}{6} = \frac{6}{9}$
            alors on égalité des quotients,
            il sagit donc d'un tableau de proportionnalité.
        \end{enumerate}
    }
}

% \ctr{}{
%     \Sialors{$\frac{a}{b} \neq \frac{c}{d}$}{l'égalité des quotients n'est pas respécté et on a pas proportionnalité}
% }

\mthd{Calcul 4e proportionnelle}{
    Si l'on connait 3 valeurs par exemple $b,c,d$.\\
    On peu calculer $a$ avec l'égalité $a = \awsr{\frac{b \times c}{d}}$.
}

\demo{}{
    En partant de l'égalité des produits en croix : $a \times d = b \times c$.\\
    Alors $a$ est le nombre qui multiplié par \awsr{$d$} done \awsr{$b \times c$}.\\
    D'après la définition du quotient : $a = \awsr{\frac{b \times c}{d}}$.
}

\expl{Compléter les tableaux de proportionnalité suivants}{
    \def\cW{2.5cm}
    \multiColEnumerate{2}{
        \item \begin{tabular}{|C{\cW}|C{1cm}|}
            \hline
            \np{9.6} & 3 \\ \hline
            \awsr{$\frac{\np{9.6}\times2}{3} = \np{6.4}$} & 2 \\ \hline
        \end{tabular}
        \item \begin{tabular}{|C{1cm}|*{2}{C{\cW}|}}
            \hline
            3 & 5 & \awsr{$\frac{\np{21.7}\times5}{7} = \np{15.5}$} \\ \hline
            \np{4.2} & \awsr{$\frac{\np{4.2}\times6}{3} = 7$} & \np{21.7} \\ \hline
        \end{tabular}
    }
}

\bsec{Représentation graphique}

\pr{Représentation graphique}{
    Sur un graphique, une situation de proportionnalité est représentée par des points alignés
    avec l'origine.
}[\ym]

\expl{}{
    Chaque graphique suivant représente-t-il une situation de proportionnalité ?
    \def\repere{%
        \tkzInit[xmin=0,xmax=8,ymin=0,ymax=8]
        \tkzGrid[sub,color=gradeColor!50!white,subxstep=1,subystep=1]        
        \tkzLabelX[step=2]
        \tkzLabelY[step=2]
        \tkzDrawY[step=1]
        \tkzDrawX[step=1]
    }
    \def\size{0.55}\def\crossWidth{0.25mm}
    \vspace{-0.75cm}
    \multiColItemize{3}{
        \item[]\ctikz[\size]{
            \repere
            \node at (4,9) {\cir[gradeColor]{1}};
            \drawPoint{}{2}{2.4}[Red]
            \drawPoint{}{4}{4.8}[Red]
            \drawPoint{}{6.5}{7.8}[Red]
            \ifthenelse{\boolean{answer}}{\draw[answer] (-1,-1.2) -- (7.5,9);}{}
        }
        \item[]\ctikz[\size]{
            \node at (4,9) {\cir[gradeColor]{2}};
            \repere
            \drawPoint{}{1}{1.3}[Red]
            \drawPoint{}{3}{3}[Red]
            \drawPoint{}{6}{7.8}[Red]
            \ifthenelse{\boolean{answer}}{\draw[answer] (-1,-1.3) -- (9/1.3,9);}{}
        }
        \item[]\ctikz[\size]{
            \repere
            \node at (4,9) {\cir[gradeColor]{3}};
            \drawPoint{}{1}{2}[Red]
            \drawPoint{}{4}{5}[Red]
            \drawPoint{}{6}{7}[Red]
            \drawPoint{}{7}{8}[Red]
            \ifthenelse{\boolean{answer}}{\draw[answer] (-1,0) -- (8,9);}{}
        }
    }
    \awsr[6]{Seuls les points du graphique \cir[gradeColor]{1} sont alignés avec l'origine.
    Ainsi, parmi les trois graphiques,
    c'est le seul qui représente une situation de proportionnalité.}
}



% \slide{qf}{
%     Les situations présentées dans ces tableaux sont-elles proportionnelles ?
%     \multiColEnumerate{2}{
%         \item \begin{center}
%             \Propor[Simple,
%             Math,
%             Stretch=1.25,%
%             ]{12/3,16/4,40/10}
%         \end{center}
%         \item \begin{center}
%             \Propor[Simple,
%             Math,
%             Stretch=1.25,%
%             ]{15/5,9/3,20/6}
%         \end{center}
%     }
% }

% \slide{qf}{\calculator \\ Completer les tableaux suivants :
%     \multiColEnumerate{3}{
%         \item \begin{center}
%             \Propor[Simple,
%             Math,
%             Stretch=1.25,%
%             ]{6/5,\awsr{\np{2.4}}/2}
%         \end{center}
%         \item \begin{center}
%             \Propor[Simple,
%             Math,
%             Stretch=1.25,%
%             ]{\np{237.6}/\awsr{66},\np{46.8}/13}
%         \end{center}
%         \item \begin{center}
%             \Propor[Simple,
%             Math,
%             Stretch=1.25,%
%             ]{\awsr{12}/18,-3/-4.5}
%         \end{center}
%     }
% }

% \slide{qf}{
%     \nullsubsec{}{
%         Sachant que huit briques de masse identique pèsent 13,6 kg, calcule la masse de six de ces
%         briques.
%     }[\afa{4e}[6]]
% }

% \slide{qf}{
%     \nullsubsec{}{
%         \begin{enumerate}
%             \item Sachant que la longueur $\mathcal{P}$ d'un cercle
%             est proportionnelle à son rayon $r$
%             avec un \cp $2\pi$.
%             Donnez la formule permettant de calculer $\mathcal{P}$ en fonction de $r$.
%             \item Sachant que la tension $U$ aux bornes d'une résistance
%             est proportionnelle à l'intensité $I$ du courant qui la traverse
%             avec un \cp égal à la valeur de la résistance $R$.
%             Donnez la formule permettant de calculer $U$ en fonction de $I$.
%         \end{enumerate}
%     }
% }


% %%% VARIABLES %%%
\setSeq{6}{Gestion de donnés - Moyenne et médiane}
\setGrade{5e}
\def\imgPath{enseignement/6e/geometrie-plane/polygones/}
\def\ym{\href{https://www.maths-et-tiques.fr/telech/19Proba-stat.pdf}{Yvan Monka}}
% \forPrint
% \def\caPrefix{6e-juin-2022-}
%%

\obj{
    \item Calculer et interpréter la moyenne d'une série de données.
    \item Interpréter la médiane d'une série de données.
    \item Recueillir et organiser des données, sous forme de tableaux, de graphiques.
    \item Traduire la relation de dépendance entre deux grandeurs par un tableau de valeurs.
    \item Produire une formule représentant la dépendance de deux grandeurs.
}

\scn{Découvrir la moyenne et la médiane}

% \slide{qf}{
%     Rappelle : Pour calculer une \key{moyenne}, on utilise les étapes suivantes :
%     \begin{enumerate}
%         \item Additionne toutes les valeurs données.
%         \item Divise cette somme par le nombre total de valeurs.
%     \end{enumerate}
%     \bvspace{-0.2cm}
%     Questions :
%     \begin{enumerate}
%         \item Quelle est la moyenne des nombres suivants : \( 4, \; 6, \; 8 \) ?
%         \item Quelle est la moyenne de \( 2, \; 3, \; 5, \; 10 \) ?
%     \end{enumerate}
% }

\def\iconPath{minecraft/}\def\iconSize{25pt}
\slide{qf}{
    \nullsubsec{}{
        Steve \icon{player-head} utilise une pioche \icon{diamond-pickaxe} Fortune III pour miner 6 minerais de charbon \icon{coal-ore}. 
        Chaque \icon{coal-ore} lui donne une quantité variable de charbon \icon{coal},
        correspondant aux valeurs suivantes : $4\,;1\,;3\,;2\,;3\,;4$.
        Détermine combien de morceaux de \icon{coal} est obtenue en moyenne par \icon{coal-ore}.        
    }[\href{https://fr.minecraft.wiki/w/Charbon}{Minecraft wiki}]
}

\slide{exo}{
    \act{}{
        Steve \icon{player-head} explore une grotte et trouve des coffres \icon{chest}
        contenant des diamants \icon{diamond}.  
        Les \icon{chest} suivants contiennent respectivement :  
        $ 8, \; 4, \; 6, \; 10, \; 2, \; 12, \; 5 \; \icon{diamond}. $
    
        \begin{enumerate}
            \item Combien de \icon{diamond} obtient-il en moyenne par \icon{chest}? \saveenumi
        \end{enumerate}
    
        \icon{player-head} veut estimer le nombre typique de \icon{diamond} qu'il peut trouver par \icon{chest} pour ses prochaines explorations.  
        Pour cela, il décide de calculer la \key{médiane}, c'est-à-dire le nombre qui partage cette série en deux groupes contenant autant de valeurs.
    
        \begin{enumerate} \loadenumi
            \item Classe les nombres de \icon{diamond} par \icon{chest} dans l'ordre croissant.
            \item Détermine la médiane et explique son interprétation dans ce contexte.
        \end{enumerate}
    }
}

\bsec{Définir la moyenne et la médiane}

\slide{cr}{
    \df{}{
        La \key{moyenne} d'une série de données est un nombre qui permet de représenter l'ensemble des valeurs de manière synthétique. Elle se calcule en ajoutant toutes les valeurs, puis en divisant par leur nombre total.
    }[\wiki{Moyenne}]
}

\slide{cr}{
    \df{}{
        La \key{médiane} d'une série de données ordonnées est une valeur qui partage cette série en deux groupes de même effectif :
        \begin{itemize}
            \item Si le nombre de données est impair, la médiane est la valeur centrale.
            \item Si le nombre de données est pair, la médiane est la moyenne des deux valeurs centrales.
        \end{itemize}
    }[\wiki{Médiane}]
}

\slide{exo}{\bsmall
    \begin{enumerate}\setItemColor{act}
        \item Trouve la moyenne et la médiane des séries suivantes :
        \begin{itemize}
            \item \( 7, \; 10, \; 12, \; 15, \; 8 \)
            \item \( 5, \; 6, \; 8, \; 9, \; 10, \; 12 \)
        \end{itemize}
        \item Explique pourquoi la moyenne et la médiane ne sont pas toujours égales.
        \item Propose une situation où la médiane serait plus utile que la moyenne pour représenter les données.
    \end{enumerate}
}
% % VARIABLES %%%
\setSeq{7}{Nombres rationnels}
\setGrade{4e}

\def\imgPath{enseignement/4e/nombres-rationnels/}

\dym{https://www.maths-et-tiques.fr/telech/19Fractions1.pdf}
\def\caPrefix{4e-mars-2022-}

\forStudents
% \forPrint

\obj{
    \item Definition d'un nombre rationnel.
    \item Calculer des sommes, difference, produit et quotient de fractions.
    \item Utiliser la notion d'inverse.
}

\scn{Rappel sur les definitions d'ensembles de nombres}

\slide{qf}{
    \exo{Fraction quotient}{
    \noCalculator\\
    Ecrires les nombres suivant sous forme décimale :
    \multiColEnumerate{2}{
        \item $\frac{5}{2} = \nswr{\np{2.5}}$
        \item $\frac{9 \times 6}{3} = \nswr{27}$
        \item $\frac{4 + 6}{6-11} = \nswr{-2}$
        \item $\frac{6 \times (-6)}{-100 \div 10} = \nswr{\np{3.6}}$
    }
}

}

\slide{exo}{
    \act{Nature des nombres}{
    En maternelle, nous avons appris à compter des objets en utilisant les nombres $1, 2, 3$, etc.
    Ces nombres sont les premiers utilisés « naturellement »,
    et on les appelle les nombres entiers naturels.
    Depuis l'école primaire et au collège, nous avons découvert d'autres nombres.
    Voici une liste de nombres :
    \multiColItemize{3}{
        \item $-\np{27.2}$
        \item $-\sqrt{4}$
        \item $\frac{10371}{100}$
        \item $\frac{27}{13}$
        \item $\frac{3}{2}$
        \item $\frac{-21}{15}$
        \item $\np{0.33333}\ldots$
        \item $\pi$
        \item $\frac{-10}{5}$
        \item $\sqrt{2}$
        \item $\frac{47}{21}$
        \item $-15 + 20$
        \item $\frac{-10}{3}$
        \item $37$
        \item $1 \div 7$
    }
    \begin{enumerate}
        \item Indique par une pastille :
        \begin{itemize}
            \item \textcolor{Blue}{bleue} les nombres entiers naturels.
            \item \textcolor{Red}{rouge} les nombres entiers relatifs.
            \item \textcolor{Green}{verte} les nombres décimaux.
        \end{itemize} 
        \hint{Certains nombres peuvent être indiqués plusieurs fois.}
        \item Quels sont les nombres restants ?
        Essaie de les classer dans deux catégories de nombres différentes.
    \end{enumerate}
}[\href{https://clg-monnet-briis.ac-versailles.fr/IMG/pdf/cours_fractions-3.pdf}{Collège Monnet}]
}

\scn{Définition des nombres rationnels}

\slide{qf}{
    \exo{Repartition de vols}{
    On a représenté sur le diagramme suivant les vols du mois de février d'une compagnie aérienne.
    
    \ctikz[0.45]{
    % \boundingBox[11.76][10.56][0.5pt][1][(-2.41,-2.86)]
    \draw [shift={(3.50,1.18)},thick,color=gradeColor,fill=gradeColor,fill opacity=0.10] (0,0) -- (89.85:1.13) arc (89.85:119.85:1.13) -- cycle;
    \draw [shift={(3.50,1.18)},thick,color=gradeColor,fill=gradeColor,fill opacity=0.10] (0,0) -- (119.85:1.13) arc (119.85:149.85:1.13) -- cycle;
    \draw [shift={(3.50,1.18)},thick,color=gradeColor,fill=gradeColor,fill opacity=0.10] (0,0) -- (149.85:1.13) arc (149.85:179.85:1.13) -- cycle;
    \draw[thick,color=gradeColor,fill=gradeColor,fill opacity=0.10] (2.70,1.18) -- (2.70,0.38) -- (3.50,0.38) -- (3.50,1.18) -- cycle;
    \draw [thick] (3.50,1.18) circle (7.48cm);
    \draw [shift={(3.50,1.18)},thick,color=gradeColor] (89.85:1.13) arc (89.85:119.85:1.13);
    \draw [shift={(3.50,1.18)},thick,color=gradeColor] (119.85:1.13) arc (119.85:149.85:1.13);
    \draw [shift={(3.50,1.18)},thick,color=gradeColor] (149.85:1.13) arc (149.85:179.85:1.13);
    \draw [shift={(3.50,1.18)},thick,color=gradeColor,fill=gradeColor,fill opacity=0.35]  (0,0) --  plot[domain=-1.57:1.57,variable=\t]({1*7.48*cos(\t r)+0*7.48*sin(\t r)},{0*7.48*cos(\t r)+1*7.48*sin(\t r)}) -- cycle ;
    \draw [shift={(3.50,1.18)},thick,color=gradeColor,fill=gradeColor,fill opacity=0.28]  (0,0) --  plot[domain=3.14:4.71,variable=\t]({1*7.48*cos(\t r)+0*7.48*sin(\t r)},{0*7.48*cos(\t r)+1*7.48*sin(\t r)}) -- cycle ;
    \draw [shift={(3.50,1.18)},thick,color=gradeColor,fill=gradeColor,fill opacity=0.21]  (0,0) --  plot[domain=2.62:3.14,variable=\t]({1*7.48*cos(\t r)+0*7.48*sin(\t r)},{0*7.48*cos(\t r)+1*7.48*sin(\t r)}) -- cycle ;
    \draw [shift={(3.50,1.18)},thick,color=gradeColor,fill=gradeColor,fill opacity=0.14]  (0,0) --  plot[domain=2.09:2.62,variable=\t]({1*7.48*cos(\t r)+0*7.48*sin(\t r)},{0*7.48*cos(\t r)+1*7.48*sin(\t r)}) -- cycle ;
    \draw [shift={(3.50,1.18)},thick,color=gradeColor,fill=gradeColor,fill opacity=0.07]  (0,0) --  plot[domain=1.57:2.09,variable=\t]({1*7.48*cos(\t r)+0*7.48*sin(\t r)},{0*7.48*cos(\t r)+1*7.48*sin(\t r)}) -- cycle ;
    \draw[color=gradeColor] (9.2,1.65) node {France};
    \draw[color=gradeColor] (-0.70,-2.86) node {Europe};
    \draw[color=gradeColor] (-1.6,2.2) node {Amérique};
    \draw[color=gradeColor] (-0.8,5.4) node {Afrique};
    \draw[color=gradeColor] (1.88,7.70) node {Asie};
    \draw [thick,gradeColor] (3.25,2.12) -- (3.17,2.42);
    \draw [thick,gradeColor] (2.81,1.87) -- (2.60,2.09);
    \draw [thick,gradeColor] (2.56,1.43) -- (2.26,1.51);
}
    Dans chaque cas, indiquer quelle fraction représentent les vols vers :
    \multiColItemize{3}{\item  la France \item l'Europe \item l'Asie}

    Au mois de février, cette compagnie a affrété 576 vols. Calculer le nombre de vols vers :
    \multiColItemize{3}{\item  la France \item l'Europe \item l'Asie}
}[\href{https://cache.media.education.gouv.fr/file/Fractions/22/7/RA16_C4_MATH_fractions_flash1_part_fractions_554227.pdf}
{Utiliser les nombres pour comparer, calculer et résoudre des problèmes :
Les fractions - Un exemple de question flash - « Vision-partage » de la fraction}]
}

\slide{cr}{
    \sseq

    Dans ce cours $a,b,c \et d$ designent des nombres.
    
    \section{Définition}
    \df{}{
    On appelle \key{nombre rationnel} est, en mathématiques,
    un nombre qui peut s'exprimer comme le quotient de deux entiers relatifs.
}[\wiki{Nombre_rationnel}]
    \expl{}{
    % \hspace{-1.6cm}%
    \Tableau[%
        DoubleEntree,
        Stretch=1.5,
        Couleur=gradeColor!15,
        LegendesH={$\np{0.6}$,$\num{13.2}$,$\frac{1}{10}$,$\frac{1}{2}$,$60$,$\frac{1}{3}$,$\frac{30}{3}$,$\pi$,$0$,$58\div100$},
        LegendesV={Nombre rationnel ?},
        Largeur=1cm
    ]{\nswr{Oui},\nswr{Oui},\nswr{Oui},\nswr{Oui},\nswr{Oui},\nswr{Non},\nswr{Oui},\nswr{Non},\nswr{Oui},\nswr{Oui}}
}
}

\scn{Sommes et différences de fractions}

\slide{qf}{
    \exo{D'accord ou pas d'accord ?}{
    \begin{align*}
        \dfrac{6+2}{6+4}
        = \dfrac{\cancel{6}+2}{\cancel{6}+4}
        = \dfrac{2}{4}
        = \dfrac{1}{2}
    \end{align*}
}[\href{https://cache.media.education.gouv.fr/file/Fractions/23/4/RA16_C4_MATH_fractions_flash4_travail_erreur_554234.pdf}
{Utiliser les nombres pour comparer, calculer et résoudre des problèmes :
les fractions - Un exemple de questions flash - Travail sur l'erreur}]
}

\slide{exo}{
    \act{Somme de fractions}{
    \begin{enumerate}
        \item Vrai ou Faux ?
        \multiColEnumerate{2}{
            \item $\frac{1}{2} + \frac{3}{2} = \frac{1+3}{2+2}$
            \item $\frac{1}{2} + \frac{3}{4} = \frac{1+3}{2+4}$
            \item $\frac{6}{2} + \frac{4}{2} = \frac{6+4}{2}$
            \item $\frac{1}{3} + \frac{1}{3} = \frac{1}{3+3}$
            \item $\frac{15}{5} + \frac{2}{1} = \frac{15+2\times5}{5}$
            \item $\frac{1}{5} + \frac{5}{5} = \frac{1+5}{5}$
        }
        \item Quelle est la méthode pour additionner deux nombres de même dénominateur ?
        \item Quelle est la méthode pour additionner deux nombres de dénominateur différents ? Par exemple $\frac{3}{4}$ et $\frac{5}{2}$.
    \end{enumerate}
}
}

\slide{cr}{
    % \section{Somme et différence}
    \subsection{De même dénominateur}
    \mthd{Somme ou différence de fractions avec le \key{même dénominateur}}{
    \begin{enumerate}
        \item On prend comme numérateur : la somme ou différence des numérateurs.
        \item On conserve le dénominateur commun.
    \end{enumerate}
}
    \expl{}{
    \multiColEnumerate{2}{
        \item $\frac{1}{2} + \frac{5}{2}
        = \nswr{\frac{1+5}{2} = \frac{6}{2}}$
        \item $\frac{14}{21} - \frac{12}{21}
        = \nswr{\frac{14-12}{21} = \frac{2}{21}}$
        \item $\frac{a}{d} + \frac{b}{d}
        = \nswr{\frac{a+b}{d}}$
        \item $\frac{a}{d} - \frac{b}{d}
        = \nswr{\frac{a-b}{d}}$
    }
}
}

\slide{cr}{
    \subsection{De dénominateur différents}
    \mthd{Somme et différence de fractions de \key{dénominateurs différents}}{
    \begin{enumerate}
        \item On met les fractions au même dénominateur.
        \item On fait leur somme ou différence comme précedement.
    \end{enumerate}
}
    \expl{}{
    \multiColEnumerate{1}{
        \item $\dfrac{1}{2} - \dfrac{1}{4}
        = \nswr{\dfrac{2}{4} - \dfrac{1}{4}
        = \dfrac{2-1}{4} = \dfrac{1}{4}
        }$
        \item $\dfrac{2}{12} + \dfrac{2}{3}
        = \nswr{\dfrac{2}{12} + \dfrac{2\times4}{3\times4}
        = \dfrac{2}{12} + \dfrac{8}{12}
        = \dfrac{2+8}{12}
        = \dfrac{10}{12}
        }$
        \item $\dfrac{3}{5} + \dfrac{6}{7}
        = \nswr{\dfrac{3\times7}{5\times7} + \dfrac{6\times5}{7\times5}
        = \dfrac{21}{35} + \dfrac{30}{35}
        = \dfrac{21+30}{35} = \dfrac{51}{35}
        }$
        \item $\dfrac{12}{3} - \dfrac{6}{2}
        = \nswr{\dfrac{12\times2}{3\times2} - \dfrac{6\times3}{2\times3}
        = \dfrac{24}{6} - \dfrac{18}{6}
        = \dfrac{24-18}{6} = \dfrac{6}{6}
        }$
        \item $\dfrac{a}{b} + \dfrac{c}{d}
        = \nswr{\dfrac{a\times d}{b\times d} + \dfrac{c \times b}{d\times b}
        = \dfrac{ad}{bd} + \dfrac{cb}{bd}
        = \dfrac{ad+cb}{bd}
        }$
    }
}
}

\scn{Exercices - Sommes et différences de fractions}

\slide{qf}{
    \exo{QCM sur les fractions 1}{
    \begin{enumerate}
        \item Dans un ruisseau, il s'écoule, en moyenne, $120 \Vol{}$ d'eau en 45 $\minute$.
        Le débit de ce ruisseau, en $\Vol{}/\hour$, est égal à :
        \multiColEnumerate{3}{
            \item $120 \div \frac{3}{4}$
            \item $120 \times \frac{3}{4}$
            \item $120 \times \np{0.75}$
        }
        \item La fraction $\frac{143}{132}$ est :
        \multiColEnumerate{3}{
            \item irréductible
            \item comprise entre $1$ et $\np{1.1}$
            \item décimale
        }
    \end{enumerate}
}
}

\bookSlide{21p121,24p121,27p121,28p121,57p124}[7.5cm][2]

\scn{Definition - Produits de fractions}

\caSlide{5-6-7}

\slide{cr}{
    \section{Produit et quotient}
    \subsection{Produit}
    \mthd{Produits de fractions}{
    \begin{enumerate}
        \item On prend comme numérateur : le produit des numérateurs.
        \item On prend comme dénominateur : le produit des dénominateurs.
    \end{enumerate}
}[][\cmdGeoGebra[azrs82xa]]
    \expl{}{
    \multiColEnumerate{2}{
        \item $\frac{2}{5} \times \frac{3}{4} = \nswr{\frac{2 \times 3}{5 \times 4} = \frac{6}{20} = \frac{3}{10}}$
        \item $\frac{5}{3} \times \frac{1}{3} = \nswr{\frac{5 \times 1}{3 \times 3} = \frac{5}{9}}$
        \item $\frac{7}{8} \times \frac{-2}{3} = \nswr{\frac{7 \times (-2)}{8 \times 3} = \frac{-14}{24} = \frac{-7}{12}}$
        \item $\frac{a}{c} \times \frac{b}{d} = \nswr{\frac{a \times b}{c \times d}}$
    }
}
}

\scn{Exercices - Produits de fractions}

\caSlide{8-9-10}

\bookSlide{4p132,5p132,7p132,48p135}[7.5cm][2]

\scn{Definition - Inverse et Quotients}

\slide{qf}{
    \exo{QCM sur les fractions 2}{
    Dans chacune des ci-dessous, une seule réponse est correcte. Laquelle ?
    \begin{enumerate}
        \item L'inverse de $\dfrac{2}{7}$ est :
        \multiColEnumerate{3}{
            \item supérieur à $7$
            \item égale à $\np{3.5}$
            \item inférieur à $2$
        }
        \item $\dfrac{1}{15}$ est égale à :
        \multiColEnumerate{3}{
            \item $\np{0.0666666667}$
            \item $\dfrac{2}{5} \div \dfrac{1}{6}$
            \item $\dfrac{2}{30}$
        }
    \end{enumerate}
}
% [\href{https://cache.media.education.gouv.fr/file/Fractions/23/2/RA16_C4_MATH_fractions_flash3_sens_quotient_554232.pdf}
% {Utiliser les nombres pour comparer, calculer et résoudre des problèmes : les fractions}]
}

\slide{cr}{
    \subsection{Inverse}
    \df{}{
    On appelle \key{inverse} d'un nombre $x$, le nombre qui, multiplié par $x$, donne $1$.
}[\wiki{Inverse}]
    \expl{}{
    \multiColEnumerate{1}{
        \item $0.5 \times \nswr{2} = 1$. L'inverse de $0.5$ est donc $\nswr{2}$.
        \item $10 \times \nswr{0.1} = 1$. L'inverse de $10$ est donc $\nswr{0.1}$.
        \item $\dfrac{1}{16} \times \nswr{16} = 1$. L'inverse de $\dfrac{1}{9}$ est donc $\nswr{9}$.
        \item $23 \times \nswr{\dfrac{1}{23}} = 1$. L'inverse de $23$ est donc $\nswr{\dfrac{1}{23}}$.
        \item $\dfrac{4}{5} \times \nswr{\dfrac{5}{4}} = 1$. L'inverse de $\dfrac{4}{5}$ est donc $\nswr{\dfrac{5}{4}}$.
    }
}
    \rmk{}{
    \begin{enumerate}
        \item L'inverse d'un nombre $x$ est $\nswr{\dfrac{1}{x}}$.
        \item L'inverse d'un nombre $\dfrac{a}{b}$ est $\nswr{\dfrac{b}{a}}$.
    \end{enumerate}
}
}

\slide{cr}{
    \subsection{Quotient}
    \mthd{Quotient de fractions}{
    \key{Diviser} un nombre par une fraction est équivalent à la \key{multiplier par l'inverse} de cette même fraction.
}
    \expl{}{
    \multiColEnumerate{1}{
        \item $\dfrac{2}{5} \div \dfrac{3}{4} = \nswr{\dfrac{2}{5} \times \dfrac{4}{3} = \dfrac{2 \times 4}{5 \times 3} = \dfrac{8}{15}}$
        \item $\dfrac{5}{3} \div \dfrac{1}{3} = \nswr{\dfrac{5}{3} \times \dfrac{3}{1} = \dfrac{5 \times 3}{3 \times 1} = 5}$
        \item $\dfrac{7}{8} \div \dfrac{-2}{3} = \nswr{\dfrac{7}{8} \times \dfrac{3}{-2} = \dfrac{7 \times 3}{8 \times (-2)} = \dfrac{21}{-16} = -\dfrac{21}{16}}$
        \item $\dfrac{a}{c} \div \dfrac{b}{d} = \nswr{\dfrac{a}{c} \times \dfrac{d}{b} = \dfrac{a \times d}{c \times b}}$
    }
}
}

\scn{Exercices - Inverse et Quotients}

\caSlide{11-12-13}

\bookSlide{22p133,28p133,37p134,42p134}[7.5cm][2]

\scn{Problèmes mettant en jeu des fractions}

\caSlide{14-15-16}

\slide{exo}{
    \exo{Le compte est bon}{
    Voici une liste de nombres : 
    $1$, $2$, $3$, $4$,
    $\dfrac{1}{3}$, $\dfrac{5}{4}$, $\dfrac{1}{4}$, $\dfrac{3}{6}$, $\dfrac{4}{7}$, $\dfrac{5}{8}$, $\dfrac{6}{9}$.
    \begin{enumerate}
        \item Pour obtenir le nombre $\dfrac{9}{8}$,
    vous pouvez effectuer toutes les opérations que vous souhaitez et décrire les résultats intermédiaires.
    Attention,
    chaque nombre ci-dessus est unique et ne peut être utilisé qu'une seule fois,
    en les convertissant si nécessaire en fractions égales.
    \\\bonus
        \item Trouvez le résultat en utilisant le moins d'opérations possible.
        \item Faites de même avec $\dfrac{6}{7}$, $\dfrac{4}{9}$, $\dfrac{9}{24}$, $\dfrac{9}{28}$.
    \end{enumerate}
}[\prbltq{le-compte-est-bon-avec-des-fractions}]

\nswr[0]{
    Pour obtenir $\dfrac{9}{8}$, on peut procéder comme suit :
    \begin{itemize}
        \item $\dfrac{5}{4} + \dfrac{1}{4} = \dfrac{6}{4} = \dfrac{3}{2}$
        \item $\dfrac{3}{2} \div \dfrac{1}{3} = \dfrac{3}{2} \times 3 = \dfrac{9}{2}$
        \item $\dfrac{9}{2} \times \dfrac{1}{4} = \dfrac{9}{8}$
    \end{itemize}
    On peut aussi faire :
    \begin{itemize}
        \item $\dfrac{5}{8} + \dfrac{3}{6} = \dfrac{5}{8} + \dfrac{1}{2} = \dfrac{5}{8} + \dfrac{4}{8} = \dfrac{9}{8}$
    \end{itemize}
}

}

\bookSlide{62p137,60p137}[7.5cm][2]

\scn{Tache complexe mettant en jeu des fractions}

\slide{qf}{
    \exo{Chocolat au lait - Olé}{
    \begin{enumerate}
        \item On prépare une boisson chocolatée en mélangeant du chocolat et du lait.
        \begin{itemize}
            \item La recette $A$ mélange $3$ doses de chocolat pour $2$ doses de lait.
            \item La recette $B$ mélange $2$ doses de chocolat pour $1$ dose de lait.
        \end{itemize}
        Quel est le Mélange qui a le plus le goût du chocolat ?
        \item On remplit deux récipients identiques,
        l'un avec le liquide A,
        l'autre avec le liquide B.
        \begin{itemize}
            \item $5$ litres du liquide $A$ pèsent $3$ kg.
            \item $7$ litres du liquide $B$ pèsent $4$ kg.
        \end{itemize}
        Lequel des deux récipients ainsi remplis est le plus lourd ?
    \end{enumerate}
}[\prbltq{chocolat-au-lait-ole}]
}

\bookSlide{61-1p137,61-2p137}[7.5cm][2]

\caSlide{17-18-19}

% %%%
\setGrade{4e}
\evaluation{2}
% [corr]
%%%
% \def\imgPath{enseignement/6e/}

\seqEvaluation{3}{Nombres Relatifs}{
    Déterminer le produit et le quotient de nombres relatifs.
    /2,
    Déterminer la somme et la différence de nombres relatifs.
    /2,
    Déterminer le signe d'un produit comportant plusieurs facteurs relatifs.
    /1,
    Trouver les antécédents du carré d'un nombre donné.
    /1%
}

\seqEvaluation{4}{Théorème de pythagore - Contraposé et réciproque}{
    Montrer qu'un triangle n'est pas rectangle.
    /2,
    Montrer qu'un triangle est rectangle.
    /2,
    Déterminer si un triangle est rectangle.
    /1,
    Maîtriser les notions de contraposée et de réciproque. /0%
    }

\seqEvaluation{5}{Proportionnalité - Tableaux et graphiques}{
    Reconnaître un tableau de proportionnalité.
    /2,
    Compléter un graphique cartésien.
    /1,
    Reconnaître une situation de proportionnalité sur un graphique.
    /1%
}

\evalutionEnd[3][2]

% \newpage

% \exo{Calcul de 4e proportionnelle avec des nombres relatifs \tt}{
%     Trouver les valeurs manquantes des tableaux de proportionnalité ci-dessous :
%     \multiColEnumerate{3}{
%         \item \Propor[Simple,Math,Stretch=1.25,%
%         ]{-18/30,-6/\awsr{\np{10}}}
%         % \item \Propor[Simple,Math,Stretch=1.25,%
%         % ]{\awsr{\np{3}}/22,\np{1.5}/11}
%         \item \Propor[Simple,Math,Stretch=1.25,%
%         ]{\np{2.5}/-5,-60/\awsr{\np{120}},\awsr{\np{-360}}/720}
%     }
% }
    
% \answerFill[Calculs][%
%     \begin{enumerate}
%         \item On peut utiliser l'égalité des produits en croix pour calculer la quatrième proportionnelle :
%         $\frac{22\times\np{1.5}}{11} = \frac{33}{11} = 3$
%         % \item $\frac{30\times(-6)}{-18} = \frac{-180}{-18} = 10$
%         \item On a $\np{2,5} \times (-2) = -5$ alors $-2$ est le coefficient qui permet de passer de la première à la deuxième ligne.
%         On calcule alors : $-60 \times (-2) = 120$ et $720 \div (-2) = -360$
%     \end{enumerate}
% ]

\exo{Manipulation de relatifs}{
    \multiColEnumerate{1}{
        \item $[-3-(-7+5)] \times (-\np{0.5}) = \awsr[4]{
            [-3-2] \times (-\np{1.5})
            = -5 \times (-\np{1.5})
            = \np{7.5}
        }$
        \item Quel est le résultat d'un produit de $\np{4815162342}$ fracteurs égales à $-1$.\\
        \awsr[4]{
            $\np{4815162342}$ est pair donc le résultat est positif
            et multiplié $1$ par lui même donne $1$ donc le résultat est $1$.
        }
        \item Donner deux nombres dont le carré vaut $18$.
        \awsr[4]{
            $\sqrt{18}^2 = 18$ et $(-\sqrt{18})^2 = 18$.
            $18$ et $-18$ sont deux nombres dont le carré vaut $18$.
        }
        % \item $\frac{2-[5-3\times(2-4)]}{2-15\div5} = \awsr[3]{
        %     \frac{2-[5-3\times(-2)]}{2-3}
        %     = \frac{2-[5-6]}{-1}
        %     = \frac{2-[-1]}{-1}
        %     = \frac{3}{-1}
        %     = -3
        % }$
    }
}[\ching{4}{nombres-relatifs-operations}[$26$a et E.$35$a]]

\newpage

\exo{Déterminer si un triangle est rectangle}{
    Déterminer la nature de chacun des triangles ci-dessous.
    \ctikz[0.6]{
        \draw[gray!40] (-6,-4) rectangle (8,5);
        \draw [penciline, thick] (-3.2,3.86)-- (-0.84,-0.1);
        \draw [penciline,thick] (-0.84,-0.1)-- (-4.5,-1.86);
        \draw [penciline,thick] (-4.5,-1.86)-- (-3.2,3.86);
        \draw [penciline,thick] (-0.28,-2.28)-- (2.28,1.94);
        \draw [penciline,thick] (6.14,-2.72)-- (2.28,1.94);
        \draw [penciline,thick] (-0.28,-2.28)-- (6.14,-2.72);
        \draw (-4.86,1.04) node[anchor=north west] {6cm};
        \draw (-2.8,-1.22) node[anchor=north west] {5cm};
        \draw (-1.8,2.04) node[anchor=north west] {3cm};
        \draw (0.26,0.32) node[anchor=north west] {6dm};
        \draw (2.26,-2.78) node[anchor=north west] {8dm};
        \draw (4.46,0.14) node[anchor=north west] {10dm};
        \drawPoint{A}{-3.20}{3.86}
        \drawPoint{B}{-4.50}{-1.86}
        \drawPoint{C}{-0.84}{-0.10}
        \drawPoint{D}{-0.28}{-2.28}
        \drawPoint{E}{2.28}{1.94}
        \drawPoint{F}{6.14}{-2.72}
    }
}[\ching{4}{reciproque-pythagore}[$8$]]


\answerSec{14}[Triangle ABC][
    \Pythagore[Reciproque,Unite=cm]{ABC}{3}{5}{6}
]

\answerFill[Triangle EDF][
    \Pythagore[Reciproque,Unite=dm]{EDF}{10}{8}{6}
]

\exo{}{
    Chez Zoro, des tee-shirts sont en vente. Les prix normaux ainsi que les prix en période de soldes sont indiqués dans le tableau ci-dessous.
    \begin{enumerate}
        \begin{table}[h!]
            \centering
            \renewcommand{\arraystretch}{1.5} % Ajuste la hauteur des lignes
            \setlength{\tabcolsep}{8pt} % Ajuste l'espacement des colonnes
            \begin{tabular}{|>{\bfseries}c|*{7}{c|}} % Colonne en gras pour la première colonne
                \hline
                \rowcolor{gray!15} 
                Tee-shirts vendus & 1 & 2 & 3 & 4 & 5 & 6 & 7 \\ \hline
                Prix normal (en \euro) & 5 & 10 & 15 & 20 & 25 & 30 & 35 \\ \hline
                Prix soldé (en \euro)  & 5 & 10 & 12 & 17 & 22 & 24 & 29 \\ \hline
            \end{tabular}
        \end{table}
        \item Complétez le graphique cartésien ci-dessous en plaçant les points correspondant :
        \begin{itemize}
            \item en \textcolor{Blue}{bleu}, les points représentant les prix normaux ;
            \item en \textcolor{Red}{rouge}, les points représentant les prix en période de soldes.
        \end{itemize}
        \vspace{-1cm}
        \begin{center}
            \begin{tikzpicture}[yscale = 0.2, xscale = 1.5]
                \tkzInit[xmin=0,xmax=7.5,ymin=0,ymax=37]
                \tkzGrid[sub,color=gradeColor!50!white,subxstep=1,subystep=1]        
                \tkzLabelX[step=1]
                \tkzLabelY[step=5]
                \tkzDrawY[label={Prix (en \euro)}, above , step=5]
                \tkzDrawX[label={Tee-shirts vendus}, right, step=1]
                % Tracer les points
                \ifthenelse{\boolean{answer}}{
                    \foreach \x/\y/\z in {1/5/5, 2/10/10, 3/15/12, 4/20/17, 5/25/22, 6/30/24, 7/35/29}{
                        \drawPoint{}{\x}{\y}[Blue];
                        \drawPoint{}{\x}{\z}[Red];
                    }
                    \draw[Blue] (0,0) -- (7,35);
                    \draw[Red] (0,0) -- (2,10) -- (3,12) -- (4,17) -- (5,22) -- (6,24) -- (7,29);
                }{}
            \end{tikzpicture}
        \end{center}
        \item Les prix normaux et soldés sont-ils proportionnels au nombre de tee-shirts vendus ?
        Justifiez votre réponse à l'aide d'un argument graphique pour chaque cas.
    \end{enumerate}
}[\sesa{4}{2021}[5][61]]

\answerFill[Réponse][
    \begin{enumerate}\loadenumi[exo][1]
        \item \begin{itemize}
            \item Les points des prix des t-shirts non soldés sont alignés avec l'origine,
            il y a donc bien proportionnalité en période normale.
            \item Dans le cas des prix soldés,
            les points ne sont pas alignés,
            il n'y a donc pas proportionnalité.
        \end{itemize}
    \end{enumerate}
]

\exo{\tiersTemps Reconnaître un tableau de proportionnalité}{
    Les tableaux suivants présentent-ils des situations de proportionnalités ?
    \vspace{-1cm}
    \multiColEnumerate{2}{
        \item \Propor[Simple,Math,Stretch=1.25,%
        ]{-18/30,-6/10}
        % \item \Propor[Simple,Math,Stretch=1.25,%
        % ]{\awsr{\np{3}}/22,\np{1.5}/11}
        \item \Propor[Simple,Math,Stretch=1.25,%
        ]{\np{2.5}/-5,-60/120,-360/710}
    }
}

\answerSec{11}[Réponse][%
    \begin{enumerate}
        \item $-6 \times 30 = 180 = -18 \times 10$\\
        L'égalité des produits en croix étant vérifier,
        on a bien proportionnalité.
        \item $\frac{22\times\np{1.5}}{11} = \frac{33}{11} = 3$
        % \item $\frac{30\times(-6)}{-18} = \frac{-180}{-18} = 10$
        \item $\frac{-5}{\np{2.5}} = \frac{1}{2} = \frac{-360}{\np{720}} \neq \frac{-360}{\np{710}}$\\
        L'égalité des quotients n'étant pas vérifier, il n'y a pas proportionnalité. 
    \end{enumerate}
]

\exo{\bonus Contraposée et réciproque}{
    On s'intéresse à un quadrilatère.
    \begin{enumerate}  
        \item La proposition suivante est-elle vraie ?  
        «\Sialors{c'est un losange}{ses diagonales sont perpendiculaires.}»
        \item Écrivez la contraposée de cette proposition. Cette contraposée est-elle vraie ?
        \item Écrivez la réciproque de cette proposition. Cette réciproque est-elle vraie ? Justifiez votre réponse.  
    \end{enumerate}  
    
}

\answerFill[Réponse][%
    \begin{enumerate}
        \item La proposition est vraie.
        \item La contraposée de cette proposition est :
        «\Sialors{les diagonales ne sont pas perpendiculaires}{ce n'est pas un losange}».
        Cette contraposée est également vraie.
        \item La réciproque de cette proposition est :
        «\Sialors{les diagonales sont perpendiculaires}{c'est un losange}».
        Cette réciproque est fausse,
        car un cerf-volant est un quadrilatère dont les diagonales sont perpendiculaires sans être nécessairement un losange.
    \end{enumerate}
]

% \dm{Nombres de 2025}
\setGrade{6e}

\exo{}{
    Le chiffre en gras est obtenu par la somme des nombres
    situés sur une branche de l'étoile. Complète l'étoile
    
    \NombreAstral[%
    Graine=314,%
    Echelle=0.7%
    ]
}

% \isometric{
    \draw [thick,dotted] (10,29) -- (10,19) -- (9,18) -- (9,28) -- (10,29) -- (20,29) -- (20,19) -- (10,19);
    \draw [thick,dotted] (9,18) -- (19,18) -- (20,19);
}
% \dm{IsoPolis}
\setGrade{6e}

% \section*{IsoPolis : }

\subsection*{Introduction}
Vous êtes responsable de l'aménagement urbain de la nouvelle ville dont vous aurez également la liberté de choisir le nom.  

\subsection*{Consignes obligatoires :}
Votre plan urbain devra inclure les éléments suivants :
\begin{itemize}
    \item Au moins trois bâtiments.
    \item Au moins un bâtiment composé de deux solides simples ou plus.
    \item Des bâtiments utilisant des cubes et des pavés droits.
    \item Au moins un bâtiment utilisant un prisme droit ou une pyramide.
\end{itemize}

\subsection*{Consignes supplémentaires (facultatives) :}
Pour enrichir votre ville, vous pouvez intégrer :
\begin{itemize}
    \item Des bâtiments utilisant des cylindres.
    \item Des bâtiments utilisant des cônes.
    % \item Des solides ayant une base polygonale non rectangulaire.
    \item Une ou plusieurs îlots adjacents.
\end{itemize}

\subsection*{Travail préliminaire :}
Avant de commencer, réfléchissez aux points suivants :
\begin{enumerate}
    \item Voulez-vous suivre un thème spécifique ? 
    (Par exemple : une ville futuriste, far-west, médiévale, fantastique, etc.)
    \item Quels bâtiments voulez-vous intégrer dans votre ville ? 
    (Par exemple : un hôpital, une école, des maisons, etc.)
    \item Dessinez un brouillon sur une feuille de papier isométrique de quelques inspirations.
\end{enumerate}

\subsection*{Procédure de construction :}
\begin{enumerate}
    \item Tracer l'empreinte de votre ville :
    \begin{itemize}
        \item Sur une première feuille, dessinez l'empreinte de votre ville et nommez-y les batiments placés.
        \item Conseil : réfléchissez à l'usage prévu pour chaque bâtiment et leur emplacement dans la ville.
    \end{itemize}
    \item Construire les bâtiments :
    \begin{itemize}
        \item Sur une deuxième feuille isométrique, reproduisez l'empreinte de votre ville au crayon à papier.
        \item Toujours au crayon à papier, dessinez les solides correspondant à vos bâtiments en respectant les proportions.
        \item Effacez les traits cachés (ceux qui se trouvent derrière un bâtiment ou à l'intérieur d'une structure complexe).
    \end{itemize}
    \item Laissez libre cours à votre imagination : ajoutez des détailles à vos batiments comme vous le souhaitez !
\end{enumerate}

% \setSeq{6bis}{Représentation isometrique}
\setGrade{6e}

% \exo{Compléter les grilles isométriques suivantes}{
%     \multiColEnumerate{1}{
%         \item \dividePage{
%             Solides :
%             \sIso{
    \draw [thick] (4,4) -- (4,6) -- (6,8) -- (6,7) -- (7,7) -- (7,8) -- (6,8) -- (4,8) -- (3,7) -- (3,6) -- (4,6);
    \draw [thick] (6,7) -- (5,6) -- (6,6) -- (7,7);
    \draw [thick] (6,6) -- (6,5) -- (5,4) -- (4,4) -- (5,5) -- (5,6);
    \draw [thick] (6,5) -- (5,5);
    \draw [thick] (3,6) -- (4,7) -- (4,8);
    \draw [thick] (4,7) -- (5,7);
}
%         }{
%             Empruntes :
%             \sIso{}
%         }
%         \item \dividePage{
%             \sIso{}
%         }{
%             \sIso{
    \draw [thick] (3,7) -- (3,6) -- (4,6) -- (4,5) -- (5,5) -- (5,7) -- (3,7) -- cycle;
}
%         }
%         \item \dividePage{
%             \sIso{
    % \fill[thick,color=gradeColor,fill=gradeColor,fill opacity=0.10] (5,9) -- (6,8) -- (6,7) -- (7,7) -- (7,9) -- cycle;
    % \fill[thick,color=gradeColor,fill=gradeColor,fill opacity=0.10] (5,6) -- (5,7) -- (4,7) -- cycle;
    \draw [thick] (3,7) -- (4,8) -- (5,9) -- (7,9) -- (7,8) -- (7,7) -- (6,7) -- (6,8) -- (5,7) -- (5,6) -- (4,7) -- (3,6) -- (4,5) -- (6,7);
    \draw [thick] (7,7) -- (5,5) -- (4,5);
    \draw [thick] (4,7) -- (5,7);
    \draw [thick] (3,7) -- (3.50,6.50);
    \draw [thick] (5,9) -- (6,8);
}
%         }{
%             \sIso{}
%         }
%         \item \dividePage{
%             \sIso{}
%         }{
%             \csiso{
    \draw [thick] (3,5) -- (3,7) -- (6,7) -- (3,5) -- cycle;
}
%         }
%     }
% }

% \exo{}{Nommer le plus de solide particulier composant cet assemblages de solides.
%     \begin{center}
    \isometric[0.75][1pt][0.5]{
        \draw [shift={(9,11)},thick]  plot[domain=2.36:5.50,variable=\t]({1*1*cos(\t r)+0*1*sin(\t r)},{0*1*cos(\t r)+1*1*sin(\t r)});
        \draw [shift={(12,14)},thick]  plot[domain=2.36:5.50,variable=\t]({1*1*cos(\t r)+0*1*sin(\t r)},{0*1*cos(\t r)+1*1*sin(\t r)});
        \draw [shift={(13,15)},thick]  plot[domain=2.36:5.50,variable=\t]({1*1*cos(\t r)+0*1*sin(\t r)},{0*1*cos(\t r)+1*1*sin(\t r)});
        \draw [shift={(13,15)},thick]  plot[domain=-0.79:2.36,variable=\t]({1*1*cos(\t r)+0*1*sin(\t r)},{0*1*cos(\t r)+1*1*sin(\t r)});
        \draw [shift={(10,17)},thick]  plot[domain=1.99:5.86,variable=\t]({1*1*cos(\t r)+0*1*sin(\t r)},{0*1*cos(\t r)+1*1*sin(\t r)});
        % \draw [thick,dashed] (0.50,12.50) -- (8.50,4.50) -- (18.50,14.50) -- (10.50,22.50) -- (0.50,12.50) -- cycle;
        % \draw [thick,dashed] (5,11) -- (0.50,6.50) -- (4.50,2.50) -- (9,7) -- (5,11) -- cycle;
        \draw [thick] (13,13) -- (11,13) -- (11,15) -- (11,16) -- (9,16) -- (9,18) -- (9.78,18);
        \draw [thick] (11,13) -- (10,12) -- (10,14) -- (10,15) -- (11,16);
        \draw [thick] (10,14) -- (9,13) -- (8,12) -- (7.50,12);
        \draw [thick] (10,12) -- (12,12) -- (13,13) -- (12.97,13.55);
        \draw [thick] (9,16) -- (8,15) -- (6,13) -- (6,12) -- (8,14) -- (7,14) -- (7,15) -- (5,13) -- (6,13);
        \draw [thick] (6,12) -- (6,11) -- (7,13) -- (8,13) -- (7,11) -- (6,11);
        \draw [thick] (11.41,12) -- (9.71,10.29);
        \draw [thick] (10,13.41) -- (8.29,11.71);
        \draw [thick] (12.29,15.71) -- (11.29,14.71);
        \draw [thick] (13.71,14.29) -- (12.71,13.29);
        \draw [thick] (13,18) -- (13,16);
        \draw [thick] (9.59,17.91) -- (12,19) -- (10.91,16.59);
        \draw [thick] (9,18) -- (6,15) -- (6,14) -- (5,15) -- (4,14) -- (5,13);
        \draw [thick] (11.55,18) -- (13,18);
        \draw [thick] (5,15) -- (6,15);
        \draw [thick] (4,12) -- (4,11) -- (5,11) -- (5.50,12.50) -- (4,12) -- cycle;
        \draw [thick] (5.50,12.50) -- (4,11);
        \draw [thick] (7,10) -- (8,10) -- (8,7) -- (6,9) -- (8,10);
        \draw [thick] (6,9) -- (7,10);
    }
\end{center}
% }

\df{Cube}{
    \sIso{
    % \draw [thick,dashed] (0.50,12.50) -- (8.50,4.50) -- (18.50,14.50) -- (10.50,22.50) -- (0.50,12.50) -- cycle;
    % \draw [thick,dashed] (5,11) -- (0.50,6.50) -- (4.50,2.50) -- (9,7) -- (5,11) -- cycle;
    \draw [thick] (3,7) -- (5,9) -- (5,7) -- (7,7) -- (5,5) -- (3,5) -- (5,7) -- (3,5);
    \draw [thick] (5,9) -- (7,9) -- (7,7);
    \draw [thick] (3,5) -- (3,7);
    \draw [thick,dashed] (5,7) -- (7,9);
    \draw [thick,dashed] (5,5) -- (5,7);
    \draw [thick,dashed] (5,7) -- (3,7);
}  
}

\df{Cône}{
    \csiso{
    \draw [shift={(4,6)},thick]  plot[domain=1.83:6.02,variable=\t]({1*1.41*cos(\t r)+0*1.41*sin(\t r)},{0*1.41*cos(\t r)+1*1.41*sin(\t r)});
    \draw [shift={(4,6)},thick,dashed]  plot[domain=-0.26:1.83,variable=\t]({1*1.41*cos(\t r)+0*1.41*sin(\t r)},{0*1.41*cos(\t r)+1*1.41*sin(\t r)});
    \draw [thick] (3.48,7.32) -- (6.50,8.50) -- (5.32,5.48);
}
}




% \setSeq{6}{Espace - Solides}
\setGrade{5e}

\def\ym{https://www.maths-et-tiques.fr/telech/19Solides5e.pdf}

\obj{
    \item Reconnaître des solides (pavé droit, cube, cylindre, prisme droit, pyramide, cône, boule) à partir
    d'un objet réel, d'une image, d'une représentation en perspective cavalière.
    \item Construire et mettre en relation une représentation en perspective cavalière et un patron d'un pavé droit,
    d'un cylindre.
    \item Calculer le volume d'un pavé droit, d'un prisme droit, d'un cylindre.
    \item Calculer le volume d'un assemblage de ces solides.
    \item Effectuer des conversions d'unités de longueurs, d'aires, de volumes.
}

\df{}{
    On appel \key{polyhèdre} un solide qui possède des \key{faces} polygonales
}{\wiki{Polyèdre}}

\df{}{
    On appel \awsr{\key{prisme droit}} un polyhèdre délimité par 2 \key{bases} superposable reliés entrelles par des rectangles.
}{\wiki{Prisme_(solide)}}

\df{}{
    On appel \awsr{\key{pavé droit}} un \awsr{prisme droit} dont les bases sont rectangulaires.
}{\wiki{Pavé_droit}}

\df{}{
    On appel \awsr{\key{cube}} un \awsr{pavé droit} dont les bases sont carré.
}{\wiki{Cube}}

\df{}{
    On appel \awsr{\key{pyramide}} un polyhèdre formé d'une base polygoneale relié à un \key{sommet} par des faces \awsr{\key{triangulaires}}.
}{\wiki{Pyramide}}

\df{}{
    On appel \awsr{\key{cylindre}} un solide formé de deux base circulaire superposable relié par une surface courbe.
}{\wiki{Cylindre}}

\df{}{
    On appel \awsr{\key{cône}} un solide formé d'une base circulaire rélié à un point superposable au centre de la base par un sommet.
}{\wiki{Cône}}



% \slide{cr}{
%     \ctikz[1]{
%         \dotGrid[10][16][0.5pt][0.5][(10,-5)]
%         \fill[thick,color=gradeColor,fill=gradeColor,fill opacity=0.10] (6,-4) -- (9,-4) -- (8.50,-4.50) -- (7.50,-4.50) -- (7,-5) -- (5,-5) -- cycle;
%         \fill[thick,color=gradeColor,fill=gradeColor,fill opacity=0.10] (14,-6) -- (13,-7) -- (15,-7) -- (15.50,-6.50) -- (14.50,-6.50) -- (15,-6) -- cycle;
%         \fill[thick,color=gradeColor,fill=gradeColor,fill opacity=0.10] (7,-6) -- (9,-6) -- (7.50,-7.50) -- (6.50,-7.50) -- (7,-7) -- (5,-7) -- (5.50,-6.50) -- (6.50,-6.50) -- cycle;
%         \fill[thick,color=gradeColor,fill=gradeColor,fill opacity=0.10] (10,-6) -- (12.51,-6.01) -- (12.01,-6.51) -- (11.50,-6.50) -- (11,-7) -- (9,-7) -- cycle;
%         \fill[thick,color=gradeColor,fill=gradeColor,fill opacity=0.10] (10,-4) -- (13,-4) -- (12,-5) -- (9,-5) -- cycle;
%         \fill[thick,color=gradeColor,fill=gradeColor,fill opacity=0.10] (15,-4) -- (16,-4) -- (15.50,-4.50) -- (14.50,-4.50) -- cycle;
%         \node at (6.38, 3.99) {\cir[gradeColor]{1}};
%         \node at (10.42, 4.73) {\cir[gradeColor]{2}};
%         \node at (13.53, 3.7) {\cir[gradeColor]{3}};
%         \node at (4.93, 0.19) {\cir[gradeColor]{4}};
%         \node at (9.4, -1.35) {\cir[gradeColor]{5}};
%         \node at (14.41, -0.36) {\cir[gradeColor]{6}};
%         \draw [thick] (5,-2) -- (6,-2) -- (6.50,-1.50) -- (6.50,-0.50) -- (6,-1) -- (5,-1) -- (4,-1) -- (4,-2) -- (5,-2) -- cycle;
%         \draw [thick] (6,-2) -- (6,-1);
%         \draw [thick] (6.50,-0.50) -- (7.50,-0.50) -- (7.50,-1.50) -- (6.50,-1.50);
%         \draw [thick] (4,-1) -- (4.50,-0.50) -- (5.50,-0.50) -- (6,0) -- (7,0) -- (8,0) -- (8,-1) -- (7.50,-1.50);
%         \draw [thick] (4.50,-0.50) -- (4.50,0.50) -- (5.50,0.50) -- (6,1) -- (6,0);
%         \draw [thick] (5.50,-0.50) -- (5.50,0.50);
%         \draw [thick] (6,1) -- (5,1) -- (4.50,0.50);
%         \draw [thick] (7.50,-0.50) -- (8,0);
%         \draw [thick] (10,-2) -- (11,-2) -- (11.50,-1.50) -- (11.50,-0.50) -- (10.50,-0.50) -- (11,0) -- (10,0) -- (9.50,-0.50) -- (9,-1) -- (10,-1) -- (11,-1) -- (11.50,-0.50);
%         \draw [thick] (11,-1) -- (11,-2);
%         \draw [thick] (9,-1) -- (9,-2) -- (10,-2);
%         \draw [thick] (11,0) -- (11,-0.50);
%         \draw [thick,gradeColor] (6,-4) -- (9,-4) -- (8.50,-4.50) -- (7.50,-4.50) -- (7,-5) -- (5,-5) -- (6,-4) -- cycle;
%         \draw [thick,gradeColor] (14,-6) -- (13,-7) -- (15,-7) -- (15.50,-6.50) -- (14.50,-6.50) -- (15,-6) -- (14,-6) -- cycle;
%         \draw [thick] (14,0) -- (16,0) -- (17,1) -- (17,-1) -- (16,-2) -- (15.50,-2.50) -- (15.50,-1.50) -- (16,-1) -- (15,-1) -- (14.50,-1.50) -- (15.50,-1.50);
%         \draw [thick] (16,0) -- (16,-1);
%         \draw [thick] (17,1) -- (15,1) -- (14,0) -- (14,-1) -- (13,-1) -- (13,-2) -- (14,-2) -- (14.50,-2);
%         \draw [thick] (13,-1) -- (13.50,-0.50) -- (14,-0.50);
%         \draw [thick] (14.50,-1.50) -- (14.50,-2.50) -- (15.50,-2.50);
%         \draw [thick,gradeColor] (7,-6) -- (9,-6) -- (7.50,-7.50) -- (6.50,-7.50) -- (7,-7) -- (5,-7) -- (5.50,-6.50) -- (6.50,-6.50) -- (7,-6) -- cycle;
%         \draw [thick] (10,3) -- (11,3) -- (11.50,3.50) -- (11.50,6.50) -- (11,6) -- (10,6) -- (10,3) -- cycle;
%         \draw [thick] (10,6) -- (10.50,6.50) -- (11.50,6.50);
%         \draw [thick] (11,3) -- (11,6);
%         \draw [thick] (12.50,3) -- (14.50,3) -- (14,5) -- (12.50,3) -- cycle;
%         \draw [thick] (14,5) -- (15,4.33);
%     }
% }

% %%% VARIABLES %%%
\setSeq{6}{Gestion de donnés - Moyenne et médiane}
\setGrade{5e}
\def\imgPath{enseignement/6e/geometrie-plane/polygones/}
\def\ym{\href{https://www.maths-et-tiques.fr/telech/19Proba-stat.pdf}{Yvan Monka}}
% \forPrint
% \def\caPrefix{6e-juin-2022-}
%%

\obj{
    \item Calculer et interpréter la moyenne d'une série de données.
    \item Interpréter la médiane d'une série de données.
    \item Recueillir et organiser des données, sous forme de tableaux, de graphiques.
    \item Traduire la relation de dépendance entre deux grandeurs par un tableau de valeurs.
    \item Produire une formule représentant la dépendance de deux grandeurs.
}

\scn{Découvrir la moyenne et la médiane}

% \slide{qf}{
%     Rappelle : Pour calculer une \key{moyenne}, on utilise les étapes suivantes :
%     \begin{enumerate}
%         \item Additionne toutes les valeurs données.
%         \item Divise cette somme par le nombre total de valeurs.
%     \end{enumerate}
%     \bvspace{-0.2cm}
%     Questions :
%     \begin{enumerate}
%         \item Quelle est la moyenne des nombres suivants : \( 4, \; 6, \; 8 \) ?
%         \item Quelle est la moyenne de \( 2, \; 3, \; 5, \; 10 \) ?
%     \end{enumerate}
% }

\def\iconPath{minecraft/}\def\iconSize{25pt}
\slide{qf}{
    \nullsubsec{}{
        Steve \icon{player-head} utilise une pioche \icon{diamond-pickaxe} Fortune III pour miner 6 minerais de charbon \icon{coal-ore}. 
        Chaque \icon{coal-ore} lui donne une quantité variable de charbon \icon{coal},
        correspondant aux valeurs suivantes : $4\,;1\,;3\,;2\,;3\,;4$.
        Détermine combien de morceaux de \icon{coal} est obtenue en moyenne par \icon{coal-ore}.        
    }[\href{https://fr.minecraft.wiki/w/Charbon}{Minecraft wiki}]
}

\slide{exo}{
    \act{}{
        Steve \icon{player-head} explore une grotte et trouve des coffres \icon{chest}
        contenant des diamants \icon{diamond}.  
        Les \icon{chest} suivants contiennent respectivement :  
        $ 8, \; 4, \; 6, \; 10, \; 2, \; 12, \; 5 \; \icon{diamond}. $
    
        \begin{enumerate}
            \item Combien de \icon{diamond} obtient-il en moyenne par \icon{chest}? \saveenumi
        \end{enumerate}
    
        \icon{player-head} veut estimer le nombre typique de \icon{diamond} qu'il peut trouver par \icon{chest} pour ses prochaines explorations.  
        Pour cela, il décide de calculer la \key{médiane}, c'est-à-dire le nombre qui partage cette série en deux groupes contenant autant de valeurs.
    
        \begin{enumerate} \loadenumi
            \item Classe les nombres de \icon{diamond} par \icon{chest} dans l'ordre croissant.
            \item Détermine la médiane et explique son interprétation dans ce contexte.
        \end{enumerate}
    }
}

\bsec{Définir la moyenne et la médiane}

\slide{cr}{
    \df{}{
        La \key{moyenne} d'une série de données est un nombre qui permet de représenter l'ensemble des valeurs de manière synthétique. Elle se calcule en ajoutant toutes les valeurs, puis en divisant par leur nombre total.
    }[\wiki{Moyenne}]
}

\slide{cr}{
    \df{}{
        La \key{médiane} d'une série de données ordonnées est une valeur qui partage cette série en deux groupes de même effectif :
        \begin{itemize}
            \item Si le nombre de données est impair, la médiane est la valeur centrale.
            \item Si le nombre de données est pair, la médiane est la moyenne des deux valeurs centrales.
        \end{itemize}
    }[\wiki{Médiane}]
}

\slide{exo}{\bsmall
    \begin{enumerate}\setItemColor{act}
        \item Trouve la moyenne et la médiane des séries suivantes :
        \begin{itemize}
            \item \( 7, \; 10, \; 12, \; 15, \; 8 \)
            \item \( 5, \; 6, \; 8, \; 9, \; 10, \; 12 \)
        \end{itemize}
        \item Explique pourquoi la moyenne et la médiane ne sont pas toujours égales.
        \item Propose une situation où la médiane serait plus utile que la moyenne pour représenter les données.
    \end{enumerate}
}

% VARIABLES %%%
\setTitle{test}
%%%%%%%%%%%%%%%

\setSeq{2}{TEST SEQ}

\setGrade{6e}

\bsec{test}

\def\aspc{\ifbool{answer}{}{\vspace{1cm}}}

\begin{frame}
    \begin{enumerate}
        \item \multiColEnumerate{2}{
            \item zvgew 
            \item zvgew 
            \item zvgew 
            \item zvgew
            \item zvgew 
            \item zvgew
        }
    \end{enumerate}
    \framebreak
    eqgrpoiesqjgr
\end{frame}

\slide{exo}{
    \begin{enumerate} \loadenumi \setItemColor{RoyalBlue}
        \item \multiColEnumerate{2}{
            \item zvgew 
            \item zvgew 
            \item zvgew 
            \item zvgew
            \item zvgew 
            \item zvgew
        }
        \item zvgew 
        \item zvgew \framebreak
        \item zvgew 
        \item zvgew
        \item zvgew 
        \item zvgew 
    \end{enumerate}
}

\slide{exo}{
    efhuzeiugfh
}

\slide{cr}{
    z"gqrzg
}
\end{document}