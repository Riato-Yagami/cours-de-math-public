\documentclass[aspectratio=169, usenames,dvipsnames,xcolor=table]{beamer}
\usepackage[fontsize=14pt]{fontsize}

\usepackage[T1]{fontenc}
\usepackage[french]{babel}

\usepackage[utf8]{inputenc}
\usepackage{amsmath}
\usepackage{amsthm}
\usepackage{amssymb}
\usepackage{graphicx}
\usepackage{dashundergaps}
\usepackage{array}
\usepackage{multicol}
\usepackage{wrapfig}
\usepackage{numprint}
\usepackage{ulem}
\usepackage{hyperref}
\usepackage{mathrsfs}
\usepackage{mathtools}
\usepackage[many]{tcolorbox}
\usepackage{xparse}
\usepackage{float}
\usepackage{lipsum}
\usepackage{pgf}
\usepackage{ifthen}
\usepackage{caption}
\usepackage{tikz}
\usepackage{xifthen}

% \usepackage[squaren,Gray]{SIunits}

% BREVET

% \usepackage{makeidx}
% \usepackage{fancybox}
% \usepackage{tabularx}
% \usepackage[normalem]{ulem}
% \usepackage{pifont}
% \usepackage{lscape}
% \usepackage{diagbox}
% \usepackage{multirow} 
% \usepackage{textcomp}
% \usepackage{scratch3}
% \usepackage[T1]{fontenc}
% \usepackage{fourier}
% \usepackage[french]{babel}
% \usepackage{pstricks}

% \usepackage[scaled=0.875]{helvet}
% \usepackage{pst-plot,pst-text,pst-tree,pstricks-add}

% fancyhdr

\setlength{\headheight}{18pt}
\fancyhead[C]{\normalsize \title}
% \renewcommand{\headrulewidth}{0pt} % Remove header line
\fancyhead[R]{}
\fancyfoot[L]{\author}
\fancyfoot[C]{\textbf{Page \thepage/\pageref{LastPage}}}
\fancyfoot[R]{\date}

\fancypagestyle{firstpage}{
    \setlength{\headheight}{29pt}
    \fancyhead[C]{\LARGE \title}
    \fancyhead[R]{}
    \fancyfoot[L]{\author}
    \fancyfoot[C]{\textbf{Page \thepage/\pageref{LastPage}}}
    \fancyfoot[R]{\date}
}

\thispagestyle{firstpage}

% \fancyfoot[C]{\textbf{Page 1/1}}

% HYPERREF

\hypersetup{
    colorlinks=true,       % false: boxed links; true: colored links
    linkcolor=red,          % color of internal links (change box color with linkbordercolor)
    citecolor=green,        % color of links to bibliography
    filecolor=magenta,      % color of file links
    urlcolor=blue,          % color of external links
    urlbordercolor=blue,    % borders of external links
    linkbordercolor=red,    % borders of internal links
    pdfborderstyle={/S/U/W 1}% border style will be underline of width 1pt
}

\usepackage[fontsize=14pt]{fontsize}

\usepackage[T1]{fontenc}
\usepackage[french]{babel}
\usepackage[utf8]{inputenc}

\frenchbsetup{StandardItemLabels=true}

% GLOBAL VARIABLES %%%
\graphicspath{{images}}
\def\cwidth{4cm}
\def\tspace{0.5cm}

% BOOLEAN %%%
\newboolean{anwser}
\newboolean{demonstration}
\newboolean{boxedProperties}
\newboolean{showID}
\newboolean{parenthisedID}
\newboolean{animated}
\newboolean{outline}

\setboolean{anwser}{false}
\setboolean{demonstration}{true}
\setboolean{parenthisedID}{true}
\setboolean{showID}{true}
\setboolean{boxedProperties}{false} % false = edge
\setboolean{outline}{false}

\def\DefinitionColor{PineGreen}
\def\PropertyColor{Blue}
\def\TheoremColor{Plum}

\def\SectionColor{Red}
\def\SubSectionColor{Green}

\setboolean{animated}{true}

% \DeclareMathOperator{\PGCD}{PGCD}
% \DeclareMathOperator{\PPCM}{PPCM}

\DeclareMathOperator{\sh}{sh}
\DeclareMathOperator{\ch}{ch}
% \DeclareMathOperator{\th}{th}

\DeclareMathOperator{\argsh}{argsh}
\DeclareMathOperator{\argch}{argch}
\DeclareMathOperator{\argth}{argth}
\DeclareMathOperator{\I}{I}
\DeclareMathOperator{\Id}{Id}
\DeclareMathOperator{\Ker}{Ker}
% \DeclareMathOperator{\dl}{o}
\newcommand{\dl}[1]{
    \operatorname*{o}_{#1}
}

\def\deg{\ensuremath{^\circ}}
\def\prll{\mathbin{\!/\mkern-5mu/\!}}
\renewcommand{\parallel}{\mathbin{\!/\mkern-5mu/\!}}

\def\octet{\textrm{o}}
\def\byte{\textrm{B}}

\def\hour{\textrm{h}}
\def\minute{\textrm{min}}
\def\second{\textrm{s}}
% ENVIRONMENT
\newenvironment{mysection}[1][gray!20]{%
    \begin{sectionBox}[#1]
}{%
    \end{sectionBox}
}

\newenvironment{mysubsection}[1][gray!20]{%
    \begin{subsectionBox}[#1]
}{%
    \end{subsectionBox}
}

% Switch implementation
\newboolean{default}
\newcommand{\case}{}
\newcommand{\default}{}

\newenvironment{switch}[1]{%
    \setboolean{default}{true}
    \renewcommand{\case}[2]{\ifthenelse{\equal{#1}{##1}}{%
        \setboolean{default}{false}##2}{}}%
    \renewcommand{\default}[1]{\ifthenelse{\boolean{default}}{##1}{}}
}{}

% SECTIONS
\input{header/command/sections.tex}

% ANSWERS
\newlength{\parline}
\newlength{\paroutindent}
\newlength{\lineheight}
\setlength{\lineheight}{\heightof{abcdefghijklmnoprstuvwxyz}}

\newcommand{\countlines}[1]{%
    \setlength{\paroutindent}{\expandafter\parindent}
    \setlength{\parline}{\heightof{\noindent\begin{minipage}{\linewidth}%
                \setlength{\parindent}{\paroutindent}#1\end{minipage}}}%
    \pgfmathparse{round(\parline / (0.9*\lineheight))}
    \newcount\linecount
    \pgfmathsetcount{\linecount}{\pgfmathresult}
}

\newcommand{\looptext}[2]{%
    \noindent
    \newcount\printcount
    \printcount=#2
    \loop
        #1
        \advance\printcount by -1
        \ifnum\printcount>0
    \repeat
}

\newcommand{\awsr}[1]{%
    \ifthenelse{\boolean{answer}}{
        \result{#1}
    }{
        \countlines{#1}
        \pgfmathsetcount{\linecount}{\linecount + 1}
        \noindent\hspace{-9pt}
        \looptext{
            \noindent\dotfill
    
        }{\the\linecount}
    }
}

\newcommand{\dottedLines}[1]{%
    \noindent\hspace{-9pt}%

    \looptext{%
        \noindent\dotfill%

    }{#1}
}

\newcommand{\result}[1]{\color{OrangeRed}#1\color{black}}

% MATH
\input{header/command/math.tex}

% IMAGES
\input{header/command/image.tex}

% COMMANDS

\newcommand{\fsize}[1]{\fontsize{#1}{#1}\selectfont}

\NewDocumentCommand{\ifNotNull}{mmo}{
    \IfValueT{#1}{
        \ifx\relax#1\relax
            \IfValueT{#3}{
                #3
            }
        \else
            #2
        \fi
    }
}

\NewDocumentCommand{\ilink}{m g}{%
    \item
    \IfValueTF{#2}{\link{#1}{#2}}{\link{#1}}
}

\NewDocumentCommand{\link}{m g}{%
    \csn{#1}%
    \IfValueT{#2}{(#2)}%
}

\NewDocumentCommand{\TODO}{g}{%
    {\color{Red} $\rightarrow$ \textbf{TODO}
    \IfValueT{#1}{(#1)}}
    % \color{black}
}

\newcommand{\leconInfoBox}[2]{
    \textbf{#1 :}\vspace{-0.25cm}
        \begin{multicols}{2}
            \begin{itemize}[label=$\blacktriangleright$, font = \small \color{Red}]
                #2
            \end{itemize}
        \end{multicols}
        \vspace{-0.4cm}
}

% TCOLORBOX

\input{header/command/tcolorbox.tex}

\NewDocumentCommand{\leconInfo}{mooo}{
    \begin{infoBox}
        \leconInfoBox{Niveaux}{#1}
        \ifNotNull{#2}{
            \tcbline
            \leconInfoBox{Prérequis}{#2}
        }
        \ifNotNull{#3}{
            \tcbline
            \leconInfoBox{Thèmes}{#3}
        }
        \ifNotNull{#4}{
            \tcbline
            \textbf{Motivation :} 
            #4
        }
    \end{infoBox}
}

\NewDocumentCommand{\seanceInfo}{oooooooo}{
    \begin{infoBox}
        \vspace{-0.05cm}
        \begin{tcbitemize}[raster rows=1,raster columns=20,raster height=1.65cm,
            raster every box/.style={colframe=red!50!black,colback=red!10!white}]
            \tcbitem[raster multicolumn=6] \textbf{Date :} #1
            \tcbitem[raster multicolumn=10] \textbf{Séquence :} #2
            \tcbitem[raster multicolumn=4] \textbf{Séance :} #3
        \end{tcbitemize}
        \vspace{-0.25cm}
        \ifNotNull{#4}{\tcbline \textbf{Objectif :} #4}
        \ifNotNull{#5}{\tcbline \leconInfoBox{Classe(s)}{#5}}
        \ifNotNull{#6}{\tcbline \leconInfoBox{Prérequi(s)}{#6}}
        \ifNotNull{#7}{\tcbline \textbf{Séance précédente :} #7}
        \ifNotNull{#7}{\tcbline \leconInfoBox{Matériel(s)}{#8}}
    \end{infoBox}
}

\def\pDscr{\tcbitem[enhanced jigsaw, breakable,
    raster multicolumn=6]
}
\def\pMdlt{\tcbitem[enhanced jigsaw, breakable,
    raster multicolumn=11]
}
\def\pTime{\tcbitem[enhanced jigsaw, breakable,
    raster multicolumn=3, halign=center]
}

\newcommand{\prepRow}[3]{
    \tcbitem[raster multicolumn=20]
    \tcblower

    \pDscr #1
    \pMdlt #2
    \pTime #3
}

\newcommand{\prepTable}[1]{
    \begin{prepBox}
        \begin{tcbitemize}[enhanced jigsaw, breakable, raster rows=1,raster columns=20,raster height=1.1cm, halign=center,
            raster every box/.style={enhanced jigsaw, breakable, colframe=Blue!50!black,colback=Blue!10!white}]
            \pDscr \textbf{Descriptif}
            \pMdlt \textbf{Modalité}
            \pTime \textbf{Durée}
        \end{tcbitemize}
        \begin{tcbitemize}[enhanced jigsaw, breakable,
            raster equal height = rows, 
            raster columns=20, frame hidden,
            raster every box/.style={
                enhanced jigsaw, breakable,
                opacityback=0, valign=top, 
                size = tight
            }]
            #1
        \end{tcbitemize}
    \end{prepBox}
}

% TIKZ

\newcommand{\ctikz}[1]{
    \begin{center}
        \begin{tikzpicture}
            #1
        \end{tikzpicture}
    \end{center}
}

\newcommand{\axis}[1]{%Draw coordinate axes
    \draw[thin, -Stealth] (-0.5,0) -- (#1,0);% node[right] {$x$}; % x-axis
    \draw[thin, -Stealth] (0,-0.5) -- (0,#1);% node[above] {$y$}; % y-axis
}

\newcommand{\drawGrid}[3]{
    \foreach \n in {0,...,#1}
        \draw[line width = #3] (\n,0) -- (\n,#2);
    \foreach \n in {0,...,#2}
        \draw[line width = #3] (0,\n) -- (#1,\n);
}

\newcommand{\drawPoint}[4]{
    \node[shift={#4}, color = \pointColor] at (#2 - 0.5,#3 - 0.5) {#1};
    \draw[line width = \crossWidth, shift={#4}, color = \pointColor] (#2 - 0.25,#3) -- (#2 + 0.25,#3);
    \draw[line width = \crossWidth, shift={#4}, color = \pointColor] (#2,#3 - 0.25) -- (#2,#3 + 0.25);
}

% Tabular
\newcolumntype{C}[1]{>{\centering\arraybackslash}p{#1}}
\newcolumntype{M}[1]{>{\centering\arraybackslash}m{#1}}
\newcolumntype{K}{@{}m{0pt}@{}}

% GEOMETRY

% \newcommand{\restoregeometry}{def}

\newcommand{\multiColItemize}[2]{
    \begin{multicols}{#1}
        \begin{itemize}
            #2
        \end{itemize}
    \end{multicols}
}

\newcommand{\multiColEnumerate}[2]{
    \ifthenelse{\isequivalentto{#1}{1}}{
        \begin{enumerate}
            #2
        \end{enumerate}
    }{
        \begin{multicols}{#1}
            \begin{enumerate}
                #2
            \end{enumerate}
        \end{multicols}
    }
}

\makeatletter
\newcommand\pgfinvisible{\pgfsys@begininvisible}
\newcommand\pgfshown{\pgfsys@endinvisible}
\makeatother

\renewcommand*{\phantom}[1]{
    \pgfinvisible #1 \pgfshown
}

\newcounter{size}
\newcommand{\listSize}[1]{%
    \setcounter{size}{0}%
    \foreach \n in {#1}{\stepcounter{size}}%
    % \thesize
}

\newcounter{elemPos}
\newcommand{\listElement}[2]{
    \setcounter{elemPos}{0} % Start counting from 1
    \def\resultVal{0} % Default value
    \renewcommand*{\do}[1]{%
        \ifnumequal{\value{elemPos}}{#2}{%
            \def\resultVal{##1}%
            \listbreak% Break out of the loop
        }{}%
        \stepcounter{elemPos}%
    }
    % \docsvlist{#1}
    \expandafter\docsvlist\expandafter{#1} % Expand the list before passing it to \docsvlist
    \resultVal
}

% \NewDocumentCommand{\exoslide}{m O{10cm}}{
%     \slide{}{
%         \img{\imgf{#1}}[#2]
%     }
% }

\NewDocumentCommand{\exoSlide}{m O{10cm} O{1} O{} O{exo}}{%
    \slide{#5}{%
        \ifthenelse{\equal{#3}{1}}{\vspace{-0.5cm}}{\vspace{-1cm}}
        \def\exercices{\foreach \q in {#1}{\imgp{\q}[#2]\vspace{-0.5cm}}}
        \exo{#1}{\wideFrame[7em]{\bvspace{0.25cm}\avspace{-0.25cm}
            \ifthenelse{\equal{#3}{1}}{\exercices}
            {\begin{multicols}{#3}\exercices\end{multicols}}}
            \avspace{0.75cm}
        }[#4]
    }
}

\NewDocumentCommand{\exoList}{m O{} O{3}}{%
    \section*{Exercices}%
    \slide{EXERCICES}{
        \exo{#2}{
            \vspace{-0.25cm}
            \multiColEnumerate{#3}{
                \foreach \q in {#1}{
                    \item \q
                }
            }
        }
    }
}

\newcommand{\questions}[1]{
    \begin{enumerate}
        \foreach \q in {#1}{
            \item \q\\
            \vspace*{-0.45cm}
            \dottedLines{3}
        }
    \end{enumerate}
}

% Define a new boolean for checking if the section is starred
\newboolean{section@star}

\makeatletter
% Redefine \section and \section* to set the boolean
\let\old@section\section
\renewcommand{\section}{%
    \@ifstar
        {\setboolean{section@star}{true}\old@section*}
        {\setboolean{section@star}{false}\old@section}%
}
\makeatother

\newcommand{\qt}[1]{«\textit{#1}»}

\newcommand{\calc}[1]{\numexpr#1\relax}
\newcommand{\ncalc}[1]{\number\calc{#1}}
\newcommand{\pcalc}[1]{\numprint{\ncalc{#1}}}

\newcommand{\setgrade}[1]{
    \def\grade{#1}
    % \begin{switch}{#1}
    %     \case{6e}{\global\definecolor{gradeColor}{hex}{FA8072}}
    %     \default{
    %         Default
    %         \global\definecolor{gradeColor}{RGB}{200, 50, 50}
    %     }
    % \end{switch}
    \ifthenelse{\equal{#1}{6e}}{
        \definecolor{gradeColor}{HTML}{C6233D} % FA8072 in hex
    }{
    \ifthenelse{\equal{#1}{5e}}{
        \definecolor{gradeColor}{HTML}{088255}
    }{
    \ifthenelse{\equal{#1}{4e}}{
        \definecolor{gradeColor}{HTML}{1466A8}
    }{
    \ifthenelse{\equal{#1}{3e}}{
        \definecolor{gradeColor}{HTML}{844499}
    }{
        \definecolor{gradeColor}{RGB}{0, 0, 0}
    }}}}
}

\gdef\phase{}
\newcommand{\setPhase}[1]{%
    \begin{switch}{#1}
        \case{exo}{\gdef\phase{EXERCICES}}
        \case{cr}{\gdef\phase{COURS}}
        \case{qf}{\gdef\phase{QUESTIONS FLASH}}
        \case{dm}{\gdef\phase{DEVOIR MAISON}}
        \default{\gdef\phase{#1}}
    \end{switch}
}

\newcounter{savedenumi}
\setcounter{savedenumi}{0}
\xdef\savedenumi{0}
% \newcommand{\saveenumi}{
%     % \xdef\savedenumi{\calc{\theenumi-1}}
%     \setcounter{savedenumi}{0}
% }

\newcommand{\saveenumi}[1]{
    \setcounter{savedenumi}{#1}
}

\newcommand{\loadenumi}{
    \setItemColor{\currentColor}
    \setcounter{enumi}{\thesavedenumi}
}

\newcommand\csn[1]{\csname #1\endcsname}

\newcommand{\vect}[1]{\ensuremath{\overrightarrow{#1}}}
% \newcommand{\vect}[1]{\overrightarrow{\,\mathstrut#1\,}}
\newcommand{\m}[1]{\ensuremath{\mathbf{#1}}}
\newcommand\lm[2]{\lim_{#1\to#2}}

\def\eqv{\Leftrightarrow}
\def\ssi{si et seulement si }
\def\pt{pour tout }
\def\poly2{fonction polynôme du second degré }
\def\eq2{équation second degré }
\def\discr{b^2-4ac}

% MATH TEXT
\def\et{\textrm{ et }}
\def\si{\textrm{ si }}
\def\avec{\textrm{ avec }}
\def\car{\textrm{ car }}
\def\alors{\textrm{ alors }}
\def\ou{\textrm{ ou }}
\def\ona{\textrm{ on a }}

\def\iet{\shortintertext{et}}
\def\ialors{\shortintertext{alors}}
\def\idou{\shortintertext{d'où}}
\def\ior{\shortintertext{or}}
\def\iona{\shortintertext{on a}}

\def\studentinfo{
    \vspace*{-1cm}
    \begin{minipage}{0.35\linewidth}
        nom: \dotfill
    \end{minipage}
    \begin{minipage}{0.35\linewidth}
        prénom: \dotfill
    \end{minipage}
    \begin{minipage}{0.15\linewidth}
        classes: \dotfill
    \end{minipage}
    
    \noindent\hrulefill
}

% UNITS
\def\cm{\,\centi\meter}
\def\km{\,\kilo\meter}
\newcommand{\defl}[2]{%
    \expandafter\def\csname #1\endcsname{\href{#2}{#1}\space}%
}

% Page Eduscol
\defl{Eduscol Cycle 3}{https://eduscol.education.fr/251/mathematiques-cycle-3}
\defl{Eduscol Cycle 4}{https://eduscol.education.fr/280/mathematiques-cycle-4}
\defl{Eduscol Lycée Général et technologique}{https://eduscol.education.fr/1723/programmes-et-ressources-en-mathematiques-voie-gt}
\defl{Eduscol Lycée Professionnel}{https://eduscol.education.fr/1793/programmes-et-ressources-en-mathematiques-voie-professionnelle}

% Repères annuels
\defl{Cycle 3}{https://eduscol.education.fr/document/14026/download}
\defl{Cycle 4}{https://eduscol.education.fr/document/14080/download}

% Attendus de fin d'année
\defl{5e}{https://eduscol.education.fr/document/14044/download}
\defl{4e}{https://eduscol.education.fr/document/14056/download}
\defl{3e}{https://eduscol.education.fr/document/14068/download}

% Programme de mathématiques
\defl{cycle 3}{https://eduscol.education.fr/document/50990/download}
\defl{cycle 4}{https://cache.media.education.gouv.fr/file/31/89/1/ensel714_annexe3_1312891.pdf}
\defl{2nd}{https://eduscol.education.fr/document/24553/download}
\defl{1re}{https://eduscol.education.fr/document/24565/download}
\defl{1re STL}{https://eduscol.education.fr/document/23098/download}
\defl{1re STI2D}{https://eduscol.education.fr/document/24919/download}
\defl{Terminale Option Spécialité}{https://eduscol.education.fr/document/24568/download}
\defl{Terminale Option Complémentaire}{https://eduscol.education.fr/document/24571/download}
\defl{Terminale Option Expertes}{https://eduscol.education.fr/document/24574/download}
\defl{Terminale STL}{https://eduscol.education.fr/document/23107/download}
\defl{Terminale STI2D}{https://eduscol.education.fr/document/24922/download}
% Ressources thématiques
\defl{Proportionnalité}{https://eduscol.education.fr/document/17281/download}
\defl{Probabilités}{https://eduscol.education.fr/document/17275/download}
\defl{Fonctions}{https://eduscol.education.fr/document/17287/download}
\defl{Traitement des données}{https://eduscol.education.fr/document/17269/download}

\defl{Fonctions}{https://eduscol.education.fr/document/17287/download}
\defl{Fractions}{https://eduscol.education.fr/document/17239/download}
\defl{Nombres relatifs}{https://eduscol.education.fr/document/17245/download}
\defl{Puissances}{https://eduscol.education.fr/document/17251/download}
\defl{Divisibilité et nombres premiers}{https://eduscol.education.fr/document/17257/download}
\defl{Calcul littéral}{https://eduscol.education.fr/document/17263/download}

\defl{Grandeurs et mesures}{https://eduscol.education.fr/document/17293/download}
\defl{Algorithmique et programmation}{https://eduscol.education.fr/document/17311/download}

\defl{Suites}{https://eduscol.education.fr/document/24586/download}
\defl{Produit Scalaire}{https://eduscol.education.fr/document/24589/download}
\defl{Raisonnement et démonstration (seconde)}{https://eduscol.education.fr/document/24580/download}
\defl{Raisonnement et démonstrations (première)}{https://eduscol.education.fr/document/24583/download}

\captionsetup{labelformat=empty,labelsep=none}

% \setboolean{boxedProperties}{true} % false = edge
% \setboolean{parenthisedID}{false}
% \setboolean{showID}{false}

% \def\DefinitionColor{Red}
\def\PropertyColor{Red}
\def\TheoremColor{Red}

% TIKZ
\def\crossWidth{0.25mm}
\def\pointColor{blue}

\usepackage{bookmark}
% THEMES
% http://mcclinews.free.fr/latex/beamergalerie/completsgalerie.html
% default
% \usetheme{Madrid}
% \usetheme{CambridgeUS}

% tree
% \usetheme{Montpellier}
% \usetheme{Juanlespins}

\newcommand{\headerBox}[2]{
    \begin{beamercolorbox}[wd=\paperwidth,ht=2.125ex,dp=1.125ex,leftskip=.3cm,rightskip=.3cm plus1fil]{#1}%
        \usebeamerfont{#1}#2%
    \end{beamercolorbox}
}

\usetheme{Antibes}
\setbeamertemplate{headline}
{%  
    \headerBox{title in head/foot}{
        \insertshorttitle
        \hfill
        \color{links}\insertauthor
    }
    \ifx\insertsectionhead\empty\else
    \headerBox{section in head/foot}{
        \hskip6pt \ifthenelse{\boolean{section@star}}{$\rightarrow$}{\Roman{sec}.} \insertsectionhead
        \hfill
        \insertframenumber{} / \inserttotalframenumber
    }
    \fi
    \ifx\insertsubsectionhead\empty\else
    \headerBox{subsection in head/foot}{
        \hskip12pt \thesubsection{} \insertsubsectionhead
    }
    \fi
    \ifx\insertsubsubsectionhead\empty\else
    \headerBox{subsubsection in head/foot}{
        \hskip18pt \thesubsubsection{} \insertsubsubsectionhead
    }
    \fi
}

\setbeamerfont{frametitle}{size=\small,series=\bfseries}

\setbeamertemplate{frametitle}
{
    \vspace{-1.5pt} % Adjust the vertical space before the title
    \begin{beamercolorbox}[ht=2.5ex,dp=1.0ex,wd=\paperwidth,leftskip=.3cm,rightskip=.3cm]{frametitle}
        \usebeamerfont{frametitle}\insertframetitle
    \end{beamercolorbox}
}


% \setbeamertemplate{footline}
% {%
%     \leavevmode%
%     \hbox{%
%     \begin{beamercolorbox}[wd=.5\paperwidth,ht=2.25ex,dp=1ex,left]{author in head/foot}%
%         \usebeamerfont{author in head/foot}\hspace*{2ex}\insertauthor
%     \end{beamercolorbox}%
%     \begin{beamercolorbox}[wd=.5\paperwidth,ht=2.25ex,dp=1ex,right]{title in head/foot}%
%         \usebeamerfont{title in head/foot}\insertframenumber{} / \inserttotalframenumber\hspace*{2ex}
%     \end{beamercolorbox}}%
%     \vskip0pt%
% }

% lateral
% \usetheme{Hannover}

% navigation
% \usetheme{Frankfurt}

% sections and sub
% \usetheme{Warsaw}

% \usetheme{shadow}
% \usetheme{AnnArbor}

% color
% \usecolortheme{beaver}
\usecolortheme{spruce}
% \definecolor{UBCblue}{rgb}{0.04706, 0.13725, 0.26667} % UBC Blue (primary)
% \definecolor{UBCgrey}{rgb}{0.3686, 0.5255, 0.6235} % UBC Grey (secondary)
\setbeamercolor{palette primary}{bg=gradeColor!20,fg=gradeColor}
\setbeamercolor{palette secondary}{bg=gradeColor!95,fg=white}
\setbeamercolor{palette tertiary}{bg=gradeColor!70,fg=white}
\setbeamercolor{palette quaternary}{bg=gradeColor!60,fg=white}

% \setbeamercolor{structure}{fg=gradeColor} % itemize, enumerate, etc
% \setbeamercolor{section in toc}{fg=gradeColor} % TOC sections


% \usecolortheme[named=gradeColor]{structure}

\definecolor{links}{HTML}{e6ffe6}
\definecolor{hyperlinks}{HTML}{e6ffe6}
\hypersetup{
    colorlinks = true,
    linkcolor = links, % Apply the color to internal links
    urlcolor = hyperlinks   % Apply the color to URLs
}

% \setbeamercolor{item}{fg=ForestGreen}
\setbeamercolor{item}{fg=MidnightBlue}

\setbeamersize{
    text margin left=1.5cm,
    text margin right=1.5cm
}

\setbeamercovered{transparent = 25}
\setbeamertemplate{navigation symbols}{}

% \setbeamertemplate{enumerate subitem}{(\alph{enumii})}

% \setbeamertemplate{enumerate subitem}[square]
% \setbeamertemplate{enumerate items}[default]

\newcommand*{\setItemColor}[1]{
    \setbeamercolor{item}{fg=#1}
}
\def\authors{Jules PESIN}
\def\longTitle{long Title}
\def\shortTitle{short Title}
% \def\day{XX/XX/XX}

\title[\shortTitle]{\longTitle}
% \date{\day}

\newcommand{\slide}[2]{
    \begin{frame}
    \frametitle[#1]{#1}
        #2
    \end{frame}
}

\newcounter{sec}
% \stepcounter{sec}
\newcounter{subsec}
% \stepcounter{subsec}

% \newcommand{\bchap}[1]{
%     \color{Red} CHAPITRE : #1\color{black}\\
% }

\newcommand{\bseq}[1]{
    \def\sseq{\color{Red} CHAPITRE \theseq{} : #1\color{black}\\}
    \def\shortTitle{\MakeUppercase{#1}}
    \def\theme{#1}
    \setcounter{sec}{0}
}

\newcommand{\bsec}[1]{
    \section{#1}
    \def\ssec{\color{Red} \Roman{sec}. #1\color{black}\\}
    \stepcounter{sec}
    \setcounter{subsec}{0}
}

\newcommand{\bsubsec}[1]{
    \subsection{#1}
    \def\ssubsec{\color{Green} \thesubsec) #1\color{black}\\}
    \stepcounter{subsec}
}

\newcommand{\palt}[2]{
    \alt<#1->{\result{#2}}{\phantom{#2}}
    % \alt<#1->{\result{#2}}{\pgfinvisible #2 \pgfshown}
    % \alt<#1->{\result{#2}}{\textcolor{white}{#2}}
    % \alt<#1->{#2}{\awsr{#2}}
}

\NewDocumentCommand{\aalt}{O{2} m m}{%
    \alt<#1>{#2}{#3}
}

\newcounter{question}

\newcommand{\startQuestions}{
    \setcounter{question}{2}
}

\newcommand{\iquestion}[2]{
    \item $\question{#1}{#2}$
}

\newcommand{\question}[2]{
    #1 = \palt{\thequestion}{#2}
    \stepcounter{question}
}

% \renewcommand{\question}[2]{
%         #1 = #2
% }


\newcommand{\disableAnimation}{
    % \renewcommand{\question}[2]{
    %     ##1 = ##2
    % }
    
    \renewcommand{\palt}[2]{
        \result{##2}
    }

    \renewcommand{\pause}{}
}

\newcommand{\shortAnimation}{
    \renewcommand{\palt}[2]{
        % \alt<2->{\result{#2}}{\phantom{#2}}
    }
}

\newcommand{\firstSlide}{
    % \renewcommand{\question}[2]{
    %     ##1 =
    % }

    \renewcommand{\palt}[2]{
        \phantom{##2}
        % \pgfinvisible ##2 \pgfshown
    }
}

\newcounter{timer}
\NewDocumentCommand{\qf}{m O{15}}{
    \setcounter{qf}{0}
    \slide{EXERCICES}{\qfSUB{}{
        \begin{itemize}
            \item \large $#2\sec$ par question
            \item \listSize{#1}\thesize{} questions
        \end{itemize}
    }}
    \foreach \q in {#1}{
        \stepcounter{qf}
        \setcounter{choice}{1}
        % Timer slides
        \setcounter{timer}{#2}
        \whiledo{\thetimer>0}{
            \addtocounter{timer}{-1}
            \slide{QUESTIONS FLASH}{
                % \hspace{0.25cm}
                \large \color{Blue}\theqf.\color{black}
                \hspace*{-1cm} \huge \listElement{\q}{0}\\
                \ifthenelse{\boolean{qftimer}}{
                    \vspace{1cm}
                    \transduration{1}
                    \centering
                    \normalsize \color{CadetBlue}$\thetimer\sec$
                }{
                    \transduration{#2}
                    \setcounter{timer}{0}
                }
            }
        }
    }

    \slide{QUESTIONS FLASH}{
        \qfRes{#1}
    }
}

\NewDocumentCommand{\dividePage}{mm O{0.5}}{
    \pgfmathparse{1-#3}
    \begin{columns}[T]
        \begin{column}{#3\textwidth}
            #1
        \end{column}
        \begin{column}{\pgfmathresult\textwidth}
            #2
        \end{column}
    \end{columns}
}

% \documentclass[a4paper, 12pt
% % , landscape
% ]{extarticle}
% \usepackage[top=1.5cm, bottom=2cm, left=2cm, right=2cm]{geometry}
\usepackage[dvipsnames, table]{xcolor}
\usepackage{lastpage}
\usepackage{fancyhdr}
\usepackage{titlesec}
\usepackage{enumitem}
\usepackage{longtable}
\usepackage{pdfpages}
% FANCYHDR

\def\background{
    \fancyhead[L]{
        \begin{tikzpicture}[overlay]
            \fill[gradeColor!65] (-3cm,-\paperheight) rectangle (-1cm,2cm);
        \end{tikzpicture}%
    }
    \fancyhead[R]{
        \begin{tikzpicture}[overlay]
            \node[anchor=north east, font=\fontsize{40}{36}\selectfont] at (1.81cm,0.1cm + 0.55\headheight)
            {\hypersetup{urlcolor=gradeColor!65}\link{\grade}};
        \end{tikzpicture}%
    }
}

\def\emptyBackground{
    \fancyhead[L]{}
    \fancyhead[R]{}
}

\setlength{\headheight}{18pt}
\fancyhead[C]{\normalsize \title}
\background
\fancyfoot[L]{\authors}
\fancyfoot[C]{$\textbf{Page}\;\mathbf{\thepage / {\hypersetup{linkcolor=black}\pageref{LastPage}}}$}
\fancyfoot[R]{\date}

\fancypagestyle{firstpage}{
    \setlength{\headheight}{29pt}
    \fancyhead[C]{\LARGE \title}
}

\def\assignmentNameWidth{7.5cm}
\fancypagestyle{assignment}{
    \setlength{\headheight}{29pt}
    \fancyhead[C]{}
    \fancyhead[L]{\large \title}
    \fancyhead[R]{%
        \begin{tabular}{p{\assignmentNameWidth}p{2.5cm}}%
            \normalsize nom:& \normalsize classe: \link{\grade}\_\\%
            \normalsize prénom:& \normalsize date:\\%
            % \normalsize date:\hspace*{3.5cm}%
        \end{tabular}%
    }
}

\fancypagestyle{empty}{
    \renewcommand{\headrulewidth}{0pt}
    \setlength{\headheight}{-10pt}
    \fancyhead[C]{}
    \fancyhead[R]{}
    \fancyhead[L]{}
    \fancyfoot[L]{}
    \fancyfoot[C]{}
    \fancyfoot[R]{}
}

\fancypagestyle{empty-head}{
    \renewcommand{\headrulewidth}{0pt}
    \setlength{\headheight}{-10pt}
    \fancyhead[C]{}
    \fancyhead[R]{}
    \fancyhead[L]{}
}

\fancypagestyle{assignment-empty-foot}{
    \setlength{\headheight}{29pt}
    \fancyhead[C]{}
    \fancyhead[L]{\large \title}
    \fancyhead[R]{%
        \begin{tabular}{p{\assignmentNameWidth}p{0.15\pdfpagewidth}}%
            \normalsize nom:& \normalsize classe:\\%
            \normalsize prénom:& \normalsize date:\\%
            % \normalsize date:\hspace*{3.5cm}%
        \end{tabular}%
    }
    \fancyfoot[L]{}
    \fancyfoot[C]{}
    \fancyfoot[R]{}
}

\fancypagestyle{small}{
    \setlength{\headheight}{20pt}
    \fancyhead[C]{}
    \fancyhead[C]{\large \title}
    \fancyhead[L]{}
    \fancyhead[R]{}
    \fancyfoot[L]{}
    \fancyfoot[C]{}
    \fancyfoot[R]{}
}

\fancypagestyle{screenread}{
    \fancyhead[C]{}
    \fancyfoot[L]{}
    \fancyfoot[C]{}
    \fancyfoot[R]{}
    % \background
}

\thispagestyle{firstpage}

% \fancyfoot[C]{\textbf{Page 1/1}}

\def\title{\theme}
\def\authors{Jules PESIN}

\pagestyle{fancy}

% \titleformat*{\section}{\small\bfseries}

\titleformat{\section}
{\normalfont\large\bfseries\color{\SectionColor}}{\thesection}{0.6em}{}

\titleformat{\subsection}
{\normalfont\normalsize\bfseries\color{\SubSectionColor}}{\thesubsection}{0.6em}{}

\titleformat{\subsubsection}
{\normalfont\small\bfseries\color{\SubSubSectionColor}}{\thesubsubsection}{0.6em}{}


% \renewcommand{\theenumii}{.\arabic{enumii}}
% \frenchbsetup{StandardItemLabels=true}
\renewcommand{\theenumi}{\small\color{Blue}\arabic{enumi}}
\renewcommand{\labelenumii}{\scriptsize\color{RoyalBlue}\alph{enumii})}
% \renewcommand{\labelitemi}{$\color{Blue}.$}

\newcommand*{\setItemColor}[1]{
    \renewcommand{\theenumi}{\small\color{#1}\arabic{enumi}}
    \renewcommand{\labelenumii}{\scriptsize\color{#1}\alph{enumii})}
    \renewcommand{\labelitemi}{$\color{#1}\blacksquare$}
    \renewcommand{\labelitemii}{$\color{#1}\blacktriangleright$}
    \renewcommand{\labelitemiii}{$\color{#1}\bullet$}
}

\definecolor{gradeColor}{RGB}{200, 50, 50}

\backgroundsetup{
    scale=1,
    color=gradeColor,
    opacity=0.65,
    position=current page.south west,
    angle = 0,
    contents={%
        % Crée une bande colorée sur la gauche
        \begin{tikzpicture}[remember picture, overlay]
            \fill[gradeColor] (0,0) rectangle (1cm, \paperheight);
            % Texte en haut à droite
            \node[shift={(-0.32,-0.25)},anchor=north east, color=gradeColor, font=\fontsize{40}{36}\selectfont] 
                at (current page.north east) {6e};
        \end{tikzpicture}%
    }
}

\newcommand{\emptyBackground}{\backgroundsetup{
    position=current page.south west,
    angle=0,
    scale=1,
    color=gradeColor,
    hshift=0cm,  % No shift
    vshift=0cm,
    contents={}
}}

% BEAMER CONVERSION

% \newcommand{\bchap}[1]{\def\title{Chapitre: #1}}
\newcommand{\bseq}[1]{\def\title{Séquence: #1}}
\newcommand{\bsec}[1]{\section{#1}}
\newcommand{\bsubsec}[1]{\subsection{#1}}

\newcommand{\ssec}{}
\newcommand{\ssubsec}{}

\newcommand{\slide}[2]{#2}

\newcommand{\startQuestions}{}
\newcommand{\iquestion}[2]{\item $#1 = \result{#2}$}

\newcommand{\palt}[2]{\result{#2}}
\NewDocumentCommand{\aalt}{o m m}{%
    \noindent #2\\#3%
}

\newcommand{\disableAnimation}{}
\newcommand{\shortAnimation}{}

\newcommand{\firstSlide}{
    \renewcommand{\iquestion}[2]{\item $##1 = \phantom{##2}$}
    \renewcommand{\palt}[2]{
        \phantom{##2}
    }
}

\newenvironment{columns}[1][T]{}{}
\newenvironment{column}[1]{\begin{minipage}{#1}}{\end{minipage}}

\newcounter{qf}
\NewDocumentCommand{\qf}{m O{15}}{
    \qfSUB{}{
        \qfRes{#1}
    }
}

\newcounter{annex}
\renewcommand{\theannex}{\Alph{annex}} % Define how the annex counter will be displayed

\newcommand{\annex}[1]{%
    \changelocaltocdepth{0}
    \setcounter{section}{0}%
    \setcounter{subsection}{0}%
    \setcounter{subsubsection}{0}%
    \newpage%
    \fancyhead[L]{\color{Red} ANNEXE \theannex}
    \refstepcounter{annex}
    \label{annex:\theannex}
    \input{#1}
    \changelocaltocdepth{2}
}

\newcommand*{\rannex}[1]{
    (\hyperref[annex:#1]{Annexe #1})
}

\newcommand{\changelocaltocdepth}[1]{%
    \addtocontents{toc}{\protect\setcounter{tocdepth}{#1}}%
    \setcounter{tocdepth}{#1}%
}

\usepackage[fontsize=14pt]{fontsize}

\usepackage[T1]{fontenc}
\usepackage[french]{babel}

\usepackage[utf8]{inputenc}
\usepackage{amsmath}
\usepackage{amsthm}
\usepackage{amssymb}
\usepackage{graphicx}
\usepackage{dashundergaps}
\usepackage{array}
\usepackage{multicol}
\usepackage{wrapfig}
\usepackage{numprint}
\usepackage{ulem}
\usepackage{hyperref}
\usepackage{mathrsfs}
\usepackage{mathtools}
\usepackage[many]{tcolorbox}
\usepackage{xparse}
\usepackage{float}
\usepackage{lipsum}
\usepackage{pgf}
\usepackage{ifthen}
\usepackage{caption}
\usepackage{tikz}
\usepackage{xifthen}

% \usepackage[squaren,Gray]{SIunits}

% BREVET

% \usepackage{makeidx}
% \usepackage{fancybox}
% \usepackage{tabularx}
% \usepackage[normalem]{ulem}
% \usepackage{pifont}
% \usepackage{lscape}
% \usepackage{diagbox}
% \usepackage{multirow} 
% \usepackage{textcomp}
% \usepackage{scratch3}
% \usepackage[T1]{fontenc}
% \usepackage{fourier}
% \usepackage[french]{babel}
% \usepackage{pstricks}

% \usepackage[scaled=0.875]{helvet}
% \usepackage{pst-plot,pst-text,pst-tree,pstricks-add}

% fancyhdr

\setlength{\headheight}{18pt}
\fancyhead[C]{\normalsize \title}
% \renewcommand{\headrulewidth}{0pt} % Remove header line
\fancyhead[R]{}
\fancyfoot[L]{\author}
\fancyfoot[C]{\textbf{Page \thepage/\pageref{LastPage}}}
\fancyfoot[R]{\date}

\fancypagestyle{firstpage}{
    \setlength{\headheight}{29pt}
    \fancyhead[C]{\LARGE \title}
    \fancyhead[R]{}
    \fancyfoot[L]{\author}
    \fancyfoot[C]{\textbf{Page \thepage/\pageref{LastPage}}}
    \fancyfoot[R]{\date}
}

\thispagestyle{firstpage}

% \fancyfoot[C]{\textbf{Page 1/1}}

% HYPERREF

\hypersetup{
    colorlinks=true,       % false: boxed links; true: colored links
    linkcolor=red,          % color of internal links (change box color with linkbordercolor)
    citecolor=green,        % color of links to bibliography
    filecolor=magenta,      % color of file links
    urlcolor=blue,          % color of external links
    urlbordercolor=blue,    % borders of external links
    linkbordercolor=red,    % borders of internal links
    pdfborderstyle={/S/U/W 1}% border style will be underline of width 1pt
}

\usepackage[fontsize=14pt]{fontsize}

\usepackage[T1]{fontenc}
\usepackage[french]{babel}
\usepackage[utf8]{inputenc}

\frenchbsetup{StandardItemLabels=true}

% GLOBAL VARIABLES %%%
\graphicspath{{images}}
\def\cwidth{4cm}
\def\tspace{0.5cm}

% BOOLEAN %%%
\newboolean{anwser}
\newboolean{demonstration}
\newboolean{boxedProperties}
\newboolean{showID}
\newboolean{parenthisedID}
\newboolean{animated}
\newboolean{outline}

\setboolean{anwser}{false}
\setboolean{demonstration}{true}
\setboolean{parenthisedID}{true}
\setboolean{showID}{true}
\setboolean{boxedProperties}{false} % false = edge
\setboolean{outline}{false}

\def\DefinitionColor{PineGreen}
\def\PropertyColor{Blue}
\def\TheoremColor{Plum}

\def\SectionColor{Red}
\def\SubSectionColor{Green}

\setboolean{animated}{true}

% \DeclareMathOperator{\PGCD}{PGCD}
% \DeclareMathOperator{\PPCM}{PPCM}

\DeclareMathOperator{\sh}{sh}
\DeclareMathOperator{\ch}{ch}
% \DeclareMathOperator{\th}{th}

\DeclareMathOperator{\argsh}{argsh}
\DeclareMathOperator{\argch}{argch}
\DeclareMathOperator{\argth}{argth}
\DeclareMathOperator{\I}{I}
\DeclareMathOperator{\Id}{Id}
\DeclareMathOperator{\Ker}{Ker}
% \DeclareMathOperator{\dl}{o}
\newcommand{\dl}[1]{
    \operatorname*{o}_{#1}
}

\def\deg{\ensuremath{^\circ}}
\def\prll{\mathbin{\!/\mkern-5mu/\!}}
\renewcommand{\parallel}{\mathbin{\!/\mkern-5mu/\!}}

\def\octet{\textrm{o}}
\def\byte{\textrm{B}}

\def\hour{\textrm{h}}
\def\minute{\textrm{min}}
\def\second{\textrm{s}}
% ENVIRONMENT
\newenvironment{mysection}[1][gray!20]{%
    \begin{sectionBox}[#1]
}{%
    \end{sectionBox}
}

\newenvironment{mysubsection}[1][gray!20]{%
    \begin{subsectionBox}[#1]
}{%
    \end{subsectionBox}
}

% Switch implementation
\newboolean{default}
\newcommand{\case}{}
\newcommand{\default}{}

\newenvironment{switch}[1]{%
    \setboolean{default}{true}
    \renewcommand{\case}[2]{\ifthenelse{\equal{#1}{##1}}{%
        \setboolean{default}{false}##2}{}}%
    \renewcommand{\default}[1]{\ifthenelse{\boolean{default}}{##1}{}}
}{}

% SECTIONS
\input{header/command/sections.tex}

% ANSWERS
\newlength{\parline}
\newlength{\paroutindent}
\newlength{\lineheight}
\setlength{\lineheight}{\heightof{abcdefghijklmnoprstuvwxyz}}

\newcommand{\countlines}[1]{%
    \setlength{\paroutindent}{\expandafter\parindent}
    \setlength{\parline}{\heightof{\noindent\begin{minipage}{\linewidth}%
                \setlength{\parindent}{\paroutindent}#1\end{minipage}}}%
    \pgfmathparse{round(\parline / (0.9*\lineheight))}
    \newcount\linecount
    \pgfmathsetcount{\linecount}{\pgfmathresult}
}

\newcommand{\looptext}[2]{%
    \noindent
    \newcount\printcount
    \printcount=#2
    \loop
        #1
        \advance\printcount by -1
        \ifnum\printcount>0
    \repeat
}

\newcommand{\awsr}[1]{%
    \ifthenelse{\boolean{answer}}{
        \result{#1}
    }{
        \countlines{#1}
        \pgfmathsetcount{\linecount}{\linecount + 1}
        \noindent\hspace{-9pt}
        \looptext{
            \noindent\dotfill
    
        }{\the\linecount}
    }
}

\newcommand{\dottedLines}[1]{%
    \noindent\hspace{-9pt}%

    \looptext{%
        \noindent\dotfill%

    }{#1}
}

\newcommand{\result}[1]{\color{OrangeRed}#1\color{black}}

% MATH
\input{header/command/math.tex}

% IMAGES
\input{header/command/image.tex}

% COMMANDS

\newcommand{\fsize}[1]{\fontsize{#1}{#1}\selectfont}

\NewDocumentCommand{\ifNotNull}{mmo}{
    \IfValueT{#1}{
        \ifx\relax#1\relax
            \IfValueT{#3}{
                #3
            }
        \else
            #2
        \fi
    }
}

\NewDocumentCommand{\ilink}{m g}{%
    \item
    \IfValueTF{#2}{\link{#1}{#2}}{\link{#1}}
}

\NewDocumentCommand{\link}{m g}{%
    \csn{#1}%
    \IfValueT{#2}{(#2)}%
}

\NewDocumentCommand{\TODO}{g}{%
    {\color{Red} $\rightarrow$ \textbf{TODO}
    \IfValueT{#1}{(#1)}}
    % \color{black}
}

\newcommand{\leconInfoBox}[2]{
    \textbf{#1 :}\vspace{-0.25cm}
        \begin{multicols}{2}
            \begin{itemize}[label=$\blacktriangleright$, font = \small \color{Red}]
                #2
            \end{itemize}
        \end{multicols}
        \vspace{-0.4cm}
}

% TCOLORBOX

\input{header/command/tcolorbox.tex}

\NewDocumentCommand{\leconInfo}{mooo}{
    \begin{infoBox}
        \leconInfoBox{Niveaux}{#1}
        \ifNotNull{#2}{
            \tcbline
            \leconInfoBox{Prérequis}{#2}
        }
        \ifNotNull{#3}{
            \tcbline
            \leconInfoBox{Thèmes}{#3}
        }
        \ifNotNull{#4}{
            \tcbline
            \textbf{Motivation :} 
            #4
        }
    \end{infoBox}
}

\NewDocumentCommand{\seanceInfo}{oooooooo}{
    \begin{infoBox}
        \vspace{-0.05cm}
        \begin{tcbitemize}[raster rows=1,raster columns=20,raster height=1.65cm,
            raster every box/.style={colframe=red!50!black,colback=red!10!white}]
            \tcbitem[raster multicolumn=6] \textbf{Date :} #1
            \tcbitem[raster multicolumn=10] \textbf{Séquence :} #2
            \tcbitem[raster multicolumn=4] \textbf{Séance :} #3
        \end{tcbitemize}
        \vspace{-0.25cm}
        \ifNotNull{#4}{\tcbline \textbf{Objectif :} #4}
        \ifNotNull{#5}{\tcbline \leconInfoBox{Classe(s)}{#5}}
        \ifNotNull{#6}{\tcbline \leconInfoBox{Prérequi(s)}{#6}}
        \ifNotNull{#7}{\tcbline \textbf{Séance précédente :} #7}
        \ifNotNull{#7}{\tcbline \leconInfoBox{Matériel(s)}{#8}}
    \end{infoBox}
}

\def\pDscr{\tcbitem[enhanced jigsaw, breakable,
    raster multicolumn=6]
}
\def\pMdlt{\tcbitem[enhanced jigsaw, breakable,
    raster multicolumn=11]
}
\def\pTime{\tcbitem[enhanced jigsaw, breakable,
    raster multicolumn=3, halign=center]
}

\newcommand{\prepRow}[3]{
    \tcbitem[raster multicolumn=20]
    \tcblower

    \pDscr #1
    \pMdlt #2
    \pTime #3
}

\newcommand{\prepTable}[1]{
    \begin{prepBox}
        \begin{tcbitemize}[enhanced jigsaw, breakable, raster rows=1,raster columns=20,raster height=1.1cm, halign=center,
            raster every box/.style={enhanced jigsaw, breakable, colframe=Blue!50!black,colback=Blue!10!white}]
            \pDscr \textbf{Descriptif}
            \pMdlt \textbf{Modalité}
            \pTime \textbf{Durée}
        \end{tcbitemize}
        \begin{tcbitemize}[enhanced jigsaw, breakable,
            raster equal height = rows, 
            raster columns=20, frame hidden,
            raster every box/.style={
                enhanced jigsaw, breakable,
                opacityback=0, valign=top, 
                size = tight
            }]
            #1
        \end{tcbitemize}
    \end{prepBox}
}

% TIKZ

\newcommand{\ctikz}[1]{
    \begin{center}
        \begin{tikzpicture}
            #1
        \end{tikzpicture}
    \end{center}
}

\newcommand{\axis}[1]{%Draw coordinate axes
    \draw[thin, -Stealth] (-0.5,0) -- (#1,0);% node[right] {$x$}; % x-axis
    \draw[thin, -Stealth] (0,-0.5) -- (0,#1);% node[above] {$y$}; % y-axis
}

\newcommand{\drawGrid}[3]{
    \foreach \n in {0,...,#1}
        \draw[line width = #3] (\n,0) -- (\n,#2);
    \foreach \n in {0,...,#2}
        \draw[line width = #3] (0,\n) -- (#1,\n);
}

\newcommand{\drawPoint}[4]{
    \node[shift={#4}, color = \pointColor] at (#2 - 0.5,#3 - 0.5) {#1};
    \draw[line width = \crossWidth, shift={#4}, color = \pointColor] (#2 - 0.25,#3) -- (#2 + 0.25,#3);
    \draw[line width = \crossWidth, shift={#4}, color = \pointColor] (#2,#3 - 0.25) -- (#2,#3 + 0.25);
}

% Tabular
\newcolumntype{C}[1]{>{\centering\arraybackslash}p{#1}}
\newcolumntype{M}[1]{>{\centering\arraybackslash}m{#1}}
\newcolumntype{K}{@{}m{0pt}@{}}

% GEOMETRY

% \newcommand{\restoregeometry}{def}

\newcommand{\multiColItemize}[2]{
    \begin{multicols}{#1}
        \begin{itemize}
            #2
        \end{itemize}
    \end{multicols}
}

\newcommand{\multiColEnumerate}[2]{
    \ifthenelse{\isequivalentto{#1}{1}}{
        \begin{enumerate}
            #2
        \end{enumerate}
    }{
        \begin{multicols}{#1}
            \begin{enumerate}
                #2
            \end{enumerate}
        \end{multicols}
    }
}

\makeatletter
\newcommand\pgfinvisible{\pgfsys@begininvisible}
\newcommand\pgfshown{\pgfsys@endinvisible}
\makeatother

\renewcommand*{\phantom}[1]{
    \pgfinvisible #1 \pgfshown
}

\newcounter{size}
\newcommand{\listSize}[1]{%
    \setcounter{size}{0}%
    \foreach \n in {#1}{\stepcounter{size}}%
    % \thesize
}

\newcounter{elemPos}
\newcommand{\listElement}[2]{
    \setcounter{elemPos}{0} % Start counting from 1
    \def\resultVal{0} % Default value
    \renewcommand*{\do}[1]{%
        \ifnumequal{\value{elemPos}}{#2}{%
            \def\resultVal{##1}%
            \listbreak% Break out of the loop
        }{}%
        \stepcounter{elemPos}%
    }
    % \docsvlist{#1}
    \expandafter\docsvlist\expandafter{#1} % Expand the list before passing it to \docsvlist
    \resultVal
}

% \NewDocumentCommand{\exoslide}{m O{10cm}}{
%     \slide{}{
%         \img{\imgf{#1}}[#2]
%     }
% }

\NewDocumentCommand{\exoSlide}{m O{10cm} O{1} O{} O{exo}}{%
    \slide{#5}{%
        \ifthenelse{\equal{#3}{1}}{\vspace{-0.5cm}}{\vspace{-1cm}}
        \def\exercices{\foreach \q in {#1}{\imgp{\q}[#2]\vspace{-0.5cm}}}
        \exo{#1}{\wideFrame[7em]{\bvspace{0.25cm}\avspace{-0.25cm}
            \ifthenelse{\equal{#3}{1}}{\exercices}
            {\begin{multicols}{#3}\exercices\end{multicols}}}
            \avspace{0.75cm}
        }[#4]
    }
}

\NewDocumentCommand{\exoList}{m O{} O{3}}{%
    \section*{Exercices}%
    \slide{EXERCICES}{
        \exo{#2}{
            \vspace{-0.25cm}
            \multiColEnumerate{#3}{
                \foreach \q in {#1}{
                    \item \q
                }
            }
        }
    }
}

\newcommand{\questions}[1]{
    \begin{enumerate}
        \foreach \q in {#1}{
            \item \q\\
            \vspace*{-0.45cm}
            \dottedLines{3}
        }
    \end{enumerate}
}

% Define a new boolean for checking if the section is starred
\newboolean{section@star}

\makeatletter
% Redefine \section and \section* to set the boolean
\let\old@section\section
\renewcommand{\section}{%
    \@ifstar
        {\setboolean{section@star}{true}\old@section*}
        {\setboolean{section@star}{false}\old@section}%
}
\makeatother

\newcommand{\qt}[1]{«\textit{#1}»}

\newcommand{\calc}[1]{\numexpr#1\relax}
\newcommand{\ncalc}[1]{\number\calc{#1}}
\newcommand{\pcalc}[1]{\numprint{\ncalc{#1}}}

\newcommand{\setgrade}[1]{
    \def\grade{#1}
    % \begin{switch}{#1}
    %     \case{6e}{\global\definecolor{gradeColor}{hex}{FA8072}}
    %     \default{
    %         Default
    %         \global\definecolor{gradeColor}{RGB}{200, 50, 50}
    %     }
    % \end{switch}
    \ifthenelse{\equal{#1}{6e}}{
        \definecolor{gradeColor}{HTML}{C6233D} % FA8072 in hex
    }{
    \ifthenelse{\equal{#1}{5e}}{
        \definecolor{gradeColor}{HTML}{088255}
    }{
    \ifthenelse{\equal{#1}{4e}}{
        \definecolor{gradeColor}{HTML}{1466A8}
    }{
    \ifthenelse{\equal{#1}{3e}}{
        \definecolor{gradeColor}{HTML}{844499}
    }{
        \definecolor{gradeColor}{RGB}{0, 0, 0}
    }}}}
}

\gdef\phase{}
\newcommand{\setPhase}[1]{%
    \begin{switch}{#1}
        \case{exo}{\gdef\phase{EXERCICES}}
        \case{cr}{\gdef\phase{COURS}}
        \case{qf}{\gdef\phase{QUESTIONS FLASH}}
        \case{dm}{\gdef\phase{DEVOIR MAISON}}
        \default{\gdef\phase{#1}}
    \end{switch}
}

\newcounter{savedenumi}
\setcounter{savedenumi}{0}
\xdef\savedenumi{0}
% \newcommand{\saveenumi}{
%     % \xdef\savedenumi{\calc{\theenumi-1}}
%     \setcounter{savedenumi}{0}
% }

\newcommand{\saveenumi}[1]{
    \setcounter{savedenumi}{#1}
}

\newcommand{\loadenumi}{
    \setItemColor{\currentColor}
    \setcounter{enumi}{\thesavedenumi}
}

\newcommand\csn[1]{\csname #1\endcsname}

\newcommand{\vect}[1]{\ensuremath{\overrightarrow{#1}}}
% \newcommand{\vect}[1]{\overrightarrow{\,\mathstrut#1\,}}
\newcommand{\m}[1]{\ensuremath{\mathbf{#1}}}
\newcommand\lm[2]{\lim_{#1\to#2}}

\def\eqv{\Leftrightarrow}
\def\ssi{si et seulement si }
\def\pt{pour tout }
\def\poly2{fonction polynôme du second degré }
\def\eq2{équation second degré }
\def\discr{b^2-4ac}

% MATH TEXT
\def\et{\textrm{ et }}
\def\si{\textrm{ si }}
\def\avec{\textrm{ avec }}
\def\car{\textrm{ car }}
\def\alors{\textrm{ alors }}
\def\ou{\textrm{ ou }}
\def\ona{\textrm{ on a }}

\def\iet{\shortintertext{et}}
\def\ialors{\shortintertext{alors}}
\def\idou{\shortintertext{d'où}}
\def\ior{\shortintertext{or}}
\def\iona{\shortintertext{on a}}

\def\studentinfo{
    \vspace*{-1cm}
    \begin{minipage}{0.35\linewidth}
        nom: \dotfill
    \end{minipage}
    \begin{minipage}{0.35\linewidth}
        prénom: \dotfill
    \end{minipage}
    \begin{minipage}{0.15\linewidth}
        classes: \dotfill
    \end{minipage}
    
    \noindent\hrulefill
}

% UNITS
\def\cm{\,\centi\meter}
\def\km{\,\kilo\meter}
\newcommand{\defl}[2]{%
    \expandafter\def\csname #1\endcsname{\href{#2}{#1}\space}%
}

% Page Eduscol
\defl{Eduscol Cycle 3}{https://eduscol.education.fr/251/mathematiques-cycle-3}
\defl{Eduscol Cycle 4}{https://eduscol.education.fr/280/mathematiques-cycle-4}
\defl{Eduscol Lycée Général et technologique}{https://eduscol.education.fr/1723/programmes-et-ressources-en-mathematiques-voie-gt}
\defl{Eduscol Lycée Professionnel}{https://eduscol.education.fr/1793/programmes-et-ressources-en-mathematiques-voie-professionnelle}

% Repères annuels
\defl{Cycle 3}{https://eduscol.education.fr/document/14026/download}
\defl{Cycle 4}{https://eduscol.education.fr/document/14080/download}

% Attendus de fin d'année
\defl{5e}{https://eduscol.education.fr/document/14044/download}
\defl{4e}{https://eduscol.education.fr/document/14056/download}
\defl{3e}{https://eduscol.education.fr/document/14068/download}

% Programme de mathématiques
\defl{cycle 3}{https://eduscol.education.fr/document/50990/download}
\defl{cycle 4}{https://cache.media.education.gouv.fr/file/31/89/1/ensel714_annexe3_1312891.pdf}
\defl{2nd}{https://eduscol.education.fr/document/24553/download}
\defl{1re}{https://eduscol.education.fr/document/24565/download}
\defl{1re STL}{https://eduscol.education.fr/document/23098/download}
\defl{1re STI2D}{https://eduscol.education.fr/document/24919/download}
\defl{Terminale Option Spécialité}{https://eduscol.education.fr/document/24568/download}
\defl{Terminale Option Complémentaire}{https://eduscol.education.fr/document/24571/download}
\defl{Terminale Option Expertes}{https://eduscol.education.fr/document/24574/download}
\defl{Terminale STL}{https://eduscol.education.fr/document/23107/download}
\defl{Terminale STI2D}{https://eduscol.education.fr/document/24922/download}
% Ressources thématiques
\defl{Proportionnalité}{https://eduscol.education.fr/document/17281/download}
\defl{Probabilités}{https://eduscol.education.fr/document/17275/download}
\defl{Fonctions}{https://eduscol.education.fr/document/17287/download}
\defl{Traitement des données}{https://eduscol.education.fr/document/17269/download}

\defl{Fonctions}{https://eduscol.education.fr/document/17287/download}
\defl{Fractions}{https://eduscol.education.fr/document/17239/download}
\defl{Nombres relatifs}{https://eduscol.education.fr/document/17245/download}
\defl{Puissances}{https://eduscol.education.fr/document/17251/download}
\defl{Divisibilité et nombres premiers}{https://eduscol.education.fr/document/17257/download}
\defl{Calcul littéral}{https://eduscol.education.fr/document/17263/download}

\defl{Grandeurs et mesures}{https://eduscol.education.fr/document/17293/download}
\defl{Algorithmique et programmation}{https://eduscol.education.fr/document/17311/download}

\defl{Suites}{https://eduscol.education.fr/document/24586/download}
\defl{Produit Scalaire}{https://eduscol.education.fr/document/24589/download}
\defl{Raisonnement et démonstration (seconde)}{https://eduscol.education.fr/document/24580/download}
\defl{Raisonnement et démonstrations (première)}{https://eduscol.education.fr/document/24583/download}

\captionsetup{labelformat=empty,labelsep=none}

% \setboolean{boxedProperties}{true} % false = edge
% \setboolean{parenthisedID}{false}
% \setboolean{showID}{false}

% \def\DefinitionColor{Red}
\def\PropertyColor{Red}
\def\TheoremColor{Red}

% TIKZ
\def\crossWidth{0.25mm}
\def\pointColor{blue}

\begin{document}

\ifBeamer{%
    \renewcommand*{\theenumii}{\alph{enumii}}

    \firstSlide
    \setboolean{showRef}{false}
}

\ifArticle{%
    \renewcommand*{\theenumii}{\alph{enumii}}
    
    \disableAnimation
}



% DOCUMENTS

% % VARIABLES %%%
\def\authors{\jules}
% \date{\today}
\def\longTitle{Géometrie plane - points et droites}
\def\shortTitle{\MakeUppercase{\longTitle}}
% \bseq{\longTitle}
% \def\theme{\longTitle}

\setboolean{showRef}{false}

\def\dim{Dimension 6e 2016}

\def\imgPath{enseignement/6e/geometrie-plane/points-et-droites/}
\def\imgExtension{.png}

%%

% Yvan Monka : https://www.maths-et-tiques.fr/telech/19Para_Perp.pdf

\ifArticle{\vspace*{0.1cm}}

\scn{Objet géométrique}{}

\bseq{\longTitle}
\bsec{Objet géométrique}
\bsubsec{Le point}

\slide{COURS}{
    \ssec\ssubsec
    \vc{}{
        On nomme \key{point} est le plus petit élément géométrique.
        Il est infiniment petit,
        tant qu'il n'a pas de dimension.
    }
}

\slide{EXERCICES}{
    \act{}{
        Représenter un point.
    }
}

\slide{COURS}{
    \rmk{}{
        On représente le point par une croix
    }

    \df{}{
        Deux points sont dits distincts s'ils ne sont pas confondus
    }

    \expl{}{
        Les points $A$ et $B$ sont confondus et les points $A$ et $C$ sont distincts.
    }
}

\bsubsec{La droite}
\slide{COURS}{
    \ssubsec

    \vc{}{On nomme \key{droite} un tracé rectiligne infini.}
}

\slide{EXERCICES}{
    \act{}{Représenter une droite passant par deux points $A$ et $B$}
}

\slide{COURS}{
    \rmk{}{On ne peut pas représenter une droite entièrement}

    \axio{}{
        Il existe une unique droite passant par deux points.
    }

    \expl{}{
        Pour deux points $D$ et $E$ disctincts.
        On peut noter la droite qui passe par $D$ et $E$, $(DE)$ ou $(ED)$.
    }
}

\bsubsec{Le segment et la demi-droite}

\slide{COURS}{
    \ssubsec

    \df{}{Un \key{segment} est une portion de droite,
    limité par deux \key{extrémités}
    }

    \expl{}{
        Pour deux points $D$ et $E$ disctincts.
        Le segment reliant $D$ et $E$ est le chemin le plus court entre ces deux points.
        On le note $[DE]$.
    }
}


\slide{COURS}{
    \ssubsec

    \df{}{Une \key{demi-droite} est une portion de droite,
    limité par une seule extrémité, son \key{origine}.
    }

    \expl{}{
        Pour deux points $D$ et $E$ disctincts.
        On peut noter, la demi-droite d'origine $D$ passant par $E$, $[DE)$.
    }
}

\scn{Alignement}{}

\slide{QUESTIONS FLASH}{%
    \sqf Comment peut-on nommer la droite ci-dessous?
    \vspace*{-0.5cm}
    \imgp{qf1}[5cm]
    \qfs $(AB)$ \hfill \qfs $[AB]$ \hfill \qfs $(BA)$ \hfill \qfs $(d)$ \hfill \qfs $AB$ \hfill
    \vspace*{0.5cm}
    \\ \hint{plusieurs réponses sont attendues}
}

\slide{}{
    \sqf Quels points semblent alignés?
    \vspace*{-0.5cm}
    \imgp{qf2}[5cm]
    \qfs $A$ et $F$ \hfill \qfs $A$,$D$ et $F$ \hfill \qfs $A$,$C$ et $D$ \hfill \qfs $C$,$A$ et $D$ \hfill
    \vspace*{0.5cm}
    \\ \hint{plusieurs réponses sont attendues}
}

\bsec{Alignement}

\slide{EXERCICES}{
    \vspace*{-0.5cm}
    \act{1 p208}{%
        \vspace*{-1cm}
        \imgp{dim-6e-act-1-p208}[10cm]
    }[\dim]
}

% \bsubsec{Alignement}
\slide{COURS}{
    \sseq\ssec%

    \df{}{
        Des points sont dit \key{alignés} s'il existe une droite passant par tous ces points.
    }
}

\slide{}{
    \vspace*{-0.45cm}
    \expl{}{
        \dividePage{
            \imgp{alignement}[4cm]
        }{
            Les points $A$, $C$ et $B$ sont alignés.\\
            Les points $A$, $D$ et $B$ ne sont pas alignés.
        }[0.35]
    }
    \vspace*{-1.2cm}
    \rmk{}{
        \begin{enumerate}
            \item Deux points peuvent toujours être reliés par une droite et sont donc toujours alignés.
            \item Dans l'exemple précédent ; C est placé sur la droite $(AB)$. On dit que $C$ appartient à la droite $(AB)$ et on note : $C\in(AB)$.
        \end{enumerate}
        
    }
}

% \exoList{5 p211,7 p211}[][3]

\slide{EXERCICES}{
    \exo{p211}{}
    \vspace*{-1cm}
    \imgp{dim-6e-exo-5-p211}[10cm]
    \imgp{dim-6e-exo-7-p211}[10cm]
}

\scn{Perpendicularité et parallèlisme}{}

\bsec{Droites}
\bsubsec{Definitions}

\setcounter{qf}{0}
\slide{QUESTIONS FLASH}{
    \sqf Les droites $(f)$ et $(g)$ semblent être:
    \vspace*{-0.5cm}
    \imgp{qf3}[4cm]
    \qfs Sécantes \hfill \qfs Parallèles \hfill \qfs Perpendiculaires \hfill
    \vspace*{0.5cm}
    \\ \hint{plusieurs réponses sont attendues}
}

\slide{}{
    \sqf Quelles droites semblent parallèles :
    \vspace*{-0.5cm}
    \imgp{qf4}[5cm]
    \qfs $(AB) \et (CE)$ \hfill \qfs $(AB) \et (DC)$ \hfill \qfs $(CD) \et (EB)$ \hfill
    \vspace*{0.5cm}
    % \\ \hint{plusieurs réponses sont attendus}
}

\slide{COURS}{
    \ssec\ssubsec%
    %
    \df{}{
        Deux droites sont dites \key{sécantes} si elles se coupent en un unique point.
    }
    \vspace*{-1cm}
    \expl{}{
        \dividePage{
            \imgp{secantes}[4cm]
        }{
            Les droites $(AE)$ et$(BD)$ sont sécantes.\\ $C$ est leur point d'intersection.
        }[0.45]
    }[\myl{https://biblio.manuel-numerique.com?openBook=9782047392935\%3FY29udGV4dGVSZXNvdXJjZT17InR5cGUiOiJhcnRpY2xlIiwiaWRyZWYiOiJpZF9DaGFwdGVyXzAxMl9ab29tX0dyYXBoaWNfNTUxX1NDUl94aHRtbCIsImFydGljbGVUeXBlIjoiem9vbSJ9}]
}

\slide{}{
    \df{}{
        Deux droites sont dites \key{perpendiculaires} si elles sont sécantes et leur intersection forme un angle droit.
    }
    \vspace*{-1cm}
    \expl{}{
        \dividePage{
            \imgp{perpendiculaires}[4cm]
        }{
            Les droites $(EF)$ et$(GF)$ sont perpendiculaires.\\ On note $(EF) \perp (GF)$.
        }[0.35]
    }[\myl{https://biblio.manuel-numerique.com?openBook=9782047392935\%3FY29udGV4dGVSZXNvdXJjZT17InR5cGUiOiJhcnRpY2xlIiwiaWRyZWYiOiJpZF9DaGFwdGVyXzAxMl9ab29tX0dyYXBoaWNfNTUyX1NDUl94aHRtbCIsImFydGljbGVUeXBlIjoiem9vbSJ9}]
}

\slide{}{
    \df{}{
        Deux droites sont dites \key{parallèles} si elles ne sont pas sécantes.
    }
    \vspace*{-1cm}
    \expl{}{
        \dividePage{
            \imgp{paralleles}[4cm]
        }{
            Les droites $(d)$ et$(d')$ sont parallèles.\\ On note $(d) \parallel (d')$.
            \rmk{}{Deux droites parallèles conservent le même écartement}
        }[0.3]
    }[\myl{https://biblio.manuel-numerique.com?openBook=9782047392935\%3FY29udGV4dGVSZXNvdXJjZT17InR5cGUiOiJhcnRpY2xlIiwiaWRyZWYiOiJpZF9DaGFwdGVyXzAxMl9ab29tX0dyYXBoaWNfNTUzX1NDUl94aHRtbCIsImFydGljbGVUeXBlIjoiem9vbSJ9}]
}

\slide{EXERCICES}{
    \vspace*{-0.5cm}
    \exo{11 p213}{
        \vspace*{-0.75cm}
        \imgp{dim-6e-exo-11-p213}[6.25cm]
    }
}

\scn{Construction}{}

\bsubsec{Construction}

\slide{}{
    \act{}{
        \begin{enumerate}
            \item Placer trois points $A,B$ et $C$.
            \item Tracer $(AB)$.
            \item Tracer une droite perpendiculaire à $(AB)$ passant par $C$.
        \end{enumerate}
    }
}

\slide{COURS}{
    \dividePage{
        \mthd{}{
            \imgp{construction-perpendiculaire}[5cm]
        }
    }{
        \expl{}{Construire la droite perpendiculaires à $(d)$ passant par $A$.
        \imgp{construction}[5cm]}
    }
}

\slide{EXERCICES}{
    \act{}{
        \begin{enumerate}
            \item Placer trois points $A,B$ et $C$.
            \item Tracer $(AB)$.
            \item Tracer une droite parallèle à $(AB)$ passant par $C$.
        \end{enumerate}
    }
}

\slide{COURS}{
    \dividePage{
        \mthd{}{
            \imgp{construction-parallele}[5cm]
        }
    }{
        \expl{}{Construire la droite parallèles à $(d)$ passant par $A$.
        \imgp{construction}[5cm]}
    }
}

\slide{EXERCICES}{
    \vspace*{-0.5cm}
    \exo{14 p213}{
        \vspace*{-0.75cm}
        \imgp{dim-6e-exo-14-p213}[10cm]
    }
}

\scn{Propriétés}{}

\bsec{Propriétés}
\slide{}{
    \ssec
    \pr{}{}
}

\slide{}{
    \pr{}{}
}

\slide{}{
    \pr{}{}
}
% % VARIABLES %%%
\def\authors{\jules}
% \date{\today}
\def\longTitle{Nombres Relatifs - Repérage et comparaison}
\def\shortTitle{\MakeUppercase{\longTitle}}
\bseq{\longTitle}
% \def\theme{\longTitle}

\setgrade{5e}

\def\imgPath{enseignement/5e/nombres-relatifs/reperage-et-comparaison/}
\def\imgExtension{.png}

\setboolean{showAllPhases}{true}
% \def\showPhase{EXERCICES}
%%

% Yvan Monka : https://www.maths-et-tiques.fr/telech/19Nomb_rel1.pdf

% \article{\vskip}
\ifArticle{\vspace*{0.1cm}}

\obj{
    \item Introduire les nombres relatifs.
    \item Repérer sur une droite graduée les nombres décimaux relatifs.
    \item Repérer sur une droite graduée et dans le plan muni d'un repère orthogonal.
    \item Utiliser la notion d'opposé.
}

\scn{Introduction aux nombres négatifs}

\bsec{Définitions}

\slide{exo}{
    \vspace*{-0.5cm}
    \act{}{
        \dividePage{
            \imgp{carte}[6cm]
        }{
            \begin{enumerate}
                \item Quelles informations nous apporte ce document?
                \item Classer les nombres en deux catégories et donner un nom à chaque catégorie.
                \item Ranger les températures par ordre croissant.
            \end{enumerate}
        }
    }
    % [\href{https://www.facebook.com/groups/994675223903586/search/?q=activite\%20m%C3\%A9t\%C3\%A9o\%20nombres\%20relatifs\&locale=fr\_FR}{Vanessa Cazier}]
}

\scn{Institutionnalisation des nombres relatifs}{}

\slide{cr}{
    \sseq\ssec
    \df{}{
        Un nombre est dit :
        \begin{itemize}
            \item \key{positif} si il est supérieur ou égal à zéro
            \item \key{négatif} si il est supérieur ou égal à zéro
        \end{itemize}
    }
}

\slide{cr}{
    \rmk{}{
        \begin{itemize}
            \item $O$ est à la fois positif et négatif
            \item On nomme nombres relatifs l'ensemble des nombres positifs et négatifs.
            \item On parle de nombres relatifs car on les considère relativement à zéro.
        \end{itemize}
    }
}

\def\imgPrefix{mm-c4/exo-}
\exoSlide{28p196,29p196,30p196}[9cm][1][\mm]
\def\imgPrefix{}

\scn{Résolution d'équations grâce aux nombres relatifs}{}

\slide{qf}{%
    \sqf Comparer à l'aide du signe $>$ ou $<$ ou $=$ les nombres :
    \begin{align*}
        &\qfs \; 10 \hole 10,075\\
        &\qfs \; 0,5 \hole \frac{1}{2}\\
        &\qfs \; \frac{6}{10} \hole \frac{6}{9}
    \end{align*}
}

\slide{cr}{
    \sqf Completer les opérations à trous :
    \begin{align*}
        &\qfs \; 6 + \hole = 18\\
        &\qfs \; 710,5 + \hole = 770\\
        &\qfs \; 30 - \hole = 22,2
    \end{align*}
}

\slide{exo}{
    \act{}{
        \begin{enumerate}
            \item $ 5 + \textrm{?} = 13$ : Trouve le nombre représenté par « ? »
            \begin{itemize}
                \item On pourrait écrire : $5 + \icon{sun} = 13$ \\
                La question deviendrait alors :
                Trouve le nombre représenté par « \icon{sun} ».
                \item On pourrait encore écrire : $5 + x = 13$ \\
                La question deviendrait alors :
                Trouve le nombre représenté par « $x$ ».
            \end{itemize}
        \end{enumerate}
    }[\href{https://clairelommeblog.fr/2020/10/11/introduire-les-relatifs-en-5e/}{Claire Lomné}]
}


\slide{exo}{
    \begin{enumerate}
        \setcounter{enumi}{1}
        \item Ces égalités sont des équations car elles contiennent une \key{inconnue}.
        \begin{itemize}
            \item Tu les appelais aussi des « opérations à trou ».
            \item Pour les résoudre, tu te demandes combien vaut $13-5$ ?
        \end{itemize}
    \end{enumerate}

}

\slide{exo}{
    \begin{enumerate}
        \setcounter{enumi}{2}
        \item Résout les équations suivantes :
        \multiColEnumerate{3}{
            \item $10 + x = 21$ \item $x + 6 = 13$\item $97 + y = 100$
            \item $19 = 3 + t$ \item $13+x = 13$ \item $10 + x = 4$
        }
        \item De même :
        \multiColEnumerate{2}{
            \item $x + 15 = 10$ \item $50 + y = 40$ \item $t + 19 = 0$
            \item $a + 7 = 1$
        }
    \end{enumerate}
}

% \slide{EXERCICES}{
%     \act{}{
%         Exprimer un résultat aux opérations suivante :
%         \multiColEnumerate{3}{
%             \item $9 - 3$ \item $0 - 5$ \item $10 - 12$
%             \item $0 - 9,3$ \item $100 - 250$ \item $\frac{1}{5} - \frac{3}{5}$
%         }
%     }
% }

\slide{cr}{
    \hist{}{
        L'invention des nombres négatifs est souvent attribuée à Brahmagupta,
        mathématicien et astronome indien du VIe siècle.\\
        Il les introduit dans le but de résoudre des opérations autrement impossibles.
        Dans les cadres de calculs dettes notamment.
    }
    \expl{}{
        Brahmagupta possède $2\euro$ et achète $4$ baguettes à $1,5\euro$ l'unité.
        Quelle quantité d'argent a-t-il après son achat ?
    }
}

\slide{cr}{
    % \df{}{
    %     L'\key{opposé} d'un nombre $n$ est le nombre qui, lorsqu'il est ajouté à $n$, donne $0$.
    % }[\href{https://fr.wikipedia.org/wiki/Opposé}{Wikipédia}]

    \df{}{
        Deux nombres dont la somme est égale à 0 sont dits \key{opposé}.
    }

    \expl{}{
        \begin{itemize}
            \item L'opposé de $7$ est \palt{2}{$-7$}.
            \item L'opposé de $-18,6$ est \palt{2}{$18,6$}.
            \item L'opposé d'un nombre relatif $b$ est \palt{2}{$-b$}.
        \end{itemize}
    }
}

\def\imgPrefix{mm-c4/exo-}
\exoSlide{46p197,47p197,48p197}[9cm][1][\mm]
\def\imgPrefix{}

\scn{Repérage sur une droite graduée}{}

\slide{qf}{%
    \setcounter{qf}{0}
    \sqf Ranger les nombres suivant dans l'ordre croissant les nombres :
    \begin{align*}
        1,2 \pv 6 \pv 1,15 \pv 2 \pv 100 \pv 0,584 \pv 2
    \end{align*}
}

\slide{qf}{%
    \sqf Donner dans chaque cas l'abscisse du point $P$:\\
    \noindent \qfs \imgp{qf-reperage-droite-a}[6cm]
    \noindent \qfs \imgp{qf-reperage-droite-b}[9cm]
    \noindent \qfs \imgp{qf-reperage-droite-c}[6cm]
}

\bsec{Repérage}
\bsubsec{Repérage sur une droite}

\slide{exo}{
    \act{}{
        \begin{enumerate}
            \item Tracer une droite numérique sur laquelle placer les points :
            $O$ d'abscisse 0, $A$ d'abscisse $2,5$
            \item Placer $B$ d'abscisse $-1$
            \item Placer $C$ d'abscisse $-2,5$
            \item Comment sont $A$ et $C$ relativement à $O$ ?
        \end{enumerate}
    }
}

\def\imgPrefix{mm-c4/expl-}
\slide{cr}{
    \ssec\ssubsec
    \df{}{L'\key{abscisse} d'un point est le nombre qui permet de repérer ce point sur la droite graduée.}
    \bvspace{-1cm}
    \expl{}{\bvspace{-0.5cm}\imgp{5p190}[10cm]
        Le point $A$ est d'abscisse \palt{2}{$-3$} et $B$ d'abscisse \palt{2}{$3$}.
        Ils ont alors la même \key{distance à $0$}, mais sont de signes différents. 
    }[\mi]
}

\slide{}{
    \rmk{}{
        Sur une droite graduée, deux points qui ont des abscisses opposées sont symétrique par rapport à l'origine.
    }[\mi]
}

\def\imgPrefix{mm-c4/exo-}
\exoSlide{51p197,53p197,55p197}[7cm][2][\mi]
\def\imgPrefix{}

\scn{Découverte repérage dans un plan}{}

\bsubsec{Repérage dans le plan}

\def\imgPrefix{mm-c4/exo-}
\exoSlide{26p196,31p196}[8cm][1][\mi][qf]

\def\imgPrefix{}

\slide{exo}{
    \small
    \bvspace{-0.75cm}
    \act{}{Guybrush, repéré par le point $G$. est à la recherche du trésor de l'île des singes.

        \bvspace{-0.1cm}
        \wideFrame[6.8em]{
            \dividePage{\imgp{reperage-plan-activite}[6.7cm]}{
                \begin{enumerate}
                    \item Combien de nombres sont nécessaires pour repérer sa position ? Donnez ces nombres pour Guybrush.
                    \item Que doit-on faire pour qu'il n'y ait pas plusieurs positions possibles repérées par ces nombres?
                    \item Repérer ainsi Guybrush et son bateau accosté au point $B$.
                    \item Demandez de l'aide au capitaine PESIN pour qu'il vous donne l'emplacement du trésor.  
                \end{enumerate}
            }[0.35]
        }
        \begin{enumerate}
            \setcounter{enumi}{4}
            \item Placer un point $T$ à l'emplacement du trésor.
        \end{enumerate}
    }
}

\scn{Institutionnalisation repérage dans un plan}{}

\def\imgPrefix{mm-c4/exo-}
\exoSlide{64p198}[6cm][1][\mm][qf]
\def\imgPrefix{}

\slide{cr}{
    \vc{}{%
        Les points du plan sont repérés par deux nombres qui forment ces \key{coordonnées}:
        \begin{itemize}
            \item L'\key{abscisse} qui se lit sur l'axe horizontal.
            \item L'\key{ordonnée} qui se lit sur l'axe vertical.
        \end{itemize}
    }[\mi]
}

\slide{cr}{
    \expl{}{
        \imgp{axes-plan}[5cm]
        $A$ est d'abscisse $3$ et d'ordonnée $2$. On note ces coordonnées $A(3;2)$.
    }[\href{https://www.maths-et-tiques.fr/telech/19Nomb_rel1.pdf}{Yvan Monka}]
}

\def\imgPrefix{mm-c4/exo-}
\exoSlide{57p197,58p198}[5.5cm][2][\mi]
\exoSlide{59p198}[6cm][1][\mi]

\scn{Comparaison de nombres relatifs}{}

\bsec{Comparaison de nombres relatifs}
\slide{cr}{
    \ssec
    \mthd{}{
        Lorsque l'on compare  deux nombres relatifs,
        si les deux nombres sont :
        \begin{itemize}
            \item positifs, le plus grand est celui avec la plus grande distance à 0.
            \item négatifs, le plus grand est celui avec la plus petite distance à 0.
            \item de signes opposées, le nombre positif est toujours plus grand.
        \end{itemize}
    }
}

\slide{cr}{
    \expl{Comparer :}{
        \multiColEnumerate{3}{
            \item $10$ et $ 10,09$
            \item $-1$ et $-1,1$
            \item $-100$ et $50$
        }
    }
    \rmk{Comparaison avec droite graduée}{Le point le plus à droite correspond au nombre le plus grand.}
}

\exoSlide{66p198}[6cm][1][\mi]

\exoSlide{70p198,87p200}[6cm][2][\mi][dm]
% % VARIABLES %%%
\def\authors{\href{https://juels.dev/}{Jules PESIN}}
% \date{\today}
\def\longTitle{Divisibilité et nombres premiers}
\setcounter{seq}{1}
\bseq{\longTitle}

\setboolean{showRef}{false}

\def\my{Myriade 5e}
% \newcommand{\myl}[1]{\href{#1}{\my}}

\def\imgPath{enseignement/4e/divisibilite-et-nombres-premiers/}
\def\imgExtension{.png}
%%

% Yvan Monka : https://www.maths-et-tiques.fr/telech/19Divi-np.pdf
% Crible d'Ératosthène : https://fr.wikipedia.org/wiki/Crible_d%27%C3%89ratosth%C3%A8ne
% Juniper Green : https://fr.wikipedia.org/wiki/Juniper_Green_(jeu)

% \qf{
%     {$2$ divise : \choice{$4$} \choice{$5$} \choice{$6$}, \choicea{1} et \choicea{3}},
%     {$6$ divise : \choice{$18$} \choice{$12$} \choice{$2$}, \choicea{1} et \choicea{2}},
%     {$30$ est divisible par : \choice{$3$} \choice{$10$} \choice{$5$} \choice{$4$}, \choicea{1}{,} \choicea{2} et \choicea{3}},
%     {Quels nombres sont premiers ? : \choice{$6$} \choice{$13$} \choice{$2$} \choice{$1$}, \choicea{2} et \choicea{3}}%
% }

\bsec{Nombres premiers}
\bsubsec{Nombres premiers}

\slide{COURS}{
    \sseq\ssec
    \df{}{
        Un entier est \key{premier} s'il a exactement deux diviseurs différents 1 et lui même.
    }
}

\slide{}{
    \act{Crible d'Ératosthène}{
        \imgp{crible-d-eratosthene}[5cm]
    }
}

\bsec{Décomposition en facteurs premiers}

% % VARIABLES %%%
% \date{\today}
\def\longTitle{Organisation et gestion de données}
\def\shortTitle{\MakeUppercase{\longTitle}}

\setcounter{seq}{2}
\bseq{\longTitle}

\setgrade{6e}
\def\imgPath{enseignement/6e/organisation-et-gestion-de-donnees/}
%%

\def\ym{\href{https://www.maths-et-tiques.fr/telech/19Tab_Graph.pdf}{Yvan Monka}}

\avspace{0.1cm}

\obj{
    \item Prélever des données numériques à partir de supports variés.
    \item Produire des tableaux, diagrammes et graphiques organisant des données numériques.
    \item Exploiter et communiquer des résultats de mesures.
    \item Situations de proportionnalité/situations qui ne sont pas de proportionnalité
}

\scn{Tableaux}{}

\bsec{Tableaux}
% \bsubsec{Tableaux de données}

\def\imgPrefix{dim-6e/qf-}
\exoSlide{1-2p96}[10cm][1][\dim][qf]
\def\imgPrefix{}

\def\imgPrefix{dim-6e/act-}
\slide{exo}{
    \bvspace{-0.75cm}
    \act{1p98}{\bvspace{-0.75cm}\imgp{1p98}[11.5cm]}[\dim]
}

\slide{cr}{
    \sseq\ssec
    \vc{}{
        Un \key{tableau} permet de rassembler et d'organiser des données pour les lire plus facilement.
    }
}

\slide{cr}{
    \expl{}{
        Au collège de la Paix,
        les enfants ont le choix entre 3 LV2 :
        italien, allemand ou espagnol.\\

        En 6eA, il y a 25 élèves. 12 ont choisi espagnol, 6 allemand et les autres italien.

        En 6eB, 13 élèves ont choisi espagnol et 5 élèves allemand.

        Dans ces deux classes, 12 élèves ont choisi italien.\\

        Présenter ces données dans un tableau à double entrée.
    }[\href{https://eduscol.education.fr/document/14014/download}{Attendues 6e}]
}

\slide{cr}{\bvspace{-0.5cm}
    \mthd{Construire un tableau}{\bvspace{-0.75cm}
        \begin{enumerate}
            \item On réalise un tableau à double entrée avec les données de l'énoncé et on ajoute une colonne et une ligne total. 
            \item On le complète avec les données de l'énoncé.
            \item On finit de compléter le tableau en effectuant les calculs.
        \end{enumerate}
        \begin{center}
            \palt{2}{
            \begin{tabular}{|c|c|c|c|c|}
                \hline
                & Espagnol & Allemand & Italien & Total\\
                \hline
                5eA    & 12       & 6        & 7       & 25        \\
                \hline
                5eB    & 13       & 5        & 5       & 23        \\
                \hline
                Total  & 25       & 11       & 12      & 48        \\
                \hline
            \end{tabular}
            }
        \end{center}
    }[\ym]
}

\def\imgPrefix{dim-6e/exo-}
\exoSlide{17p104,6p101,19p104}[6cm][2][\dim]

\scn{Situations de proportionnalité}{}

\bsec{Situations de proportionnalité}
\bsubsec{Définition}

\def\imgPrefix{}
\slide{qf}{%
    \begin{enumerate}
        \item \imgp{cn-CM2-juin-2023-exo-18}[8cm]
        \item \imgp{cn-CM2-juin-2023-exo-21-22}[8cm]
    \end{enumerate}
}

\slide{exo}{
    \bvspace{-0.5cm}
    \act{}{
        \begin{itemize}
            \item Pour une recette de dahl,
            Christopher a besoin de 4 gousses d'ail pour 6 personnes.
            Il déjeune avec ses amis Sarah et Jean.
            Peut-il prévoir combien de gousses d'ail il aura besoin ?\\
            Pourquoi ? et si oui, combien ?
            \item Jean l'a félicité 2 fois pour sa cuisine en 20 minutes.
            Peut-il prévoir combien de félicitations il recevra de Jean en 40 minutes?\\
            Pourquoi ? et si oui, combien ?
        \end{itemize}
        
    }
}

\slide{cr}{
    \ssec\ssubsec
    % \df{}{
    %     Deux grandeurs sont dites \key{proportionnelles} si les valeurs de l'une s'obtiennent en multipliant les valeurs de l'autre par un même nombre non nul,
    %     appelé le \key{coefficient de proportionnalité}.
    % }
    \df{}{
        Deux \key{grandeurs proportionnelles} sont deux grandeurs qui varient dans les mêmes proportions.
    }
}


\slide{cr}{
    \expl{}{
        Les $2kg$ de lentilles $5\EUR$.
        Combien en coutent $10kg$ de lentilles ? $12kg$ de lentilles?
    }
}

\bsubsec{Tableaux de proportionnalité}

\slide{cr}{
    \ssubsec

    \rmk{}{
        On peut présenter l'exemple précédent dans un tableau.
    }
    
    \vc{}{
        Un \key{tableau de proportionnalité} est un tableau où chaque ligne est proportionnelle aux autres.
    }[\wiki{Proportionnalité}]
}


\def\imgPrefix{dim-6e/exo-}
\exoSlide{2p83}[6cm][1][\dim]

\scn{Diagrammes en baton}{}

\bsec{Diagrammes}
\bsubsec{Diagrammes en baton}

\def\imgPrefix{dim-6e/qf-}
\exoSlide{3-4p96}[8cm][1][\dim][qf]

\def\imgPrefix{dim-6e/act-}
\slide{exo}{
    \bvspace{-0.75cm}
    \act{2p98}{\bvspace{-0.75cm}\imgp{2p98}[9cm]}[\dim]
}

\slide{cr}{
    \ssec\ssubsec
    \vc{}{
        Un \key{diagramme en bâton} (ou à barres) permet de comparer visuellement des données.
    }[\dim]
}

\def\imgPrefix{}
\slide{exo}{
    \bvspace{-0.5cm}
    \exo{Vrai ou faux?}{
        Le nombre de tablettes vendues de la marque B est trois fois plus important que le nombre de tablettes vendues de la marque A.
        \imgp{diagramme-en-baton-attendus-6e}[7cm]
    }
}

\slide{cr}{
    \pr{}{
        Si un diagramme en bâtons a pour origine 0,
        alors la hauteur des barres est proportionnelle aux effectifs.
    }
}


% \newpage
\slide{exo}{%
    \bvspace{-0.5cm}
    \exo{}{
        On a demandé aux élèves des trois classes de 6e du collège Anatole France combien d'animaux de compagnie vivaient avec eux.
        On a représenté les résultats dans le diagramme suivant.
        \imgp{nos-amis-les-betes}[9cm]
    }[\rpmc[28]]
}

\slide{exo}{
    Les affirmations suivantes sont-elles vraies ou fausses? Justifier.
    \begin{enumerate}
        \setItemColor{\currentColor}
        \item  21 élèves ont un seul animal de compagnie.
        \item Il y a 75 élèves en 6e au collège Anatole France.
        \item Les élèves qui ont deux animaux de compagnie sont trois fois plus nombreux que les élèves qui ont trois animaux de compagnie.
        \item 70 élèves ont moins de trois animaux de compagnie.
        \item Plus de la moitié des élèves ont au moins un animal de compagnie.
    \end{enumerate}
}

\scn{Diagrammes circulaires}{}

\bsubsec{Diagrammes circulaires}

\def\imgPrefix{dim-6e/qf-}
\exoSlide{5-6p96}[8cm][1][\dim][qf]

\def\imgPrefix{dim-6e/act-}
\slide{exo}{
    \bvspace{-0.75cm}
    \act{4p99}{\bvspace{-0.75cm}\imgp{4p99}[11cm]}[\dim]
}

\def\imgPrefix{}
\slide{exo}{
    \bvspace{-0.5cm}
    \exo{}{\ifBeamer{\small}
        Dans un collège,
        112 élèves viennent en voiture,
        autant viennent à vélo,
        56 viennent en bus et 280 viennent à pied.
        \bvspace{-0.25cm}
        \begin{enumerate}
            \dividePage{%
                \item Un seul de ces diagrammes circulaires représente le mode de déplacement des élèves de ce collège.
                Lequel?
            }{\imgp{vers-de-mobilites-douces-1}[6.5cm]}
            
            \ifArticle{
                \dividePage{\imgp{vers-de-mobilites-douces-2}[8cm]}{%
                \item Compléter le tableau ci-contre, puis choisir les nombres appropriés pour graduer
                le diagramme en bâtons qui représente ces données.
                }
            }
            \saveenumi{1}
        \end{enumerate}
    }[\rpmc[34]]
}

\ifBeamer{
    \slide{exo}{\bvspace{-0.25cm}
        \begin{enumerate}
            \loadenumi
            \item Compléter le tableau ci-contre, puis choisir les nombres appropriés pour graduer
        le diagramme en bâtons qui représente ces données.
        \bvspace{-0.45cm}\imgp{vers-de-mobilites-douces-2}[8cm]
        \end{enumerate}
    }
}


\scn{Graphiques cartésiens}{}

\bsec{Graphiques cartésiens}

\def\imgPrefix{dim-6e/act-}
\slide{exo}{
    \bvspace{-0.70cm}
    \act{3p99}{\bvspace{-0.75cm}\imgp{3p99}[9cm]}[\dim]
}

\def\imgPrefix{}
\slide{exo}{
    \exo{}{
        Que pourrait représenter ce graphique à propos d'une salle de classe?
        Le décrire avec le plus de précision possible.
        Justifier et compléter le graphique ci-dessous.
        \imgp{l-allure-de-la-courbe}[10cm]
    }[\rpmc[31]]
}

\bsec{Reconnaître une situation de proportionnalité}
% VARIABLES %%%
\def\authors{\jules}
% \date{\today}
\def\longTitle{Proportionnalité - Situation de proportionnalité et conversions}
\def\shortTitle{\MakeUppercase{\longTitle}}

\setcounter{seq}{2}
\bseq{\longTitle}
% \def\theme{\longTitle}

\setgrade{5e}

\def\imgPath{enseignement/5e/proportionnalite/situation-de-proportionnalite-et-conversions/}
\def\imgExtension{.png}
%%

% Yvan Monka : https://www.maths-et-tiques.fr/telech/19Prop1.pdf
\ifArticle{\vspace*{0.1cm}}

\obj{
    \item Reconnaître une situation de proportionnalité ou de non proportionnalité entre deux grandeurs.
    \item Résoudre des problèmes de proportionnalité par passage à l'unité.
    \item Effectuer des calculs de durées et d'horaires.
    \item Effectuer des conversions d'unités de longueurs, et de durées.
    \item Partager une quantité en deux ou trois parts selon un ratio donné.
}

\scn{Passage à l'unité}{}

\qfSlide{
    Donner le resultat sous la forme décimale de :
    \multiColEnumerate{2}{
        \item $14 \times 3 = \palt{2}{\ncalc{14*3}} $
        \item $3 \div 2 = \palt{2}{1,5} $
        \item $6 \div 4 = \palt{2}{1,5} $
        \item $5,6 \times 2 = \palt{2}{11,2} $
        \item $\frac{1}{4} = \palt{2}{0.25}$
        \item $\frac{3}{4} = \palt{2}{0.75}$
    }
}

\bsec{Proportionnalité}
\bsubsec{Définition}

\slide{exo}{
    \bvspace{-0.5cm}
    \act{}{
        \begin{itemize}
            \item Pour une recette de dahl,
            Christopher a besoin de 8 gousses d'ail pour 4 personnes.
            Il déjeune avec ses amis Sarah et Jean.
            Peut-il prévoir combien de gousses d'ail il aura besoin ?\\
            Pourquoi ? et si oui, combien ?
            \item Jean l'a félicité 2 fois pour sa cuisine en 20 minutes.
            Peut-il prévoir combien de félicitations il recevra de Jean en 40 minutes?\\
            Pourquoi ? et si oui, combien ?
        \end{itemize}
        
    }
}

\slide{cr}{
    \sseq\ssec\ssubsec
    \df{}{
        Deux grandeurs sont dites \key{proportionnelles} si les valeurs de l'une s'obtiennent en multipliant les valeurs de l'autre par un même nombre non nul,
        appelé le \key{coefficient de proportionnalité}.
    }
}

\slide{cr}{
    \expl{}{
        Les $2kg$ de lentilles $5\EUR$.
        Combien en coutent $13kg$ de lentilles ?
    }

    \mthd{Passage à l'unité}{
        \palt{2}{
            \begin{tabular}{|c|c|c|c|}
                \hline
                Masse (en $\kilo\gram$)  & 2 & 1 & 13\\
                \hline
                Prix (en \EUR)    & 5 & 2,5 & 32,5\\
                \hline
            \end{tabular}
        }
        \palt{3}{Le coefficient de proportionnalité est 2,5.}
    }
}

\slide{cr}{
    \vc{}{
        Un \key{tableau de proportionnalité} permet de présenter une situation de proportionnalité.
        Sa deuxième ligne s'obtient en multipliant la première par le coefficient de proportionnalité.
    }
}

\def\imgPrefix{mm-c4/exo-}
\exoSlide{32p27,51p28,53p28}[7cm][2][\mm]

\scn{Reconnaître une situation de proportionnalité}{}

\def\imgPrefix{mm-c4/qf-}
\exoSlide{15p26,17p26}[7cm][2][\mm][qf]

\bsubsec{Reconnaître une situation de proportionnalité}

\slide{cr}{
    \bvspace{-0.25cm}
    \ssubsec
    \bvspace{-0.5cm}
    \expl{}{
        Une marque d'épices vend différentes tailles de pots de curcuma et de cumin.
        Présenter avec leur prix dans les deux tableaux ci-dessous.\\

        \begin{tabular}{|c|c|c|c|}
            \hline
            Masse de curcuma (en $\gram$)  & 45 & 60 & 90\\
            \hline
            Prix (en \EUR)    & 2.5 & 4 & 5\\
            \hline
        \end{tabular}\ifArticle{\quad}\ifBeamer{\\ \\}
        \begin{tabular}{|c|c|c|c|}
            \hline
            Masse de cumin (en $\gram$)  & 45 & 60 & 90\\
            \hline
            Prix (en \EUR)    & 2.7 & 4.05 & 5.4\\
            \hline
        \end{tabular}\\

        Les prix de ces deux épices sont-ils proportionnels à leur masse ?
    }
}

\slide{cr}{
    \mthd{Reconnaître une situation de proportionnalité}{
        \palt{2}{
            On divise chaque nombre de la première grandeur par ceux de la 2e grandeurs correspondant.
            Si on obtient le même résultat, il y a proportionnalité.
        }
    }
}

% \exoSlide{}[][][][]

\bsec{Grandeurs}


\def\imgPrefix{}
\slide{qf}{
    \begin{enumerate}
        \item \imgp{qf-cn-3-5e-mars-2023}[6cm]
        \item \imgp{qf-cn-8-5e-mars-2023}[6cm]
    \end{enumerate}
}

\bsubsec{Longueurs}

\bsubsec{Durées}

\bsec{Ratio}
% % VARIABLES %%%
% \date{\today}
\setSeq{2}{Théorème de Pythagore - Sens direct}

\setGrade{4e}
\def\imgPath{enseignement/4e/theoreme-de-pythagore/sens-direct/}

% \firstSlide
%%

\def\ym{\href{https://www.maths-et-tiques.fr/telech/19Pyth1.pdf}{Yvan Monka}}

\avspace{0.1cm}

\obj{
    \item Utiliser la calculatrice pour déterminer une valeur approchée de la racine carrée d'un nombre positif.
    \item Utiliser la racine carrée d'un nombre positif en lien avec des situations géométriques.
    \item Utiliser le sens direct du Théorème de Pythagore.
}

\scn{Decouvrir l'égalité de Pythagore}
\bsec{Egalité de Pythagore}

\slide{qf}{
    \dividePage{
        \begin{enumerate}
            \item Peut-on construire un triangle $ABC$ avec:
            \begin{enumerate}
                \item $AB = 5cm \pv BC = 11cm \pv AC = 4cm$
                \item $AB = 8cm \pv BC = 9cm \pv AC = 6cm$
            \end{enumerate}
        \end{enumerate}
    }{
        \imgp{mi-c4/qf-5p426}[7cm]
    }
}

\slide{exo}{
    \bvspace{-0.5cm}
    \act{}{
        \bvspace{-0.5cm}
        \dividePage{%
            \imgp{activite-decouverte-fig}[5cm]
        }
        {
            \ifBeamer{\small}
            \begin{enumerate}
                \item Donner une expression mathématique permettant de calculer l'aire des carrés 1; 2 et 3.
                \item Découper les quatre triangles et le carré fournis en annexe.
                \item Disposer les quatre triangles dans le carré afin de former deux nouveaux carrés.
                \item Quelles sont les aires des deux carrés formés ?
            \end{enumerate}
        }[0.35]
    }[\href{https://drive.google.com/drive/folders/1ipPqxysYc8GHNOIT0u6HSi8cAnUiYfxx}{Audrey Belay}]
}

\slide{exo}{
    \begin{enumerate}\setcounter{enumi}{4}
        \item Disposer les quatre triangles dans le carré afin de former un seul carré.
        \item Quelle est l'aire de ce carré ?
        \item Quelle relation peut-on établir entre les aires des carrés 1, 2 et l'aire du carré 3 ?
        \item Quelle relation peut-on établir entre les longueurs $a$, $b$ et $c$ ?
    \end{enumerate}
    \ifArticle{Annexe:\imgp{activite-decouverte-print}[6cm]}
}

\scn{Utiliser l'égalité de Pythagore}

\slide{qf}{
    \calculator  \\
    Ces tableaux présentent-ils des situations de proportionnalité ?
    \multiColEnumerate{2}{
        \item\begin{tabular}{|C{1.5cm}|C{1.5cm}|}
            \hline
            89.08 & 147.39\\
            \hline
            52.4 & 86.7\\
            \hline
        \end{tabular}
        
        \item\begin{tabular}{|C{1.5cm}|C{1.5cm}|C{1.5cm}|}
            \hline
            277.5 & 7.83 & 139.2\\
            \hline
            92.5 & 2.7 & 46.4\\
            \hline
        \end{tabular}
    }
}

\slide{cr}{
    \sseq\ssec
    \df{}{
        Dans un triangle rectangle, le côté opposé à l'angle droit est appelé \key{hypoténuse}.
    }[\href{https://fr.wikipedia.org/wiki/Hypoténuse}{Wikipédia}]
}

\slide{cr}{
    \setboolean{showID}{false}
    \thm{de Pythagore}{
        \Sialors{un triangle est rectangle}
        {le carré de la longueur de l'hypoténuse est égal à la somme des carrés des longueurs des deux autres côtés.}
        \color{black}
        \begin{center}
            \begin{tikzpicture}[scale=1]%,cap=round,>=latex]
                \coordinate (A) at (-1.5cm,-1.cm);
                \coordinate (C) at (1.5cm,-1.0cm);
                \coordinate (B) at (1.5cm,1.0cm);
                \draw (A) -- node[above] {$a$} (B) -- node[right] {$c$} (C) -- node[below] {$b$} (A);
                \draw[color = BlueViolet] (1.25cm,-1.0cm) rectangle (1.5cm,-0.75cm);
            \end{tikzpicture}\\
            \color{ForestGreen}$c^2=a^2+b^2$
        \end{center}
    }
    \setboolean{showID}{true}
}

\slide{cr}{
    \expl{}{
        Soit $ABC$ un triangle rectangle en $A$, tel que $AB = 6$, $AC = 8$, $BC=10$.
        Verifier si le triangle $ABC$ respecte bien l'égalité de Pythagore.
    }
}

\slide{exo}{\bvspace{-1cm}
    \exo{}{
        \vspace{-0.5cm}
        \imgp{pythagore-a-trou-1}[11cm]
    }
}

\def\imgPrefix{mi-c4/exo-}
\exoSlide{3p430,4p430}[7cm][2][\mi]

\scn{Decouvrir la racine carré}

\def\imgPrefix{mi-c4/qf-}
\exoSlide{3p426}[10cm][1][\mi][qf]

\bsec{Racine carré}

\slide{exo}{
    \act{}{
        \ifBeamer{\small \vspace{-0.5cm}}
        \begin{enumerate}\bvspace{0.2cm}
            \item Calculer l'aire d'un carré de côté:
            \multiColEnumerate{4}{
                \item $3\cm$ \item $12\cm$ \item $5,5\cm$ \item $x\cm$
            }
            \item Trouvé le côté d'un carré d'aire:
            \multiColEnumerate{4}{
                \item $4\cmd$ \item $25\cmd$ \item $1\cmd$ \item $169\cmd$
            }
            \item Encadrer par deux entiers le côté d'un carré d'aire $40\cmd$.
            \item Trouver le côté d'un carré d'aire $20.25\cmd$.
            \item Approcher au centième près le côté d'un carré d'aire $2\cmd$.
        \end{enumerate}
    }
}

\scn{Utiliser la racine carré}

\slide{qf}{
    \multiColEnumerate{2}{
        \item $1 - 6 = $
        \item $10 + (-5) =$
        \item $21 - (- 6) =$
        \item $-15 + 4 = $
        \item $7,6 - 10 = $
        \item $-\frac{5}{4} + \frac{1}{4} = $
    }
}

\slide{cr}{
    \ssec
    \df{}{
        La \key{racine carrée} d'un nombre positif $x$ représente la longueur des côtés d'un carré dont l'aire est égale à $x$.
        On la note $\sqrt{x}$.
    }
    \bvspace{-1cm}
    \expl{}{\bvspace{-0.25cm}
        \multiColEnumerate{2}{
            \item $\sqrt{4} = \palt{2}{2}$
            \item $\sqrt{9} = \palt{2}{3}$
        }
        \calculator
        \multiColEnumerate{2}{
            \item $\sqrt{5,5225} = \palt{2}{2,35}$
            \item $\sqrt{12} \approx \palt{2}{3.46410161514}$
        }
    }
}

\slide{cr}{
    \df{}{
        Un \key{carré parfait} est le carré d'un entier positif.
    }[\wiki{Carré_parfait}]

    \expl{}{
        \multiColItemize{5}{
            \foreach \nb in {0,1,...,12} {
                \item  $\nb^2 = \palt{2}{\ncalc{\nb*\nb}}$
            }
        }
    }
}

\scn{Calculer une longueur grace au Théorème de Pythagore}

\bsec{Calcul de longueur}

\def\imgPrefix{mi-c4/qf-}
\exoSlide{1p430}[8cm][1][\mi][qf]

\slide{cr}{
    \expl{}{
        Soit $ABC$ un triangle rectangle en $A$, tel que $AB = 3\cm$ et $AC = 4\cm$.
        Déterminer $BC$.
    }

    \mthd{}{
        \palt{2}{
            \Ona le triangle ABC rectangle en A.\\
            \Alors d'après le théorème de Pythagore :
            \begin{align*}
                BC^2 &= AB^2 + AC^2\\
                &= 3^2 + 4^2\\
                &= 9 + 16\\
                &= 25\\
                \alors BC &= \sqrt{25}\\
                \donc &= 5\centi\meter\\
            \end{align*}
        }
    }
}

\def\imgPrefix{mi-c4/}
\exoSlide{17p431,18p431,38p433}[8cm][2][\mi]

\scn{Résoudre des problèmes mettant en jeux le théorème de pythagore}

\def\imgPrefix{}
\slide{qf}{
    \multiColEnumerate{2}{
        \item \imgp{cn-18-4e-juin-2023}[9cm]
        \item \imgp{cn-28-4e-juin-2022}[9cm]
    }
}

\slide{exo}{
    \ifBeamer{\small \vspace{-0.75cm}}
    \exo{}{%
        Un professeur d'EPS trace un circuit de course à pied avec des plots :
        \begin{itemize}
            \item le plot n°2 est situé à 36 m au nord du plot n°1, qui est le plot de départ ;
            \item le plot n°3 est situé à 69 m à l'est du plot n°2 ;
            \item le plot n°4 est situé à 72 m au sud du plot n°3.
        \end{itemize}
        Chaque élève va d'un plot au suivant en ligne droite,
        et parcourt un certain nombre de fois le circuit 1-2-3-4-1.
        Il continue sur le même circuit jusqu'au plot d'arrivée,
        placé sur ce circuit de telle sorte le trajet total ait une longueur de 1,5 km.
        \begin{enumerate}
            \item Où le professeur doit-il placer le plot d'arrivée ?
            \item Combien de fois un élève doit-il parcourir le circuit ?
        \end{enumerate}
    }[\href{https://eduscol.education.fr/document/17305/download\#page=8}
    {Utiliser les notions de géométrie plane pour démontrer}]
}

\slide{exo}{
    \exo{}{
        On assimile la Terre à une sphère de centre $O$ et de rayon $R = 6371\kilo\meter$.
        Un observateur se tient à la surface de la Terre en un point A.
        Le point H représente l'endroit le plus éloigné que l'observateur peut voir, c'est-à-dire son horizon.
        \begin{enumerate}
            \item Représentez la situation par un schéma.
            \item À quelle distance de l'observateur se trouve environ l'horizon?
            \item Quelle serait cette distance si l'observateur se trouve au $3^e$ étage de la tour Eiffel?
        \end{enumerate}
    }[\href{https://blogdemaths.wordpress.com/2015/06/27/jusquou-peut-on-voir-a-lhorizon/}{Jusqu'où peut on voir à l'horizon?}]
}

% % VARIABLES %%%
\setTitle{Correction - Devoir Maison - Séquence 1}
\setGrade{6e}
\def\imgPath{enseignement/6e/geometrie-plane/points-et-droites/}
%%

\exo{44p217}{
    \imgp{dim-6e/exo-44p217}[5cm]
    \imgp{dm-44p217}[5cm]
    \color{BurntOrange} d. \color{black}
    La droite $(AC)$ semble passer par le point $P$.
}

\exo{71p221}{
    \imgp{dim-6e/exo-71p221}[5cm]
    \imgp{dm-71p221}[5cm]
    \color{BurntOrange} e. \color{black}
    $A,C$ et $D$ sont positionnés au milieu des segments qui composent les côtés du triangle $IJK$.
}

% % VARIABLES %%%
\setTitle{Correction - Devoir Maison - Séquence 1}
\setGrade{5e}
\def\imgPath{enseignement/5e/nombres-relatifs/reperage-et-comparaison/}
%%

\exo{70p198}{
    \imgp{mm-c4/exo-70p198}[6cm]
    Il y a \key{7 chemins} possible :
    \multiColEnumerate{2}{
        \item $-7,8 < -5,2 < -3 < 2$
        \item $-7,8 < -5,2 < -3 < 0 < 2$
        \item $-7,8 < -5,2 < -3 < -2 < 0 < 2$
        \item $-7,8 < -6,5 < -5,2 < -3 < 2$
        \item $-7,8 < -6,5 < -5,2 < -3 < 0 < 2$
        \item $-7,8 < -6,5 < -5,2 < -3 < -2 < 0 < 2$
        \item $-7,8 < -6,5 < -2 < 0 < 2$
    }
}

\exo{87p200}{
    \imgp{mm-c4/exo-87p200}[6cm]

    \begin{enumerate}
        \item On peut trouver l'altitude du fond du lac Ontario en enlevant sa profondeur à l'altitude de sa surface.\\
        74,2 - 244 = -169,8\\
        Le fond du lac Ontario se trouve à une altitude de -169,8 \meter.
        \item On peut trouver l'altitude du sommet des Chutes du Niagara en ajoutant leurs hauteur à l'altitude de la surface du lac Ontario.\\
        74,2 + 52 = 126.2\\
        Le sommet des Chutes du Niagara se trouve à une altitude de 126,2 \meter.
    \end{enumerate}
}
% % VARIABLES %%%
\setTitle{Correction - Devoir Maison - Séquence 1}
\setgrade{4e}
\def\imgPath{enseignement/4e/divisibilite-et-nombres-premiers/}
%%

\exo{20p17}{
    \imgp{mi-c4/exo-20p17}[6cm]

    \imgp{dm-20p17}[8cm]
}

\exo{45p19}{
    \imgp{mi-c4/exo-45p19}[6cm]
    \begin{enumerate}
        \item\begin{itemize}
            \item Pour trouver les diviseurs de $6$, on commence par le décomposer en facteurs premiers.
            \item $6 = 2 \times 3$
            \item Les diviseurs de $6$ sont donc $1;2;3$ et $6$.
            \item Or $1+2+3 = 6$.
            \item $6$ est donc bien un nombre parfait.
        \end{itemize}
        \item\begin{itemize}
            \item Pour trouver les diviseurs de $28$, on commence par le décomposer en facteurs premiers.
            \item $28 = 2 \times 2 \times 7$
            \item Les diviseurs de $6$ sont donc $1;2;7;2\times2 = 4;2\times7 = 14$ et $28$.
            \item Or $1+2+7+4+14 = 28$.
            \item $28$ est donc bien un nombre parfait.
        \end{itemize}
    \end{enumerate}
}

\exo{56p20}{
    \imgp{mi-c4/exo-56p20}[6cm]
    \begin{itemize}
        \item Le prénom Annabelle comporte $9$ lettres.
        \item Or la division euclidienne de $1000$ par $9$ donne : $1000 = 9 \times 111 + 1$.
        \item Alors à la $999$ lettre Annabelle aura écrit son prénom pour la $111^e$ fois.
        \item Et la $1000^e$ lettre sera donc un $A$.
    \end{itemize}
}

% % VARIABLES %%%
\setTitle{Interrogation - Séquence 1}
\setgrade{6e}
\thispagestyle{assignment}
%%

\def\Ona{\key{On a} }
\def\Or{\key{Or} }
\def\Donc{\key{Donc} }


\exo{}{
    La droite (FD) est perpendiculaire à la droite (AB),
    et la droite (AB) est parallèle à la droite (GH). 
    Quelle est la relation entre les droites (FD) et (GH) ?
}

\corr{}{
    \Ona $(AB) \prll (GH)$ et $(FD) \perp (AB)$.\\
    \Or \sialors{deux droites sont parallèles}{toute perpendiculaire à l'une
    est perpendiculaire à l'autre.} \\ 
    \Donc $(FD) \perp (GH)$.
}
% % VARIABLES %%%
\setTitle{Interrogation - Séquence 2}
\setGrade{5e}
\thispagestyle{assignment}
%%

% \hint{Calculatrice autorisée}
\calculator

\def\colWidth{1.5cm}
\exo{Les tableaux ci-dessous représentent-ils une situation de proportionnalité ?}{\vspace{-0.5cm}%
    \multiColEnumerate{2}{
        % \item \propTable{5}{8}{7,5}{12}
        \item \begin{tabular}{|C{\colWidth}|C{\colWidth}|C{\colWidth}|}
            \hline
            $36$ & $13,95$ & $40,5$\\
            \hline
            $8$ & $3,1$ & $9$\\
            \hline
        \end{tabular}
        \item \begin{tabular}{|C{\colWidth}|C{\colWidth}|}
            \hline
            $651,3$ & $128$\\
            \hline
            $100,2$ & $20$\\
            \hline
        \end{tabular}
    }
}

\corr{}{
    \begin{enumerate}
        \item \begin{align*}
            \frac{36}{8} = 4,5\qquad
            \frac{13,95}{3,1} &= 4,5\qquad
            \frac{40,5}{9} = 4,5\\
            \ialors \frac{36}{8} = \frac{13,95}{3,1} &= \frac{40,5}{9}
        \end{align*}
        L'\key{égalité} des quotients indique qu'\key{il s'agit bien} d'une situation de proportionnalité.
        \item \begin{align*}
            \frac{655,36}{128} = 5,12\qquad
            \frac{100,2}{20} &= 5,01\qquad
            \ialors \frac{651,3}{128} \neq \frac{100,2}{20}
        \end{align*}
        L'\key{inégalité} des quotients indique qu'\key{il ne sagit pas} d'une situation de proportionnalité.
    \end{enumerate}
}

% % VARIABLES %%%
\setTitle{Interrogation - Séquence 1}
\setGrade{4e}
\thispagestyle{assignment}
%%

\exo{}{
    \begin{enumerate}
        \item Donner la décomposition en facteurs premiers de : $1980$
        \item Écrire sous forme de fraction irréductible : $\dfrac{3150}{84}$
    \end{enumerate}
}

\corr{}{
    \begin{enumerate}
        \item \begin{align*}
            1980 &= 10 \times 198\\
            &= 2 \times 5 \times 2 \times 99\\
            &= 2 \times 2 \times 5 \times 9 \times 11\\
            &= 2 \times 2 \times 3 \times 3 \times 5 \times 11
        \end{align*}
        \item \begin{align*}
            3150 &= 2 \times 3 \times 3 \times 5 \times 5 \times 7\\
            \iet 84 &= 2 \times 2 \times 3 \times 7\\
            \ialors \dfrac{3150}{84} &= 
            \dfrac{\cancel{2} \times \cancel{3} \times 3 \times 5 \times 5 \times \cancel{7}}
            {\cancel{2} \times 2 \times \cancel{3} \times \cancel{7}}\\
            &= \dfrac{3 \times 5 \times 5}{2}\\
            & = \dfrac{75}{2}
        \end{align*}
    \end{enumerate}
}

% % VARIABLES %%%
\setTitle{Interrogation - Entrainement - Séquence 2}
\setGrade{5e}
%%

\calculator

\exo{Les tableaux ci-dessous représentent-ils une situation de proportionnalité ?}{%
\multiColEnumerate{2}{
\item\begin{tabular}{|C{1.5cm}|C{1.5cm}|C{1.5cm}|C{1.5cm}|}
    \hline
    89.08 & 147.39 & 18.53 & 23.46\\
    \hline
    52.4 & 86.7 & 10.9 & 13.8\\
    \hline
\end{tabular}

\item\begin{tabular}{|C{1.5cm}|C{1.5cm}|C{1.5cm}|}
    \hline
    277.5 & 7.83 & 139.2\\
    \hline
    92.5 & 2.7 & 46.4\\
    \hline
\end{tabular}

\item\begin{tabular}{|C{1.5cm}|C{1.5cm}|C{1.5cm}|}
    \hline
    124 & 19.26 & 172.44\\
    \hline
    77.5 & 10.7 & 95.8\\
    \hline
\end{tabular}

\item\begin{tabular}{|C{1.5cm}|C{1.5cm}|}
    \hline
    469.44 & 171.36\\
    \hline
    97.8 & 35.7\\
    \hline
\end{tabular}

\item\begin{tabular}{|C{1.5cm}|C{1.5cm}|C{1.5cm}|C{1.5cm}|}
    \hline
    18.16 & 6.32 & 64.48 & 38.56\\
    \hline
    22.7 & 7.9 & 80.6 & 48.2\\
    \hline
\end{tabular}

\item\begin{tabular}{|C{1.5cm}|C{1.5cm}|C{1.5cm}|C{1.5cm}|}
    \hline
    56.84 & 281.06 & 389.76 & 331.73\\
    \hline
    11.6 & 61.1 & 81.2 & 67.7\\
    \hline
\end{tabular}

\item\begin{tabular}{|C{1.5cm}|C{1.5cm}|C{1.5cm}|C{1.5cm}|}
    \hline
    101.83 & 29.07 & 80.07 & 109.65\\
    \hline
    59.9 & 17.1 & 47.1 & 64.5\\
    \hline
\end{tabular}

\item\begin{tabular}{|C{1.5cm}|C{1.5cm}|}
    \hline
    214.5 & 123.25\\
    \hline
    85.8 & 49.3\\
    \hline
\end{tabular}

\item\begin{tabular}{|C{1.5cm}|C{1.5cm}|C{1.5cm}|}
    \hline
    261.8 & 108.16 & 150.04\\
    \hline
    77 & 33.8 & 48.4\\
    \hline
\end{tabular}

\item\begin{tabular}{|C{1.5cm}|C{1.5cm}|C{1.5cm}|C{1.5cm}|}
    \hline
    58 & 101.4 & 33.2 & 118.2\\
    \hline
    29 & 50.7 & 16.6 & 59.1\\
    \hline
\end{tabular}

\item\begin{tabular}{|C{1.5cm}|C{1.5cm}|C{1.5cm}|}
    \hline
    72.1 & 197.28 & 336.6\\
    \hline
    20.6 & 54.8 & 93.5\\
    \hline
\end{tabular}

\item\begin{tabular}{|C{1.5cm}|C{1.5cm}|C{1.5cm}|}
    \hline
    29.25 & 66.13 & 127.36\\
    \hline
    19.5 & 38.9 & 79.6\\
    \hline
\end{tabular}

\item\begin{tabular}{|C{1.5cm}|C{1.5cm}|}
    \hline
    71.25 & 36.9\\
    \hline
    47.5 & 24.6\\
    \hline
\end{tabular}

\item\begin{tabular}{|C{1.5cm}|C{1.5cm}|C{1.5cm}|C{1.5cm}|}
    \hline
    4.86 & 8.67 & 20.28 & 20.55\\
    \hline
    8.1 & 28.9 & 67.6 & 41.1\\
    \hline
\end{tabular}

\item\begin{tabular}{|C{1.5cm}|C{1.5cm}|}
    \hline
    10.92 & 5.07\\
    \hline
    36.4 & 16.9\\
    \hline
\end{tabular}

\item\begin{tabular}{|C{1.5cm}|C{1.5cm}|C{1.5cm}|C{1.5cm}|}
    \hline
    123.28 & 152.55 & 222.18 & 280.14\\
    \hline
    26.8 & 33.9 & 48.3 & 60.9\\
    \hline
\end{tabular}

\item\begin{tabular}{|C{1.5cm}|C{1.5cm}|}
    \hline
    35.28 & 58.14\\
    \hline
    19.6 & 32.3\\
    \hline
\end{tabular}

\item\begin{tabular}{|C{1.5cm}|C{1.5cm}|C{1.5cm}|C{1.5cm}|}
    \hline
    51 & 121.75 & 178.36 & 125.58\\
    \hline
    20.4 & 48.7 & 68.6 & 54.6\\
    \hline
\end{tabular}

\item\begin{tabular}{|C{1.5cm}|C{1.5cm}|}
    \hline
    209.96 & 40.23\\
    \hline
    72.4 & 14.9\\
    \hline
\end{tabular}

\item\begin{tabular}{|C{1.5cm}|C{1.5cm}|C{1.5cm}|C{1.5cm}|}
    \hline
    94.64 & 13.77 & 279 & 378.93\\
    \hline
    18.2 & 2.7 & 55.8 & 74.3\\
    \hline
\end{tabular}

\item\begin{tabular}{|C{1.5cm}|C{1.5cm}|C{1.5cm}|}
    \hline
    21.76 & 7.48 & 17.6\\
    \hline
    12.8 & 4.4 & 11\\
    \hline
\end{tabular}

\item\begin{tabular}{|C{1.5cm}|C{1.5cm}|}
    \hline
    369 & 43\\
    \hline
    90 & 10\\
    \hline
\end{tabular}

\item\begin{tabular}{|C{1.5cm}|C{1.5cm}|C{1.5cm}|}
    \hline
    267.43 & 339.02 & 401.58\\
    \hline
    56.9 & 73.7 & 87.3\\
    \hline
\end{tabular}

\item\begin{tabular}{|C{1.5cm}|C{1.5cm}|}
    \hline
    16.53 & 178.6\\
    \hline
    8.7 & 94\\
    \hline
\end{tabular}

\item\begin{tabular}{|C{1.5cm}|C{1.5cm}|C{1.5cm}|C{1.5cm}|}
    \hline
    36.96 & 40.59 & 26.4 & 22.77\\
    \hline
    11.2 & 12.3 & 8 & 6.9\\
    \hline
\end{tabular}

\item\begin{tabular}{|C{1.5cm}|C{1.5cm}|C{1.5cm}|}
    \hline
    167.67 & 86.02 & 24.61\\
    \hline
    72.9 & 37.4 & 10.7\\
    \hline
\end{tabular}

\item\begin{tabular}{|C{1.5cm}|C{1.5cm}|C{1.5cm}|C{1.5cm}|}
    \hline
    159.16 & 95.76 & 220.56 & 128.64\\
    \hline
    69.2 & 39.9 & 91.9 & 53.6\\
    \hline
\end{tabular}

\item\begin{tabular}{|C{1.5cm}|C{1.5cm}|C{1.5cm}|}
    \hline
    327.32 & 328.8 & 210.21\\
    \hline
    66.8 & 68.5 & 42.9\\
    \hline
\end{tabular}

\item\begin{tabular}{|C{1.5cm}|C{1.5cm}|C{1.5cm}|C{1.5cm}|}
    \hline
    252.48 & 92.8 & 205.76 & 48\\
    \hline
    78.9 & 29 & 64.3 & 15\\
    \hline
\end{tabular}

\item\begin{tabular}{|C{1.5cm}|C{1.5cm}|C{1.5cm}|}
    \hline
    130.41 & 20.46 & 169.4\\
    \hline
    62.1 & 9.3 & 77\\
    \hline
\end{tabular}

}}

\newpage

\corr{}{%
\begin{enumerate}\item\begin{align*}
\frac{89.08}{52.4} = 1.7\qquad \frac{147.39}{86.7} = 1.7\qquad \frac{18.53}{10.9} = 1.7\qquad \frac{23.46}{13.8} = 1.7\qquad 
\end{align*}
L'\key{égalité} des quotients indique qu'\key{il s'agit bien} d'une situation de proportionnalité.

\item\begin{align*}
\frac{277.5}{92.5} = 3\qquad \frac{7.83}{2.7} = 2.9\qquad \frac{139.2}{46.4} = 3\qquad 
\end{align*}
L'\key{inégalité} des quotients indique qu'\key{il ne sagit pas} d'une situation de proportionnalité.

\item\begin{align*}
\frac{124}{77.5} = 1.6\qquad \frac{19.26}{10.7} = 1.8\qquad \frac{172.44}{95.8} = 1.8\qquad 
\end{align*}
L'\key{inégalité} des quotients indique qu'\key{il ne sagit pas} d'une situation de proportionnalité.

\item\begin{align*}
\frac{469.44}{97.8} = 4.8\qquad \frac{171.36}{35.7} = 4.8\qquad 
\end{align*}
L'\key{égalité} des quotients indique qu'\key{il s'agit bien} d'une situation de proportionnalité.

\item\begin{align*}
\frac{18.16}{22.7} = 0.8\qquad \frac{6.32}{7.9} = 0.8\qquad \frac{64.48}{80.6} = 0.8\qquad \frac{38.56}{48.2} = 0.8\qquad 
\end{align*}
L'\key{égalité} des quotients indique qu'\key{il s'agit bien} d'une situation de proportionnalité.

\item\begin{align*}
\frac{56.84}{11.6} = 4.9\qquad \frac{281.06}{61.1} = 4.6\qquad \frac{389.76}{81.2} = 4.8\qquad \frac{331.73}{67.7} = 4.9\qquad 
\end{align*}
L'\key{inégalité} des quotients indique qu'\key{il ne sagit pas} d'une situation de proportionnalité.

\item\begin{align*}
\frac{101.83}{59.9} = 1.7\qquad \frac{29.07}{17.1} = 1.7\qquad \frac{80.07}{47.1} = 1.7\qquad \frac{109.65}{64.5} = 1.7\qquad 
\end{align*}
L'\key{égalité} des quotients indique qu'\key{il s'agit bien} d'une situation de proportionnalité.

\item\begin{align*}
\frac{214.5}{85.8} = 2.5\qquad \frac{123.25}{49.3} = 2.5\qquad 
\end{align*}
L'\key{égalité} des quotients indique qu'\key{il s'agit bien} d'une situation de proportionnalité.

\item\begin{align*}
\frac{261.8}{77} = 3.4\qquad \frac{108.16}{33.8} = 3.2\qquad \frac{150.04}{48.4} = 3.1\qquad 
\end{align*}
L'\key{inégalité} des quotients indique qu'\key{il ne sagit pas} d'une situation de proportionnalité.

\item\begin{align*}
\frac{58}{29} = 2\qquad \frac{101.4}{50.7} = 2\qquad \frac{33.2}{16.6} = 2\qquad \frac{118.2}{59.1} = 2\qquad 
\end{align*}
L'\key{égalité} des quotients indique qu'\key{il s'agit bien} d'une situation de proportionnalité.

\item\begin{align*}
\frac{72.1}{20.6} = 3.5\qquad \frac{197.28}{54.8} = 3.6\qquad \frac{336.6}{93.5} = 3.6\qquad 
\end{align*}
L'\key{inégalité} des quotients indique qu'\key{il ne sagit pas} d'une situation de proportionnalité.

\item\begin{align*}
\frac{29.25}{19.5} = 1.5\qquad \frac{66.13}{38.9} = 1.7\qquad \frac{127.36}{79.6} = 1.6\qquad 
\end{align*}
L'\key{inégalité} des quotients indique qu'\key{il ne sagit pas} d'une situation de proportionnalité.

\item\begin{align*}
\frac{71.25}{47.5} = 1.5\qquad \frac{36.9}{24.6} = 1.5\qquad 
\end{align*}
L'\key{inégalité} des quotients indique qu'\key{il ne sagit pas} d'une situation de proportionnalité.

\item\begin{align*}
\frac{4.86}{8.1} = 0.6\qquad \frac{8.67}{28.9} = 0.3\qquad \frac{20.28}{67.6} = 0.3\qquad \frac{20.55}{41.1} = 0.5\qquad 
\end{align*}
L'\key{inégalité} des quotients indique qu'\key{il ne sagit pas} d'une situation de proportionnalité.

\item\begin{align*}
\frac{10.92}{36.4} = 0.3\qquad \frac{5.07}{16.9} = 0.3\qquad 
\end{align*}
L'\key{égalité} des quotients indique qu'\key{il s'agit bien} d'une situation de proportionnalité.

\item\begin{align*}
\frac{123.28}{26.8} = 4.6\qquad \frac{152.55}{33.9} = 4.5\qquad \frac{222.18}{48.3} = 4.6\qquad \frac{280.14}{60.9} = 4.6\qquad 
\end{align*}
L'\key{inégalité} des quotients indique qu'\key{il ne sagit pas} d'une situation de proportionnalité.

\item\begin{align*}
\frac{35.28}{19.6} = 1.8\qquad \frac{58.14}{32.3} = 1.8\qquad 
\end{align*}
L'\key{égalité} des quotients indique qu'\key{il s'agit bien} d'une situation de proportionnalité.

\item\begin{align*}
\frac{51}{20.4} = 2.5\qquad \frac{121.75}{48.7} = 2.5\qquad \frac{178.36}{68.6} = 2.6\qquad \frac{125.58}{54.6} = 2.3\qquad 
\end{align*}
L'\key{inégalité} des quotients indique qu'\key{il ne sagit pas} d'une situation de proportionnalité.

\item\begin{align*}
\frac{209.96}{72.4} = 2.9\qquad \frac{40.23}{14.9} = 2.7\qquad 
\end{align*}
L'\key{inégalité} des quotients indique qu'\key{il ne sagit pas} d'une situation de proportionnalité.

\item\begin{align*}
\frac{94.64}{18.2} = 5.2\qquad \frac{13.77}{2.7} = 5.1\qquad \frac{279}{55.8} = 5\qquad \frac{378.93}{74.3} = 5.1\qquad 
\end{align*}
L'\key{inégalité} des quotients indique qu'\key{il ne sagit pas} d'une situation de proportionnalité.

\item\begin{align*}
\frac{21.76}{12.8} = 1.7\qquad \frac{7.48}{4.4} = 1.7\qquad \frac{17.6}{11} = 1.6\qquad 
\end{align*}
L'\key{inégalité} des quotients indique qu'\key{il ne sagit pas} d'une situation de proportionnalité.

\item\begin{align*}
\frac{369}{90} = 4.1\qquad \frac{43}{10} = 4.3\qquad 
\end{align*}
L'\key{inégalité} des quotients indique qu'\key{il ne sagit pas} d'une situation de proportionnalité.

\item\begin{align*}
\frac{267.43}{56.9} = 4.7\qquad \frac{339.02}{73.7} = 4.6\qquad \frac{401.58}{87.3} = 4.6\qquad 
\end{align*}
L'\key{inégalité} des quotients indique qu'\key{il ne sagit pas} d'une situation de proportionnalité.

\item\begin{align*}
\frac{16.53}{8.7} = 1.9\qquad \frac{178.6}{94} = 1.9\qquad 
\end{align*}
L'\key{égalité} des quotients indique qu'\key{il s'agit bien} d'une situation de proportionnalité.

\item\begin{align*}
\frac{36.96}{11.2} = 3.3\qquad \frac{40.59}{12.3} = 3.3\qquad \frac{26.4}{8} = 3.3\qquad \frac{22.77}{6.9} = 3.3\qquad 
\end{align*}
L'\key{égalité} des quotients indique qu'\key{il s'agit bien} d'une situation de proportionnalité.

\item\begin{align*}
\frac{167.67}{72.9} = 2.3\qquad \frac{86.02}{37.4} = 2.3\qquad \frac{24.61}{10.7} = 2.3\qquad 
\end{align*}
L'\key{égalité} des quotients indique qu'\key{il s'agit bien} d'une situation de proportionnalité.

\item\begin{align*}
\frac{159.16}{69.2} = 2.3\qquad \frac{95.76}{39.9} = 2.4\qquad \frac{220.56}{91.9} = 2.4\qquad \frac{128.64}{53.6} = 2.4\qquad 
\end{align*}
L'\key{inégalité} des quotients indique qu'\key{il ne sagit pas} d'une situation de proportionnalité.

\item\begin{align*}
\frac{327.32}{66.8} = 4.9\qquad \frac{328.8}{68.5} = 4.8\qquad \frac{210.21}{42.9} = 4.9\qquad 
\end{align*}
L'\key{inégalité} des quotients indique qu'\key{il ne sagit pas} d'une situation de proportionnalité.

\item\begin{align*}
\frac{252.48}{78.9} = 3.2\qquad \frac{92.8}{29} = 3.2\qquad \frac{205.76}{64.3} = 3.2\qquad \frac{48}{15} = 3.2\qquad 
\end{align*}
L'\key{égalité} des quotients indique qu'\key{il s'agit bien} d'une situation de proportionnalité.

\item\begin{align*}
\frac{130.41}{62.1} = 2.1\qquad \frac{20.46}{9.3} = 2.2\qquad \frac{169.4}{77} = 2.2\qquad 
\end{align*}
L'\key{inégalité} des quotients indique qu'\key{il ne sagit pas} d'une situation de proportionnalité.

\end{enumerate}}
% % VARIABLES %%%
\setTitle{Interrogation - Entrainement - Séquence 1}
\setGrade{4e}
%%

\exo{Écrire sous forme de fractions irréductibles :}{%
\multiColEnumerate{6}{
\item$\frac{5500}{275}$

\item$\frac{33}{72}$

\item$\frac{2548}{8112}$

\item$\frac{2184}{21}$

\item$\frac{99372}{4840}$

\item$\frac{4620}{130}$

\item$\frac{770}{572}$

\item$\frac{66}{6}$

\item$\frac{70}{91}$

\item$\frac{1618617}{78}$

\item$\frac{5544}{4158}$

\item$\frac{12}{622545}$

\item$\frac{25}{42}$

\item$\frac{784}{33}$

\item$\frac{248430}{14014}$

\item$\frac{4914}{220}$

\item$\frac{660}{8250}$

\item$\frac{7280}{1540}$

\item$\frac{76050}{3780}$

\item$\frac{507}{1087515}$

\item$\frac{10}{588}$

\item$\frac{3675}{6}$

\item$\frac{91}{572}$

\item$\frac{1375}{5005}$

\item$\frac{429}{104}$

\item$\frac{362505}{9438}$

\item$\frac{15}{1386}$

\item$\frac{275275}{1001}$

\item$\frac{26}{350350}$

\item$\frac{2556125}{248430}$

\item$\frac{102245}{231}$

\item$\frac{5250}{2860}$

\item$\frac{108}{22}$

\item$\frac{330}{33957}$

\item$\frac{726}{420}$

\item$\frac{210}{27}$

\item$\frac{308}{8316}$

\item$\frac{10}{484}$

\item$\frac{78}{372645}$

\item$\frac{5148}{5460}$

\item$\frac{1274}{7872865}$

\item$\frac{195}{2640}$

\item$\frac{65}{520}$

\item$\frac{168}{22}$

\item$\frac{84084}{1144}$

\item$\frac{280}{13552}$

\item$\frac{11550}{1386}$

\item$\frac{15}{153790}$

\item$\frac{1470}{1617}$

\item$\frac{546}{504}$

\item$\frac{429}{2600}$

\item$\frac{8}{42588}$

\item$\frac{2366}{3300}$

\item$\frac{20}{5775}$

\item$\frac{819}{390}$

\item$\frac{175}{6}$

\item$\frac{650}{10010}$

\item$\frac{3120}{168}$

\item$\frac{13860}{1680}$

\item$\frac{50820}{10}$

\item$\frac{98}{11011}$

\item$\frac{308}{1014}$

\item$\frac{51975}{48}$

\item$\frac{35}{676}$

\item$\frac{4}{2860}$

\item$\frac{84}{280}$

\item$\frac{3432}{484}$

\item$\frac{5005}{5915}$

\item$\frac{1014}{14}$

\item$\frac{12}{630}$

\item$\frac{1848}{364}$

\item$\frac{4}{1680}$

\item$\frac{15288}{296450}$

\item$\frac{14520}{286}$

\item$\frac{195}{22}$

\item$\frac{286}{6}$

\item$\frac{352}{330}$

\item$\frac{2310}{6468}$

\item$\frac{650650}{80}$

\item$\frac{2184}{27027}$

\item$\frac{252}{10164}$

\item$\frac{14014}{150}$

\item$\frac{10}{14}$

\item$\frac{4620}{15}$

\item$\frac{1287}{16}$

\item$\frac{85995}{936}$

\item$\frac{14742}{385}$

\item$\frac{3003}{1274}$

\item$\frac{37180}{5460}$

\item$\frac{975975}{4}$

\item$\frac{70}{131820}$

\item$\frac{9800}{140}$

\item$\frac{10}{6468}$

\item$\frac{15400}{700}$

\item$\frac{210}{30}$

\item$\frac{33}{780}$

\item$\frac{32340}{217503}$

\item$\frac{792}{165}$

\item$\frac{126}{264}$

\item$\frac{936}{1320}$

}}

\newpage

\corr{}{%
\multiColEnumerate{2}{
\item\begin{align*}
    \frac{5500}{275} &=
    \frac{2 \times 2 \times \cancel{5} \times \cancel{5} \times 5 \times \cancel{11}}
    {\cancel{5} \times \cancel{5} \times \cancel{11}}\\ &=
    \frac{2 \times 2 \times 5}
    {1} =
    \frac{20}{1}
    \end{align*}

\item\begin{align*}
    \frac{33}{72} &=
    \frac{\cancel{3} \times 11}
    {2 \times 2 \times 2 \times \cancel{3} \times 3}\\ &=
    \frac{11}
    {2 \times 2 \times 2 \times 3} =
    \frac{11}{24}
    \end{align*}

\item\begin{align*}
    \frac{2548}{8112} &=
    \frac{\cancel{2} \times \cancel{2} \times 7 \times 7 \times \cancel{13}}
    {\cancel{2} \times \cancel{2} \times 2 \times 2 \times 3 \times \cancel{13} \times 13}\\ &=
    \frac{7 \times 7}
    {2 \times 2 \times 3 \times 13} =
    \frac{49}{156}
    \end{align*}

\item\begin{align*}
    \frac{2184}{21} &=
    \frac{2 \times 2 \times 2 \times \cancel{3} \times \cancel{7} \times 13}
    {\cancel{3} \times \cancel{7}}\\ &=
    \frac{2 \times 2 \times 2 \times 13}
    {1} =
    \frac{104}{1}
    \end{align*}

\item\begin{align*}
    \frac{99372}{4840} &=
    \frac{\cancel{2} \times \cancel{2} \times 3 \times 7 \times 7 \times 13 \times 13}
    {\cancel{2} \times \cancel{2} \times 2 \times 5 \times 11 \times 11}\\ &=
    \frac{3 \times 7 \times 7 \times 13 \times 13}
    {2 \times 5 \times 11 \times 11} =
    \frac{24843}{1210}
    \end{align*}

\item\begin{align*}
    \frac{4620}{130} &=
    \frac{\cancel{2} \times 2 \times 3 \times \cancel{5} \times 7 \times 11}
    {\cancel{2} \times \cancel{5} \times 13}\\ &=
    \frac{2 \times 3 \times 7 \times 11}
    {13} =
    \frac{462}{13}
    \end{align*}

\item\begin{align*}
    \frac{770}{572} &=
    \frac{\cancel{2} \times 5 \times 7 \times \cancel{11}}
    {\cancel{2} \times 2 \times \cancel{11} \times 13}\\ &=
    \frac{5 \times 7}
    {2 \times 13} =
    \frac{35}{26}
    \end{align*}

\item\begin{align*}
    \frac{66}{6} &=
    \frac{\cancel{2} \times \cancel{3} \times 11}
    {\cancel{2} \times \cancel{3}}\\ &=
    \frac{11}
    {1} =
    \frac{11}{1}
    \end{align*}

\item\begin{align*}
    \frac{70}{91} &=
    \frac{2 \times 5 \times \cancel{7}}
    {\cancel{7} \times 13}\\ &=
    \frac{2 \times 5}
    {13} =
    \frac{10}{13}
    \end{align*}

\item\begin{align*}
    \frac{1618617}{78} &=
    \frac{\cancel{3} \times 7 \times 7 \times 7 \times 11 \times 11 \times \cancel{13}}
    {2 \times \cancel{3} \times \cancel{13}}\\ &=
    \frac{7 \times 7 \times 7 \times 11 \times 11}
    {2} =
    \frac{41503}{2}
    \end{align*}

\item\begin{align*}
    \frac{5544}{4158} &=
    \frac{\cancel{2} \times 2 \times 2 \times \cancel{3} \times \cancel{3} \times \cancel{7} \times \cancel{11}}
    {\cancel{2} \times \cancel{3} \times \cancel{3} \times 3 \times \cancel{7} \times \cancel{11}}\\ &=
    \frac{2 \times 2}
    {3} =
    \frac{4}{3}
    \end{align*}

\item\begin{align*}
    \frac{12}{622545} &=
    \frac{2 \times 2 \times \cancel{3}}
    {\cancel{3} \times 5 \times 7 \times 7 \times 7 \times 11 \times 11}\\ &=
    \frac{2 \times 2}
    {5 \times 7 \times 7 \times 7 \times 11 \times 11} =
    \frac{4}{207515}
    \end{align*}

\item\begin{align*}
    \frac{25}{42} &=
    \frac{5 \times 5}
    {2 \times 3 \times 7}\\ &=
    \frac{5 \times 5}
    {2 \times 3 \times 7} =
    \frac{25}{42}
    \end{align*}

\item\begin{align*}
    \frac{784}{33} &=
    \frac{2 \times 2 \times 2 \times 2 \times 7 \times 7}
    {3 \times 11}\\ &=
    \frac{2 \times 2 \times 2 \times 2 \times 7 \times 7}
    {3 \times 11} =
    \frac{784}{33}
    \end{align*}

\item\begin{align*}
    \frac{248430}{14014} &=
    \frac{\cancel{2} \times 3 \times 5 \times \cancel{7} \times \cancel{7} \times \cancel{13} \times 13}
    {\cancel{2} \times \cancel{7} \times \cancel{7} \times 11 \times \cancel{13}}\\ &=
    \frac{3 \times 5 \times 13}
    {11} =
    \frac{195}{11}
    \end{align*}

\item\begin{align*}
    \frac{4914}{220} &=
    \frac{\cancel{2} \times 3 \times 3 \times 3 \times 7 \times 13}
    {\cancel{2} \times 2 \times 5 \times 11}\\ &=
    \frac{3 \times 3 \times 3 \times 7 \times 13}
    {2 \times 5 \times 11} =
    \frac{2457}{110}
    \end{align*}

\item\begin{align*}
    \frac{660}{8250} &=
    \frac{\cancel{2} \times 2 \times \cancel{3} \times \cancel{5} \times \cancel{11}}
    {\cancel{2} \times \cancel{3} \times \cancel{5} \times 5 \times 5 \times \cancel{11}}\\ &=
    \frac{2}
    {5 \times 5} =
    \frac{2}{25}
    \end{align*}

\item\begin{align*}
    \frac{7280}{1540} &=
    \frac{\cancel{2} \times \cancel{2} \times 2 \times 2 \times \cancel{5} \times \cancel{7} \times 13}
    {\cancel{2} \times \cancel{2} \times \cancel{5} \times \cancel{7} \times 11}\\ &=
    \frac{2 \times 2 \times 13}
    {11} =
    \frac{52}{11}
    \end{align*}

\item\begin{align*}
    \frac{76050}{3780} &=
    \frac{\cancel{2} \times \cancel{3} \times \cancel{3} \times \cancel{5} \times 5 \times 13 \times 13}
    {\cancel{2} \times 2 \times \cancel{3} \times \cancel{3} \times 3 \times \cancel{5} \times 7}\\ &=
    \frac{5 \times 13 \times 13}
    {2 \times 3 \times 7} =
    \frac{845}{42}
    \end{align*}

\item\begin{align*}
    \frac{507}{1087515} &=
    \frac{\cancel{3} \times \cancel{13} \times \cancel{13}}
    {\cancel{3} \times 3 \times 5 \times 11 \times \cancel{13} \times \cancel{13} \times 13}\\ &=
    \frac{1}
    {3 \times 5 \times 11 \times 13} =
    \frac{1}{2145}
    \end{align*}

\item\begin{align*}
    \frac{10}{588} &=
    \frac{\cancel{2} \times 5}
    {\cancel{2} \times 2 \times 3 \times 7 \times 7}\\ &=
    \frac{5}
    {2 \times 3 \times 7 \times 7} =
    \frac{5}{294}
    \end{align*}

\item\begin{align*}
    \frac{3675}{6} &=
    \frac{\cancel{3} \times 5 \times 5 \times 7 \times 7}
    {2 \times \cancel{3}}\\ &=
    \frac{5 \times 5 \times 7 \times 7}
    {2} =
    \frac{1225}{2}
    \end{align*}

\item\begin{align*}
    \frac{91}{572} &=
    \frac{7 \times \cancel{13}}
    {2 \times 2 \times 11 \times \cancel{13}}\\ &=
    \frac{7}
    {2 \times 2 \times 11} =
    \frac{7}{44}
    \end{align*}

\item\begin{align*}
    \frac{1375}{5005} &=
    \frac{\cancel{5} \times 5 \times 5 \times \cancel{11}}
    {\cancel{5} \times 7 \times \cancel{11} \times 13}\\ &=
    \frac{5 \times 5}
    {7 \times 13} =
    \frac{25}{91}
    \end{align*}

\item\begin{align*}
    \frac{429}{104} &=
    \frac{3 \times 11 \times \cancel{13}}
    {2 \times 2 \times 2 \times \cancel{13}}\\ &=
    \frac{3 \times 11}
    {2 \times 2 \times 2} =
    \frac{33}{8}
    \end{align*}

\item\begin{align*}
    \frac{362505}{9438} &=
    \frac{\cancel{3} \times 5 \times \cancel{11} \times \cancel{13} \times 13 \times 13}
    {2 \times \cancel{3} \times \cancel{11} \times 11 \times \cancel{13}}\\ &=
    \frac{5 \times 13 \times 13}
    {2 \times 11} =
    \frac{845}{22}
    \end{align*}

\item\begin{align*}
    \frac{15}{1386} &=
    \frac{\cancel{3} \times 5}
    {2 \times \cancel{3} \times 3 \times 7 \times 11}\\ &=
    \frac{5}
    {2 \times 3 \times 7 \times 11} =
    \frac{5}{462}
    \end{align*}

\item\begin{align*}
    \frac{275275}{1001} &=
    \frac{5 \times 5 \times \cancel{7} \times \cancel{11} \times 11 \times \cancel{13}}
    {\cancel{7} \times \cancel{11} \times \cancel{13}}\\ &=
    \frac{5 \times 5 \times 11}
    {1} =
    \frac{275}{1}
    \end{align*}

\item\begin{align*}
    \frac{26}{350350} &=
    \frac{\cancel{2} \times \cancel{13}}
    {\cancel{2} \times 5 \times 5 \times 7 \times 7 \times 11 \times \cancel{13}}\\ &=
    \frac{1}
    {5 \times 5 \times 7 \times 7 \times 11} =
    \frac{1}{13475}
    \end{align*}

\item\begin{align*}
    \frac{2556125}{248430} &=
    \frac{\cancel{5} \times 5 \times 5 \times 11 \times 11 \times \cancel{13} \times \cancel{13}}
    {2 \times 3 \times \cancel{5} \times 7 \times 7 \times \cancel{13} \times \cancel{13}}\\ &=
    \frac{5 \times 5 \times 11 \times 11}
    {2 \times 3 \times 7 \times 7} =
    \frac{3025}{294}
    \end{align*}

\item\begin{align*}
    \frac{102245}{231} &=
    \frac{5 \times \cancel{11} \times 11 \times 13 \times 13}
    {3 \times 7 \times \cancel{11}}\\ &=
    \frac{5 \times 11 \times 13 \times 13}
    {3 \times 7} =
    \frac{9295}{21}
    \end{align*}

\item\begin{align*}
    \frac{5250}{2860} &=
    \frac{\cancel{2} \times 3 \times \cancel{5} \times 5 \times 5 \times 7}
    {\cancel{2} \times 2 \times \cancel{5} \times 11 \times 13}\\ &=
    \frac{3 \times 5 \times 5 \times 7}
    {2 \times 11 \times 13} =
    \frac{525}{286}
    \end{align*}

\item\begin{align*}
    \frac{108}{22} &=
    \frac{\cancel{2} \times 2 \times 3 \times 3 \times 3}
    {\cancel{2} \times 11}\\ &=
    \frac{2 \times 3 \times 3 \times 3}
    {11} =
    \frac{54}{11}
    \end{align*}

\item\begin{align*}
    \frac{330}{33957} &=
    \frac{2 \times \cancel{3} \times 5 \times \cancel{11}}
    {\cancel{3} \times 3 \times 7 \times 7 \times 7 \times \cancel{11}}\\ &=
    \frac{2 \times 5}
    {3 \times 7 \times 7 \times 7} =
    \frac{10}{1029}
    \end{align*}

\item\begin{align*}
    \frac{726}{420} &=
    \frac{\cancel{2} \times \cancel{3} \times 11 \times 11}
    {\cancel{2} \times 2 \times \cancel{3} \times 5 \times 7}\\ &=
    \frac{11 \times 11}
    {2 \times 5 \times 7} =
    \frac{121}{70}
    \end{align*}

\item\begin{align*}
    \frac{210}{27} &=
    \frac{2 \times \cancel{3} \times 5 \times 7}
    {\cancel{3} \times 3 \times 3}\\ &=
    \frac{2 \times 5 \times 7}
    {3 \times 3} =
    \frac{70}{9}
    \end{align*}

\item\begin{align*}
    \frac{308}{8316} &=
    \frac{\cancel{2} \times \cancel{2} \times \cancel{7} \times \cancel{11}}
    {\cancel{2} \times \cancel{2} \times 3 \times 3 \times 3 \times \cancel{7} \times \cancel{11}}\\ &=
    \frac{1}
    {3 \times 3 \times 3} =
    \frac{1}{27}
    \end{align*}

\item\begin{align*}
    \frac{10}{484} &=
    \frac{\cancel{2} \times 5}
    {\cancel{2} \times 2 \times 11 \times 11}\\ &=
    \frac{5}
    {2 \times 11 \times 11} =
    \frac{5}{242}
    \end{align*}

\item\begin{align*}
    \frac{78}{372645} &=
    \frac{2 \times \cancel{3} \times \cancel{13}}
    {\cancel{3} \times 3 \times 5 \times 7 \times 7 \times \cancel{13} \times 13}\\ &=
    \frac{2}
    {3 \times 5 \times 7 \times 7 \times 13} =
    \frac{2}{9555}
    \end{align*}

\item\begin{align*}
    \frac{5148}{5460} &=
    \frac{\cancel{2} \times \cancel{2} \times \cancel{3} \times 3 \times 11 \times \cancel{13}}
    {\cancel{2} \times \cancel{2} \times \cancel{3} \times 5 \times 7 \times \cancel{13}}\\ &=
    \frac{3 \times 11}
    {5 \times 7} =
    \frac{33}{35}
    \end{align*}

\item\begin{align*}
    \frac{1274}{7872865} &=
    \frac{2 \times \cancel{7} \times 7 \times \cancel{13}}
    {5 \times \cancel{7} \times 11 \times 11 \times 11 \times \cancel{13} \times 13}\\ &=
    \frac{2 \times 7}
    {5 \times 11 \times 11 \times 11 \times 13} =
    \frac{14}{86515}
    \end{align*}

\item\begin{align*}
    \frac{195}{2640} &=
    \frac{\cancel{3} \times \cancel{5} \times 13}
    {2 \times 2 \times 2 \times 2 \times \cancel{3} \times \cancel{5} \times 11}\\ &=
    \frac{13}
    {2 \times 2 \times 2 \times 2 \times 11} =
    \frac{13}{176}
    \end{align*}

\item\begin{align*}
    \frac{65}{520} &=
    \frac{\cancel{5} \times \cancel{13}}
    {2 \times 2 \times 2 \times \cancel{5} \times \cancel{13}}\\ &=
    \frac{1}
    {2 \times 2 \times 2} =
    \frac{1}{8}
    \end{align*}

\item\begin{align*}
    \frac{168}{22} &=
    \frac{\cancel{2} \times 2 \times 2 \times 3 \times 7}
    {\cancel{2} \times 11}\\ &=
    \frac{2 \times 2 \times 3 \times 7}
    {11} =
    \frac{84}{11}
    \end{align*}

\item\begin{align*}
    \frac{84084}{1144} &=
    \frac{\cancel{2} \times \cancel{2} \times 3 \times 7 \times 7 \times \cancel{11} \times \cancel{13}}
    {\cancel{2} \times \cancel{2} \times 2 \times \cancel{11} \times \cancel{13}}\\ &=
    \frac{3 \times 7 \times 7}
    {2} =
    \frac{147}{2}
    \end{align*}

\item\begin{align*}
    \frac{280}{13552} &=
    \frac{\cancel{2} \times \cancel{2} \times \cancel{2} \times 5 \times \cancel{7}}
    {\cancel{2} \times \cancel{2} \times \cancel{2} \times 2 \times \cancel{7} \times 11 \times 11}\\ &=
    \frac{5}
    {2 \times 11 \times 11} =
    \frac{5}{242}
    \end{align*}

\item\begin{align*}
    \frac{11550}{1386} &=
    \frac{\cancel{2} \times \cancel{3} \times 5 \times 5 \times \cancel{7} \times \cancel{11}}
    {\cancel{2} \times \cancel{3} \times 3 \times \cancel{7} \times \cancel{11}}\\ &=
    \frac{5 \times 5}
    {3} =
    \frac{25}{3}
    \end{align*}

\item\begin{align*}
    \frac{15}{153790} &=
    \frac{3 \times \cancel{5}}
    {2 \times \cancel{5} \times 7 \times 13 \times 13 \times 13}\\ &=
    \frac{3}
    {2 \times 7 \times 13 \times 13 \times 13} =
    \frac{3}{30758}
    \end{align*}

\item\begin{align*}
    \frac{1470}{1617} &=
    \frac{2 \times \cancel{3} \times 5 \times \cancel{7} \times \cancel{7}}
    {\cancel{3} \times \cancel{7} \times \cancel{7} \times 11}\\ &=
    \frac{2 \times 5}
    {11} =
    \frac{10}{11}
    \end{align*}

\item\begin{align*}
    \frac{546}{504} &=
    \frac{\cancel{2} \times \cancel{3} \times \cancel{7} \times 13}
    {\cancel{2} \times 2 \times 2 \times \cancel{3} \times 3 \times \cancel{7}}\\ &=
    \frac{13}
    {2 \times 2 \times 3} =
    \frac{13}{12}
    \end{align*}

\item\begin{align*}
    \frac{429}{2600} &=
    \frac{3 \times 11 \times \cancel{13}}
    {2 \times 2 \times 2 \times 5 \times 5 \times \cancel{13}}\\ &=
    \frac{3 \times 11}
    {2 \times 2 \times 2 \times 5 \times 5} =
    \frac{33}{200}
    \end{align*}

\item\begin{align*}
    \frac{8}{42588} &=
    \frac{\cancel{2} \times \cancel{2} \times 2}
    {\cancel{2} \times \cancel{2} \times 3 \times 3 \times 7 \times 13 \times 13}\\ &=
    \frac{2}
    {3 \times 3 \times 7 \times 13 \times 13} =
    \frac{2}{10647}
    \end{align*}

\item\begin{align*}
    \frac{2366}{3300} &=
    \frac{\cancel{2} \times 7 \times 13 \times 13}
    {\cancel{2} \times 2 \times 3 \times 5 \times 5 \times 11}\\ &=
    \frac{7 \times 13 \times 13}
    {2 \times 3 \times 5 \times 5 \times 11} =
    \frac{1183}{1650}
    \end{align*}

\item\begin{align*}
    \frac{20}{5775} &=
    \frac{2 \times 2 \times \cancel{5}}
    {3 \times \cancel{5} \times 5 \times 7 \times 11}\\ &=
    \frac{2 \times 2}
    {3 \times 5 \times 7 \times 11} =
    \frac{4}{1155}
    \end{align*}

\item\begin{align*}
    \frac{819}{390} &=
    \frac{\cancel{3} \times 3 \times 7 \times \cancel{13}}
    {2 \times \cancel{3} \times 5 \times \cancel{13}}\\ &=
    \frac{3 \times 7}
    {2 \times 5} =
    \frac{21}{10}
    \end{align*}

\item\begin{align*}
    \frac{175}{6} &=
    \frac{5 \times 5 \times 7}
    {2 \times 3}\\ &=
    \frac{5 \times 5 \times 7}
    {2 \times 3} =
    \frac{175}{6}
    \end{align*}

\item\begin{align*}
    \frac{650}{10010} &=
    \frac{\cancel{2} \times \cancel{5} \times 5 \times \cancel{13}}
    {\cancel{2} \times \cancel{5} \times 7 \times 11 \times \cancel{13}}\\ &=
    \frac{5}
    {7 \times 11} =
    \frac{5}{77}
    \end{align*}

\item\begin{align*}
    \frac{3120}{168} &=
    \frac{\cancel{2} \times \cancel{2} \times \cancel{2} \times 2 \times \cancel{3} \times 5 \times 13}
    {\cancel{2} \times \cancel{2} \times \cancel{2} \times \cancel{3} \times 7}\\ &=
    \frac{2 \times 5 \times 13}
    {7} =
    \frac{130}{7}
    \end{align*}

\item\begin{align*}
    \frac{13860}{1680} &=
    \frac{\cancel{2} \times \cancel{2} \times \cancel{3} \times 3 \times \cancel{5} \times \cancel{7} \times 11}
    {\cancel{2} \times \cancel{2} \times 2 \times 2 \times \cancel{3} \times \cancel{5} \times \cancel{7}}\\ &=
    \frac{3 \times 11}
    {2 \times 2} =
    \frac{33}{4}
    \end{align*}

\item\begin{align*}
    \frac{50820}{10} &=
    \frac{\cancel{2} \times 2 \times 3 \times \cancel{5} \times 7 \times 11 \times 11}
    {\cancel{2} \times \cancel{5}}\\ &=
    \frac{2 \times 3 \times 7 \times 11 \times 11}
    {1} =
    \frac{5082}{1}
    \end{align*}

\item\begin{align*}
    \frac{98}{11011} &=
    \frac{2 \times \cancel{7} \times 7}
    {\cancel{7} \times 11 \times 11 \times 13}\\ &=
    \frac{2 \times 7}
    {11 \times 11 \times 13} =
    \frac{14}{1573}
    \end{align*}

\item\begin{align*}
    \frac{308}{1014} &=
    \frac{\cancel{2} \times 2 \times 7 \times 11}
    {\cancel{2} \times 3 \times 13 \times 13}\\ &=
    \frac{2 \times 7 \times 11}
    {3 \times 13 \times 13} =
    \frac{154}{507}
    \end{align*}

\item\begin{align*}
    \frac{51975}{48} &=
    \frac{\cancel{3} \times 3 \times 3 \times 5 \times 5 \times 7 \times 11}
    {2 \times 2 \times 2 \times 2 \times \cancel{3}}\\ &=
    \frac{3 \times 3 \times 5 \times 5 \times 7 \times 11}
    {2 \times 2 \times 2 \times 2} =
    \frac{17325}{16}
    \end{align*}

\item\begin{align*}
    \frac{35}{676} &=
    \frac{5 \times 7}
    {2 \times 2 \times 13 \times 13}\\ &=
    \frac{5 \times 7}
    {2 \times 2 \times 13 \times 13} =
    \frac{35}{676}
    \end{align*}

\item\begin{align*}
    \frac{4}{2860} &=
    \frac{\cancel{2} \times \cancel{2}}
    {\cancel{2} \times \cancel{2} \times 5 \times 11 \times 13}\\ &=
    \frac{1}
    {5 \times 11 \times 13} =
    \frac{1}{715}
    \end{align*}

\item\begin{align*}
    \frac{84}{280} &=
    \frac{\cancel{2} \times \cancel{2} \times 3 \times \cancel{7}}
    {\cancel{2} \times \cancel{2} \times 2 \times 5 \times \cancel{7}}\\ &=
    \frac{3}
    {2 \times 5} =
    \frac{3}{10}
    \end{align*}

\item\begin{align*}
    \frac{3432}{484} &=
    \frac{\cancel{2} \times \cancel{2} \times 2 \times 3 \times \cancel{11} \times 13}
    {\cancel{2} \times \cancel{2} \times \cancel{11} \times 11}\\ &=
    \frac{2 \times 3 \times 13}
    {11} =
    \frac{78}{11}
    \end{align*}

\item\begin{align*}
    \frac{5005}{5915} &=
    \frac{\cancel{5} \times \cancel{7} \times 11 \times \cancel{13}}
    {\cancel{5} \times \cancel{7} \times \cancel{13} \times 13}\\ &=
    \frac{11}
    {13} =
    \frac{11}{13}
    \end{align*}

\item\begin{align*}
    \frac{1014}{14} &=
    \frac{\cancel{2} \times 3 \times 13 \times 13}
    {\cancel{2} \times 7}\\ &=
    \frac{3 \times 13 \times 13}
    {7} =
    \frac{507}{7}
    \end{align*}

\item\begin{align*}
    \frac{12}{630} &=
    \frac{\cancel{2} \times 2 \times \cancel{3}}
    {\cancel{2} \times \cancel{3} \times 3 \times 5 \times 7}\\ &=
    \frac{2}
    {3 \times 5 \times 7} =
    \frac{2}{105}
    \end{align*}

\item\begin{align*}
    \frac{1848}{364} &=
    \frac{\cancel{2} \times \cancel{2} \times 2 \times 3 \times \cancel{7} \times 11}
    {\cancel{2} \times \cancel{2} \times \cancel{7} \times 13}\\ &=
    \frac{2 \times 3 \times 11}
    {13} =
    \frac{66}{13}
    \end{align*}

\item\begin{align*}
    \frac{4}{1680} &=
    \frac{\cancel{2} \times \cancel{2}}
    {\cancel{2} \times \cancel{2} \times 2 \times 2 \times 3 \times 5 \times 7}\\ &=
    \frac{1}
    {2 \times 2 \times 3 \times 5 \times 7} =
    \frac{1}{420}
    \end{align*}

\item\begin{align*}
    \frac{15288}{296450} &=
    \frac{\cancel{2} \times 2 \times 2 \times 3 \times \cancel{7} \times \cancel{7} \times 13}
    {\cancel{2} \times 5 \times 5 \times \cancel{7} \times \cancel{7} \times 11 \times 11}\\ &=
    \frac{2 \times 2 \times 3 \times 13}
    {5 \times 5 \times 11 \times 11} =
    \frac{156}{3025}
    \end{align*}

\item\begin{align*}
    \frac{14520}{286} &=
    \frac{\cancel{2} \times 2 \times 2 \times 3 \times 5 \times \cancel{11} \times 11}
    {\cancel{2} \times \cancel{11} \times 13}\\ &=
    \frac{2 \times 2 \times 3 \times 5 \times 11}
    {13} =
    \frac{660}{13}
    \end{align*}

\item\begin{align*}
    \frac{195}{22} &=
    \frac{3 \times 5 \times 13}
    {2 \times 11}\\ &=
    \frac{3 \times 5 \times 13}
    {2 \times 11} =
    \frac{195}{22}
    \end{align*}

\item\begin{align*}
    \frac{286}{6} &=
    \frac{\cancel{2} \times 11 \times 13}
    {\cancel{2} \times 3}\\ &=
    \frac{11 \times 13}
    {3} =
    \frac{143}{3}
    \end{align*}

\item\begin{align*}
    \frac{352}{330} &=
    \frac{\cancel{2} \times 2 \times 2 \times 2 \times 2 \times \cancel{11}}
    {\cancel{2} \times 3 \times 5 \times \cancel{11}}\\ &=
    \frac{2 \times 2 \times 2 \times 2}
    {3 \times 5} =
    \frac{16}{15}
    \end{align*}

\item\begin{align*}
    \frac{2310}{6468} &=
    \frac{\cancel{2} \times \cancel{3} \times 5 \times \cancel{7} \times \cancel{11}}
    {\cancel{2} \times 2 \times \cancel{3} \times \cancel{7} \times 7 \times \cancel{11}}\\ &=
    \frac{5}
    {2 \times 7} =
    \frac{5}{14}
    \end{align*}

\item\begin{align*}
    \frac{650650}{80} &=
    \frac{\cancel{2} \times \cancel{5} \times 5 \times 7 \times 11 \times 13 \times 13}
    {\cancel{2} \times 2 \times 2 \times 2 \times \cancel{5}}\\ &=
    \frac{5 \times 7 \times 11 \times 13 \times 13}
    {2 \times 2 \times 2} =
    \frac{65065}{8}
    \end{align*}

\item\begin{align*}
    \frac{2184}{27027} &=
    \frac{2 \times 2 \times 2 \times \cancel{3} \times \cancel{7} \times \cancel{13}}
    {\cancel{3} \times 3 \times 3 \times \cancel{7} \times 11 \times \cancel{13}}\\ &=
    \frac{2 \times 2 \times 2}
    {3 \times 3 \times 11} =
    \frac{8}{99}
    \end{align*}

\item\begin{align*}
    \frac{252}{10164} &=
    \frac{\cancel{2} \times \cancel{2} \times \cancel{3} \times 3 \times \cancel{7}}
    {\cancel{2} \times \cancel{2} \times \cancel{3} \times \cancel{7} \times 11 \times 11}\\ &=
    \frac{3}
    {11 \times 11} =
    \frac{3}{121}
    \end{align*}

\item\begin{align*}
    \frac{14014}{150} &=
    \frac{\cancel{2} \times 7 \times 7 \times 11 \times 13}
    {\cancel{2} \times 3 \times 5 \times 5}\\ &=
    \frac{7 \times 7 \times 11 \times 13}
    {3 \times 5 \times 5} =
    \frac{7007}{75}
    \end{align*}

\item\begin{align*}
    \frac{10}{14} &=
    \frac{\cancel{2} \times 5}
    {\cancel{2} \times 7}\\ &=
    \frac{5}
    {7} =
    \frac{5}{7}
    \end{align*}

\item\begin{align*}
    \frac{4620}{15} &=
    \frac{2 \times 2 \times \cancel{3} \times \cancel{5} \times 7 \times 11}
    {\cancel{3} \times \cancel{5}}\\ &=
    \frac{2 \times 2 \times 7 \times 11}
    {1} =
    \frac{308}{1}
    \end{align*}

\item\begin{align*}
    \frac{1287}{16} &=
    \frac{3 \times 3 \times 11 \times 13}
    {2 \times 2 \times 2 \times 2}\\ &=
    \frac{3 \times 3 \times 11 \times 13}
    {2 \times 2 \times 2 \times 2} =
    \frac{1287}{16}
    \end{align*}

\item\begin{align*}
    \frac{85995}{936} &=
    \frac{\cancel{3} \times \cancel{3} \times 3 \times 5 \times 7 \times 7 \times \cancel{13}}
    {2 \times 2 \times 2 \times \cancel{3} \times \cancel{3} \times \cancel{13}}\\ &=
    \frac{3 \times 5 \times 7 \times 7}
    {2 \times 2 \times 2} =
    \frac{735}{8}
    \end{align*}

\item\begin{align*}
    \frac{14742}{385} &=
    \frac{2 \times 3 \times 3 \times 3 \times 3 \times \cancel{7} \times 13}
    {5 \times \cancel{7} \times 11}\\ &=
    \frac{2 \times 3 \times 3 \times 3 \times 3 \times 13}
    {5 \times 11} =
    \frac{2106}{55}
    \end{align*}

\item\begin{align*}
    \frac{3003}{1274} &=
    \frac{3 \times \cancel{7} \times 11 \times \cancel{13}}
    {2 \times \cancel{7} \times 7 \times \cancel{13}}\\ &=
    \frac{3 \times 11}
    {2 \times 7} =
    \frac{33}{14}
    \end{align*}

\item\begin{align*}
    \frac{37180}{5460} &=
    \frac{\cancel{2} \times \cancel{2} \times \cancel{5} \times 11 \times \cancel{13} \times 13}
    {\cancel{2} \times \cancel{2} \times 3 \times \cancel{5} \times 7 \times \cancel{13}}\\ &=
    \frac{11 \times 13}
    {3 \times 7} =
    \frac{143}{21}
    \end{align*}

\item\begin{align*}
    \frac{975975}{4} &=
    \frac{3 \times 5 \times 5 \times 7 \times 11 \times 13 \times 13}
    {2 \times 2}\\ &=
    \frac{3 \times 5 \times 5 \times 7 \times 11 \times 13 \times 13}
    {2 \times 2} =
    \frac{975975}{4}
    \end{align*}

\item\begin{align*}
    \frac{70}{131820} &=
    \frac{\cancel{2} \times \cancel{5} \times 7}
    {\cancel{2} \times 2 \times 3 \times \cancel{5} \times 13 \times 13 \times 13}\\ &=
    \frac{7}
    {2 \times 3 \times 13 \times 13 \times 13} =
    \frac{7}{13182}
    \end{align*}

\item\begin{align*}
    \frac{9800}{140} &=
    \frac{\cancel{2} \times \cancel{2} \times 2 \times \cancel{5} \times 5 \times \cancel{7} \times 7}
    {\cancel{2} \times \cancel{2} \times \cancel{5} \times \cancel{7}}\\ &=
    \frac{2 \times 5 \times 7}
    {1} =
    \frac{70}{1}
    \end{align*}

\item\begin{align*}
    \frac{10}{6468} &=
    \frac{\cancel{2} \times 5}
    {\cancel{2} \times 2 \times 3 \times 7 \times 7 \times 11}\\ &=
    \frac{5}
    {2 \times 3 \times 7 \times 7 \times 11} =
    \frac{5}{3234}
    \end{align*}

\item\begin{align*}
    \frac{15400}{700} &=
    \frac{\cancel{2} \times \cancel{2} \times 2 \times \cancel{5} \times \cancel{5} \times \cancel{7} \times 11}
    {\cancel{2} \times \cancel{2} \times \cancel{5} \times \cancel{5} \times \cancel{7}}\\ &=
    \frac{2 \times 11}
    {1} =
    \frac{22}{1}
    \end{align*}

\item\begin{align*}
    \frac{210}{30} &=
    \frac{\cancel{2} \times \cancel{3} \times \cancel{5} \times 7}
    {\cancel{2} \times \cancel{3} \times \cancel{5}}\\ &=
    \frac{7}
    {1} =
    \frac{7}{1}
    \end{align*}

\item\begin{align*}
    \frac{33}{780} &=
    \frac{\cancel{3} \times 11}
    {2 \times 2 \times \cancel{3} \times 5 \times 13}\\ &=
    \frac{11}
    {2 \times 2 \times 5 \times 13} =
    \frac{11}{260}
    \end{align*}

\item\begin{align*}
    \frac{32340}{217503} &=
    \frac{2 \times 2 \times \cancel{3} \times 5 \times 7 \times 7 \times \cancel{11}}
    {\cancel{3} \times 3 \times \cancel{11} \times 13 \times 13 \times 13}\\ &=
    \frac{2 \times 2 \times 5 \times 7 \times 7}
    {3 \times 13 \times 13 \times 13} =
    \frac{980}{6591}
    \end{align*}

\item\begin{align*}
    \frac{792}{165} &=
    \frac{2 \times 2 \times 2 \times \cancel{3} \times 3 \times \cancel{11}}
    {\cancel{3} \times 5 \times \cancel{11}}\\ &=
    \frac{2 \times 2 \times 2 \times 3}
    {5} =
    \frac{24}{5}
    \end{align*}

\item\begin{align*}
    \frac{126}{264} &=
    \frac{\cancel{2} \times \cancel{3} \times 3 \times 7}
    {\cancel{2} \times 2 \times 2 \times \cancel{3} \times 11}\\ &=
    \frac{3 \times 7}
    {2 \times 2 \times 11} =
    \frac{21}{44}
    \end{align*}

\item\begin{align*}
    \frac{936}{1320} &=
    \frac{\cancel{2} \times \cancel{2} \times \cancel{2} \times \cancel{3} \times 3 \times 13}
    {\cancel{2} \times \cancel{2} \times \cancel{2} \times \cancel{3} \times 5 \times 11}\\ &=
    \frac{3 \times 13}
    {5 \times 11} =
    \frac{39}{55}
    \end{align*}

}}
% % VARIABLES %%%
\setTitle{Fiche de Prep - Séance 3: Lire et construire un diagrammes en bâton}
\date{27/09/2024}
\setGrade{6e}
\def\imgPath{enseignement/6e/organisation-et-gestion-de-donnees/}
%%

\def\imgPrefix{dim-6e/}
\prepTable{
    \prepRow{
        \slide{SEANCES PRECEDENTES}{}
    }{
        \begin{itemize}
            \item Lire un tableau
            \item Construire un tableau
        \end{itemize}
    }{-}
    \prepRow{
        \slide{qf}{}
        \imgp{qf-3-4p96}
    }{
        \begin{itemize}[wide=0pt, leftmargin=*]
            \item 3p96: 
            \begin{itemize}[wide=0pt, leftmargin=*]
                \item Tâche: Lire un diagramme en bâton.
                \item Modalité de correction : correction d'un élève au tableau qui entourera les données recherché dans le document.
            \end{itemize}
        \end{itemize}
    }{
        $5\min$
    }
    \prepRow{
        \slide{CORRECTION}{}
        \imgp{exo-6p101.png}
    }{
        \begin{itemize}[wide=0pt, leftmargin=*]
            \item 6p101: 
            \begin{itemize}[wide=0pt, leftmargin=*]
                \item Tâche: Constrution de tableau a partir de série statique.
                \item Difficultés attendues : 
                identification des données à organiser,
                classement des données,
                différence entre modalité et effectif,
                utilisation de la ligne ou de la colonne
                \item Modalité de correction : création d'un tableau sur libreOffice Calc.
            \end{itemize}
        \end{itemize}
    }{
        $7\min$
    }
    \prepRow{
        \slide{ACTIVITE}{}
        \imgp{exo-6p101.png}
    }{
        \begin{itemize}[wide=0pt, leftmargin=*]
            \item A partir des données de l'exercice corrigé contruire un diagramme en bâton: 
            \begin{itemize}[wide=0pt, leftmargin=*]
                \item Tâche: Constrution d'un diagramme en bâton a partir d'un tableau.
                \item Difficultés attendues :
                gestion des axes,
                présentation graphique,
                respect de la proportionnalité,
                compréhension de la consigne,
                choix des unités et des échelles
                \item Modalité de correction :
                Sur un fond de graphique,
                un élève dessine un diagramme en bâton.
                Un éleve dessine au tableau le diagramme en bâton.
            \end{itemize}
        \end{itemize}
    }{
        $15\min$
    }
    \prepRow{
        \slide{cr}{}
        \setcounter{section}{1}
        \section{Diagrammes}
        \subsection{Diagrammes en bâton}
    }{
        \begin{itemize}
            \item \vc{Diagrammes en bâton}{à recopier}
            \item \expl{Differentes représentation}{à coller (correction de l'exercice/activité)}
        \end{itemize}
    }{
        $5\min$
    }

    \def\imgPrefix{}
    \prepRow{
        \slide{exo}
        \def\imgPrefix{}
        \imgp{diagramme-en-baton-attendus-6e}
    }{
        \begin{itemize}
            \item Vrai ou Faux?
            \begin{itemize}
                \item Tâche : Porter un regard critique sur une représentation graphique.
                \item Difficultés attendues:
                confusion entre les nombres réels et la représentation visuelle,
                compréhension de l'échelle
            \end{itemize}
        \end{itemize}
    }{10min}

    \prepRow{
        \slide{COURS (si le temps)}{}
    }{
        \begin{itemize}
            \item \pr{bâtons proportionnels}{à copier}
        \end{itemize}
    }{6min}

    \prepRow{
        \slide{EXERCICES A LA MAISON}{}
        \begin{itemize}
            \item 27p105
        \end{itemize}
    }{\imgp{exo-27p105}[5cm]}{3min}
    \prepRow{
        \slide{SEANCES SUIVANTES}{}
    }{
        \begin{itemize}
            \item Diagrammes circulaires
            \item Situations de proportionnalité
            \item Graphiques cartésiens
        \end{itemize}
    }{-}
}
% % VARIABLES %%%
\setTitle{\jules}
\def\imgPath{enseignement/4e/divisibilite-et-nombres-premiers/}
%%

\definecolor{gradeColor}{HTML}{E46C4B}
\def\currentColor{gradeColor}

\newcommand{\adress}[1]{{\small #1}}
\newcommand{\yr}[1]{{\color{Gray} #1}}
\def\SU{\adress{Sorbonne Université, 4 Place Jussieu, 75005 Paris }}

\slide{Occupation}{
    \begin{itemize}
        \item \key{Professeur stagiaire à temps complet} -
        \adress{Collège de la Paix, 76 Rue du Fort, 92130 Issy-les-Moulineaux} -
        \yr{2024-2025}
    \end{itemize}
}

\slide{Formation}{
    \begin{itemize}
        \item \key{Master 2 MEEF mathématiques} -
        \SU -
        \yr{2023-2024}
        \item \key{Master 1 MEEF mathématiques} -
        \SU -
        \yr{2022-2023}
        \item \key{Licence 3 Mono-Math} -
        \SU -
        \yr{2020-2022}
        \item \key{Prépa Math Physique} -
        \adress{lycée Fénelon, 2 rue de l'Eperon, 75006 Paris} -
        \yr{2018-2020}
    \end{itemize}
}

\slide{Experiences professionnelles}{
    \begin{itemize}
        \item \key{Stage SOPA} -
        \adress{Collège Janson de Sailly, 106 Rue de la Pompe, 75016 Paris} -
        \yr{2023-2024 , 324 heures}
        \item \key{Stage SOPA} -
        \adress{Collège Jean-Moulin, 75 rue d'Alésia, 75014 Paris} -
        \yr{2022-2023 , 6 semaines}
        \item \key{Stage SOPA} -
        \adress{Collège Rodin, 19 Rue Corvisart, 75013 Paris} -
        \yr{2023 - 5 demi-journées}
        \item \key{Vente / Animation} -
        Emploi étudiant -
        \adress{Le Paysans Urbain, 14 Rue Stendhal, 75020 Paris} -
        \yr{2021-2022 , 6 mois à mi-temps}
        \item \key{Plaidoyer - écologie et alimentation} -
        Association étudiante -
        \adress{LUPA, Sorbonne Université} -
        \yr{2020-2023}
        \item \key{Projectionniste} -
        Association étudiante -
        \adress{SUper8, Sorbonne Université} -
        \yr{2020-2022}
        \item \key{Professeur particulier} - mathématiques -
        \yr{2021-2024 , 2h/semaines}
    \end{itemize}
}

\slide{Coordonnées}{
    \begin{itemize}
        \item 38 rue Fessart, 92100 Boulogne-Billancourt
        \item jules.pesin@ac-versailles.fr
        \item 0695010324
    \end{itemize}
}

\end{document}