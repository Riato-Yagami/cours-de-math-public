\act{La rumeur}{
    Le vendredi 14 février, dernier jour avant les vacances d'hiver, M. Pesin dit sur le ton de l'humour à un de ses élèves que le collège de la Paix sera fermé le jour de la rentrée, le lundi 3 mars.\\
    Le lendemain, cet élève, qui n'a pas compris qu'il s'agissait d'une blague, le raconte à trois de ses amis.
    Le jour suivant, ces trois amis le racontent chacun à trois de leurs amis, tous différents.
    \begin{enumerate}
        \item Combien d'élèves pensent que le collège sera fermé le soir du 16 février ?\saveenumi
    \end{enumerate}
    En supposant que chaque jour, toutes les nouvelles personnes mises au courant transmettent la rumeur à trois nouvelles personnes chacune.
    \begin{enumerate}\loadenumi
        \item Combien d'élèves auront entendu la rumeur le soir du 17 février et du 18 février ?
        \item Le jour de la rentrée, est-il probable que tous les élèves du collège pensent que le collège est fermé ?
        \item Combien de temps a-t-il fallu pour que tous les élèves soient au courant, en supposant que chaque personne informée est du collège et que le collège de la Paix compte environ 700 élèves ?
        \item Quel calcul permet de connaître le nombre de nouvelles personnes mises au courant le 3 mars ?
        \item Propose une notation pour raccourcir ce calcul.
    \end{enumerate}
}