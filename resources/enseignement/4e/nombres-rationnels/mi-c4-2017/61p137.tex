\newcommand{\doc}[1]{Doc. #1}

\exo{61p137}{
    \nswr[0]{
        \begin{enumerate}
            \item
            \begin{itemize}
                \item Dans cet exercice, on assimile le potager à un pavé droit.
                \begin{itemize}
                    \item La formule du volume $V$ d'un pavé droit de longueur $L$, de largeur $l$ et de hauteur $h$ est
                    $V = L \times l \times h$.
                    \item Ici, $L = \Lg{180}$, $l = \Lg{80}$ et $h = \Lg{15}$ d'après le \doc{1}.
                    \item Donc, $V = \Lg{180} \times \Lg{80} \times \Lg{15} = \Vol{216000}$.
                    \item Le volume du potager est donc de $\Vol{216000}$.
                \end{itemize}
                \item On remplit le bac aux deux-tiers.
                \begin{itemize}
                    \item Donc, $\Vol{216000} \times \dfrac{2}{3} = \Vol{144000}$.
                    \item Sachant que $\Vol[dm]{1} = \Vol{1000}$ et $\Vol[dm]{1} = \Capa{1}$, alors $\Vol{144000} = \Capa{144}$.
                    \item On a donc besoin de $\Capa{144}$ de terreau pour remplir le bac.
                \end{itemize}
                \item Et $144 = 50 \times 3 - 7$.
                \begin{itemize}
                    \item Il faut donc acheter $3$ sacs de terreau.
                \end{itemize}
                \item Un sac de terreau coûte \Prix{6.95} d'après le \doc{3}.
                \begin{itemize}
                    \item Donc, $3 \times \Prix{6.95} = \Prix{20.85}$.
                    \item Il va donc dépenser $\Prix{20.85}$.
                \end{itemize}
            \end{itemize}
            \item \begin{itemize}
                \item Le potager est divisé en $2$ carrés de même aire.
                \item D'après le \doc{2}, chacun de ces deux carrés est lui-même découpé en sous-carrés :
                \begin{itemize}
                    \item Celui de gauche est divisé en $3 \times 3 = 9$ carrés dont $2$ sont réservés aux carottes.
                    \item Celui de droite est divisé en $5 \times 5 = 25$ carrés dont $4$ sont réservés aux carottes.
                \end{itemize}
                \item \begin{align*}
                    \dfrac{2}{9} \times \dfrac{1}{2} + \dfrac{4}{25} \times \dfrac{1}{2}
                    &= (\dfrac{2}{9} + \dfrac{4}{25}) \times \dfrac{1}{2}\\
                    &= (\dfrac{2 \times 25}{9 \times 25} + \dfrac{4 \times 9}{25 \times 9}) \times \dfrac{1}{2}\\
                    &= \dfrac{86}{225} \times \dfrac{1}{2}\\
                    &= \dfrac{86}{450}
                    = \dfrac{43}{225}
                \end{align*}
                \item $\dfrac{43}{225}$ du potager est donc réservé aux carottes.
                \item Donc, $\Capa{144} \times \dfrac{43}{225} = \Capa*{\dfrac{6192}{225}} \approx \Capa{27.52}$.
                \item Il faut donc environ $\Capa{27.52}$ pour les carottes.
            \end{itemize}
        \end{enumerate}
    }
}[\mi]