\exo{QCM sur les fractions}{
    Dans chacune des ci-dessous, une seule réponse est correcte. Laquelle ?
    \begin{enumerate}
        \item L'inverse de $\frac{2}{7}$ est :
        \multiColEnumerate{3}{
            \item supérieur à $7$
            \item égale à $\np{3.5}$
            \item inférieur à $2$
        }
        \item $\frac{1}{15}$ est égale à :
        \multiColEnumerate{3}{
            \item $\np{0.0666666667}$
            \item $\frac{2}{5} \div \frac{1}{6}$
            \item $\frac{2}{30}$
        }
        \item Dans un ruisseau, il s'écoule, en moyenne, $120 \Vol{}$ d'eau en 45 $\minute$.
        Le débit de ce ruisseau, en $\Vol{}/\hour$, est égal à :
        \multiColEnumerate{3}{
            \item $120 \div \frac{3}{4}$
            \item $120 \times \frac{3}{4}$
            \item $120 \times \np{0.75}$
        }
        La fraction $\frac{143}{132}$ est :
        \multiColEnumerate{3}{
            \item irréductible
            \item comprise entre $1$ et $\np{1.1}$
            \item décimale
        }
    \end{enumerate}
}
% [\href{https://cache.media.education.gouv.fr/file/Fractions/23/2/RA16_C4_MATH_fractions_flash3_sens_quotient_554232.pdf}
% {Utiliser les nombres pour comparer, calculer et résoudre des problèmes : les fractions}]