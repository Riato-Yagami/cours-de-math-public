\exo{Réussite au brevet}{
    Au collège de la Paix, $162$ élèves ont présenté le brevet en $2024$ et sept élèves sur neuf l'ont obtenu.
    En $2025$, M. Pesin prédit que, sur les $150$ élèves inscrits, quatre cinquièmes d'entre eux réussiront.
    \begin{enumerate}
        \item M. Pesin prédit-il un meilleur taux de réussite que l'année précédente ?
        \nswr[\answerHeight]{
            \begin{itemize}
                \item D'une part $\dfrac{7}{9} = \dfrac{7 \times 5}{9 \times 5} = \dfrac{35}{45}$.
                \item D'autre part $\dfrac{4}{5} = \dfrac{4 \times 9}{5 \times 9} = \dfrac{36}{45}$.
                \item Donc, $\dfrac{36}{45} > \dfrac{35}{45}$, M. Pesin prédit un meilleur taux de réussite.
            \end{itemize}
        }
        \item Si ses prédictions sont correctes, quelle année, entre $2024$ et $2025$, comptera le plus d'élèves ayant réussi le brevet ?
        \nswr[\answerHeight]{
            \begin{itemize}
                \item D'une part $162 \times \dfrac{7}{9} = 126$.
                \item D'autre part $150 \times \dfrac{4}{5} = 120$.
                \item Alors $162 \times \dfrac{7}{9} > 150 \times \frac{4}{5}$
                \item Ainsi en $2024$, il y aura plus d'élèves ayant réussi le brevet.
            \end{itemize}
        }
    \end{enumerate}
}
%[\href{https://www.letudiant.fr/college/annuaire-des-colleges/fiche/college-la-paix-92.html#success-rates}{L'Etudiant}]