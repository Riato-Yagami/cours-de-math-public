\act{Nature des nombres}{
    En maternelle, nous avons appris à compter des objets en utilisant les nombres $1, 2, 3$, etc.
    Ces nombres sont les premiers utilisés « naturellement »,
    et on les appelle les nombres entiers naturels.
    Depuis l'école primaire et au collège, nous avons découvert d'autres nombres.
    Voici une liste de nombres :
    \multiColItemize{3}{
        \item $-\np{27.2}$
        \item $-\sqrt{4}$
        \item $\dfrac{10371}{100}$
        \item $\dfrac{27}{13}$
        \item $\dfrac{3}{2}$
        \item $\dfrac{-21}{15}$
        \item $\np{0.33333}\ldots$
        \item $\pi$
        \item $\dfrac{-10}{5}$
        \item $\sqrt{2}$
        \item $\dfrac{47}{21}$
        \item $-15 + 20$
        \item $\dfrac{-10}{3}$
        \item $37$
        \item $1 \div 7$
    }
    \begin{enumerate}
        \item Indique par une pastille :
        \begin{itemize}
            \item \textcolor{Blue}{bleue} les nombres entiers naturels.
            \item \textcolor{Red}{rouge} les nombres entiers relatifs.
            \item \textcolor{Green}{verte} les nombres décimaux.
        \end{itemize} 
        \hint{Certains nombres peuvent être indiqués plusieurs fois.}
        \item Quels sont les nombres restants ?
        Essaie de les classer dans deux catégories de nombres différentes.
    \end{enumerate}
}[\href{https://clg-monnet-briis.ac-versailles.fr/IMG/pdf/cours_fractions-3.pdf}{Collège Monnet}]