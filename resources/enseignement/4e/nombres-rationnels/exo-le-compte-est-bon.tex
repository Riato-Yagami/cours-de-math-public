\exo{Le compte est bon}{
    Voici une liste de nombres : 
    $1$, $2$, $3$, $4$,
    $\dfrac{1}{3}$, $\dfrac{5}{4}$, $\dfrac{1}{4}$, $\dfrac{3}{6}$, $\dfrac{4}{7}$, $\dfrac{5}{8}$, $\dfrac{6}{9}$.
    \begin{enumerate}
        \item Pour obtenir le nombre $\dfrac{9}{8}$,
    vous pouvez effectuer toutes les opérations que vous souhaitez et décrire les résultats intermédiaires.
    Attention,
    chaque nombre ci-dessus est unique et ne peut être utilisé qu'une seule fois,
    en les convertissant si nécessaire en fractions égales.
    \\\bonus
        \item Trouvez le résultat en utilisant le moins d'opérations possible.
        \item Faites de même avec $\dfrac{6}{7}$, $\dfrac{4}{9}$, $\dfrac{9}{24}$, $\dfrac{9}{28}$.
    \end{enumerate}
}[\prbltq{le-compte-est-bon-avec-des-fractions}]

\nswr[0]{
    Pour obtenir $\dfrac{9}{8}$, on peut procéder comme suit :
    \begin{itemize}
        \item $\dfrac{5}{4} + \dfrac{1}{4} = \dfrac{6}{4} = \dfrac{3}{2}$
        \item $\dfrac{3}{2} \div \dfrac{1}{3} = \dfrac{3}{2} \times 3 = \dfrac{9}{2}$
        \item $\dfrac{9}{2} \times \dfrac{1}{4} = \dfrac{9}{8}$
    \end{itemize}
    On peut aussi faire :
    \begin{itemize}
        \item $\dfrac{5}{8} + \dfrac{3}{6} = \dfrac{5}{8} + \dfrac{1}{2} = \dfrac{5}{8} + \dfrac{4}{8} = \dfrac{9}{8}$
    \end{itemize}
}
