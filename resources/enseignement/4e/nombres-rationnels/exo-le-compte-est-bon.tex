\exo{Le compte est bon}{
    Voici une liste de nombres : 
    $1$, $2$, $3$, $4$,
    $\frac{1}{3}$, $\frac{5}{4}$, $\frac{1}{4}$, $\frac{3}{6}$, $\frac{4}{7}$, $\frac{5}{8}$, $\frac{6}{9}$.
    Pour obtenir le nombre $\frac{9}{8}$,
    vous pouvez effectuer toutes les opérations que vous souhaitez et décrire, si nécessaire,
    les résultats intermédiaires.
    Attention, chaque nombre ci-dessus est unique et ne peut être utilisé qu'une seule fois,
    en les convertissant si nécessaire en fractions équivalentes.
    \bonus
    \begin{enumerate}
        \item Trouvez le résultat en utilisant le moins d'opérations possible.
        \item Faites de même avec $\frac{6}{7}$, $\frac{4}{9}$, $\frac{9}{24}$, $\frac{9}{28}$.
    \end{enumerate}
}[\prbltq{le-compte-est-bon-avec-des-fractions}]

\nswr[0]{
    Pour obtenir $\frac{9}{8}$, on peut procéder comme suit :
    \begin{itemize}
        \item $\frac{5}{4} + \frac{1}{4} = \frac{6}{4} = \frac{3}{2}$
        \item $\frac{3}{2} \div \frac{1}{3} = \frac{3}{2} \times 3 = \frac{9}{2}$
        \item $\frac{9}{2} \times \frac{1}{4} = \frac{9}{8}$
    \end{itemize}
    On peut aussi faire :
    \begin{itemize}
        \item $\frac{5}{8} + \frac{3}{6} = \frac{5}{8} + \frac{1}{2} = \frac{5}{8} + \frac{4}{8} = \frac{9}{8}$
    \end{itemize}
}[\prbltq{le-compte-est-bon-avec-des-fractions}]
