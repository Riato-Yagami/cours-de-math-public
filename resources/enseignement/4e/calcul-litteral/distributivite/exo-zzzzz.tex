\exo{Zzzzz...}{
    Dans le motif suivant, illustré par des exemples de carrés de tailles 3, 4 et 5, nous étudions le nombre de carrés blancs et colorés présents en fonction de la taille du carré.
    \input{resources/enseignement/4e/calcul-litteral/distributivite/tikz/zzzzz.tex}
    \begin{enumerate}\setItemColor{exo}
        \item \textbf{Carrés colorés :}  
            \begin{enumerate}
                \item Combien y aura-t-il de carrés colorés dans un carré de taille 6 ?  
                \item Donner une expression littérale permettant de calculer le nombre de carrés colorés dans un carré de taille $n$.  
                \item À l'aide de votre calculatrice, déterminer le nombre de carrés coloré dans un carré de taille $30$ et dans un carré de taille $176$.  
            \end{enumerate}
    
        \item \textbf{Nombre total de carrés :}  
        \begin{enumerate}
            \item Combien y aura-t-il de carrés au total dans un carré de taille $6$ ? Et dans un carré de taille $7$ ?
            \item Combien y aura-t-il de carrés au total (blancs et colorés) dans un carré de taille $n$ ?  
        \end{enumerate}

        \item \textbf{Carrés blancs :}  
        \begin{enumerate}
            \item Donner une relation d'égalité entre le nombre total de carrés, le nombre de carrés blancs et le nombre de carrés colorés.  
            \item En déduire une expression littérale permettant de calculer le nombre de carrés blancs dans un carré de taille $n$. 
            \item À l'aide de votre calculatrice, déterminer le nombre de carrés colorés dans un carré de taille $610$.
        \end{enumerate}

        \item \textbf{Parité des carrés :}  
        \begin{enumerate}
            \item Peut-on obtenir un nombre impair de carrés colorés ?  
            \item Peut-on obtenir un nombre impair de carrés blancs ?  
        \end{enumerate}
    \end{enumerate}
}