\exo{Rangez des pommes !}{\calculator\\
    Un marchand possède des caisses contenant chacune $99$ pommes.
    Pour optimiser l'espace dans son entrepôt, il décide d'empiler les caisses en tours.
    Cependant, les caisses étant fragiles,
    à chaque nouvelle caisse ajoutée dans une tour,
    il retire une pomme de chacune des caisses de cette tour.\\
    Voici la composition des tours de $1$, $2$ et $3$ caisses :
    % \ctikz[\ifBA{1}{2}]{
%     \boundingBox[5][3][0.5pt][0.5][(0,0)]
%     \fill[thick,color=gradeColor,fill=gradeColor,fill opacity=0.10] (0.00,0.00) -- (1.00,0.00) -- (1.00,1.00) -- (0.00,1.00) -- cycle;
%     \fill[thick,color=gradeColor,fill=gradeColor,fill opacity=0.10] (2.00,1.00) -- (3.00,1.00) -- (3.00,2.00) -- (2.00,2.00) -- cycle;
%     \fill[thick,color=gradeColor,fill=gradeColor,fill opacity=0.10] (2.00,0.00) -- (3.00,0.00) -- (3.00,1.00) -- (2.00,1.00) -- cycle;
%     \fill[thick,color=gradeColor,fill=gradeColor,fill opacity=0.10] (4.00,0.00) -- (5.00,0.00) -- (5.00,1.00) -- (4.00,1.00) -- cycle;
%     \fill[thick,color=gradeColor,fill=gradeColor,fill opacity=0.10] (4.00,1.00) -- (5.00,1.00) -- (5.00,2.00) -- (4.00,2.00) -- cycle;
%     \fill[thick,color=gradeColor,fill=gradeColor,fill opacity=0.10] (4.00,2.00) -- (5.00,2.00) -- (5.00,3.00) -- (4.00,3.00) -- cycle;
%     \draw[color=gradeColor] (0.5,0.57) node[text width=2cm, align=center] {99\\pommes};
%     \draw[color=gradeColor] (2.5,1.57) node[text width=2cm, align=center] {98\\pommes};
%     \draw[color=gradeColor] (2.5,0.57) node[text width=2cm, align=center] {98\\pommes};
%     \draw[color=gradeColor] (4.5,0.57) node[text width=2cm, align=center] {97\\pommes};
%     \draw[color=gradeColor] (4.5,1.57) node[text width=2cm, align=center] {97\\pommes};
%     \draw[color=gradeColor] (4.5,2.58) node[text width=2cm, align=center] {97\\pommes};
%     \draw [thick,gradeColor] (0.00,0.00) -- (1.00,0.00) -- (1.00,1.00) -- (0.00,1.00) -- (0.00,0.00);
%     \draw [thick,gradeColor] (2.00,1.00) -- (3.00,1.00) -- (3.00,2.00) -- (2.00,2.00) -- (2.00,1.00) -- (2.00,0.00) -- (3.00,0.00) -- (3.00,1.00);
%     \draw [thick,gradeColor] (4.00,0.00) -- (5.00,0.00) -- (5.00,1.00) -- (4.00,1.00) -- (4.00,0.00);
%     \draw [thick,gradeColor] (5.00,1.00) -- (5.00,2.00) -- (4.00,2.00) -- (4.00,1.00);
%     \draw [thick,gradeColor] (5.00,2.00) -- (5.00,3.00) -- (4.00,3.00) -- (4.00,2.00);
% }

\ctikz[\ifBA{0.75}{1}]{
    \boundingBox[4][3][0.5pt][0.5][(0,0)]
    \fill[thick,color=gradeColor,fill=gradeColor,fill opacity=0.10] (0.00,0.00) -- (1.00,0.00) -- (1.00,1.00) -- (0.00,1.00) -- cycle;
    \fill[thick,color=gradeColor,fill=gradeColor,fill opacity=0.10] (1.50,1.00) -- (2.50,1.00) -- (2.50,2.00) -- (1.50,2.00) -- cycle;
    \fill[thick,color=gradeColor,fill=gradeColor,fill opacity=0.10] (1.50,0.00) -- (2.50,0.00) -- (2.50,1.00) -- (1.50,1.00) -- cycle;
    \fill[thick,color=gradeColor,fill=gradeColor,fill opacity=0.10] (3.00,0.00) -- (4.00,0.00) -- (4.00,1.00) -- (3.00,1.00) -- cycle;
    \fill[thick,color=gradeColor,fill=gradeColor,fill opacity=0.10] (3.00,1.00) -- (4.00,1.00) -- (4.00,2.00) -- (3.00,2.00) -- cycle;
    \fill[thick,color=gradeColor,fill=gradeColor,fill opacity=0.10] (3.00,2.00) -- (4.00,2.00) -- (4.00,3.00) -- (3.00,3.00) -- cycle;
    \draw[color=gradeColor] (0.48,0.53) node {99};
    \draw[color=gradeColor] (1.99,1.53) node {98};
    \draw[color=gradeColor] (1.99,0.53) node {98};
    \draw[color=gradeColor] (3.49,0.53) node {97};
    \draw[color=gradeColor] (3.49,1.53) node {97};
    \draw[color=gradeColor] (3.49,2.53) node {97};
    \draw [thick,gradeColor] (0.00,0.00) -- (1.00,0.00) -- (1.00,1.00) -- (0.00,1.00) -- (0.00,0.00);
    \draw [thick,gradeColor] (1.50,1.00) -- (2.50,1.00) -- (2.50,2.00) -- (1.50,2.00) -- (1.50,1.00) -- (1.50,0.00) -- (2.50,0.00) -- (2.50,1.00);
    \draw [thick,gradeColor] (3.00,0.00) -- (4.00,0.00) -- (4.00,1.00) -- (3.00,1.00) -- (3.00,0.00);
    \draw [thick,gradeColor] (4.00,1.00) -- (4.00,2.00) -- (3.00,2.00) -- (3.00,1.00);
    \draw [thick,gradeColor] (4.00,2.00) -- (4.00,3.00) -- (3.00,3.00) -- (3.00,2.00);
}

    \begin{enumerate} \loadenumi[exo][0]
        \item Combien de pommes y aurait-il au total dans une tour de $4$ caisses ? Et dans une tour de $5$, $10$ ou $23$ caisses ?
        \item Expliquez, en une phrase ou à l'aide d'un programme de calcul, comment déterminer le nombre de pommes pour un nombre quelconque de caisses.
        \item Écrivez une expression littérale permettant de calculer le nombre de pommes dans une tour composée de $n$ caisses.
        \item Utilisez cette expression pour calculer le nombre de pommes dans une tour de $35$, $46$, $58$ et $70$ caisses.
        \item Que remarquez-vous à propos des résultats obtenus ?
        \item Quel est le nombre maximal de pommes pouvant être rangées dans une tour ?
    \end{enumerate}
}

% \nswr[0]{
%     \begin{enumerate}
%         \item On a : $(99 - 4) \times 4 = 380$
%         \item 
%         \item L'expression littérale permettant de calculer le nombre de pommes dans une tour composée de $n$ caisses est $(99 - n) \times n$.
%         \item Pour une tour de $35$ caisses : $(99 - 35) \times 35 = 2240$ \\
%         Pour une tour de $46$ caisses : $(99 - 46) \times 46 = 2438$ \\
%         Pour une tour de $58$ caisses : $(99 - 58) \times 58 = 2378$ \\
%         Pour une tour de $70$ caisses : $(99 - 70) \times 70 = 2030$
%         \item On remarque que le nombre de pommes augmente jusqu'à un certain point puis commence à diminuer.
%     \end{enumerate}
% }