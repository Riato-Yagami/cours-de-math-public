\exo{\bonus Patron du cône}{
    On souhaite construire un patron d'un cône dont la génératrice mesure \Lg{5} et le rayon de la base \Lg{1.5}.
    
    \dividePage{
        \newcommand*{\ArcAngle}{108}%
\newcommand*{\ArcRadius}{5.0}%
\ctikz[0.75]{
    \draw [thick] (-0.11,-0.98) circle (1.50cm);
    \draw [rotate around={-178.73:(-8.72,-3.26)},thick] (-8.72,-3.26) ellipse (1.50cm and 0.75cm);
    % \draw [rotate around={-178.73:(-8.72,-3.26)},thick,dashed] (-8.72,-3.26) ellipse (1.50cm and 0.75cm);
    \draw [thick,gradeColor] ($(-5.99,-3.74)+(\ArcRadius,0)$) 
    arc (0:\ArcAngle:\ArcRadius);
    \drawPoint{O}{-5.99}{-3.74}
    \drawPoint{A}{-0.99}{-3.75}
    \drawPoint{B}{-7.53}{1.02}
    \draw[color=gradeColor] (-9.72,-0.89) node {$\Lg{5}$};
    \draw[color=gradeColor] (-9.44,-3.03) node {$\Lg{1.50}$};
    \draw [thick,gradeColor] (-8.72,1.51) -- (-10.22,-3.26);
    \draw [thick] (-8.72,1.51) -- (-7.22,-3.26);
    \draw [thick,dashed,gradeColor] (-8.72,-3.26) -- (-10.22,-3.26);
    \draw [thick,dashed] (-5.99,-3.74) -- (-0.11,-0.98);
    \draw [thick] (-7.53,1.02) -- (-5.99,-3.74) -- (-0.99,-3.75);
}
    }{
        \begin{enumerate}
            \item Calculer le périmètre de la base du cône.
            \item Quelle est la longueur de l'arc de cercle $\overset{\frown}{AB}$ ?
            \item Calculer le périmètre du cercle ayant pour rayon la génératrice.
            \item Déterminer la mesure de l'angle $\widehat{AOB}$.
            \item Construire un patron du cône.
        \end{enumerate}
    }
}[\href{http://chica.chevere.free.fr/quatrieme/14/a14.pdf}{B. TRUCHETET}][\cmdGeoGebra[pstm9puc]]

\nswr[0]{
    \begin{enumerate}
        \item Le périmètre $P$ de la base du cône est donné par la formule du périmètre d'un cercle : $P = 2 \pi r$.
        Ici, $r = \Lg{1.5}$,
        donc $P = 2 \pi \times 1,5\Lg{} = 3 \pi\Lg{}$.
        \item La longueur de l'arc de cercle $\overset{\frown}{AB}$ correspondant à la surface conique est égale au périmètre de la base du cône, soit $3 \pi\Lg{}$.
        \item Le périmètre $P'$ du cercle ayant pour rayon la génératrice est donné par la formule du périmètre d'un cercle :
        $P' = 2 \pi R$. Ici, $R = \Lg{5}$, donc $P' = 2 \pi \times 5 = 10 \pi$.
        \item La mesure de l'angle $\widehat{AOB}$ est déterminée par la proportion entre le périmètre de la base du cône et le périmètre du cercle ayant pour rayon la génératrice.
        Donc, $\widehat{AOB} 
        = \frac{P}{P'} \times 360^\circ
        = \frac{3 \pi}{10 \pi} \times 360^\circ
        = \frac{3}{10} \times 360^\circ
        = 108^\circ$.
        \item Pour construire un patron du cône, on dessine un secteur circulaire de rayon \Lg{5} et d'angle $108^\circ$, puis on découpe et on assemble ce secteur avec un cercle de rayon 1,5 cm pour former le cône.
    \end{enumerate}
}