\act{Patron d'un cylindre}{
    Voici deux représentations en perspective cavalière d'un même cylindre :
    \ctikz[\ifBA{1.5}{3}]{
    \boundingBox[2.6][1.35][0.5pt][0.5][(0,0)]
    \draw [rotate around={-0.12:(0.50,0.25)},thick,dashed] (0.50,0.25) ellipse (0.50cm and 0.29cm);
    \draw [rotate around={0:(0.50,1.25)},thick] (0.50,1.25) ellipse (0.50cm and 0.29cm);
    \draw [thick] (1.75,0.50) circle (0.50cm);
    \draw [shift={(2.25,1.00)},thick]  plot[domain=-0.79:2.36,variable=\t]({1*0.50*cos(\t r)+0*0.50*sin(\t r)},{0*0.50*cos(\t r)+1*0.50*sin(\t r)});
    \draw [shift={(2.25,1.00)},thick,dashed]  plot[domain=2.36:5.50,variable=\t]({1*0.50*cos(\t r)+0*0.50*sin(\t r)},{0*0.50*cos(\t r)+1*0.50*sin(\t r)});
    \draw [rotate around={0:(0.50,0.25)},thick]  plot[domain=3.14:6.28,variable=\t]({0.50 + 0.50*cos(\t r)}, {0.25 + 0.29*sin(\t r)});
    \draw[color=gradeColor] (1.09,0.79) node {6cm};
    \draw[color=gradeColor] (2.43,0.37) node {6cm};
    \draw[color=gradeColor] (1.61,0.29) node {3cm};
    \draw[color=gradeColor] (0.74,1.35) node {3cm};
    \draw [thick] (0.00,1.25) -- (0.00,0.25);
    \draw [thick] (1.00,1.25) -- (1.00,0.25);
    \draw [thick] (1.40,0.85) -- (1.90,1.35);
    \draw [thick] (2.10,0.15) -- (2.60,0.65);
    \draw [thick] (1.75,0.50) -- (1.75,0.00);
    \draw [thick] (0.50,1.25) -- (1.00,1.25);
}
    \bvspace{-0.5cm}\begin{enumerate}
        \item Combien de figures géométriques composent ce cylindre ?
        \item Quelle est la nature de chacune de ces figures ?
        \item Dessine à main levée le patron de ce cylindre en ajoutant les mesures.
        \item Construis le patron en respectant les proportions.
        \item Découpe le patron pour vérifier sa cohérence.
    \end{enumerate}
}

% \slide{exo}{\bshrink
%     \act{Patron d'un cylindre}{
%         Voici deux représentations en perspective cavalière d'un même cylindre :
%         \ctikz[\ifBA{1.5}{3}]{
    \boundingBox[2.6][1.35][0.5pt][0.5][(0,0)]
    \draw [rotate around={-0.12:(0.50,0.25)},thick,dashed] (0.50,0.25) ellipse (0.50cm and 0.29cm);
    \draw [rotate around={0:(0.50,1.25)},thick] (0.50,1.25) ellipse (0.50cm and 0.29cm);
    \draw [thick] (1.75,0.50) circle (0.50cm);
    \draw [shift={(2.25,1.00)},thick]  plot[domain=-0.79:2.36,variable=\t]({1*0.50*cos(\t r)+0*0.50*sin(\t r)},{0*0.50*cos(\t r)+1*0.50*sin(\t r)});
    \draw [shift={(2.25,1.00)},thick,dashed]  plot[domain=2.36:5.50,variable=\t]({1*0.50*cos(\t r)+0*0.50*sin(\t r)},{0*0.50*cos(\t r)+1*0.50*sin(\t r)});
    \draw [rotate around={0:(0.50,0.25)},thick]  plot[domain=3.14:6.28,variable=\t]({0.50 + 0.50*cos(\t r)}, {0.25 + 0.29*sin(\t r)});
    \draw[color=gradeColor] (1.09,0.79) node {6cm};
    \draw[color=gradeColor] (2.43,0.37) node {6cm};
    \draw[color=gradeColor] (1.61,0.29) node {3cm};
    \draw[color=gradeColor] (0.74,1.35) node {3cm};
    \draw [thick] (0.00,1.25) -- (0.00,0.25);
    \draw [thick] (1.00,1.25) -- (1.00,0.25);
    \draw [thick] (1.40,0.85) -- (1.90,1.35);
    \draw [thick] (2.10,0.15) -- (2.60,0.65);
    \draw [thick] (1.75,0.50) -- (1.75,0.00);
    \draw [thick] (0.50,1.25) -- (1.00,1.25);
}
%         \bvspace{-0.5cm}\begin{enumerate}
%             \item Combien de figures géométriques composent ce cylindre ?
%             \item Quelle est la nature de chacune de ces figures ? \saveenumi
%         \end{enumerate}
%     }
% }

% \slide{exo}{
%     \begin{enumerate}\loadenumi[act]
%         \item Dessine à main levée le patron de ce cylindre en ajoutant les mesures.
%         \item Construis le patron en respectant les proportions.
%         \item Découpe le patron pour vérifier sa cohérence.
%     \end{enumerate}
% }