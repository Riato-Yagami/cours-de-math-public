\def\figScale{2.25}
% \newcommand{\solide}[2]{\item \dividePage{#1}{\input{resources/enseignement/5e/geometrie-dans-l-espace/solides/tikz/#2.tex}}}%

\newcommand{\solide}[2]{\item #1
    \input{resources/enseignement/5e/geometrie-dans-l-espace/solides/tikz/#2.tex}
}

\df{Solides de révolution}{%
    On appelle :
    \multiColItemize{1}{
        \item \key{solide de révolution}, un solide obtenu en faisant tourner une figure plane autour d'un axe.
        \solide{\nswr{\key{cylindre de révolution}}[5cm], un solide de révolution dont les bases sont des cercles superposables}{cylindre}        
        \solide{\nswr{\key{cône}}[5cm], un solide de révolution obtenu par rotation d'un triangle rectangle autour d'un côté adjacent à l'angle droit.}{cone}
        \solide{\nswr{\key{sphère}}[5cm], une surface constituée de tous les points situés à une même distance d'un point appelé \key{centre}.}{sphere}
    }
}[\wiki{Solide_de_révolution}][\cmdGeoGebra[agzsnw5r]]