\newcommand{\sld}[2]{
    \item \dividePage{#1}{\input{resources/enseignement/5e/geometrie-dans-l-espace/solides/tikz/#2.tex}}
}

\def\figScale{1}

\df{Polyèdres}{On appelle :%
    \multiColEnumerate{1}{%
        \sld{%
            \key{polyhèdre}, un solide constitué de \key{faces} polygonales.
        }{polyedre}
        \sld{%
            \nswr{\key{prisme droit}}[5cm], un polyhèdre constitué de deux \key{bases} polygonales superposables,
            reliées entre elles par des faces \nswr{\key{rectangulaires}}.
        }{prisme-droit}
        \sld{%
            \nswr{\key{pavé droit}}[5cm], un \nswr{prisme droit}[5cm] dont les bases sont \nswr{\key{rectangulaires}}.
        }{pave-droit}
        \sld{%
            \nswr{\key{cube}}[5cm], un \nswr{pavé droit}[5cm] dont toutes les faces sont des \nswr{\key{carrés}}[5cm].
        }{cube}
        \sld{%
            \nswr{\key{pyramide}}[5cm], un polyhèdre formé d'une base polygonale reliée à un \key{sommet} par des faces \nswr{\key{triangulaires}}[5cm].
        }{pyramide}
    }
}[\wiki{Polyèdre}[Polyèdres_simples]]

% \df{Polyèdres}{On appelle :
% \multiColEnumerate{1}{ 
%     \item \dividePage{
%         \key{polyhèdre} un solide constitué de \key{faces} polygonales.
%     }{\ctikz{
    \boundingBox[7.789999999999999][6][0.5pt][1][(0,0)][iso]
    \draw [thick] (1.73,1.50) -- (3.46,0.50) -- (5.19,0.50) -- (6.93,1.50) -- (5.19,3.50) -- (2.60,4.00) -- (1.73,1.50);
    \draw [thick,dashed] (6.93,1.50) -- (5.19,2.50);
    \draw [thick,dashed] (1.73,1.50) -- (5.19,2.50);
    \draw [thick,dashed] (5.19,2.50) -- (5.19,3.50);
    \draw [thick] (2.60,4.00) -- (3.46,3.50) -- (5.19,3.50);
    \draw [thick] (3.46,3.50) -- (5.19,0.50);
    \draw [thick] (3.46,3.50) -- (3.46,0.50);
}}
%     \item \dividePage{
%         \nswr{\key{prisme droit}}[5cm], un polyhèdre constitué de deux \key{bases} polygonales superposables,
%         reliées entre elles par des faces \nswr{\key{rectangulaires}}.
%     }{\ctikz[\figScale]{
    \boundingBox[2.2][2.8][0.5pt][0.5][(0,0)][cavalier]
    \draw [thick] (0.00,2.00) -- (0.00,0.40) -- (1.40,0.00) -- (1.60,0.60) -- (2.20,1.20);
    \draw [thick,dashed] (2.20,1.20) -- (1.00,1.20) -- (0.00,0.40);
    \draw [thick,dashed] (1.00,1.20) -- (1.00,2.80);
    \draw [thick] (0.00,2.00) -- (1.00,2.80) -- (2.20,2.80) -- (1.60,2.20) -- (1.40,1.60) -- (0.00,2.00);
    \draw [thick] (1.40,1.60) -- (1.40,0.00);
    \draw [thick] (2.20,2.80) -- (2.20,1.20);
    \draw [thick] (1.60,0.60) -- (1.60,2.20);
}}\saveenumi
%     \ifArticle{
%         \item \dividePage{
%             \nswr{\key{pavé droit}}[5cm] un \nswr{prisme droit}[5cm] dont les bases sont \nswr{\key{rectangulaires}}.
%         }{\ctikz[\figScale]{
    \boundingBox[4.5][2.5][0.5pt][0.5][(0,0)][cavalier]
    \draw [thick] (0.00,2.00) -- (0.50,2.50) -- (4.50,2.50) -- (4.00,2.00) -- (0.00,2.00) -- (0.00,0.00) -- (4.00,0.00) -- (4.00,2.00);
    \draw [thick,dashed] (0.00,0.00) -- (0.50,0.50) -- (4.50,0.50);
    \draw [thick,dashed] (0.50,0.50) -- (0.50,2.50);
    \draw [thick] (4.00,0.00) -- (4.50,0.50) -- (4.50,2.50);
}}
%         \item \dividePage{
%             \nswr{\key{cube}}[5cm] un \nswr{pavé droit}[5cm] dont les toutes les faces sont des \nswr{\key{carrés}}[5cm].
%         }{\ctikz[\figScale]{
    \boundingBox[3][3][0.5pt][0.5][(0,0)][cavalier]
    \draw [thick] (0.00,2.00) -- (1.00,3.00) -- (3.00,3.00) -- (2.00,2.00) -- (0.00,2.00) -- (0.00,0.00) -- (2.00,0.00) -- (3.00,1.00) -- (3.00,3.00);
    \draw [thick,dashed] (0.00,0.00) -- (1.00,1.00) -- (3.00,1.00);
    \draw [thick,dashed] (1.00,1.00) -- (1.00,3.00);
    \draw [thick] (2.00,2.00) -- (2.00,0.00);
}}
%         \item \dividePage{
%             \nswr{\key{pyramide}}[5cm] un polyhèdre formé d'une base polygoneale relié à un \key{sommet} par des faces \nswr{\key{triangulaires}}[5cm].
%         }{\ctikz[\figScale]{
    \boundingBox[2.5][1.5][0.5pt][0.5][(0,0)][cavalier]
    \draw [thick,dashed] (0.50,0.50) -- (0.50,1.50) -- (1.30,0.80) -- (2.50,0.50);
    \draw [thick] (0.50,1.50) -- (2.00,0.00) -- (2.50,0.50) -- (0.50,1.50) -- (0.00,0.00) -- (2.00,0.00);
    \draw [thick,dashed] (0.00,0.00) -- (0.50,0.50) -- (1.30,0.80);
}}
%     }
% }
% }[\wiki{Polyèdre}[Polyèdres_simples]]