\def\qrv{Que remarquez-vous ?}
\def\mt{\textcolor{Red}{médiatrices}}
\def\mn{\textcolor{Green}{médianes}}
\def\h{\textcolor{Blue}{hauteurs}}

\hint{
    \begin{itemize}
        \item Faites des tracés bien précis pour obtenir les résultats attendus aux questions «\qrv».
        \item Une figure de grande taille vous permettra de rester plus précis.
        \item Vous pouvez effacer vos traits de construction.
        \item Lorsqu'il est demandé de construire un objet géométrique d'une certaine couleur, commencez par le faire au crayon à papier.
        Assurez-vous qu'il est bien tracé, puis repassez-le au stylo, feutre fin ou crayon de couleur.
    \end{itemize}
}

\exo{Droite d'Euler}{
    \begin{enumerate}
        \item Sur une feuille blanche, tracez un triangle quelconque. \nswr[0]{\cmdGeoGebra[fjehbrn3]}
        \item \begin{enumerate}
            \item Tracez en \textcolor{Red}{rouge} les trois \mt{} des côtés du triangle.
            \item \qrv
            \nswr[0]{\\On remarque que les trois \mt{} se coupent en un même point.}
            \item Tracez en \textcolor{Red}{rouge} un cercle centré à l'intersection des \mt{} passant par un sommet du triangle.
            \item \qrv
            \nswr[0]{\\On remarque que le cercle passe par les trois sommets du triangle (appelé \key{cercle circonscrit}).}
        \end{enumerate}
        \item \begin{enumerate}
            \item Tracez en \textcolor{Blue}{bleu} les trois \h{} des côtés du triangle.
            \item \qrv
            \nswr[0]{\\On remarque que les trois \h{} se coupent en un même point (appelé \key{orthocentre}).}
        \end{enumerate}
        \df{}{On appelle \key{médiane} d'un triangle, une droite qui relie un sommet du triangle au milieu du côté opposé.
}[\wiki{Médiane\_(géométrie)}]
        \def\currentColor{exo}
        \item \begin{enumerate}
            \item Tracez en \textcolor{Green}{vert} les trois \mn{} des côtés du triangle.
            \item \qrv
            \nswr[0]{\\On remarque que les trois \mn{} se coupent en un même point (appelé \key{centre de gravité}).}
        \end{enumerate}
        \item \begin{enumerate}
            \item Tracez dans une nouvelle couleur de votre choix une droite passant par l'intersection des \mt{} et des \h.
            \item \qrv
            \nswr[0]{\\On remarque que cette droite passe également par l'intersection des \mn{} (appelée \key{droite d'Euler}).}
        \end{enumerate}
    \end{enumerate}
}