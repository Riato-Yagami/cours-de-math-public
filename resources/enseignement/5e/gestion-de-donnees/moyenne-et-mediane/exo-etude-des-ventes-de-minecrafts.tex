\exo{\ttps Étude des ventes de Minecraft}{
    Le graphique ci-dessous présente les ventes de Minecraft de 2011 à 2023 :
    \begin{center}
    \Stat[% 
        Qualitatif,
        Graphique,
        Donnee=Année,
        Effectif=Ventes (en millions),
        Unitex=1,AngleRotationAbscisse=60,
        Unitey=0.15,Pasy=4,
        Grille,PasGrilley=2,LectureFine,
        CouleurDefaut = ForestGreen,
        EpaisseurBatons=2.5,
        Origine=2010,
    ]{%
        2011/4,2012/5,2013/24,2014/21,2015/18,2016/28,2017/22,2018/32,2019/22,2020/24,2021/42,2022/33,2023/25%
    }
    \end{center}
    \begin{enumerate}
        \item Calculez la moyenne annuelle des ventes de Minecraft entre 2011 et 2023, arrondie au million près.
        \nswr[12]{
            \begin{itemize}
                \item Les ventes annuelles sont : 4, 5, 24, 21, 18, 28, 22, 32, 22, 24, 42, 33, 25.
                \item La somme des ventes est :
                \\ $4 + 5 + 24 + 21 + 18 + 28 + 22 + 32 + 22 + 24 + 42 + 33 + 25 = 300$.
                \item Le nombre d'années est : 13.
                \item La moyenne est donc : $\text{Moyenne} = \dfrac{\text{Somme des valeurs}}{\text{Effectif total}} = \dfrac{300}{13} \approx 23$
                \item La moyenne annuelle des ventes est donc de 23 millions.
            \end{itemize}
        }
        \item Déterminez la médiane des ventes annuelles de Minecraft sur la période donnée.
        \nswr[\remaininglines]{
        \begin{itemize}
            \item On classe les ventes (en millions) par ordre croissant :
            \\ $4 < 5 < 18 < 21 < 22 < 22 < 24 < 24 < 25 < 28 < 32 < 33 < 42$.
            \item L'effectif total étant de 13 (nombre impair), la médiane est la valeur centrale, soit la $7^e$ valeur : 24 millions.
        \end{itemize}
        }
    \end{enumerate}
}[\href{https://www.businessofapps.com/data/minecraft-statistics/}{Business of Apps}]