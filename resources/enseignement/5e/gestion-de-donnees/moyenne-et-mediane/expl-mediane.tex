\mthd{Calcule d'une médiane}{
    On distingue deux cas :
    \begin{itemize}
        \item Si l'effectif total est impair, la médiane est la valeur centrale de la série.
        \item Si l'effectif total est pair, la médiane est n'importe quelle valeur entre les deux valeurs centrales, en pratique,
        on prend la moyenne des deux valeurs centrales.
    \end{itemize}
}[\wiki{Glossaire_des_statistiques}[Médiane]]

\exo{}{
    Trouver la médiane des listes
    \multiColEnumerate{2}{
        \item $10;9;1;12;16$
        \item $16;2;3;60;5;6$
    }
}

\nswr[0]{
    \begin{enumerate}
        \item \begin{itemize}
            \item On commence par trier la liste : $1<9<10<12<16$.
            \item L'effectif total est impair,
            la médiane est donc la valeur centrale : $10$.
        \end{itemize}
        \item \begin{itemize}
            \item On commence par trier la liste : $2<3<5<6<16<60$.
            \item L'effectif total est pair,
            la médiane est donc la moyenne des deux valeurs centrales : $\frac{5+6}{2} = 5.5$.
        \end{itemize}
    \end{enumerate}
}