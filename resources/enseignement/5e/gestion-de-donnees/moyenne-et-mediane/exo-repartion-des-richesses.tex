\exo{Repartition des richesses}{
    Sur un serveur Minecraft, la richesse de chaque joueur \icon{steve} est mesurée par le nombre de lingots d'or \icon{gold-ingot} qu'il possède. 
    L'histogramme ci-dessous présente la répartition des joueurs en fonction de leur richesse. Chaque classe correspond à un intervalle de lingots d'or possédés. 
    
    \begin{center}
        \Stat[Graphique,Histogramme,UniteAire=2,Pasx=32,Pasy=5,Unitex=0.5,Unitey=0.5,Lecture,DonneesSup,
        Donnee=\icon{gold-ingot},
        Effectif=\icon{steve},
        ListeCouleurs={Orange,Purple,Crimson,Cornsilk}]{%
        1/64/12,
        64/128/25,
        128/256/10,
        256/512/3
        }
    \end{center}
    
    \begin{enumerate}
        \item Estime la richesse moyenne des \icon{steve} sur ce serveur. 
        \item Détermine la richesse médiane à partir des données. 
        \item Compare et analyse tes résultats : que peux-tu conclure sur la manière dont les richesses sont réparties parmi les joueurs ?
    \end{enumerate}
}

\nswr[0]{
    \begin{enumerate}
        \item Pour estimer la richesse moyenne, nous devons calculer la moyenne des classes. Les classes sont :
        \multiColItemize{2}{
            \item 1 à 64 \icon{gold-ingot} : 12 \icon{steve}
            \item 64 à 128 \icon{gold-ingot} : 25 \icon{steve}
            \item 128 à 256 \icon{gold-ingot} : 10 \icon{steve}
            \item 256 à 512 \icon{gold-ingot} : 3 \icon{steve}
        }
        La richesse moyenne de chaque classe est approximée par le milieu de l'intervalle :
        \multiColItemize{2}{
            \item $(1+64)/2 = 32.5$ \icon{gold-ingot}
            \item $(64+128)/2 = 96$ \icon{gold-ingot}
            \item $(128+256)/2 = 192$ \icon{gold-ingot}
            \item $(256+512)/2 = 384$ \icon{gold-ingot}
        }
        La richesse moyenne est donc :
        \begin{align*}
            \dfrac{32.5 \times 12 + 96 \times 25 + 192 \times 10 + 384 \times 3}{12 + 25 + 10 + 3}
            &= \dfrac{390 + 2400 + 1920 + 1152}{50}\\
            &= \dfrac{5862}{50} = 117.24
        \end{align*}
        \item Pour déterminer la richesse médiane, nous devons trouver la classe médiane. Le nombre total de joueurs est 50, donc la médiane est la 25ème et 26ème valeur.
        \begin{itemize}
            \item Les 12 premiers \icon{steve} ont entre 1 et 64 \icon{gold-ingot}.
            \item Les 25 \icon{steve} suivants ont entre 64 et 128 \icon{gold-ingot}.
        \end{itemize}
        La médiane se situe donc dans la classe 64 à 128 \icon{gold-ingot}. Puisque cette classe contient les 25ème et 26ème valeurs, la médiane est approximativement 96 \icon{gold-ingot}.
        \item En comparant la moyenne (117.24 \icon{gold-ingot}) et la médiane (96 \icon{gold-ingot}),
        nous pouvons conclure que la distribution des richesses est légèrement asymétrique.
        Cela signifie qu'il y a quelques joueurs avec des richesses beaucoup plus élevées que la majorité,
        ce qui tire la moyenne vers le haut par rapport à la médiane.
    \end{enumerate}
}
