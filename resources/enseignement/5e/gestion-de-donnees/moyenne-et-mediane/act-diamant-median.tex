\act{Diamant médian}{
    Steve \icon{steve} explore une grotte et trouve des coffres \icon{chest}
    contenant des diamants \icon{diamond}.  
    Les \icon{chest} suivants contiennent respectivement:\begin{align*}
        8, \; 4, \; 6, \; 19, \; 2, \; 14, \et 4 \icon{diamond}
    \end{align*}
    \begin{enumerate}
        \item Combien de \icon{diamond} obtient-il en moyenne par \icon{chest}? \saveenumi
    \end{enumerate}

    \icon{steve} décide de calculer le nombre \key{médian} de \icon{diamond} par \icon{chest}, c'est-à-dire le nombre qui partage cette série en deux groupes contenant autant de valeurs.

    \begin{enumerate} \loadenumi
        \item Classe les nombres de \icon{diamond} par \icon{chest} dans l'ordre croissant.
        \item Détermine la médiane et explique son interprétation dans ce contexte.
    \end{enumerate}
}
\nswr[0]{
    \begin{enumerate}
        \item La moyenne de \icon{diamond} par \icon{chest} est calculée en additionnant tous les \icon{diamond} et en divisant par le nombre de \icon{chest}.
        Donc, $(8 + 4 + 6 + 19 + 2 + 14 + 4) / 7 = 54 / 7 \approx \np{8.14}$ \icon{diamond} par \icon{chest}.
        \item Les nombres de \icon{diamond} par \icon{chest} classés dans l'ordre croissant sont :\begin{align*}
            2, \; 4, \; 4, \; 6, \; 8, \; 14, \et 18 \icon{diamond}
        \end{align*}
        La médiane est la valeur centrale, donc ici c'est $6$.
        \item Cela signifie que la moitié des \icon{chest} contient 6 \icon{diamond} ou moins et l'autre moitié contient plus de 6 \icon{diamond} ou plus.
    \end{enumerate}
}