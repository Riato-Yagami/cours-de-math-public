\exo{Repartition de vols}{
    On a représenté sur le diagramme suivant les vols du mois de février d'une compagnie aérienne.
    
    \ctikz[0.45]{
    % \boundingBox[11.76][10.56][0.5pt][1][(-2.41,-2.86)]
    \draw [shift={(3.50,1.18)},thick,color=gradeColor,fill=gradeColor,fill opacity=0.10] (0,0) -- (89.85:1.13) arc (89.85:119.85:1.13) -- cycle;
    \draw [shift={(3.50,1.18)},thick,color=gradeColor,fill=gradeColor,fill opacity=0.10] (0,0) -- (119.85:1.13) arc (119.85:149.85:1.13) -- cycle;
    \draw [shift={(3.50,1.18)},thick,color=gradeColor,fill=gradeColor,fill opacity=0.10] (0,0) -- (149.85:1.13) arc (149.85:179.85:1.13) -- cycle;
    \draw[thick,color=gradeColor,fill=gradeColor,fill opacity=0.10] (2.70,1.18) -- (2.70,0.38) -- (3.50,0.38) -- (3.50,1.18) -- cycle;
    \draw [thick] (3.50,1.18) circle (7.48cm);
    \draw [shift={(3.50,1.18)},thick,color=gradeColor] (89.85:1.13) arc (89.85:119.85:1.13);
    \draw [shift={(3.50,1.18)},thick,color=gradeColor] (119.85:1.13) arc (119.85:149.85:1.13);
    \draw [shift={(3.50,1.18)},thick,color=gradeColor] (149.85:1.13) arc (149.85:179.85:1.13);
    \draw [shift={(3.50,1.18)},thick,color=gradeColor,fill=gradeColor,fill opacity=0.35]  (0,0) --  plot[domain=-1.57:1.57,variable=\t]({1*7.48*cos(\t r)+0*7.48*sin(\t r)},{0*7.48*cos(\t r)+1*7.48*sin(\t r)}) -- cycle ;
    \draw [shift={(3.50,1.18)},thick,color=gradeColor,fill=gradeColor,fill opacity=0.28]  (0,0) --  plot[domain=3.14:4.71,variable=\t]({1*7.48*cos(\t r)+0*7.48*sin(\t r)},{0*7.48*cos(\t r)+1*7.48*sin(\t r)}) -- cycle ;
    \draw [shift={(3.50,1.18)},thick,color=gradeColor,fill=gradeColor,fill opacity=0.21]  (0,0) --  plot[domain=2.62:3.14,variable=\t]({1*7.48*cos(\t r)+0*7.48*sin(\t r)},{0*7.48*cos(\t r)+1*7.48*sin(\t r)}) -- cycle ;
    \draw [shift={(3.50,1.18)},thick,color=gradeColor,fill=gradeColor,fill opacity=0.14]  (0,0) --  plot[domain=2.09:2.62,variable=\t]({1*7.48*cos(\t r)+0*7.48*sin(\t r)},{0*7.48*cos(\t r)+1*7.48*sin(\t r)}) -- cycle ;
    \draw [shift={(3.50,1.18)},thick,color=gradeColor,fill=gradeColor,fill opacity=0.07]  (0,0) --  plot[domain=1.57:2.09,variable=\t]({1*7.48*cos(\t r)+0*7.48*sin(\t r)},{0*7.48*cos(\t r)+1*7.48*sin(\t r)}) -- cycle ;
    \draw[color=gradeColor] (9.2,1.65) node {France};
    \draw[color=gradeColor] (-0.70,-2.86) node {Europe};
    \draw[color=gradeColor] (-1.6,2.2) node {Amérique};
    \draw[color=gradeColor] (-0.8,5.4) node {Afrique};
    \draw[color=gradeColor] (1.88,7.70) node {Asie};
    \draw [thick,gradeColor] (3.25,2.12) -- (3.17,2.42);
    \draw [thick,gradeColor] (2.81,1.87) -- (2.60,2.09);
    \draw [thick,gradeColor] (2.56,1.43) -- (2.26,1.51);
}
    Dans chaque cas, indiquer quelle fraction représentent les vols vers :
    \multiColItemize{3}{\item  la France \item l'Europe \item l'Asie}

    Au mois de février, cette compagnie a affrété 576 vols. Calculer le nombre de vols vers :
    \multiColItemize{3}{\item  la France \item l'Europe \item l'Asie}
}[\href{https://cache.media.education.gouv.fr/file/Fractions/22/7/RA16_C4_MATH_fractions_flash1_part_fractions_554227.pdf}
{Utiliser les nombres pour comparer, calculer et résoudre des problèmes : Les fractions - Un exemple de question flash - « Vision-partage » de la fraction}]
[\cmdGeoGebra[azrs82xa]]