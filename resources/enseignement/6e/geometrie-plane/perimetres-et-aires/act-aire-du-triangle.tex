\act{Aire du triangle}{
    \begin{enumerate}
        \item Construisez un triangle $ABC$ tel que :
        \begin{itemize}
            \item $AB = \Lg{6}$.
            \item La hauteur issue de $C$ coupe le segment $[AB]$ en un point $H$.
            \item $CH = \Lg{4}$.
        \end{itemize}
        \item Placez les points $E$ et $F$ de sorte que $AHCE$ et $BHCF$ soient des rectangles.
        \item \begin{enumerate}
            \item Trouvez l'aire du rectangle $AEFB$.
            \item Donnez une formule pour calculer l'aire de $AEFB$ en fonction des longueurs des segments $AB$ et $HC$.
        \end{enumerate}
        \item \begin{enumerate} 
            \item Comparez l'aire du rectangle $AHCE$ avec celle du triangle $AHC$.
            \item Faites de même pour $BHCF$ et $BHC$.
            \item En déduisez une comparaison des aires du rectangle $AEFB$ et du triangle $ABC$.
        \end{enumerate}
        \item \begin{enumerate}
            \item Quelle est l'aire du triangle $ABC$ ?
            \item Donnez une formule pour calculer l'aire de $ABC$ en fonction des longueurs des segments $AB$ et $HC$.
        \end{enumerate}
    \end{enumerate}
}