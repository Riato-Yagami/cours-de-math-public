% DOC https://ctan.math.illinois.edu/macros/latex/contrib/scratch3/scratch3-fr.pdf

\setscratch{scale=.75}

\def\block{{\setscratch{scale=.5}\begin{scratch}\blockmove{\Large bloc}\end{scratch} }}

\newcommand{\scr}[1]{\begin{scratch}#1\end{scratch}}

\definecolor{smotion}{HTML}{4C97FF} % #4C97FF
\definecolor{slooks}{HTML}{9966FF} % #9966FF
\definecolor{ssound}{HTML}{D65CD6} % #D65CD6
\definecolor{sevents}{HTML}{FFD500} % #FFD500
\definecolor{scontrol}{HTML}{FFAB19} % #FFAB19
\definecolor{ssensing}{HTML}{4CBFE6} % #4CBFE6
\definecolor{soperators}{HTML}{6DB26E} % #6DB26E
\definecolor{svariables}{HTML}{F28011} % #F28011
\definecolor{smyblocks}{HTML}{FF6680} % #FF6680

\def\smotion{\textcolor{smotion}{\faCircle\,Mouvement}} % Déplacement du lutin
\def\slooks{\textcolor{slooks}{\faCircle\,Apparence}} % Modifier l'apparence du lutin ou de la scène
\def\ssound{\textcolor{ssound}{\faCircle\,Son}} % Jouer des sons ou de la musique
\def\sevents{\textcolor{sevents}{\faCircle\,Événement}} % Déclencher des scripts en réponse à des actions
\def\scontrol{\textcolor{scontrol}{\faCircle\,Contrôle}} % Boucles, conditions, et contrôle du flux
\def\ssensing{\textcolor{ssensing}{\faCircle\,Capteur}} % Réagir à des informations extérieures ou internes
\def\soperators{\textcolor{soperators}{\faCircle\,Opérateur}} % Calculs mathématiques et logiques
\def\svariables{\textcolor{svariables}{\faCircle\,Variable}} % Stockage et manipulation de données
\def\smyblocks{\textcolor{smyblocks}{\faCircle\,Mes blocs}} % Création de blocs personnalisés

\def\spen{{\icon{scratch/pen} Stylo}}
\def\spenExtension{{\icon{scratch/pen-extension} Stylo}}
\def\sextensions{{\icon{scratch/extensions} $\lbrack$ Ajouter une extensions $\rbrack$}}
\def\sflag{{\icon{scratch/flag}%
%  Drapeau
}}

% \setscratch{scale=.75}
% \setscratch{print=true}
% \setscratch{fill blocks=true}

\dividePage{
    \exo{}{
        \begin{center}
            \begin{scratch}
                \blockinit{quand \greenflag est cliqué}
                \blockpen{effacer tout}
                \blockpen{stylo en position d'écriture}
                \blockrepeat{répéter \ovalnum{2} fois}{
                    \blockmove{avancer de \ovalnum{40} pas}
                    \blockmove{tourner \turnright{} de \ovalnum{90} degrés}
                    \blockmove{avancer de \ovalnum{60} pas}
                    \blockmove{tourner \turnright{} de \ovalnum{90} degrés}
                }
            \end{scratch}
        \end{center}
    }
}{
    \begin{enumerate}
        \item Tracer un schéma codé de ce qui est construit par ce programme \Scratch dans l'encadré.
        \ctikz[1]{
    \boundingBox[8][6][0.5pt][1][(-1,-1)][dot]
    \nswr[0]{
        \draw[thick,color=gradeColor,fill=gradeColor,fill opacity=0.10] (0,3.76) -- (0.24,3.76) -- (0.24,4) -- (0,4) -- cycle;
        \draw[thick,color=gradeColor,fill=gradeColor,fill opacity=0.10] (5.76,4) -- (5.76,3.76) -- (6,3.76) -- (6,4) -- cycle;
        \draw[thick,color=gradeColor,fill=gradeColor,fill opacity=0.10] (6,0.24) -- (5.76,0.24) -- (5.76,0) -- (6,0) -- cycle;
        \draw[thick,color=gradeColor,fill=gradeColor,fill opacity=0.10] (0.24,0) -- (0.24,0.24) -- (0,0.24) -- (0,0) -- cycle;
        \draw [thick] (0,4) -- (6,4) -- (6,0) -- (0,0) -- cycle;
        \draw [thick] (3,4.11) -- (3,3.89);
        \draw [thick] (6.11,2.05) -- (5.89,2.05);
        \draw [thick] (6.11,1.95) -- (5.89,1.95);
        \draw [thick] (3,-0.11) -- (3,0.11);
        \draw [thick] (-0.11,1.95) -- (0.11,1.95);
        \draw [thick] (-0.11,2.05) -- (0.11,2.05);
    }
    \draw [thick, ->] (0,4) -- (1,4);
    \draw [thick, |-|] (6,-1) -- (7,-1) ;
    \draw[color=gradeColor] (6.48,-.6) node {$10$ pas};
    \draw[color=gradeColor] (0.50,4.39) node {Départ};
}
        \item De quelle figure sagit-il ?
        \nswr[3]{Ce programme \Scratch trace un rectangle.}
    \end{enumerate}
    % \answerSec{5}[\textcolor{exo}{2.}][Ce programme \Scratch trace un rectangle.]
}[0.4]