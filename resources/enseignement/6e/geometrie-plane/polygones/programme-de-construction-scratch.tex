\dividePage{
    \exo{}{
        \begin{center}
            \begin{scratch}
                \blockinit{quand \greenflag est cliqué}
                \blockpen{effacer tout}
                \blockpen{stylo en position d'écriture}
                \blockrepeat{répéter \ovalnum{2} fois}{
                    \blockmove{avancer de \ovalnum{40} pas}
                    \blockmove{tourner \turnright{} de \ovalnum{90} degrés}
                    \blockmove{avancer de \ovalnum{60} pas}
                    \blockmove{tourner \turnright{} de \ovalnum{90} degrés}
                }
            \end{scratch}
        \end{center}
    }
}{
    \begin{enumerate}
        \item Tracer un schéma codé de ce qui est construit par ce programme \Scratch dans l'encadré.
        \ctikz[1]{
    \boundingBox[8][6][0.5pt][1][(-1,-1)][dot]
    \nswr[0]{
        \draw[thick,color=gradeColor,fill=gradeColor,fill opacity=0.10] (0,3.76) -- (0.24,3.76) -- (0.24,4) -- (0,4) -- cycle;
        \draw[thick,color=gradeColor,fill=gradeColor,fill opacity=0.10] (5.76,4) -- (5.76,3.76) -- (6,3.76) -- (6,4) -- cycle;
        \draw[thick,color=gradeColor,fill=gradeColor,fill opacity=0.10] (6,0.24) -- (5.76,0.24) -- (5.76,0) -- (6,0) -- cycle;
        \draw[thick,color=gradeColor,fill=gradeColor,fill opacity=0.10] (0.24,0) -- (0.24,0.24) -- (0,0.24) -- (0,0) -- cycle;
        \draw [thick] (0,4) -- (6,4) -- (6,0) -- (0,0) -- cycle;
        \draw [thick] (3,4.11) -- (3,3.89);
        \draw [thick] (6.11,2.05) -- (5.89,2.05);
        \draw [thick] (6.11,1.95) -- (5.89,1.95);
        \draw [thick] (3,-0.11) -- (3,0.11);
        \draw [thick] (-0.11,1.95) -- (0.11,1.95);
        \draw [thick] (-0.11,2.05) -- (0.11,2.05);
    }
    \draw [thick, ->] (0,4) -- (1,4);
    \draw [thick, |-|] (6,-1) -- (7,-1) ;
    \draw[color=gradeColor] (6.48,-.6) node {$10$ pas};
    \draw[color=gradeColor] (0.50,4.39) node {Départ};
}
        \item De quelle figure sagit-il ?
        \nswr[3]{Ce programme \Scratch trace un rectangle.}
    \end{enumerate}
    % \answerSec{5}[\textcolor{exo}{2.}][Ce programme \Scratch trace un rectangle.]
}[0.4]