\exo{Décrire des solides}{
    \ctikz[0.7]{
    \boundingBox[18][14][0.75pt][1][(-1,-1)][iso]
    \node at (4.41, 4.43) {\cir[gradeColor]{1}};
    \node at (9.61, 8.95) {\cir[gradeColor]{2}};
    \node at (13.07, 3.44) {\cir[gradeColor]{3}};
    \drawPoint{A}{1.73}{0}
    \drawPoint{B}{0}{1}
    \drawPoint{C}{3.46}{1}
    \drawPoint{D}{5.20}{0}
    \drawPoint{E}{7.79}{1.50}
    \drawPoint{F}{3.46}{4}
    \drawPoint{G}{0}{4}
    \drawPoint{H}{2.60}{5.50}
    \drawPoint{I}{7.79}{4.50}
    \drawPoint{J}{5.20}{3}
    \drawPoint{K}{1.73}{3}
    \drawPoint{L}{2.60}{2.50}
    \draw [thick] (0,4) -- (2.60,5.50) -- (7.79,4.50) -- (7.79,1.50) -- (5.20,0) -- (3.46,1) -- (1.73,0) -- (0,1) -- (0,4) -- (1.73,3) -- (3.46,4) -- (5.20,3) -- (7.79,4.50);
    \draw [thick] (5.20,0) -- (5.20,3);
    \draw [thick] (3.46,1) -- (3.46,4);
    \draw [thick] (1.73,0) -- (1.73,3);
    \draw [thick] (9.53,2.50) -- (12.99,6.50) -- (11.26,1.50) -- (9.53,2.50);
    \draw [thick] (12.99,6.50) -- (14.72,1.50) -- (11.26,1.50);
    \draw [thick] (12.99,6.50) -- (16.45,2.50) -- (14.72,1.50);
    \draw [thick,dashed] (12.99,6.50) -- (12.99,4.50);
    \draw [thick,dashed] (16.45,2.50) -- (12.99,4.50);
    \draw [thick,dashed] (12.99,4.50) -- (9.53,2.50);
    \draw [thick] (6.06,7.50) -- (6.06,9.50) -- (8.66,6) -- (12.99,8.50) -- (10.39,12) -- (6.06,9.50);
    \draw [thick,dashed] (10.39,12) -- (10.39,10);
    \draw [thick,dashed] (10.39,10) -- (6.06,7.50);
    \draw [thick] (6.06,7.50) -- (8.66,6);
    \draw [thick,dashed] (10.39,10) -- (12.99,8.50);
    \draw [thick,dashed] (0,1) -- (2.60,2.50);
    \draw [thick,dashed] (2.60,2.50) -- (2.60,5.50);
    \draw [thick,dashed] (2.60,2.50) -- (7.79,1.50);
}
    \begin{enumerate}
        \item À quelle famille commune appartiennent tous ces solides ?
        \\\nswr[2]{Il s'agit de polyèdres.}
        \item À quelle famille particulière appartient chacun de ces solides ?
        \\\nswr[4]{Les solides \cir[gradeColor]{1} et \cir[gradeColor]{2} sont des prismes droits
        et le solide \cir[gradeColor]{3} est une pyramide.}
        \item Combien de faces, arêtes et sommets compte le solide \cir[gradeColor]{2} ?
        \\\nswr[3]{Le solide \cir[gradeColor]{2} a $5$ faces, $9$ arêtes et $6$ sommets.}
        \item Dans le solide \cir[gradeColor]{1}, nommez :
        \begin{enumerate}
            \item Une face parallèle à la face $BGKA$.
            \\\nswr[3]{La face $KACF$ semble perpendiculaire à la face $BGKA$.}
            \item Deux arêtes adjacentes et perpendiculaires à l'arête $[HI]$.
            \\\nswr[3]{Les arêtes $[IE]$ et $[HL]$ sont perpendiculaires à l'arête $[HI]$.}
        \end{enumerate}
    \end{enumerate}
}