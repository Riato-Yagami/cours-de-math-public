\def\figScale{0.6}
\newcommand{\sld}[2]{\item \dividePage{#1}
{\input{resources/enseignement/6e/geometrie-dans-l-espace/solides/tikz/#2.tex}}[0.5]
}

% \newcommand{\sld}[2]{\item #1
%     \input{resources/enseignement/6e/geometrie-dans-l-espace/solides/tikz/#2.tex}
% }

\df{Polyèdres}{%
    On appelle :%
    \multiColItemize{1}{%
        \sld{%
            \key{polyèdre}, un solide constitué de \key{faces} polygonales.\\
            Les côtés de ces polygones sont appelés  \key{arêtes}.\\
            Les extrémités des arêtes sont des points appelés  \key{sommets}.
        }{polyhedre}
        \sld{%
            \nswr{\key{prisme droit}}[5cm], un polyèdre constitué de deux \key{bases} polygonales superposables,
            reliées entre elles par des faces \nswr{\key{rectangulaires}}[5cm].
        }{prisme-droit}
        \sld{%
            \nswr{\key{pavé droit}}[5cm], un \nswr{prisme droit}[5cm] dont les bases sont \nswr{\key{rectangulaires}}.
        }{pave-droit}
        \sld{%
            \nswr{\key{cube}}[5cm], un \nswr{pavé droit}[5cm] dont toutes les faces sont des \nswr{\key{carrés}}[5cm].
        }{cube}
        \sld{%
            \nswr{\key{pyramide}}[5cm], un polyèdre formé d'une base polygonale reliée à un \key{sommet} par des faces \nswr{\key{triangulaires}}[5cm].
        }{pyramide}
    }
}[\wiki{Polyèdre}[Polyèdres_simples]]