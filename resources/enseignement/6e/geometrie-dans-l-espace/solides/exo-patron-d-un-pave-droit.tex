\act{Patron d'un pavé droit}{
    On considère le pavé droit ci-dessous :

    \ctikz[1]{
    \boundingBox[8][7][0.5pt][1][(0,-0.5)][iso]
    \drawPoint{A}{3.46}{0}
    \drawPoint{B}{0.87}{1.50}
    \drawPoint{C}{0.87}{3.50}
    \drawPoint{G}{4.33}{5.50}
    \drawPoint{E}{6.93}{2}
    \drawPoint{D}{3.46}{2}
    \drawPoint{H}{6.93}{4}
    \drawPoint{F}{4.33}{3.50}
    \draw[color=gradeColor] (1.96,0.64) node {3};
    \draw[color=gradeColor] (3.26,1.15) node {2};
    \draw[color=gradeColor] (5.34,0.87) node {4};
    \draw [thick] (0.87,1.50) -- (3.46,0) -- (3.46,2) -- (6.93,4) -- (4.33,5.50) -- (0.87,3.50) -- (0.87,1.50);
    \draw [thick] (6.93,4) -- (6.93,2) -- (3.46,0);
    \draw [thick] (3.46,2) -- (0.87,3.50);
    \draw [thick,dashed] (4.33,3.50) -- (0.87,1.50);
    \draw [thick,dashed] (4.33,3.50) -- (6.93,2);
    \draw [thick,dashed] (4.33,3.50) -- (4.33,5.50);
}

    \begin{enumerate}
        \item Combien de faces compose ce pavé droit ?
        \item Sans considérer les noms des sommets, combien de polygones différents composent ces faces ?
        Préciser leurs natures.
        \item Dessinez ces polygones. (Les unités de mesure sont données en \Lg[cm]{})
        \item Combien de fois chacun de ces polygones apparaît-il dans le pavé droit?
        \item Dessinez à main levée tous les polygones avec les noms des sommets.
        \item Sur une feuille blanche, dessinez tous les polygones avec les noms de leurs sommets.
        Chaque polygone doit partager au moins un côté avec un autre polygone.
        Les sommets doivent être nommés et peuvent apparaître plusieurs fois.
        \item Découpez le patron du pavé droit obtenu à l'étape précédente et pliez-le pour obtenir un pavé droit.
    \end{enumerate}
}