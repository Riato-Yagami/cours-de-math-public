\def\size{6.5}
\def\wid{\size cm}
\def\len{2*\size cm}

% \newcommand{\billard}[2]{
%     \begin{center}
%         \ifthenelse{\boolean{answer}}
%         {%
%             \phantom{#1}
%             \vspace*{-\wid}\vspace*{-1.5cm}
%             #2
%         }
%         {#1}
%     \end{center}
% }

\newcommand{\billard}[2]{%
    \begin{center}
        #1\nswr[0]{\\#2}
    \end{center}
}

%% dimensions billard : 2,54 m x 1,27 m -> ratio 1:2

\exo{Billard - Mots cachés}{
    Dans cet exercice, nous allons étudier le déplacement d'une boule de billard envoyée depuis le point $S$.
    \begin{center}
        \billard{\Billard[Longueur=\len,Largeur=\wid]{"SKIEUR"}}
        {\Billard[Solution,Longueur=\len,Largeur=\wid]{"SKIEUR"}}
    \end{center}
    \begin{enumerate}
        \item Relie les points $S, K, I, E, U$ et $R$ dans cet ordre pour représenter le trajet de la boule.
        \item Comment sont les angles $\widehat{SKI}$, $\widehat{KIE}$, $\widehat{IEU}$ et $\widehat{EUR}$ ?
        \nswr[0]{\\Ce sont tous des angles droits.}
        \item Écris une phrase expliquant le comportement de la boule lorsqu'elle heurte un bord de billard.
        \nswr[0]{\\Lorsque la boule heurte un bord de billard, elle rebondit en formant un angle droit.}
        \item Le billard précédent cachait le mot «SKIEUR». Trace le trajet de la boule dans le billard suivant pour découvrir le mot caché en respectant les mêmes règles de rebonds. Place les lettres sur les tirets ci-dessous, qui sont au nombre de 8, tu devrais donc trouver un mot de 8 lettres.
    \end{enumerate}
    \billard{\Billard[Depart=1.6,Angle=30,Longueur=\len,Largeur=\wid]{"MONTAGNE"}}
    {\Billard[Solution,Depart=1.6,Angle=30,Longueur=\len,Largeur=\wid]{"MONTAGNE"}}
}

\newpage

\exo{Billard - Des rebonds plus réalistes}{
    \vspace*{0.25cm}
    \billard{\Billard[Vrai,Depart=2.6,Angle=30,Longueur=\len,Largeur=\wid]{"ALPIN"}}
    {\Billard[Solution,Vrai,Depart=2.6,Angle=30,Longueur=\len,Largeur=\wid]{"ALPIN"}}
    \vspace*{-\size cm}
    \vspace*{-2.75cm}
    \hspace*{1cm}
    \begin{tikzpicture}
        \draw (0,\size) node[above left] {$D_1$};
        \draw (2*\size,\size) node[above right] {$C_1$};
        \draw (2*\size,0) node[below right] {$B_1$};
        \draw (0,0) node[below left] {$A_1$};
    \end{tikzpicture}
    \vspace*{1cm}
    \begin{enumerate}
        \item Dans ce billard, le mot «ALPIN» est caché. Trace le trajet de la boule pour retrouver ce mot.
        \item \begin{enumerate}
            \item Compare les angles $\widehat{ALD_1}$ et $\widehat{PLA_1}$.
            \nswr[0]{\\Les deux angles $\widehat{ALD_1}$ et $\widehat{PLA_1}$ mesurent tous les deux $\ang{60}$.}
            \item Compare l'angle $\widehat{LPA_1}$ avec l'angle «en face» de même sommet.
            \nswr[0]{\\Les deux angles $\widehat{LPA_1}$ et l'angle «en face» $\widehat{IPB_1}$ mesurent tous les deux $\ang{30}$.}
            \item Fais de même pour le dernier rebond.
            \nswr[0]{\\Pour le dernier rebond, on peut comparer les angles $\widehat{C_1IN}$ et l'angle «en face» $\widehat{B_1IP}$ mesurant tous les deux $\ang{60}$.}
        \end{enumerate}
        \item Écris une phrase expliquant le comportement d'une boule lorsqu'elle heurte un bord de billard.
        \nswr[0]{\\Lorsque la boule heurte un bord de billard, elle rebondit en formant des angles égaux avec le bord.}
        \item Trace le trajet de la boule dans le billard pour découvrir le mot caché en respectant les mêmes règles de rebonds.
    \end{enumerate}
    \billard{\Billard[Vrai,Depart=0.5,Angle=55,Longueur=\len,Largeur=\wid]{"SLALOM"}}
    {\Billard[Solution,Vrai,Depart=0.5,Angle=55,Longueur=\len,Largeur=\wid]{"SLALOM"}}
}[\href{https://www.mathmaurer.com/maths_6/mes_01/exercices/6_mes1_exe_a.html}{Mathmaurer}]
