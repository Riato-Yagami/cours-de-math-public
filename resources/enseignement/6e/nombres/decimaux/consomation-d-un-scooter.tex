\exo{Consommation d'un scooter}{
    Pierrick fait plusieurs fois le même trajet de \Lg[km]{30} avec son scooter.
    
    Il calcul à chaque fois sa vitesse moyenne et note sa consommation en carburant.
    
    Voici ces relevés :
    \multiColItemize{6}{
        \item \Capa{2} \item \Capa{1.10} \item \Capa{2.3} \item \Capa{1.13} \item \Capa{2.03} \item \Capa{1.2}
    }

    La consommation de carburant augmente lorsque la vitesse moyenne augmente.
    
    Place les consommations dans le tableau.

    \begin{center}
        \begin{tabular}{|C{3cm}|*{6}{C{\cW}|}}
            \hline
            Vitesse en \Vitesse[kmh]{} & $20$ & $22$ & $25$ & $35$ & $40$ & $45$ \\
            \hline
            Consommation en \Capa[l]{}
            & \nswr[0]{\np{1.10}}
            & \nswr[0]{\np{1.13}}
            & \nswr[0]{\np{1.2}}
            & \nswr[0]{\np{2}}
            & \nswr[0]{\np{2.03}}
            & \nswr[0]{\np{2.3}} \\
            \hline
        \end{tabular}
    \end{center}

}[\dmeepc{6}[234]]