% BEAMER CONVERSION

% \newcommand{\bchap}[1]{\def\title{Chapitre: #1}}
\newcommand{\bseq}[1]{\def\title{\textbf{\color{Red} Séquence \theseq{} : #1}}}
\newcommand{\bsec}[1]{\section{#1}}
\newcommand{\bsubsec}[1]{\subsection{#1}}

\newcommand{\sseq}{}
\newcommand{\ssec}{}
\newcommand{\ssubsec}{}

\newcommand{\slide}[2]{%
\ifNotNull{#1}{%
    \begin{center}
        \color{Maroon}\textbf{- #1 -}
    \end{center}
}

    #2
}

\newcommand{\startQuestions}{}
\newcommand{\iquestion}[2]{\item $#1 = \result{#2}$}

\newcommand{\palt}[2]{\result{#2}}
\NewDocumentCommand{\aalt}{o m m}{%
    \noindent #2\\#3%
}

\newcommand{\disableAnimation}{}
\newcommand{\shortAnimation}{}

\newcommand{\firstSlide}{
    \renewcommand{\iquestion}[2]{\item $##1 = \phantom{##2}$}
    \renewcommand{\palt}[2]{
        \phantom{##2}
    }
}

\newenvironment{columns}[1][T]{}{}
\newenvironment{column}[1]{\begin{minipage}{#1}}{\end{minipage}}

\NewDocumentCommand{\qf}{m O{15}}{
    \qfSUB{}{
        \qfRes{#1}
    }
}

\newcounter{annex}
\renewcommand{\theannex}{\Alph{annex}} % Define how the annex counter will be displayed

\newcommand{\annex}[1]{%
    \changelocaltocdepth{0}
    \setcounter{section}{0}%
    \setcounter{subsection}{0}%
    \setcounter{subsubsection}{0}%
    \newpage%
    \fancyhead[L]{\color{Red} ANNEXE \theannex}
    \refstepcounter{annex}
    \label{annex:\theannex}
    \input{#1}
    \changelocaltocdepth{2}
}

\newcommand*{\rannex}[1]{
    (\hyperref[annex:#1]{Annexe #1})
}

\newcommand{\changelocaltocdepth}[1]{%
    \addtocontents{toc}{\protect\setcounter{tocdepth}{#1}}%
    \setcounter{tocdepth}{#1}%
}

\NewDocumentCommand{\dividePage}{mm O{0.5}}{
    \pgfmathparse{1-#3}
    % \newcount\secondPage
    % \pgfmathsetcount{\secondPage}{\pgfmathresult}
    \begin{minipage}{#3\linewidth}
        #1
    \end{minipage}
    \begin{minipage}{\pgfmathresult\linewidth}
        #2
    \end{minipage}
}

\newcommand{\ifArticle}[1]{#1}
\newcommand{\ifBeamer}[1]{}

\newcommand\wideFrame[2][3em]{#2}
\newcommand{\bvspace}[1]{}
\newcommand{\avspace}[1]{\vspace*{#1}}