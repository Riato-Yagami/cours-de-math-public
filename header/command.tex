% ENVIRONMENT
\newenvironment{mysection}[1][gray!20]{%
    \ifthenelse{\boolean{boxedProperties}}{
        \begin{tcolorbox}[colback=white,
            colframe=#1!50,
            boxrule=5pt,
            outer arc=0pt,
            arc=0pt,
            grow to left by= -5pt
    ]
    }{
        \begin{tcolorbox}[colback=white,
            colframe=#1!50,
            boxrule=-1pt,
            arc=0pt,
            leftrule=3pt,
            left = 8pt,
            outer arc=0pt,
            grow to left by= -5pt
        ]
    }
}{%
    \end{tcolorbox}
}

\newenvironment{mysubsection}[1][gray!20]{%
    \begin{tcolorbox}[colback=white,
        colframe=#1!50,
        boxrule=-1pt,
        outer arc=0pt,
        arc=0pt,
        grow to left by= -5pt
    ]
}{%
    \end{tcolorbox}
}

% Switch implementation
\newboolean{default}
\newcommand{\case}{}
\newcommand{\default}{}

\newenvironment{switch}[1]{%
    \setboolean{default}{true}
    \renewcommand{\case}[2]{\ifthenelse{\equal{#1}{##1}}{%
        \setboolean{default}{false}##2}{}}%
    \renewcommand{\default}[1]{\ifthenelse{\boolean{default}}{##1}{}}
}{}

% SECTIONS
\newcommand{\parseSecContent}[2]{% Color | Content
    \ifNotNull{#2}{%
        \begin{mysection}[#1]
            #2
        \end{mysection}
    }
}

\newcommand{\parseSecTitle}[3]{% Color | Title | Type
    \color{#1}\target
    {#3}\ifthenelse{\boolean{showID}}{\ifthenelse{\boolean{parenthisedID}}
    {\small$_{(\sectionID)}$}
    { \sectionID}}{}\normalsize : 
    \color{black}#2
}

\makeatletter
\newcommand{\parseSec}[4]{% Color | Title | Content | Type
    \def\secContent{\parseSecContent{#1}{#3}}
    \def\secTitle{\parseSecTitle{#1}{#2}{#4}}

    \@ifclassloaded{beamer}{
        \textbf{\secTitle}
    }{
        \subsection*{\secTitle}
    }

    \ifthenelse{\boolean{outline}}{}{\secContent}
}
\makeatother

\newcommand{\parseSubsecContent}[2]{ % Color | Content
    #2
}

\newcommand{\parseSubsecTitle}[3]{ % Color | Title | Type
    \color{#1}\target
    {#3}\normalsize : 
    \color{black}#2
}

\makeatletter
\newcommand{\parseSubsec}[4]{ % Color | Title | Content | Type
    \def\secContent{\parseSubsecContent{#1}{#3}}
    \def\secTitle{\parseSubsecTitle{#1}{#2}{#4}}

    \ifthenelse{\boolean{outline} \and \not \boolean{subsectionInOutline}}{}{
        \@ifclassloaded{beamer}{
            \begin{mysubsection}[#1]
                \textbf{\secTitle \ifx\relax#2\relax\else \\ \fi}
                \hspace*{-0.35cm}\secContent
            \end{mysubsection}
        }{
            \subsubsection*{\secTitle}
            \ifthenelse{\boolean{outline}}{}{\secContent}
        }
    }
}
\makeatother

\makeatletter
\newcommand{\newsec}[4]{ % #1 Level | #2 Code | #3 Color | #4 Type
    \newcounter{#2}

    \expandafter\NewDocumentCommand\csname #2\endcsname{m +m o}{ % ##1 Title | ##2 Content
        \stepcounter{#2}

        \def\sectionID{\csname the#2\endcsname}

        % Define the label for the section
        \def\sectionLabel{#2-\sectionID}

        % Create the section label
        \label{\sectionLabel}
        
        % Define the target for hyperlink
        \def\target{\hypertarget{\sectionLabel}}
        
        % Create Ref Hyperlink
        \ifNotNull{##1}{
            \expandafter\newcommand\csname #2-##1\endcsname[1]{%
                \hyperlink{#2-\sectionID}{
                    \textbf{\color{#3!80}#4 ##1}\color{black}
                }
            }%
        }

        \begin{switch}{#1}
            \case{sec}{\parseSec{#3}{##1}{##2}{#4}}
            \default{
                \parseSubsec{#3}{##1}{##2}{#4}
            }
        \end{switch}

        % Reference
        \ifNotNull{##3}{ 
            \ifthenelse{\boolean{outline} \or \not \boolean{showRef}}{}{
                \vspace{-0.5cm}
                \begin{flushright}
                \small\textit{##3 $\leftarrow$}\hspace*{0.5cm}
                \end{flushright}
            }
        }
    }
}
\makeatother

\newcommand{\refsec}[2]{%
    \expandafter\csname #1-#2\endcsname{#2}%
}

\newsec{sec}{pr}{\PropertyColor}{Propriété}
\newsec{sec}{df}{\DefinitionColor}{Définition}
\newsec{sec}{thm}{\TheoremColor}{Théorème}
\newsec{sec}{scnSUB}{OrangeRed}{Séance}

\newsec{sub}{cor}{NavyBlue}{Corollaire}
\newsec{sub}{lem}{NavyBlue}{Lemme}
\newsec{sub}{ctr}{NavyBlue}{Contraposée}
\newsec{sub}{rmdr}{Thistle}{Rappel}
\newsec{sub}{mthd}{OrangeRed}{Méthode}

\newsec{subsub}{rmk}{Thistle}{Remarque}
\newsec{subsub}{expl}{gray}{Exemple}
\newsec{subsub}{app}{RedViolet}{Application}
\newsec{subsub}{exo}{Gray}{Exercice}
\newsec{subsub}{act}{BurntOrange}{Activité}
\newsec{subsub}{dm}{Gray}{Devoir maison}

\newsec{subsub}{demoSUB}{Red}{$\rightarrow$ Démonstration}
\newsec{subsub}{ticeSUB}{Orchid}{TICE}
\newsec{subsub}{qfSUB}{Red}{\Large Questions flash}

\NewDocumentCommand{\demo}{mmo}{
    \ifthenelse{\boolean{demonstration} \and \not \boolean{outline}}{
        \demoSUB{#1}{
            \ifNotNull{#2}{
                \noindent #2 $\square$
            }
        }[#3]
    }{
    }
}

\newcommand*{\scn}[2]{%
    \scnSUB{#1}{%
        \ifNotNull{#2}{%
            \begin{itemize}[wide = 0pt, label=]%
                #2%
            \end{itemize}%
        }%
    }%
}

\newcommand{\scni}[2]{
    \item \color{OrangeRed}#1 : \color{black} #2
}

\NewDocumentCommand{\tice}{mmmo}{
    \ticeSUB{#1 - #2}{#3}[#4]
}

\newcommand{\hint}[1]{
    \ifthenelse{\boolean{outline}}{}{\textit{Indication: #1}}
}

% QUESTIONS FLASHS

% \NewDocumentCommand{\qfb}{m O{15}}{
%     % Initialize an empty token list to accumulate the formatted questions and answers
%     \def\formattedList{}
%     % Iterate over the input list
%     \foreach \q in {#1}{
%         % Extract the question and answer parts
%         \xdef\formattedList{\formattedList{${\listElement{\q}{0}} = $,${\listElement{\q}{1}}$},}
%     }
%     % Remove the last comma
%     \edef\formattedList{\unexpanded\expandafter{\formattedList}}
%     % Pass the formatted list to \qf
%     \expandafter\qf\expandafter{\formattedList}[#2]
% }

\newcommand{\qfRes}[1]{
    \begin{enumerate}
        \foreach \q in {#1}{
            \item \listElement{\q}{0}\palt{2}{\listElement{\q}{1}}
        }
    \end{enumerate}
}

% ANSWERS
\newlength{\parline}
\newlength{\paroutindent}
\newlength{\lineheight}
\setlength{\lineheight}{\heightof{abcdefghijklmnoprstuvwxyz}}

\newcommand{\countlines}[1]{%
    \setlength{\paroutindent}{\expandafter\parindent}
    \setlength{\parline}{\heightof{\noindent\begin{minipage}{\linewidth}%
                \setlength{\parindent}{\paroutindent}#1\end{minipage}}}%
    \pgfmathparse{round(\parline / (0.9*\lineheight))}
    \newcount\linecount
    \pgfmathsetcount{\linecount}{\pgfmathresult}
}

\newcommand{\looptext}[2]{%
    \noindent
    \newcount\printcount
    \printcount=#2
    \loop
        #1
        \advance\printcount by -1
        \ifnum\printcount>0
    \repeat
}

\newcommand{\awsr}[1]{%
    \ifthenelse{\boolean{answer}}{
        \result{#1}
    }{
        \countlines{#1}
        \pgfmathsetcount{\linecount}{\linecount + 1}
        \noindent\hspace{-9pt}
        \looptext{
            \noindent\dotfill
    
        }{\the\linecount}
    }
}

\newcommand{\dottedLines}[1]{%
    \noindent\hspace{-9pt}%

    \looptext{%
        \noindent\dotfill%

    }{#1}
}

\newcommand{\result}[1]{
    \color{OrangeRed}#1\color{black}
}

% MATH
\newcommand{\modxy}[2]{\sqrt{#1^2+#2^2}}
\newcommand{\acroissement}[2]{\frac{#1(#2+h)-#1(#2)}{h}}

\newcommand{\pow}[2]{
    #1
    \foreach \n in {2,...,#2}{
        \times #1
    }
}

% \newcommand{\powTen}[1]{
%     \ifnum#1=1 10
%     \else \ifnum#1=0 1
%     \else \ifnum#1>0 1\foreach \n in {2,...,#1}{0} \fi
% }

% \newcounter{rec}
% \newcommand{\powTen}[1]{
%     \setcounter{rec}{#1}
%     \ifnum\therec=0 1
%     \else \addtocounter{rec}{-1} \powTen{\therec}0 \fi
% }

\newcommand{\powTen}[1]{
    \ifnum#1<0
        0,\powTenNegative{\numexpr#1*(-1)\relax}
    \else
        \powTenPositive{#1}
    \fi
}

\newcommand{\powTenPositive}[1]{
    \ifnum#1=0 1
    \else \powTen{\numexpr#1-1\relax}0 \fi
}

\newcommand{\powTenNegative}[1]{
    \ifnum#1=1 1
    \else 0\powTenNegative{\numexpr#1-1\relax} \fi
}

\newcommand{\lPowBrace}[2]{
    \underbrace{\pow{#1}{2} \times ... \times \pow{#1}{2}}_{#2 \textrm{ fois}}
}
\newcommand{\powBrace}[2]{
    \underbrace{#1 \times ... \times #1}_{#2 \textrm{ fois}}
}

\newcommand{\lfbrace}[1]{
    \left\{
        \begin{array}{ll}
            #1
        \end{array}
    \right.
}

\newcommand{\pgcd}[2]{
    % \PGCD(#1,#2)
    #1 \wedge #2
}

\newcommand{\ppcm}[2]{
    % \PPCM(#1,#2)
    #1 \vee #2
}

\def\colWidth{1.5cm}
\newcommand{\propTable}[4]{
    \begin{tabular}{|C{\colWidth}|C{\colWidth}|}
        \hline
        $#1$ & $#2$ \\ \hline
        $#3$ & $#4$ \\ \hline
    \end{tabular}
}

\newcommand{\pscal}[2]{\langle #1, #2 \rangle}

\newcommand{\coord}[2]{% 
    \ensuremath{
        \begin{pmatrix} 
            #1\\ 
            #2
        \end{pmatrix}
    }
}

% COMMANDS

\newcommand{\fsize}[1]{\fontsize{#1}{#1}\selectfont}

\NewDocumentCommand{\ifNotNull}{mmo}{
    \IfValueT{#1}{
        \ifx\relax#1\relax
            \IfValueT{#3}{
                #3
            }
        \else
            #2
        \fi
    }
}

\NewDocumentCommand{\limg}{m O{\linewidth}}{%
    \includegraphics[width=#2]{#1}%
}

\newsavebox{\picbox}

\newcommand{\cutpic}[3]{
    \savebox{\picbox}{\includegraphics[width=#2]{#3}}
    \tikz\node [draw, rounded corners=#1, line width=4pt,
    color=white, minimum width=\wd\picbox,
    minimum height=\ht\picbox, path picture={
        \node at (path picture bounding box.center) {
        \usebox{\picbox}};
    }] {};
}

\NewDocumentCommand{\rimg}{m O{\linewidth}}{%
    \begin{center}
        \cutpic{0.4cm}{#2}{#1}
    \end{center}
}

\NewDocumentCommand{\imgp}{m O{\linewidth} O{0cm} o}{%
    \img{\imgf{#1}}[#2]
}

\NewDocumentCommand{\img}{m O{\linewidth} O{0cm} o}{%
    \begin{figure}[H]%
        \centering%
        \hspace*{#3}
        \includegraphics[width=#2]{#1}%
        \IfValueTF{#4}{
            \caption{\underline{#4}}
            % \vspace{-0.4cm}
            }{
            % \vspace{-0.4cm}
        }
        \vspace{-0.45cm}
    \end{figure}%
}

\NewDocumentCommand{\ilink}{m g}{%
    \item
    \IfValueTF{#2}{\link{#1}{#2}}{\link{#1}}
}

\NewDocumentCommand{\link}{m g}{%
    \csn{#1}%
    \IfValueT{#2}{(#2)}%
}

\NewDocumentCommand{\TODO}{g}{%
    \color{Red} $\rightarrow$ \textbf{TODO}
    \IfValueT{#1}{(#1)}%
}

\newcommand{\leconInfoBox}[2]{
    \textbf{#1 :}\vspace{-0.25cm}
        \begin{multicols}{2}
            \begin{itemize}[label=$\blacktriangleright$, font = \small \color{Red}]
                #2
            \end{itemize}
        \end{multicols}
        \vspace{-0.4cm}
}

% TCOLORBOX

\newtcolorbox{infoBox}{
    colback=red!5!white,
    colframe=red!75!black,
    breakable
}

\newtcolorbox{prepBox}{
    enhanced jigsaw,
    breakable,
    colback=Blue!2!white,
    colframe=Blue!75!black,
    grow sidewards by = 1cm,
}

\newtcolorbox{twoColBox}{
    sidebyside, sidebyside align=top,
    enhanced,
    opacityback=0,
    colframe=red!75!black,
    frame hidden,
}

\NewDocumentCommand{\leconInfo}{mooo}{
    \begin{infoBox}
        \leconInfoBox{Niveaux}{#1}
        \ifNotNull{#2}{
            \tcbline
            \leconInfoBox{Prérequis}{#2}
        }
        \ifNotNull{#3}{
            \tcbline
            \leconInfoBox{Thèmes}{#3}
        }
        \ifNotNull{#4}{
            \tcbline
            \textbf{Motivation :} 
            #4
        }
    \end{infoBox}
}

\NewDocumentCommand{\seanceInfo}{oooooooo}{
    \begin{infoBox}
        \vspace{-0.05cm}
        \begin{tcbitemize}[raster rows=1,raster columns=20,raster height=1.65cm,
            raster every box/.style={colframe=red!50!black,colback=red!10!white}]
            \tcbitem[raster multicolumn=6] \textbf{Date :} #1
            \tcbitem[raster multicolumn=10] \textbf{Séquence :} #2
            \tcbitem[raster multicolumn=4] \textbf{Séance :} #3
        \end{tcbitemize}
        \vspace{-0.25cm}
        \ifNotNull{#4}{\tcbline \textbf{Objectif :} #4}
        \ifNotNull{#5}{\tcbline \leconInfoBox{Classe(s)}{#5}}
        \ifNotNull{#6}{\tcbline \leconInfoBox{Prérequi(s)}{#6}}
        \ifNotNull{#7}{\tcbline \textbf{Séance précédente :} #7}
        \ifNotNull{#7}{\tcbline \leconInfoBox{Matériel(s)}{#8}}
    \end{infoBox}
}

\def\pDscr{\tcbitem[enhanced jigsaw, breakable,
    raster multicolumn=6]
}
\def\pMdlt{\tcbitem[enhanced jigsaw, breakable,
    raster multicolumn=11]
}
\def\pTime{\tcbitem[enhanced jigsaw, breakable,
    raster multicolumn=3, halign=center]
}

\newcommand{\prepRow}[3]{
    \tcbitem[raster multicolumn=20]
    \tcblower

    \pDscr #1
    \pMdlt #2
    \pTime #3
}

\newcommand{\prepTable}[1]{
    \begin{prepBox}
        \begin{tcbitemize}[enhanced jigsaw, breakable, raster rows=1,raster columns=20,raster height=1.1cm, halign=center,
            raster every box/.style={enhanced jigsaw, breakable, colframe=Blue!50!black,colback=Blue!10!white}]
            \pDscr \textbf{Descriptif}
            \pMdlt \textbf{Modalité}
            \pTime \textbf{Durée}
        \end{tcbitemize}
        \begin{tcbitemize}[enhanced jigsaw, breakable,
            raster equal height = rows, 
            raster columns=20, frame hidden,
            raster every box/.style={
                enhanced jigsaw, breakable,
                opacityback=0, valign=top, 
                size = tight
            }]
            #1
        \end{tcbitemize}
    \end{prepBox}
}

% TIKZ

\newcommand{\ctikz}[1]{
    \begin{center}
        \begin{tikzpicture}
            #1
        \end{tikzpicture}
    \end{center}
}

\newcommand{\axis}[1]{%Draw coordinate axes
    \draw[thin, -Stealth] (-0.5,0) -- (#1,0);% node[right] {$x$}; % x-axis
    \draw[thin, -Stealth] (0,-0.5) -- (0,#1);% node[above] {$y$}; % y-axis
}

\newcommand{\drawGrid}[3]{
    \foreach \n in {0,...,#1}
        \draw[line width = #3] (\n,0) -- (\n,#2);
    \foreach \n in {0,...,#2}
        \draw[line width = #3] (0,\n) -- (#1,\n);
}

\newcommand{\drawPoint}[4]{
    \node[shift={#4}, color = \pointColor] at (#2 - 0.5,#3 - 0.5) {#1};
    \draw[line width = \crossWidth, shift={#4}, color = \pointColor] (#2 - 0.25,#3) -- (#2 + 0.25,#3);
    \draw[line width = \crossWidth, shift={#4}, color = \pointColor] (#2,#3 - 0.25) -- (#2,#3 + 0.25);
}

% Tabular
\newcolumntype{C}[1]{>{\centering\arraybackslash}p{#1}}
\newcolumntype{M}[1]{>{\centering\arraybackslash}m{#1}}
\newcolumntype{K}{@{}m{0pt}@{}}

% GEOMETRY

% \newcommand{\restoregeometry}{def}

\NewDocumentCommand{\dividePage}{mm O{0.5}}{
    \pgfmathparse{1-#3}
    % \newcount\secondPage
    % \pgfmathsetcount{\secondPage}{\pgfmathresult}
    \begin{minipage}{#3\linewidth}
        #1
    \end{minipage}
    \begin{minipage}{\pgfmathresult\linewidth}
        #2
    \end{minipage}
}

\newcommand{\multiColItemize}[2]{
    \begin{multicols}{#1}
        \begin{itemize}
            #2
        \end{itemize}
    \end{multicols}
}

\makeatletter
\newcommand\pgfinvisible{\pgfsys@begininvisible}
\newcommand\pgfshown{\pgfsys@endinvisible}
\makeatother

\renewcommand*{\phantom}[1]{
    \pgfinvisible #1 \pgfshown
}

\newcounter{size}
\newcommand{\listSize}[1]{%
    \setcounter{size}{0}%
    \foreach \n in {#1}{\stepcounter{size}}%
    % \thesize
}

\newcounter{elemPos}
\newcommand{\listElement}[2]{
    \setcounter{elemPos}{0} % Start counting from 1
    \def\resultVal{0} % Default value
    \renewcommand*{\do}[1]{%
        \ifnumequal{\value{elemPos}}{#2}{%
            \def\resultVal{##1}%
            \listbreak% Break out of the loop
        }{}%
        \stepcounter{elemPos}%
    }
    % \docsvlist{#1}
    \expandafter\docsvlist\expandafter{#1} % Expand the list before passing it to \docsvlist
    \resultVal
}

\def\imgPath{}
\def\imgExtension{}

\newcommand{\imgf}[1]{\imgPath #1\imgExtension}

\NewDocumentCommand{\exoslide}{m O{10cm}}{
    \slide{}{
        \img{\imgf{#1}}[#2]
    }
}

\NewDocumentCommand{\exoList}{m O{} O{3}}{
    \slide{EXERCICES}{
        \exo{#2}{
            \vspace{-0.25cm}
            \multiColItemize{#3}{
                \foreach \q in {#1}{
                    \item \q
                }
            }
        }
    }
}

\newcommand{\questions}[1]{
    \begin{enumerate}
        \foreach \q in {#1}{
            \item \q\\
            \vspace*{-0.45cm}
            \dottedLines{3}
        }
    \end{enumerate}
}

\newcommand{\qt}[1]{«\textit{#1}»}

\newcommand{\calc}[1]{\numexpr#1\relax}
\newcommand{\ncalc}[1]{\number\calc{#1}}
\newcommand{\pcalc}[1]{\numprint{\ncalc{#1}}}