% COUNTERS
\newcounter{ex}
\stepcounter{ex}
% \newcommand\incrcount[1]{%
%     \expandafter\the\csname#1\endcsname
%     \stepcounter{#1}%
% }

% SECTIONS

\newenvironment{mysection}[1][gray!20]{%
    \begin{tcolorbox}[colback=white,
        colframe=#1!50,
        boxrule=-1pt,
        arc=0pt,
        leftrule=3pt,
        left = 8pt,
        % boxsep= -3pt,
        % show bounding box
        grow to left by= -5pt
        % fill downwards
        ]
}{%
    \end{tcolorbox}
}

\makeatletter
\newcommand{\createsection}[3]{
    \newcounter{#1}
    \stepcounter{#1}

    \expandafter\newcommand\csname #1\endcsname[2]{
        \subsection*{
            \color{#2}#3\small$_{(\csname the#1\endcsname)}$\normalsize : 
            \color{black}##1}
        \stepcounter{#1}
        \begin{mysection}[#2]
        ##2
        \end{mysection}
    }
}
\makeatother

\createsection{pr}{Blue}{Propriété}
\createsection{df}{ForestGreen}{Définition}
\createsection{thm}{Plum}{Théorème}

% \newcounter{thm}
% \stepcounter{thm}

% \newcommand{\thm}[2]{
%     % \begin{mysection}[Blue]
%     \subsection*{
%         \color{Plum}Théorème p\small$_{(\thethm)}$\normalsize : \color{black}#1
%         }
%     \stepcounter{thm}
%     % \end{mysection}
%     \begin{mysection}[Blue]
%     #2
%     \end{mysection}
% }

% \newcommand{\pr}[2]{
%     \subsection*{
%         \color{Blue}Propriété\small$_{(\thepr)}$\normalsize : 
%         \color{black}#1
%         }
%     \stepcounter{pr}
%     \begin{mysection}[Blue]
%     #2
%     \end{mysection}
% }

\newcommand{\ex}[2]{
    \subsubsection*{
        \color{Thistle}Exercice\small$_{(\theex)}$\normalsize :
        \color{black}#1
        }
    \stepcounter{ex}
    #2
}

\newcommand{\rap}[2]{
    \subsection*{Rappel : #1}
    #2
}

\newcommand{\cor}[2]{
    \subsection*{\color{NavyBlue}Corolaire : \color{black}#1}
    #2
}

\newcommand{\ctr}[2]{
    \subsection*{Contraposée : #1}
    #2
}

\newcommand{\app}[2]{
    \subsection*{\color{RedViolet}Application : \color{black}#1}
    #2
}

% \newcommand\ex[1]{\subsubsection*{\color{Thistle}Exercice : \color{black}#1}}
\newcommand{\rmq}[2]{
    \subsubsection*{\color{Thistle}Remarque : \color{black}#1}
    #2
}

\newcommand{\demo}[2]{
    \subsubsection*{\color{Red}$\rightarrow$ Démonstration: \color{black}#1} 
    #2 $\square$
}

\newcommand{\tice}[3]{
    \subsection*{\color{Orchid}TICE - \underbar{#1} : \color{black}#2}
    #3
}

% MATH

\newcommand{\modxy}[2]{\sqrt{#1^2+#2^2}}

% COMMANDS

\newcommand{\infoLecon}[3]{
    \begin{tcolorbox}[colback=red!5!white,
        colframe=red!75!black,
       ]
        \textbf{Niveau :}\vspace{-0.25cm}
        \begin{multicols}{2}
            \begin{itemize}[label=$\blacktriangleright$, font = \small \color{Red}]
                #1
            \end{itemize}
        \end{multicols}
        \tcblower
        \textbf{Prérequis :}\vspace{-0.25cm}
        \begin{multicols}{2}
            \begin{itemize}[label=$\blacktriangleright$, font = \small \color{Red}]
                #2
            \end{itemize}
        \end{multicols}
    \end{tcolorbox}
}

\newcommand{\fsize}[1]{\fontsize{#1}{#1}\selectfont}

\NewDocumentCommand{\img}{m O{\linewidth}}{%
    \begin{figure}[H]%
        \centering%
        \includegraphics[width=#2]{#1}%
    \end{figure}%
}

\NewDocumentCommand{\niv}{m g}{%
  \item \csn{#1}%
  \IfValueT{#2}{(#2)}%
}