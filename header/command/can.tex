% Définition des exercices (peut être étendue avec d'autres exercices)
% \newcommand{\can@4e@entrainement@22}{Le périmètre d'un rectangle de longueur $14,6$ et de largeur $5,5$ est :}
% \newcommand{\can@4e@entrainement@23}{Compléter le calcul par le bon nombre :
% $9 \times \dotfill = 36$}

% % Commande pour générer le tableau
% \NewDocumentCommand{\can}{>{\SplitList{,}}m}{%
%     \renewcommand{\do}[1]{\expandafter\lookupAndInsertExo\expandafter{##1}}%
%     \begin{tabular}{| M{1cm} | M{7cm} | M{5cm} |}
%     \hline
%     \docsvlist{#1} % Exécute la commande pour chaque item dans la liste
%     \end{tabular}
% }

% % Recherche et insertion du contenu de l'exercice
% \newcommand{\lookupAndInsertExo}[1]{%
%     \ifcsname can@#1\endcsname
%         \expandafter\gobbleAndInsertRow\csname can@#1\endcsname
%     \else
%         \errmessage{Exercice #1 non trouvé}
%     \fi
% }

% % Insertion de la ligne du tableau avec le contenu de l'exercice
% \newcommand{\gobbleAndInsertRow}[1]{%
%     #1 & #1 & \\
%     \hline
% }