\xdef\currentSection{null}
\xdef\currentColor{black}

\newcommand{\parseSecContent}[2]{% Color | Content
    \ifNotNull{#2}{%
        \begin{sectionBox}{#1}
            #2
        \end{sectionBox}
    }
}

\newcommand{\parseSecTitle}[3]{% Color | Title | Type
    \color{#1}%\target
    {#3}\ifthenelse{\boolean{showID}}{\ifthenelse{\boolean{parenthisedID}}
    {\small$_{(\sectionID)}$}
    { \sectionID} }{}\normalsize : 
    \color{black}#2
}

\makeatletter
\newcommand{\parseSec}[4]{% Color | Title | Content | Type
    \def\secContent{\parseSecContent{#1}{#3}}
    \def\secTitle{\parseSecTitle{#1}{#2}{#4}}
    %
    \@ifclassloaded{beamer}{
        \textbf{\secTitle}\vspace*{-0.25cm}
    }{
        \subsection*{\secTitle}\vspace*{-0.2cm}
    }
    %
    \ifthenelse{\boolean{outline}}{}{\secContent}
}
\makeatother

\newcommand{\parseSubsecContent}[2]{ % Color | Content
    #2
}

\newcommand{\parseSubsecTitle}[3]{ % Color | Title | Type
    \color{#1}%\target
    {#3}\ifthenelse{\boolean{showSubID}}{\ifthenelse{\boolean{parenthisedID}}
    {\small$_{(\sectionID)}$}
    { \sectionID} }{}\normalsize \ifNotNull{#3}{:}[\vspace{-1cm}]
    \color{black}#2
}

\makeatletter
\newcommand{\parseSubsec}[4]{ % Color | Title | Content | Type
    \def\secContent{\parseSubsecContent{#1}{#3}}
    \def\secTitle{\parseSubsecTitle{#1}{#2}{#4}}

    \ifthenelse{\boolean{outline} \and \not \boolean{subsectionInOutline}}{}{
        \@ifclassloaded{beamer}{
            \begin{subsectionBox}{#1}
                \textbf{\secTitle \ifx\relax#2\relax\else \\ \fi}
                \hspace*{-0.35cm}\secContent
            \end{subsectionBox}
        }{
            % \ifNotNull{#2}{}{\titleformat{\subsection}[runin]}
            \subsubsection*{\secTitle}\vspace*{-0.25cm}
            % \textbf{\secTitle \ifx\relax#2\relax\else \\ \fi}
            \ifthenelse{\boolean{outline}}{}{\secContent}
        }
    }
}
\makeatother

\makeatletter
\newcommand{\newsec}[3]{ % #1 Level | #2 Code | #3 Type
    \newcounter{#2}

    \expandafter\NewDocumentCommand\csname #2\endcsname{m +m o}{ % ##1 Title | ##2 Content
        
        \xdef\currentSection{#2}
        \def\currentColor{#2}
        \setItemColor{\currentColor}

        \stepcounter{#2}

        \def\sectionID{\csname the#2\endcsname}

        % % Define the label for the section
        % \def\sectionLabel{#2-\sectionID}

        % % Create the section label
        % \label{\sectionLabel}
        
        % % Define the target for hyperlink
        % \def\target{\hypertarget{\sectionLabel}}
        
        % Create Ref Hyperlink
        \ifNotNull{##1}{
            \expandafter\newcommand\csname #2-\detokenize{##1}\endcsname[1]{%
                \hyperlink{#2-\sectionID}{
                    \textbf{\color{#2!80}#3 ##1}\color{black}
                }
            }%
        }

        \begin{switch}{#1}
            \case{sec}{\parseSec{#2}{##1}{##2}{#3}}
            \default{
                \parseSubsec{#2}{##1}{##2}{#3}
            }
        \end{switch}

        % Reference
        \ifNotNull{##3}{ 
            \ifthenelse{\boolean{outline} \or \not \boolean{showRef}}{}{
                \vspace{-0.5cm}
                \begin{flushright}
                \small\textit{##3 $\leftarrow$}\hspace*{0.5cm}
                \end{flushright}
            }
        }
    }
}
\makeatother

\newcommand{\refsec}[2]{%
    \expandafter\csname #1-#2\endcsname{#2}%
}