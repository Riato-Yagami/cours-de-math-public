\xdef\currentSection{null}
\xdef\currentColor{black}

\newcommand{\parseSecContent}[2]{% Color | Content
    \ifNotNull{#2}{%
        \begin{sectionBox}{#1}
            #2
        \end{sectionBox}
    }
}

\newcommand{\parseSecTitle}[3]{% Color | Title | Type
    \color{#1}\target
    {#3}\ifthenelse{\boolean{showID}}{\ifthenelse{\boolean{parenthisedID}}
    {\small$_{(\sectionID)}$}
    { \sectionID} }{}\normalsize : 
    \color{black}#2
}

\makeatletter
\newcommand{\parseSec}[4]{% Color | Title | Content | Type
    \def\secContent{\parseSecContent{#1}{#3}}
    \def\secTitle{\parseSecTitle{#1}{#2}{#4}}
    %
    \@ifclassloaded{beamer}{
        \textbf{\secTitle}\vspace*{-0.25cm}
    }{
        \subsection*{\secTitle}
    }
    %
    \ifthenelse{\boolean{outline}}{}{\secContent}
}
\makeatother

\newcommand{\parseSubsecContent}[2]{ % Color | Content
    #2
}

\newcommand{\parseSubsecTitle}[3]{ % Color | Title | Type
    \color{#1}\target
    {#3}\ifthenelse{\boolean{showSubID}}{\ifthenelse{\boolean{parenthisedID}}
    {\small$_{(\sectionID)}$}
    { \sectionID} }{}\normalsize \ifNotNull{#3}{:}[\vspace{-1cm}]
    \color{black}#2
}

\makeatletter
\newcommand{\parseSubsec}[4]{ % Color | Title | Content | Type
    \def\secContent{\parseSubsecContent{#1}{#3}}
    \def\secTitle{\parseSubsecTitle{#1}{#2}{#4}}

    \ifthenelse{\boolean{outline} \and \not \boolean{subsectionInOutline}}{}{
        \@ifclassloaded{beamer}{
            \begin{subsectionBox}{#1}
                \textbf{\secTitle \ifx\relax#2\relax\else \\ \fi}
                \hspace*{-0.35cm}\secContent
            \end{subsectionBox}
        }{
            \subsubsection*{\secTitle}
            \ifthenelse{\boolean{outline}}{}{\secContent}
        }
    }
}
\makeatother

\makeatletter
\newcommand{\newsec}[4]{ % #1 Level | #2 Code | #3 Color | #4 Type
    \newcounter{#2}

    \expandafter\NewDocumentCommand\csname #2\endcsname{m +m o}{ % ##1 Title | ##2 Content
        
        \xdef\currentSection{#2}
        \xdef\currentColor{#3}
        \setItemColor{\currentColor}

        \stepcounter{#2}

        \def\sectionID{\csname the#2\endcsname}

        % Define the label for the section
        \def\sectionLabel{#2-\sectionID}

        % Create the section label
        \label{\sectionLabel}
        
        % Define the target for hyperlink
        \def\target{\hypertarget{\sectionLabel}}
        
        % Create Ref Hyperlink
        \ifNotNull{##1}{
            \expandafter\newcommand\csname #2-##1\endcsname[1]{%
                \hyperlink{#2-\sectionID}{
                    \textbf{\color{#3!80}#4 ##1}\color{black}
                }
            }%
        }

        \begin{switch}{#1}
            \case{sec}{\parseSec{#3}{##1}{##2}{#4}}
            \default{
                \parseSubsec{#3}{##1}{##2}{#4}
            }
        \end{switch}

        % Reference
        \ifNotNull{##3}{ 
            \ifthenelse{\boolean{outline} \or \not \boolean{showRef}}{}{
                \vspace{-0.5cm}
                \begin{flushright}
                \small\textit{##3 $\leftarrow$}\hspace*{0.5cm}
                \end{flushright}
            }
        }
    \setItemColor{gradeColor}
    }
}
\makeatother

\newcommand{\refsec}[2]{%
    \expandafter\csname #1-#2\endcsname{#2}%
}

\newsec{sec}{pr}{Red}{Propriété} % \PropertyColor
\newsec{sec}{df}{\DefinitionColor}{Définition}
\newsec{sec}{thm}{Maroon}{Théorème} % \TheoremColor
\newsec{sec}{scnSUB}{OrangeRed}{Séance}

\newsec{sub}{cor}{Red}{Corollaire} % NavyBlue
\newsec{sub}{lem}{Red}{Lemme} % NavyBlue
\newsec{sub}{ctr}{Red}{Contraposée} % NavyBlue
\newsec{sub}{rmdr}{BrickRed}{Rappel} % Thistle
\newsec{sub}{mthd}{OrangeRed}{Méthode}
\newsec{sub}{vc}{ForestGreen}{Vocabulaire}
\newsec{sub}{nt}{SeaGreen}{Notation}
\newsec{sub}{hist}{Black}{Histoire}
\newsec{sub}{axio}{Mahogany}{Axiome}
\newsec{sub}{objSUB}{gradeColor}{Objectifs}

\newsec{subsub}{rmk}{BrickRed}{Remarque} % Thistle
\newsec{subsub}{expl}{gray}{Exemple}
\newsec{subsub}{app}{NavyBlue}{Application} %RedViolet
\newsec{subsub}{exo}{Gray}{Exercice}
\newsec{subsub}{corr}{OrangeRed}{Correction}
\newsec{subsub}{act}{NavyBlue}{Activité} % BurntOrange
\newsec{subsub}{dm}{Gray}{Devoir maison}

\newsec{subsub}{nullsubsec}{Gray}{}

\newsec{subsub}{demoSUB}{Red}{$\rightarrow$ Démonstration}
\newsec{subsub}{ticeSUB}{Orchid}{TICE}
\newsec{subsub}{qfSUB}{Red}{\Large Questions flash}
\newsec{sub}{rl}{OrangeRed}{Règle n\deg\therl}

\NewDocumentCommand{\demo}{mmo}{
    \ifthenelse{\boolean{demonstration} \and \not \boolean{outline}}{
        \demoSUB{#1}{
            \ifNotNull{#2}{
                \noindent #2 $\square$
            }
        }[#3]
    }{
    }
}

\newcommand*{\obj}[1]{
    \begin{objBox}
        \objSUB{}{%
        \begin{enumerate}#1\end{enumerate}%
        }
    \end{objBox}
}

\NewDocumentCommand{\scn}{m O{}}{
    \begin{sectionBox}{OrangeRed}
        \scnSUB{#1}{#2}%
        \vspace*{-0.25cm}
    \end{sectionBox}
}

% \newcommand*{\scn}[2]{%
%     \begin{mysection}[OrangeRed]
%         \scnSUB{#1}{#2}%
%         \vspace*{-0.25cm}
%     \end{mysection}
%     %
%         % \ifNotNull{#2}{%
%         %     \begin{itemize}[wide = 0pt, label=]%
%         %         #2%
%         %     \end{itemize}%
%         % }%
%     %}%
% }

\newcommand{\scni}[2]{
    \item \color{OrangeRed}#1 : \color{black} #2
}

\NewDocumentCommand{\tice}{mmmo}{
    \ticeSUB{#1 - #2}{#3}[#4]
}

\newcommand{\hint}[1]{
    \ifthenelse{\boolean{outline}}{}{%
        {\color{Black!70}\textit{Indication: #1}}%
    }
}

\newcounter{qf}
\newcounter{choice}
\newcommand*{\choice}[1]{%
    {\scalefont{0.75}\color{Blue}\Alph{choice}.\color{black}}%
    #1%
    \stepcounter{choice}%
}

\newcommand*{\choicea}[1]{\setcounter{choice}{#1}%
    {\scalefont{0.75}\color{Blue}\Alph{choice}.\color{black}}%
}

\newcounter{qfs}
\def\qfs{\stepcounter{qfs}\color{Blue}\alph{qfs}) \color{black}}
\def\sqf{\setcounter{qfs}{0} \stepcounter{qf}\color{blue}\arabic{qf}. \color{black}}

\newcommand{\qfSlide}[1]{
    \slide{qf}{
        \setcounter{qf}{0}
        \setcounter{qfs}{0}
        #1
    }
}
% QUESTIONS FLASH

% \NewDocumentCommand{\qfb}{m O{15}}{
%     % Initialize an empty token list to accumulate the formatted questions and answers
%     \def\formattedList{}
%     % Iterate over the input list
%     \foreach \q in {#1}{
%         % Extract the question and answer parts
%         \xdef\formattedList{\formattedList{${\listElement{\q}{0}} = $,${\listElement{\q}{1}}$},}
%     }
%     % Remove the last comma
%     \edef\formattedList{\unexpanded\expandafter{\formattedList}}
%     % Pass the formatted list to \qf
%     \expandafter\qf\expandafter{\formattedList}[#2]
% }

\newcommand{\qfRes}[1]{
    \begin{enumerate}
        \foreach \q in {#1}{\setcounter{choice}{1}%
            \item \listElement{\q}{0}\palt{2}{\listElement{\q}{1}}
        }
    \end{enumerate}
}

\setul{0.4ex}{0.15ex}
\newcommand*{\key}[1]{\setulcolor{\currentColor}\ul{\textbf{#1}}}
% \newcommand*{\key}[1]{\sethlcolor{\currentColor!25}\hl{\textbf{#1}}}