\newcommand{\parseSecContent}[2]{% Color | Content
    \ifNotNull{#2}{%
        \begin{mysection}[#1]
            #2
        \end{mysection}
    }
}

\newcommand{\parseSecTitle}[3]{% Color | Title | Type
    \color{#1}\target
    {#3}\ifthenelse{\boolean{showID}}{\ifthenelse{\boolean{parenthisedID}}
    {\small$_{(\sectionID)}$}
    { \sectionID}}{}\normalsize : 
    \color{black}#2
}

\makeatletter
\newcommand{\parseSec}[4]{% Color | Title | Content | Type
    \def\secContent{\parseSecContent{#1}{#3}}
    \def\secTitle{\parseSecTitle{#1}{#2}{#4}}

    \@ifclassloaded{beamer}{
        \textbf{\secTitle}
    }{
        \subsection*{\secTitle}
    }

    \ifthenelse{\boolean{outline}}{}{\secContent}
}
\makeatother

\newcommand{\parseSubsecContent}[2]{ % Color | Content
    #2
}

\newcommand{\parseSubsecTitle}[3]{ % Color | Title | Type
    \color{#1}\target
    {#3}\normalsize : 
    \color{black}#2
}

\makeatletter
\newcommand{\parseSubsec}[4]{ % Color | Title | Content | Type
    \def\secContent{\parseSubsecContent{#1}{#3}}
    \def\secTitle{\parseSubsecTitle{#1}{#2}{#4}}

    \ifthenelse{\boolean{outline} \and \not \boolean{subsectionInOutline}}{}{
        \@ifclassloaded{beamer}{
            \begin{mysubsection}[#1]
                \textbf{\secTitle \ifx\relax#2\relax\else \\ \fi}
                \hspace*{-0.35cm}\secContent
            \end{mysubsection}
        }{
            \subsubsection*{\secTitle}
            \ifthenelse{\boolean{outline}}{}{\secContent}
        }
    }
}
\makeatother

\makeatletter
\newcommand{\newsec}[4]{ % #1 Level | #2 Code | #3 Color | #4 Type
    \newcounter{#2}

    \expandafter\NewDocumentCommand\csname #2\endcsname{mmo}{ % ##1 Title | ##2 Content
        \stepcounter{#2}

        \def\sectionID{\csname the#2\endcsname}

        % Define the label for the section
        \def\sectionLabel{#2-\sectionID}

        % Create the section label
        \label{\sectionLabel}
        
        % Define the target for hyperlink
        \def\target{\hypertarget{\sectionLabel}}
        
        % Create Ref Hyperlink
        \ifNotNull{##1}{
            \expandafter\newcommand\csname #2-##1\endcsname[1]{%
                \hyperlink{#2-\sectionID}{
                    \textbf{\color{#3!80}#4 ##1}\color{black}
                }
            }%
        }

        \begin{switch}{#1}
            \case{sec}{\parseSec{#3}{##1}{##2}{#4}}
            \default{
                \parseSubsec{#3}{##1}{##2}{#4}
            }
        \end{switch}

        % Reference
        \ifNotNull{##3}{ 
            \ifthenelse{\boolean{outline} \or \not \boolean{showRef}}{}{
                \vspace{-0.5cm}
                \begin{flushright}
                \small\textit{##3 $\leftarrow$}\hspace*{0.5cm}
                \end{flushright}
            }
        }
    }
}
\makeatother

\newcommand{\refsec}[2]{%
    \expandafter\csname #1-#2\endcsname{#2}%
}

\newsec{sec}{pr}{\PropertyColor}{Propriété}
\newsec{sec}{df}{\DefinitionColor}{Définition}
\newsec{sec}{thm}{\TheoremColor}{Théorème}
\newsec{sec}{scnSUB}{OrangeRed}{Séance}

\newsec{sub}{cor}{NavyBlue}{Corollaire}
\newsec{sub}{lem}{NavyBlue}{Lemme}
\newsec{sub}{ctr}{NavyBlue}{Contraposée}
\newsec{sub}{rmdr}{Thistle}{Rappel}
\newsec{sub}{mthd}{OrangeRed}{Méthode}

\newsec{subsub}{rmk}{Thistle}{Remarque}
\newsec{subsub}{expl}{gray}{Exemple}
\newsec{subsub}{app}{RedViolet}{Application}
\newsec{subsub}{exo}{Gray}{Exercice}
\newsec{subsub}{act}{BurntOrange}{Activité}
\newsec{subsub}{dm}{Gray}{Devoir maison}

\newsec{subsub}{demoSUB}{Red}{$\rightarrow$ Démonstration}
\newsec{subsub}{ticeSUB}{Orchid}{TICE}
\newsec{subsub}{qfSUB}{Red}{\Large Questions flash}

\NewDocumentCommand{\demo}{mmo}{
    \ifthenelse{\boolean{demonstration} \and \not \boolean{outline}}{
        \demoSUB{#1}{
            \ifNotNull{#2}{
                \noindent #2 $\square$
            }
        }[#3]
    }{
    }
}

\newcommand*{\scn}[2]{%
    \scnSUB{#1}{%
        \ifNotNull{#2}{%
            \begin{itemize}[wide = 0pt, label=]%
                #2%
            \end{itemize}%
        }%
    }%
}

\newcommand{\scni}[2]{
    \item \color{OrangeRed}#1 : \color{black} #2
}

\NewDocumentCommand{\tice}{mmmo}{
    \ticeSUB{#1 - #2}{#3}[#4]
}

\newcommand{\hint}[1]{
    \textit{Indication: #1}
}

% QUESTIONS FLASHS

% \NewDocumentCommand{\qfb}{m O{15}}{
%     % Initialize an empty token list to accumulate the formatted questions and answers
%     \def\formattedList{}
%     % Iterate over the input list
%     \foreach \q in {#1}{
%         % Extract the question and answer parts
%         \xdef\formattedList{\formattedList{${\listElement{\q}{0}} = $,${\listElement{\q}{1}}$},}
%     }
%     % Remove the last comma
%     \edef\formattedList{\unexpanded\expandafter{\formattedList}}
%     % Pass the formatted list to \qf
%     \expandafter\qf\expandafter{\formattedList}[#2]
% }

\newcommand{\qfRes}[1]{
    \begin{enumerate}
        \foreach \q in {#1}{
            \item \listElement{\q}{0}\palt{2}{\listElement{\q}{1}}
        }
    \end{enumerate}
}