\NewDocumentCommand{\ctikz}{O{1} m}{
    \begin{center}
        \begin{tikzpicture}[scale = #1]
            #2
        \end{tikzpicture}
    \end{center}
}

\newcommand{\axis}[1]{%Draw coordinate axes
    \draw[thin, -Stealth] (-0.5,0) -- (#1,0);% node[right] {$x$}; % x-axis
    \draw[thin, -Stealth] (0,-0.5) -- (0,#1);% node[above] {$y$}; % y-axis
}

\newcommand{\drawGrid}[3]{
    \foreach \n in {0,...,#1}
        \draw[line width = #3] (\n,0) -- (\n,#2);
    \foreach \n in {0,...,#2}
        \draw[line width = #3] (0,\n) -- (#1,\n);
}

\def\pointColor{blue}
\def\crossWidth{0.15mm}
\def\crossSize{0.2}
\def\nodeShift{0.4}
\NewDocumentCommand{\drawPoint}{mmm O{\pointColor} O{(0,0)}}{%
    \pgfgettransformentries{\aa}{\ab}{\ba}{\bb}{\xshift}{\yshift}
    \node[shift={#5}, color = #4] at (#2 - \nodeShift / \aa,#3 - \nodeShift / \bb) {#1};
    \draw[line width = \crossWidth, shift={#5}
    , color = #4] (#2 - \crossSize / \aa,#3) -- (#2 + \crossSize / \aa,#3);
    \draw[line width = \crossWidth, shift={#5}, color = #4] (#2,#3 - \crossSize / \bb) -- (#2,#3 + \crossSize / \bb);
}

% \NewDocumentCommand{\drawPoint}{mmm O{\pointColor} O{(0,0)}}{
%     \pgftransformscale{1}
%     \pgftransformxscale{1/\pgfkeysvalueof{/pgf/scale x}}
%     \pgftransformyscale{1/\pgfkeysvalueof{/pgf/scale y}}
%     \node[shift={#5}, color = #4] at (#2 - \nodeShift,#3 - \nodeShift) {#1};
%     \draw[line width = \crossWidth, shift={#5}, color = #4] 
%         (#2 - \crossSize,#3) -- (#2 + \crossSize,#3);
%     \draw[line width = \crossWidth, shift={#5}, color = #4] 
%         (#2,#3 -\crossSize) -- (#2,#3 + \crossSize);
%     \pgftransformreset
% }


\NewDocumentCommand{\drawTick}{m O{} O{gradeColor}}{
    \draw[line width = 0.75mm, #3] (#1, 0.08) -- (#1, -0.08);
    \draw[line width = 0.75mm, #3] (#1 - 0.08, 0) -- (#1 + 0.08, 0);
    \node[#3] at (#1, -0.3) {#2};
}

\usetikzlibrary{calc,decorations,patterns,arrows,decorations.pathmorphing}

\pgfdeclaredecoration{penciline}{initial}{
    \state{initial}[width=+\pgfdecoratedinputsegmentremainingdistance,
        auto corner on length=1pt,
    ]{
        \ifthenelse
            {\pgfkeysvalueof{/tikz/penciline/jag ratio}=0} {
            \pgfpathcurveto%
                {% 1st control point
                    \pgfpoint
                        {\pgfdecoratedinputsegmentremainingdistance/2}
                        {2*rnd*\pgfdecorationsegmentamplitude}
                }
                {%% 2nd control point
                    \pgfpoint
                    %% Make sure random number is always between origin and target points
                        {\pgfdecoratedinputsegmentremainingdistance/2}
                        {2*rnd*\pgfdecorationsegmentamplitude}
                }
                {% 2nd point (1st one is implicit)
                    \pgfpointadd
                        {\pgfpointdecoratedinputsegmentlast}
                        {\pgfpoint{0*rand*1pt}{0*rand*1pt}}
                }          
            } {
            \pgfpathcurveto%
                {% 1st control point
                    \pgfpoint
                        {\pgfdecoratedinputsegmentremainingdistance*rnd*1pt}
                        {\pgfkeysvalueof{/tikz/penciline/jag ratio}*
                        rand*\pgfdecorationsegmentamplitude}
                }
                {%% 2nd control point
                    \pgfpoint
                    %% Make sure random number is always between origin and target points
                        {(.5+0.25*rand)*\pgfdecoratedinputsegmentremainingdistance}
                        {\pgfkeysvalueof{/tikz/penciline/jag ratio}*
                        rand*\pgfdecorationsegmentamplitude}
                }
                {% 2nd point (1st one is implicit)
                    \pgfpointadd
                        {\pgfpointdecoratedinputsegmentlast}
                        {\pgfpoint{rand*1pt}{rand*1pt}}
                }
            }
    }
    \state{final}{}
}

\tikzset{
    penciline/.code={\pgfqkeys{/tikz/penciline}{#1}},
    penciline={
        jag ratio/.initial=2,
        decoration/.initial = penciline,
    },
    penciline/.style = {
        decorate,
        %%decoration={\pgfkeysvalueof{/tikz/penciline/decoration}},
        penciline/.cd,
        #1,
        /tikz/.cd,
    },
    decorate,
    decoration={\pgfkeysvalueof{/tikz/penciline/decoration}},
}

\tikzset{
    penthick/.style={
        penciline, % Ajuster selon vos préférences
        thick                        % Épaisseur du trait
    }
}


\NewDocumentCommand{\cir}{O{black} m}{
    \tikz[baseline=(X.base)] 
    \node (X) [draw, shape=circle, inner sep=-1pt, color = #1] {\small\strut #2};
}

% ISOMETRIC %

\NewDocumentCommand{\isometric}{O{1} O{0.5pt} O{1} m}{
    \begin{tikzpicture}[scale = #1,
        x={(0.86cm,0.5cm)},
        y={(-0.86cm,0.5cm)}]
        \begin{scope}
            \clip (0,#3*25.5) rectangle (#3*37.5,#3*29);
            \foreach \x in {0,...,50}
            \foreach \y in {0,...,50}
            {
                \fill (\x,\y) circle (#2);
            }
            #4
        \end{scope} 
        \draw [gray] (-0.5,#3*25.5 - 0.5) rectangle (#3*37.5 + 1,#3*29 + 0.5);
    \end{tikzpicture}
}

\newcommand{\sIso}[1]{
    \begin{center}
        \isometric[0.75][1pt][0.25]{#1}
    \end{center}
}

\newcommand{\mIso}[1]{
    \begin{center}
        \isometric[0.75][1pt][0.5]{#1}
    \end{center}
}

% \NewDocumentCommand{\dotGrid}{O{(0,0)} O{(2,2)} O{0.5} O{0.5} O{0.5pt}}{%
%     % #1: Bottom-left corner of the grid (default: (0,0))
%     % #2: Top-right corner of the grid (default: (0,5))
%     % #3: Step size for x (default: 0.5)
%     % #4: Step size for y (default: 0.5)
%     % #5: Radius of points (default: 0.5pt)
%     % \foreach \x in {#1x,#1x+#3,...,#2x} {
%     %     \foreach \y in {#1y,#1y+#4,...,#2y} {
%     %         \fill (\x,\y) circle (#5); % Draw points
%     %     }
%     % }
%     \draw[gray!40] #1 rectangle #2; % Draw rectangle
% }

\NewDocumentCommand{\dotGrid}{O{0} O{0} O{5} O{5} O{1} O{0.5pt}}{%
    \pgfmathsetmacro{\xStart}{#1}
    \pgfmathsetmacro{\xEnd}{#3}
    \pgfmathsetmacro{\yStart}{#2}
    \pgfmathsetmacro{\yEnd}{#4}
    \pgfmathsetmacro{\step}{#5}

    \foreach \x in {1,...,100} { % Arbitrary large range
        \pgfmathsetmacro{\currentX}{\xStart + (\x-1)*\step} % Calculate current x
        \ifdim \currentX pt > \xEnd pt \breakforeach \fi % Stop if past end
        \foreach \y in {1,...,100} { % Arbitrary large range
            \pgfmathsetmacro{\currentY}{\yStart + (\y-1)*\step} % Calculate current y
            \ifdim \currentY pt > \yEnd pt \breakforeach \fi % Stop if past end
            \fill (\currentX,\currentY) circle (#6); % Draw point
        }
    }
}
