\NewDocumentCommand{\ctikz}{O{1} m}{
    \begin{center}
        \begin{tikzpicture}[scale = #1]
            #2
        \end{tikzpicture}
    \end{center}
}

\newcommand{\axis}[1]{%Draw coordinate axes
    \draw[thin, -Stealth] (-0.5,0) -- (#1,0);% node[right] {$x$}; % x-axis
    \draw[thin, -Stealth] (0,-0.5) -- (0,#1);% node[above] {$y$}; % y-axis
}

\newcommand{\drawGrid}[3]{
    \foreach \n in {0,...,#1}
        \draw[line width = #3] (\n,0) -- (\n,#2);
    \foreach \n in {0,...,#2}
        \draw[line width = #3] (0,\n) -- (#1,\n);
}

\def\pointColor{blue}
\def\crossWidth{0.15mm}
\def\crossSize{0.2}
\def\nodeShift{0.4}
\NewDocumentCommand{\drawPoint}{mmm O{\pointColor} O{(0,0)}}{%
    \pgfgettransformentries{\aa}{\ab}{\ba}{\bb}{\xshift}{\yshift}
    \node[shift={#5}, color = #4] at (#2 - \nodeShift / \aa,#3 - \nodeShift / \bb) {#1};
    \draw[line width = \crossWidth, shift={#5}
    , color = #4] (#2 - \crossSize / \aa,#3) -- (#2 + \crossSize / \aa,#3);
    \draw[line width = \crossWidth, shift={#5}, color = #4] (#2,#3 - \crossSize / \bb) -- (#2,#3 + \crossSize / \bb);
}

% \NewDocumentCommand{\drawPoint}{mmm O{\pointColor} O{(0,0)}}{
%     \pgftransformscale{1}
%     \pgftransformxscale{1/\pgfkeysvalueof{/pgf/scale x}}
%     \pgftransformyscale{1/\pgfkeysvalueof{/pgf/scale y}}
%     \node[shift={#5}, color = #4] at (#2 - \nodeShift,#3 - \nodeShift) {#1};
%     \draw[line width = \crossWidth, shift={#5}, color = #4] 
%         (#2 - \crossSize,#3) -- (#2 + \crossSize,#3);
%     \draw[line width = \crossWidth, shift={#5}, color = #4] 
%         (#2,#3 -\crossSize) -- (#2,#3 + \crossSize);
%     \pgftransformreset
% }


\NewDocumentCommand{\drawTick}{m O{} O{gradeColor}}{
    \draw[line width = 0.75mm, #3] (#1, 0.08) -- (#1, -0.08);
    \draw[line width = 0.75mm, #3] (#1 - 0.08, 0) -- (#1 + 0.08, 0);
    \node[#3] at (#1, -0.3) {#2};
}

\usetikzlibrary{calc,decorations,patterns,arrows,decorations.pathmorphing}

\pgfdeclaredecoration{penciline}{initial}{
    \state{initial}[width=+\pgfdecoratedinputsegmentremainingdistance,
        auto corner on length=1pt,
    ]{
        \ifthenelse
            {\pgfkeysvalueof{/tikz/penciline/jag ratio}=0} {
            \pgfpathcurveto%
                {% 1st control point
                    \pgfpoint
                        {\pgfdecoratedinputsegmentremainingdistance/2}
                        {2*rnd*\pgfdecorationsegmentamplitude}
                }
                {%% 2nd control point
                    \pgfpoint
                    %% Make sure random number is always between origin and target points
                        {\pgfdecoratedinputsegmentremainingdistance/2}
                        {2*rnd*\pgfdecorationsegmentamplitude}
                }
                {% 2nd point (1st one is implicit)
                    \pgfpointadd
                        {\pgfpointdecoratedinputsegmentlast}
                        {\pgfpoint{0*rand*1pt}{0*rand*1pt}}
                }          
            } {
            \pgfpathcurveto%
                {% 1st control point
                    \pgfpoint
                        {\pgfdecoratedinputsegmentremainingdistance*rnd*1pt}
                        {\pgfkeysvalueof{/tikz/penciline/jag ratio}*
                        rand*\pgfdecorationsegmentamplitude}
                }
                {%% 2nd control point
                    \pgfpoint
                    %% Make sure random number is always between origin and target points
                        {(.5+0.25*rand)*\pgfdecoratedinputsegmentremainingdistance}
                        {\pgfkeysvalueof{/tikz/penciline/jag ratio}*
                        rand*\pgfdecorationsegmentamplitude}
                }
                {% 2nd point (1st one is implicit)
                    \pgfpointadd
                        {\pgfpointdecoratedinputsegmentlast}
                        {\pgfpoint{rand*1pt}{rand*1pt}}
                }
            }
    }
    \state{final}{}
}

\tikzset{
    penciline/.code={\pgfqkeys{/tikz/penciline}{#1}},
    penciline={
        jag ratio/.initial=4,
        decoration/.initial = penciline,
    },
    penciline/.style = {
        decorate,
        %%decoration={\pgfkeysvalueof{/tikz/penciline/decoration}},
        penciline/.cd,
        #1,
        /tikz/.cd,
    },
    decorate,
    decoration={\pgfkeysvalueof{/tikz/penciline/decoration}},
}