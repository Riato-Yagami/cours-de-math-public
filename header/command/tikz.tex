\def\figScale{1}

\NewDocumentCommand{\ctikz}{O{\figScale} m}{%
    \begin{center}%
        \begin{tikzpicture}[scale = #1]%
            #2%
        \end{tikzpicture}%
    \end{center}%
}%

\newcommand{\axis}[1]{%Draw coordinate axes
    \draw[thin, -Stealth] (-0.5,0) -- (#1,0);% node[right] {$x$}; % x-axis
    \draw[thin, -Stealth] (0,-0.5) -- (0,#1);% node[above] {$y$}; % y-axis
}

\newcommand{\drawGrid}[3]{
    \foreach \n in {0,...,#1}
        \draw[line width = #3] (\n,0) -- (\n,#2);
    \foreach \n in {0,...,#2}
        \draw[line width = #3] (0,\n) -- (#1,\n);
}

\def\pointColor{blue}
\def\crossWidth{0.15mm}
\def\crossSize{0.2}
\def\nodeShift{0.4}
\NewDocumentCommand{\drawPoint}{mmm O{\pointColor} O{(0,0)}}{%
    \pgfgettransformentries{\aa}{\ab}{\ba}{\bb}{\xshift}{\yshift}
    \node[shift={#5}, color = #4] at (#2 - \nodeShift / \aa,#3 - \nodeShift / \bb) {#1};
    \draw[line width = \crossWidth, shift={#5}
    , color = #4] (#2 - \crossSize / \aa,#3) -- (#2 + \crossSize / \aa,#3);
    \draw[line width = \crossWidth, shift={#5}, color = #4] (#2,#3 - \crossSize / \bb) -- (#2,#3 + \crossSize / \bb);
}

% \NewDocumentCommand{\drawPoint}{mmm O{\pointColor} O{(0,0)}}{
%     \pgftransformscale{1}
%     \pgftransformxscale{1/\pgfkeysvalueof{/pgf/scale x}}
%     \pgftransformyscale{1/\pgfkeysvalueof{/pgf/scale y}}
%     \node[shift={#5}, color = #4] at (#2 - \nodeShift,#3 - \nodeShift) {#1};
%     \draw[line width = \crossWidth, shift={#5}, color = #4] 
%         (#2 - \crossSize,#3) -- (#2 + \crossSize,#3);
%     \draw[line width = \crossWidth, shift={#5}, color = #4] 
%         (#2,#3 -\crossSize) -- (#2,#3 + \crossSize);
%     \pgftransformreset
% }


\NewDocumentCommand{\drawTick}{m O{} O{gradeColor}}{
    \draw[line width = 0.75mm, #3] (#1, 0.08) -- (#1, -0.08);
    \draw[line width = 0.75mm, #3] (#1 - 0.08, 0) -- (#1 + 0.08, 0);
    \node[#3] at (#1, -0.3) {#2};
}

\usetikzlibrary{calc,decorations,patterns,arrows,decorations.pathmorphing}

\pgfdeclaredecoration{penciline}{initial}{
    \state{initial}[width=+\pgfdecoratedinputsegmentremainingdistance,
        auto corner on length=1pt,
    ]{
        \ifthenelse
            {\pgfkeysvalueof{/tikz/penciline/jag ratio}=0} {
            \pgfpathcurveto%
                {% 1st control point
                    \pgfpoint
                        {\pgfdecoratedinputsegmentremainingdistance/2}
                        {2*rnd*\pgfdecorationsegmentamplitude}
                }
                {%% 2nd control point
                    \pgfpoint
                    %% Make sure random number is always between origin and target points
                        {\pgfdecoratedinputsegmentremainingdistance/2}
                        {2*rnd*\pgfdecorationsegmentamplitude}
                }
                {% 2nd point (1st one is implicit)
                    \pgfpointadd
                        {\pgfpointdecoratedinputsegmentlast}
                        {\pgfpoint{0*rand*1pt}{0*rand*1pt}}
                }          
            } {
            \pgfpathcurveto%
                {% 1st control point
                    \pgfpoint
                        {\pgfdecoratedinputsegmentremainingdistance*rnd*1pt}
                        {\pgfkeysvalueof{/tikz/penciline/jag ratio}*
                        rand*\pgfdecorationsegmentamplitude}
                }
                {%% 2nd control point
                    \pgfpoint
                    %% Make sure random number is always between origin and target points
                        {(.5+0.25*rand)*\pgfdecoratedinputsegmentremainingdistance}
                        {\pgfkeysvalueof{/tikz/penciline/jag ratio}*
                        rand*\pgfdecorationsegmentamplitude}
                }
                {% 2nd point (1st one is implicit)
                    \pgfpointadd
                        {\pgfpointdecoratedinputsegmentlast}
                        {\pgfpoint{rand*1pt}{rand*1pt}}
                }
            }
    }
    \state{final}{}
}

\tikzset{
    penciline/.code={\pgfqkeys{/tikz/penciline}{#1}},
    penciline={
        jag ratio/.initial=2,
        decoration/.initial = penciline,
    },
    penciline/.style = {
        decorate,
        %%decoration={\pgfkeysvalueof{/tikz/penciline/decoration}},
        penciline/.cd,
        #1,
        /tikz/.cd,
    },
    decorate,
    decoration={\pgfkeysvalueof{/tikz/penciline/decoration}},
}

\tikzset{
    penthick/.style={
        penciline, % Ajuster selon vos préférences
        thick                        % Épaisseur du trait
    }
}


\NewDocumentCommand{\cir}{O{black} m}{
    \tikz[baseline=(X.base)] 
    \node (X) [draw, shape=circle, inner sep=-1pt, color = #1] {\small\strut #2};
}

% ISOMETRIC %

\NewDocumentCommand{\isometric}{O{1} O{0.5pt} O{1} m}{
    \begin{tikzpicture}[scale = #1,
        x={(0.86cm,0.5cm)},
        y={(-0.86cm,0.5cm)}]
        \begin{scope}
            \clip (0,#3*25.5) rectangle (#3*37.5,#3*29);
            \foreach \x in {0,...,50}
            \foreach \y in {0,...,50}
            {
                \fill (\x,\y) circle (#2);
            }
            #4
        \end{scope} 
        \draw [gray] (-0.5,#3*25.5 - 0.5) rectangle (#3*37.5 + 1,#3*29 + 0.5);
    \end{tikzpicture}
}

\newcommand{\sIso}[1]{
    \begin{center}
        \isometric[0.75][1pt][0.25]{#1}
    \end{center}
}

\newcommand{\mIso}[1]{
    \begin{center}
        \isometric[0.75][1pt][0.5]{#1}
    \end{center}
}


% \NewDocumentCommand{\dotGrid}{O{5} O{5} O{0.5pt} O{0.5} O{(0,0)}}{
%     \foreach \x in {0,#4,...,#1} {
%         \foreach \y in {0,#4,...,#2} {
%             \fill[shift={#5}] (\x,\y) \bigDot{#3} % Draw points
%         }
%     }
%     \draw[gray!40, shift={#5}] (-0.5,-0.5) rectangle (#1+0.5,#2+0.5);
% }

\def\isoSpace{0.8660254037844386}

\newcommand{\smallDot}[1]{circle (#1/1.75);}
\newcommand{\bigDot}[1]{circle (#1);}
\newcommand{\medDot}[1]{circle (#1/1.45);}

\NewDocumentCommand{\boundingBox}{O{5} O{5} O{0.5pt} O{0.5} O{(0,0)} O{plain} O{gray!40} O{black}}{
    \ifstrequal{#6}{plain}{% Only a rectangle
    \draw[#7, shift={#5}] (-0.5,-0.5) rectangle (#1+0.5,#2+0.5);
    }{}
    \ifstrequal{#6}{grid}{% Grid
    \draw[#7, shift={#5}] (-0.5,-0.5) rectangle (#1+0.5,#2+0.5);
    \draw[#7, shift={#5}] (-0.5,-0.5) grid (#1+0.5,#2+0.5);
    }{}
    \ifstrequal{#6}{dot}{% Dot Grid
        \foreach \x in {0,#4,...,#1} {
            \foreach \y in {0,#4,...,#2} {
                \fill[shift={#5}] (\x,\y) \bigDot{#3} % Draw points
            }
        }
        \draw[#7, shift={#5}] (-0.5,-0.5) rectangle (#1+0.5,#2+0.5);
    }{}
    \ifstrequal{#6}{sdot}{% Dot with sub grid
    \foreach \x [count=\xi from 0] in {0,#4,...,#1} {
        \foreach \y [count=\yi from 0] in {0,#4,...,#2} {
            \ifodd\xi \ifodd\yi
                \fill[shift={#5}] (\x,\y) \medDot{#3};
            \else
                \fill[shift={#5}] (\x,\y) \smallDot{#3}
            \fi\else \ifodd\yi
                \fill[shift={#5}] (\x,\y) \smallDot{#3}
            \else
                \fill[shift={#5}] (\x,\y) \bigDot{#3}
            \fi\fi
        }
        \draw[#7, shift={#5}] (-0.5,-0.5) rectangle (#1+0.5,#2+0.5);
    }
    }{}
    \ifstrequal{#6}{cavalier}{% Dot with sub grid
        \foreach \x [count=\xi from 0] in {0,#4,...,#1} {
            \foreach \y [count=\yi from 0] in {0,#4,...,#2} {
                \ifodd\xi \ifodd\yi
                    \fill[shift={#5}] (\x,\y) \smallDot{#3}
                    \fill[shift={#5}] (\x+#4/2,\y+#4/2) \medDot{#3}
                \else
                    \fill[shift={#5}] (\x,\y) \bigDot{#3}
                    \fill[shift={#5}] (\x+#4/2,\y+#4/2) \smallDot{#3}
                \fi\else \ifodd\yi
                    \fill[shift={#5}] (\x,\y) \bigDot{#3}
                    \fill[shift={#5}] (\x+#4/2,\y+#4/2) \smallDot{#3}
                \else
                    \fill[shift={#5}] (\x,\y) \smallDot{#3}
                    \fill[shift={#5}] (\x+#4/2,\y+#4/2) \medDot{#3}
                \fi\fi
            }
            \draw[#7, shift={#5}] (-0.25-0.125,-0.25-0.125) rectangle (#1+0.5+0.125,#2+0.5+0.125);
        }
    }{}
    \ifstrequal{#6}{iso}{% Iso Dot Grid
        \pgfmathsetmacro{\maxX}{int(#1/\isoSpace)}%
        \pgfmathsetmacro{\maxY}{int(#2-1)}%
        \begin{scope}[xscale=\isoSpace]
            \foreach \y [count=\yi from 0] in {0,#4,...,\maxY} {
                \foreach \x [count=\xi from 0] in {0,#4,...,\maxX} {
                    % Offset every other column by half of the step size in the y-direction
                    \pgfmathsetmacro{\yoffset}{mod(\xi,2) == 0 ? #4/2 : 0}
                    \fill[shift={#5}, color = #8] (\x,\y+\yoffset) \bigDot{#3} % Draw points
                    % \ifodd\xi \ifodd\yi
                    %     \fill[shift={#5}] (\x,\y+\yoffset) \bigDot{#3}
                    % \else
                    %     \fill[shift={#5}] (\x,\y+\yoffset) \bigDot{#3}
                    % \fi\else \ifodd\yi
                    %     \fill[shift={#5}] (\x,\y+\yoffset) \bigDot{#3}
                    % \else
                    %     \fill[shift={#5}] (\x,\y+\yoffset) \smallDot{#3}
                    % \fi\fi
                }
            }
            \draw[#7, shift={#5}] (-0.5,-0.5) rectangle (\maxX+.5,\maxY+1);
        \end{scope}
    }{
        % \draw[#7, shift={#5}] (-0.5,-0.5) rectangle (#1+0.5,#2+0.5);
    }  
}
% \boundingBox[13][14][0.5pt][0.5]