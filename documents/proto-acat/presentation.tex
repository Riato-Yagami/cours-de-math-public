% VARIABLES %%%
\def\authors{Jules Pesin}
% \date{\today}
\def\longTitle{Proto - PIXEL - ACAT}
\def\shortTitle{Acat}
\def\theme{\longTitle}
%%

% \maketitle
% \disableAnimation
% \shortAnimation
% \firstSlide

\bsec{Présentation}
\slide{}{
    \ssec
    Fusionnant l'essence palpitante de Super Smash Bros. Melee avec le Gameplay précis de Celeste;
    le projet ACAT vous plongera dans un monde post-apocalyptique où les rues de Paris se mêlent aux anciennes cités mayas,
    créant un décor unique et mystique.
    Incarnant Maya,
    une adolescente dotée de pouvoirs mystiques inspirés des capacités de Fox dans SSBM,
    vous vous embarquerez dans une quête héroïque guidée par la déesse Ixchel.
    \rimg{proto-acat/concept-art/post_apocalyptic_paris.png}[0.9\linewidth]
    (Concepts arts réalisés avec \href{https://labs.openai.com/}{DALL.E})
}

\bsec{Gameplay}

\slide{}{
    \ssec Le jeu se découpera de la même manière que Celeste,
    avec de longs niveaux découpés en différents écrans,
    sans aucun risque de Game Over.
    Le moveset sera très proche de celui de Fox.
    Un coup standard servira à éliminer différents ennemis,
    à la manière des jeux Rayman Origins et Legends.
    Maya aura également la possibilité de s'accrocher aux rebords (pas implémenté),
    offrant ainsi un level design similaire à celui de The End is Nigh.\\
    \dividePage{\rimg{proto-acat/ability/rayman_kick.png}}{\rimg{proto-acat/ability/ash_ledge_grab.png}}[0.65]
    
    Certaines plateformes seront également traversables à l'aide du bouton de saut (pas implémenté) lorsqu'elle est accroupie.
    Des capacités spéciales (décrites ci-dessous) seront débloquées au fur et à mesure de la progression dans le jeu.
}

\bsec{Abilités}
\bsubsec{Neutral Special - Frape Sacrificiel}

\slide{}{
    \ssec
    \ssubsec
    Maya manie le Frape Sacrificiel,
    grace à l'obtention du Cœur d'Ah Puch.
    Lorsqu'elle l'utilise,
    Maya libère une rafale d'énergie de sa paume sous la forme d'un rayon rapide et droit.
    Cette attaque peut être utilisée pour attaquer ces adversaires,
    ainsi qu'a activer certains méchanismes.

    \rimg{proto-acat/ability/fox_neutral_special.png}[5cm]
}

\bsubsec{Side Special - Ruée de Quetzal}

\slide{}{
    \ssubsec
    Maya possède la Plume de Quetzal,
    une plume mythique de l'oiseau légendaire Quetzal.
    En activant cet artefact,
    Maya effectue la "Ruée du Quetzal",
    une ruée rapide et réctiligne qui peut être utilisée pour passer de grands gouffres.

    \rimg{proto-acat/ability/fox_side_special.png}[5cm]
}

\bsubsec{Up Special - Ascension Céleste}

\slide{}{
    \ssubsec
    Lorsque Maya utilise le Bâton de l'Ascension Céleste,
    elle canalise le pouvoir d'Ixchel pour effectuer un mouvement de ruée,
    plus court que celui de la plume mais dont l'angle peut etre modifier.

    \rimg{proto-acat/ability/fox_up_special.png}[5cm]
}

\bsubsec{Down Special - Rayonnement de Chaac}

\slide{}{
    \ssubsec
    Maya utilise les larmes de Chaac,
    afin de frapper son alentours d'une vague d'electricité rapide,
    reflètant les projectiles et permetant d'activer certains interupteurs.

    \rimg{proto-acat/ability/fox_down_special.png}[5cm]
}

\bsec{Histoire}

\slide{}{
    \ssec
    Maya vit dans un monde post-apocalyptique,
    où les vestiges de la civilisation moderne de Paris coéxiste avec les mystères et la magie de l'ancienne cité maya.
    La ville fusionnée,
    résultat d'une anomalie cosmique inexpliquée,
    abritait des zones délabrées,
    des ruines majestueuses,
    et des environnements marqués par la symbiose de deux époques distinctes.\\
}

\slide{}{
    Adolescente solitaire,
    née au cœur de ce paysage étrange,
    élevée par des survivants de différentes cultures qui s'étaient regroupés pour former des communautés hétéroclites.
    Sa vie quotidienne était imprégnée de la recherche constante de resources,
    de navigation à travers les terrains changeants de Paris,
    et de la découverte des mystères de ce monde fusionné.\\
}

\slide{}{
    Bien qu'elle ait grandi parmi les vestiges d'une société passée et les reliques de la cité maya,
    Maya était spéciale.
    Des rumeurs circulaient sur ses aptitudes inhabituelles,
    une connexion mystique avec l'énergie qui imprégnait la ville.
    Cependant,
    elle ignorait la véritable étendue de ses pouvoirs jusqu'à ce jour crucial où Ixchel,
    déesse maya de la Lune et de la Maternité,
    se manifesta dans ses rêves,
    l'invitant à entreprendre une quête divine.\\
}

\slide{}{
    La vie de Maya,
    jusqu'alors façonnée par la survie au sein de ce monde fusionné,
    allait prendre un tournant extraordinaire alors qu'elle se lançait dans une aventure héroïque guidée par la déesse Ixchel,
    le destin de la ville fusionnée reposant sur ses épaules.\\
}

\bsec{Progression du jeu}
\bsubsec{L'Appel à l'Aventure}

\slide{}{
    \ssec
    \ssubsec
    Maya reçoit une vision mystique d'Ixchel,
    qui lui donnera comme quetes de réunir 4 artefacts divins;
    le premier étant la Plume de Quetzal qu'elle trouvera en gravissant la Tour Effeil.\\
    \dividePage{\rimg{proto-acat/concept-art/eiffel_tower_1.png}}{\rimg{proto-acat/concept-art/eiffel_tower_2.png}}
}

\bsubsec{Catacombes et ces algues Bioluminescentes}

\slide{}{
    \ssubsec
    Maya plonge dans les catacombes pour collecter les Larmes de Chaac,
    affrontant des énigmes et des obstacles au milieu des rayonnements bioluminescent.\\
    \dividePage{\rimg{proto-acat/concept-art/catacombs_1.png}}{\rimg{proto-acat/concept-art/catacombs_2.png}}
}

\bsubsec{Montmartre et les Épreuves de l'Âme}

\slide{}{
    \ssubsec
    En escaladant Montmartre,
    Maya fait face à des épreuves liées à l'âme pour obtenir le Cœur d'Ah Puch,
    dévoilant le lien entre les dieux mayas et l'anomalie cosmique.
}

\bsubsec{Louvre et la Vision Divine}

\slide{}{
    \ssubsec
    Explorant les décombres cosmiques du Louvre,
    Maya récupère les Yeux de Hunab Ku,
    acquérant un aperçu des souhaits des dieux et de la véritable raison de l'anomalie cosmique.
    \rimg{proto-acat/concept-art/cosmic_louvre.png}[10cm]
}

\bsubsec{La Convergence Harmonieuse}

\slide{}{
    \ssubsec
    Kukulkan libéré,
    gigantesque serpent,
    divinité de la résurection.
    Maya aura à ce rendre au cœur de la ville fusionnée.
    Ixchel lui livrera le bâton de d'Ascension Céleste pour l'aider dans cette quête.
    Elle y assemblera les artefacts,
    déclenchant une séquence climax où le monde répondra enfin aux souhaits des dieux.
    Et l'anomalie cosmique,
    maintenant en équilibre,
    dévoilera un chemin vers une nouvelle ère.
}

\bsubsec{Conclusion}

\slide{}{
    \ssubsec
    Le jeu se termine avec le succès de Maya;
    dans le rétablissement de l'harmonie dans la ville fusionnée.
    Le paysage se transforme,
    montrant la coexistence d'éléments mayas anciens et de Paris moderne, dans un mélange harmonieux.
    Le dénouement du jeu reflète la résilience de l'humanité et la nature cyclique de la vie,
    Maya devenant un symbole d'espoir pour un monde revitalisé.
    \rimg{proto-acat/concept-art/mayan_paris.png}[10cm]
}

\bsec{Motivation - Un jeu communautaire}
\slide{}{
    \ssec
    Je pense que le développement de ce jeu permettra de mettre en place une collaboration profonde,
    entre l'équipe de développement permanant et le reste des membres de PSU.
    En effet j'ai fait le choix de présenter un concept de jeu qui ne sort pas trop de l'odinaire;
    afin de faciliter un travail ponctuel sur le projet. 
}

\bsubsec{Level Design Collaboratif}

\slide{}{
    \ssubsec
    Le choix d'un jeu de plateforme divisé en sous-écrans favorise des sessions de développement courtes.
    Cela permet également de concevoir facilement des niveaux sans même avoir besoin de toucher à un PC,
    par exemple,
    sur une simple feuille A4.
    De plus,
    il existe d'excellents outils tels que \href{https://ldtk.io/}{LDtk},
    qui permettent une implémentation rapide de nouvelles idées.\\

    \begin{center}
        \animategraphics[loop,autoplay,width=\linewidth]{5}{images/proto-acat/ldtk/ldtk/ldtk_}{12}{30}
    \end{center}

    L'utilisation d'un gameplay fait de références permettra rapidement à ceux qui le souhaiteront,
    d'imaginer des niveaux sans même avoir à tester le jeu.
}

\bsubsec{Création Artistique}

\slide{}{
    \ssubsec
    Une résolution très réduite de $8\times8$ px,
    particulièrement adaptée à ce type de jeu,
    offre l'avantage de conférer une homogénéité appréciable au travail de plusieurs artistes.
    Cette approche facilitera grandement le respect de la Direction Artistique du Jeu.
}

\bsubsec{Testing et Feedback}

\slide{}{
    \ssubsec
    Le choix de modéliser le moveset du héros sur le gameplay de Fox dans Super Smash Bros.,
    une licence de jeu phare dans cette association,
    offre une solution élégante pour pallier à une jeu comportant de nombreux contrôles différents.
    En effet nous utiliserons la familiarité des joueurs avec le gameplay de la licence,
    Lors des phases de play-test;
    notamment durant les phases de développement où les tutorielles n'aurait pas encore était réalisé.
}

\slide{}{
    \section*{Liens}
    \begin{itemize}
        \item \href{https://github.com/Riato-Yagami/proto-PIXEL-ACAT}{ACAT - GitHub repository}
        \item \href{https://juels.dev/}{juels.dev}
        \item asset pack: \href{https://greenpixels.itch.io/}{greenpixels},
            \href{https://octoshrimpy.itch.io/}{octoshrimpy},
            \href{https://lionheart963.itch.io/}{Warren Park},
            \href{https://free-game-assets.itch.io/}{Craftpix.net}
            \href{https://greatdocbrown.itch.io/}{greatdocbrown}.
        \item scripts: \href{https://github.com/Ev01/PlatformerController2D}{PlatformerController2D (Evan Barac)},
        \href{https://youtu.be/lPJMj-ov7ys?si=FkPn3sAEAGh878MM}{Room Based Camera (Nick Vatanshenas)},
        \href{https://youtu.be/n0X_BBbq_pw?si=hOGEogJx5Z27XyRQ}{Screen Transition Framework (SporkTank)}
        
    \end{itemize}
}


