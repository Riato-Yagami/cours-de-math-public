% VARIABLES %%%
\def\theme{Trigonométrie}
% \def\date{date}
%%%%%%%%%%%%%%%

% DEFINITIONS %
\newcommand{\norm}[1]{||\vect{#1}||}
\newcommand{\vangle}[2]{(\widehat{\vect{#1},\vect{#2}})}
\newcommand{\pscalv}[2]{\pscal{\vect{#1}}{\vect{#2}}}
\newcommand{\nsquare}[1]{\pscal{#1}{#1}}

\def\u{\vect{u}}
\def\v{\vect{v}}
\def\w{\vect{w}}
\def\vi{\vect{i}}
\def\vj{\vect{j}}
\def\hpi{\dfrac{\pi}{2}}

\def\co{\textrm{côté opposé}}
\def\ca{\textrm{côté adjacent}}
\def\hy{\textrm{hypoténuse}}
%%%%%%%%%%%%%%

% \setboolean{subsectionInOutline}{true}
% \setboolean{outline}{true}
% \setboolean{demonstration}{false}

% \setboolean{showRef}{false}

\hbox{}
\leconInfo{
    \ilink{3e}
    \ilink{1re}
    \ilink{Terminale Option Expertes}
} % Niveaux
[\item propriété Triangle \item théorème de Pythagore \item fonction] % Prérequis
[] % Thèmes
[] % Motivation

\section{Triangle rectangle}

\subsection{Définitions}

\df{côté du triangle rectangle}{
    Soit $ABC$ triangle rectangle en $A$.
    \begin{center}
        \begin{tikzpicture}[scale=0.6]
            \tkzDefPoint(0,0){A}
            \tkzDefPoint(5,0){B}
            \tkzDefPoint(0,3){C}
            \draw[very thick] (A)--(B)--(C)--cycle;
            \tkzLabelSegment[below](A,B){$\ca$}
            \tkzLabelSegment[above right](B,C){$\hy$}
            \tkzLabelSegment[left](C,A){$\co$}
            \tkzMarkAngle[size=0.9cm,arc=l](C,B,A)
            \tkzMarkRightAngle[scale=1.5](B,A,C)
            \tkzLabelAngle[pos=1.5](C,B,A){$b$}
            \tkzLabelPoints[below left](A)
            \tkzLabelPoints[below right](B)
            \tkzLabelPoints[above left](C)
        \end{tikzpicture}
    \end{center} 
}


\df{Sinus}{\vspace*{-0.25cm}
    \begin{align*}
        \sin(\widehat{b}) = \dfrac{\co}{\hy} = \dfrac{AC}{BC}
    \end{align*}
}

\df{Cosinus}{\vspace*{-0.25cm}
    \begin{align*}
        \cos(\widehat{b}) = \dfrac{\ca}{\hy} = \dfrac{AB}{BC}
    \end{align*}
}

\df{Tangente}{\vspace*{-0.25cm}
    \begin{align*}
        \tan(\widehat{b}) = \dfrac{\co}{\ca} = \dfrac{AC}{AB}
    \end{align*}
}

\rmk{Moyen mnémotechnique}{
    CAH SOH TOA
}

\pr{}{
    Soit $\widehat{a}$ un angle aigue,
    on a :
    \begin{align*}
        0 < &\cos{\widehat{a}} < 1\\
        0 < &\sin{\widehat{a}} < 1
    \end{align*}
}

\subsection{Formules}

Soit $x\in\m{R}$

\pr{Somme des carrés}{\vspace*{-0.25cm}
    \begin{equation*}
        \cos^2{x} + \sin^2{x} = 1\\
    \end{equation*}
}

\demo{}{
    Avec $x = \widehat{ABC}$,
    on a:
    \begin{align*}
        \cos{x} &= \dfrac{AB}{BC} \et \sin{x} = \dfrac{AC}{BC}
        \ialors \cos^2{x} + \sin^2{x} &= \left(\dfrac{AB}{BC}\right)^2 + \left(\dfrac{AC}{BC}\right)^2\\
        &= \dfrac{AB^2+AC^2}{BC^2}\\
        &= 1 \;\textrm{d'après le théorème de Pythagore.}
    \end{align*}
}[\href{https://docplayer.fr/210338046-Lecon-n-o-11-trigonometrie-applications-capes-session-cbmaths-cbmaths-fr-derniere-mise-a-jour-1-er-avril-2021.html}{CBMaths.fr}]

\pr{Tangente}{\vspace*{-0.25cm}
    \begin{equation*}
        \tan{x} = \dfrac{\sin{x}}{\cos{x}}
    \end{equation*}
}

\demo{}{\vspace*{-1cm}
    \begin{align*}
        \dfrac{\sin{x}}{\cos{x}} &= \dfrac{\dfrac{AB}{BC}}{\dfrac{AC}{BC}}\\
        &= \dfrac{AB}{BC} \times \dfrac{BC}{AC}\\
        &= \dfrac{AC}{AB} = \tan{x}
    \end{align*}
}

\section{Produit scalaire}

\subsection{Définitions}

\df{Forme bilinéaire}{
    Une \textbf{forme bilinéaire} est une application qui,
    à un couple de vecteurs associe un scalaire,
    et qui a la particularité d'être linéaire en ses deux arguments.
}[\href{https://fr.wikipedia.org/wiki/Forme_bilinéaire}{Wikipedia}]

\df{Forme bilinéaire symétrique}{
    Une forme bilinéaire est dite \textbf{symétrique},
    si l'échange des deux vecteurs ne modifie pas le résultat.
}

\df{Forme bilinéaire définie}{
    Une forme bilinéaire $f$ est dite \textbf{définie},
    si elle n'a pas de vecteur $x\neq0$ tel que $f(x,x) = 0$.
}

\df{Produit scalaire}{
    Un \textbf{produit scalaire} est une forme bilinéaire symétrique définie positive d'un espace vectoriel sur les nombres réels.
}[\href{https://fr.wikipedia.org/wiki/Produit_scalaire}{Wikipedia}]

\df{Norme}{
    La norme donne la longueur d'un vecteur.
}

\subsection{Espace euclidien}

% \df{Norme euclidienne}{
%     La \textbf{norme euclidienne} d'un vecteur peut se calculer à l'aide de ses coordonnées dans un repère orthonormé à l'aide du théorème de Pythagore;
%     pour un vecteur \u elle s'écrit:
%     \begin{equation*}
%         \norm{u} = \sqrt{x^2+y^2}
%     \end{equation*}
% }

\df{Norme euclidienne}{
    Soient $A$ et $B$ deux points du plan usuel, la norme du vecteur \vect{AB} est la longueur du segment $[AB]$.
}

Soient \m{E} un espace vectoriel muni de la norme usuelle,
et $\u, \v \in\m{E}$.

\df{Produit scalaire euclidien}{
    Le produit scalaire est donné par:
    \begin{equation*}
        \pscalv{u}{v} = \norm{u} \times \norm{v} \times \cos\vangle{u}{v}
    \end{equation*}
}

\pr{Symetrie du produit scalaire euclidien}{
    \begin{equation*}
        \pscalv{u}{v} = \pscalv{v}{u}
    \end{equation*}
}

\pr{Bilinéarité du produit scalaire euclidien}{
    Soient $\lambda\in\m{R}$ et $\w\in\m{E}$
    \begin{align*}
        \pscal{\lambda \u + \v}{\w} &= \lambda\pscalv{u}{w} + \pscalv{v}{w}\\
        \et \pscal{\u}{\lambda\v + \w} &= \lambda\pscalv{u}{v} + \pscalv{u}{w}
    \end{align*}
}

% \demo{}{
%     \begin{align*}
%         \pscal{\lambda \u + \v}{\w} &= ||\lambda \u + \v|| \times \norm{w} \times \cos(\vangle{\lambda u + \v}{v})
%     \end{align*}
% }

\demo{}{
    \begin{align*}
        \pscalv{u}{v} &= \norm{u} \times \norm{v} \times \cos\vangle{u}{v}\\
        &= \norm{v} \times \norm{u} \times \cos -\vangle{v}{u}\\
        &= \norm{v} \times \norm{u} \times \cos\vangle{v}{u}\\
        &= \pscalv{v}{u}
    \end{align*}
}

\pr{Norme euclidienne}{
    \begin{equation*}
        \norm{u} = \sqrt{\pscalv{u}{u}}
    \end{equation*}
}

\pr{Carré scalaire}{
    \begin{equation*}
        \u^2 = \pscalv{u}{u} = \norm{u}^2
    \end{equation*}
}

\demo{}{
    \begin{align*}
        \pscalv{u}{u} &= \norm{u} \times \norm{u} \times \cos\vangle{u}{u}\\
        &= \norm{u}^2 \cos(0)\\
        &= \norm{u}^2
    \end{align*}
}

\pr{Identité remarquables}{
    \begin{enumerate}
        \item $(\u+\v)^2 = \u^2 + \v^2 + 2\pscalv{u}{v}$
        \item $(\u-\v)^2 = \u^2 + \v^2 - 2\pscalv{u}{v}$
        \item $(\u+\v)(\u-\v) = \u^2 - \v^2$
    \end{enumerate}
}

\demo{}{
    \begin{align*}
        (\u \pm \v)^2 &= \nsquare{\u\pm\v}\\
        &= \nsquare{\u} \pm \pscalv{u}{v} \pm \pscalv{v}{u} + \nsquare{\v}\; (\textrm{bilinéarité})\\
        &= \nsquare{\u} + \nsquare{\v} \pm 2\pscalv{v}{u}
    \end{align*}
    \begin{align*}
        \pscalv{\u+\v}{\u-\v} &= \nsquare{\u\pm\v}\\
        &= \nsquare{\u} \pm \pscalv{u}{v} \pm \pscalv{v}{u} + \nsquare{\v}\; (\textrm{bilinéarité})\\
        &= \nsquare{\u} + \nsquare{\v} \pm 2\pscalv{v}{u}
    \end{align*}
}

\cor{Norme au carré}{
    \begin{itemize}
        \item $||\u + \v||^2 = \norm{u}^2 + \norm{v}^2 + 2\pscalv{u}{v}$
        \item $||\u - \v||^2 = \norm{u}^2 + \norm{v}^2 - 2\pscalv{u}{v}$
    \end{itemize}
}

\cor{}{
    Soit $ABC$ un triangle,
    on a:
    \begin{align*}
        \pscalv{AB}{AC} = \dfrac{1}{2} (AB^2 + AC^2 - BC^2)
    \end{align*}
}

\pr{Inégalité de Cauchy-Schwarz}{
    \begin{equation*}
        |\u.\v| \leqslant \norm{u} \times \norm{v}
    \end{equation*}
}

\demo{}{
    On suppose \v non nul (sinon la propriété est clair),
    posons:
    \begin{align*}
        P(t) &= \norm{u+tv}^2\\
        &= \pscal{\u+t\v}{\u+t\v}\\
        &= \nsquare{\u} + t\pscalv{u}{v} + \nsquare{\v} + t\pscalv{u}{v}\; \textrm{(bilinéarité)}\\
        &= \norm{u}^2 + t^2\norm{v}^2 + 2t\pscalv{u}{v} \geqslant 0\; \textrm{(polynome de 2nd degré)}\\
        \ialors \Delta = 4&\pscalv{u}{v}^2 - 4\norm{u}^2\norm{v}^2 \leqslant 0
        \idou &\pscalv{u}{v} \leqslant \norm{u}\times\norm{v}
    \end{align*}
}

\subsection{Orthogonalité}

\df{Orthogonalité}{
    Dans l'espace, deux droites sont orthogonales si elles sont chacune parallèles à des droites se coupant en angle droit.
}[\href{https://fr.wikipedia.org/wiki/Orthogonalité}{Wikipédia}]

\pr{Vecteurs orthogonaux}{
    \u et \v sont orthogonaux \ssi $\pscalv{u}{v} = 0$.
}

\demo{}{
    On suppose \u et \v non nuls (sinon c'est évident).
    \begin{align*}
        &\pscalv{u}{v} = 0\\
        &\eqv \norm{u} \times \norm{v} \times \cos\vangle{u}{v} = 0\\
        &\eqv \cos\vangle{u}{v} = 0\\
        &\eqv \u \perp \v
    \end{align*}
}

\df{Projeté orthogonale}{
    Le \textbf{projeté orthogonale} d'un point $A$ sur une droite $(d)$ est le point:
    $H = p_{(d)}(A)$ sur $(d)$ tel que $(d)$ et $(AH)$ soient perpendiculaires.
}[\href{https://fr.wikipedia.org/wiki/Projection_orthogonale}{Wikipédia}]

\pr{}{
    Soient les points $O,A,B$ et $H$ le projeté orthogonale de $B$ sur $(OA)$,
    on a:
    \begin{equation*}
        \pscalv{OA}{OB} = \pscalv{OA}{OH}
    \end{equation*}
}

\demo{}{
    \begin{align*}
        \pscalv{OA}{OB} &= \pscal{\vect{OA}}{\vect{OB} + \vect{HB}}\\
        &= \pscalv{OA}{OH} + \pscalv{OA}{HB}
        \ior \pscalv{OA}{HB} &= 0 \car \vect{OA} \perp \vect{HB}
        \ialors \pscalv{OA}{OB} &= \pscalv{OA}{OH}
    \end{align*}
}

\subsection{Produit scalaire dans un repère orthonormé}
Le plan est muni d'un repère orthonormé $(O;\vi;\vj)$.
Soient $\u\coord{x}{y}$ et  $\u\coord{x'}{y'}$.
\pr{Produit scalaire dans un repère orthonormé}{
    On a:
    \begin{equation*}
        \pscalv{u}{v} = xx' + yy'
    \end{equation*}
}

\demo{}{
    \begin{align*}
        \pscalv{u}{v} &= \pscal{x\vi+y\vj}{x'\vi+y'\vj}\\
        &= xx'\pscalv{i}{i} + xy'\pscalv{i}{j} + yx'\pscalv{j}{i} + yy'\pscalv{j}{j}\\
        \ior \pscalv{j}{i} &= 0 \car \vi \perp \vj\\
        \ialors \pscalv{u}{v} &= xx'\norm{i}^2 + yy' xx'\norm{j}^2\\
        \ior \norm{i}^2 &= 1 = \norm{j}^2\\
        \ialors \pscalv{u}{v} &= xx' + yy'\\
    \end{align*}
}

\cor{Norme dans un repère orthonormé}{
    \begin{equation*}
        \norm{u} = \sqrt{x^2 + y^2}
    \end{equation*}
}

\rmk{Orthogonalité}{
    \begin{equation*}
        \u \perp \v \eqv xx' = yy'
    \end{equation*}
}

\demo{Bilinéarité}{
    \TODO
}

\section{Trigonométrie}

\thm{d'Al-Kashi}{
    \dividePage{
        Soit un triangle de côté $a,b,c$ et d'angle $\gamma$ opposé à $c$,
        on a:
        \begin{equation*}
            c^2 = a^2+b^2 - 2ab cos(\gamma)
        \end{equation*}
    }{
        \begin{center}
            \begin{tikzpicture}[scale=0.6]
                \tkzDefPoint(0,0){A}
                \tkzDefPoint(5,0){B}
                \tkzDefShiftPoint[A](60:4){C} % Adjust the angle for gamma and the length for side 'c'
                \draw[very thick] (A)--(B)--(C)--cycle;
                \tkzLabelSegment[below](A,B){$c$}
                \tkzLabelSegment[above right](B,C){$a$}
                \tkzLabelSegment[above left](C,A){$b$}
                \tkzMarkAngle[size=0.9cm,arc=l](A,C,B)
                \tkzLabelAngle[pos=1](A,C,B){$\gamma$}
                \tkzLabelPoints[below left](A)
                \tkzLabelPoints[below right](B)
                \tkzLabelPoints[above](C)
            \end{tikzpicture}
        \end{center} 
    }
}

\demo{}{
    \begin{align*}
        c^2 &= AB^2 = \norm{AB}^2\\
        &= ||\vect{AC} + \vect{CB}||^2 \; (\textrm{relation de Chalse})\\
        &= \norm{AC}^2+\norm{CB}^2 + 2\pscalv{AC}{CB}\\
        &= \norm{AC}^2+\norm{CB}^2 - 2\pscalv{CA}{CB}\\
        &= a^2 + b^2 - 2ab\cos(\gamma)
    \end{align*}    
}[\href{https://share.miple.co/content/AtXi78spumnFd}{Démos Maths MPSI}]

\pr{Formules de décalage d'angle}{
    \dividePage{
        Soit $x\in\m{R}$
        \begin{enumerate}
            \item \begin{itemize}
                \item $\cos(\pi + x) = -\cos(x)$
                \item $\sin(\pi + x) = -\sin(x)$
                \item $\cos(\pi - x) = -\cos(x)$
                \item $\sin(\pi - x) = \sin(x)$
            \end{itemize}
            \item \begin{itemize}
                \item $\cos(\hpi + x) = -\sin(x)$
                \item $\sin(\hpi + x) = \cos(x)$
                \item $\cos(\hpi - x) = \sin(x)$
                \item $\sin(\hpi - x) = \cos(x)$
            \end{itemize}
        \end{enumerate}
    }{
        \begin{tikzpicture}[scale=2.5,cap=round,>=latex]
            % Draw the unit circle
            \draw[thick] (0,0) circle(1cm);
            % Draw the axes
            \draw[->] (-1.2,0) -- (1.2,0) node[right,fill=white] {};
            \draw[->] (0,-1.2) -- (0,1.2) node[above,fill=white] {};
            % Define points
            \def\x{40} % Angle for demonstration
            \coordinate (M) at (\x:1);
            \coordinate (N) at (\x+180:1);
            \coordinate (Mp) at (-\x:1);
            \tkzDefPoint(0,0){O}
            \coordinate (I) at (0:1);
            \coordinate (J) at (90:1);
            \coordinate (H) at (-90:1);
            % Draw points and lines
            \foreach \point in {M,N,Mp}
            \fill (\point) circle(1.2pt);
            \draw (0,0) -- (M);
            \draw (0,0) -- (N);
            \draw (0,0) -- (Mp);
            % Label points
            \node at (M) [above right] {$M(\cos x, \sin x)$};
            \node at (N) [below left] {$N(-\cos x, -\sin x)$};
            \node at (Mp) [below right] {$M'(\cos x, -\sin x)$};
            \node at (O) [below left, xshift = 4pt, yshift = -3pt] {$O$};
            \node at (I) [below right] {$I$};
            \node at (J) [above right] {$J$};
            \node at (H) [below right] {$H$};
            % Draw angles
            \draw[->,red] (0.3,0) arc (0:\x:0.3);
            \node[red] at (\x/2:0.4) {$x$};
            \draw[->,blue] (0.5,0) arc (0:\x+180:0.5);
            \node[blue] at (\x/2+90:0.7) [xshift = -3pt] {$\pi+x$};
            \draw[->,Green] (0.4,0) arc (0:-\x:0.4);
            \node[Green] at (-\x/2:0.6) {$-x$};
        \end{tikzpicture}
    }[0.4]
}

\demo{}{
    Soit $M$ un point du cercle trigonométrique d'argument $x$.
    On a $M(\cos x, \sin x)$
    \begin{enumerate}
        \item Le point $N$ d'argument $\pi+x$ est le symétrique de $M$ par rapport à $O$.\\
        Alors $N(-\cos x, -\sin x)$.\\
        Les deux autres formules découles des égalités:
        $\lfbrace{\cos(-x) = \cos{x} \\ \sin(-x) = -\sin{x}}$
        \item Soit $H$ le symétrique de $J$ par rapport à $O$.
        Soit $M'$ d'argument $-x$ (symétrique de $M$ par rapport à l'axe des abscisses).\\
        On a $M'(\cos x, - \sin x)$.
        \begin{align*}
            \vangle{h}{i} &\equiv \vangle{h}{OM'} + \vangle{OM'}{i} [2\pi]\\
            \ialors \hpi &\equiv \vangle{h}{OM'} + x [2\pi]
            \ialors \vangle{h}{OM'} &\equiv \hpi - x [2\pi]
        \end{align*}
        Alors $M'(\cos(\hpi - x), \sin(\hpi - x))$ dans le repère $(O,H,I)$.\\
        Or pour $P(a,b)$ dans $(O,I,J)$, on a $P(-b,a)$ dans $(O,H,I)$.\\
        D'ou les égalités.
    \end{enumerate}
}