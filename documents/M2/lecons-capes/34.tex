% VARIABLES %%%
\def\theme{Fonctions Logarithmes}
% \def\date{date}
%%%%%%%%%%%%%%%

% DEFINITIONS %
\def\Logn{Logarithme népérien}
\def\logn{logarithme népérien}
\def\expn{exponentielle}

\newcommand{\limto}[1]{
    \lim\limits_{x\to#1}
}

\def\limP{\limto{+\infty}}
\def\limM{\limto{-\infty}}
\def\limO{\limto{0}}

\def\hyp{Hyperbole Terminale Option Spécialité 2021}
\def\bib{\href{https://www.bibmath.net/resources/index.php?action=affiche&quoi=capes/demos/analyse_logarithme.html}{Bibmath}}
%%%%%%%%%%%%%%

\setboolean{subsectionInOutline}{true}
% \setboolean{outline}{true}
% \setboolean{demonstration}{false}

% \setboolean{showRef}{false}

\hbox{}
\leconInfo{
    \ilink{Terminale Option Spécialité}
    \ilink{Terminale Technologique}
    \ilink{Terminale Option Complémentaire}
} % Niveaux
[\item Continuité et dérivation
\item Fonction exponentielle
\item Limites de fonction
\item Calcul de puissances
\item Théorème des valeurs intermédiaires
\item Fonction réciproque
\item Suites
] % Prérequis
[\ilink{Fonctions} \ilink{Puissances}] % Thèmes
[Va permettre d'alléger des calculs très lourds,
de multiplications de nombres avec beaucoup de chiffres avant et après la virgule;
en physique et finances notamment.] % Motivation

\section{Fonctions Logarithme}

\subsection{Définitions}

\df{Fonctions Logarithme}{
    Un \textbf{logarithme} est la fonction réciproque d'une exponentiation,
    c'est-à-dire que le logarithme de base $b$ d'un nombre réel strictement positif,
    est la puissance à laquelle il faut élever la base $b$ pour obtenir ce nombre.
    On le note $\log_b$.
}[\href{https://fr.wikipedia.org/wiki/Logarithme}{Wikipédia}]

\pr{Bijectivité de la fonction \expn}{Pour $x>0$,
il existe un unique $y\in\m{R}$ tel que $\exp{(y)} = x $.
}[\bib]

\demo{}{
    La fonction $\exp$ est continue et strictement croissante sur $\m{R}$.
    De plus, $\limM\exp{(x)} = 0$ et $\limP\exp{(x)} = +\infty$.
    Ainsi, d'après le théorème des valuers intermédiaires,
    il existe un unique $y\in\m{R}$ tel que $\exp{(y)}=x$.
}[\href{https://www.maths-et-tiques.fr/telech/LogTS.pdf}{Yvan Monka}]

\df{Fonction \Logn}{
    La fonction \textbf{\logn}, notée $\ln$, est la bijection réciproque de la fonction \expn.
}[\href{https://fr.wikipedia.org/wiki/Logarithme_népérien}{Wikipédia}]

\rmk{Logarithme de base $e$}{
    Le logarithme népérien est alors le logarithme de base $e$.
}

\cor{Conséquence de la bijection}{
    On a alors:
    \begin{itemize}
        \item pour $(x,y)\in\m{R}_+^* \times \m{R}, e^x=\ln(y) \eqv y = \ln(x)$
        \item pour $x\in\m{R}^*, e^{\ln(x)}=x$
        \item pour $x\in\m{R}, \ln(e^x)=x$
    \end{itemize}
}

\rmk{Quelques résultats numériques}{
    On a alors:
    \begin{itemize}
        \item $\ln(1)=\ln(e^0)$
        \item $\ln(e)=\ln(e^1)$
        \item $\ln(\frac{1}{e})= \ln(e^{-1}) = -1$
    \end{itemize}
}

\pr{Symétrie graphique}{
    Dans un repère orthonormé,
    les courbes représentatives des fonctions \expn{} et \logn,
    sont symétrique par rapport à la droite d'équation $y=x$.
}[\hyp]

\demo{}{
    \vspace*{-1.2cm}
    \rmdr{Symétrie de droite $y=x$}{
    Pour $(a,b)\in\m{R}^2$,
    le symétrique du point $(a,b)$ par rapport à la droite d'équation $y=x$,
    est le point de coordonnés $(b,a)$.
    }\vspace*{0.5cm}
    Soit $a\in\m{R}$,
    On considère un point $(a,\ln(a))$ avec $a>0$ sur la courbe représentative du \logn.
    Son symétrique est le point $(\ln(a),a)$.
    Il est bien sur la courbe représentative de l'\expn,
    en effet, on a bien: $e^{\ln(a)} = a$.\\
    De manière semblable,
    on obtient que le symétrique d'un point sur la courbe représentative de l'\expn,
    se trouve sur la courbe représentative du \logn.
}[\bib]

\subsection{Changement de base}

\pr{Formule de changement de base}{
    Deux fonctions logarithmes ne diffèrent que d'une constante multiplicative:
    pour $(a,b)\in(\m{R}_+\backslash\{0,1\})^2$ et $x\in\m{R}_+^*$,
    on a:
    \begin{align*}
        \log_b(x) = \frac{\log_a(x)}{\log_a(b)}
    \end{align*}
}

\demo{}{
    Soient $b\in\m{R}_+\backslash\{0,1\}$ et $x\in\m{R}_+^*$,
    On a:
    \begin{align*}
        x &= b^{\log_b(x)}
        \ialors \log_a(x) &= \log_a(b^{\log_b(x)})
        \ialors \log_a(x) &= \log_b(x) \times \log_a(b)
        \idou \log_b(x) &= \frac{\log_a(x)}{\log_a(b)}
    \end{align*}
}[\href{https://exomath.info.ucl.ac.be/syllabus/functions-series/exp_and_log/4.Fonctions\%20logarithmes.html}{Syllabus interactif : Mathématiques}]

\cor{Changement de base avec le \logn}{
    Pour $b\in\m{R}_+\backslash\{0,1\}$ et $x\in\m{R}_+^*$,
    on a:
    \begin{align*}
        \log_b(x) = \frac{\ln(x)}{\ln(b)}
    \end{align*}
}

\rmk{Réduction du problème au \logn}{
    On peut alors, sans perte de généralité, s'interesser uniquement au \logn.
}

\section{Étude Analytique du \logn}

\subsection{Variations}

\pr{Croissance du \logn}{
    La fonction \logn{} est strictement croissante sur $]0;+\infty[$.
}

\demo{}{
    Soient $(a,b)\in\m{R}^2$ tels que:
    \begin{align*}
        0<a&<b\\
        \ialors e^{ln(a)}&<e^{\ln(b)}\\
        \ialors \ln(a)&<\ln(b) \textrm{ par croissance de l'\expn}
    \end{align*}
}[\hyp]

\rmk{Conséquence}{
    On a alors, pour $a\in\m{R}_+^*$:
    \begin{itemize}
        \item $\ln(a) > 0 \eqv a > 1$
        \item $\ln(a) < 0 \eqv a < 1$
    \end{itemize}
}

\subsection{Limites}

\pr{Limites du \logn}{
    \begin{itemize}
        \item $\limP\ln(x) = +\infty$
        \item $\limO\ln(x) = -\infty$
    \end{itemize}
}

\demo{Avec la fonction \expn}{
    \vspace*{-1.2cm}
    \rmdr{Limites infinie}{
        Pour $f$ définie sur $\m{R}$:
        \begin{equation*}
            \limP f(x) = +\infty \eqv
            \forall M \in \m{R} \; \exists A \in \m{R}, \; \forall x \geq A \; f(x) \geq M
        \end{equation*}
    }
    Soit $M$.
    Posons $A=\exp(M)$.
    Alors puisque la fonction logarithme est croissante,
    pour tout $x \geqslant A$, on a:
    $\ln(x) \geqslant \ln(A) = \ln(\exp(M)) = M$.
}[\bib]

\demo{Avec la relation fonctionnelle}{
    Soit $M$,
    et $ \forall n \in \m {N}, u_n = \ln(2^n) = n\ln(2)$,
    on a $\lim\limits_{n\to+\infty}u_n = +\infty$.
    Il existe alors $n_0\in\m{N}$ tel que $\ln(2^{n_0}) \geq M$.
    Posons $A=2^{n_0}$.
    Par croissance de la fonction \logn,
    pour tout $x \geq A $,
    on a:
    $\ln(x) \geq \ln(A)\geq \ln(2^{n_0}) \geq M$
}[\bib]

\subsection{Continuité et dérivation}

\pr{Continuité du \logn}{
    La fonction \logn{} est continue sur $\m{R}_+^*$.
}

\demo{}{
    Le \logn{} est bijection réciproque de l'\expn{} continue de $\m{R}$ dans $\m{R}_+^*$.
    Il est alors continue de $\m{R}_+^*$ dans $\m{R}$, d'après le théorème de la bijection.
}

\pr{Dérivabilité du \logn}{
    La fonction \logn{} est dérivable sur $\m{R}_+^*$.
}

\demo{}{
    Le \logn{} est bijection réciproque de l'\expn{},
    dérivable et strictement croissante de $\m{R}$ dans $\m{R}_+^*$.
    Aussi la dérivée de l'\expn{} est strictement positive.
    Alors le \logn{} est dérivable en $\m{R}_+^*$.
}[\href{https://fr.wikiversity.org/wiki/Fonctions_d\%27une_variable_réelle/Dérivabilité\#_et_opérations}{Wikiversity}]

\pr{Dérivée du \logn}{
    Pour $x>0$,
    on a:
    $\ln'(x) = \frac{1}{x}$
}

\demo{}{
    \vspace*{-1.2cm}
    \rmdr{Dérivation de fonctions composées}{
        Soit $f$ dérivable sur $\m{I}$ et $g$ dérivable sur $f(\m{I})$.
        Alors $(g \circ f)$ est dérivable sur $I$ et:
        \begin{equation*}
            (g \circ f)' = (g' \circ f) \times f'
        \end{equation*}
    }
    Pour $x\in\m{R^*}$,
    \begin{align*}
        \exp(\ln(x)) &= x\\
        \ialors (\exp(\ln(x)))' &= (x)'\\
        \ialors \ln'(x)\exp'(\ln(x)) &= 1 \textrm{ (dérivation de fonctions composées)}\\
        \ialors \ln'(x)\exp(\ln(x)) &= 1\\
        \ialors \ln'(x) \times x &= 1\\
        \idou \ln'(x) &= \frac{1}{x}
    \end{align*}
}[\bib]

\demo{Avec le taux d'accroissement}{
}[\href{https://www.lyceedadultes.fr/sitepedagogique/documents/math/mathTermS/06_fonction_logarithme/06_Cours_fonction_logarithme_neperien.pdf}
    {Le Lycée d'Adultes de la ville de Paris}]

\pr{Dérivations composées avec le \logn}{
    Soient $u$ fonction dérivable et strictement positive sur un intervalle $I$.
    Alors $\ln(u)$ est dérivable et $\ln'(u) = \frac{u'}{u}$.
}

\demo{}{
    Soient $f : x \mapsto \ln(x)$ et $g : x \mapsto u(x)$ et $x\in\m{I}$.
    \begin{align*}
        (f \circ g)' &= g' \times (f' \circ g)
        \ior f'(x) &= \frac{1}{x} \et g'(x) = u'(x)
        \ialors (f \circ g)'(x) &= u'(x) \times \frac{1}{u(x)}
        \idou \ln'(u) &= \frac{u'}{u}
    \end{align*}
}[\href{https://www.math-linux.com/mathematiques/derivee-de-fonction/article/derivee-de-ln-u}{Math-Linux.com}]

\subsection{Croissance comparée}

\def\lnx{\frac{\ln(x)}{x}}
\def\lnxn{\frac{\ln(x)}{x^n}}

\pr{Croissance comparée de $ln(x)$ et $x$ en l'infinie}{
    \begin{equation*}
        \limP \lnx = 0
    \end{equation*}
}

\demo{}{
    Pour $x \geq 1$,
    on pose:
    \begin{align*}
        f(x) &= \ln(x)-2\sqrt{x}
        \ialors f'(x) &= \frac{1}{x} - \frac{1}{\sqrt{x}}\;\;  (f \textrm{ de classe } C^1 \textrm{ sur } [1,+\infty[)
        \ior x &\geq \sqrt{x} > 0 \car x \geq 1
        \ialors f'(x) &\leq 0
        \ior f \textrm{ décroissante sur } &[1,+\infty[ \et f(1) = -2 \leq 0
        \ialors \ln(x) &\leq 2 \sqrt{x}
        \ialors 0 \leq \lnx &\leq \frac{2}{\sqrt{x}}
        \idou \limP \lnx &= 0 \textrm{ par le théorème des gendarmes}
    \end{align*}
}[\bib]

\cor{Croissance comparée de $ln(x)$ et $x^n$ en l'infinie}{
    Pour $n\in\m{N}\backslash\{0\}$
    \begin{equation*}
        \limP \lnxn = 0
    \end{equation*}
}

\demo{}{
    Pour $x \geq 1$ et $n\in\m{N}\backslash\{0\}$,
    on a:
    \begin{align*}
        \lnxn &= \frac{1}{x^{n-1}} \times \lnx
        \ior \limP \frac{1}{x^{n-1}} &= 0 \et \limP \lnx = 0
        \idou \limP \lnxn &= 0
    \end{align*}
}[\bib]

\def\xln{x \ln(x)}
\def\xnln{x^n \ln(x)}
\pr{Croissance comparée de $ln(x)$ et $x$ en $0$}{
    \begin{equation*}
        \limO \xln = 0
    \end{equation*}
}

\demo{}{
    Pour $x>0$,
    on a:
    \begin{align*}
        \xln &= \frac{1}{u} \ln\left(\frac{1}{u}\right) = -\frac{\ln(u)}{u} \avec u = \frac{1}{x}
        \ialors \limO \xln &= \lim\limits_{u\to+\infty} -\frac{\ln(u)}{u} = 0
    \end{align*}
}[\bib]

\cor{Croissance comparée de $ln(x)$ et $x^n$ en $0$}{
    Pour $n\in\m{N}\backslash\{0\}$,
    \begin{equation*}
        \limO \xnln = 0
    \end{equation*}
}

\demo{}{
    Pour $x>0$ et $n\in\m{N}\backslash\{0\}$,
    on a:
    \begin{align*}
        \xnln &= x^{n-1} \times \xln
        \ior \limO x^{n-1} = 0 \et \limO \xln = 0
        \idou \limO \xnln = 0
    \end{align*}
}[\bib]

\section{Propriétés algébriques}

\pr{Relation fonctionnelle}{
    Pour $(a,b)\in(\m{R}_+)^2$,
    on a:
    \begin{equation*}
        \ln(a \times b) = \ln(a) + \ln(b)
    \end{equation*}
}

\demo{}{
    Soient $(a,b)\in(\m{R}_+)^2$,
    on a:
    \begin{align*}
        e^{\ln(ab)} &= ab
        \iet e^{\ln(a)+\ln(b)} = e^{\ln(a)}e^{\ln(b)} &= ab
        \idou \ln(ab) &= \ln(a) + \ln(b) \textrm{ par injectivité de l'\expn}
    \end{align*}
}

\pr{Logarithme népérien d'un inverse}{
    Pour $a\in(\m{R}_+)$
    \begin{equation*}
        \ln\left(\frac{1}{a}\right) = -\ln(a)
    \end{equation*}
}

\demo{}{
    Soit $a\in(\m{R}_+)$,
    on a:
    \begin{align*}
        \ln\left(\frac{1}{a} \times a\right) &= \ln(1) = 0
        \iet \ln\left(\frac{1}{a} \times a\right) &= \ln\left(\frac{1}{a}\right) + \ln(a)
        \idou \ln\left(\frac{1}{a}\right) &= -\ln(a)
    \end{align*}
}[\hyp]


\pr{Logarithme népérien d'un quotient}{
    Pour $(a,b)\in(\m{R}_+)^2$,
    \begin{equation*}
        \ln\left(\frac{a}{b}\right) = \ln(b)-\ln(a)
    \end{equation*}
}

\demo{}{
    Soient $(a,b)\in(\m{R}_+)^2$,
    on a:
    \begin{align*}
        \ln\left(\frac{a}{b}\right) &= \ln\left(a \times \frac{1}{b}\right)\\
        &= \ln(a) + \ln\left(\frac{1}{b}\right)\\
        &= \ln(a) - \ln(b)
    \end{align*}
}[\hyp]

\pr{Logarithme népérien d'une puissance}{
    Pour $a\in\m{R}_+$ et $n\in\m{N}$,
    \begin{equation*}
        \ln(a^n) = n \ln(a)
    \end{equation*}
}

\demo{}{
    Soit $a\in\m{R}_+$ et $n\in\m{N}$,
    \begin{align*}
        e^{\ln(a^n)} &= a^n
        \iet e^{n\ln(a)} = (e^{\ln(a)})^n &= a^n
        \idou  \ln(a^n) = n \ln(a) \textrm{ par injectivité de l'\expn}
    \end{align*}
}

\pr{Logarithme népérien d'une racine carré}{
    Pour $a\in\m{R}_+$,
    \begin{equation*}
        \ln(\sqrt{a}) = \frac{1}{2} \ln(a)
    \end{equation*}
}

\demo{}{
    Soit $a\in\m{R}_+$,
    \begin{align*}
        (\sqrt{a})^2 &= a
        \ialors \ln((\sqrt{a})^2) &= \ln(a)
        \iet \ln((\sqrt{a})^2) &= 2\ln(\sqrt{a})
        \ialors \ln(a) &= 2\ln(\sqrt{a})
        \idou \frac{1}{2} \ln(a) &= \ln(\sqrt{a})
    \end{align*}
}[\hyp]

\section{Méthodes d'approximations}

\mthd{Algorithme de Briggs}{
    \vspace*{-0.6cm}
    \begin{algorithm}
        \KwData{$a\in\m{R_+^*}$ et $n\in\m{N}$}
        \KwRes{$\ln(a)$ a $10^{-n}$ près ($\rightarrow 2^i \times (a^{\frac{1}{2^i}} - 1)$)}
        $i \leftarrow 0$\;
        $u \leftarrow a$\;
        \Tq{$|u-1| \geq 10^{-n}$}{
            $i \leftarrow i+1$\;
            $a \leftarrow \sqrt{u}$\;
        }
        \KwRet{$2^i \times (u-1)$}\;
    \end{algorithm}        
}

\demo{}{
    Soit $(u_n)_{n\in\m{N}}$,
    définie telle que:
    \begin{align*}
        &\lfbrace{
            &\forall n \in \m{N}, u_{n+1} = \sqrt{u_n} \\
            &u_0 = a
        }\\
        \ialors &\lim\limits_{n \rightarrow +\infty} u_n = 1\\
        \intertext{d'où la terminaison de la boucle.}
        \iet \ln(1+x) &= x + \dl{x \rightarrow 0} (x) 
        \ialors \ln(z) &= z - 1 + \dl{z \rightarrow 1} (z-1) \textrm{ changement de variable } z = 1 + x
        \ialors 2^i\ln(z) &= 2^i(z - 1 + \dl{z \rightarrow 1} (z-1))
        \ialors \ln(z^{2^i}) &= 2^i(z - 1) + \dl{z \rightarrow 1} (z-1)
        \ialors \ln(y) &= 2^i(y^{\frac{1}{2^i}} - 1) + \dl{y \rightarrow 1^{2^i} = 1} (y^{\frac{1}{2^i}}-1) \textrm{ changement de variable } y = z^{2^i}
        \ior \lim\limits_{i\rightarrow+\infty} y^{\frac{1}{2^i}} &= 1
        \idou \ln(y) &\approx 2^i(y^{\frac{1}{2^i}} - 1) \textrm{ quand $i$ grand}
        % \ialors \ln(x) &\approx x-1 \textrm{ quand $x$ est proche de 1.}
        % \ior 2^i\ln(x) &= \ln(x^{2^i})
        % \ialors 2^i\ln(y^{\frac{1}{2^i}}) &= \ln(y^{\frac{2^i}{2^i}}) = \ln(y) \textrm{ par changement de variable } y = x^{2^i}
        % \ialors \ln(y) &\approx 2^i \times (y^{\frac{1}{2^i}} - 1) \textrm{ quand $i$ grand}  
        % \ior u &= a^{\frac{1}{2^i}}
    \end{align*}
    d'où le retour de la boucle.
}

\tice{Implémentation de l'algorithme de Briggs}{Python}{
    \lstinputlisting[language=Python]{resources/lecons-capes/34/algo_briggs.py}
}

\mthd{Algorithme de Brouncker}{
    $\rightarrow$ \href{https://www.geogebra.org/m/dcbuqkjx}{Animation Géogebra}
    \begin{equation*}
        \ln(2) = \sum_{k=0}^{+\infty}\frac{1}{(2k+1)(2k+2)}
    \end{equation*}
}[\href{https://www.geogebra.org/m/dcbuqkjx}{Jean-Paul Berroir}]