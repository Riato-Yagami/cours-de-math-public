% VARIABLES %%%
\def\theme{Transformations du plan. Frises et pavages.}
\def\date{24/10/2023}
%%%%%%%%%%%%%%%

\leconInfo{
    \niv{cycle 3}{Symetrie axiale}
    \niv{5e}
    \niv{3e}
}{
    \item Prerequis
}

\df{Transformation géométrique}
Une \textbf{transformation géométrique} est 
une bijection d'une partie d'un ensemble géométrique dans lui-même.

\df{Similitude}
Une \textbf{similitude} est une transformation 
qui multiplie toutes les distances par une constante fixe, appelée son \textbf{rapport}.\\
L'image de toute figure par une similitude est une \textbf{figure semblable}\\
Une similitude est dite \textbf{directe} quand elle conserve l'orientation des figures,
\textbf{indirecte} quand elle inverse leur orientation.

\thm{}
Une transformation du plan est une similitude:
\begin{itemize}
    \item directe 
    \ssi{} son écriture complexe est de la forme:
    $z'=az+b$ avec $(a,b)\in\m{C^*}\times\m{C}$
    \item indirecte
    \ssi{} son écriture complexe est de la forme:
    $z'=a\overline{z}+b$ avec $(a,b)\in\m{C^*}\times\m{C}$
\end{itemize}

\demo{}\href{https://www.wikiwand.com/fr/Similitude_(g%C3%A9om%C3%A9trie)#Expression_complexe}{Wikipédia}

\df{Isométrie}
Une \textbf{isométrie} est une similitude qui conserve les mesures de longueurs et d'angles.

\df{Rotation plane}
Une \textbf{rotation plane} est une isométrie
qui fait tourner les figures autour d'un point et d'un certain angle.

\pr{Expression complexe d'un rotation}
La rotation de centre $R$ et d'angle $\Theta$ a pour expression complexe:
\begin{equation*}
    z'=e^{i\Theta}(z-z_r)+z_r
\end{equation*}

\df{Rotation affine}
Dans un espace affine euclidien orienté, une \textbf{rotation affine} est définie par la donnée d'un point $\Omega$
(le centre de la rotation, qui reste invariant par celle-ci) et d'une rotation vectorielle $r$ associée.
Si $P$ est un point de l'espace affine, son image par la rotation affine est le point $Q$ tel que:
$\vect{\Omega Q} = r(\vect{\Omega Q})$

\df{Symétrie centrale}
Dans le plan euclidien, les symétries centrales sont les rotations d'un demi-tour.
$\vect{\Omega Q} = -\vect{\Omega Q}$

\df{Symétrie axiale}
Une \textbf{symétrie axiale} ou \textbf{réflexion}, isométrie indirecte,
est une symétrie orthogonale par rapport à une droite.

\df{Translation}
Une \textbf{translation} est une isométrie directe 
qui correspond à l'idée intuitive de "glissement" d'un objet, sans rotation, retournement ni déformation de cet objet.

\pr{Expression complexe d'une translation}
La translation de vecteur $\vect{u}$ à pour expression complexe:
\begin{equation*}
    z' = z + t
\end{equation*}

\df{Homothétie}
Une \textbf{homothétie} est une similitude par agrandissement ou réduction.\\
Elle se caractérise par son centre, point invariant, et un rapport qui est un nombre réel.

\pr{Expression complexe d'une homothétie}
L'homothétie de centre $A$ et de rapport $k$ à pour expression complexe:
\begin{equation*}
    z'= k(z-a) + a
\end{equation*}
