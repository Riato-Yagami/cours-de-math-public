% VARIABLES %%%
\def\theme{Périmètres, aires, volumes}
% \def\date{date}
%%%%%%%%%%%%%%%

% DEFINITIONS %
\def\md{\ensuremath{\meter^2}}
\def\mt{\ensuremath{\meter^3}}
\def\Lt{\ensuremath{\liter}}

\def\Per{\ensuremath{\mathcal{P}}}
\def\A{\ensuremath{\mathcal{A}}}
\def\V{\ensuremath{\mathcal{V}}}
\def\B{\ensuremath{\mathcal{B}}}
\def\C{\ensuremath{\mathcal{C}}}

\def\u{\vect{u}}
\def\v{\vect{v}}

\def\cbmath{\href{https://docplayer.fr/210413521-Lecon-n-o-18-perimetres-aires-volumes-capes-session-cbmaths-cbmaths-fr-derniere-mise-a-jour-2-avril-2021.html}{CBMaths.fr}}
\def\jean{\href{https://www.mathenjeans.fr/sites/default/files/comptes-rendus/polygones_-_saint_dominique_nancy.pdf}{mathenjeans.fr}}

\newcommand{\drawArea}[1][blue]{\fill[fill = #1, opacity = 0.2]}
%%%%%%%%%%%%%%

\setboolean{subsectionInOutline}{true}
% \setboolean{outline}{true}
% \setboolean{demonstration}{false}

% \setboolean{showRef}{false}

\hbox{}
\leconInfo{
    \ilink{Cycle 3}
    \ilink{Cycle 4}
    \ilink{Terminale Option Spécialité}
    \ilink{Terminale Option Complémentaire}
    \ilink{Terminale Technologique}
} % Niveaux
[\item mesures de longueurs
\item transformation géométrique
\item continuité] % Prérequis
[] % Thèmes
[] % Motivation

\section{Mesures du plan}
\subsection{Périmètre}
\subsection{Découpages et recollements}
% \href{https://wimsauto.universite-paris-saclay.fr/wims/wims.cgi?session=8W0948C5F7.2&+lang=fr&+module=U1\%2Fgeometry\%2Fdocbolyai.fr&+cmd=reply&+job=read&+doc=1&+block=rectanglecarre2}{Université Paris-Saclay}
\subsection{Mesures d'aires}
\subsection{Formules d'aires}
\subsection{Théorèmes}

\pr{partage de la Médiane}{
    Une médiane d'un triangle partage ce triangle en deux triangles d'aires égales. 
}[\href{https://www.apmep.fr/IMG/pdf/S\_e9rie\_1-4_D\_e9montrer\_avec\_des\_aires\_-\_Copie.pdf}{Mathématique et mathématiciens}]

\demo{}{
    \dividePage{
        \begin{center}
            \begin{tikzpicture}[scale=1]
                % Définir les points du triangle
                \tkzDefPoint(0,0){A}
                \tkzDefPoint(4,0){B}
                \tkzDefPoint(1.5,2){C}
                \tkzDefMidPoint(A,B) \tkzGetPoint{D}
                % Ajouter des étiquettes
                \tkzLabelPoints[below left](A)
                \tkzLabelPoints[below right](B)
                \tkzLabelPoints[above](C)
                % Marquer le point médian
                \tkzDrawPoints(D)
                \tkzLabelPoints[below](D)
                % Calculer le pied de la hauteur depuis C sur AB
                \tkzDefPointBy[projection=onto A--B](C) \tkzGetPoint{H}
                \tkzLabelPoints[below](H)
                \drawArea (A) -- (C) -- (D) -- cycle;
                \drawArea[red] (C) -- (D) -- (B) -- cycle;
                \draw[very thick] (A) -- (B) -- (C) -- cycle;
                \draw[dotted, very thick] (C) -- (D);
                \draw[dashed, very thick] (C) -- (H);
                \tkzMarkSegment[blue,pos=.5,mark=|](A,D)
                \tkzMarkSegment[blue,pos=.5,mark=|](D,B)
                \tkzMarkRightAngle[blue](C,H,A)
            \end{tikzpicture}
        \end{center}
    }{
        \begin{align*}
            \color{Blue}\A\color{black} &= \dfrac{AD \times CH }{2}\\
            &= \dfrac{BD \times CH }{2} = \color{Red}\A\color{black}
        \end{align*}
    }
}

% \pr{des proportions}{

% }

% \thm{du chevron}{
%     Soit $N$ point dans intérieur au triangle $ABC$ et $A'$ intersection des droites $(AN)$ et $(BC)$.\\
%     On a :\begin{align*}
%         \dfrac{\A_{ANB}}{\A_{ANC}} = \dfrac{BA'}{CA'}
%     \end{align*}
% }

% \demo{}{
%     \TODO
% }
\subsection{Dichotomie périmètre et aire}

\section{Le cercle et le disque}
\subsection{La Circonférence}
\subsection{Aire du disque}

% \act{Approximation aire du disque}{}

\section{Mesures de l'espace}
\subsection{Aire latérale}
\subsection{Contenance}
\subsection{Volume}
\subsection{Cube et pavé droit}
\subsection{Prisme et cylindre}
\subsection{Pyramide et cône}
\subsection{Sphère}

\section{Aire sous la courbe}

\section{Déterminants dans le plan}