\def\theme{Correction DM - Puissances}
\def\date{26/10/2023}
\def\authors{
    PESIN - CADOT - COURTIN
}

\def\km{\kilo\meter}
\def\kmps{\kilo\meter\per\second}
\ex{7}{
    On cherche à calculer la distance
    $d$ parcourue par la lumière en une année.\\
    On sait que la vitesse de la lumière est:
    \begin{align*}
        v &= \numprint{300 000} \kmps\\
        &= 3 \times 10^5 \kmps
    \end{align*}
    On va utiliser la formule: $d=v \times t$ avec $t$ la durée d'une année.\\
    On cherche alors le temps en seconde dans une année:
    \begin{align*}
        t &= 60 \times 60 \times 24 \times 365\\
        &= \numprint{31 536 000} \second\\
        &= \numprint{3,1536} \times 10^7 \second
    \end{align*}
    Car il y a:
    \begin{itemize}
        \item $60$ secondes dans $1$ minute.
        \item $60$ minutes dans $1$ heure.
        \item $24$ heures dans $1$ journée.
        \item $365$ jours dans $1$ année. 
        % (on pourrait également prendre $365,25$ jours dans $1$ année pour être plus précis.)
    \end{itemize}
    On a alors:
    \begin{align*}
        d &= v \times t\\
        &= 3 \times 10^5 \times \numprint{3,1536} \times 10^7\\
        &= 3 \times \numprint{3,1536} \times 10^{5+7}\\
        &= \numprint{9,4608} \times 10^{12} \km
    \end{align*}
    La lumière parcourt donc $\numprint{9,4608} \times 10^{12} \km$ en une année.
}