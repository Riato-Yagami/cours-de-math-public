% VARIABLES %%%
\def\authors{CADOT - COURTIN - PESIN}
% \date{\today}
\def\theme{Myriade 2016 3e 63 p 151}
\thispagestyle{empty}
%%

% \img{images/stage-M2/3e/proportionnalite/exo_63_p151_myr_3e_2016.png}[10cm]

\exo{Myriade 63 p151}{%
    On cherche à comparer la rentabilité des deux emplacements; plage ou centre-ville.\\
    Peio établira son commerce du $1^{er}$ juin au $31$ aout inclus,
    c'est-à-dire : $30+31+31=\pcalc{30+31+31}$ jours.\\
    D'après l'information 2, le soleil brillera $75\%$ du temps,
    c'est-à-dire: $92 \times \dfrac{75}{100} = \pcalc{92*75/100}$ jours,
    et il y aura donc $92-69 = \pcalc{92-69}$ jours nuageux ou pluvieux.\\
    L'information 3 donne  l'estimation du chiffre d'affaires par jour selon la météo pour les deux locations.
    On peut alors estimer le chiffre d'affaires total:
    \begin{itemize}
        \item plage : $500 \times 69 + 50 \times 23 = \pcalc{500*69+50*23}\euro$
        \item centre-ville : $350 \times 69 + 300 \times 23 = \pcalc{350*69+300*23}\euro$
    \end{itemize}
    Le coût de l'emplacement peut être calculé à l'aide de l'information 1:
    \begin{itemize}
        \item plage : $2500 \times 3 = \pcalc{2500*3}\euro$
        \item centre-ville : $60 \times 92 = \pcalc{60*92}\euro$
    \end{itemize}
    En ôtant le coût de l'emplacement au chiffre d'affaires,
    on obtient alors le bénéfice total:
    \begin{itemize}
        \item plage : $35650 - 7500 = \pcalc{35650-7500}\euro$
        \item centre-ville : $31050 - 5520 = \pcalc{31050-5520}\euro$
    \end{itemize}
    Si Peio veut optimiser son bénéfice total alors il devrait choisir de s'établir sur la plage.
}