% VARIABLES %%%
\def\authors{CADOT - COURTIN - PESIN}
% \date{\today}
\def\longTitle{La Proportionnalité}
\def\shortTitle{Proportionnalité}
%%

% \disableAnimation
% \shortAnimation
\firstSlide

\def\imgPath{stage-M2/4e/proportionnalite/exercices/}
\def\imgExtension{_myriade_4e_2016.png}

\exoList{12 p139,20 p140,16 p138}[Myriade 4e 2016][3]

\exoslide{ex_12_p139}
\exoslide{ex_20_p140}[8cm]
\exoslide{ex_2_p138}[9cm]

\bsec{Rappel}
\bsubsec{Représentation Graphique}

\slide{Cours}{
    \bchap{\longTitle}
    \ssec\ssubsec

    \pr{Représentation graphique}{
        Un graphique présente une situation de proportionnalité si l'ensemble de ces points sont alignés avec l'origine.
    } 
}

\slide{}{
    \expl{}{
        \img{stage-M2/4e/proportionnalite/graphique_situation_proportionnelle.png}[8cm]
        Le graphique ci-dessus présente une situation de proportionnalité.
    }
}

\bsubsec{Produit en croix}
\slide{}{
    \ssubsec

    \pr{Egalité des produits en croix}{
        Dans le tableau de proportionnalité:
        \begin{center}
            \propTable{a}{c}{b}{d}
        \end{center}
        On a l'égalité: $a \times d = b \times c$. 
    }
    \vspace*{-0.5cm}
    \rmk{Egalité des quotients}{
        On a aussi:%
        \vspace*{-0.5cm}
        \begin{align*}%
            \dfrac{a}{b} = \dfrac{c}{d}%
        \end{align*}%
    }
}

\exoList{7 p143,2 p142,13 p144}[Myriade 3e 2016 - à terminer pour vendredi][3]

\def\imgPath{stage-M2/3e/proportionnalite/}
\def\imgExtension{_myr_3e_2016.png}

\exoslide{exo_7_p143}[5.4cm]
\exoslide{exo_2_p142}[7cm]
\exoslide{exo_13_p144}

\bsec{Pourcentage}
\bsubsec{Pourcentage d'un nombre}
\slide{Cours}{
    \ssec\ssubsec
    \df{Pourcentage}{
        Calculer $t\%$ d'un nombre revient à multiplier ce nombre par $\dfrac{t}{100}$.
    }
    
    \expl{}{
        Calculer $30\%$ de $60$.
    }
}

\bsubsec{Pourcentage d'évolution}

\slide{}{
    \pr{Pourcentage d'évolution}{
        \begin{itemize}
            \item Augmenter un nombre de $t\%$ revient à le multiplier par $1+\dfrac{t}{100}$.
            \item Diminuer un nombre de $t\%$ revient à le multiplier par $1-\dfrac{t}{100}$.
        \end{itemize}
    }

    \rmk{}{
        $1-\dfrac{t}{100}$ et $1+\dfrac{t}{100}$ sont appelés les \textbf{coefficients multiplicateurs}.
    }
}

\slide{}{
    \expl{Myriade}{
        \vspace{-0.25cm}
        \multiColItemize{3}{%
            \item 14 p144
            \item 15 p144
            \item 17 p144
            \item 24 p145
        }%
    }
}

\exoslide{exo_14_p144}
\exoslide{exo_15_p144}
\exoslide{exo_17_p144}
\exoslide{exo_24_p145}

\exoList{60 p151,63 p151,16 p144}[Devoir-Maison - Lundi 25 mars][3]

\qf{
    {$70\euro$ diminué de $10\%$, $63\euro$},
    {$50\euro$ augmenté de $20\%$, $\pcalc{50*120/100}\euro$}%
}[15]

\bsec{Ratio}
\bsubsec{Définition}

\slide{COURS}{
    \ssec\ssubsec

    Un ratio est une façon d'exprimer comment deux grandeurs (ou plus) se comparent.
    
    \df{}{
        On dit que deux nombres $a$ et $b$ sont dans le ratio $3:4$ (notation) si:
        \begin{align*}
            \dfrac{a}{3} = \dfrac{b}{4}
        \end{align*}
    }  
}

\slide{}{
    \expl{}{
        \begin{enumerate}
            \item 2 personnes se sont réparties des chocolats dans le ratio $3:4$,
            la première personne à 6 chocolats.\\
            Combien en a la deuxième personne ?
            \item 3 personnes se répartissent du jus d'orange dans le ratio $3:5:8$,
            la seconde personne recupère $15\centi\liter$.\\
            Combien y avait-il de jus d'orange au départ?
            \item \begin{enumerate} \item 2 personnes souhaitent se répartir $90\euro$ selon leur temps de travail,
            dans le ratio $4:5$.\\
            Combien chacune d'elles va-t-elle avoir?
            \item $90\euro$ dans le ratio $1:2:2$
            \end{enumerate}
        \end{enumerate}
    }  
}

\bsubsec{Echelle}

\slide{}{
    \ssubsec
    \df{Echelle}{
        Une carte à l'échelle $1:\np{1000}$ signifie que:
        $1u$ sur la carte représente $\np{1000}u$ dans la réalité.
        ($u$ : unité de longueur)
    }
}

\slide{}{
    \expl{}{
        On mesure sur une carte à l'aide de l'échelle graphique que $1\cm$ représente $1\km$.\\
        Quelle est l'échelle de la carte?
        \img{stage-M2/4e/proportionnalite/carte-paris.jpg}[6cm]
        \palt{2}{
            $1\km = \np{100 000}\cm$\\
            On alors une échelle de $1:\np{100 000}$
        }
    }
}

\exoList{4 p142, 6 p142}[Ratios][3]

\exoslide{exo_4_p142}
\exoslide{exo_6_p142}

\bsec{Grandeur composé}

\bsubsec{Grandeur quotient}

\slide{}{
    \ssec\ssubsec

    \df{}{
        Une \textbf{grandeur quotient} est une grandeur obtenue en effectuant le quotient de deux grandeurs.
    }

    \expl{}{\vspace*{-0.85cm}
        \begin{equation*}
            Vitesse = \palt{4}{\dfrac{Distance}{Temps}}
        \end{equation*}
    }
}

\slide{}{
    Une voiture parcours $120\km$ en $1\hour$.
    En supposant sa vitesse constante;
    combien parcourt-elle de $\km$ en $1\hour18$?

    \rmdr{Conversion}{
        \vspace*{-0.7cm}
        \begin{align*}
            60\min &= \palt{2}{1,0\hour}\\
            \alors 1\min &= \palt{4}{\dfrac{1}{60}\hour \approx 0,01667\hour}\\
            \alors 18\min &= \palt{5}{\dfrac{18}{60}\hour = 0,3\hour}\\
            \et 1\hour18\min &= \palt{5}{1,3\hour}
        \end{align*}
    }
}

\bsubsec{Grandeur produit}

\slide{}{
    \df{}{
        Une \textbf{grandeur produit} est une grandeur obtenue en effectuant le produit de deux grandeurs.
    }

    \expl{}{
        \begin{itemize}
            \item énergie (Wh) $=$ puissance $\times$ durée\\
            Un appareil d'une puissance de $100$W utilisé pendant $3$h consomme ainsi \palt{2}{$300$Wh}.
            \item volume ($\meter^3$) $=$ longueur $\times$ longueur $\times$ longueur
        \end{itemize}
    }
}

\exoList{51 p149, 36 p147, 37p147, 31p146}[Grandeur produit et quotient][4]

\exoslide{exo_51_p149}
\exoslide{exo_36_p147}
\exoslide{exo_37_p147}[7.5cm]
\exoslide{exo_31_p146}

\exoList{73p153, 67p153}[à finir pour la rentré]

\exoslide{exo_73_p153}[4.2cm]
\exoslide{exo_67_p153}

\exoList{44 p 148, 45p148, 50p149, 57p150, 61p151}[]

\exoslide{exo_44_p148}[6cm]
\exoslide{exo_45_p148}[5cm]
\exoslide{exo_50_p149}[8cm]
\exoslide{exo_57_p150}[5cm]
\exoslide{exo_61_p151}[8cm]