% VARIABLES %%%
\def\authors{CADOT - COURTIN - PESIN}
% \date{\today}
\def\longTitle{Les Transformations}
\def\shortTitle{Transformations}
\def\theme{\longTitle}
%%

% \disableAnimation
% \shortAnimation
% \firstSlide

\def\imgPath{stage-M2/4e/translation/decouverte-translation/}
\def\imgExtension{.png}

\newcounter{qfs}

\def\qfs{\stepcounter{qfs}\color{Blue}\alph{qfs}) \color{black}}
\def\sqf{\setcounter{qfs}{0} \stepcounter{qf}\color{blue}\arabic{qf}. \color{black}}

\def\deg{\ensuremath{^\circ}}

% \qfSUB{}{}
\slide{QUESTIONS FLASH}{%
    \sqf \qfs S'agit il d'une symétrie axiale?
    \imgp{exo2_symetrie_axiale_1}[5cm]
}

\slide{}{%
    \qfs S'agit il d'une symétrie axiale?
    \imgp{exo2_symetrie_axiale_2}[5cm]
}

\slide{}{%
    \qfs S'agit il d'une symétrie axiale?
    \imgp{exo2_symetrie_axiale_3}[5cm]
}

\slide{}{%
    \qfs S'agit il d'une symétrie axiale?
    \imgp{exo2_symetrie_axiale_4}[5cm]
}

\slide{}{%
    \sqf \qfs S'agit il d'une symétrie centrale?
    \imgp{exo2_symetrie_centrale_1}[5cm]
}

\slide{}{%
    \qfs S'agit il d'une symétrie centrale?
    \imgp{exo2_symetrie_centrale_2}[5cm]
}

\slide{}{%
    \qfs S'agit il d'une symétrie centrale?
    \imgp{exo2_symetrie_centrale_3}[5cm]
}

\slide{}{%
    \qfs S'agit il d'une symétrie centrale?
    \imgp{exo2_symetrie_centrale_4}[5cm]
}

\slide{}{%
    \sqf \qfs S'agit d'une translation?
    \imgp{exob_translation_1}[5cm]
}

\slide{}{%
    \qfs S'agit il d'une translation?
    \imgp{exob_translation_2}[5cm]
}

\slide{}{%
    \qfs S'agit il d'une translation?
    \imgp{exob_translation_3}[5cm]
}

\slide{}{%
    \qfs S'agit il d'une translation?
    \imgp{exob_translation_4}[5cm]
}

\slide{}{
    \begin{enumerate}
        \item S'agit il d'une symétrie axiale ?\\
        \begin{tabular}{c c c c}
            \includegraphics[width=2cm]{\imgf{exo2_symetrie_axiale_1}}&
            \includegraphics[width=2cm]{\imgf{exo2_symetrie_axiale_2}}&
            \includegraphics[width=2cm]{\imgf{exo2_symetrie_axiale_3}}&
            \includegraphics[width=2cm]{\imgf{exo2_symetrie_axiale_4}}
        \end{tabular}
        \item S'agit il d'une symétrie centrale ?\\
        \begin{tabular}{c c c c}
            \includegraphics[width=1.5cm]{\imgf{exo2_symetrie_centrale_1}}&
            \includegraphics[width=2.5cm]{\imgf{exo2_symetrie_centrale_2}}&
            \includegraphics[width=2.5cm]{\imgf{exo2_symetrie_centrale_3}}&
            \includegraphics[width=2.5cm]{\imgf{exo2_symetrie_centrale_4}}
        \end{tabular}
        \item S'agit il d'une translation ?\\
        \begin{tabular}{c c c c}
            \includegraphics[width=2.5cm]{\imgf{exob_translation_1}}&
            \includegraphics[width=2.5cm]{\imgf{exob_translation_2}}&
            \includegraphics[width=2.5cm]{\imgf{exob_translation_3}}&
            \includegraphics[width=2.5cm]{\imgf{exob_translation_4}}
        \end{tabular}
    \end{enumerate}
}

\slide{}{
    \exo{Décrire les transformations suivantes}{
        \begin{columns}[T] % align columns
            \begin{column}{.3\textwidth}
                \imgp{symetrie_axiale}[3.5cm]
            \end{column}%
            \hfill%
            \begin{column}{.3\textwidth}
                \imgp{symetrie_centrale}
            \end{column}%
            \hfill%
            \begin{column}{.3\textwidth}
                \imgp{translation_goose}
            \end{column}%
        \end{columns}
    }
}

\def\imgPath{stage-M2/3e/transformations/}

\slide{EXERCICES}{
    \vspace*{-0.5cm}
    \exo{Recopier la figure et tracer les transformations suivantes}{%
        \vspace*{-1cm}
        \begin{columns}[T] % align columns
            \begin{column}{.6\textwidth}
                \small
                \begin{enumerate}
                    \item $\mathcal{F}_2$ est l'image de $\mathcal{F}_1$ par la symétrie axiale d'axe $(CG)$.
                    \item $\mathcal{F}_3$ est l'image de $\mathcal{F}_1$ par la translation qui transforme $A$ en $J$. 
                    \item $\mathcal{F}_4$ est l'image de $\mathcal{F}_1$ par la symétrie centrale de centre $A$.
                    % \item azef
                    % \item zef
                \end{enumerate}
            \end{column}%
            \hfill%
            \begin{column}{.45\textwidth}
                \imgp{activite_1}
            \end{column}%
        \end{columns}
    }
}

\bsec{Rappels}

\def\width{6cm}
\newcommand*{\imgt}[1]{
    \includegraphics[width=\width]{\imgf{#1}}
}

\def\imgExtension{_myr_3e_2016.png}

\bsubsec{Symétrie Axiale}
\slide{Cours}{
    \bseq{\longTitle}
    \ssec\ssubsec
    \begin{tabular}{cc}
        \palt{2}{\imgt{symetrie_axiale_def}}&
        \imgt{symetrie_axiale_expl}
    \end{tabular}
}

\bsubsec{Symétrie Centrale}
\slide{}{
    \ssubsec
    \begin{tabular}{cc}
        \palt{2}{\imgt{symetrie_centrale_def}}&
        \imgt{symetrie_centrale_expl}
    \end{tabular}
}

\bsubsec{Translation}
\slide{}{
    \ssubsec
    \begin{tabular}{cc}
        \palt{2}{\imgt{translation_def}}&
        \imgt{translation_expl}
    \end{tabular}
}

\def\imgExtension{_myr_4e_2016.png}

\exoslide{activite_3p179}

\bsec{Rotation}
\bsubsec{Définition}
\slide{}{
    \ssec\ssubsec
    \rmk{}{
        Lorsque l'on décrit une rotation il faut préciser le \textbf{sens} horaire (sens des aiguilles d'une montre) ou anti-horaire (sens inverse des aiguilles d'une montre).
    }
}

\exoslide{exo_13p184}

\slide{COURS}{
    \df{Rotation}{
        Transformer un point ou une figure par rotation,
        c'est faire tourner ce point ou cette figure par rapport à un \textbf{centre de rotation},
        un \textbf{angle} et un \textbf{sens}.
    }
    \vspace*{-0.75cm}
    \expl{}{
        \begin{columns}[T] % align columns
            \begin{column}{.1\textwidth}
                \vspace*{-0.5cm}
                \imgp{expl_rotation}[1.7cm]
            \end{column}%
            % \hfill%
            \begin{column}{.8\textwidth}
                \small
                Le triangle $A'B'C'$ est l'image du triangle $ABC$ par la rotation de centre $O$ et d'angle $60\deg$ dans le sens anti-horaire.
            \end{column}%
        \end{columns}
    }
}

\exoList{17, 20, 12}[sur pronote]
% \exoList{13 p184}[Construction]

\exoslide{exo_17p185}[5cm]

\exoslide{exo_20p185}[6cm]

\exoslide{exo_12p184}

\bsubsec{Construction}
\slide{COURS}{
    \ssubsec
    \vspace*{-0.5cm}
    \begin{columns}[T]
        \begin{column}{.75\textwidth}
            \mthd{Construction rotation}{\small%
                Pour construire $A'$ l'image de $A$, par la rotation de centre $O$,
                d'angle $60^{\circ}$ dans le sens anti-horaire.
                \begin{enumerate}
                    \item On trace la demi-droite $[OA)$.
                    \item Avec le rapporteur on trace une demi-droite passant par $O$,
                    tel que l'angle entre les deux demi-droites mesure $60^{\circ}$ dans le sens-antihoraire.
                    \item On repporte la longueur $OA$ sur la nouvelle demi-droite pour placer $A'$.
                \end{enumerate}
            }
        \end{column}
        \begin{column}{.25\textwidth}
            \vspace*{2cm}
            % \hspace*{-1cm}
            \ctikz{
                \tkzDefPoint(0,0){O}
                \tkzDefPoint(3,1){A}
                \tkzDrawPoints(O)
                \tkzDrawPoints(A)
                \tkzLabelPoints[below](O)
                \tkzLabelPoints[below](A)
            }
        \end{column}
    \end{columns}
}

\slide{}{
    \exo{}{
        Recopier $D$ et $ABC$ puis construire son image $A'B'C'$ par la rotation d'angle $140\deg$ dans le sens anti-horaire par rapport à $D$.
        \begin{flushright}
            \begin{tikzpicture}[scale=0.8]
                \tkzDefPoint(-2,1){D}
                \tkzDefPoint(0,0){A}
                \tkzDefPoint(3,0){B}
                \tkzDefPoint(3,4){C}
                \tkzLabelPoints[left](A)
                \tkzLabelPoints[below](B)
                \tkzLabelPoints[above](C)
                \tkzLabelSegment[below](A,B){$3\cm$}
                \tkzLabelSegment[right](B,C){$4\cm$}
                \tkzLabelSegment[above left](A,C){$5\cm$}
                \tkzMarkRightAngle[blue](A,B,C)
                \tkzDrawPoints(D)
                \tkzLabelPoints[below](D)
                \draw[very thick] (A) -- (B) -- (C) -- cycle;
            \end{tikzpicture}
        \end{flushright}
    }
}

\slide{}{
    \pr{}{
        La rotation conserve les mesures d'angle, de longueurs et d'aires.
    }
}

\def\imgExtension{.png}

\slide{}{
    \begin{columns}[T]
        \begin{column}{.4\textwidth}
            \imgp{exo_reconnaitre}[6.1cm]
        \end{column}
        \begin{column}{.6\textwidth}
            \small
            Compléter les phrases :
            \begin{itemize}
                \item $\mathcal{F}_2$ est l'image de $\mathcal{F}_1$ par la rotation \dottedLines{1}
                \item $\mathcal{T}_2$ est l'image de $\mathcal{T}_1$ par la rotation \dottedLines{1}
                \item $\mathcal{Q}_2$ est l'image de $\mathcal{Q}_1$ par la rotation \dottedLines{1}
            \end{itemize}
        \end{column}
    \end{columns}
}

\slide{}{
    \begin{columns}[T]
        \begin{column}{.4\textwidth}
            \imgp{exo_rotation_carreaux}[5.5cm]
        \end{column}
        \begin{column}{.6\textwidth}
            \vspace*{2cm}
            Recopier $DEFG$ puis tracer $D'E'F'G'$ son image par la rotation de centre $P$,
            d'angle $90\deg$ et
            de sens horaire.
        \end{column}
    \end{columns}
}

\exoslide{exo_rotation_papier_blanc}[6cm]

\bsec{Homothétie}
\bsubsec{Définition}

\slide{}{
    \ssec\ssubsec
    \df{}{
        Transformer un point $A$ par homothétie de rapport $k$ et de centre $O$,
        c'est placer $A'$ sur la droite $(OA)$ tel que:
        \begin{itemize}
            \item si $k>0$, alors $A$ et $A'$ sont du même coté par rapport $O$ et $OA'$ = $k \times OA$
            \item si $k<0$, alors $A$ et $A'$ ne sont pas du même coté par rapport à $O$ et $OA'$ = $-k \times OA$
        \end{itemize}
    }
}

% \slide{}{
%     \rmk{}{
%         \begin{itemize}
%             \item si $k>1$,
%             \item si $k=1$
%             \item si $k=-1$
%             \item 
%         \end{itemize} 
%     }
% }

\exoslide{exo_31p189}[6cm]

\exoList{5 p185, 6p185, 10p185 (justifier chaque réponse par un dessin)}[DM pour Lundi 13 mai][2]

\exoslide{exo_5p185}[6cm]
\exoslide{exo_6p185}
\exoslide{exo_10p185}

\bsubsec{Construction}

% \exoslide{exo_23-27p188}

\slide{}{
    \ssubsec
    \mthd{Construction homothétie}{
        Pour construire $A'$ l'image de $A$,
        par l'homothétie de rapport $k$ et de centre $O$.
        \begin{enumerate}
            \item On trace la droite $(OA)$.
            \item On place $A'$ sur $(OA)$ à une distance $OA$ de $O$,
            du même coté que $A$ par rapport à $O$ si $k>0$,
            sinon du coté opposé.
        \end{enumerate}
    }
}

\exoList{14 p 186, 18 p 186}
% \exoslide{exo_14p186}
% \exoslide{exo_18p186}

\slide{}{
    \rmk{}{
        Pour une homothétie de rapport $k$;
        \begin{itemize}
            \item Les aires sont multiplié par $k^2$.
            \item Les volumes sont multiplié par $k^3$ si $k>0$ sinon par $-k^3$.
        \end{itemize}
    }
}

\exoList{40 p 189, 36 p 189 1. (prendre ABCD et K différent de l'illustration)}[Pour vendredi][2]
\exoList{21 p 187,21 p 187,45 p 191,ex. 4 Brevet Amérique du Nord 2023, ex. 3 Brevet Polynésie 2022}[Travail en groupe]