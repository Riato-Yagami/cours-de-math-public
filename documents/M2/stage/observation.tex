% VARIABLES %%%
\def\theme{Observation de Stage}
\def\date{13/11/2023}
\def\authors{\fsize{8pt}
    Jules PESIN - Antonin CADOT - Laurent COURTIN
}

Notre tutrice,
Anne Lambert,
exerce en tant que professeure de mathématiques au collège Janson de Sailly,
situé dans le 16e arrondissement de Paris.
Elle assure l'enseignement pour deux classes de troisième et deux classes de quatrième.

\section{Observations sur le travail en classe}

Elle entame ses séances en faisant corriger par un élève,
les exercices qui étaient à faire à la maison.
Elle affiche sur le tableau numérique le travail accompli par l'élève afin d'accélérer la correction,
puis lui demande de partager son raisonnement avec ses camarades tout en répondant à leurs éventuelles questions.
Pendant cette correction interactive,
elle en profite pour circuler entre les rangs et s'assurer que chaque élève a bien réalisé le travail demandé.
\\\\
Par la suite,
elle aborde le contenu du cours et propose des exercices à réaliser en classe.
Un élève volontaire se charge de corriger l'exercice,
puis un autre prend la relève pour le suivant.
Elle souligne l'importance pour le professeur de minimiser son intervention,
laissant ainsi la responsabilité du travail aux élèves.
\\\\
Elle donne toujours un ou deux exercices du manuel à faire chez soi en fin de séance.
Elle exige que les élèves le notent dans leur agenda et elle note le travail demandé sur Pronote.

\section{Remarques sur les préparations de cours}

Mme Lambert rédige l'intégralité du cours à domicile avant de le projeter en classes,
une méthode qui lui permet de gagner un temps précieux,
et permet de mettre plus rapidement les élèves au travail sur les activités prévues.

\section{Attentes et exigences scolaires de la tutrice vis-à-vis de ses élèves :}

Elle exige que tous les élèves aient fait leurs exercices à la maison,
elle commence toujours par vérifier élèves après élèves.
Les élèves doivent écrire le cours et prendre la correction après chaque exercice.
A la maison,
ils doivent apprendre leurs cours entre deux séances. En classe tout le monde doit travailler,
elle ne laisse personne attendre que les autres fassent.

\section{Règles de vie de classe}

Notre tutrice ne veut pas qu'il y ait de bruit,
même des chuchotements, elle ne laisse rien passer.
Excepté lorsqu'un élève a terminé un exercice,
il peut se lever et aider son camarade.
\\\\
Elle conçoit un plan de classe rigoureux qu'elle ajuste au fil de l'année afin d'écarter les groupes d'élèves dont le comportement perturbe la tranquillité du cours.
Lorsqu'elle divise la classe en demi-groupe,
le plan demeure inchangé mais les élèves sont tenus d'avancer jusqu'à ce que les premiers rangs soient complets.
Elle ne dispose que rarement de la classe dans une autre configuration qu'en autobus.

\section{Le déroulement de la première séance de l'année et la mise en place des règles de vie et de travail}

Elle nous a expliqué que lors des premières séances, elle pouvait être vraiment
«dure» pour que,
dès le début,
les élèves la respectent.
Elle nous a pris l'exemple d'une élève qui est arrivée sans sac lors du premier cours,
elle l'a incendié pour lui montrer tout de suite qu'ici on est là pour travailler.
\\\\
Cette approche s'étend sur le début de l'année,
où elle accorde une attention particulière au maintien du calme en classe et à l'engagement sérieux des élèves dans leurs études.
Une fois que ces attentes sont bien comprises par les élèves,
elle peut alors se permettre de relâcher un peu la pression au fil de l'année.