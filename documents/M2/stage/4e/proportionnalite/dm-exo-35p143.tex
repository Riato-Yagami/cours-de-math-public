\def\theme{Correction DM - Proportionnalité}
\def\date{04/01/2024}
\def\authors{
    PESIN - CADOT - COURTIN
}

\setboolean{answer}{true}

\exo{35p143}{
    \def\cW{3cm}
    % \def\rowHeight{1.5cm}
    On cherche à trouver la distance Nantes-Angers $d_{NA}$,
    Angers-Cholet $d_{AC}$,
    et Cholet-Nantes $d_{CN}$.\\
    On peut mesurer à la règle les distances qui séparent ces points sur la carte:
    \begin{align*}
        NA &= \nswr{5,3\cm}\\
        AC &= \nswr{3,3\cm}\\
        CN &= \nswr{3,5\cm}
    \end{align*}
    En utilisant une règle pour mesurer l'échelle graphique,
    on détermine que $0,7\cm$ sur la carte équivaut à $10\km$ dans le monde réel.\\
    On peut alors monter le tableau de proportionnalité suivant:
    \begin{center}
        \begin{tabular}{|M{\cW}|M{\cW}|M{\cW}|M{\cW}|}
            \hline
            $0.7\cm$ & $NA$ & $AC$ & $CN$ \\
            \hline
            $10\km$ & $d_{NA}$ & $d_{AC}$ & $d_{CN}$ \\
            \hline
        \end{tabular}
    \end{center}
    On a donc un coefficient de proportionnalité de $\frac{10\km}{0.7\cm}$,
    pour passer de la ligne 1 à la ligne 2 du tableau.\\
    On trouve alors en multipliant chaque mesure par le coefficient de proportionnalité:
    \begin{center}
        \begin{tabular}{|M{\cW}|M{\cW}|M{\cW}|M{\cW}|}
            \hline
            $0.7\cm$ & $5,3\cm$ & $3,3\cm$ & $3,5\cm$\\
            \hline
            $10\km$ & $\nswr{75,7\km}$ & $\nswr{47,1\km}$ & $\nswr{50,0\km}$\\
            \hline
        \end{tabular}
    \end{center}
    On a donc :
    \begin{align*}
        d_{NA} &= \nswr{75,7\km}\\
        d_{AC} &= \nswr{47,1\km}\\
        d_{CN} &= \nswr{50,0\km}
    \end{align*}
}
