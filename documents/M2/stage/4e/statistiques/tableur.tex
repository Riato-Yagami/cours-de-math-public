% VARIABLES %%%
\def\authors{PESIN - CADOT - COURTIN}
\def\theme{\large Formulaire tableur : Statistiques}
\def\imgPath{libre-office/}
\thispagestyle{small}
\def\cu{cel1}
\def\cd{cel2}
%%
\vspace*{-1cm}
\begin{itemize}
    \item une formule commence toujours par le signe «=»
    \item \textbf{NB(\cu : \cd)} donne le \textbf{nombre de valeurs} dans la plage \cu : \cd
    \item \textbf{SOMME(\cu : \cd)} donne la \textbf{somme des valeurs} dans la plage \cu : \cd
    \item \textbf{MEDIANE(\cu  : \cd)} donne la \textbf{médiane} dans la plage \cu : \cd
    \item \textbf{MIN(\cu : \cd)} donne la \textbf{plus petite valeur} dans la plage \cu : \cd
    \item \textbf{MAX(\cu : \cd)} donne la \textbf{plus grande valeur} dans la plage \cu : \cd
    \item Pour construire un \textbf{diagramme} pour la plage à \textbf{deux colonnes} \cu : \cd
    \begin{enumerate}
        \item sélectionner la plage \cu : \cd
        \item choisir : $\rightarrow$ Insertion $\rightarrow$ \icon{diagramme} Diagramme
        \item choisir : Étapes $\rightarrow$ 1. Type de diagramme  $\rightarrow$ \icon{colonne} Colonne (ou \icon{secteur} Secteur)
        \item cocher : Étapes $\rightarrow$ 2. Plage de données $\rightarrow$ Première colonne comme étiquette
    \end{enumerate}
\end{itemize}