% VARIABLES %%%
\def\authors{CADOT - COURTIN - PESIN}
% \date{\today}
\def\longTitle{Les Puissances}
\def\shortTitle{Puissances}
%%

\disableAnimation
% \shortAnimation
% \firstSlide

\def\imgPath{stage-M2/4e/puissance/}
\def\imgExtension{_myr_4e_2016.png}

\newcommand{\powq}[2]{\iquestion{\pow{#1}{#2}}{#1^{#2}}}
\newcommand{\powiq}[3]{\iquestion{#1^{#2}}{\pow{#1}{#2} = #3}}

\scn{Notation puissance}{}

\qf{
    {$\pow{10}{3}$, $= 1000$},
    {$6^2$, $= 36$},
    {$10^2$, $= 100$},
    {$\pow{2}{4}$, $= 16$},
    {$\pow{9}{2}$, $= 81$}%
}[10]

\slide{COURS}{
    \bchap{\longTitle}
    \act{Myriade 1p76}{
        \vspace{-1cm}
        \img{\imgf{activite_1_p76}}[9.7cm]
    }
}

\bsec{Puissance d'un nombre}

\slide{}{
    \ssec
    \df{}{
        Soient $n$ un entier supérieur ou égal à $1$ et $a$ un nombre relatif.
        \begin{align*}
            a^n = \palt{2}{
                \lPowBrace{a}{n}
            }
        \end{align*}
    }
}

\slide{}{
    \expl{}{
        \startQuestions
        Ecrire avec la notation puissance:
        \begin{enumerate}
            \powq{6}{5}
            \powq{(-5)}{3}
            \powq{0,2}{4}
            \iquestion{\pow{3}{2}\times \pow{2}{4}}{3^3 \times 2^4}
        \end{enumerate}
    }
}

\slide{}{
    \expl{}{
        \startQuestions
        % \fsize{13pt}
        Calculer:
        \begin{enumerate}
            \powiq{2}{5}{32}
            \powiq{(-3)}{2}{9}
            \iquestion{-3^2}{-3 \times -3 = -9}
            \iquestion{2000^1}{2000}
            \powiq{0}{12}{0}
            \powiq{1}{6}{1}
            \powiq{(\frac{2}{3})}{3}{\frac{8}{27}}
            \powiq{0,5}{2}{0,25}
        \end{enumerate}
    }
}

\slide{}{
    \expl{}{
        \startQuestions
        Donner le signe du resultat:
        \begin{enumerate}
            \item $(-1)^{12} \palt{2}{\geqslant 0}$
            \item $(-1)^{211} \palt{3}{\leqslant 0}$
            \item $(-879)^{75} \palt{4}{\leqslant 0}$
            \item $(-1984,6)^{2024} \palt{5}{\geqslant 0}$
        \end{enumerate}
    }

    % \rmk{}{
    %     Soit $a$ un nombre négatif et $n$ un entier,
    %     $a^n$ est\palt{2}{positif } si\palt{2}{$n$ est pair },
    %     et\palt{3}{négatif } si\palt{3}{$n$ est impair}.
    % }
}

\slide{}{
    \rmk{}{
        Soit $a$ un nombre négatif et $n$ un entier,
        $a^n$ est positif si $n$ est pair,
        et négatif si $n$ est impair.
    }
}

\exoList{2 p80,4 p80,6 p80}[Myriade p80]

\exoslide{exo_2_p80}
\exoslide{exo_4_p80}
\exoslide{exo_6_p80}

\scn{Première propriété sur les puissances de 10}{}

\qf{
    {Ecrire avec la notation puissance :\\$\pow{9}{6}$, $= 9^6$},
    {$\pow{0,15}{4}$, $= 0{,}15^4$},
    {Calculer :\\$2^4$, $= 16$},
    {$10^4$, $= 1000$}%
}[10]

\bsec{Puissances de 10}
\bsubsec{Propriétés}
\slide{COURS}{
    \ssec\ssubsec
    \vspace{-0.5cm}
    \act{Découverte}{
        \vspace{-0.7cm}
        \img{\imgf{activite_2_p76_1.a}}[10cm]
        % \vspace{-0.3cm}
        \small \color{BurntOrange} b. \color{black} Que remarque t-on?
        \vspace{-0.3cm}
        \img{\imgf{activite_2_p76_2.a}}[10cm]
        \small \color{BurntOrange} b. \color{black} Que remarque t-on?
    }
}

\slide{}{
    \def\tenPow{1\underbrace{0...0}_{n \textrm{ zéros}}}
    \pr{}{
        Avec $n$ un entier positif, les puissances de $10$ s'écrivent :
        \startQuestions
        \begin{itemize}
            \iquestion{10^n}{\tenPow}
            \iquestion{10^{-n}}{
                \frac{1}{10^n} 
                = \frac{1}{\tenPow} 
                = \underbrace{0,0..0}_{n \textrm{ zéros}}1
            }
        \end{itemize}
    }
}

\exoList{21 p80,22 p82,23 p82}[Myriade 4e 2016 p82]

\exoslide{exo_21_p82}
\exoslide{exo_22_p82}
\exoslide{exo_23_p82}

\scn{Propriétés de calcule sur les puissance de 10}{
    Etendu aux puissances de bases quelconques, mais non exigé
}

\qf{
    {$10^6$, $= \powTen{6}$},
    {$5^3$, $= 125$},
    {$10^{-1}$, $= \powTen{-1}$},
    {$10^2 \times 10^3$, $= \powTen{5}$},
    {$10^{-3}$, $= \powTen{-3}$}%
}[10]

\exoList{24 p82,26 p82,27 p82, 30p83}[Myriade 4e 2016 p82-83][2]

\exoslide{exo_24_p82}

\slide{}{
    \color{black} Soient $a$ et $b$ deux entiers.
    \rmk{}{
        \begin{equation*}
            10^a\times 10^b = \palt{2}{10^{a+b}}
        \end{equation*}
    }
}

\exoslide{exo_26_p82}

\slide{}{
    \rmk{}{
        \begin{equation*}
            \frac{10^a}{10^b} = \palt{2}{10^{a-b}}
        \end{equation*}
    }
}

\exoslide{exo_27_p82}

\slide{}{
    \rmk{}{
        \begin{equation*}
            (10^a)^b = \palt{2}{10^{a \times b}}
        \end{equation*}
    }
}

\slide{}{
    \pr{}{
        Pour $x$ un nombre et $a$ et $b$ deux entiers.
        \begin{align*}
            x^a\times x^b = x^{a+b}
            \iet
            \frac{x^a}{x^b} = x^{a-b}
            \iet 
            (x^a)^b = x^{a\times b}
        \end{align*}
    }

    \rmk{Convention}{
        Pour $x$ un nombre: $x^0 = 1$ 
    }
}

\slide{EXERCICES}{
    \exo{}{
        \multiColItemize{3}{
            \item 30 p83
        }
    }
    \exo{Devoir Maison}{
        \multiColItemize{3}{
            \item 29 p83 \item 15 p81 \item 16p81
        }
    }
}

\exoslide{exo_30_p83}

\scn{Préfixes et Notation scientifique}{}

\qf{
    {$\pow{3}{5}$, $= 3^5$},
    {$(\frac{1}{2})^3$, $= \frac{1}{8}$},
    {$10^6$, $= \powTen{6}$},
    {$\powTen{5}$, $= 10^5$},
    {$10^{-4}\times 10^{2}$, $= \powTen{-2}$}%
}

\bsubsec{Prefixes}

\slide{}{
    \ssubsec
    \def\cW{0.95cm}
    \begin{tabular}{|C{2.5cm}|C{\cW}|C{\cW}|C{\cW}|C{\cW}|C{\cW}|C{\cW}|}
        \hline
        Prefixes & giga & méga & kilo & milli & micro & nano \\
        \hline
        Symbole & G & M & k & m & $\mu$ & n \\
        \hline
        Signification & $10^9$ & $10^6$ & $10^3$ & $10^{-3}$ & $10^{-6}$ & $10^{-9}$ \\
        \hline
    \end{tabular}

    \expl{}{
        \begin{itemize}
            \item Un gigaoctet, noté Go, correspond à une quantité de données numériques de $10^9$ octets,
            soit un milliard d'octets.
            \item Un microgramme, noté $\mu$g, correspond à $10^{-6}$ grammes,
            soit un millionième de gramme.
        \end{itemize}
    }
}

\exoList{34 p83}[Myriade 4e 2016 p83][2]

\exoslide{exo_34_p83}

\bsec{Notations scientifiques}

\slide{COURS}{
    \ssec
    \act{Myriade 4p77}{
        \vspace{-0.7cm}
        \img{\imgf{activite_4_p77}}[9.7cm]
    }
}

\bsubsec{Définition}

\slide{}{
    \ssubsec
    \df{}{
        La notation scientifique d'un nombre est la seule écriture de la forme:
        \begin{equation*}
            \palt{2}{a \times 10^n}
        \end{equation*}
        avec $a$ un nombre entre 1 et 10 (exclu) et $n$ un entier.
    }
}

\slide{}{
    \expl{}{
        \startQuestions
        \begin{itemize}
            \iquestion{\num{632000}}{6,32 \times 10^5}
            \iquestion{0,0012}{1,2 \times 10^{-3}}
        \end{itemize}
    }
}

\exoList{41 p84,42 p84,52 p85}[Myriade 4e 2016 p84-85][2]

\exoslide{exo_41_p84}
\exoslide{exo_42_p84}
\exoslide{exo_52_p85}

\scn{Ordre de grandeur}{}

\qf{
    {Comparer avec le bon signe d'inégalité ($< \ou > \ou =$):\\ $10^6 \et 10^5$ , $10^6 > 10^5$},
    {$10^2\times10^{-3} \et 10^{-2}\times10^3$, $10^2\times10^{-3} < 10^{-2}\times10^3$},
    {$\frac{10^5}{10^2} \et \frac{10^5}{10^{-2}}$, $\frac{10^5}{10^2} < \frac{10^5}{10^{-2}}$},
    {$\frac{1}{(10^2)^3} \et 10^{-5}$, $\frac{1}{(10^2)^3} < {10^-5}$}%
}[10]

% \qf{
%     {Ecrire en notation scientifique:\\$\num{136000}$, $= 1{,}36\times10^5$},
%     {$\num{50224}$, $= 5{,}0224\times10^4$},
%     {$\num{0{,}002}$, $= 2\times10^{-3}$},
%     {$\num{0{,}00002334}$, $=2{,}334 \times 10^{-6}$}%
% }

\bsubsec{Ordre de grandeurs}

\slide{}{
    \df{}{
        L'\textbf{ordre de grandeur} d'un nombre est la puissance de 10 la plus proche de la grandeur étudiée.
    }
}

\slide{}{
    \expl{}{
        \begin{align*}
            A &= \num{32657000}\\
            \alors A &= \palt{2}{\num{3,2657} \times 10^7}\\
            \et \palt{3}{1 \times 10^7} < A &< \palt{3}{10\times 10^7}
        \end{align*}
        et \palt{4}{$\num{3,2657}$ est plus proche de 10 que de 1},
        alors l'ordre de grandeur de $A$ est \palt{5}{$10^7$}.
    }
}

\slide{}{
    \vspace*{-0.5cm}
    \begin{align*}
        B &= \num{732000}\\
        \alors B &= \palt{2}{\num{7,32} \times 10^5}\\
        \palt{2}{\et 1 \times 10^5 < B } &\palt{2}{< 10\times 10^5}
    \end{align*}
    \palt{2}{et $7,32$ est plus proche de 1 que de 10,
    alors l'ordre de grandeur de $B$ est $10\times 10^5 = 10^6$.}
    \begin{align*}
        C &= \num{0,000586}\\
        \alors C &= \palt{3}{\num{5,86} \times 10^{-4}}\\
        \palt{3}{\et 1 \times 10^{-4} < C} &\palt{3}{< 10\times 10^{-4}}
    \end{align*}
    \palt{3}{et $5,46$ est plus proche de 10 que de 1,
    alors l'ordre de grandeur de $C$ est $10\times 10^{-4} = 10^{-3}$.}
}

\exoList{53 p86,58 p86,68 p87}[Myriade 4e 2016 p86-87][2]

\exoslide{exo_53_p86}
\exoslide{exo_58_p86}
\exoslide{exo_68_p87}

\scn{Exercices d'entrainements sur tous le chapitre}{}

% \exoList{53 p86,58 p86,68 p87}[Myriade 4e 2016 p86-87][2]