% VARIABLES %%%
\def\theme{Une nouvelle transformation du plan}
\def\date{26/10/2023}
\def\authors{\fsize{8pt}
    PESIN - CADOT - COURTIN
}

\setboolean{answer}{false}
%%%%%%%%%%%%%%%

\section{Révisions}

\ex{Reconnaitre et décrire les transformations suivantes}{
    \begin{multicols}{2}
        \img{stage-M2/decouverte-translation/symetrie_axiale.png}[4cm][Image 1]
        \img{stage-M2/decouverte-translation/symetrie_centrale.png}[5.4cm][Image 2]
    \end{multicols}

    \nswr{
        \begin{enumerate}
            \item Dans l'image 1 
            la figure en bas à droite est l'image 
            par symétrie axiale d'axe $(d)$ 
            de la figure en haut à gauche.
            \item Dans l'image 2 
            la figure en bas à droite est l'image 
            par symétrie centrale de centre $C$ 
            de la figure en haut à gauche.
        \end{enumerate}
    }
}

\ex{Pour chaque cas, dire s'il s'agit de symétrie centrale par rapport à un point ou non}{
    \img{stage-M2/decouverte-translation/exo2_symetrie_centrale.png}[\linewidth]
    \nswr{
        % \def\cW{0.25\linewidth}
        % \begin{tabular}{p{\cW}p{\cW}p{\cW}p{\cW}}
        %     oui & non & non & oui \\
        % \end{tabular}
        a. oui b. non c. non d. oui
    }
}


\ex{Pour chaque cas, dire s'il s'agit de symétrie axiale par rapport à la droite $(d)$ ou non}{
    \begin{multicols}{4}
        \img{stage-M2/decouverte-translation/exo2_symetrie_axiale_1.png}
        \img{stage-M2/decouverte-translation/exo2_symetrie_axiale_2.png}
        \img{stage-M2/decouverte-translation/exo2_symetrie_axiale_3.png}
        \img{stage-M2/decouverte-translation/exo2_symetrie_axiale_4.png}
    \end{multicols}

    \nswr{
        a. non b. non c. oui d. oui
    }
}

\newpage
\section{Nouvelle Transformation}

\ex{Décrire une démarche qui a permis de passer de la figure f1 à la figure f2, de même de la figure f3 à figure f4}{
    \img{stage-M2/decouverte-translation/exo3_decrire_demarche.png}[10cm]
    \nswr{
        \begin{itemize}
            \item Pour passer de f1 à f2 
            on a fait glisser la figure de 
            3 carreaux vers la droite.
            \item Pour passer de f3 à f4 
            on a fait glisser la figure de 
            3 carreaux vers le bas.
        \end{itemize}
    }
}

\ex{Décrire une démarche qui a permis de passer de la figure $F_1$ à la figure $F_2$}{
    \img{stage-M2/decouverte-translation/translation_goose.png}[8cm]

    \nswr{
        Pour passer de $F_1$ à $F_2$ 
        on a fait glisser la figure 
        parallèlement à la flèche entre $A$ et $B$.\\
        % $F_1$ pour devenir $F_2$ à subit la meme transformation que $A$ pour devenir $B$.
    }
}