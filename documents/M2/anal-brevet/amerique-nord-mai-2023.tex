\documentclass[11pt]{article}
\usepackage[T1]{fontenc}
\usepackage[utf8]{inputenc}
\usepackage{fourier}
\usepackage[scaled=0.875]{helvet}
\renewcommand{\ttdefault}{lmtt}
\usepackage{amsmath,amssymb,makeidx}
\usepackage{fancybox}
\usepackage{tabularx}
\usepackage{graphicx}
\usepackage[normalem]{ulem}
\usepackage{pifont}
\usepackage{lscape}
\usepackage{multicol}
\usepackage{enumitem}
\usepackage{diagbox}
\usepackage{multirow}
\usepackage{scratch3} 
\usepackage{textcomp} 
\newcommand{\euro}{\eurologo{}}
\DeclareUnicodeCharacter{0301}{~}
%\usepackage[dvipsnames]{xcolor}
%\xdefinecolor{abricot}{named}{Apricot}
%Sujet aimablement fourni par Amélie Daniel
%Tapuscrit : Denis Vergès
\usepackage[dvipsnames]{pstricks}
\usepackage{pst-plot,pst-text,pst-tree,pstricks-add}
\newcommand{\R}{\mathbb{R}}
\newcommand{\N}{\mathbb{N}}
\newcommand{\D}{\mathbb{D}}
\newcommand{\Z}{\mathbb{Z}}
\newcommand{\Q}{\mathbb{Q}}
\newcommand{\C}{\mathbb{C}}
\usepackage[left=3.5cm, right=3.5cm, top=3cm, bottom=3cm,headheight=14pt]{geometry}
\newcommand{\vect}[1]{\overrightarrow{\,\mathstrut#1\,}}
\renewcommand{\theenumi}{\textbf{\arabic{enumi}}}
\renewcommand{\labelenumi}{\textbf{\theenumi.}}
\renewcommand{\theenumii}{\textbf{\alph{enumii}}}
\renewcommand{\labelenumii}{\textbf{\theenumii.}}
\def\Oij{$\left(\text{O},~\vect{\imath},~\vect{\jmath}\right)$}
\def\Oijk{$\left(\text{O},~\vect{\imath},~\vect{\jmath},~\vect{k}\right)$}
\def\Ouv{$\left(\text{O},~\vect{u},~\vect{v}\right)$}
\usepackage{fancyhdr}
%\usepackage[colorlinks=true,pdfstartview=FitV,linkcolor=blue,citecolor=blue,urlcolor=blue]{hyperref}
\usepackage[dvips]{hyperref}
\hypersetup{%
pdfauthor = {APMEP},
pdfsubject = {Brevet},
pdftitle = {Amérique du Nord 31 mai 2023},
allbordercolors = white,
pdfstartview=FitH}
\usepackage[french]{babel}
\usepackage[np]{numprint}
\begin{document}
\setlength\parindent{0mm}
\lhead{\small L'année 2023}
\rhead{\small \textbf{A. P{}. M. E. P{}.}}
\rfoot{\small Amérique du Nord }
\lfoot{\small 31 mai 2023}
\pagestyle{fancy}
\thispagestyle{empty}
\begin{center} {\huge \textbf{\decofourleft~Brevet Amérique du Nord 31 mai 2023 \decofourright}}
\end{center}

\bigskip

\textbf{Exercice 1 \hfill 20 points}

\medskip

Les cinq situations suivantes sont indépendantes.

\medskip

\textbf{Situation 1}

\medskip

Décomposer en produit de facteurs premiers le nombre $780$. 

Aucune justification n'est attendue.

\medskip

\textbf{Situation 2}

\medskip

On rappelle qu'un jeu de 32 cartes est composé de quatre familles (trèfle, carreau, cœur, pique).

Chaque famille est composée de huit cartes: 7, 8, 9, 10, valet, dame, roi et as.

L'expérience aléatoire consiste à tirer une carte au hasard dans ce jeu de 32 cartes.

\medskip

\textbf{a.~} Quelle est la probabilité d'obtenir le 8 de pique ? 

Aucune justification n'est attendue.

\textbf{b.~} Quelle est la probabilité d'obtenir un roi ou un cœur ? 

Aucune justification n'est attendue.

\medskip

\textbf{Situation 3}

\medskip

Développer et réduire l'expression  $A = (2x + 5)(3x - 4)$.

\medskip

\textbf{Situation 4}

\medskip

\begin{minipage}{0.48\linewidth}
\textbf{a.~} Quel est le volume, en cm$^3$, de ce prisme droit ?

\textbf{b.~} Convertir ce résultat en litre.

Rappel: 1 L = 1 dm$^3$.

\end{minipage}\hfill
\begin{minipage}{0.48\linewidth}
\psset{unit=0.9cm,arrowsize=2pt 3}
\begin{pspicture}(8.3,5.5)
\pspolygon(0.7,0.7)(5.7,0.7)(0.7,2.9)
\psline(5.7,0.7)(7.9,2.8)(2.9,5)(0.7,2.9)
\psline[linestyle=dashed](0.7,0.7)(2.9,2.8)(2.9,5)
\psline[linestyle=dashed](2.9,2.8)(7.9,2.8)
\psframe(2.9,2.8)(3.1,3)\psframe(0.7,0.7)(0.9,0.9)
\psline[linewidth=0.6pt]{<->}(0.7,0.5)(5.7,0.5)\uput[d](3.1,0.5){80 cm}
\psline[linewidth=0.6pt]{<->}(0.5,0.7)(0.5,2.9)\rput{90}(0.3,1.8){60 cm}
\psline[linewidth=0.6pt]{<->}(5.9,0.7)(8.1,2.8)\rput{46}(7,1.5){120 cm}
\psline[linewidth=0.6pt]{<->}(7.9,3)(2.9,5.2)\rput{-25}(5.4,4.4){100 cm}
\end{pspicture}
\end{minipage}

\medskip

\textbf{Situation 5}

\medskip

\begin{minipage}{0.48\linewidth}
Le polygone 2 est un agrandissement du polygone 1. 

Le coefficient de cet agrandissement est 3.

L'aire du polygone 1 est égale à 11 cm$^2$.

Quelle est l'aire du polygone 2 ?
\end{minipage}\hfill
\begin{minipage}{0.48\linewidth}
\begin{tabular}{|m{3.4cm} m{3.4cm}|}\hline
\multicolumn{2}{|m{6.8cm}|}{Représentation de la situation qui n'est pas à l'échelle:
}\\
\psset{unit=1cm}
\begin{pspicture}(1.3,1)
\pspolygon(0.4,0)(1.1,0.2)(0.9,0.7)(0.4,0.9)(0,0.4)
\end{pspicture}&\psset{unit=3cm}
\begin{pspicture}(1.3,1)
\pspolygon(0.4,0)(1.1,0.2)(0.9,0.7)(0.4,0.9)(0,0.4)
\end{pspicture}\\ 
Polygone 1&Polygone 2\\ \hline
\end{tabular}
\end{minipage}

\bigskip

\textbf{Exercice 2 \hfill 22 points}

\medskip

\begin{minipage}{0.48\linewidth}
On considère la figure ci-contre. On donne les mesures suivantes:

\begin{itemize}
\item[$\bullet~$] AN = 13 cm
\item[$\bullet~$] LN = 5 cm
\item[$\bullet~$] AL = 12 cm
\item[$\bullet~$] ON = 3 cm
\item[$\bullet~$] O appartient au segment [LN]
\item[$\bullet~$] H appartient au segment [NA]
\end{itemize}
\end{minipage}\hfill
\begin{minipage}{0.48\linewidth}
\psset{unit=1cm}
\begin{pspicture}(6.5,8.3)
\rput(3.25,8.1){Cette figure n'est pas à l'échelle. }
\pspolygon(1.9,0.4)(6,0.4)(1.9,7.6)%LNA
\psline(3.4,0.4)(3.4,5)%OH
\psframe(3.4,0.4)(3.6,0.6)
\uput[dl](1.9,0.4){L} \uput[dr](6,0.4){N} \uput[l](1.9,7.6){A} \uput[d](3.4,0.4){O} \uput[ur](3.4,5){H} 
\end{pspicture}
\end{minipage}


\begin{enumerate}
\item Montrer que le triangle LNA est rectangle en L.
\item Montrer que la longueur OH est égale à $7,2$~cm.
\item Calculer la mesure de l'angle $\widehat{\text{LNA}}$. Donner une valeur approchée à l'unité près. 
\item Pourquoi les triangles LNA et ONH sont-ils semblables ?
\item 
	\begin{enumerate}
		\item Quelle est l'aire du quadrilatère LOHA ?
		\item Quelle proportion de l'aire du triangle LNA représente l'aire du quadrilatère LOHA ?
	\end{enumerate}
\end{enumerate}

\bigskip

\textbf{Exercice 3 \hfill 20 points}

\medskip

\textbf{Les deux parties sont indépendantes}

\medskip

\textbf{Partie A : évolution du nombre de visiteurs sur un site touristique}

\medskip

\begin{enumerate}
\item Le diagramme ci-dessous représente le nombre de visiteurs par an de 2010 à 2021 sur ce site.

\begin{center}
\psset{xunit=0.95cm,yunit=0.000018cm}
\begin{pspicture}(-1.5,-70000)(12.5,450000)
\psaxes[linewidth=0pt,Dx=50,Dy=500000,labelFontSize=\scriptstyle](0,0)(0,0)(12.5,450000)
\psframe[fillstyle=solid,fillcolor=purple](0.7,0)(1.3,300000)\psframe[fillstyle=solid,fillcolor=purple](1.7,0)(2.3,310000)
\psframe[fillstyle=solid,fillcolor=purple](2.7,0)(3.3,320000)\psframe[fillstyle=solid,fillcolor=purple](3.7,0)(4.3,320000)
\psframe[fillstyle=solid,fillcolor=purple](4.7,0)(5.3,300000)\psframe[fillstyle=solid,fillcolor=purple](5.7,0)(6.3,320000)
\psframe[fillstyle=solid,fillcolor=purple](6.7,0)(7.3,330000)\psframe[fillstyle=solid,fillcolor=purple](7.7,0)(8.3,350000)
\psframe[fillstyle=solid,fillcolor=purple](8.7,0)(9.3,360000)\psframe[fillstyle=solid,fillcolor=purple](9.7,0)(10.3,400000)
\psframe[fillstyle=solid,fillcolor=purple](10.7,0)(11.3,187216)\psframe[fillstyle=solid,fillcolor=purple](11.7,0)(12.3,219042)
\uput[d](6.25,-50000){Année}\rput{90}(-1.5,225000){Nombre de visiteurs}
\rput(6.25,475000){Nombre de visiteurs sur le site touristique par année
}
\multido{\n=1+1,\na=2010+1}{12}{\uput[d](\n,0){\small \na}}
\multido{\n=0+50000}{10}{\psline[linewidth=0.6pt](0,\n)(12.5,\n)\uput[l](0,\n){\footnotesize \np{\n}}}
\end{pspicture}
\end{center}

	\begin{enumerate}
		\item Quel a été le nombre de visiteurs en 2010 ? Aucune justification n'est attendue.
		\item En quelle année le nombre de visiteurs a-t-il été le plus élevé ? Aucune justification n'est attendue.
	\end{enumerate}
\item Le tableau ci-dessous indique le nombre de visiteurs sur le site touristique de cette ville en 2020 et en 2021 :
\begin{center}
\begin{tabularx}{0.75\linewidth}{|l|*{2}{>{\centering \arraybackslash}X|}}\hline
Année 				&2020 		&2021\\ \hline
Nombre de visiteurs &\np{187216}&\np{219042}\\ \hline
\end{tabularx}
\end{center}

Le maire de cette ville avait pour objectif que le nombre de visiteurs progresse d'au moins 15\,\% entre 2020 et 2021.

L'objectif a-t-il été atteint ?
\end{enumerate}

\medskip

\textbf{Partie B :  étude des prix des hôtels de cette ville}

\medskip

Sur une période donnée, on relève les prix facturés pour une nuit par les hôtels de cette ville.

\begin{center}
\begin{tabularx}{\linewidth}{|m{4.5cm}|*{8}{>{\centering \arraybackslash}X|}}\hline
Prix facturés pour une nuit (en euro)&60 &80 &85 &90 &110 &120 &350 &500\\ \hline
Effectif &\np{1200} &\np{1350} &\np{1000} &\np{1100} &\np{1200} &\np{1300} &900 &300\\ \hline
\end{tabularx}
\end{center}

\begin{enumerate}[resume]
\item Déterminer l'étendue des prix facturés.
\item Quelle est la moyenne des prix facturés pour une nuit ? Arrondir à l'euro près.
\item L'association des hôteliers de cette ville cherche à attirer des touristes et annonce : \og Dans les hôtels de notre ville, au moins la moitié des nuits est facturée à moins de $100$~\euro{} \fg. Est-ce vrai ?
\end{enumerate}

\bigskip

\textbf{Exercice 4 \hfill 20 points}

\medskip

À l'aide d'un logiciel de programmation, on veut réaliser le motif \og Fleur\fg suivant.

\begin{center}
\begin{tabular}{|c|}\hline
\textbf{Motif \og Fleur \fg}\\
\psset{unit=1cm}
\begin{pspicture}(-1.6,-1.5)(1.5,1.5)
\def\para{\pspolygon(0,0)(0.9,0)(1.1,0.4)(0.2,0.4)}
\multido{\n=0+72}{5}{\rput{\n}(0,0){\para}}
\rput(-1,-1.35){Un pétale}\psline[linewidth=1.25pt]{->}(-1,-1.1)(-0.4,-0.5)
\end{pspicture}\\ \hline
\end{tabular}
\end{center}

\begin{enumerate}
\item 
	\begin{enumerate}
		\item Le parallélogramme KLMN ci-dessous représente un des pétales du motif \og Fleur \fg.

Construire ce parallélogramme sur la copie en prenant $1$~cm pour 5 pas.

\begin{center}
\psset{unit=0.8cm}
\begin{pspicture}(9.5,4)
\pspolygon(0.2,0.4)(7.2,0.4)(9.2,3.8641)(2.2,3.8641)
\uput[dl](0.2,0.4){K} \uput[dr](7.2,0.4){L} \uput[ur](9.2,3.8641){M} \uput[ul](2.2,3.8641){N} \uput[d](3.7,0.4){35 pas} \uput[ul](7,0.6){$120\degres$} \uput[dl](9,3.6641){$60\degres$} \uput[r](8.2,2.13){20 pas} 
\psarc(7.2,0.4){0.5}{60}{180}\psarc(9.2,3.8641){0.4}{180}{240}
\end{pspicture}
\end{center}
	\end{enumerate}
\end{enumerate}

\medskip

\begin{minipage}{0.56\linewidth}

\textbf{b.~} On définit le bloc \og Pétale \fg{} ci-contre afin de dessiner ce parallélogramme.

On commence la construction du parallélogramme au point K en s'orientant vers la droite.

Par quelles valeurs doit-on compléter les lignes 4, 5, 6, et 7 du bloc \og Pétale \fg{} ci-contre ?

\emph{Aucune justification n'est attendue, écrire sur la copie le numéro de la ligne du bloc \og Pétale\fg{} et la valeur correspondante.}
\end{minipage}\hfill
\begin{minipage}{0.4\linewidth}

\begin{tabular}{|l|}\hline
\multicolumn{1}{|c|}{Bloc \og Pétale \fg}\\
\begin{scratch}[num blocks]
\initmoreblocks{définir \namemoreblocks{Pétale}}
\blockpen{stylo en position d'écriture}
\blockrepeat{répéter \ovalnum{2} fois}
{\blockmove{avancer de \ovalnum{} pas}
\blockmove{tourner \turnleft{} de \ovalnum{} degr\'es}
\blockmove{avancer de \ovalnum{} pas}
\blockmove{tourner \turnleft{} de \ovalnum{} degr\'es}
}
\end{scratch}\\ \hline
\end{tabular}
%\end{enumerate}
\end{minipage}

\begin{enumerate}[resume]
\item Le bloc ci-dessous permet de construire un motif \og Fleur\fg{} en partant de son centre.

\begin{minipage}{0.48\linewidth}
\begin{tabular}{|l|}\hline
\multicolumn{1}{|c|}{Bloc \og Fleur \fg}\\
\begin{scratch}[num blocks]
\initmoreblocks{définir \namemoreblocks{Fleur}}
\blockrepeat{r\'ep\'eter \ovalnum{} fois}
{\blocklook{Pétale}
\blockmove{tourner \turnright{} de \ovalnum{72} degr\'es}
}
\end{scratch}\\ \hline
\end{tabular}
\end{minipage}\hfill
\begin{minipage}{0.48\linewidth}
\begin{tabular}{|c|}\hline
\textbf{Motif \og Fleur \fg}\\
\psset{unit=1cm}
\begin{pspicture}(-1.5,-1.5)(1.5,1.5)
\def\para{\pspolygon(0,0)(0.9,0)(1.1,0.4)(0.2,0.4)}
\multido{\n=0+72}{5}{\rput{\n}(0,0){\para}}
\end{pspicture}\\ \hline
\end{tabular}
\end{minipage}
	\begin{enumerate}
		\item Par quelle valeur doit-on compléter la ligne 2 du bloc \og Fleur \fg{} ci-dessus ? 
\emph{Aucune justification n'est attendue.}
		\item Expliquer le choix de la valeur \og 72 \fg{} dans la ligne 4.
		\item On modifie le bloc \og Fleur \fg{} pour construire le motif suivant:

\begin{center}
\psset{unit=0.2cm}
\begin{pspicture}(-7.75,-7.75)(7.75,7.75)
\def\para{\pspolygon(0,0)(0.9,0)(1.12,0.4)(0.22,0.4)}
%%
\def\para2{\pspolygon(0,0)(7,0)(9,3.4641)(2,3.4641)}
\multido{\n=0+30}{12}{\rput{\n}(0,0){\para2}}
\end{pspicture}
\end{center}

Quelles sont alors les modifications à apporter aux lignes 2 et 4 du bloc \og Fleur\fg{} ? \emph{Aucune justification n'est attendue}.
	\end{enumerate}
\end{enumerate}

\bigskip

\textbf{Exercice 5 \hfill 18 points}

\medskip

Un hippodrome est un lieu où se déroule des courses de chevaux.

On s'intéresse à la piste d'un hippodrome.

Cette piste est composée de :
\begin{itemize}
\item deux lignes droites modélisées par des segments de $850$~mètres; 
\item deux virages modélisés par deux demi-cercles de rayon $40$~mètres.
\end{itemize}

\begin{center}
\psset{unit=1cm,arrowsize=2pt 3}
\begin{pspicture}(11.6,5)
\psline(1.4,0.8)(11.4,0.8)\psline(1.4,3.5)(11.4,3.5)
\psarc(1.4,2.15){1.35}{90}{270}\psarc(11.4,2.15){1.35}{-90}{90}
\psline[linestyle=dashed](1.4,0.8)(1.4,3.5)\psline[linestyle=dashed](11.4,0.8)(11.4,2.15)
\psline{<->}(11.4,2.15)(11.4,3.5)\uput[l](11.4,2.825){rayon : 40 m}
\psline{<->}(1.4,0.5)(11.4,0.5)\uput[d](6.4,0.5){Segment de longueur : 850 m}
\rput(5.8,4.5){Schéma de la piste de cet hippodrome}
\end{pspicture}
\end{center}

\begin{enumerate}
\item Montrer que la longueur d'un tour de piste est d'environ \np{1951}~m. 
\item Un cheval parcourt un tour de piste en 2~min~9~s.
	\begin{enumerate}
		\item Calculer la vitesse moyenne de ce cheval sur un tour de piste en mètre par seconde (m/s). Donner une valeur approchée à l'unité près.
		\item Convertir cette vitesse en kilomètre par heure (km/h).
	\end{enumerate}	
\item On admet que la surface de la piste a une aire d'environ \np{73027} m$^2$.

On souhaite semer du gazon sur la totalité de la surface de la piste.

On doit choisir des sacs de gazon à semer parmi les trois marques ci-dessous :

\begin{center}
\begin{tabularx}{\linewidth}{|*{3}{>{\centering \arraybackslash}X|}}\hline
&Surface couverte par sac& Prix d'un sac\\ \hline
Marque A &500 m$^2$& 141,95~\euro\\ \hline
Marque B& 400 m$^2$&87,90~\euro\\ \hline
Marque C&300 m$^2$&66,50~\euro\\ \hline
\end{tabularx}
\end{center}

Quelle marque doit-on choisir pour que cela coûte le moins cher possible ?
\end{enumerate}
\end{document}