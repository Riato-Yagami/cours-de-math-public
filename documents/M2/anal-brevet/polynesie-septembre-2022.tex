% \renewcommand{\ttdefault}{lmtt}

% \newcommand{\euro}{\eurologo{}}
% %Tapuscrit : Denis Vergès
% %Relecture : François Hache

% \newcommand{\R}{\mathbb{R}}
% % \newcommand{\N}{\mathbb{N}}
% \newcommand{\D}{\mathbb{D}}
% \newcommand{\Z}{\mathbb{Z}}
% \newcommand{\Q}{\mathbb{Q}}
% % \newcommand{\C}{\mathbb{C}}

% \newcommand{\vect}[1]{\overrightarrow{\,\mathstrut#1\,}}
% \renewcommand{\theenumi}{\textbf{\arabic{enumi}}}
% \renewcommand{\labelenumi}{\textbf{\theenumi.}}
% \renewcommand{\theenumii}{\textbf{\alph{enumii}}}
% \renewcommand{\labelenumii}{\textbf{\theenumii.}}
% \def\Oij{$\left(\text{O},~\vect{\imath},~\vect{\jmath}\right)$}
% \def\Oijk{$\left(\text{O},~\vect{\imath},~\vect{\jmath},~\vect{k}\right)$}
% \def\Ouv{$\left(\text{O},~\vect{u},~\vect{v}\right)$}

% %\usepackage[colorlinks=true,pdfstartview=FitV,linkcolor=blue,citecolor=blue,urlcolor=blue]{hyperref}
% %\usepackage[dvips]{hyperref}
% %\hypersetup{%
% %pdfauthor = {APMEP},
% %pdfsubject = {Brevet},
% %pdftitle = {Polynésie 6 septembre 2022},
% %allbordercolors = white,
% %pdfstartview=FitH}

% \setlength\parindent{0mm}
% \rhead{}
% \lhead{\small L'année 2022}
% \rfoot{\small Polynésie}
% \lfoot{\small 6 septembre 2022}
% \pagestyle{fancy}
% \thispagestyle{empty}
% \begin{center}
% {\Large\textbf{\decofourleft~ Diplôme national du Brevet Polynésie
%  ~\decofourright}}\\[6pt]
% {\Large \textbf{6 septembre 2022}}

% \vspace{0,5cm}

% \textbf{Durée : 2 heures}
% \end{center}
% \medskip

% %\emph{L'usage de calculatrice avec mode examen activé est autorisé.\\
% %L'usage de calculatrice sans mémoire \og type collège \fg{} est autorisé}
% %
% %\vspace{0,5cm}
% %\framebox[16cm]{
% %\begin{minipage}{15cm}
% %\begin{center}
% %\textbf{Indications portant sur l'ensemble du sujet.}
% %\end{center}
% % Toutes les réponses doivent être justifiées, sauf si une indication contraire est donnée.\\
% %Pour chaque question, si le travail n'est pas terminé, laisser tout de même une trace de la 
% %recherche ; elle sera prise en compte dans la notation.
% %\end{minipage}
% %}
% %\vspace{0.75cm}

% \textbf{{\large \textsc{Exercice 1}} \hfill 22 points}

% \medskip

% %Cet exercice est constitué de six questions indépendantes.
% %
% %\medskip

% \begin{enumerate}
% \item %Calculer $\dfrac56 + \dfrac78$ et donner le résultat sous la forme d'une fraction irréductible.

% %On détaillera les calculs.
% $\dfrac56 + \dfrac78 = \dfrac{5 \times 4}{6 \times 4} + \dfrac{7 \times 3}{8\times 3} = \dfrac{20 + 21}{24} = \dfrac{41}{24}$.
% \item 
% 	\begin{enumerate}
% 		\item %Donner, sans justifier, la décomposition en facteurs premiers de $198$ et de $84$.
		
% $\bullet~~$$198 = 9 \times 22 = 3 \times 3 \times 2 \times 11 = 2 \times 3^2 \times 11$ ;
		
% $\bullet~~$$84 = 4 \times 21 = 2 \times 2 \times 3 \times 7 = 2^2 \times 3 \times 7$.
% 		\item %En déduire la forme irréductible de la fraction $\dfrac{198}{84}$.
% $\dfrac{198}{84} = \dfrac{2 \times 3^2 \times 11}{2^2 \times 3 \times 7} = \dfrac{3 \times 11}{2 \times 7} = \dfrac{33}{14}$.
% 	\end{enumerate}	
% \item  %On donne l'expression littérale suivante : $E = 5(3x - 4) - (2x - 7)$.

% %Développer et réduire $E$.
% $E = 5(3x - 4) - (2x - 7) = 15x - 20 - 2x + 7 = 13x - 13$.
% \item  On désigne par $b$ un nombre positif.

% \begin{minipage}{0.58\linewidth}
% %Déterminer la valeur de $b$ telle que le périmètre du rectangle ci-contre soit égal à 25.
% Le rectangle a une largeur de 4,5 et une longueur de 3b + 2,9.

% Son périmètre est égal à : $2(4,5 + 3b + 2,9) =$

% $ 2(7,4 + 3b) = 14,8 + 6b$.

% Il faut que $14,8 + 6b = 25$, soit $6b = 25 - 14,8$ ou 

% $6b = 10,2$, soit $b = \dfrac{10,2}{6} = \dfrac{3 \times 3,4}{3 \times 2} = 1,7$.
% \end{minipage}\hfill
% \begin{minipage}{0.38\linewidth}
% \psset{unit=1cm,arrowsize=2pt 3}
% % \begin{pspicture}(5.8,3.5)
% % \psframe(1,0)(5.8,2.8)
% % \psline(1.9,2.7)(1.9,2.9)\psline(2.8,2.7)(2.8,2.9)\psline(3.7,2.7)(3.7,2.9)
% % \psline(1.4,2.7)(1.5,2.9)\psline(2.3,2.7)(2.4,2.9)\psline(3.2,2.7)(3.3,2.9)
% % \psline[linewidth=0.6pt]{<->}(1,3)(1.9,3)\uput[u](1.45,3){$b$}
% % \psline[linewidth=0.6pt]{<->}(3.7,3)(5.8,3)\uput[u](4.75,3){$2,9$}
% % \psline[linewidth=0.6pt]{<->}(0.8,0)(0.8,2.8)\uput[l](0.8,1.4){4,5}
% % \end{pspicture}
% \end{minipage}

% \item ~

% \begin{minipage}{0.58\linewidth}
% %Calculer le volume de la pyramide à base rectangulaire de hauteur SH $= 6$ ci-contre.
% On sait que $V = \dfrac13 \times B \times h$, avec $B = 3 \times 4 = 12$ et $h = 6$, d'où :

% $V = \dfrac13 \times 12 \times 6 = 24$.
% \end{minipage}\hfill
% \begin{minipage}{0.38\linewidth}
% \psset{unit=1cm,arrowsize=2pt 3}
% % \begin{pspicture}(-2.5,-1)(2.6,4.6)
% % \psline(0,4.2)(-2.5,-0.5)(0.9,-0.5)(0,4.2)(2.5,0.5)(0.9,-0.5)
% % \pspolygon[linestyle=dashed](-2.5,-0.5)(2.5,0.5)(-0.9,0.5)
% % \psline[linestyle=dashed](0,0)(0,4.2)(-0.9,0.5)(0.9,-0.5)
% % \uput[l](0,2.1){6}\uput[d](-0.8,-0.5){4}\uput[dr](1.7,0){3}
% % \uput[u](0,4.2){S}\uput[d](0,0){H}
% % \end{pspicture}
% \end{minipage}
% \item %Le nombre d'habitants d'une ville a augmenté de 12\,\% entre 2019 et 2020.

% %Cette ville compte \np{20692}~habitants en 2020.

% %Quel était le nombre d'habitants de cette ville en 2019 ?
% Augmenter de 12\,\%, c'est multiplier par $1 + \dfrac{12}{100} = 1 + 0,12 = 1,12$.

% Si $x$ est le nombre d'habitants en 2019, alors :

% $x \times 1,12 = \np{20692}$, d'où en multipliant chaque membre par $\dfrac{1}{1,12}$, \quad $x = \dfrac{\np{20692}}{1,12} = \np{18475}$.

% Il y avait en 2019, \np{18475} habitants.
% \end{enumerate}

% \bigskip

% \textbf{{\large \textsc{Exercice 2}} \hfill 22 points}

% \medskip

% \begin{minipage}{0.53\linewidth}
% Un poteau électrique vertical [BC] de $5,2$ m de haut est retenu par un câble métallique [AC] comme montré sur le schéma 1 qui n'est pas en vraie grandeur.
% \end{minipage}\hfill
% \begin{minipage}{0.43\linewidth}
% \psset{unit=1cm}
% \begin{pspicture}(5.5,5)
% \psline[linewidth=1.5pt](3.7,0.3)(3.7,4.7)%BC
% \psline(3.7,4.7)(0.5,0.3)(3.7,0.3)%CAB
% \psframe(3.7,0.3)(3.5,0.5)
% \uput[r](3.7,2.5){Poteau : 5,2 m}
% \uput[d](2.1,0.3){Sol : 3,9 m}
% \uput[ul](2.1,2.5){Câble}
% \rput(1.7,4){\textbf{Schéma 1}}
% \uput[dl](0.5,0.3){A} \uput[dr](3.7,0.3){B} \uput[ur](3.7,4.7){C} 
% \end{pspicture}
% \end{minipage}

% \medskip

% \begin{enumerate}
% \item %Montrer que la longueur du câble [AC] est égale à $6,5$~m.
% le théorème de Pythagore appliqué au triangle ABC rectangle en B, s'écrit :

% $\text{AB}^2 + \text{BC}^2 = \text{AC}^2$, soit $3,9^2 + 5,2^2 = \text{AC}^2$, ou encore $15,21 + 27,04 = \text{AC}^2$, soit $\text{AC}^2 = 42,25$.

% On a donc AC $= \sqrt{42,25} = 6,5$~(m).
% \item %Calculer la mesure de l'angle $\widehat{\text{ACB}}$ au degré près.
% On a par exemple $\cos \widehat{\text{ACB}} = \dfrac{\text{BC}}{\text{AC}} = \dfrac{5,2}{6,5} = \dfrac{52}{65} = \dfrac{4 \times 13}{5 \times 13} = \dfrac{4}{5} = \dfrac{8}{10} = 0,8$.

% La calculatrice donne $\widehat{\text{ACB}} \approx 36,9$.

% La mesure de l'angle $\widehat{\text{ACB}}$ est $37\degres$ au degré près.
% \end{enumerate}

% Deux araignées se trouvant au sommet du poteau (point C) décident de rejoindre le bas du câble (point A) par deux chemins différents.

% \begin{enumerate}[resume]
% \item %La première araignée se déplace le long du câble [AC] à une vitesse de $0,2$~m/s. 

% %Vérifier qu'il lui faut 32,5 secondes pour atteindre le bas du câble.
% On a $v = \dfrac dt$, avec $v = 0,2$ et $d = \text{CA} = 6,5$.

% Donc $0,2 = \dfrac{6,5}{t}$, d'où $0,2t = 6,5$ et $t = \dfrac{6,5}{0,2} = 6,5 \times 5 = 32,5$~(s).
% \item La deuxième araignée décide de parcourir le chemin CFHA indiqué en pointillés sur le schéma 2 (qui n'est pas en vraie grandeur) : elle suit le morceau de câble [CF] en marchant, puis descend verticalement le long de [FH] grâce à son fil et enfin marche sur le sol le long de [HA].

% %Calculer les longueurs FH et HA.
% Les droites (FH) et (CB) étant toutes les deux perpendiculaires à la droite (AB) sont parallèles.

% $\bullet~~$D'après le théorème de Thalès : $\dfrac{\text{FH}}{\text{BC}} = \dfrac{\text{AF}}{\text{AC}}$, soit $\dfrac{\text{FH}}{5,2} = \dfrac{4}{6,5}$, d'où en multipliant chaque membre par 6,5 : $\text{FH} = \dfrac{4 \times 5,2}{6,5} = \dfrac{4 \times 52}{65} = \dfrac{4 \times 4 \times 13}{5 \times 13} = \dfrac{16}{5} = \dfrac{32}{10} = 3,2$~(m).

% $\bullet~~$On a de même toujours d'après Thalès : $\dfrac{\text{AH}}{\text{AB}} = \dfrac{\text{AF}}{\text{AC}}$, soit $\dfrac{\text{AH}}{3,9} = \dfrac{4}{6,5}$, d'où en multipliant chaque membre par 3,9 : $\text{AH} = \dfrac{3,9 \times 4}{6,5} = \dfrac{39 \times 4}{65} = \dfrac{3 \times 13 \times 4}{5 \times 13} = \dfrac{12}{5} = \dfrac{24}{10} = 2,4$~(m).
% % \begin{center}
% % \psset{unit=1cm,arrowsize=2pt 3}
% % \begin{pspicture}(7,6.8)
% % %\psgrid
% % \psline[linewidth=1.5pt](5.4,0.8)(5.4,6.1)%BC
% % \psline(3.9,4.1)(1.5,0.8)%FA
% % \psline[linestyle=dashed,ArrowInside=->](5.4,6.1)(3.9,4.1)(3.9,0.8)(1.5,0.8)%CFHA
% % \psline(3.9,0.8)(5.4,0.8)%HB
% % \uput[dl](1.5,0.8){A} \uput[dr](5.4,0.8){B} \uput[ur](5.4,6.1){C} \uput[r](3.9,4.1){F} \uput[ur](3.9,0.8){H} 
% % \rput(1.4,6){\textbf{Schéma 2}}
% % \psline[linewidth=0.6pt]{<->}(5.8,0.8)(5.8,6.1)\rput{90}(6.05,3.45){5,2~m}
% % \psline[linewidth=0.6pt]{<->}(1.5,0.4)(5.4,0.4)\uput[d](3.45,0.4){3,9 m}
% % \psline[linewidth=0.6pt]{<->}(1.3,1)(3.7,4.3)\rput{57.03}(2.4,2.8){4 m}
% % \psline[linewidth=0.6pt]{<->}(1,1.4)(4.9,6.7)\rput{57.03}(2.8,4.2){6,5 m}
% % \psframe(3.9,0.8)(3.7,1)\psframe(5.4,0.8)(5.2,1)
% % \end{pspicture}
% % \end{center}

% \item %La deuxième araignée marche à une vitesse de 0,2 m/s le long des segments [CF] et [HA] et descend le long du segment [FH] à une vitesse de 0,8 m/s.

% %Laquelle des deux araignées met le moins de temps à arriver en A ?
% De $v = \dfrac{d}{t}$, on tire $d = v \times t$ et $t = \dfrac{d}{v}$.

% La deuxième araignée parcourt CF + HA $= (6,5 - 4)  + 2,4 = 4,9$~(m) à la vitesse de 0,2~(m/s).

% Elle met donc $t_1 = \dfrac{4,9}{0,2} = 4,9 \times 5 = 24,5$~(s) pour parcourir ces deux segments.

% Pour parcourir le segment [FH], elle met $t_2 = \dfrac{3,2}{0,8} = \dfrac{32}{8} = 4$~(s).

% Elle met donc au total : $24,5 + 4 = 28,5$~(s) : c'est elle qui met le moins de temps pour arriver en A.
% \end{enumerate}

% \bigskip

% \textbf{{\large \textsc{Exercice 3}} \hfill 17 points}

% \medskip

On utilise un logiciel de programmation.
%
%On rappele que \og s'orienter à $0\degres$ \fg{} signifie qu'on oriente le stylo vers le haut.
%
%On considère les deux scripts suivants:
%
%\begin{center}
%\begin{tabularx}{\linewidth}{X X}
%\multicolumn{1}{c}{Script 1}&\multicolumn{1}{c}{Script 2}\\
%\begin{scratch}
%\blockinit{Quand \greenflag est cliqué}
%\blockpen{effacer tout}
%\blockpen{stylo en position d'écriture}
%\blockmove{s'orienter à \ovalnum{0} }
%\blockrepeat{répéter \ovalnum{2} fois}
%{\blockmove{avancer de \ovalnum{20}}
%\blockmove{tourner \turnright{} de \ovalnum{90} degrés}
%\blockmove{avancer de \ovalnum{40}}
%\blockmove{tourner \turnleft{} de \ovalnum{90} degrés}
%}
%\end{scratch}&
%\begin{scratch}
%\blockinit{Quand \greenflag est cliqué}
%\blockpen{effacer tout}
%\blockpen{stylo en position d'écriture}
%\blockmove{s'orienter à \ovalnum{0} }
%\blockvariable{mettre \selectmenu{longueur} à \ovalnum{20}}
%\blockrepeat{répéter \ovalnum{2} fois}
%{\blockmove{avancer de \selectmenu{longueur}}
%\blockmove{tourner \turnright{} de \ovalnum{90} degrés}
%\blockmove{avancer de \selectmenu{longueur}}
%\blockmove{tourner \turnleft{} de \ovalnum{90} degrés}
%\blockvariable{ajouter à \selectmenu{longueur} \ovalnum{20}}
%}
%\end{scratch}\\
%\end{tabularx}
%\end{center}
%
%\medskip

\begin{enumerate}
\item On exécute le script 1 ci-dessus.

Représenter le chemin parcouru par le stylo sur l'ANNEXE à rendre avec la copie.

Le tracé est en rouge sur l'ANNEXE.
\item %Quel dessin parmi les trois ci-dessous correspond au script 2 ? 

%On expliquera pourquoi les deux autres dessins ne correspondent pas au script 2.

%Chaque côté de carreau mesure 20 pixels.
Le dessin 1  n'est  pas correct car après avoir avancé deux fois de 20 on doit avancer de 40.

Le dessin 3  n'est  pas correct car on ne se dirige pas au départ vers le haut.

Il reste donc le dessin 2 seul correct.
% \begin{center}
% \begin{tabularx}{\linewidth}{*{3}{>{\centering \arraybackslash}X}}
% \textbf{Dessin 1}&\textbf{Dessin 2} &\textbf{Dessin 3}\\
% \psset{unit=4.5mm,arrowsize=2pt 3}
% \begin{pspicture}(9,12)
% \multido{\n=0+1}{10}{\psline[linewidth=0.2pt](\n,3)(\n,12)}
% \multido{\n=3+1}{10}{\psline[linewidth=0.2pt](0,\n)(9,\n)}
% \psframe(9,2)
% \rput(4.5,1){Position de départ}
% \psline[linewidth=1.5pt](1,4)(1,5)(2,5)(2,6)(3,6)
% \psline{->}(4.5,2)(1,4)
% \end{pspicture}&
% \psset{unit=4.5mm,arrowsize=2pt 3}
% \begin{pspicture}(9,12)
% \multido{\n=0+1}{10}{\psline[linewidth=0.2pt](\n,3)(\n,12)}
% \multido{\n=3+1}{10}{\psline[linewidth=0.2pt](0,\n)(9,\n)}
% \psframe(9,2)
% \rput(4.5,1){Position de départ}
% \psline[linewidth=1.5pt](1,4)(1,5)(2,5)(2,7)(4,7)
% \psline{->}(4.5,2)(1,4)
% \end{pspicture}&
% \psset{unit=4.5mm,arrowsize=2pt 3}
% \begin{pspicture}(9,12)
% \multido{\n=0+1}{10}{\psline[linewidth=0.2pt](\n,3)(\n,12)}
% \multido{\n=3+1}{10}{\psline[linewidth=0.2pt](0,\n)(9,\n)}
% \psframe(9,2)
% \rput(4.5,1){Position de départ}
% \psline[linewidth=1.5pt](1,4)(2,4)(2,5)(4,5)(4,7)
% \psline{->}(4.5,2)(1,4)
% \end{pspicture}
% \end{tabularx}
% \end{center}

\item On souhaite maintenant obtenir le motif représenté sur le dessin 4 : 

\begin{center}
\psset{unit=4.5mm,arrowsize=2pt 3}
\begin{pspicture}(9,13)
\multido{\n=0+1}{10}{\psline[linewidth=0.2pt](\n,3)(\n,12)}
\multido{\n=3+1}{10}{\psline[linewidth=0.2pt](0,\n)(9,\n)}
\psframe(9,2)
\uput[u](4.5,12){\textbf{Dessin 4}}
\rput(4.5,1){Position de départ}
\pspolygon[linewidth=1.5pt](1,4)(1,5)(3,5)(3,9)(5,9)(5,10)(6,10)(6,4)
\psline{->}(4.5,2)(1,4)
\end{pspicture}
\end{center}

Compléter sans justifier les trois cases du script 3 donné en ANNEXE à rendre avec la copie, permettant d'obtenir le dessin 4.

Les compléments sont en rouge dans l'annexe.
\item À partir du motif représenté sur le dessin 4, on peut obtenir le pavage ci-dessous :

\begin{center}
\psset{unit=4.5mm}
\begin{pspicture}(24,18)
\psline[linewidth=1.5pt](0,0)(1,0)(1,1)(3,1)(3,5)(5,5)(5,7)(3,7)(3,11)(1,11)(1,13)(3,13)(3,17)(5,17)(5,18)(7,18)(7,17)(9,17)(9,13)(11,13)(11,11)(9,11)(9,7)(7,7)(7,5)(9,5)(9,1)(11,1)(11,0)(13,0)
\psline[linewidth=1.5pt](13,0)(13,1)(15,1)(15,5)(17,5)(17,7)(15,7)(15,11)(13,11)(13,13)(15,13)(15,17)(17,17)(17,18)(19,18)(19,17)(21,17)(21,13)(23,13)(23,11)(21,11)(21,7)(19,7)(19,5)(21,5)(21,1)(23,1)(23,0)(25,0)
\psframe[linewidth=1.5pt](18,12)\psframe[linewidth=1.5pt](6,0)(12,18)
\psframe[linewidth=1.5pt](18,6)
\psline[linewidth=1.5pt](6,12)(6,18)\psline[linewidth=1.5pt](12,12)(12,18)
\psline[linewidth=1.5pt](18,12)(18,18)
\psline[linewidth=1.5pt](12,18)(19,18)\psline[linewidth=1.5pt](18,12)(23,12)
\psline[linewidth=1.5pt](18,6)(19,6)\psline[linewidth=1.5pt](18,0)(23,0)
\uput[ul](1,12){A} \uput[ul](6,12){B} \uput[dl](12,18){C} \uput[ur](13,12){D} 
\uput[ur](18,12){E} \uput[ur](0,6){F} \uput[ur](12,6){G}
\multido{\n=4.5+3.0,\na=1+1}{6}{\rput(\n,14){\na}}
\multido{\n=2.+3.0,\na=7+1}{7}{\rput(\n,9){\na}}
\multido{\n=2.+3.0,\na=14+1}{7}{\rput(\n,3){\na}}
\end{pspicture}
\end{center}

Répondre aux questions suivantes sur votre copie en indiquant le numéro du motif qui convient (on ne demande pas de justifier la réponse) :

	\begin{enumerate}
		\item Quelle est l'image du motif 1 par la translation qui transforme le point B en E ? Le motif 5.
		\item Quelle est l'image du motif 1 par la symétrie de centre B ? Le motif 9.
		\item Quelle est l'image du motif 16 par la symétrie de centre G ? Le motif 12.
		\item Quelle est l'image du motif 2 par la symétrie d'axe (CG) ? Le motif 5.
	\end{enumerate}	
\end{enumerate}

% \bigskip

% \textbf{{\large \textsc{Exercice 4}} \hfill 20 points}

% \medskip

% \begin{enumerate}
% \item Voici un tableau de valeurs d'une fonction $f$ :

% \begin{center}
% \begin{tabularx}{\linewidth}{|*{8}{>{\centering \arraybackslash}X|}}\hline
% $x$		&$-2$	&$-1$	&0	&1		& 3		& 4		& 5\\ \hline
% $f(x)$	&5 		&3 		&1 	&$-1$ 	&$-5$	& $-7$	& $-9$\\ \hline
% \end{tabularx}
% \end{center}

% 	\begin{enumerate}
% 		\item Quelle est l'image de $3$ par la fonction $f$ ?
		
% L'image de $3$ par la fonction $f$ est $f(3) = - 5$.
% 		\item Donner un nombre qui a pour image $5$ par la fonction $f$.
		
% On a $f(- 2) = 5$, donc $- 2 $ a pour image 5 par la fonction $f$.
% 		\item Donner un antécédent de $1$ par la fonction $f$.
		
% On a $f(0) = 1$, donc 1 a pour antécédent $0$ par $f$.
% 	\end{enumerate}
% \item On considère le programme de calcul suivant:
% \begin{center}
% \begin{tabular}{|l|}\hline
% Choisir un nombre\\
% Ajouter 1\\
% Calculer le carré du résultat\\ \hline
% \end{tabular}
% \end{center}
% 	\begin{enumerate}
% 		\item Quel résultat obtient-on en choisissant $1$ comme nombre de départ?
		
% On a $1 \to 1 + 1 = 2 \to 2^2 = 4$ : 1 donne 4 comme résultat. 

% Et en choisissant $- 2$ comme nombre de départ ?

% On a $- 2  \to - 2 + 1 = -1  \to (- 1)^2 = 1$ : $- 2$ donne 1 comme résultat.
% 		\item On note $x$ le nombre choisi au départ et on appelle $g$ la fonction qui à $x$ fait correspondre le résultat obtenu avec le programme de calcul.
		
% Exprimer $g(x)$ en fonction de $x$.

% On a $x \to x + 1 \to (x + 1)^2$. Donc $g(x) = (x + 1)^2$.
% 	\end{enumerate}
% \item La fonction $h$ est définie par $h(x) = 2x^2 - 3$.
% 	\begin{enumerate}
% 		\item Quelle est l'image de $3$ par la fonction $h$ ?
		
% On a $h(3) = 2 \times 3^2 - 3 = 2 \times 9 - 3 = 18 - 3 = 15$.
% 		\item Quelle est l'image de $-4$ par la fonction $h$ ?
		
% On a $h(- 4) = 2 \times (- 4)^2 - 3 = 2 \times 16 - 3 = 32 - 3 = 29$.
% 		\item Donner un antécédent de $5$ par la fonction $h$. En existe-t-il un autre ?
		
% Il faut trouver $x$ tel que $2x^2 - 3 = 5$, soit $2x^2 = 8$ ou $x^2 = 4$ ou $x^2 - 4 = 0$, c'est-à-dire $(x - 2)(x + 2) = 0$ et enfin $\left\{\begin{array}{l c l}
% x - 2&=&0\\
% x + 2&=&0
% \end{array}\right.$ : il y a deux solutions : 2 et $- 2$.
% 	\end{enumerate}	
% \item On donne les trois représentations graphiques suivantes qui correspondent chacune à une des fonctions $f$, $g$ et $h$ citées dans les questions précédentes.

% Associer à chaque courbe la fonction qui lui correspond, en expliquant la réponse.

% \begin{center}
% \begin{tabularx}{\linewidth}{*{3}{>{\centering \arraybackslash}X}}
% \psset{unit=4.5mm,arrowsize=2pt 3}
% \begin{pspicture*}(-4,-5)(4,6)
% \rput(0,5.5){Représentation \no 1}
% \psaxes[linewidth=1.25pt,labelFontSize=\scriptstyle]{->}(0,0)(-4,-4.9)(4,5)
% \psplot[plotpoints=1000,linewidth=1.25pt]{-2}{5}{1 2 x mul sub}
% \end{pspicture*} &
% \psset{xunit=4.4mm,yunit=2.2mm,arrowsize=2pt 3}
% \begin{pspicture*}(-4,-4.5)(4,18.2)
% \rput(0,17){Représentation \no 2}
% \psaxes[linewidth=1.25pt,Dy=2,labelFontSize=\scriptstyle]{->}(0,0)(-4,-4)(4,16)
% \psplot[plotpoints=1000,linewidth=1.25pt]{-3.1}{3.1}{x dup mul 2 mul 3 sub}
% \end{pspicture*}&
% \psset{xunit=4.5mm,yunit=4mm,arrowsize=2pt 3}
% \begin{pspicture*}(-5,-1.5)(3,11)
% \rput(-1,10.5){Représentation \no 3}
% \psaxes[linewidth=1.25pt,labelFontSize=\scriptstyle]{->}(0,0)(-5,-1.5)(3,10)
% \psplot[plotpoints=1000,linewidth=1.25pt]{-4.1}{2.1}{1  x add  dup mul}
% \end{pspicture*}\\
% \end{tabularx}
% \end{center}

% La représentation \no 1 est celle de $f$ : c'est la seule pour laquelle l'image de 1 est $-1$.

% La représentation \no 2 est celle de $h$ : on a bien $h(0) = - 3$.

% La représentation \no 3 est celle de $g$ : on a bien $g(0) = 1$.
% \end{enumerate}

% \bigskip

% \textbf{{\large \textsc{Exercice 5}} \hfill 19 points}

% \medskip

% Une urne contient 20 boules rouges, 10 boules vertes, 5 boules bleues et 1 boule noire.

% Un jeu consiste à tirer une boule au hasard dans l'urne.

% Lorsqu'un joueur tire une boule noire, il gagne 10 points.

% Lorsqu'il tire une boule bleue, il gagne 5 points.

% Lorsqu'il tire une boule verte, il gagne 2 points.

% Lorsqu'il tire une boule rouge, il gagne 1 point.

% \medskip

% \begin{enumerate}
% \item Un joueur tire au hasard une boule dans l'urne.
% 	\begin{enumerate}
% 		\item %Quelle est la probabilité qu'il gagne $10$ points?
% Il gagne 10 points s'il tire une boule noire ; il y a 1 boule noire sur un total de $20 + 10 + + 5 + 1 = 36$ : la probabilité est égale à $\dfrac{1}{36}$.
% 		\item %Quelle est la probabilité qu'il gagne plus de $3$ points?
% 		Il gagnera plus de 3 points s'il tire une boule noire (1 seule) ou une boule bleue (5 boules bleues) : la probabilité est égale à $\dfrac{6}{36} = \dfrac{6 \times 1}{6 \times 6} = \dfrac16$.
% 		\item %A-t-il plus de chance de gagner $2$ points ou de gagner $5$ points ?
% Il y a plus de boules vertes que de boules bleues : Il a plus de chance de gagner $2$ points que de gagner $5$ points.
% 	\end{enumerate}	

% \item~

% %\begin{minipage}{0.48\linewidth}	
% %Le tableau ci-contre récapitule les scores obtenus par 15 joueurs:
% 	\begin{enumerate}
% 		\item %Quelle est la moyenne des scores obtenus par ces joueurs ?
% 		La moyenne des scores est : $\dfrac{2 + 1 + 1 + \ldots + 2}{15} = \dfrac{50}{15} = \dfrac{5 \times 10}{5 \times 3} = \dfrac{10}{3} = 3,333$~(points).
% 		\item %Quelle est la médiane des scores ?
% Les scores sont dans l'ordre croissant :
		
% 1 1 1 1 1 2 2 2 2 2 \ldots : la médiane est entre la 7\up{e} et la 8\up{e} valeur soit 2.
% 		\item %Déterminer la fréquence du score \og 10 points \fg.
% 		La fréquence du score 10 est $\dfrac{2}{15}$.
% 		\end{enumerate}
% %\end{minipage}	\hfill
% %\begin{minipage}{0.48\linewidth}
% %\begin{tabularx}{\linewidth}{|*{2}{>{\centering \arraybackslash}X|}}\hline
% %\textbf{JOUEUR}&\textbf{SCORE}\\ \hline
% %JOUEUR A&2 points\\ \hline
% %JOUEUR B& 1 point\\ \hline
% %JOUEUR C&1 point\\ \hline
% %JOUEUR D&5 points \\ \hline
% %JOUEUR E&10 points\\ \hline
% %JOUEUR F&2 points\\ \hline
% %JOUEUR G&2 points\\ \hline
% %JOUEUR H&5 points\\ \hline
% %JOUEUR I&1 point\\ \hline
% %JOUEUR J&2 points\\ \hline
% %JOUEUR K&5 points\\ \hline
% %JOUEUR L&10 points\\ \hline
% %JOUEUR M&1 point\\ \hline
% %JOUEUR N&1 point\\ \hline
% %JOUEUR O&2 points\\ \hline
% %\end{tabularx}
% %\end{minipage}

% \item Mille joueurs ont participé au jeu. Peut-on estimer le nombre de joueurs ayant obtenu le score de $10$ points ?

% La réponse, affirmative ou négative, devra être argumentée.

% On a vu à la question précédente que la fréquence du score 10 points est égale à $\dfrac{1}{36}$.

% Donc pour \np{1000} joueurs il y aura à peu près :

% $\np{1000} \times \dfrac{1}{36} = \dfrac{\np{1000}}{36} = \dfrac{250}{9} \approx 27,7$

% Environ 28 joueurs auront un score de 10 points.
% \end{enumerate}

% \newpage

% \begin{center}
% \textbf{\large ANNEXE à compléter et à rendre avec la copie }
% \end{center}

% \bigskip

% \textbf{Exercice 3. Question 1}

% \medskip

% \begin{minipage}{0.48\linewidth}
% \psset{unit=5mm,arrowsize=2pt 3}
% \begin{pspicture}(10,12)
% \multido{\n=0+1}{10}{\psline[linewidth=0.2pt](\n,3)(\n,12)}
% \multido{\n=3+1}{10}{\psline[linewidth=0.2pt](0,\n)(9,\n)}
% \multido{\n=0+1}{10}{\multido{\na=3+1}{10}{\psdots(\n,\na)}}
% \psframe(9,2)
% \rput(4.5,1){Position de départ}
% \psline[linewidth=1.5pt,linecolor=red](1,4)(1,5)(3,5)(3,6)(5,6)
% \psline{->}(4.5,2)(1,4)
% \end{pspicture}
% \end{minipage}\hfill
% \begin{minipage}{0.48\linewidth}
% Chaque côté de carreau mesure 20 pixels.

% La position de départ du stylo est indiquée sur la figure ci-contre.
% \end{minipage}

% \bigskip

% \textbf{Exercice 3. Question 3}

% %\begin{center}
% \psset{unit=1cm,arrowsize=2pt 3}
% \begin{pspicture}(12,12)
% %\psgrid
% \rput(1,11.5){Script 3}\psframe(0,11)(2,12)
% \psline{->}(8,6)(2,6)
% \psline{->}(8,1.5)(2,1.5)
% \psline{->}(8,-1.5)(2,-1.5)
% \rput(8,2.25){\psscaleboxto(1,8){\}}}
% \rput(2,4){\begin{scratch}[scale=0.8]
% \blockinit{Quand \greenflag est cliqué}
% \blockpen{effacer tout}
% \blockpen{stylo en position d'écriture}
% \blockmove{s'orienter à \ovalnum{0}}
% \blockmove{avancer de \ovalnum{20}}
% \blockmove{tourner \turnright{} de \ovalnum{90} degrés}
% \blockmove{avancer de {\red 40}}
% \blockmove{tourner \turnleft{} de \ovalnum{90} degrés}
% \blockmove{avancer de \ovalnum{80}}
% \blockmove{tourner \turnright{} de \ovalnum{90} degrés}
% \blockmove{avancer de \ovalnum{40}}
% \blockmove{tourner \turnleft{} de \ovalnum{90} degrés}
% \blockmove{avancer de {\red 20}}
% \blockmove{tourner \turnright{} de \ovalnum{90} degrés}
% \blockmove{avancer de \ovalnum{20}}
% \blockmove{tourner \turnright{} de \ovalnum{90} degrés}
% \blockmove{avancer de {\red 120}}
% \blockmove{tourner \turnright{} de \ovalnum{90} degrés}
% \blockmove{avancer de \ovalnum{100}}
% \end{scratch}}
% \rput(10.5,2.2){Trois cases à compléter}
% \psframe(8.5,1.7)(13,2.7)
% \end{pspicture}
% %\end{center}










% % \end{document}