% VARIABLES %%%
\def\theme{ \large Présentation pour l'oral 2 du CAPES de mathématiques}
% \def\date{20/01/2024}
%%%%%%%%%%%%%%%

\section*{Introduction}

\begin{itemize}
    \item \textbf{Présentation personnelle :} Jules PESIN.
    \item \textbf{Objectif :} expliquer mon parcours et mes motivations pour devenir professeur de mathématiques.
\end{itemize}

\section{Lycée et Prépa}

\begin{itemize}
    \item \textbf{Bac S :} intérêt prononcé pour les sciences,
    surtout mathématiques et physique.
    \item \textbf{Classe préparatoire MPSI :} passion pour les maths et la physique,
    découverte du plaisir d'enseigner grâce à des professeurs inspirants.
    \item \textbf{Inscription en L3 :} désintérêt pour l'ingénierie et à devenir employé d'une entreprise à but lucratif.
\end{itemize}


\section{Début de l'expérience d'enseignement}

\begin{itemize}
    \item \textbf{Licence 3 :} interview OIP avec ancien professeur de MP,
    début des cours particuliers suite à sa recommandation.
    \item \textbf{Première expérience de cours particuliers :} appréciation de l'enseignement et du retour aux fondamentaux,
    redecouverte des programmes de Terminale,
    construction de la connaissances chez un élève.
\end{itemize}

\section{Renforcement de la vocation}

\begin{itemize}
    \item \textbf{Redoublement en L3 :} continuité des cours particuliers,
    enseignement à une élève de collège.
    \item \textbf{Inscription en M1 MEEF :} confirmation du plaisir d'enseigner et de l'intérêt pour l'enseignement des mathématiques,
    découverte d'une vocation.
\end{itemize}


\section{Master MEEF}

\begin{itemize}
    \item Poursuite des cours particuliers avec un élève de collège pendant de 2ans.
    \item \textbf{Compétences :} Découverte de toutes les compétences à maitriser,
    nécessaire à l'enseignement.
    \item \textbf{Participation à trois stages SOPA :} enseignement à des classes de 5ème, 4ème et 3ème.
    \item \textbf{Expériences en classe :} moments formateurs,
    consolidation de l'envie de devenir professeur.
\end{itemize}

\section{Développement des compétences didactiques}

\begin{itemize}
    \item \textbf{Etude de cas et suivi de stage:} Acquisition de connaissances didactiques et compétences d'analyse des situations d'apprentissage.
    \item \textbf{Rédaction d'un mémoire :} « Exploration des jeux de cadres et de la dialectique outil-objet appliqués à la remédiation dans l'apprentissage des puissances au collège ».
    \item \textbf{Impact du mémoire :} approfondissement des méthodes pédagogiques, expérience formatrice.
\end{itemize}

\section*{Conclusion}

\begin{itemize}
    \item \textbf{Synthèse :} parcours marqué par la passion pour les maths et l'enseignement.
    \item Confirmation de la vocation grâce aux diverses expériences d'encadrement.
    \item Détermination à poursuivre cette voie et à contribuer à l'éducation des jeunes générations.
\end{itemize}

\section*{Remerciements}

\begin{itemize}
    \item Remerciement du jury.
    \item Disponibilité pour les questions.
\end{itemize}