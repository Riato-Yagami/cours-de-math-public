\section*{Résumé}

Ce mémoire explore l'application de la dialectique outil-objet et des jeux de cadres dans la remédiation mathématique,
en se concentrant particulièrement sur le concept des puissances au collège.
Le document est structuré entre cadres théoriques et mises en œuvre pratiques en classe,
utilisant les théories didactiques de Régine Douady.
Il présente une expérimentation didactique nouvelle utilisant le logiciel GeoGebra pour travailler sur le changement de cadres géométriques à algébrique,
dans le but d'approfondir la compréhension des concepts mathématiques par les élèves grâce à une activité interactive.
Le mémoire évalue également l'efficacité de ces approches par une analyse \textit{a priori} et \textit{a posteriori} de la séance,
et compare la confiance et la compréhension mathématique des élèves avant et après l'intervention.
Les résultats suggèrent que l'activité a favorisé une meilleure compréhension des puissances et a mis en lumière plusieurs difficultés que les élèves rencontrent lors de la phase de changement de cadre.