% VARIABLES %%%
\def\theme{présentation mémoire}
\def\date{14/05/2024}
%%%%%%%%%%%%%%%


\section*{Introduction}
\textbf{Contexte:} Séquence sur les puissances donnée il y a 3 mois à des élèves de 4ème au collège Janson de Sailly.

\textbf{Problématiques rencontrées et objectif:} Recherche de méthodes de remédiation pour adresser les difficultés observées, avec une analyse réflexive utilisant les concepts de \textbf{jeux de cadres} et de \textbf{dialectique outil-objet}.

\section{Retour sur le mémoire}
\subsection{Résumé théorique}
\textbf{Auteurs et concepts clés:}
\begin{itemize}
    \item \textbf{Régine Douady:} Focus sur "\textbf{Jeux de cadre et dialectique outil-objet}"
    \begin{itemize}
        \item \textbf{Jeux de Cadre:} Exploration et compréhension des concepts mathématiques à travers diverses perspectives.
        Ces cadres peuvent être algébriques,
        géométriques,
        numériques,
        ou liés à des situations réelles,
        chacun offrant une vision unique et des outils spécifiques pour aborder les problèmes mathématiques.
        En classe, utiliser les jeux de cadres signifie guider les élèves à passer d'un mode de pensée à un autre.
        \item \textbf{Dialectique outil-objet:} Utilisation des concepts mathématiques comme outils pour résoudre des problèmes et comme objets d'étude théorique.
        Par exemple, un nombre peut être utilisé comme outil pour calculer une surface ou comme objet d'étude pour explorer ses propriétés numériques.
    \end{itemize}
    \item \textbf{Pertinence:} Ces approches enrichissent la remédiation mathématique en offrant une compréhension plus adaptative et réfléchie.
\end{itemize}

\subsection{Première expérimentation}
\textbf{Explications des applications pratiques:}
\begin{itemize}
    \item \textbf{Régine Douady:} "Enseignement de la dialectique outil-objet et des jeux de cadres en formation mathématique des professeurs d'école."
    \begin{itemize}
        \item \textbf{Jeu des Cibles et Défi du Rectangle:} Des activités qui intègrent les principes théoriques pour améliorer la compréhension arithmétique et la résolution de problèmes géométriques complexes.
        \item \textbf{Expérimentation avec Cabri-Géomètre:} Utilisation de logiciels pour une approche interactive et visuelle de problèmes algébriques.
    \end{itemize}
    \item Bernard Capponi et Rosamund Sutherland "Interaction des cadres algébrique et graphiques dans la résolution de problèmes": \textbf{expérimentation avec Cabri-Géomètre}, problèmes algébriques à travers des manipulations géométriques interactives.
    \item Guy Brousseau "Fondements et méthodes de la didactique des mathématiques" : situation \textbf{a-didactique}, découverte autonome mais encadrée et encourageant le développement de stratégies personnelles.
\end{itemize}

\newpage
\section{Seconde expérimentation}
\subsection{Analyse a priori}
\textbf{Objectifs et adaptations:}
\begin{itemize}
    \item Rappel sur les hauteurs issue d'un sommet dans un triangle en début de cours.
    \item Adaptation des outils et des supports (\textbf{GeoGebra simplifié}) pour maximiser l'efficacité pédagogique et la compréhension des élèves.
    \item Premier triangle construit avec les élèves au bout de quelques minutes.
\end{itemize}
\subsection{Analyse a posteriori}
\begin{itemize}
    \item \textbf{Observations:} Progression des élèves dans l'activité, avec une meilleure confrontation aux défis mathématiques.
    \item \textbf{Points clés:} Discussion sur la croissance exponentielle vs linéaire, et sur les notations appropriées pour exprimer des réponses numériques.
    \item \textbf{Impacts:} Les élèves ont mieux compris la distinction entre multiplication séquentielle et puissances, et l'importance de la notation appropriée dans les expressions mathématiques.
\end{itemize}

\section*{Conclusion}
\begin{itemize}
    \item \textbf{Synthèse:} L'usage des \textbf{jeux de cadres} et de la \textbf{dialectique outil-objet} a permis de cibler et d'exploiter efficacement les difficultés des élèves, enrichissant ainsi leur expérience d'apprentissage et leur compréhension des mathématiques.
    \item \textbf{Réflexion finale:} Ces approches donnent aux élèves l'occasion de se confronter activement à des défis mathématiques, favorisant une compréhension plus profonde et une réflexion plus critique sur les mathématiques.
\end{itemize}

\section*{Entretiens}
\subsection*{Différences cadre et registre}
\begin{itemize}
    \item \textbf{Cadres (Régine Douady) :} concernent les contextes et les perspectives multiples dans lesquels un concept peut être exploré. Ils mettent l'accent sur la capacité des élèves à utiliser différents points de vue pour résoudre des problèmes, reliant ainsi diverses disciplines mathématiques.
    \item \textbf{Registres (Raymond Duval) :} Les registres se concentrent sur les systèmes de représentation et les transformations entre ces systèmes. Ils soulignent l'importance de la capacité des élèves à traduire un concept d'une forme de représentation à une autre pour faciliter la compréhension et la résolution des problèmes.
\end{itemize}

