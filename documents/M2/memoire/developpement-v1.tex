\section{Cadre théorique et revue de littérature}
\subsection{Principes des Jeux de Cadres}

Dans le contexte de la didactique des mathématiques,
particulièrement dans les travaux de Régine Douady et d'autres chercheurs en éducation,
le terme «cadre» désigne un contexte ou un environnement conceptuel et pédagogique;
dans lequel se déroule l'apprentissage ou l'exploration d'un concept mathématique.\\

Un «cadre» peut être compris de plusieurs manières.
Cela peut faire référence à un ensemble particulier de règles,
de procédures,
de définitions et de notations qui sont utilisées pour explorer et comprendre un concept mathématique.
Par exemple,
le cadre algébrique implique l'utilisation de symboles et de formules algébriques,
tandis que le cadre géométrique s'appuie sur des représentations visuelles et spatiales.\\

Un cadre peut aussi se rapporter à l'ensemble des stratégies,
des activités et des interactions en classe qui soutiennent l'apprentissage.
Cela inclut la manière dont l'enseignant structure la leçon,
les types de questions posées,
les tâches assignées aux élèves et la manière dont les élèves sont encouragés à réfléchir et à interagir avec le matériel mathématique.\\

Dans une dernière perspective,
le cadre peut également se référer aux préconceptions,
aux attitudes et aux expériences antérieures des élèves,
qui influencent la manière dont ils perçoivent et s'engagent dans l'apprentissage mathématique.
Cela peut inclure leur façon de voir les mathématiques,
leur confiance en leur capacité à résoudre des problèmes et la valeur qu'ils attribuent à la connaissance mathématique.\\

Les jeux de cadres,
un concept associé,
impliquent de passer d'un cadre à l'autre pour aider les élèves à voir les concepts mathématiques sous différents angles,
et à développer une compréhension plus riche et plus flexible.
Cette approche vise à encourager les élèves à penser de manière critique et créative,
en explorant les connexions entre différents domaines mathématiques,
et entre les mathématiques et d'autres disciplines ou situations de la vie réelle.

\subsection{Fondements de la Dialectique Outil-Objet}

Dans la didactique des mathématiques,
la dialectique outil-objet,
concept élaboré par Régine Douady,
sert de fondement à la compréhension et à l'enseignement des concepts mathématiques.
Ce cadre théorique souligne la nature dualiste des éléments mathématiques,
lesquels peuvent être considérés simultanément comme des «outils» facilitant la résolution de problèmes,
et comme des «objets» d'étude et de réflexion.\\

La notion d'«outil» dans cette dialectique fait référence à la fonction pratique et opérationnelle des concepts mathématiques.
Par exemple,
un concept tel que la fraction peut servir d'outil pour diviser des quantités ou résoudre des problèmes concrets.
En contraste,
le même concept considéré comme un « objet » se focalise sur ses propriétés,
définitions et théorèmes,
indépendamment de son utilisation dans des contextes spécifiques.\\

La dialectique outil-objet propose ainsi une approche dynamique de l'apprentissage,
où les élèves sont amenés à explorer les concepts mathématiques à travers différentes lunettes :
en tant qu'outils pour l'action et en tant qu'objets de connaissance pure.
Cette alternance entre les perspectives encourage une compréhension plus profonde et nuancée des mathématiques,
permettant aux élèves de mieux saisir les liens entre la théorie et la pratique.\\

L'importance de cette dialectique réside dans son potentiel pédagogique :
elle incite les enseignants à concevoir des situations d'apprentissage qui amènent les élèves à réfléchir sur les mathématiques de manière active,
en passant de l'application pratique à la conceptualisation et vice versa.\\
En appliquant cette approche,
les enseignants favorisent un environnement d'apprentissage où les élèves construisent leur savoir de manière significative,
en reconnaissant et en exploitant la nature versatile des concepts mathématiques.

\subsection{Impact Épistémologique et Didactique}

La dialectique outil-objet redéfinit la façon dont les concepts mathématiques sont perçus et utilisés.
Cette approche encourage les élèves à envisager les mathématiques comme un domaine dynamique et évolutif,
où les idées peuvent changer de statut en fonction du contexte et de l'usage.
En adoptant cette perspective,
les élèves développent une compréhension plus profonde et nuancée des séquences prévu par le programme,
ce qui les aide à intégrer de nouvelles informations dans leur cadre conceptuel existant.\\

Dans le contexte pédagogique,
la dialectique outil-objet et les jeux de cadres offrent aux enseignants des stratégies pour structurer les leçons de manière à promouvoir une exploration mathématique active.
En alternant entre différents cadres,
les enseignants peuvent aider les élèves à établir des liens entre diverses représentations et applications des concepts mathématiques,
facilitant ainsi une compréhension plus intégrée et applicative.
Cette approche encourage également les élèves à devenir des apprenants actifs et réflexifs,
capables de questionner et de critiquer les idées mathématiques plutôt que de les recevoir passivement.

\subsection{Stratégies de Formation Pratique pour les Enseignants}

Dans son second article,
Régine Douady met en avant l'importance d'intégrer la dialectique outil-objet et les jeux de cadres dans la formation des enseignants pour renouveler les méthodes pédagogiques en mathématiques.
Cette démarche vise non seulement à enrichir le contenu enseigné mais aussi à éveiller une pensée critique chez les formateurs,
ainsi qu'à affiner leur compréhension des dynamiques d'apprentissage.\\

Douady suggère spécifiquement l'organisation d'ateliers pratiques pour les enseignants,
les plongeant dans des expériences similaires à celles de leurs élèves.
Cela permet aux formateurs de saisir les enjeux mathématiques du point de vue des apprenants,
favorisant le développement de méthodes d'enseignement empathiques et fondées sur l'expérience réelle des élèves.
Ces ateliers jouent un rôle clé dans la préparation des enseignants à concevoir des activités de classe qui stimulent l'apprentissage actif et la réflexion.\\

Cette approche transforme l'enseignement mathématique en une aventure contextualisée et significative,
plaçant les élèves au centre de leur parcours éducatif.
En intégrant ces méthodes,
les enseignants favorisent un cadre d'apprentissage propice à l'éveil de la curiosité et de la pensée critique en mathématiques.

\section{Exemples de mise en place en classe}

\subsection{Application Pratique : Le Jeu des Cibles et l'Apprentissage Numérique}

Présenté dans «Jeux de cadres et dialectique outil-objet»,
le «jeu des cibles»,
conçu pour les élèves de cours préparatoire (CP),
représente un exemple pratique de la mise en œuvre de la dialectique outil-objet et des jeux de cadres dans l'enseignement des nombres et de l'arithmétique.
Cette activité didactique,
étalée sur trois semaines,
se décompose en phases progressives,
allant de la familiarisation avec les nombres à l'application de concepts mathématiques avancés,
et illustre une approche dynamique et participative de l'apprentissage mathématique.\\

Le jeu vise principalement à encourager les élèves à:
explorer et étendre leur compréhension des nombres à travers une série de jeux et d'activités interactives;
développer un langage algébrique et utiliser des représentations graphiques pour structurer et résoudre des problèmes mathématiques;
aborder de manière ludique la notion de multiples,
posant ainsi les bases pour la compréhension de concepts arithmétiques plus complexes.\\

L'utilisation d'une cible avec des marques de score et de balles en caoutchouc transforme l'apprentissage en une expérience concrète et engageante.
Les règles du jeu,
conçues pour être variées,
défient les élèves à atteindre des scores spécifiques ou à maximiser leur total.\\

Le jeu des cibles, au-delà de son aspect ludique,
sert d'outil pédagogique efficace pour impliquer les élèves dans leur propre processus d'apprentissage.
Il incarne l'approche de Douady,
où l'apprentissage mathématique est vu comme une aventure interactive,
ancrée dans l'expérience et guidée par la curiosité et l'investigation.
Cette activité illustre parfaitement comment les jeux de cadres peuvent être utilisés pour faciliter la transition des élèves entre différents contextes mathématiques,
les aidant ainsi à construire et à transformer leurs savoirs de manière active et réfléchie.\\

À travers le jeu,
les élèves abordent les nombres et les opérations arithmétiques de base,
telles que l'addition et la soustraction.
Ce cadre encourage les élèves à utiliser et à comprendre les nombres non seulement comme des symboles,
mais aussi en tant qu'outils pour résoudre des problèmes concrets.\\

Alors que les élèves progressent dans le jeu et commencent à utiliser des stratégies pour atteindre certains scores,
ils entament implicitement l'utilisation de raisonnements algébriques.
Ils commencent à reconnaître des motifs,
à former des équations simples et à manipuler des expressions pour atteindre des objectifs spécifiques.

\subsection{Approfondissement Mathématique : Le Défi du Rectangle et les Phases de Résolution}

Douady décrit une autre expérimentation et les phases de résolutions mise en jeux par les élèves.
Il s'agit d'un défi géométrique:
identifier les dimensions d'un rectangle ayant un demi-périmètre spécifique,
et une aire déterminée,
Ce problème,
initialement abordé dans un contexte géométrique,
requiert non seulement une visualisation spatiale mais aussi une manipulation algébrique.\\

\textbf{Phase A - Utilisation d'Outils Existant :}
Armés de leur connaissance des formules classiques du périmètre et de l'aire,
les élèves tentent d'appliquer ces outils à la situation donnée.
Ils recherchent des dimensions satisfaisant à la fois les conditions de périmètre et d'aire,
confrontant ainsi théorie et pratique.\\

\textbf{Phase B - Rencontre de Difficultés et Nouvelles Questions :}
Lorsque les solutions entières se font rares,
les élèves explorent l'idée de dimensions non entières.
Cette étape les amène à questionner et à étendre leurs connaissances préexistantes,
les introduisant à des manipulations plus complexes comme les nombres décimaux ou les racines carrées.\\

\textbf{Phase C - Jeux de Cadres et Exploration :}
La transition vers d'autres cadres se manifeste lorsque les élèves utilisent des graphiques pour illustrer diverses combinaisons de longueur et de largeur.
Cette représentation graphique les aide à visualiser le problème sous un nouvel angle,
facilitant l'identification de modèles ou de solutions potentielles.\\

\textbf{Phase D - Explicitation et Institutionnalisation Locale :}
Cette phase voit les élèves consolider et formaliser leurs découvertes.
Les solutions intuitives deviennent des concepts étudiés explicitement.
Ils élaborent une compréhension renouvelée des liens entre les dimensions du rectangle et son aire,
conceptualisant le processus mathématique qui sous-tend leurs observations.\\

\textbf{Phase E - Familiarisation et Réinvestissement :}
Avec de nouvelles connaissances en main,
les élèves testent leur compréhension à travers des problèmes similaires ou des variations de la situation initiale.
Ce processus de réinvestissement consolide leur compréhension et augmente leur aisance avec les concepts récemment acquis.\\

Cet exemple illustre efficacement le potentiel transformatif des jeux de cadres et de la dialectique outil-objet dans l'enseignement des mathématiques.
En naviguant entre différents cadres et en s'engageant activement avec le problème,
les élèves développent une compréhension profonde et versatile des concepts mathématiques,
transformant les formules et procédures standards en outils dynamiques et adaptatifs pour la résolution de problèmes concrets.

\subsection{Institutionnalisation et Réinvestissement du Savoir}

Douady aborde le processus final de l'apprentissage mathématique dans le cadre de la dialectique outil-objet et des jeux de cadres :
l'institutionnalisation et le réinvestissement du savoir.
Ces phases cruciales garantissent que les connaissances acquises par les élèves deviennent intégrées dans leur compréhension globale des mathématiques,
et qu'elles peuvent être appliquées de manière autonome dans de nouveaux contextes.\\

L'institutionnalisation se réfère à la formalisation et à l'acceptation des concepts et méthodes mathématiques découverts au cours des phases précédentes d'exploration et d'expérimentation.
Cela implique une transition des solutions intuitives et des manipulations concrètes vers des principes mathématiques reconnus et articulés,
qui sont discutés,
validés et résumés par l'enseignant.
Cette phase solidifie les concepts dans le cadre de la classe et les intègre dans le corpus plus large des connaissances mathématiques des élèves.\\

Le réinvestissement,
quant à lui,
concerne l'application de ces concepts nouvellement institutionnalisés à des situations différentes ou plus complexes.
Cela permet aux élèves de tester et de renforcer leur compréhension dans un éventail de contextes,
facilitant ainsi la transition du savoir spécifique à la classe vers un ensemble d'outils polyvalents pour la pensée mathématique.\\

Ce processus est intrinsèquement cyclique,
reflétant la nature continue de la dialectique outil-objet.
Les élèves passent de l'utilisation des concepts mathématiques comme outils pour résoudre des problèmes (outil),
à l'examen approfondi de ces mêmes concepts comme sujets d'étude en soi (objet).
Ce cycle,
de l'introduction des concepts à leur application autonome,
assure que les élèves non seulement comprennent les mathématiques de manière abstraite,
mais sont également capables de les utiliser de manière concrète et significative dans diverses situations.

\subsection{Intégration des Cadres Algébriques et Graphiques: L'Expérimentation avec Cabri-Géomètre}

L'étude menée par Sutherland et Capponi illustre l'utilisation du logiciel de géométrie dynamique Cabri-Géomètre par des élèves de collège pour résoudre des problèmes alliant géométrie et algèbre.
Cet outil de géométrie dynamique favorise une interaction productive entre les cadres algébriques et graphiques,
enrichissant la compréhension des élèves.
Il offre aussi aux élèves une manière de visualiser et de résoudre des problèmes géométriques de manière innovante.\\

Dans cet environnement,
les élèves,
notamment Frédéric et Farid,
explorent activement les problèmes en alternant entre représentations graphiques et formules algébriques.
Le logiciel agit comme pont entre la théorie et la pratique.
Cette démarche facilite une exploration approfondie et développe une compréhension nuancée des concepts mathématiques et sous-jacents.\\

L'étude souligne la valeur,
de l'exploration de problème de façon multimodale,
et des environnements numériques dans l'enseignement des mathématiques.
En facilitant les transitions entre différents cadres conceptuels,
Cabri-Géomètre stimule de par ces fonctionnalité interactive l'apprentissage actif et encourage l'intégration des connaissances mathématiques,
montrant l'importance de permettre aux élèves de naviguer entre divers systèmes de signes (symboles, notations, diagrammes…) pour une compréhension globale.
Ce processus renforce également une approche plus exploratoire et critique de la résolution de problèmes,
un aspect essentiel de la didactique moderne des mathématiques.

\section{Optimisation de la Remédiation Mathématique par la Dialectique Outil-Objet et les Jeux de Cadres}

\subsection{Fondement de la Remédiation par la Dialectique et les Jeux de Cadres}

Bien que les articles examinés ne mentionnent pas explicitement la remédiation,
le cadre théorique qu'ils présentent,
la dialectique outil-objet et les jeux de cadres,
s'avère intrinsèquement lié à cette notion.
En effet,
ces approches pédagogiques visent à corriger les conceptions erronées et à approfondir la compréhension mathématique des élèves,
objectifs fondamentaux de la remédiation.
Ces méthodologies proposent une voie prometteuse pour aborder les défis d'apprentissage,
en mettant l'accent sur la correction des erreurs conceptuelles et l'encouragement d'une compréhension mathématique robuste.\\

En conséquence,
je m'efforcerai d'identifier et d'intégrer,
en complément aux exemples décrits dans les articles étudiés qui se concentrent sur l'institutionnalisation,
une activité spécifiquement conçue pour corriger les erreurs et les conceptions erronées observées dans mes classes.

\subsection{Stratégies de Remédiation : Application de la Dialectique et Exploration des Cadres}

La remédiation mathématique,
visant à adresser et à surmonter les difficultés d'apprentissage des élèves,
peut bénéficier grandement de l'application de la dialectique outil-objet et des jeux de cadres,
tels que conceptualisés par Régine Douady.
Ces approches offrent des stratégies dynamiques et flexibles pour renforcer la compréhension des élèves,
en facilitant la transition entre différents modes de pensée mathématique et en rendant les concepts plus accessibles et significatifs.\\

La remédiation commence par un diagnostic précis des difficultés rencontrées par les élèves.
Une fois ces difficultés identifiées,
l'enseignant peut utiliser la dialectique outil-objet pour déterminer si un concept mathématique doit être abordé comme un "outil" pour résoudre des problèmes pratiques ou comme un "objet" d'analyse théorique.
Cette flexibilité permet d'adapter l'enseignement aux besoins spécifiques de chaque élève ou groupe d'élèves.\\

Les jeux de cadres encouragent les élèves à explorer un concept mathématique à travers différents "cadres" ou perspectives.
En remédiation,
cela signifie présenter un concept de plusieurs manières,
par exemple,
à travers des manipulations concrètes,
des visualisations graphiques,
des représentations symboliques ou des contextes de problèmes du monde réel.
Cette variété d'approches aide les élèves à construire des connexions entre différentes représentations du même concept,
renforçant ainsi leur compréhension et leur capacité à appliquer ce concept dans divers contextes.\\

Plusieurs stratégies de remédiation basées sur le jeu de cadres pourraient être envisagées:
\begin{itemize}%
    \item \textbf{Manipulation Concrète :}
    Utiliser des objets physiques pour représenter des concepts mathématiques permet aux élèves de mieux saisir des idées abstraites.

    \item \textbf{Visualisation Graphique :}
    Dessiner ou utiliser des logiciels de géométrie dynamique aide les élèves à visualiser des problèmes et à comprendre les relations entre les variables.
    
    \item \textbf{Exploration Algorithme :}
    Encourager les élèves à créer leurs propres méthodes pour résoudre des problèmes développe leur capacité à penser de manière algébrique et à reconnaître des modèles.
    
    \item \textbf{Contextualisation des Problèmes :}
    Présenter des problèmes dans des contextes significatifs pour les élèves peut faciliter la compréhension et l'engagement.
\end{itemize}%

La remédiation devrait être cyclique,
permettant aux élèves de réfléchir sur leur apprentissage et d'appliquer de nouvelles connaissances à des problèmes inédits.
Ce processus de réflexion et d'application renforce la compréhension et prépare les élèves à utiliser de manière autonome les concepts mathématiques.

\subsection{Conception d'une Activité de Remédiation sur les Puissances}

Face aux défis observés durant la séquence sur les puissances avec mes élèves de quatrième,
particulièrement concernant l'abstraction des puissances de dix négatives,
j'ai décidé de reconceptualiser l'approche pédagogique pour cette notion.
La difficulté semble résider dans la nature fortement abstraite et algébrique avec laquelle le concept a été initialement abordé,
traitant principalement les puissances comme des objets d'étude plutôt que comme des outils pratiques.
Pour remédier à cela,
j'envisage de réintroduire ces concepts trois mois après la fin de la séquence initiale,
à travers une activité ancrée dans un cadre graphique où les puissances seront utilisées comme des outils pour résoudre des problèmes concrets,
plutôt que comme de simples entités algébriques à manipuler.\\

Cette activité vise à encourager les élèves à redécouvrir les puissances,
en particulier leur transition de positive à négative,
par le biais de la construction géométrique plutôt que par des explications directes.
L'enjeu sera de les amener à introduire spontanément les puissances comme une stratégie pour répondre à des questions qui autrement les confineraient dans un cadre numérique limité aux mesures de grandeurs.
Ce faisant,
l'activité aspire non seulement à corriger les conceptions erronées mais également à promouvoir une compréhension plus intuitive et appliquée des mathématiques,
démontrant la pertinence de la dialectique outil-objet et des jeux de cadres dans la remédiation mathématique.

\subsection{Exploration Géométrique des Puissances à travers les Hauteurs de Triangles Rectangles}

Inspirée par l'activité de l'escargot de Pythagore,
cette activité se concentre sur la construction géométrique successive de triangles \cite{villemin},
non pas pour explorer les racines successives,
mais pour illustrer les puissances successives d'un nombre,
notamment les puissances de dix.
Cette approche vise à transformer un concept mathématique abstrait en une expérience visuelle et tangible,
facilitant la compréhension des puissances négatives et positives par les élèves.\\

L'activité s'appuie sur une propriété spécifique de la hauteur issue de l'angle
droit, laquelle peut être démontrée de deux manières : algébriquement, en
utilisant trois applications du théorème de Pythagore, ou géométriquement, dans
un cadre plus visuel. Ces deux approches de démonstration, adaptées pour des
élèves de quatrième, sont détaillées en annexe et pourraient faire l'objet d'une
séance complémentaire pour approfondir la compréhension du concept.


\subsection{Méthodologie de l'Activité : Puissances travers la Géométrie}

Cette activité sur les puissances,
inspirée par les principes méthodologiques développés par Régine Douady dans son second article,
vise à engager les élèves de quatrième dans une exploration géométrique pour renforcer leur compréhension des puissances positives et négatives.
En suivant les préceptes de Douady,
l'activité est conçue pour être à la fois a-didactique et riche en opportunités de découverte et de réflexion.\\

Le terme "a-didactique" désigne une situation d'apprentissage où les élèves,
sans directives explicites de l'enseignant,
utilisent leurs propres stratégies et connaissances pour résoudre des problèmes.
Cela promeut un apprentissage autonome,
encourageant la découverte et la construction individuelle du savoir par une interaction directe avec les matériaux mathématiques.
Cette approche,
essentielle dans la théorie de la situation didactique de Guy Brousseau \cite{brousseau}
vise à activer le processus d'apprentissage des élèves,
leur permettant de développer une compréhension approfondie et des connaissances durables basées sur leur exploration personnelle.

\subsubsection{Conception de la Situation A-didactique}

\textbf{Choix de la Situation :}
L'activité débute par une construction géométrique qui sert de base à l'exploration des puissances successives d'un nombre.
Les élèves sont placés dans un contexte où les puissances deviennent un outil nécessaire pour résoudre des problèmes géométriques concrets,
favorisant ainsi une compréhension profonde du concept.\\

\textbf{Adaptation des Énoncés :}
Les instructions données aux élèves sont soigneusement élaborées pour encourager l'autonomie et la réflexion.
Les problèmes sont présentés de manière à ce que les élèves doivent intuitivement recourir aux puissances de dix,
sans que le concept soit explicitement mentionné.

\subsubsection{Gestion de l'Interaction Élève-Activité}

\textbf{Observation et Soutien :}
L'enseignant observe attentivement les stratégies développées par les élèves,
intervenant de manière minimale pour maintenir l'aspect a-didactique de la situation.
L'objectif est de permettre aux élèves de naviguer par eux-mêmes à travers les défis proposés,
tout en étant prêt à fournir des pistes de réflexion si nécessaire.

\subsubsection{Diffusion et Institutionnalisation des Résultats}

\textbf{Dépersonnalisation et Généralisation :}
À l'issue de l'activité,
un temps est consacré à la discussion collective des découvertes faites par les élèves.
Cette phase vise à partager les différentes approches utilisées et à aider les élèves à voir au-delà de leur contexte immédiat,
vers une compréhension plus généralisée des puissances.\\

\textbf{Dépersonnalisation et Généralisation :}
L'enseignant guide les élèves dans le processus de décontextualisation,
les encourageant à conceptualiser les puissances comme des outils mathématiques applicables dans une variété de contextes,
au-delà de l'activité spécifique.

\subsubsection{Évaluation des Acquis}

\textbf{Évaluation par Application :}
L'évaluation des connaissances acquises se fait à travers l'introduction de nouveaux problèmes qui requièrent l'application des concepts explorés.
Cela permet à l'enseignant d'évaluer la disponibilité opérationnelle des connaissances en tant qu'outils.

\subsubsection{Maintien de la Disponibilité des Connaissances}

\textbf{Réinvestissement :}
L'enseignant prévoit des occasions futures pour les élèves de réutiliser et de consolider leur compréhension des puissances dans de nouveaux contextes,
assurant ainsi le maintien de leur disponibilité pour l'apprentissage futur.\\

En appliquant ces principes méthodologiques,
cette activité aspire non seulement à remédier aux conceptions erronées autour des puissances de dix négatives mais aussi à encourager une exploration mathématique riche et significative pour les élèves.

