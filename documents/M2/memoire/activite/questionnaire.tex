% VARIABLES %%%
\def\theme{Questionnaire - Mémoire}
% \def\date{05/04/2024}
\thispagestyle{assignment}
%%%%%%%%%%%%%%%

Ce questionnaire,
visant à améliorer l'enseignement des mathématiques,
nécessite vos réponses sincères pour évaluer des activités didactiques.
Vos données restent confidentielles et non évaluées.
Merci pour votre honnêteté et votre participation.\\

Merci de bien vouloir déposer vos réponses avant le \textbf{15 avril 2024},
à cette adresse: \href{https://juels.dev/math/}{https://juels.dev/math/},
mot de passe: \textbf{TRH},
au format \textbf{NOM\_qTRH.pdf}
(autre format acceptés : .png .jpg .txt .odt .docx).

\subsection{Questions générals}

\questions{
    Sur une échelle de 0 à 10{,} évalue ta compréhension des maths en général.
    Peux-tu expliquer pourquoi ?
    ,%
    Sur une échelle de 0 à 10{,} évalue ta compréhension des puissances.
    Peux-tu expliquer pourquoi ?
    ,%
    Qu'est-ce qui te semble difficile avec les puissances ?
    ,%
    Trouves-tu que les puissances sont utiles dans d'autres cours de maths? Lesquels?
    ,%
    Vois-tu une différence entre utiliser 10 et d'autres nombres comme bases pour les puissances ?
    ,%
    Préfères-tu écrire $10^{-n}$ ou $\dfrac{1}{10^n}$? Pourquoi?
    ,%
    Penses-tu utiliser les puissances en dehors de tes études ? Comment ?
    ,%
    As-tu trouvé la séquence sur les puissances intéressantes?
    ,%
    Trouves-tu les maths intéressantes de manière générale?
    Qu'est-ce qui pourrait rendre les maths plus intéressantes pour toi ?
}

\subsection{Questions relative à l'activité: Triangles rectangles et hauteurs}

\questions{
    Cette activité t'a-t-elle intéressée ? Qu'est-ce qui t'a plu ou déplu ?
    ,%
    Quelles difficultés as-tu rencontrées pendant l'activité ?
    Comment les as-tu surmontées ?
    ,%
    Penses-tu avoir mieux compris les puissances grâce à cette activité ?
    Quels aspects sont maintenant plus clairs ?
    ,%
    Cette activité a-t-elle changé ta façon de voir les puissances négatives ?
    Si oui{,} comment ?
    ,%
    Aimerais-tu participer à plus d'activités de ce type{,} qui reprennent des concepts déjà étudiés d'une manière différente ?
}