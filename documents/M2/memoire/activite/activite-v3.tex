% VARIABLES %%%
\def\theme{Activités géométriques : Hauteurs et Triangles}
\def\date{04/11/2023}
%%%%%%%%%%%%%%%
\vspace*{-1cm}
\hint{
    \begin{itemize}
        \item Fichier amorce : "amorce\_activite\_hauteur\_et\_triangle.ggb" à modifier.
        \item Rendre le fichier avec le nom : "NOM1\_NOM2\_activite.ggb"
        \item Répondre au questions sur une feuille à rendre.
    \end{itemize}
}\vspace*{0.2cm}

\newcommand*{\tri}[4]{
    \item Placer $#4$ sur $(O#2)$, pour construire le triangle #1: $#2#3#4$ rectangle en $#3$.
}

Le fichier amorce présente deux droites $(OA)$ et $(OB)$ perpendiculaires avec $OA=1\cm$ et $OB=2\cm$.
\vspace*{-0.2cm}
\section{$OB = 2cm$}
\begin{enumerate}\setlength{\itemsep}{15pt}%
    \item \begin{enumerate}
        \tri{1}{A}{B}{C}

        \hint{On s'intéressera dans la suite du problème à la valeur de la hauteur issue de l'angles droit des triangles rectangles.}
        \item Quelle est la hauteur du triangle 1?
    \end{enumerate}
    \item \begin{enumerate}
        \tri{2}{B}{C}{D}
        \tri{3}{C}{D}{E}
        \item Quelles sont les hauteurs de ces deux triangles?
        
        \hint{Pour mesurer utiliser l'outil: "Distance ou Longueur".}
    \end{enumerate}
    \item \begin{enumerate}
        \item Si l'on construisait suivant le même procédé le triangle 4,
        quelle serait sa hauteur ? 
        \item Des triangles 5, 7 et 56 ?
        \item Quelle serait la hauteur du triangle $n$ ?
    \end{enumerate}
\end{enumerate}

\section{$OB = 10cm$}

\begin{enumerate}\setlength{\itemsep}{15pt}%
    \item Déplacer le curseur $OB$ pour que $[OB]$ soit de longueur $10\cm$.
    \item \begin{enumerate} \item Quelles sont les nouvelles hauteurs des triangles 1, 2, 3?
            \item Quelles seraient les hauteurs des triangles 4, 9, 123?
            \item \label{itm:first} Quelle serait la hauteur du triangle $n$ ?
    \end{enumerate}
    \item \begin{enumerate} 
        \tri{0}{B}{A}{A_1}
        \tri{-1}{A}{A_1}{B_1}
        \tri{-2}{A_1}{B_1}{C_1}
        \item Quelle sont les hauteurs de ces trois triangles?
        \item Est-ce que la formule donné en \ref{itm:first} reste vrai pour ces triangles?
    \end{enumerate}
    \item \begin{enumerate}
        \item Déplacer le curseur $OB$ pour que $[OB]$ soit de nouveaux de longueur $2\cm$.
        \item Quelles sont les hauteurs des triangles 0, -1 et -2?
        \item Quelles seraient les hauteurs des triangles -4 et -23?
        \item Pouvez-vous émettre une conjecture sur une nouvelle notation,
        qui donnerait la hauteur du triangle $-n$.
    \end{enumerate}
\end{enumerate}