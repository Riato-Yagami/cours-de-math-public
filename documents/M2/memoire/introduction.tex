\section*{Introduction}

Améliorer constamment les compétences mathématiques de nos élèves est un objectif majeur de notre système éducatif.
Pour répondre à cette exigence,
il est essentiel de développer de nouvelles stratégies didactiques qui stimulent l'apprentissage et l'intérêt des élèves en mathématiques.
Ce mémoire explore certaines approches didactiques pour enrichir l'enseignement des mathématiques et améliorer la compréhension des élèves. 
Pour cela il sera nécessaire de mettre en place une réflexion apronfondie sur les procédés d'apprentissage de notions mathématiques.\\

Une des pistes explorées par la didacticienne des mathématiques Régine Douady concerne la dialectique outil-objet et les jeux de cadres.
Ces concepts,
offrent une nouvelle perspective dans l'approche des mathématiques.
À travers ce mémoire, nous nous appuierons sur deux articles majeurs de Douady,
« Jeux de cadre et dialectique outil-objet » (1986) \cite{douady1} et « Enseignement de la dialectique outil-objet et des jeux de cadres en formation mathématique des professeurs d'école » (1993) \cite{douady2},
afin de mieux comprendre les enjeux de ces notions.
Elles cherchent à diversifier les approches mathématiques sous lesquelles un problème est examiné,
pour en améliorer la compréhension et ainsi renforcer l'ancrage des connaissances chez les élèves.
Nous examinerons également l'article de Bernard Capponi et Rosamund Sutherland,
« Interaction des cadres algébriques et graphiques dans la résolution de problèmes » (1999) \cite{capponi},
qui met en avant l'utilisation du logiciel Cabri-Géomètre pour enrichir l'approche algébrique des problèmes.\\

Bien que ces concepts, notamment celui des jeux de cadres, soient de plus en plus intégrés en classe,
les tâches de type recherche restent souvent limitées à la découverte de nouvelles connaissances,
ou bien à la démonstration de certaines propositions mathématiques.
D'après mes observations lors de stages en classe,
un problème récurrent est le réinvestissement des connaissances par les élèves.
Ce mémoire vise donc à explorer comment l'utilisation de changements de cadres peut conduire à une connaissance mieux assimilée par l'élève,
permettant de construire un savoir plus durable.
Nous chercherons notamment à éliminer certaines conceptions erronées d'élèves.\\

L'expérimentation que je vais mener,
dans un but de remédiation,
se concentrera sur la séquence des puissances en classe de 4e.
Elle s'inspirera des travaux de Sutherland et Capponi en mettant en lumière l'interaction entre les cadres graphiques et algébriques,
au seins d'un problème géométrique.
Cette expérimentation consistera à manipuler les hauteurs des triangles rectangles à l'aide du logiciel Géogebra,
permettant ainsi d'explorer de façon concrète les concepts mathématiques abordés.