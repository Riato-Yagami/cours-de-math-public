% VARIABLES %%%
\setTitle{test}
%%%%%%%%%%%%%%%

\setSeq{2}{Organisation et gestion de données}

\setGrade{6e}

% \def\exoGrid{
%     \def\crossWidth{0.04cm}
%     \def\gridColor{gradeColor!50}
%     % \begin{center}
%         \begin{tikzpicture}[scale = 0.5]
%             \tkzInit[xmax=15, ymax=11]
%             \tkzGrid[color=\gridColor,subxstep=0.5,subystep=0.5]
%             \drawPoint{A}{6}{7}
%             \drawPoint{B}{10}{5}
%         \end{tikzpicture}
%         \ifArticle{\hspace{0.5cm}}
%         \begin{tikzpicture}[scale = 0.5]
%             \tkzInit[xmax=15, ymax=11]
%             \tkzGrid[color=\gridColor,subxstep=0.5,subystep=0.5]
%             \drawPoint{A}{5}{4}
%             \drawPoint{B}{11}{8}
%         \end{tikzpicture}
% }

\slide{exo}{
    \exo{}{
        Chacune des figures suivantes est constituée de deux points $A$ et $B$.
        Chacune d'elles doit être complétée par un point $J$ qui respecte les conditions suivantes :
        \begin{enumerate}
            \item $I$, $C$ et $D$ sont trois points tels que $I$ est le milieu des segments $[AC]$ et $[BD]$;
            \item $E$ est le point tel que $A$ est le milieu du segment $[DE]$;
            \item $J$ est le milieu du segment $[CE]$.
        \end{enumerate}
        % \ifArticle{\begin{center}\exoGrid\end{center}}
        \begin{itemize}
            \item Placer deux points $A$ et $B$ sur une feuille blanche et de même trouver l'emplacement de $J$.
        \end{itemize}
    }[\rpmc[149]]
}
% \imgp{tableau-exemple}

