\section*{Conclusion}

Au terme de cette étude approfondie sur l'intégration de la dialectique outil-objet et des jeux de cadres en mathématiques,
il apparaît,
comme le suggère les auteurs des trois articles étudiés,
que ces méthodes enrichissent l'expérience d'apprentissage des élèves.
L'expérimentation menée a permis de mettre en lumière l'efficacité de changer de cadres pour aider les élèves à surmonter leurs difficultés avec les concepts mathématiques.\\

Bien que l'activité sur les puissances n'ait pas complètement atteint tous les objectifs fixés,
notamment en raison de contraintes de temps et de familiarité avec les outils numériques,
elle a néanmoins favorisé chez certains élèves une meilleure compréhension des puissances.
Le dialogue entre les cadres algébriques et géométriques a encouragé les élèves à réfléchir de manière plus critique et créative,
démontrant ainsi le potentiel des approches multimodales dans l'enseignement des mathématiques.\\

La comparaison entre le groupe ayant participé à l'activité et le groupe témoin a également révélé certaines différences en termes de perception de l'utilité des puissances et de l'intérêt pour les cours de mathématiques,
suggérant que les méthodes interactives et les approches fondées sur la manipulation des concepts peuvent rendre l'apprentissage plus engageant et pertinent pour les élèves.\\

Enfin,
ce mémoire souligne l'importance d'une réflexion continue sur nos méthodes didactiques et encourage à intégrer de manière plus systématique des approches innovantes comme celles développées par Régine Douady.
En poursuivant l'exploration et l'adaptation de ces stratégies didactiques,
nous pouvons non seulement améliorer la compréhension mathématique des élèves mais aussi leur donner les outils nécessaires pour aborder les défis mathématiques de manière autonome et confiante.