% VARIABLES %%%
\def\theme{Activités géométriques : Hauteurs et Triangles}
\def\date{04/11/2023}
%%%%%%%%%%%%%%%

\newcommand*{\tri}[4]{
    \item Placer $#4$ sur $(O#2)$, pour construire le triangle #1: $#2#3#4$ rectangle en $#3$.
}

\section{Construction sur feuille blanche}

\begin{enumerate}\setlength{\itemsep}{15pt}%
    \item Construire les droites $(OA)$ et $(OB)$ perpendiculaires,
    avec: $OA = 1\cm$ et $OB = 2\cm$.
    \item \begin{enumerate}
        \tri{1}{A}{B}{C}
        \tri{2}{B}{C}{D}
        \tri{3}{C}{D}{E}
        \item Donner les hauteurs de ces trois triangles.
    \end{enumerate}
    \item \begin{enumerate}
        \item Si l'on construisait suivant le même procédé le triangle 4,
        quelle serait sa hauteur ? 
        \item Du triangle 5, 6 ?
        \item Quelle serait la hauteur du triangle $n$ ?
    \end{enumerate}
    \item \begin{enumerate} 
        \tri{0}{B}{A}{A_1}
        \tri{-1}{A}{A_1}{B_1}
        \item Donner les hauteurs de ces deux triangles.
    \end{enumerate}
\end{enumerate}

\section{Construction sur Géogebra}

\begin{enumerate}
    \item Sur Géogebra, reprendre les questions précédentes pour:
    $OA = 1\cm$ et $OB = 10\cm$.\\
    
    \hint{
        \begin{itemize}
            \item Pour créer un segment d'une longueur donné utiliser l'outil "Cercle (centre-rayon)".
            \item Pour mesurer utiliser l'outil "Distance ou Longueur".
        \end{itemize}
    }\vspace*{0.5cm}
    \item La formule de la hauteur donné pour le triangle $n$, fonctionne-t-elle pour le triangle $0$ ? le triangle $-1$ ? $-2$, $-4$ ?
\end{enumerate}