\begin{enumerate}
    \item \textbf{Contextualisation et Justification de l'Intégration de l'Algèbre et de la Géométrie}
    \begin{enumerate}
        \item \textbf{Introduction aux concepts d'algèbre et de géométrie au cycle 4 :}
        \begin{itemize}
            \item Présentation des principaux thèmes en algèbre et en géométrie.
        \end{itemize}
        
        \item \textbf{Justification de l'intégration de l'algèbre et de la géométrie :}
        \begin{itemize}
            \item Analyse des avantages pédagogiques et cognitifs de l'intégration.
        \end{itemize}
        
        \item \textbf{Problématique et Objectifs de la Recherche :}
        \begin{itemize}
            \item Reformulation de la problématique posée dans l'introduction.
            \item Présentation des objectifs spécifiques de l'étude.
        \end{itemize}
    \end{enumerate}
    
    \item \textbf{Méthodes d'Intégration de l'Algèbre et de la Géométrie}
    \begin{enumerate}
        \item \textbf{Approches pédagogiques pour l'intégration de l'algèbre et de la géométrie :}
        \begin{itemize}
            \item Utilisation de situations-problèmes.
            \item Activités basées sur la modélisation.
            \item Intégration des Technologies de l'Information et de la Communication pour l'Enseignement.
        \end{itemize}
        
        \item \textbf{Exemples d'activités d'apprentissage intégrées :}
        \begin{itemize}
            \item Description d'exemples concrets d'activités pédagogiques intégrées.
            \item Illustration à travers des cas pratiques et des exemples de situations d'apprentissage en classe.
        \end{itemize}
    \end{enumerate}
    
    \item \textbf{Évaluation de l'Impact sur l'Apprentissage des Élèves}
    \begin{enumerate}
        \item \textbf{Méthodologie de l'étude :}
        \begin{itemize}
            \item Description des méthodes de recherche utilisées.
        \end{itemize}
        
        \item \textbf{Résultats de l'étude :}
        \begin{itemize}
            \item Présentation et analyse des résultats obtenus à partir des données recueillies.
        \end{itemize}
    \end{enumerate}
    
    \item \textbf{Recommandations et Perspectives Futures}
    \begin{enumerate}
        \item \textbf{Recommandations pédagogiques :}
        \begin{itemize}
            \item Proposition de recommandations concrètes pour les enseignants.
            \item Suggestions pour l'élaboration de matériel pédagogique intégré et l'organisation de formations professionnelles.
        \end{itemize}
        
        \item \textbf{Limites de l'étude et Perspectives Futures :}
        \begin{itemize}
            \item Discussion sur les limites de la recherche.
            \item Proposition de pistes pour de futures recherches dans le domaine de l'intégration de l'algèbre et de la géométrie.
        \end{itemize}
    \end{enumerate}
\end{enumerate}

\textbf{Conclusion :} Synthèse des principales conclusions de l'étude, récapitulation des contributions de la recherche et des recommandations pratiques pour les enseignants. Ouverture sur l'importance continue de l'intégration de l'algèbre et de la géométrie dans l'enseignement des mathématiques pour favoriser une compréhension approfondie et holistique des concepts mathématiques chez les élèves du cycle 4.
