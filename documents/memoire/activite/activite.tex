% VARIABLES %%%
\def\theme{Activité : Triangles rectangles et hauteurs}
% \def\date{05/04/2024}
% \thispagestyle{assignment}
%%%%%%%%%%%%%%%
\vspace*{-1cm}
\hint{
    \begin{itemize}
        \item Amorce : \textbf{\href{https://www.geogebra.org/classic/xqyzte6y}{amorce\_TRH.ggb}} à modifier,\\
        à rendre sous le nom : \textbf{NOM1\_NOM2\_TRH.ggb}
        \item Répondre aux questions sur le document réponse en page 2.
        \item Toute recherche au brouillon est encouragée et est à écrire sur le document réponse (recto ou au verso).
    \end{itemize}
}\vspace*{0.2cm}

\newcommand*{\tri}[4]{%
    \item Placer $#4$ sur $(O#2)$ pour construire $#2#3#4$ rectangle en $#3$, le triangle #1.%
}

Le fichier amorce présente deux droites $(OA)$ et $(OB)$ perpendiculaires avec $OA=1\cm$ et $OB=2\cm$.
\vspace*{-0.2cm}
\section{$OB = 2cm$}
\begin{enumerate}\setlength{\itemsep}{15pt}%
    \item \begin{enumerate}
        \tri{1}{A}{B}{C}

        \hint{%
            \begin{itemize}%
            \item Pour placer $C$ il faudra d'abord construire une perpendiculaire à $(AB)$ et prendre son intersection avec $OA$.
            \item On s'intéressera dans la suite du problème à la \textbf{valeur} de la \textbf{hauteur issue de l'angle droit} des triangles rectangles.
            \end{itemize}%
        }
        \item Quelle est la hauteur du triangle 1?
    \end{enumerate}
    \item \begin{enumerate}
        \tri{2}{B}{C}{D}
        \tri{3}{C}{D}{E}
        \item Quelles sont les hauteurs de ces deux triangles?
        
        \hint{Pour mesurer utiliser l'outil: "Distance ou Longueur".}
    \end{enumerate}
    \item \begin{enumerate}
        \item Si l'on construisait suivant le même procédé le triangle 4,
        quelle serait sa hauteur ? 
        \item Des triangles 5, 6, 7 et 56 ?
        \item Émettre une conjecture pour la hauteur du triangle $n$ ?
    \end{enumerate}
\end{enumerate}

\section{$OB = 10cm$}

\begin{enumerate}\setlength{\itemsep}{15pt}%
    \item Déplacer le curseur $OB$ pour que $[OB]$ soit de longueur $10\cm$.
    \item \begin{enumerate} \item Quelles sont les nouvelles hauteurs des triangles 1, 2, 3?
            \item Quelles seraient les hauteurs des triangles 4, 9, 123?
            \item \label{itm:first} Quelle serait la hauteur du triangle $n$ ?
    \end{enumerate}
    \item \begin{enumerate} 
        \tri{0}{B}{A}{A_1}
        \tri{-1}{A}{A_1}{B_1}
        \tri{-2}{A_1}{B_1}{C_1}
        \item Quelle sont les hauteurs de ces trois triangles?
        \item \label{itm:second} Est-ce que la formule donné à la question \ref{itm:first}) reste vrai pour ces triangles?
    \end{enumerate}
    \item \begin{enumerate}
        \item Déplacer le curseur $OB$ pour que $[OB]$ soit à nouveau de longueur $2\cm$.
        \item Quelles sont les hauteurs des triangles 0, -1 et -2?
        \item Quelles seraient les hauteurs des triangles -3, -4, -5 et -23?
        \item Pouvez-vous émettre une conjecture sur une nouvelle notation,
        qui donnerait la hauteur du triangle $-n$.
    \end{enumerate}
\end{enumerate}

\section*{Document Réponse}
\thispagestyle{assignment}
\setcounter{section}{0}

\section{OB = 2cm}

\def\tableSpacing{0.15cm}
\def\colWidth{1.5cm}
\renewcommand{\arraystretch}{1.5}

\newcommand{\triangleTable}[1]{
    \listSize{#1}
    \xdef\headerRow{triangle}
    \xdef\heightRow{hauteur}
    \foreach \t in {#1}{
        \expandafter\xdef\expandafter\headerRow\expandafter{\headerRow & \t}
        \expandafter\xdef\expandafter\heightRow\expandafter{\heightRow &}
    }
    \begin{tabular}{|M{2cm}|*{\thesize}{M{\colWidth}|}}
        \hline%
        \headerRow \\
        \hline%
        \heightRow \\[0.35cm]
        \hline%
    \end{tabular}
}

\begin{flushleft}
    \triangleTable{1,2,3}\\
    \vspace*{\tableSpacing}
    \triangleTable{4,5,6,7}\\
    \vspace*{\tableSpacing}
    \triangleTable{56,n}
\end{flushleft}

\section{OB = 10cm}

\subsection{Question 1. à 3.}

\begin{flushleft}
    \triangleTable{1,2,3,4,9,123,n}\\
    \vspace*{\tableSpacing}
    \triangleTable{0,-1,-2} \textbf{\ref{itm:second}})
\end{flushleft}

\subsection{Question 4. OB = 2cm}
\begin{flushleft}
    \triangleTable{0,-1,-2}\\
    \vspace*{\tableSpacing}
    \triangleTable{-3,-4,-5,-23,-n}
\end{flushleft}