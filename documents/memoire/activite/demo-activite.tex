% VARIABLES %%%
\def\theme{Propriétés : Hauteur issue de l'angle droit}
% \def\date{04/11/2023}
%%%%%%%%%%%%%%%

\pr{Hauteur du triangle rectangle}{
    \dividePage{
        \def\a{2.1}
        \begin{center}
            \begin{tikzpicture}[scale=0.6]
                \tkzDefPoint(0,\a**2){A}
                \tkzDefPoint(\a,0){B}
                \tkzDefPoint(-\a**3,0){C}
                \tkzDefPoint(0,0){H}
                \draw[very thick] (A)--(B)--(C)--cycle;
                \tkzMarkRightAngle[scale=1.5](B,A,C)
                \tkzMarkRightAngle[scale=1.5](A,H,C)
                \tkzLabelPoints[above](A)
                \tkzLabelPoints[below right](B)
                \tkzLabelPoints[below left](C)
                \tkzLabelPoints[below](H)
                \draw[very thick, red] (A)--(H);
            \end{tikzpicture}
        \end{center} 
    }{
        Soit $ABC$ rectangle en $A$ et $H$ intersection de $(BC)$ et de la hauteur issue de $A$.
        On a:
        \begin{align*}
            HA ^ 2 = HB \times HC
        \end{align*}
    }
}

\demo{Algébrique}{
    D'après le théorème de Pythagore :
    \begin{align*}%
        AB^2 = HA^2 + HB^2 &\et AC^2 = HA^2 + HC^2
        \ialors AB^2 + AC^2 &= 2 \times HA^2 + HB^2 + HC^2\\
        %
        \intertext{et avec à nouveau le théorème de Pythagore :} AB^2 + AC^2 &= BC^2\\
        &= (HB+HC)^2\\
        &= 2 \times HB \times HC + HB^2 + HC^2\\
        %
        \ialors 2 \times HA^2 + HB^2 + HC^2 &= 2 \times HB \times HC + HB^2 + HC^2
        \idou HA ^ 2 = HB \times HC
    \end{align*}
}

\demo{Géometrique}{
    \def\a{2}
    \dividePage{
        \begin{tikzpicture}[scale=0.6]
            \tkzDefPoint(0,\a**2){A}
            \tkzDefPoint(\a,0){B}
            \tkzDefPoint(-\a**3,0){C}
            \tkzDefPoint(0,0){H}
            %
            \draw[fill = red, fill opacity=0.2] (A)--(H)--(C)--cycle;
            \draw[fill = blue, fill opacity=0.2] (A)--(H)--(B)--cycle;
            %
            \tkzMarkRightAngle[scale=1.5](B,A,C)
            \tkzMarkRightAngle[scale=1.5](A,H,C)
            %
            \draw[very thick] (A)--(B)--(C)--cycle;
            \draw[very thick] (A)--(H);
            %
            \tkzLabelSegment[left](H,A){$h$}
            \tkzLabelSegment[below](H,C){$b$}
            \tkzLabelSegment[below](H,B){$a$}
        \end{tikzpicture}
    }{
        \begin{tikzpicture}[scale=0.6]
            \tkzDefPoint(0,\a**2){A}
            \tkzDefPoint(\a,0){B}
            \tkzDefPoint(-\a**3,0){C}
            \tkzDefPoint(0,0){H}
            %
            \tkzDefPoint(\a**2,\a**2){H'}
            \tkzDefPoint(\a**2,\a**2+\a){B'}
            %
            \draw[very thick, fill = red, fill opacity=0.2] (A)--(H)--(C)--cycle;
            \draw[very thick, fill = blue, fill opacity=0.2] (A)--(H')--(B')--cycle;
            %
            \tkzMarkRightAngle[scale=1.5](A,H,C)
            \tkzMarkRightAngle[scale=1.5](H,A,H')
            \tkzMarkAngle[size=0.5cm,arc=l,red](B',A,C)
            %
            \draw[very thick,dotted] (A)--(B)--(H)--cycle;
            %
            \tkzLabelSegment[left](H,A){$h$}
            \tkzLabelSegment[below](H,C){$b$}
            %
            \tkzLabelSegment[below](A,H'){$h$}
            \tkzLabelSegment[right](B',H'){$a$}
            %
            \tkzLabelAngle[pos=1.2,red](B',A,C){$\alpha = 90^\circ+90^\circ$}
        \end{tikzpicture}
    }
    %
    %
    \dividePage{
        \begin{tikzpicture}[scale=0.6]
            \tkzDefPoint(0,\a**2){A}
            \tkzDefPoint(\a,0){B}
            \tkzDefPoint(-\a**3,0){C}
            \tkzDefPoint(0,0){H}
            %
            \tkzDefPoint(\a**2,\a**2){H'}
            \tkzDefPoint(\a**2,\a**2+\a){B'}
            %
            \tkzDefPoint(-\a**3,\a**2){H''}
            \tkzDefPoint(0,\a**2+\a){H'''}
            \tkzDefPoint(-\a**3,\a**2+\a){D}
            \tkzDefPoint(\a**2,0){E}
            %
            \draw[very thick, fill = red, fill opacity=0.2] (A)--(H)--(C)--cycle;
            \draw[very thick, fill = blue, fill opacity=0.2] (A)--(H')--(B')--cycle;
            \draw[fill = red, fill opacity=0.1] (A)--(H'')--(C)--cycle;
            \draw[fill = blue, fill opacity=0.1] (A)--(H''')--(B')--cycle;
            %
            \tkzMarkRightAngle[scale=1.5](A,H,C)
            \tkzMarkRightAngle[scale=1.5](H,A,H')
            %
            \draw[very thick] (A)--(H)--(C)--cycle;
            \draw[very thick] (A)--(H')--(B')--cycle;
            %
            % \draw[thick] (A)--(C)--(H'')--cycle;
            % \draw[thick] (A)--(B')--(H''')--cycle;
            % \draw[thick,dashed] (H'')--(D)--(H''');
            % \draw[thick,dashed] (H')--(E)--(H);
            \draw[thick,Green] (E)--(C)--(D)--(B')--cycle;
            %
            \tkzLabelSegment[left](H,A){$h$}
            %
            \tkzLabelSegment[above](A,H'){$h$}
            \tkzLabelSegment[right](A,H'''){$a$}
            \tkzLabelSegment[below](A,H''){$b$}
            \node[red] at (0.5*\a^2,0.5*\a^2) {$h^2$};
            \node[red] at (-0.5*\a^3,\a^2 + 0.5*\a) {$a \times b$};
        \end{tikzpicture}
    }{
        \begin{align*}
            a \times b = \frac{\color{Green}\mathcal{A}}{2} - \color{red}\mathcal{A}\color{black} - \color{blue}\mathcal{A}\color{black} = h^2
        \end{align*}
    }
}

\cor{}{
    \dividePage{
        Soit $a\in\m{R}, n\in\m{Z}$ et la figure suivante :\\ \\
        \vspace{1cm}
        alors $x= a^{n+1}$
    }{
        \def\a{1.9}
        % \begin{center}
            \begin{tikzpicture}[scale=0.6]
                \tkzDefPoint(0,\a**2){A}
                \tkzDefPoint(-\a**3,0){B}
                \tkzDefPoint(\a,0){C}
                \tkzDefPoint(0,0){H}
                \draw[very thick] (A)--(B)--(C)--cycle;
                \tkzMarkRightAngle[scale=1.5](B,A,C)
                \tkzMarkRightAngle[scale=1.5](B,H,A)
                \tkzLabelSegment[left](H,A){$a^{n}$}
                \tkzLabelSegment[below](H,C){$a^{n-1}$}
                \tkzLabelSegment[below, yshift = -5pt](H,B){$x$}
                \draw[very thick] (A)--(H);
            \end{tikzpicture}
        % \end{center}
    }
}
    
\demo{}{%
    \begin{align*}
        \iOna (a^n)^2 &= x \times a^{n-1}\\
        \ialors x &= \frac{(a^n)^2}{a^{n-1}}\\
        &= \frac{a^{2n}}{a^{n-1}}\\
        &= a^{2n - (n-1)} = a^{n+1}
    \end{align*}
}