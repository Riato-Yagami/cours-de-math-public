% VARIABLES %%%
\def\authors{Jules PESIN}
\date{mercredi 14 mai 2024}
\def\longTitle{Exploration des jeux de cadres et de la dialectique outil-objet appliqués à la remédiation dans l'apprentissage des puissances au collège}
\def\shortTitle{Jeux de cadres \& dialectique outil objet}%% - remédiation puissances collège}
\def\theme{\longTitle}
%%

% \disableAnimation
% \shortAnimation
% \firstSlide

\def\imgPath{stage-M2/4e/translation/decouverte-translation/}
\def\imgExtension{.png}

\maketitle

\section*{Introduction}
\slide{}{
    \begin{itemize}
        \item Collège Janson de Sailly.
        \item Classe de $4^e$.
        \item Séquence sur les puissances $2$ mois avant le début du mémoire.
        \item Quelques difficultés rencontrées.
        \item Recherche d’une méthode de rémédiation.
    \end{itemize}
}
\section*{Plan}
\slide{}{
    \tableofcontents
}

\section{Retour sur le mémoire}
\subsection{Résumé théorique}
\slide{}{
    \begin{itemize}
        \item Régine Douady:
        \textit{Jeux de cadre et dialectique outil-objet}
        \item \textbf{Jeux de cadre}:
        Explorer et comprendre les concepts mathématiques en \textbf{alternant} entre différents \textbf{cadres} ou \textbf{perspectives} (algébrique, géométrique, numérique...)
        \item \textbf{Dialectique outil-objet}:
        Les concepts mathématiques servent à la fois d'\textbf{outils} pour \textbf{effectuer des opérations} et \textbf{résoudre des problèmes},
        et d'\textbf{objets d'étude théoriques}.
    \end{itemize}
}
\subsection{Première expérimentation}
\slide{Méthodologie}{
    \begin{itemize}
        \item Régine Douady:
        \textit{Enseignement de la dialectique outil-objet et des jeux de cadres en formation mathématique des professeurs d’école}
        \item Bernard Capponi et Rosamund Sutherland:
        \textit{Interaction des cadres algébriques et graphiques dans la résolution de problèmes}
        \item Guy Brousseau :
        \textit{Fondements et méthodes de la didactique des mathématiques}
    \end{itemize}
}

\slide{Activité}{
    \begin{itemize}
        \item Activité géogebra.
        \item Hauteurs issue de l'angle droit (cadre géométrique)
        \item Mesure de la hauteur (cadre numérique)
        \item Passage à une formule de calcule de la hauteur du triangle $n$ : $2^n$ (cadre algébrique).
    \end{itemize}
}

\slide{}{
    \begin{itemize}
        \item Puissances utilisé comme \textbf{outil} pour calculer des hauteurs de triangle
        et comme \textbf{objet} pour la propriété de passage d'un triangle au suivant.
    \end{itemize}
    \pr{Hauteur du triangle rectangle}{
        \small
        \def\a{2.1}
        \begin{columns}[T]
            \begin{column}{.45\textwidth}
                \begin{center}
                    \begin{tikzpicture}[scale = 0.25]
                        \tkzDefPoint(0,\a**2){A}
                        \tkzDefPoint(\a,0){B}
                        \tkzDefPoint(-\a**3,0){C}
                        \tkzDefPoint(0,0){H}
                        \draw[very thick] (A)--(B)--(C)--cycle;
                        \tkzMarkRightAngle[scale=1.5](B,A,C)
                        \tkzMarkRightAngle[scale=1.5](A,H,C)
                        \tkzLabelPoints[above](A)
                        \tkzLabelPoints[below right](B)
                        \tkzLabelPoints[below left](C)
                        \tkzLabelPoints[below](H)
                        \draw[very thick, red] (A)--(H);
                    \end{tikzpicture}
                \end{center}
            \end{column}
            \begin{column}{.5\textwidth}
                Soit $ABC$ rectangle en $A$ et $H$ intersection de $(BC)$ et de la hauteur issue de $A$.
                On a:
                \begin{align*}
                    HA ^ 2 = HB \times HC
                \end{align*}
            \end{column}
        \end{columns}
    }
}

\section{Seconde expérimentation}
\subsection{Analyse a priori}
\slide{}{
}
\subsection{Analyse a posteriori}
\slide{}{
    4
}

\section*{Conclusion}
\slide{}{
    B
}