\section*{Introduction}

Améliorer le niveau de nos élèves est un défi majeur pour tous les enseignants de France,
particulièrement dans le domaine des mathématiques où les performances,
selon certaines études et évaluations internationales,
semblent être en déclin.
Il devient nécessaire de développer de nouvelles stratégies pédagogiques,
sans pour autant remettre en cause notre système éducatif,
fruit de nombreuses années de recherche et d'expérimentation didactique.\\

Une des pistes explorées par des chercheurs et des enseignants concerne la dialectique outil-objet et les jeux de cadres.
Ces concepts,
largement développés par Régine Douady,
offrent une nouvelle perspective dans l'approche des mathématiques.
À travers ce mémoire, nous nous appuierons sur deux articles majeurs de Douady,
« Jeux de cadre et dialectique outil-objet » (1986) \cite{douady1} et « Enseignement de la dialectique outil-objet et des jeux de cadres en formation mathématique des professeurs d'école » (1993) \cite{douady2},
afin de mieux comprendre les enjeux de ces notions.
Elles visent à multiplier les perspectives mathématiques à travers lesquelles un problème est étudié,
pour en améliorer la compréhension et ainsi renforcer l'ancrage des connaissances chez les élèves.
Nous examinerons également l'article de Bernard Capponi et Rosamund Sutherland,
« Interaction des cadres algébriques et graphiques dans la résolution de problèmes » (1999) \cite{capponi},
qui met en avant l'utilisation du logiciel Cabri-Géomètre pour enrichir l'approche algébrique des problèmes.\\

Bien que ces concepts soient de plus en plus intégrés en classe,
les tâches de type recherche restent souvent limitées à la découverte de nouvelles connaissances.
D'après mes observations lors de stages en classe,
un problème récurrent est le réinvestissement des connaissances par les élèves.
Ce mémoire vise donc à explorer comment l'utilisation de changements de cadres peut conduire à une connaissance mieux ancrée chez l'élève,
permettant de construire un savoir plus durable.
Nous chercherons notamment à éliminer certaines conceptions erronées qui auraient pu être internalisées par les élèves en fin de séquence.\\

L'expérimentation que je vais mener,
dans un but de remédiation,
se concentrera sur la séquence des puissances en classe de 4e.
Elle s'inspirera des travaux de Sutherland et Capponi en mettant en lumière l'interaction entre les cadres graphiques et algébriques,
au seins d'un problème géométrique.
Cette expérimentation consistera à manipuler les hauteurs des triangles rectangles à l'aide du logiciel Géogebra,
permettant ainsi d'explorer de façon concrète les concepts mathématiques abordés.