% VARIABLES %%%
\def\theme{Hauteur du triangle rectangles}
\def\date{04/11/2023}
%%%%%%%%%%%%%%%

\pr{Hauteur du triangle rectangle}{
    \dividePage{
        \def\a{2.1}
        \begin{center}
            \begin{tikzpicture}[scale=0.6]
                \tkzDefPoint(0,\a**2){A}
                \tkzDefPoint(-\a**3,0){B}
                \tkzDefPoint(\a,0){C}
                \tkzDefPoint(0,0){D}
                \draw[very thick] (A)--(B)--(C)--cycle;
                \tkzMarkRightAngle[scale=1.5](B,A,C)
                \tkzMarkRightAngle[scale=1.5](B,D,A)
                \tkzLabelPoints[above](A)
                \tkzLabelPoints[below left](B)
                \tkzLabelPoints[below right](C)
                \tkzLabelPoints[below](D)
                \draw[very thick, red] (A)--(D);
            \end{tikzpicture}
        \end{center} 
    }{
        Soit $ABC$ rectangle en $A$ et $D$ intersection de $(BC)$ et de la hauteur issue de $A$.
        On a:
        \begin{align*}
            DA ^ 2 = DB \times DC
        \end{align*}
    }
}

\demo{}{
    D'après le théorème de Pythagore :
    \begin{align*}%
        AC^2 = DA^2 + DC^2 &\et AB^2 = DA^2 + DC^2
        \ialors AB^2 + AC^2 &= 2 DA^2 + DC^2 + DB^2\\
        %
        \intertext{et avec à nouveau le théorème de Pythagore :} AB^2 + AC^2 &= BC^2\\
        &= (DB+DC)^2\\
        &= 2 \times DB \times DC + DB^2 + DC^2\\
        %
        \ialors 2 \times DA^2 + DC^2 + DB^2 &= 2 \times DB \times DC + DB^2 + DC^2
        \idou DA ^ 2 = DB \times DC
    \end{align*}
}

\cor{}{
    \dividePage{
        Soit $a\in\m{R}, n\in\m{Z}$ et la figure suivante :\\ \\
        \vspace{1cm}
        alors $x= a^{n+1}$
    }{
        \def\a{1.9}
        % \begin{center}
            \begin{tikzpicture}[scale=0.6]
                \tkzDefPoint(0,\a**2){A}
                \tkzDefPoint(-\a**3,0){B}
                \tkzDefPoint(\a,0){C}
                \tkzDefPoint(0,0){D}
                \draw[very thick] (A)--(B)--(C)--cycle;
                \tkzMarkRightAngle[scale=1.5](B,A,C)
                \tkzMarkRightAngle[scale=1.5](B,D,A)
                \tkzLabelSegment[left](D,A){$a^{n}$}
                \tkzLabelSegment[below](D,C){$a^{n-1}$}
                \tkzLabelSegment[below, yshift = -5pt](D,B){$x$}
                \draw[very thick] (A)--(D);
            \end{tikzpicture}
        % \end{center}
    }
}
    
\demo{}{%
    \begin{align*}
        \iOna (a^n)^2 &= x \times a^{n-1}\\
        \ialors x &= \frac{(a^n)^2}{a^{n-1}}\\
        &= \frac{a^{2n}}{a^{n-1}}\\
        &= a^{2n - (n-1)} = a^{n+1}
    \end{align*}
}