% VARIABLES %%%
\def\theme{Diagonalisation}
\def\date{09/01/2023}
%%%%%%%%%%%%%%%

% DEFINITIONS %
\def\ev{espace vectoriel }
\def\endo{endomorphisme }
\def\vp{vecteur propre }
\def\vlp{valeur propre }
\def\lb{\ensuremath{\lambda}}
\def\plc{polynôme caractéristique }
\def\sep{sous-espace propre }
%%%%%%%%%%%%%%

$\m{K} = \m{R} \ou \m{K} = \m{C}$ dans le reste de la leçon.
Soient \m{E} un \ev fini sur \m{K},
et $f$ \endo de \m{E}.

\df{Endomorphisme diagonalisable}{
    On dit que $f$ est \textbf{diagonalisable} s'il existe une base de \m{E} telle que:
    la matrice de $f$ par rapport à cette base soit diagonale.
}[\href{https://uel.unisciel.fr/mathematiques/reducmat1/reducmat1_ch01/co/apprendre_ch1_02.html}{Unisciel}]

\df{Vecteur propre}{
    Un vecteur $v$ de \m{E} est appelé \textbf{\vp} de $f$ s'il il verifie les deux conditions:
    \begin{enumerate}
        \item $v \neq 0$,
        \item il existe $\lb\in\m{K}$ tel que $f(v) = \lb v$. 
    \end{enumerate}
}

\df{Valeur propre}{
    $\lb\in\m{K}$ est appelé \textbf{\vlp} de $f$ s'il existe un vecteur $v\neq0$,
    tel que $f(v)=\lb v$.
}

\pr{Caracatérisation d'une \vlp}{
    $\lb\in\m{K}$ est une \vlp de $f$ \ssi :
    $\det(f-\lb \Id_{\m{E}}) = 0$.
}

\df{Polynôme caractéristique}{
    Si \m{E} de dimension $n\geqslant1$.
    Soit $A$ la matrice associée à $f$ par rapport à une base de \m{E}.\\
    Le polynôme $\chi_f = \det(f-X\Id_{\m{E}})$ égale à $\det(A-X\I_n)$;
    est appelé \textbf{\plc} de $f$.
}

\rmk{Coéfficient du \plc}{
    C'est polynôme à coefficient dans \m{K},
    de degré $n$,
    dont le coefficient dominant est $(-1)^n$.
}

\pr{Valeur propre et \plc}{
    $\lb\in\m{K}$ est une \vlp de $f$ \ssi il est racine du \plc de $f$.
}

\df{Sous-espace propre associée à une \vlp}{
    Soit \lb une \vlp de $f$.
    On appelle \sep associée à la \vlp \lb :
    le noyau $\m{E}(\lb) = \Ker(f-\lb \Id_{\m{E}})$.
}