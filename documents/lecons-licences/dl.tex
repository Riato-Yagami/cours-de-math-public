% VARIABLES %%%
\def\theme{Statistique à une ou deux variables, représentation et analyse de données}
\def\date{02/12/2023}
%%%%%%%%%%%%%%%

% DEFINITIONS %
\def\eq{&\operatorname*{=}_{x \rightarrow 0}}
\def\ppp{+ ... +}
\newcommand{\pdl}[1]{
    + \dl{#1}\\
}
%%%%%%%%%%%%%%

\hbox{\vspace{-0.7cm}}

\subsection{Développements limités usuels}
\vspace{-1cm}
\begin{align*}
    e^x \eq 1 + x + \frac{x^2}{2} \ppp \frac{x^n}{n!} \pdl{n}
    \ch(x) \eq 1 + \frac{x^2}{2} \ppp \frac{x^{2n}}{(2n)!} \pdl{2n}
    \sh(x) \eq x + \frac{x^3}{6} \ppp \frac{x^{2n+1}}{(2n+1)!} \pdl{2n+1}
    \cos(x) \eq 1 - \frac{x^2}{2} \ppp (-1)^{2n}\frac{x^{2n}}{(2n)!} \pdl{2n}
    \sin(x) \eq x - \frac{x^3}{6} \ppp (-1)^{2n+1}\frac{x^{2n+1}}{(2n+1)!} \pdl{2n+1}
    \tan(x) \eq x + \frac{x^3}{3} + \frac{2x^5}{15} + \frac{17x^7}{315} \pdl{7}
    \frac{1}{1-x} \eq 1 + x + x^2 \ppp x^n \pdl{n}
    \frac{1}{1+x} \eq 1 - x + x^2 \ppp (-1)^{n} x^n \pdl{n}
    \ln(1+x) \eq x - \frac{x^2}{2} \ppp (-1)^{n}\frac{x^{n}}{(n)!} \pdl{n}
    \ln(1-x) \eq x + \frac{x^2}{2} \ppp \frac{x^{n}}{(n)!} \pdl{n}
    \arctan(x) \eq x - \frac{x^3}{3} \ppp (-1)^{n}\frac{x^{2n+1}}{2n+1} \pdl{2n+1}
    \argth(x) \eq x + \frac{x^3}{3} \ppp \frac{x^{2n+1}}{2n+1} \pdl{2n+1}
    (1+x)^\alpha \eq  1 + \alpha x + \frac{\alpha(\alpha-1)}{2}x^2 
    \ppp \frac{\alpha(\alpha-1) ... (\alpha-(n-1))}{n!}x^n \pdl{n}
    &= \sum_{k=0}^{n}\binom{\alpha}{k}x^k \pdl{n}
\end{align*}