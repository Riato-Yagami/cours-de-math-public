% VARIABLES %%%
\def\theme{Pourcentages et taux d'évolution.}
\def\date{23/10/2023}
%%%%%%%%%%%%%%%

\hbox{}
\infoLecon{
    \csname 5e\endcsname (Pourcentages)\\
    \csname 3e\endcsname
    /\csname2e\endcsname (Taux d'évolution)\\
    \csname 1re\endcsname (Evolution successives)
}{
    Proportionnalité, fractions, suites
}

\df{Pourcentage}
Le \textbf{pourcentage} d'une partie d'un ensemble, 
est le rapport d'une mesure de cette partie à la mesure correspondante de l'ensemble total, 
exprimé sous la forme d'une fraction de cent.

\df{Taux d'évolution}
Si une grandeur passe d'une valeur de départ $V_D$ à une valeur d'arrivée $V_A$, 
le \textbf{taux d'évolution} est donné en pourcentage par la formule :
\begin{equation*}
    t = \frac{V_A-V_D}{|V_D|}
\end{equation*}

\tice{Tableur}{Evolution successives à taux constant}
\csname 4e\endcsname
\csname 5e\endcsname
% \expandafter\def\csname $#&\endcsname{Hello!}
% \csname $#&\endcsname