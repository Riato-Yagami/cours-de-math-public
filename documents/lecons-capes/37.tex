% VARIABLES %%%
\def\theme{Primitives, équations différentielles.}
\def\date{11/11/2023}
\setboolean{demonstration}{true}
%%%%%%%%%%%%%%%

\hbox{}
\infoLecon{
    \niv{Terminale Option Spécialité}
    \niv{Terminale Option Complémentaire}
}{
    \item Dérivation
    \item Dérivées usuelles
}

\df{Équation différentielle}{
    En mathématiques, une \textbf{équation différentielle}
    est une équation dont la ou les « inconnue(s) » sont des fonctions.\\
    Elle se présente sous la forme d'une relation entre
    ces fonctions inconnues et leurs dérivées successives.\\
    C'est un cas particulier d'équation fonctionnelle.
}

\df{Primitives}{
    Soit $f$ une fonction dérivable sur $\m{I}$.\\
    On appelle \textbf{primitives} de $f$ sur $\m{I}$ toutes les fonctions $F$,\\
    telles que $F' = f$.
}

\thm{Fondamental de l'Analyse}{
    Si $f$ est une fonction continue sur $[a,b]$\\
    et si, \pt $x\in[a,b]$ on pose:\\
    \begin{equation*}
        F(x) = \int_{a}^{x} f(t)dt
    \end{equation*}
    alors $F$ est dérivable sur $[a,b]$ et $F'(x)=f(x)$ \pt $x\in[a,b]$
}

\newcommand{\intxx}[1]{\int_{x_0}^{x_0+h} #1 dt}
\def\intf{\intxx{f(t)}}
\def\acrF{\acroissement{F}{x_0}}

\demo{Terminale S - On suppose f positive et croissante}{
    On calcule le taux d'acroissement de $F$ en $x_0$. Soit $h>0$, alors:
    \begin{align*}
        \acrF &= \frac{1}{h}\intf\\
        \iet hf(x_0) \leqslant \intf &\leqslant hf(x_0 + h) \car f\geqslant0 \et \textrm{croissante.}
        \ialors f(x_0) \leqslant \acrF &\leqslant f(x_0+h)
        \ialors \lim_{h\rightarrow0^+} \acrF &= f(x_0) \car f \textrm{ continue en } x_0.
    \end{align*}
    On procède de la meme maniere si $h<0$.
}

\demo{Cas général}{
    Soit $\varepsilon > 0$. Il existe $\delta > 0$ tel que,
    si $|x-x_0|<\delta$, alors $|f(x)-f(x_0)|<\varepsilon$\\
    Soit $|h|<\delta$, alors:
    \begin{align*}
        \acrF &= \frac{1}{h}\intf\\
        \ialors \acrF - f(x_0) &= \frac{1}{h} - \intxx{f(t)-f(x_0)}
    \end{align*}
    $|h|<0$, alors \pt{} $t\in[x_0,x_0+h]$, on a $|t-x_0| < \delta$\\
    et donc $|f(t)-f(x_0)| \leqslant \varepsilon$. Alors:
    \begin{align*}
        |\acrF - f(x_0)| &\leqslant \frac{1}{h}\intxx{|f(t)-f(x_0)|}\\
        &\leqslant \frac{1}{h} \intxx{\varepsilon}\\
        &\leqslant \varepsilon\\
        \idou \lim_{h\rightarrow0} \acrF &= f(x_0).
    \end{align*}
}

\pr{}{
    Toute fonction continue sur un intervalle $\m{I}$ admet des primitives sur $\m{I}$.
}

\demo{Terminale S}{
    D'après le \refsec{thm}{Fondamental de l'Analyse}, 
    toute fonctions continues et positive admet une primitive.\\
    Soit $f:[a,b]->\m{R}$ continue. $f$ admet un minimum $m$ sur $[a,b]$.\\
    Posons $g: x \mapsto f(x) - m$. 
    Alors $g$ est continue positive sur $[a,b]$ 
    et admet alors une primitive $G$ sur $[a,b]$.\\
    Posons $F: x \mapsto G(x)+mx$.
    Alors $F$ est dérivable sur $[a,b]$ et \pt $x\in[a,b]$:
    \begin{align*}
        F'(x) &= G'(x) + m\\
        &= g(x) + m\\ 
        &= f(x)
    \end{align*}
    Ainsi, $F$ est une primitive de $f$ sur $[a,b]$
}

\pr{}{
    Soient un intervalle I et $f: \m{I} \rightarrow \m{R}$ une fonction continue.
    Soient $F$ et $G$ deux primitives de $f$ sur $\m{I}$.\\
    Alors il existe $c\in\m{R}$ tel que, \pt $x\in\m{I}, F(x) = G(x)+c$.
}

\demo{}{
    Soient $F$ et $G$ deux primitives de $f$ sur $\m{I}$.
    \begin{align*}
        (F-G)' &= F'-G'\\ 
        &= f-f\\
        &= 0
    \end{align*}
    Alors $F-G$ est une fonction de dérivée nulle sur $\m{I}$.\\
    D'où $F-G$ constante sur $\m{I}$.\\
    Alors il existe $c\in\m{R}$ tel que, \pt $x\in\m{I}, F(x) = G(x)+c$.
}
