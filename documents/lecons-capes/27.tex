% VARIABLES %%%
\def\theme{Fonctions polynômes du second degré. Équations et inéquations du second degré.}
\def\date{11/10/2023}
%%%%%%%%%%%%%%%

\hbox{}
\leconInfo{
    \niv{1re}
}{
    \item Equation \item inéquation \item fonction.
}

\df{Fonctions polynômes du second degré}

Une \textbf{fonction polynômes du second degré} est une fonction de la forme:
\begin{equation*}
    x \mapsto ax^2+bx+c
\end{equation*}
avec $a,b,c\in\m{R}$ et $a\ne0$.

\rmq{}
Une \poly2 est définie sur $\m{R}$.

\df{Racine d'une fonction}
Soit $f$ une fonction définie sur un intervalle $\m{I}$. On dit que $x_0\in\m{I}$ est une \textbf{racine} de $f$ si $f(x_0)=0$

\df{Forme canonique}
Toute \poly2 $f$ possède une \textbf{forme canonique} où la variable $x$ n'apparaît qu'une seule fois:
\begin{equation*}
    f(x) = a(x+\frac{b}{2a})^2 - \frac{b^2-4ac}{4a}
\end{equation*}

\rmq{}
\begin{itemize}
    \item $(-\frac{b}{2a},f(-\frac{b}{2a})) = (-\frac{b}{2a},-\frac{b^2-4ac}{4a})$ 
sont les coordonnés du sommet de la parabole représentative de $f$.
    \item $\discr$ est appelée \textbf{discriminant} et est noté $\Delta$.
    \item La forme canonique permet d'écrire la \poly2 comme composée de fonction affines avec une fonction carré.
et donne des résultats interessants sur la fonction.
\end{itemize}

\newpage
\hbox{}
\df{Équation du second degré}
Une \textbf{\eq2} est une équation de la forme $f(x) = 0$ où $f$ est une \poly2.

\df{Racines d'une \eq2}
On dit que $r$ est une racine de l'équation de $f$ si $f(r)=0$.

\pr{Résolution d'une \eq2}
On s'interresse à l'\eq2 : $f(x)=0$ avec $x\in\m{R}$
\begin{itemize}
    \item si $\Delta>0$ alors $f$ possède deux racines distinctes:
    \begin{equation*}
        r_1 = \frac{-b-\sqrt{\Delta}}{2a} \et r_2 = \frac{-b+\sqrt{\Delta}}{2a}
    \end{equation*}
    \item si $\Delta=0$ alors $f$ possède une racine double
    \begin{equation*}
        r_0 = \frac{-b}{2a}
    \end{equation*}
    \item si $\Delta<0$ alors $f$ ne possède pas de racine dans $\m{R}$
\end{itemize}

\df{Forme factorisé}
Une \poly2 peut parfois s'écrire sous une des \textbf{forme factorisées} suivantes:
\begin{itemize}
    \item si $\Delta>0$
    \begin{equation*}
        f(x) = a(x-r_1)(x-r_2)
    \end{equation*}
    \item si $\Delta=0$
    \begin{equation*}
        f(x) = a(x-r_0)^2
    \end{equation*}
\end{itemize}

\rmq{}
Soit $f$ une \poly2 admettant deux racines $r_1,r_2\in\m{R}$, on a:
\begin{equation*}
    r_1+r_2 = \frac{-b}{a} \et r_1\times r_2 = \frac{c}{a}
\end{equation*}

\newpage
\hbox{}

\df{Inéquation du second degré}
Une \textbf{in\eq2} est une inéquation de la forme:\\
$f(x) > 0$, $f(x) < 0$, $f(x) \geqslant 0$ ou $f(x) \leqslant 0$\\
où $f$ est une \poly2.

\pr{Résolution d'une in\eq2}
\begin{itemize}
    \item si $\Delta>0$ alors $f(x)$ est du signe de $a$ sur $]r_1,r_2[$ nul en $r_1$ et $r_2$
    et du signe de $-a$ à sur $]-\infty,r_1[\bigcap]r_2,\infty[$
    \item si $\Delta=0$ alors $f(x)$ est du signe de $a$ sur $\m{R\backslash\{r_0\}}$ et nul en $r_0$
    \item si $\Delta<0$ alors $f(x)$ est du signe de $a$ sur $\m{R}$
\end{itemize}
