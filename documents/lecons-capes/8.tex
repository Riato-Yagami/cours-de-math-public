% VARIABLES %%%
\def\theme{Congruences dans $\m{Z}$}
\def\date{02/12/2023}
%%%%%%%%%%%%%%%

% DEFINITIONS %

%%%%%%%%%%%%%%

% \setboolean{outline}{true}
% \setboolean{demonstration}{false}

\hbox{}
\leconInfo{
    \ilink{Terminale Option Expertes}
}
[]
[]
[]

\pr{Relation d'équivalence}{
    La relation "$\equiv$" est une relation d'équivalence.
}

\pr{Somme et produit de congruences}{
    La relation "$\equiv$" est compatible avec les opérations $+$ et $\times$.
    \begin{align*}
        \lfbrace{ a &\equiv b[n] \\ c &\equiv d[n]}
        \Rightarrow 
        \lfbrace{ a + c &\equiv b + d[n] \\ a \times c &\equiv b \times d[n]}
    \end{align*}
}

\thm{Petit théorème de Fermat}{
    Si $p$ premier et $a\in\m{Z}$, alors $a^p\equiv a[p]$.\\
    De plus si $p\nmid a$, alors $a^{p-1}\equiv 1 [p]$.
}

\df{Inverse modulaire}{
    $a\in\m{Z}$ est \textbf{inversible modulo $n$}$\in\m{N}$,
    lorsqu'il existe un $b\in\m{Z}$ tel que $ab\equiv 1[n]$.\\
    On appelle alors $b$, son \textbf{inverse modulo $n$}.
    \begin{equation*}
        b \equiv a^{-1} [n]
    \end{equation*}
}

\pr{Existence d'un inverse modulo $n$}{
    $a\in\m{Z}$ est inversible modulo $n\in\m{N}$ \ssi{}
    $\pgcd{a}{n} = 1$
}

\mthd{Recherche d'inverse}{
    Pour trouver un inverse $u$ de $a$ modulo $n$.\\
    On cherche un couple $u,v$, tel que $au+nv= 1$ à l'aide de l'algorithme d'euclide.
}

\thm{Des restes chinois}{
    Soient $(n_i)_{i\in{[|1,k|]}}\in\m{N}$ premiers entre eux avec $k\in\m{k}$.\\
    Alors pour tout $(a_i)_{i\in{[|1,k|]}}\in\m{N}$,\\
    il existe un $x\in\m{N}$ unique modulo $N= \prod_{i=1}^{k}n_i$, tel que:
    \begin{align*}
        x\equiv a_i [n_i], \forall i \in [|1,k|]
    \end{align*}
    $x$ est donné par la formule:
    \begin{equation*}
        x = \sum_{i=1}^{k} a_i N_i y_i
    \end{equation*}
    avec $N_i = N/n_i$ et $y_i \equiv N_i^{-1}[n_i]$ \pt{} $i\in[|1,k|]$.
}

% \mthd{}{
%     \TODO
% }