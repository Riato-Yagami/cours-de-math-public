% VARIABLES %%%
\def\theme{Statistique à une ou deux variables, représentation et analyse de données}
\def\date{02/12/2023}
%%%%%%%%%%%%%%%

% DEFINITIONS %
\def\ss{série statistique}
\def\ssd{\ss{} à deux variables}
\def\sump{\sum_{i=1}^{p}}
\def\sumN{\sum_{i=1}^{N}}
\def\sN{\frac{1}{N}}
\newcommand{\moy}[1]{
    \overline{#1}
}
%%%%%%%%%%%%%%

\setboolean{outline}{true}
% \setboolean{demonstration}{false}

\hbox{}
\infoLecon{
    \ilink{2e}
    \ilink{1re}
    \ilink{Terminale Option Complémentaire}
}
[]
[\ilink{Traitement des données}]
[   
    Offre une méthode rigoureuse pour comprendre et interpréter les
    données.
    Fournit les outils nécessaires pour révéler des tendances, 
    des modèles et des relations au sein des phénomènes observés.
]

\section{Série statistique à une variable}

\subsection{Définition}

\df{Série statistique}{
    Une \textbf{\ss}
    est une liste de \textbf{valeurs};
    mesures d'un \textbf{caractère}
    d'une \textbf{population} et provient généralement d'une étude.
}

\df{Effectif}{
    \begin{itemize}
        \item L'\textbf{effectif} $n_i$ correspond au nombre de fois où une valeur à était obtenue.
        \item L'\textbf{effectif total} $N$ est la taille de la population étudié.
    \end{itemize}
}

\rmk{Représentation \ss}{
    On peut représenter les \ss{} de différentes manières:
    \begin{itemize}
        \item avec une liste
        \item dans un tableau
        \item dans un graphique
    \end{itemize}
}

\subsection{Fréquence}

\df{Fréquence}{
    La \textbf{fréquence} d'une valeur est donnée par l'effectif de la valeur divisée par l'effectif total.
    \begin{align*}
        \textrm{fréquence} &= \frac{\textrm{effectif}}{\textrm{effectif total}}\\
        f_{x_i} &= \frac{n_i}{N}, \forall i \in [|1;n|]
    \end{align*}
}

\pr{Somme des fréquences}{
    La somme de toutes les fréquences d'une \ss{} est égale à 1.
    \begin{equation*}
        \sump f_{x_i} = 1
    \end{equation*}
}

\pr{Fréquence de fréquence}{
    Soient $P$ une population et $A$ et $B$ deux sous-population de $P$,
    avec $B\subset A$.\\
    Si $A$ est de fréquence $x$ dans $P$ 
    et $B$ de fréquence $y$ dans $A$,\\
    alors $B$ est de fréquence $x\times y$ dans $P$.
}

\subsection{Distribution}

\df{Moyenne}{
    La \textbf{moyenne} d'une \ss{}
    est donnée par la somme de ces valeurs divisée par l'effectif total.
    \begin{align*}
        \textrm{moyenne} &= \frac{\textrm{somme des valeurs}}{\textrm{effectif total}}\\
        \moy{x} &= \sN \sumN x_i\\
        &= \sN \sump n_i x_{a_i} \textrm{ (\textbf{moyenne pondérée} avec les $x_{a_i}$ tous différent)}
    \end{align*}
}

\pr{Linéarité de la moyenne}{
    Soient $a,b\in\m{R}^2$.\\
    Si la \ss{} $x$ a pour moyenne $\moy{x}$, 
    alors la \ss{} $ax+b$ a pour moyenne $a\moy{x}+b$.
}

\df{Etendue}{
    Dans une \ss,
    l'\textbf{étendue} est la différence entre la plus grande valeur et la plus petite valeur.
}

\df{Médiane}{
    Dans une \ss,
    la \textbf{médiane} est la valeur telle que:
    \begin{itemize}
        \item au moins la moitié des valeurs lui sont supérieures ou égales.
        \item au moins la moitié des valeurs lui sont inférieures ou égales.
    \end{itemize}
}

\df{Quartiles}{
    Dans une \ss,
    on définie 2 \textbf{quartiles}:
    \begin{itemize}
        \item le \textbf{premier quartile} $Q_1$,
        tel que au moins $25\%$ des valuers lui soient inférieures ou égales.
        \item le \textbf{troisième quartile} $Q_3$,
        tel que au moins $75\%$ des valuers lui soient inférieures ou égales.
    \end{itemize}
}

\df{Ecart interquartile}{
    L'\textbf{écart interquartile} $EI$, est la différence entre le troisième et premier quartile.
    \begin{equation*}
        EI(x) = Q_3(x)-Q_1(x)
    \end{equation*}
}

\subsection{Dispersion}

\df{Variance}{
    La \textbf{variance} $V$ est une mesure de la dispersion.
    Elle exprime la moyenne des carrés des écarts à la moyenne.
    \begin{equation*}
        V(x) = \sN \sumN (x_i-\moy{x})
    \end{equation*}
}

\thm{de König-Huygens}{
    \begin{equation*}
        V(x) = (\sN \sumN x_i^2) - \moy{x}^2
    \end{equation*}
}

\demo{}{
    \vspace*{-1cm}
    \begin{align*}
        V(x) &= \sN \sumN (x_i-\moy{x})\\
        &= \sN \sumN (x_i^2 - 2x_i \moy{x} + \moy{x}^2)\\
        &= \sN \sumN x_i^2 - \frac{2\moy{x}}{N} \sumN x_i  + \sN N \moy{x}^2\\
        &= \sN \sumN x_i^2 - \frac{2\moy{x}}{N} N \moy{x} + \moy{x}^2\\
        &= \sN \sumN x_i^2 - 2\moy{x}^2 + \moy{x}^2\\
        &= \sN \sumN x_i^2 - \moy{x}^2
    \end{align*}
}

\df{Ecart type}{
    L'\textbf{écarts type} $\sigma$ est une mesure de la dispersion,
    racine carrée de la variance.
    \begin{equation*}
        \sigma(x) = \sqrt{V(x)}
    \end{equation*}
}

\rmk{}{
    \begin{itemize}
        \item Plus l'écart-type est grand,
        plus les valeurs sont éloignées de la moyenne.
        \item L'écart type est homogène au caractère étudié.
    \end{itemize}
}

\section{Série statistique à deux variables}

\subsection{Définition}

\df{Série statistique à deux variables}{
    On considère deux caractères $x$ et $y$,
    de valeurs $x_i$ et $y_i$ pour $i\in\m{[|1;N|]}$,
    observées sur une même population de $N$ individus.\\
    Les couples $(x_i,y_i)$ pour $i\in\m{[|1;N|]}$ forment une \textbf{\ssd}.
}

\rmk{Représentation \ssd}{
    On peut représenter les \ssd{} par un nuage de points
    $M_i(x_i,y_i)$ pour $i\in\m{[|1;N|]}$.
}

\df{Point moyen}{
    Le point $G(\moy{x};\moy{y})$ est appelé \textbf{point moyen}.
}

\subsection{Corrélation linéaire}

\df{Covariance}{
    La \textbf{covariance} mesure la co-variation entre deux \ss $x$ et $y$.
    \begin{align*}
        Cov(x,y) = \sN \sumN (x_i-\moy{x})(y_i-\moy{y})
    \end{align*}
}

\thm{König-Huygens pour la covariance}{
    \begin{equation*}
        Cov(x,y) = (\sN \sumN x_i y_i) - \moy{x} \times \moy{y}
    \end{equation*}
}

\demo{Similaire à la démonstration de König-Huygens pour la variance}{}

\df{Coefficient de corrélation linéaire}{
    Le \textbf{coefficient de corrélation linéaire} $r$,
    mesure la force et la direction de la relation linéaire entre deux variables $x$ et $y$.
    \begin{align*}
        r(x,y) = \frac{Cov(x, y)}{\sigma(x)\sigma(y)}
    \end{align*}
}

\rmk{}{
    \begin{itemize}
        \item $-1 \leqslant r(x,y) \leqslant 1$
        \item Plus $|r(x,y)|$ augmente plus la corrélation linéaire entre $x$ et $y$ est forte.
        \item Si $r(x,y) \geqslant 0$ alors $x$ et $y$ augmente ensemble.
        \item Si $r(x,y) \leqslant 0$ alors lorsque $x$ augmente $y$ diminue.
    \end{itemize}
}

\subsection{Ajustement affine}

\df{Interpolation et extrapolation}{
    L'\textbf{interpolation} et l'\textbf{extrapolation} sont des méthodes qui consistent à
    estimer une valeur inconnue dans une série statistique.
    \begin{itemize}
        \item pour une interpolation,
        le calcul est réalisé dans le domaine d'étude fourni par les valeurs de la série.
        \item pour une extrapolation,
        le calcul est réalisé en dehors du domaine d'étude.
    \end{itemize}
}

\df{Droite d'ajustement}{
    Dans un nuage de point, on peut construire une droite, appelé \textbf{droite d'ajustement} $y = ax+b$ (ou \textbf{droite de régression}),
    passant « au plus près » des points.
}

\mthd{Des points extrêmes}{
    Droite d'ajustement qui passe par
    le premier point et le dernier point.
    \begin{align*}
        &\lfbrace{
            y_1 &= a x_1 + b\\
            y_N &= a x_N + b
        }\\
        \ialors &\lfbrace{
            b &= y_1 - x_1 \frac{y_n}{x_N-x_1}\\
            a &= \frac{y_n}{x_N-x_1}
        }\\
    \end{align*}
}

\mthd{De Mayer}{
    Droite d'ajustement qui passe par
    les points moyens de deux sous-série obtenue en découpant la \ssd{} en son milieu.
    \begin{align*}
        &\lfbrace{
            \moy{y_1} &= a \moy{x_1} + b\\
            \moy{y_2} &= a \moy{x_2} + b\\
        }
    \end{align*}
}

\mthd{Des moindres carrés}{
    Droite d'ajustement qui
    minimise la somme des carrés des écarts verticals entre
    les valeurs observées et celles prédites par la droite.
    \begin{align*}
        &(a,b) = \operatorname*{argmin}_{(a,b)}\sumN (y_i - (a x_i + b))^2\\
        &\lfbrace{
            a = \frac{Cov(x,y)}{V(x)}\\
            b = \moy{y} - a \moy{x}
        }
    \end{align*}
}

\demo{Méthode Lycée}{
    On admet que le point moyen $G(\moy{x};\moy{y})$ appartient à la droite d'ajustement $(d)$ d'équation $y=ax +b$.\\
    On va chercher la valeurs de $a$ et de $b$.\\
    Soit $P_i(x_i, a x_i + b)$, points de $(d)$ pour $i\in[|1;N|]$.\\
    On a $a$ et $b$ qui minimise $\sumN ||M_i P_i||_v$ 
    avec $M_i(x_i, y_i)$ pour $i\in[|1;N|]$
    et $||M_i P_i||_v$ la distance verticale au carré du point $M_i$ à $P_i$.
    \begin{align*}
        \sumN ||M_i G|| &= \sumN (y_i - (a x_i + b))^2\\
        \ior b &= \moy{y} - a \moy{x} \car G(\moy{x};\moy{y})\in(d)\\
        \ialors \sumN ||M_i G|| &= \sumN (y_i - (a x_i + \moy{y} - a \moy{x}))^2\\
        &= \sumN ((y_i - \moy{y}) - a (x_i - \moy{x}))^2\\
        &= \sumN (y_i - \moy{y})^2 + a^2 (x_i - \moy{x})^2 - 2a(y_i - \moy{y})(x_i - \moy{x})\\
        &= \sumN (y_i - \moy{y})^2 + a^2 \sumN (x_i - \moy{x})^2 - 2a \sumN (y_i - \moy{y})(x_i - \moy{x})\\
        &= N (V(y) + a^2 V(x) - 2aCov(x,y))\\
        \ior V(y) + a^2 V(x) - 2aCov(x,y)& \textrm{ est minimal en } a = -\frac{\beta}{2\alpha} \textrm{ (polynôme du second degré)}\\
        \ialors a = \frac{2Cov(x,y)}{2V(x)} &= \frac{Cov(x,y)}{V(x)}
    \end{align*}
}

\demo{Méthode experte}{
    vsq
}

\subsection{Ajustement non affine}



