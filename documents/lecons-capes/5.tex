% VARIABLES %%%
\def\theme{Statistique à une ou deux variables, représentation et analyse de données}
\def\date{**/**/2023}
%%%%%%%%%%%%%%%

\def\ss{série statistique}

\hbox{}
\infoLecon{
    \ilink{2e}
    \ilink{1re}
    \ilink{Terminale Option Complémentaire}
}
[]
[\ilink{Traitement des données}]
[   
    Offre une méthode rigoureuse pour comprendre et interpréter les
    données.
    Fournit les outils nécessaires pour révéler des tendances, 
    des modèles et des relations au sein des phénomènes observés.
]

\df{\ss}{
    Une \textbf{\ss}
    est une liste de \textbf{valeurs};
    mesures d'un \textbf{caractère}
    d'une \textbf{population} et provient généralement d'une étude.
}

\df{Effectif}{
    \begin{itemize}
        \item L'\textbf{effectif} correspond au nombre de fois où une valeur à était obtenue.
        \item L'\textbf{effectif total} est la taille de la population étudié.
    \end{itemize}
}

\rmk{Représentation}{
    On peut représenter les \ss de différentes manières:
    \begin{itemize}
        \item avec une liste
        \item dans un tableau
        \item dans un graphique
    \end{itemize}
}

\df{Fréquence}{
    La \textbf{fréquence} d'une valeur est donnée par l'effectif de la valeur divisée par l'effectif total.
    \begin{equation*}
        \textrm{fréquence} = \frac{\textrm{effectif}}{\textrm{effectif total}}
    \end{equation*}
}

\pr{Somme des fréquences}{
    La somme de toutes les fréquences d'une \ss est égale à 1.
}

\pr{Fréquence de fréquence}{
    Soient $P$ une population et $A$ et $B$ deux sous-population de $P$,
    avec $B\subset A$.\\
    Si $A$ est de fréquence $x$ dans $P$ 
    et $B$ de fréquence $y$ dans $A$,\\
    alors $B$ est de fréquence $x\times y$ dans $P$.
}

\df{Moyenne}{
    La \textbf{moyenne} d'une \ss
    est donnée par la somme de ces valeurs divisée par l'effectif total.
    \begin{equation*}
        \textrm{moyenne} = \frac{\textrm{somme des valeurs}}{\textrm{effectif total}}
    \end{equation*}
}

\df{Médiane}{
    Dans une \ss,
    la \textbf{médiane} est la valeur telle que:
    \begin{itemize}
        \item au moins la moitié des valeurs lui sont supérieures ou égales.
        \item \item au moins la moitié des valeurs lui sont inférieures ou égales.
    \end{itemize}
}

\df{Etendue}{
    Dans une \ss,
    l'\textbf{étendue} est la différence entre la plus grande valeur et la plus petite valeur.
}



