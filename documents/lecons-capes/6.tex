% VARIABLES %%%
\def\theme{Multiples et diviseurs dans $\m{N}$, nombres premiers.}
\def\date{02/12/2023}
%%%%%%%%%%%%%%%

% DEFINITIONS %
\def\Pgcd{Plus grand diviseurs commun}
\def\Ppcm{Plus petit multiple commun}
%%%%%%%%%%%%%%

% \setboolean{outline}{true}
% \setboolean{demonstration}{false}

\hbox{}
\leconInfo{
    \ilink{Terminale Option Expertes}
}
[]
[]
[]

\section{Multiples et diviseurs dans $\m{N}$}

\subsection{Divisibilité dans $\m{N}$}

\df{Multiples}{
    \TODO
}

\df{Diviseurs}{
    \TODO
}

\mthd{Critère de divisibilité}{
    \TODO
}

\subsection{Division euclidienne}

\thm{Division Euclidienne}{
    \TODO
}

\subsection{\Pgcd}

\df{\Pgcd}{
    \TODO
}

\pr{Existence et unicité du PGCD}{
    \TODO
}

\pr{Algorithme d'Euclide}{
    Soit $a$ et $b$ sont deux entiers naturels avec $a\geqslant b$,
    et $r$ le reste de la division euclidienne de $a$ par $b$.\\
    Alors le pgcd de $a$ et $b$ vaut le pgcd de $b$ et $r$.
    \begin{align*}
        a = bq + r \avec q\in\m{N}
        \Rightarrow \pgcd{a}{b} = \pgcd{b}{r}
    \end{align*}
}

\demo{}{
    Soit $\m{D}(x,y)$ l'ensemble des diviseurs communs de $x$ et $y$.\\
    Soit $a=bq+r$,
    et $d\in\m{D}(a,b)$.\\
    On a $r=a-bq$, $d|a$ et $d|b$ alors $d|r$.\\
    Alors $d\in\m{D}(b,r)$.\\
    Réciproquement si $d\in\m{D}(b,r)$,
    on trouve $d|a$, et donc $d\in\m{D}(a,b)$.\\
    D'où $ \m{D}(a,b) = \m{D}(b,r)$.\\
    On a donc $\pgcd{a}{b} = \pgcd{b}{r}$.
}

\mthd{Algorithme d'Euclide}{
    % \vspace*{-0.6cm}
    \begin{algorithm}
        \KwData{Entiers $a$ et $b$ avec $a\geqslant b$}
        \KwRes{Le pgcd de $a$ et $b$}
        \Tq{$b \neq 0$}{
            $r \leftarrow a \mod b$\;
            $a \leftarrow b$\;
            $b \leftarrow r$\;
        }
        \KwRet{$a$}\;
        % \caption{Euclidean Algorithm}
    \end{algorithm}        
}

\tice{Algorithme d'Euclide}{Python}{
    \lstinputlisting[language=Python]{appendix/lecon-capes/5/euclide.py}
}

\thm{Identité de Bachet-Bézout}{
    Soient $a,b\in\m{Z}$ et $d = \pgcd{a}{b}$.\\
    Il existe $u,v\in\m{Z}$ tels que $au+bv=d$.
}

\demo{}{
    Soient $a,b\in\m{Z}$ et $d = \pgcd{a}{b}$.\\
    Si $d=0$, alors $a=b=0$ donc $(u,v)=(0,0)$ vérifie $au+bv$.\\
    Si $d>0$, alors $(a,b)\neq(0,0)$.\\
    Soit $\m{E}=\{au+bv|(x,y)\in(\m{Z})^2\}\cap\m{N}^*$.\\
    On a $(a^2 + b^2)\in\m{E}$ alors $\m{E}\neq\emptyset$.\\
    $\m{E}$ a un plus petit élément $d_0=au_0+bv_0$.\\
    On a $d\in\m{D}(a,b)$ alors $d|d_0$.\\
    Soit $(u,v)\in\m{Z}^2$.\\
    Il existe $(q,r)\in\m{Z}\times\m{N}$ avec $r<d_0$, tels que:
    \begin{align*}
        au+bv &= qd_0+r \textrm{ division euclidienne de $au+bv$ par $d_0$}
        \ialors au+bv &= q(au_0+bv_0)+r
        \ialors r &= a(u-qu_0) + b(v-qv_0) = au'+bv'
        \ialors r&\in\m{E} \cup \{0\} \et r<d_0 \alors r = 0
        \ialors d_0&|(au+bv)
        \ialors d_0&\in\m{D}(a,b) \alors d_0 | (pgcd{a,b}) = d
        \ialors d_0 &= d
    \end{align*}
    Il existe donc $u,v\in\m{Z}$ tels que $au+bv=d$.
}[\href{https://share.miple.co/content/9kCofIZDUtwbh}{Démos Maths MPSI}]

\pr{}{
    Soient $a,b,c\in\m{Z}$.
    On a ($c|a$ et $c|b) \eqv c|(\pgcd{a}{b})$.
}

\pr{Homogénéité du PGCD}{
    Soit $a,b,k\in\m{Z}\backslash\{0\}$.\\
    Alors $\pgcd{(ka)}{(kb)} = |k|(\pgcd{a}{b})$.
}

\thm{}{
    Soit $a,b,m\in\m{Z}$.\\
    Alors $m$ est un multiple commun de $a,b$ \ssi{}
    $m$ est un multiple de $\pgcd{a}{b}$.
}

\subsection{Nombres premiers entre eux}

\thm{de Bézout}{
    Pour $a,b\in\m{Z}$, on a $\pgcd{a}{b}=1$, \ssi,
    il existe $u,v\in\m{Z}$ tels que $au+bv=1$.
}

\demo{}{
    Le sens direct découle immédiatement de l'identité de Bézout,
    appliquée au cas où $\pgcd{a}{b}=1$.\\
    Sens réciproque:\\
    Supposons qu'il existe $(u,v)\in\m{Z}^2$, tel que $au+bv=1$.\\
    Soit $d\in\m{D}(a,b)$, on a:
    \begin{align*}
        d&|(au+bv)\\
        \ialors d&|1
        \idou d&\in\{1,-1\}
    \end{align*}
    Donc $\pgcd{a}{b} = 1$.
}[\href{https://share.miple.co/content/9kCofIZDUtwbh}{Démos Maths MPSI}]

\lem{de Gauss}{
    Soient $a,b,c\in{Z}$ avec $\pgcd{a}{b}=1$ et $a|bc$,
    alors $a|c$.
}

\demo{}{
    $a|bc$ alors,
    il existe $k\in\m{Z}$ tel que $bc=ka$.\\
    $\pgcd{a}{b}$, il existe $u,v\in\m{Z}$ tels que:
    \begin{align*}
        au+bv&=1  \textrm{ ( \refsec{thm}{de Bézout})}
        \ialors auc+bvc &= c
        \ialors auc+kav &= c
        \ialors a(au+kv) &= c
    \end{align*}
    d'où $a|c$
}[\href{https://www.bibmath.net/ressources/index.php?action=affiche&quoi=capes/demos/arithmetique.html}{Bibmath}]

\cor{du lemme de Gauss}{
    Soit $a,b,n\in\m{Z}$.
    \begin{itemize}
        \item Si $a|n$, $b|n$ et $\pgcd{a}{b} = 1$, alors $ab|n$.
        \item Si $\pgcd{a}{n} = 1$ et $\pgcd{b}{n} = 1$, alors $\pgcd{ab}{n} = 1$
    \end{itemize}
}

\rmk{Forme irréductible de $r$}{
    Le théorème de Gauss permet aussi de démontrer que $r\in\m{Q^*}$
    s'écrit de manière unique $r=\frac{p}{q}$ avec $p\in\m{Z}$, $q\in\m{N*}$ et $\pgcd{p}{q} = 1$.
}

\subsection{\Ppcm}

\df{\Ppcm}{
    \TODO
}

\pr{Existence et unicité du PPCM}{
    \TODO
}

\pr{Lien PGCD et PPCM}{
    \TODO
}

\pr{}{
    Pour $a,b\in\m{Z}$, on a: $|ab| = (\pgcd{a}{b})(\ppcm{a}{b})$.
}

\pr{Homogénéité et associativité du PPCM}{
    \TODO
}

\section{Nombres premiers}

\subsection{Définition}

\TODO

\thm{fondamental de l'arithmétique}{
    Tout entier strictement positif peut être écrit comme un produit de nombres premiers d'une unique façon,
    à l'ordre près des facteurs.\\ \\
    Pour $n\in\m{Z}$, il existe un unique $(p_i)_{i\in[|1,r|]}$ et $(\alpha_i)_{i\in[|1,r|]}$,
    avec $p_i$ premiers, $\alpha_i\in\m{N^*}$ pour $i\in[|1,r|]$ avec $r\in\m{N}$ tels que:
    \begin{equation*}
        n = \prod_{i = 1}^{r}p_i^{\alpha_i}
    \end{equation*}
}

\df{Valuation p-adique}{
    On appelle \textbf{valuation p-adique} et on note $v_p(n)$,
    le plus grand entier $k\geqslant0$ tel que $p^k|n$.
}

\pr{}{
    Soit $a,b\geqslant2$ et $p$ premiers.\\
    Alors $v_p(ab) = v_p(a) + v_p(b)$.
}