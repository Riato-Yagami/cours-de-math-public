% VARIABLES %%%
\def\theme{Périmètres, aires, volumes}
% \def\date{date}
%%%%%%%%%%%%%%%

% DEFINITIONS %
\def\md{\ensuremath{\meter^2}}
\def\mt{\ensuremath{\meter^3}}

\def\cbmath{\href{https://docplayer.fr/210413521-Lecon-n-o-18-perimetres-aires-volumes-capes-session-cbmaths-cbmaths-fr-derniere-mise-a-jour-2-avril-2021.html}{CBMaths.fr}}
\def\jean{\href{https://www.mathenjeans.fr/sites/default/files/comptes-rendus/polygones_-_saint_dominique_nancy.pdf}{mathenjeans.fr}}
%%%%%%%%%%%%%%

% \setboolean{subsectionInOutline}{true}
% \setboolean{outline}{true}
% \setboolean{demonstration}{false}

% \setboolean{showRef}{false}

\hbox{}
\leconInfo{
    \ilink{Cycle 3}
    \ilink{Cycle 4}
    \ilink{Terminale Option Spécialité}
    \ilink{Terminale Option Complémentaire}
    \ilink{Terminale Technologique}
} % Niveaux
[\item longueur] % Prérequis
[] % Thèmes
[] % Motivation

\section{Périmètre et aire}

\subsection{Périmètre}

\df{Perimètre}{
    Le \textbf{périmètre} d'une figure est la longueur de son contour.
}

\rmk{Perimètre du polygone}{
    Pour un polygone, c'est la somme des longueur de ses cotés
}

\rmk{Unité de longueur}{
    Le périmètre se mesure en unité de longueur,
    de maniere usuelle en mètre (\meter).\\
    Deux unités de longueur consécutives sont 10 fois plus grande ou plus petite les unes par rapport aux autres.
}

\pr{Perimètre de polygone particulier}{
    \begin{itemize}
        \item Carré ou losange de coté $c$: $P = 4c$
        \item Rectangle de longueur $L$ et largeur $l$: $P= 2L +2l$
        \item Triangle équilatéral de coté $c$: $P= 3c$
    \end{itemize}
}

\subsection{Aire}

\df{Comparaison d'aire}{
    On dit que deux figure ont la même aire si en découpant l'une d'entre elle,
    on peut recomposer l'autre.
}[\cbmath]

\pr{Découpage de polygone}{
    On peut toujours découper un polygone en triangles.
}

\demo{}{
    \def\polygone{
        \tkzDefPoint(0,0){A}
        \tkzDefPoint(1,2){B}
        \tkzDefPoint(1.5,2){C}
        \tkzDefPoint(1.9,1.5){D}
        \tkzDefPoint(2.6,3){E}
        \tkzDefPoint(3,2){F}
        \tkzDefPoint(3,1){G}
        \tkzDefPoint(2,-1){H}
        \draw[blue , thick] (D)--(H);
    }
    \dividePage{On le découper en polygones convexes.
        \begin{center}
            \begin{tikzpicture}
                \polygone
                \draw[very thick] (A)--(B)--(C)--(D)--(E)--(F)--(G)--(H)--cycle;
            \end{tikzpicture}
        \end{center}
    }{Puis on choisit un sommet dans chaque partie convexe et on le relie aux autres.
        \begin{center}
            \begin{tikzpicture}
                \polygone
                \draw[red , thick] (A)--(C);
                \draw[red , thick] (A)--(D);
                \draw[red , thick] (E)--(G);
                \draw[red , thick] (E)--(H);
                \draw[very thick] (A)--(B)--(C)--(D)--(E)--(F)--(G)--(H)--cycle;
            \end{tikzpicture}
        \end{center}
    }
}[\jean]

\pr{Découpage et recollage de triangle en rectangle}{
    On peut toujours découper et recoller un triangle quelconque pour trouver un rectangle de même aire.
}

\demo{}{
    \def\triangle{
        \tkzDefPoint(0,0){A}
        \tkzDefPoint(6,0){B}
        \tkzDefPoint(1.5,3){C}
        % \tkzLabelPoints[below left](A)
        % \tkzLabelPoints[below right](B)
        % \tkzLabelPoints[above](C)
        % Hauteur issue de C
        \tkzDefLine[perpendicular=through C](A,B) \tkzGetPoint{c} % Définir un point pour la direction de la ligne
        \tkzInterLL(C,c)(A,B) \tkzGetPoint{H} % Intersection de la ligne avec AB
        \draw[thick] (C)--(H);
        \tkzLabelPoints[below](H)
        % Médiatrice CH
        \tkzDefLine[mediator](C,H) \tkzGetPoint{f}
        \tkzDefLine[mediator](H,C) \tkzGetPoint{o}
        \tkzInterLL(o,f)(A,C) \tkzGetPoint{O}
        \tkzInterLL(o,f)(B,C) \tkzGetPoint{F}
        \draw[thick] (O)--(F);
        % \tkzLabelPoints[above left](O)
        % \tkzLabelPoints[above right](F)
        \tkzInterLL(O,F)(C,H) \tkzGetPoint{I}
        \tkzLabelPoints[below right](I)
        % Mark
        \tkzMarkRightAngle[blue](A,H,C)
        \tkzMarkRightAngle[blue](O,I,H)
        \draw[very thick] (A)--(B)--(C)--cycle;
        \tkzMarkSegment[blue,pos=.5,mark=|](C,I)
        \tkzMarkSegment[blue,pos=.5,mark=|](H,I)
    }
    \dividePage{
        \begin{center}
            \begin{tikzpicture}[scale=1]
                \triangle
            \end{tikzpicture}
        \end{center}
    }{
        \begin{center}
            \begin{tikzpicture}[scale=1]
                \triangle
                % D et E
                \tkzDefLine[perpendicular=through A](O,F) \tkzGetPoint{d}
                \tkzInterLL(d,A)(O,F) \tkzGetPoint{D}
                \tkzDefLine[perpendicular=through B](O,F) \tkzGetPoint{e}\tkzLabelPoints[below right](I)
                \tkzInterLL(e,B)(O,F) \tkzGetPoint{E}
                % \tkzLabelPoints[right](E)
                % \tkzLabelPoints[left](D)
                \draw[thick] (E)--(D);
                % Trianlges
                \draw[fill = red, fill opacity=0.2] (A)--(D)--(O)--cycle;
                \draw[fill = red, fill opacity=0.2] (C)--(I)--(O)--cycle;
                \draw[fill = blue, fill opacity=0.2] (B)--(E)--(F)--cycle;
                \draw[fill = blue, fill opacity=0.2] (C)--(I)--(F)--cycle;
            \end{tikzpicture}
        \end{center}
    }
}[\jean]

% \pr{Découpage et recollage de rectangle à un rectangle de longueur choisit}{
%     On peut toujours découper et recoller un rectangle quelconque pour trouver un rectangle de même aire et de longueur choisit.
% }

% \pr{Découpage et recollage de rectangle en carré}{
%     On peut toujours découper et recoller un rectangle quelconque pour trouver un carré de même aire.
% }

% \demo{}{
%     \def\lenght{5}
%     \def\width{3}
%     \begin{center}
%         \begin{tikzpicture}[scale=1]
%             \tkzDefPoint(0,0){A}
%             \tkzDefPoint(0,\width){B}
%             \tkzDefPoint(\lenght,\width){C}
%             \tkzDefPoint(\lenght,0){D}
%             \draw[very thick] (A)--(B)--(C)--(D)--cycle;
%             \tkzDefPoint(-\width,0){E}
%             \draw[dashed] (B) arc (90:180:\width);
%         \end{tikzpicture}
%     \end{center}
% }[\href{https://wimsauto.universite-paris-saclay.fr/wims/wims.cgi?session=8W0948C5F7.2&+lang=fr&+module=U1\%2Fgeometry\%2Fdocbolyai.fr&+cmd=reply&+job=read&+doc=1&+block=rectanglecarre2}{Université Paris-Saclay}]

% \pr{Un rectangle de longueur $L$ et largeur $l$ a pour aire $Ll$}

% \demo{}{
%     Sur la 
% }[\cbmath]

\df{Unité d'aire}{
    L'unité d'aire usuelle est le mètre carré (\md).\\
    1 \md{} correspond à la mesure de surface d'un carré de coté 1 \meter.
}

\df{Aire}{
    L'\textbf{aire} d'une figure est la mesure de sa surface,
    en comparaison au \md.
}

\rmk{Unités d'aire consécutives}{
    Deux unités d'aire consécutives sont 100 fois plus grande ou petites les unes par rapport aux autres.
}

\pr{Additivité des aires}{
    L'aire de l'union des surface $A$ et $B$ disjointes, est égale à la somme de leurs aires.
}

\rmk{Découpages et recollements}{
    L'additivité des aires est le fondement du découpage et recollements.
}

\subsection{Dichotomie périmètre et aire}

\act{Curvica}{
}[\href{https://www.educmat.fr/categories/jeux_reflexion/fiches_jeux/curvica/index.php}{éducmat}]

\section{Le cercle et le disque}

\subsection{La Circonférence}

\pr{Circonférence}{
    La \textbf{circonférence} est la longueur d'un cercle.
}

\df{Constante d'Archimède (Pi)}{
    $\pi$ est le rapport de la circonférence d'un cercle à son diamètre.
    \begin{equation*}
        \pi \approx 3,141592653589793
    \end{equation*}
}

\mthd{d'Archimède (Approximation de Pi)}{
    On encadre $\pi$ par les périmètres de polygones réguliers inscrits et circonscrits à un cercle de diamètre $1$.
    \href{https://www.geogebra.org/m/yqhfdqgg}{$\rightarrow$ Activité géogebra}
}

\rmk{Formule de la circonférence}{
    On a alors la circonférence d'un cercle de rayon $r$ : $2 \pi r$.
}

\pr{Longueur d'arc}{
    \begin{center}
        \def\angle{30} % Define the angle size
        \def\r{4}
        \begin{tikzpicture}
            \tkzDefPoint(0,0){A}
            \tkzDefPoint(\r,0){B}
            \tkzDefShiftPoint[A](\angle:\r){C} % Adjust the angle for gamma and the length for side 'c'
            \draw[very thick] (B)--(A)--(C);
            \tkzDrawArc[Blue](A,B)(C) % Draw arc from B to C centered at A
            % %
            \tkzMarkAngle[Red,size=0.7cm,opacity=1](B,A,C) % Mark angle BAC
            \tkzLabelAngle[Red, pos=1](B,A,C){$\theta$} % Label angle BAC
            % %
            \tkzLabelSegment[below](A,B){$r$} % Label segment AB as r
            \tkzLabelArc[Blue, right](A,B,C){$r\theta$} % Label the arc from B to C as rθ
        \end{tikzpicture}
    \end{center}
}

\demo{}{
    L'arc est une section $\frac{\theta}{2\pi}$ de la circonférence.\\
    Alors la longueur d'arc est: $2\pi r \times \frac{\theta}{2\pi} = r \theta$.
}

\subsection{Aire du disque}

\act{Approximation aire du disque}{}

\mthd{de Monte-Carlo (Approximation de l'aire du disque)}{
    \begin{algorithm}[H]
        \KwData{$r\in\m{R} \et N\in\m{N}$}
        \KwRes{approximations de l'aire du cercle de rayon $r$.}
        $c \leftarrow 2r$
        On considère une figure constiué d'un carré de coté $c$ dans lequel on place en son centre un cercle de rayon $r$\;
        $n \leftarrow 0$\;
        \Pr{$i$ de $0$ à $N$}{
            Prendre un point $A$ au hasard dans le carré\;
            \Si{$A$ est dans le cercle}{$n \leftarrow n+1$\;}
        }
        \KwRet{$\frac{n}{N} \times c^2$}\;
    \end{algorithm}
}[\href{http://pedagogie.ac-limoges.fr/maths/IMG/pdf/manipulation_aire_disque_methode_de_monte-carlo.pdf}{Académie de limoge}]

\tice{Méthode de Monte-Carlo}{Python}{
    \lstinputlisting[language=Python]{ressources/lecons-capes/17/monte-carlo.py}
}

\pr{Aire du disque}{
    L'aire d'un disque de rayon $r$ : $\pi r^2$.
}

\section{Volumes et aires latérales}

\subsection{Volumes}

\df{Unité de volume}{
    L'unité de volume usuelle est le mètre cube (\mt)\\
    1 \mt{} correspond à la mesure d'espace d'un cube de coté 1 $\meter$.
}

\df{Volume}{
    Le \textbf{volume} d'un solide est ca mesure d'espace, en comparaison \mt.
}

\rmk{Unités de volume consécutives}{
    Deux unités de volume consécutives sont 1000 fois plus grande ou petites les unes par rapport aux autres.
}

\subsection{Sphère}

\pr{Aire d'une sphère}{
    L'aire d'une shère de rayon $r$ est : $4\pi r^2$
}


\demo{}{
    $dxdy = dA = r dr d\theta$ par la formule de longueur d'arc
    \begin{center}
        \def\angle{30} % Define the angle size
        \def\r{4}
        \def\dr{\r*0.75}
        \begin{tikzpicture}
            \tkzDefPoint(0,0){A}
            \tkzDefPoint(\r,0){B}
            \tkzDefPoint(\dr,0){R}
            \tkzDefShiftPoint[A](\angle:\r){C} 
            \tkzDefShiftPoint[A](\angle:\dr){c} 
            \draw[very thick] (B)--(A)--(C);
            \tkzDrawArc(A,B)(C) % Draw arc from B to C centered at A
            \tkzDrawArc(A,R)(c) % Draw arc from B to C centered at A
            % %
            \tkzMarkAngle[size=0.7cm,opacity=1](B,A,C) % Mark angle BAC
            \tkzLabelAngle[pos=1](B,A,C){$d\theta$} % Label angle BAC
            % %
            \tkzLabelSegment[above](A,C){$r$} 
            \tkzLabelSegment[Red][below](R,B){$dr$}
            \tkzLabelArc[right](A,B,C){$rd\theta$} % Label the arc from B to C as rθ
            \draw[fill=Red, opacity = 0.25] (B) arc (0:30:\r) -- (c) arc (30:0:\dr) -- cycle;
            \node[Red] at (barycentric cs:B=1,C=1,c=0.75,R=0.75) {$dA$};
            \draw[Red] (R) -- (B);
        \end{tikzpicture}
    \end{center}
    \begin{align*}
        dxdydz = dV &= r dr d\theta dz\\
        \ior z &= r\cos(\theta)\\
        \idou dz &= r\sin(\varphi) d\varphi\\
        \ialors dV &= r^2 \sin(\varphi) dr d\theta d\varphi
    \end{align*}
}

\pr{Volume d'une sphère}{
    Le volume d'une shère de rayon $R$ est : $\frac{4\pi R^3}{3}$
}

\demo{}{
    On a les coordonnés sphériques :
    \begin{align*}
        \lfbrace{x &= R\cos(\theta)\sin(\varphi)\\
        y &= R\sin(\theta)\sin(\varphi)\\
        z &= R\cos(\theta)}
    \end{align*}
}

\section{Aire sous la courbe}

Soit $f$, fonction continue et positive sur $[a,b]$ et $\mathcal{C}$ sa courbe représentative.

\df{Aire sous la courbe}{
    L'\textbf{aire sous la courbe} $\mathcal{C}$ sur $[a,b]$ est l'aire du domaine délimité par;
    l'axe des abscisses, la courbe $\mathcal{C}$ et les droites d'équation $x=a$ et $x=b$.\\
    On note $\int_{a}^{b}f(x)dx$ et appelle \textbf{intégrale} de $f$ entre $a$ et $b$ cette aire. 
}[\cbmath]

\mthd{Approximation d'une aire par l'utilisation de suites adjacentes}{
    \def\s{\frac{b-a}{n}}
    Soit $u,v$ deux suites adjacentes tels que pour $n\in\m{N}$:\\
    \begin{align*}
        u_n = \s \sum_{k=0}^{n} f(a+k \s) \;&\et\; v_n = \s \sum_{k=1}^{n} f(a+k \s)\\
        \ona \; \int_{a}^{b} f(x) dx &\in [u_n,v_n]
    \end{align*}
    \href{https://www.geogebra.org/m/bhmjfjhs}{$\rightarrow$ Activité géogebra}
}

\pr{Relation de Chalses}{
    Soit $c\in[a,b]$, on a:
    \begin{equation*}
        \int_{a}^{b}f(x)dx = \int_{a}^{c}f(x)dx + \int_{c}^{b}f(x)dx
    \end{equation*}
}

\demo{}{
    Additivité des aires
}