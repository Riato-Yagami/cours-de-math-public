% VARIABLES %%%
\def\theme{Utilisation des nombres complexes en géométrie}
\def\date{31/10/2023}
%%%%%%%%%%%%%%%

\hbox{}
\infoLecon{
    \niv{Terminale Option Expertes}
    \item Licence 1
}{
    \item Prerequis
}

\section{Image et affixe}

\df{Image et affixe}{
\begin{itemize}
    \item A tout nombre complexe $z=a+ib$, on associe un point $M$ de coordonnés $(a;b)$.\\
    On dit que $M$ est \textbf{l'image} de $z$ et que $z$ est \textbf{l'affixe} de $M$ et on note $M(z)$.
    \item A tout vecteur $\vect{k}$ de coordonnés $(a,b)$, on associe le nombre complexe $z = a+ib$.\\
    On dit que $z$ est \textbf{l'affixe} de $\vect{k}$ et on note $\vect{k}(z)$.
\end{itemize}
}

\pr{Egalité et Linéarité}{
\begin{itemize}
    \item Deux vecteurs sont egaux \ssi, leurs affixes sont égales.
    \item Si $\vect{u}$ et $\vect{v}$ ont pour affixe $z$ et $z'$, alors $\vect{u}+\vect{v}$ a pour affixe $z+z'$
    et $\lambda\vect{u}$ avec $\lambda\in\m{R}$ a pour affixe $\lambda z$
\end{itemize}
}

\pr{Représentation graphique}{
\begin{itemize}
    \item $M$ appartient à l'axe des abscisses \ssi son affixe $z$ est réel.
    \item $M$ appartient à l'axe des ordonnés \ssi son affixe $z$ est imaginaire pur.
    \item Deux nombres complexes conjugés sont coordonnés des points symétrique par rapport à l'axe des abscisses.
\end{itemize}
}

\pr{}{
Soient $A$ et $B$ d'affixe $z_a$ et $z_b$.
\begin{itemize}
    \item L'affixe du vecteur $\vect{AB}$ est $z_B-z_A$
    \item L'affixe du milieu du segement $[A,B]$ est $z_I = \frac{z_A + z_B}{2}$
\end{itemize}
}

\section{Module}

\df{Module d'un nombre complexe}{
Soit $M$ d'affixe z. Le \textbf{module} de $z$, noté $|z|$, est la distance $OM$.
}

\pr{}{
Soit $z\in\m{C}$, alors $|z|=0 \eqv z = 0$.
}

\demo{}{
$|z| = 0 \eqv OM = 0 \eqv O \et M \textrm{ confondus } \eqv z = 0.$ 
}

\pr{}{
Pour tout nombre complexe $z=x+iy$, on a:
\begin{equation*}
    |z| = \modxy{x}{y} \et |z|^2 = z \times \overline{z}.
\end{equation*}
}

\demo{}{
    Soit $M$ d'affixe $z$ de coordonnés $(x,y)$.
    On a:
    \begin{align*}
        |z| = OM &= \modxy{(x_M-x_O)}{(y_M-y_O)}\\
        &= \modxy{(x-0)}{(y-0)}\\
        &= \modxy{x}{y}
    \end{align*}
    D'autre part:
    \begin{align*}
        |z|^2 = (\modxy{x}{y})^2 &= x^2 + y^2\\
        \iet z \times \overline{z} = (x+iy)(x-iy) &= x^2 + y^2\\
        \idou |z|^2 &= \et z \times \overline{z}
    \end{align*}
}

\rmq{}{
    Pour $z\neq 0, \frac{1}{z} = \frac{\overline{z}}{|z|^2}$
}

\pr{}{
    \begin{enumerate}
        \item $|\overline{z}| = |z| \et |-z| = |z|$
        \item \begin{itemize}
            \item $|z\times z'| = |z| \times |z'|$
            \item si $z'\neq 0, |\frac{z}{z'}| = \frac{|z|}{|z'|}$
            \item pour $\in\m{Z} |z^n| = |z|^n$ (avec $z\neq 0 \si n < 0$)
        \end{itemize}
    \end{enumerate}
}

\pr{Inégalités triangulaires}{
    \begin{enumerate}
        \item $|z+z'| \leqslant |z|+|z'|$
        \item $||z|-|z'|| \leqslant |z+z'|$
    \end{enumerate}
}

\demo{}{
    \begin{enumerate}
        \item \begin{align*}
            |z+z'|^2 &= (z+z')^2\\
            &= z^2+z'^2 + 2zz'\\
            &= |z|^2+|z'|^2+2zz'\\
            \ior zz' \leqslant |zz'|&=|z||z'|
            \ialors |z+z'|^2 &\leqslant |z|^2+|z'|^2+2|z||z'| = (|z|+|z'|)^2
        \end{align*}
        Par croissance de la fonction carré, on a: $|z+z'| \leqslant |z|+|z'|$
        \item En utilisant la première inégalité:
        \begin{align*}
            |z| = |z+z'-z'| &\leqslant |z+z'| + |z'|\\
            \iet |z'| = |z'+z-z| &\leqslant |z+z'| + |z|\\
            \ialors |z|-|z'| &\leqslant |z+z'|\\
            \iet |z'|-|z| &\leqslant |z+z'|\\
            \ior ||z|-|z'|| &= max(|z|-|z'|,|z'|-|z|)\\
            \ialors ||z'|-|z|| &\leqslant |z+z'|
        \end{align*}
    \end{enumerate}
}

\section{Nombres complexes de module 1}

\df{Nombres complexes de module 1}{
    On note $\m{U}$ l'ensemble des nombres complexes $z$ tels que $|z|=1$\\
    Dans le plan complexe, $U$ est représenté par le cercle de centre $O$ et de rayon $1$.
}

\pr{}{
    Pour tout $z,z'\in\m{U}$.
    \begin{enumerate}
        \item $zz'\in\m{U}$
        \item $\frac{1}{z}\in\m{U}$
    \end{enumerate}
}

\demo{}{
    \begin{enumerate}
        \item \begin{align*}
            |zz'|&=|z||z'|\\
            \ior |z|=|z'|&=1\\
            \ialors |z'z| &= 1
            \idou z'z&\in\m{U}
        \end{align*}
        \item \begin{align*}
            |\frac{1}{z}| = \frac{1}{|z|}\\
            \ior |z| = 1\\
            \ialors |\frac{1}{z}| = 1\\
            \idou \frac{1}{z}\in\m{U}
        \end{align*}
    \end{enumerate}
}

\section{Argument}

\df{Mesure de l'angle orienté}{
    Soit un angle orienté $(\vect{u},\vect{OM})$.\\
    On appelle \textbf{mesure} de cette orienté les réels $\alpha$ tels que:
    \begin{equation*}
        (\vect{u},\vect{OM}) = \alpha + k \times 2\pi \avec k\in\m{Z}
    \end{equation*}
}

\df{Argument}{
    Soient $M$ point d'affixe $z\in\m{C\backslash\{0\}}$.\\
    On appelle \textbf{argument} de $z$ et on note $\mathbf{arg(z)}$
    une mesure de l'angle orienté $(\vect{u},\vect{OM})$.\\
    Alors $(\vect{u},\vect{OM}) = arg(z) + 2k\times \pi$ avec $k\in\m{Z}$.\\
    La mesure d'angle dans $]-\pi;\pi]$ est appelée \textbf{argument principal} de $z$.
}

\section{Forme trigonometrique}

\newpage

\section{Ecriture complexe de transformations}

\thm{Ecriture complexe d'une translation}{
Soit un vecteur $\vect{u}$, d'affixe $u$ 
et $T_u$ la translation de vecteur $\vect{u}$.\\
Alors, si $M$ est un point d'affixe $z$ et $z'$ est l'affixe 
de son image $M'= T_u(M)$, on a:
\begin{align*}
    z' = z + u
\end{align*}
\textbf{L'écriture complexe} de $T_u$ est $t_u : z \mapsto z + u$
}

\demo{}{
On a $\vect{MM'} = u$, ce qui donne $z-z'= u$
}

\thm{Ecriture complexe d'une homothétie}{
    \newcommand{\homg}{H_{\Omega,k}}
    Soient $\Omega$ point d'affixe $w$, $k$ un réel non nul 
    et $\homg$ homothétie de centre $\Omega$ et de rapport $k$.\\
    Alors, si $M$ est un point d'affixe $z$ et $z'$ est l'affixe 
    de son image $M'= \homg(M)$, on a:
    \begin{align*}
        z' = k(z-w)+w=kz+(1-k)w
    \end{align*}
    \textbf{L'écriture complexe} de $\homg$ est $h_{\Omega,k} : z \mapsto kz + (1-k)w$
}

\demo{}{
    On a $\vect{\Omega M'} = k\vect{\Omega M}$, d'ou $z-w'= k(z-w)$
}


\thm{Ecriture complexe d'une rotation}{
    \newcommand{\romg}{R_{\Omega,\Theta}}
    Soient $\Omega$ point d'affixe $w$, $\Theta$ un réel
    et $\romg$ rotation de centre $\Omega$ et d'angle $\Theta$.\\
    Alors, si $M$ est un point d'affixe $z$ et $z'$ est l'affixe 
    de son image $M'= \romg(M)$, on a:
    \begin{align*}
        z' = e^{i\Theta}(z-w)+w
    \end{align*}
    \textbf{L'écriture complexe} de $\romg$ est $r_{\Omega,\Theta} : z \mapsto e^{i\Theta}(z-w)+w$
}

% % \demo{}
% % \begin{itemize}
% %     \item Si $\Theta = 0$
% %     on a: $r_{\Omega,0} = e^{i*0}(z-w)+w = z - w + w = z
% %     \item 
% % \end{itemize}

\thm{}{
    Une transformation du plan est une similitude:
    \begin{itemize}
        \item directe 
        \ssi{} son écriture complexe est de la forme:
        $z'=az+b$ avec $(a,b)\in\m{C^*}\times\m{C}$
        \item indirecte
        \ssi{} son écriture complexe est de la forme:
        $z'=a\overline{z}+b$ avec $(a,b)\in\m{C^*}\times\m{C}$
    \end{itemize}
}

\thm{}{
    Une transformation du plan est une similitude:
    \begin{itemize}
        \item directe 
        \ssi{} son écriture complexe est de la forme:
        $z'=az+b$ avec $(a,b)\in\m{C^*}\times\m{C}$
        \item indirecte
        \ssi{} son écriture complexe est de la forme:
        $z'=a\overline{z}+b$
        avec $(a,b)\in\m{C^*}\times\m{C}$
    \end{itemize}
}