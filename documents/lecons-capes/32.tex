% VARIABLES %%%
\def\theme{Nombre dérivé. Fonction dérivée.}
\def\date{09/10/2023}
%%%%%%%%%%%%%%%

\infoLecon{
    \csname 1re\endcsname
}{
    Limites
}

\df{Taux de variation}
Soient $f$ definie sur un intervalle $\m {I}$,
et $(x_0,h)\in\m{I}\times\m{R\backslash\{0\}}$
tels que $x_0+h\in\m{I}$.\\
Le \textbf{taux de variation} (ou \textbf{taux d'accroissement}) de la fonction $f$ entre $x_0$ et $x_0+h$ est:
\begin{equation}\label{eq:1}
    \frac{f(x_0+h)-f(x_0)}{h}
\end{equation}

\df{Fonction dérivable en un point}
Une fonction $f$ est dite \textbf{dérivable} en $x_0$ si (\ref{eq:1})
admet une certaine limite $l\in\m{R}$ quand $h$ tend vers $0$.\\
$l$ est alors appelée \textbf{nombre dérivé} de $f$ en $x_0$, on le note $f'(x_0)$.\\
\begin{equation*}
    f'(x_0) = \lm{h}{0}\frac{f(x_0+h)-f(x_0)}{h}
\end{equation*}

\rmq{}
On peut également définir le nombre dérivé de la façon suivante:
\begin{equation*}
    f'(x_0) = \lm{x}{x_0}\frac{f(x)-f(x_0)}{x-x_0}
\end{equation*}

\pr{}
Soit $f$ une fonction dérivable en $x_0\in\m{R}$ de courbe représentative $\mathscr{C}_f$\\
L'équation de la tangente à $\mathscr{C}_f$ au point d'abscisse $x_0$ est:
\begin{equation*}
    y = f'(x_0)(x-x_0) + f(x_0)
\end{equation*}

\df{Fonction dérivable sur un intervalle}
Soit $f$ une fonction défini sur $\m{I}$. On dit que $f$ est \textbf{dérivable} sur $\m{I}$ si pour tout $x\in\m{I}$,
le nombre dérivé $f'(x)$ existe.\\
La fonction qui à $x\in\m{I}$ associe le nombre dérivé de $f$ en $x$ s'appelle la \textbf{fonction dérivé} et se note $f'$.

\newpage

\pr{dérivé de fonctions usuelles}

\begin{center}
    \begin{tabular}{l | c | c}

        \textbf{fonction} &  \textbf{ensemble de dérivabilité} & \textbf{dérivé}\\
        $x \mapsto k$ avec $k\in\m{R}$ 
        & $\m{R}$ 
        & $0$ \\
        $x \mapsto x$ 
        & $\m{R}$ 
        & $1$ \\
        $x \mapsto x^n$ avec $n\in\m{N}$
        & $\m{R}$ 
        & $nx^{n-1}$ \\
        $x \mapsto \frac{1}{x^n}$ avec $n\in\m{N}$
        & $\m{R^*}$ 
        & $-\frac{n}{x^{n+1}}$ \\
        $x \mapsto \sqrt{x}$
        & $\m{R_+^*}$ 
        & $\frac{1}{2\sqrt{x}}$ \\
    \end{tabular}
\end{center}

\pr{Formule de base}

\begin{center}
    \begin{tabular}{c | c}
        \textbf{fonction} & \textbf{dérivé}\\
        $u+v$ & $u'+v'$\\
        $ku$ & $ku'$\\
        $uv$ & $u'v+uv'$\\
        $\frac{1}{u}$ avec $u$ non nul sur $\m{I}$ & -$\frac{u'}{u^2}$\\
        $\frac{u}{v}$ avec $v$ non nul sur $\m{I}$ & -$\frac{u'v-uv'}{v^2}$\\
        $\sqrt{u}$ & $\frac{u'}{2\sqrt{u}}$\\

    \end{tabular}
\end{center}

\thm{}
Soit $f$ définie sur un intervalle $\m{I}$.
\begin{itemize}
    \item $f$ est croissante sur $\m{I}$ \ssi,
    \pt{} $x\in\m{I}, f'(x) \geqslant 0$
    \item $f$ est décroissante sur $\m{I}$ \ssi,
    \pt{} $x\in\m{I}, f'(x) \leqslant 0$
    \item $f$ est strictement croissante sur $\m{I}$ \ssi,\\
    \pt{} $x\in\m{I}, f'(x) > 0$
    \item $f$ est strictement décroissante sur $\m{I}$ \ssi,\\
    \pt{} $x\in\m{I}, f'(x) < 0$
\end{itemize}

\pr{Fonction valeurs absolue}










