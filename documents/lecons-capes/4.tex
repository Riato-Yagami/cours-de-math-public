% VARIABLES %%%
\def\theme{Variables aléatoires réelles à densité.}
\def\date{02/12/2023}
%%%%%%%%%%%%%%%

% DEFINITIONS %
\def\va{variable aléatoire}
\newcommand{\intf}[2]{
    \intfx{#1}{#2}{f(x)}
}
\newcommand{\intfx}[3]{
    \int_{#1}^{#2}#3dx
}
%%%%%%%%%%%%%%

% \setboolean{outline}{true}
% \setboolean{demonstration}{false}

\hbox{}
\infoLecon{
    \ilink{Terminale Option Expertes}
}
[]
[]
[]

\df{Variable aléatoire à densité}{
    On appelle \textbf{\va{} à densité} toute fonction $X:\Omega \rightarrow \m{R}$
    telle qu'il existe une fonction $f:\m{R}\rightarrow\m{R}$ (\textbf{fonction densité}) continue par morceaux telle que:
    \begin{equation*}
        \forall a < b, P(X\in[a,b]) = \intf{a}{b}.
    \end{equation*}
}

Soit $X$ une \va et $f$ sa densité.

\pr{Caracterisation fonction densité}{
    Nécessairement: $f$ positive et $\intf{-\infty}{+\infty} = 1$.
}

\df{Fonction répartition}{
    $F_X: t \rightarrow P(X\leqslant t)$ est la \textbf{fonction répartition} de $X$.
}

\pr{}{
    Soit $a\in\m{R}$, on a:
    \begin{align*}
        P(X\leqslant a) &= P(X < a) = \intf{-\infty}{a}\\
        P(X\geqslant a) &= P(X > a) = \intf{a}{+\infty}
    \end{align*}
}

\df{Esperance}{
    $X$ admet une \textbf{espérance} E si $\intfx{-\infty}{+\infty}{|x|f(x)}$ converge.
    \begin{equation*}
        E(X) = \intfx{-\infty}{+\infty}{xf(x)}
    \end{equation*}
}

\thm{de Transfert}{
    $\varphi : \Omega \rightarrow \m{R}$ continue.\\
    Si  $\intfx{-\infty}{+\infty}{|\varphi(x)|f(x)}$ converge.
    \begin{equation*}
        E(\varphi(X)) = \intfx{-\infty}{+\infty}{\varphi(x)f(x)}
    \end{equation*}
}

\df{Variance}{
    $X$ admet une \textbf{variance} V
    si $X$ admet une espérance
    et $(X-E(X))^2$ admet une espérance.
    \begin{equation*}
        V(X) = E((X-E(X))^2) = \intfx{-\infty}{+\infty}{(x-E(X))^2f(x)}
    \end{equation*}
}

\thm{de Koenig-Huygens}{
    \begin{equation*}
        V(X) = E(X^2)-E(X)^2
    \end{equation*}
}

\df{Loi uniforme}{
    X suit une \textrm{loi uniforme} sur [a,b] si:
    \begin{equation*}
        f(x) = \lfbrace{
            \frac{1}{b-a} &\si x \in[a,b] \\
            0 &sinon
            }
    \end{equation*}
}

\pr{Loi uniforme}{
    X de loi uniforme sur [a,b]
    \begin{itemize}
        \item $E(X) = \frac{a+b}{2}$
        \item $V(X) = \frac{(b-a)^2}{12}$
    \end{itemize}
}

\df{Loi exponentielle}{
    X suit une \textrm{loi exponentielle} de paramètre $\lambda>0$ si:
    \begin{equation*}
        f(x) = \lfbrace{
            \lambda e^{-\lambda x} &\si x \geqslant 0 \\
            0 &sinon
            }
    \end{equation*}
}

\pr{Loi exponentielle}{
    X de loi exponentielle de paramètre $\lambda>0$
    \begin{itemize}
        \item $E(X) = \frac{1}{\lambda}$
        \item $V(X) = \frac{1}{\lambda^2}$
    \end{itemize}
}