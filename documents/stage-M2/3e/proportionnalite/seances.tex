% VARIABLES %%%
\def\authors{PESIN - CADOT - COURTIN}
% \date{\today}
\def\longTitle{La Proportionnalité}
\def\shortTitle{Proportionnalité}
%%

% \disableAnimation
% \shortAnimation
\firstSlide

\def\imgPath{stage-M2/3e/proportionnalite/}
\def\imgExtension{_myr_4e_2016.png}

% \exoslide{exo_12_p}
% \exoslide{exo_20_p}
% \exoslide{exo_2_p}

\bsec{Rappel}
\bsubsec{Représentation Graphique}
\slide{Cours}{
    \bchap{\longTitle}
    \ssec\ssubsec

    \pr{Représentation graphique}{
        Un graphique présente une situation de proportionnalité si l'ensemble de ces points sont alignés avec l'origine.
    } 
}

\bsubsec{Produit en croix}
\slide{}{
    \ssubsec

    \pr{Egalité des produits en croix}{
        Dans le tableau de proportionnalité:
        \begin{center}
            \propTable{a}{c}{b}{d}
        \end{center}
        On a l'égalité: $a \times d = b \times c$. 
    }

    \rmk{Egalité des quotients}{
        On a aussi : $ \frac{a}{b} = \frac{c}{d}$.
    }
}

\slide{Exercices}{
    \exo{Myriade : Proportionnalité}{
        \vspace{-0.25cm}
        \multiColItemize{3}{%
            \item 7 p143
            \item 2 p142
        }%
        %
        Pour vendredi:
        \multiColItemize{3}{%
            \item 13 p144
        }%
    }
}

% \exoslide{exo_7_p143}
% \exoslide{exo_2_p142}
% \exoslide{exo_13_p144}

\bsec{Pourcentage}
\bsubsec{Pourcentage d’un nombre}
\slide{Cours}{
    \ssec\ssubsec
    \df{Pourcentage}{
        Calculer $t\%$ d'un nombre revient à multiplier ce nombre par $\frac{t}{100}$.
    }
    
    \expl{}{
        Calculer $30\%$ de $60$.
    }
}

\bsubsec{Pourcentage d’évolution}

\slide{}{
    \pr{}{
        \begin{itemize}
            \item Augmenter un nombre de $t\%$ revient à le multiplier par $1+\frac{1}{100}$.
            \item Diminuer un nombre de $t\%$ revient à le multiplier par $1-\frac{1}{100}$.
        \end{itemize}
    }
}

\slide{Exercices}{
    \exo{Myriade : Pourcentages}{
        \vspace{-0.25cm}
        \multiColItemize{3}{%
            \item 14 p144
            \item 23 p145
        }%
        %
        Devoir-Maison pour Lundi 25 mars:
        \multiColItemize{3}{%
            \item 60 p151
            \item 63 p15113
            \item 16 p144
        }%
    }
}