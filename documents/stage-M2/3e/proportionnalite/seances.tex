% VARIABLES %%%
\def\authors{CADOT - COURTIN - PESIN}
% \date{\today}
\def\longTitle{La Proportionnalité}
\def\shortTitle{Proportionnalité}
%%

% \disableAnimation
% \shortAnimation
\firstSlide

\def\imgPath{stage-M2/4e/proportionnalite/exercices/}
\def\imgExtension{_myriade_4e_2016.png}

\exoList{12 p139,20 p140,16 p138}[Myriade 4e 2016][3]

\exoslide{ex_12_p139}
\exoslide{ex_20_p140}[8cm]
\exoslide{ex_2_p138}[9cm]

\bsec{Rappel}
\bsubsec{Représentation Graphique}

\slide{Cours}{
    \bchap{\longTitle}
    \ssec\ssubsec

    \pr{Représentation graphique}{
        Un graphique présente une situation de proportionnalité si l'ensemble de ces points sont alignés avec l'origine.
    } 
}

\slide{}{
    \expl{}{
        \img{stage-M2/4e/proportionnalite/graphique_situation_proportionnelle.png}[8cm]
        Le graphique ci-dessus présente une situation de proportionnalité.
    }
}

\bsubsec{Produit en croix}
\slide{}{
    \ssubsec

    \pr{Egalité des produits en croix}{
        Dans le tableau de proportionnalité:
        \begin{center}
            \propTable{a}{c}{b}{d}
        \end{center}
        On a l'égalité: $a \times d = b \times c$. 
    }
    \vspace*{-0.5cm}
    \rmk{Egalité des quotients}{
        On a aussi:%
        \vspace*{-0.5cm}
        \begin{align*}%
            \frac{a}{b} = \frac{c}{d}%
        \end{align*}%
    }
}

\exoList{7 p143,2 p142,13 p144}[Myriade 3e 2016 - à terminer pour vendredi][3]

\def\imgPath{stage-M2/3e/proportionnalite/}
\def\imgExtension{_myr_3e_2016.png}

\exoslide{exo_7_p143}[5.4cm]
\exoslide{exo_2_p142}[7cm]
\exoslide{exo_13_p144}

\bsec{Pourcentage}
\bsubsec{Pourcentage d'un nombre}
\slide{Cours}{
    \ssec\ssubsec
    \df{Pourcentage}{
        Calculer $t\%$ d'un nombre revient à multiplier ce nombre par $\frac{t}{100}$.
    }
    
    \expl{}{
        Calculer $30\%$ de $60$.
    }
}

\bsubsec{Pourcentage d'évolution}

\slide{}{
    \pr{}{
        \begin{itemize}
            \item Augmenter un nombre de $t\%$ revient à le multiplier par $1+\frac{t}{100}$.
            \item Diminuer un nombre de $t\%$ revient à le multiplier par $1-\frac{t}{100}$.
        \end{itemize}
    }

    \expl{Myriade}{
        \vspace{-0.25cm}
        \multiColItemize{3}{%
            \item 14 p144
            \item 23 p145
        }%
    }
}

\exoslide{exo_14_p144}
\exoslide{exo_23_p145}

\exoList{60 p151,63 p151,16 p144}[Devoir-Maison - Lundi 25 mars][3]