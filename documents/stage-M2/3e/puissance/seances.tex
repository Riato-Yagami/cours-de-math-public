% VARIABLES %%%
\def\authors{PESIN - CADOT - COURTIN}
% \date{\today}
\def\longTitle{Les Puissances}
\def\shortTitle{Puissance}
%%

% \boolean{showA}
\disableAnimation
% \shortAnimation
% \firstSlide

\bsec{Puissance d'un nombre}
\bsubsec{Puissance d'exposant positif}

\slide{Cours - PRENDRE SON CAHIER A L'ENVERS}{
    \bchap{\longTitle}
    \vspace*{0.3cm}
    \ssec\ssubsec

    \df{}{
        Soient $n$ un entier supérieur ou égal à $1$ et $a$ un nombre relatif.
        \begin{align*}
            a^n = \palt{2}{
                \lPowBrace{a}{n}
            }
        \end{align*}
    }
}

\slide{}{
    \vspace*{-0.5cm}
    \expl{}{
        \startQuestions
        \fsize{13pt}
        \begin{enumerate}
            \iquestion{2^5}{\pow{2}{5}
            = \pow{4}{2} \times 2
            = 16 \times 2 = 32}
            \iquestion{(-3)^2}{\pow{(-3)}{2} = 9}
            \iquestion{-3^2}{-(3\times 3) = -9}
            \iquestion{2000^1}{2000}
            \iquestion{0^{12}}{\powBrace{0}{12} = 0}
            \iquestion{1^6}{\powBrace{1}{6} = 1}
            \iquestion{(\frac{2}{3})^3}{\pow{\frac{2}{3}}{3} 
            = \frac{4}{9} \times \frac{2}{3} 
            = \frac{8}{27}}
            \iquestion{0,5^2}{\pow{0,5}{2} = 0,25}
        \end{enumerate}
    }
}

\bsubsec{Puissance d'exposant négatif}
\slide{}{
    \ssubsec

    \df{}{
        Soient $n$ un entier supérieur ou égal à $1$ et a un nombre relatif non nul.
        \begin{align*}
            a^{-n} = \palt{2}{
                \frac{1}{a^n}
            }
        \end{align*}
    }
}

\slide{}{
    \expl{}{
        \startQuestions
        \begin{enumerate}
            \iquestion{3^{-2}}{\frac{1}{3^2} = \frac{1}{9}}
            \iquestion{5^{-1}}{\frac{1}{5^1} = \frac{1}{5}}
            \iquestion{\frac{1}{4^{3}}}{4^{-3}}
            \iquestion{\frac{1}{x^{3}}}{x^{-3}}
        \end{enumerate}
    }
}

\newsec{sub}{cvt}{NavyBlue}{Convention}
\slide{}{
    \cvt{}{
        \begin{itemize}
            \item Pour tout nombre relatif $a$ : $a^0 = 1$
            \item En particulier: $0^0 = 1$
        \end{itemize}
    }
}

\bsec{Propriété sur les puissances}
\bsubsec{Produit et quotient de deux puissances d'un même nombre}

\slide{}{
    \ssec\ssubsec

    \pr{}{
        Soient $q$ et $p$ deux entiers et $a$ un nombre relatif.
        \startQuestions
        \begin{itemize}
            \iquestion{a^q \times a^p}{a^{q+p}}
            \iquestion{(a^q)^p}{a^{q \times p}}
            \iquestion{\si a \neq 0, \frac{a^q}{a^p}}{a^{q-p}}
        \end{itemize}
    }
}


\slide{}{
    \expl{}{
        \startQuestions\startQuestions
        \begin{enumerate}
            \iquestion{11^2 \times 11^6}{11^{(2+6)} = 11^8}
            \iquestion{8^3 \times 8^2 \times 8^4}{8^{(3+2+4)} = 8^9}
            \iquestion{(2^3)^2}{2^{(3\times 2)}= 2^6}
            \iquestion{(2023^{11})^2}{2023^{(11\times 2)}= 2023^22}
            \iquestion{\frac{(-5)^6}{(-5)^3}}{(-5)^{6-3} = (-5)^{3}}
        \end{enumerate}
    }
}

\bsubsec{Puissances d'un produit, d'un quotient.}
\slide{}{
    \ssubsec

    \pr{}{
        Soient n un entier, a et b deux nombres relatifs.
        \startQuestions
        \begin{itemize}
            \iquestion{(a\times b)^n}{a^n \times b^n}
            \iquestion{\si b\neq 0, (\frac{a}{b})^n}{\frac{a^n}{b^n}}
        \end{itemize}
    }
}

\slide{}{
    \expl{}{
        \startQuestions
        \begin{enumerate}
            \iquestion{(2\times 3)^4}{2^4 \times 3^4}
            \iquestion{(\frac{2}{3})^5}{\frac{2^5}{3^5}}
            \iquestion{4^3 \times 7^3}{(4 \times 7)^3 = 28 ^3}
            \iquestion{\frac{36^7}{3^7}}{(\frac{36}{3})^7 = 12^7}
        \end{enumerate}
    }
}

\bsec{Puissance de $10$}
\bsubsec{Définition}
\slide{}{
    \ssec\ssubsec
    \rmk{}{
        \def\tenPow{1\underbrace{0...0}_{n \textrm{ zéros}}}
        Avec $n$ un entier positif, les puissances de $10$ s'écrivent :
        \startQuestions
        \begin{itemize}
            \iquestion{10^n}{\tenPow}
            \iquestion{10^{-n}}{
                \frac{1}{10^n} 
                = \frac{1}{\tenPow} 
                = \underbrace{0,0..0}_{n \textrm{ zéros}}1
            }
        \end{itemize}
    }
}

\slide{}{
    \expl{}{
        \startQuestions
        \begin{enumerate}
            \iquestion{10^5}{100\,000}
            \iquestion{10^{-4}}{0,0001}
            \iquestion{10^0}{1}
            \iquestion{10^1}{10}
            \iquestion{10^{-1}}{0,1}
            \iquestion{100}{10^{2}}
            \iquestion{0,001}{10^{-3}}
        \end{enumerate}
    }
}

\exoList{5,6,7,8}[Devoir-Maison (fiche d'exercices)][8]



