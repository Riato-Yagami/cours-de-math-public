% VARIABLES %%%
\def\authors{PESIN - CADOT - COURTIN}
% \date{\today}
\def\longTitle{La Proportionnalité}
\def\shortTitle{Proportionnalité}
%%

\disableAnimation
% \shortAnimation
% \firstSlide

% \newcommand{\exoslide}[2]{
%     \slide{}{
%         \img{stage-M2/proportionnalite/exercices/ex_#1_myriade_4e_2016.png}[#2]
%     }
% }

\bsec{Introduction}

\slide{Cours}{
    \bchap{\longTitle}
    \vspace*{0.3cm}
    \ssec

    \exo{Myriade: Activité 2 p 134}{}
}

\slide{}{
    \vspace*{-0.5cm}
    \img{stage-M2/proportionnalite/exercices/activite_2_p137_myriade_4e_2016.png}[12cm][-0.7cm]
}

\slide{}{
    \rmdr{Coefficient de proportionnalité}{
        Dans un tableau de proportionnalité,
        on passe d'une ligne à l'autre en multipliant par le coefficient de proportionnalité.
    }

    \expl{}{
        Voir tableau 1 de l'Activité d'Introduction.
    }
}

\bsec{Proportionnalité}
\bsubsec{Représentation graphique}

\slide{}{
    \ssec\ssubsec

    \pr{Représentation graphique}{
        Un graphique présente une situation de proportionnalité si 
        \palt{2}{l'ensemble de ces points sont alignés avec l'origine.}
    } 
}

\slide{}{
    \expl{}{
        \img{stage-M2/proportionnalite/graphique_situation_proportionnelle.png}[8cm]
        Le graphique ci-dessus présente une situation de proportionnalité.
    }
}

\slide{Exercice}{
    \exo{Myriade: p138-141}{
        \vspace{-0.25cm}
        \multiColItemize{3}{
            \item 8 p138
            \item 20 p140
            \item 21 p141
        }
        A finir pour lundi 18 décembre.
    }
}

\exoslide{20_p140}{8cm}
\exoslide{21_p141}{10cm}
\exoslide{8_p138}{10cm}

\bsubsec{Produits en croix}
\slide{Cours}{
    \ssubsec
    Soit $a,b,c,d$ quatres nombres.
    
    \pr{Egalité des produits en croix}{
        Dans le tableau de proportionnalité:
        \begin{center}
            \propTable{a}{c}{b}{d}
        \end{center}
        On a l'égalité: $a \times d = b \times c$. 
    }
}

% \slide{}{
%     \expl{}{ Les tableaux suivants sont-ils des tableaux de proportionnalités?
%         \begin{center}
%             \propTable{2}{5}{3}{7,5}
%             \propTable{1,5}{2,5}{3}{6}\\
%         \end{center}
%         \begin{align*}
%             &\palt{2}{2\times7,5 = 15 = 3 \times 5}\\
%             &\palt{3}{1,5\times6 = 9 \ne 10,5 = 3 \times 3,5}
%         \end{align*}
%     }
% }

\slide{}{
    \mthd{Calcul 4e proportionnelle}{
        Si l'on connait 3 valeurs par exemple $b,c,d$.
        On peut utiliser l'égalité des produits en croix pour calculer la 4e proportionnelle $a$.\\
        En effet:
        \begin{align*}
            a \times d &= b \times c\\
            \ialors \palt{2}{\frac{a \times d}{d}} &= \palt{2}{\frac{b \times c}{d}}
            \ialors a &= \palt{3}{\frac{b \times c}{d}}
        \end{align*}
    }
}

\slide{}{
    \expl{}{ 
        Compléter les tableaux de proportionnalités suivant :
        % \vspace{0.2cm}
        \begin{center}
            \propTable{11}{6}{\palt{2}{27,5}}{15}
            \propTable{4}{5}{3,2}{\palt{2}{4}}\\
            \vspace{0.2cm}
            \propTable{x}{6}{\palt{2}{\frac{3.6x}{6}}}{3,6}
            \propTable{\palt{2}{\frac{1}{12}}}{\frac{1}{6}}{3}{6}
        \end{center}
    }

    \rmk{Egalité des quotients}{
        On a aussi
        \begin{align*}
            \frac{a}{b} = \frac{c}{d}
        \end{align*}
    }
}

\slide{Exercices}{
    \exo{Myriade: p138}{
        \vspace{-0.25cm}
        \multiColItemize{2}{
            \item 1 p138
            \item 2 p138
            \item 5 p138
            \item 4 p138
        }
        A finir pour vendredi 22 décembre.
    }
}

\exoslide{1_p138}{11cm}
\exoslide{2_p138}{11cm}
\exoslide{5_p138}{11cm}
\exoslide{4_p138}{11cm}

\bsec{Pourcentages}
\bsubsec{Définition}
\slide{Cours}{
    \ssec\ssubsec

    \df{Pourcentage}{
        Calculer $t\%$ d'un nombre revient à multiplier ce nombre par $\frac{t}{100}$.
    }
    \vspace{-0.5cm}
    \expl{}{ 
        Calculer $30\%$ de $60$.\\
        On fait : \palt{2}{$60 \times \frac{30}{100} = 10$}
    }
    \vspace{-0.7cm}
}

\slide{}{
    \rmk{}{
        Tableau de proportionnalité de la situation précédente:
        \propTable{\palt{2}{10}}{60}{30}{100}
    }
}

\slide{Exercices}{
    \exo{Myriade : p144}{
        \vspace{-0.25cm}
        \multiColItemize{2}{
            \item 41 p144
            \item 44 p144
        }
    }
    \dm{Myriade}{
        \vspace{-0.25cm}
        \multiColItemize{3}{
            \item 3 p135
            \item 35 p143
            \item 47 p144
        } 
    }
    A finir pour lundi 8 janvier.
}

\exoslide{41_p144}{11cm}

\bsubsec{Déterminer un pourcentage}

\slide{Correction dans le cours}{
    \ssubsec
    \img{stage-M2/proportionnalite/exercices/ex_44_p144_myriade_4e_2016.png}[11cm]
    \palt{2}{
    Déterminer un pourcentage d'une portion;
    c'est trouver la fraction égale à cette portion de dénominateur 100.
    }
    \begin{align*}
        \palt{3}{
            \frac{\numprint{2617}}{\numprint{8235}} \approx \frac{31,78}{100}
        }
    \end{align*}
    \palt{4}{
        On a alors environ $31,78 \%$ de la population, licenciés dans un club de sport.
    }
}

\bsec{Ratio}
\bsubsec{Définition}

\newcommand{\red}[1]{\color{Red}#1\color{black}}
\newcommand{\green}[1]{\color{ForestGreen}#1\color{black}}

\slide{}{
    \ssec\ssubsec

    \df{Ratio}{
        Un ratio est une façon d'exprimer comment deux grandeurs (ou plus) se comparent.
    }

    \expl{1}{
        Une poche de bonbons est partagée entre Assan et Esteban dans le ratio $1:4$.\\
        \palt{2}{Esteban aura 4 fois plus de bonbons que Assan.} 
    }
}

\slide{}{
    \expl{2}{
        Une poche de bonbons est partagée entre Assan et Esteban dans le ratio $3:4$.\\
    }
    Si Assan en a 3, alors Esteban en a \palt{2}{4}.\\ 
    Si Assan en a 6, alors Esteban en a \palt{3}{8}.\\
    \vspace*{0.5cm}
    La quantité de bonbons d'Assan partagée en 3 est égale à la quantité de bonbons d'Esteban partagée en \palt{4}{4}.\\
    En effet par exemple:
    \begin{align*}
        \palt{5}{\frac{6}{3} = 2 = \frac{8}{4}}
    \end{align*}
}

\slide{}{
    \df{}{
        On dit que deux nombres $a$ et $b$ sont dans le ratio $3:4$ (notation) si:
        \begin{align*}
            \frac{a}{3} = \frac{b}{4}
        \end{align*}
    }
    
    \def\colWidth{0.7cm}
    \def\colA{\cellcolor{red!25}}
    \def\colB{\cellcolor{blue!25}}

    \begin{center}
        \begin{tabular}{|C{\colWidth}|C{\colWidth}|C{\colWidth}|C{\colWidth}|C{\colWidth}|C{\colWidth}|C{\colWidth}|}
            \hline
            \colA $\frac{a}{3}$ & \colA & \colA & \colB $\frac{b}{4}$ & \colB & \colB & \colB\\
            \hline
        \end{tabular}\\
        $\underbrace{\hspace{3.3cm}}_{a}\underbrace{\hspace{4.4cm}}_{b}$
    \end{center}
}

\slide{}{
    \rmk{}{
        Un ratio permet de parler des proportions de deux grandeurs (ou plus) les unes par rapport aux autres.
    }

    Dans notre exemple;
    Assan a reçu $\palt{2}{\frac{3}{7}}$ des bonbons,
    et Esteban $\palt{2}{\frac{4}{7}}$
}

\slide{}{
    \expl{3}{
        Adrien, Angélique et Ludivine se partagent des chocolats dans le ratio ${2:3:7}$.
    }
    Si Adrien en a 2, alors Angélique en a \palt{2}{3}{} et Ludivine en a \palt{2}{7}.\\
    Si Ludivine en a 21, alors Angélique en a \palt{3}{9}{} et Adrien en a \palt{3}{6}.
}

\slide{}{
    \df{}{
        On dit que trois nombres $a,b$ et $c$ sont dans le ratio $2:3:7$ si:
        \begin{align*}
            \frac{a}{2} = \frac{b}{3} = \frac{c}{7}
        \end{align*}
    }
}

\slide{Exercices}{
    \exo{}{
        Trois personnes décident de se partager une somme de $494\euro$ proportionnellement à leur temps de travail,
        suivant le ratio $4:7:8$.\\
        Combien chaque personne va-t-elle recevoir ?
        \palt{2}{
            \begin{align*}
                4+7+8 &= 19\\
                494\euro \times \frac{4}{19} &= 104\euro\\
                494\euro \times \frac{7}{19} &= 182\euro\\
                494\euro \times \frac{8}{19} &= 208\euro
            \end{align*}
        }
    }
}

\bsubsec{Echelle}

\slide{Cours}{
    \ssubsec
    \df{Echelle}{
        Une carte à l'échelle $1:\numprint{1000}$ signifie que:\\
        $1\cm$ sur la carte représente $\numprint{1000}\cm$ dans la réalité.
    }
}

\slide{}{
    \expl{}{
        On mesure sur une carte à l'aide de l'échelle graphique que $1\cm$ représente $1\km$.\\
        Quelle est l'échelle de la carte?
        \img{stage-M2/proportionnalite/carte-paris.jpg}[7cm]
        \palt{2}{
            $1\km = \numprint{100 000}\cm$\\
            On alors une échelle de $1:\numprint{100 000}$
        }
    }
}

\bsec{Vitesse}

\slide{}{
    \ssec
    \expl{}{
        Une voiture parcours $120\km$ en $1\hour$.
        En supposant sa vitesse constante;
        combien parcourt-elle de $\km$ en $1\hour30$?
    }
    \vspace*{-0.5cm}
    \rmdr{Conversion}{
        \vspace*{-0.7cm}
        \begin{align*}
            60\min &= \palt{2}{1,0\hour}\\
            \alors 1\min &= \palt{4}{\frac{1}{60}\hour \approx 0,01667\hour}\\
            \alors 30\min &= \palt{5}{\frac{30}{60}\hour = 0,5\hour}\\
            \et 1\hour30\min &= \palt{5}{1,5\hour}
        \end{align*}
    }
}

\slide{}{
    \def\colWidth{1.5cm}
    On peut construire le tableau de proportionnalité :
    \begin{center}
        \begin{tabular}{|C{2.5cm}|C{\colWidth}|C{\colWidth}|}
            \hline
            Temps & $\palt{2}{1\hour}$ & $\palt{2}{1,5\hour}$ \\ \hline
            Distance & $\palt{2}{120\km}$ & $\palt{3}{180\km}$ \\ \hline
        \end{tabular}
    \end{center}

    \df{Vitesse}{
        On définie la \textbf{vitesse} comme le coefficient de proportionnalité pour passer du temps à la distance.
        \begin{equation*}
            Vitesse = \palt{4}{\frac{Distance}{Temps}}
        \end{equation*} 
    }
}

\slide{}{
    \expl{}{
        La voiture a donc une vitesse:
        \begin{align*}
            v = \frac{d}{t} = \palt{2}{\frac{1}{120}} = \palt{3}{120\km / \hour}
        \end{align*}
    }
    \pr{}{
        On en déduit les deux formules suivantes:
        \begin{itemize}
            \item $ d = v \times t$
            \item $ t = \frac{d}{v}$
        \end{itemize}
    }
}

\slide{Exercices}{
    \exo{Myriade : Echelle et Vitesse}{
        \vspace{-0.25cm}
        \multiColItemize{3}{
            \item 32 p142
            \item 27 p142
            \item 28 p142
        }
    }
}

\exoslide{32_p142}{11cm}
\exoslide{27_p142}{11cm}
\exoslide{28_p142}{11cm}

\slide{Exercices}{
    \exo{Myriade : Proportionnalité}{
        \vspace{-0.25cm}
        \multiColItemize{3}{
            \item 56 p145
            \item 64 p146
            \item 65 p147
            \item 66 p147
            \item 68 p147
            \item 70 p147
            \item 74 p148
        }
    }
}

\exoslide{56_p145}{10cm}
\exoslide{64_p146}{8cm}
\exoslide{65_p147}{7cm}
\exoslide{66_p147}{11cm}
\exoslide{68_p147}{11cm}
\exoslide{70_p147}{11cm}

\newpage
\slide{}{
    \def\imgWidth{5.5cm}
    \begin{multicols*}{2}
        \img{stage-M2/proportionnalite/exercices/ex_74a_p148_myriade_4e_2016.png}[\imgWidth] \columnbreak
        \img{stage-M2/proportionnalite/exercices/ex_74b_p148_myriade_4e_2016.png}[\imgWidth]
    \end{multicols*}
}