\def\theme{Fiche de préparation: Les produits en croix}
\def\date{18/10/2023}

% \fontsize{13}{14}
% \selectfont

% \setboolean{outline}{true}
\setboolean{subsectionInOutline}{true}

\seanceInfo[Lundi 18 décembre 2023]
[\link{Proportionnalité}][2]
[Utiliser le produit en croix pour calculer la 4e proportionnelle]
[\ilink{4e}{07} \ilink{4e}{05}]
[
    \item Comparation de fractions
    \item Compréhension de formule de calcul littéral
    \item Lecture graphique
    \item Egalité / Inégalité des quotients pour justifier une situation de proportionnalité
]
[Activité découverte; représentation graphique d'une situation de Proportionnalité 
(Myriade activité 2 p137)]
[\item Manuel Myriade 4e 2016
\item Règle
\item Calculatrice
]
%     \section*{Cours}
%     \section{Proportionnalité}
%     \subsection{Représentation graphique}
%     \pr{Représentation graphique}{}
%     \subsection{Produits en croix}
%     \pr{Egalité des produits en croix}{}
%     \mthd{Calcul 4e proportionnelle}{}
%     \rmk{Egalité des quotients}{}

\newpage

\prepTable{
    \prepRow{
        \section*{Correction}
        \limg{stage-M2/proportionnalite/exercices/ex_20_p140_myriade_4e_2016.png}
        \limg{stage-M2/proportionnalite/exercices/ex_21_p141_myriade_4e_2016.png}
        \limg{stage-M2/proportionnalite/exercices/ex_8_p138_myriade_4e_2016.png}
    }{
        \begin{itemize}[wide=0pt, leftmargin=*]
            \item 20p140: 
            \begin{itemize}[wide=0pt, leftmargin=*]
                \item Type: Reconnaitre situation de proportionnalité par lecture graphique.
                \item Attendus: "points alignés avec l'origines alors proportionnalité".
                \item Correction à l'oral en discussion avec la classe.
            \end{itemize}
            \item 21p140:
            \begin{itemize}[wide=0pt, leftmargin=*]
                \item Type: Constrution graphique / Reconnaitre situation de proportionnalité par lecture graphique et par le calcul.
                \item Attendus: "inégalités et egalités des quotients" / "point alignés".
                \item Correction au tableau par un élève en aillant pris en photo son cahier afin eviter le long processus de dessin d'un graphique au tableau.
            \end{itemize}
            \item 8p138
            \begin{itemize}[wide=0pt, leftmargin=*]
                \item Type: Calcul de $4^e$ proportionnelle.
                \item Attendus: "coefficient de proportionnalité" / opérations sur les colonnes.
                \item Correction au tableau par un élève.
            \end{itemize}
        \end{itemize}
    }{
        $15\min$
    }

    \prepRow{
        \section*{Cours}
        % \setcounter{section}{1}
        \color{Red}II - Proportionnalité\\
        \color{Green}1 - Représentation graphique\\
        \color{black}Prop - Représentation graphique\\
        \color{Green}2 - Produits en croix\\
        \color{black}Prop - Egalité des produits en croix\\
        Mthd - Calcul 4e proportionnelle\\
        Rmk - Egalité des quotients\\
        % \section{Proportionnalité}
        % \subsection{Représentation graphique}
        % \pr{Représentation graphique}{}
        % \subsection{Produits en croix}
        % \pr{Egalité des produits en croix}{}
        % \mthd{Calcul 4e proportionnelle}{}
        % \rmk{Egalité des quotients}{}
    }{
        \begin{enumerate}[wide=1pt, leftmargin=*]
            \item Finir Institutionnalisation du cours précédent
            (propriété à noter et graphique à dessiner).
            \item Revennir sur l'exercices 8p138 et demandé si d'autres techniques auraient pu être utilisé?
            Attendus: "produits en croix".
            \item Institutionnalisation de cette technique.
            \item Après 3 minutes de réflexion individuelles,
            démonstration en discussion avec les élèves du passage de l'égalité des produits en croix à la méthode de détermination de 4e proportionnelle.
            \item Résolution d'exemples, travail individuelles,
            avec les 4 configurations possibles dans un tableau de 2 par 2.
            \item Refaire le lien avec l'égalité des quotients car elle ne semblait pas très bien maitrisé lors de la séance précédente.
        \end{enumerate}
    }{
        $30\min$
    }

    \prepRow{
        \section*{Exercices}
        \limg{stage-M2/proportionnalite/exercices/ex_1_p138_myriade_4e_2016.png}
        \limg{stage-M2/proportionnalite/exercices/ex_2_p138_myriade_4e_2016.png}
        \limg{stage-M2/proportionnalite/exercices/ex_5_p138_myriade_4e_2016.png}
        \limg{stage-M2/proportionnalite/exercices/ex_4_p138_myriade_4e_2016.png}
    }{
        Exercices à commencer en classe et à finir à la maison\\
        Difficulté possible:
        \begin{itemize}
            \item Application à des problèmes de la vie réelle :
            Transférer les compétences de produit en croix à des situations réelles ou à des mots problèmes peut être difficile pour certains,
            surtout si le contexte est inconnu ou complexe.
            \item Constrution tableau de proportionnalité
            \item Erreurs dans la mise en place des équations :
            Positionner incorrectement les termes lors de la mise en place de l'équation pour résoudre le problème de proportionnalité.
            \item  Difficultés avec les fractions : Des difficultés à manipuler des fractions peuvent entraîner des erreurs dans le calcul du produit en croix,
            surtout si les nombres ne sont pas entiers.
            \item Identifier quand utiliser le produit en croix :
            Certains élèves peuvent ne pas reconnaître les situations qui requièrent le produit en croix,
            ou l'utiliser quand ce n'est pas nécessaire.
        \end{itemize}
    }{
        $5\min$
    }
}
\section*{Analyse à priori}

Dans l'exercice 20,
il aurait été possible de jouer sur les échelles données et ainsi modifier les variables didactiques.
Car tel que l'exercice est présenté,
il semble possible de "Calculer",
alors qu'il s'agit d'un exercice où l'on veut travailler les compétences "Raisonner" et "Communiquer".\\

On passe ensuite dans une l'exercice 21 ou il faut "Représenter" un ensemble de données dans un graphique,
puis à nouveau "Raisonner" et effectuer un changement de registre et cette fois-ci bien "Calculer" et "Raisonner".\\
Il faudra insister sur l'utilisation de l'égalité des quotients.
En effet de grandes lacunes sont apparues lors de la dernière séance sur l'utilisation de cette technique.

L'exercice 8 de calcul de $4^e$ proportionnelle peut être résolu par un grand nombre de méthode.
On cherchera cependant à favoriser l'utilisation du coefficient de proportionnalité,
on a ainsi choisi un exercice où les valeurs ne se calculent pas bien par d'autres méthodes tels que:
les opérations sur les colonnes ou le retour à l'unité.\\

Apres avoir fini de copier la fin du cours précédent dont il manquait la partie "Institutionnalisation".
Nous reviendrons sur l'exercice 8 pour introduire la propriété d'égalité des produits en croix.
Une bonne partie des élèves ont déjà la connaissance de cette égalité,
il est probable qu'elle a déjà été etudié en $5^e$.\\
Il sera alors simple de l'institutionnaliser à nouveau en faisant rappeler par un élève au tableau la méthode.
Nous en profiterons pour rappeler son lien étroit avec l'égalité des quotients;
car il est fondamental que cette propriété soit bien comprise pour que le cours sur Thalès, qui arrivera plus tard
dans cette année de $4^e$, ce déroule correctement.\\

Lors de cette séance il aurait été fortement possible d'introduire l'utilisation de TICE,
notamment de tableur afin de corriger, par exemple l'exercice 21p140.
Mais les élèves n'auraient pas pu eux même manipuler le logiciel alors nous décidons de garder cette approche pour une séance de TP numérique.\\

Pour palier aux difficultés possible il sera possible d'encourager des stratégies de vérifications;
telles que l'estimation pour s'assurer que les réponses sont raisonnables; notamment dans l'exercice 1 qui met en jeu des situations réelles.

\section*{Analyse à posteriori}

Comme on s'y attendait une majorité des élèves connaissaient la méthode de calcul de 4e proportionnelle.
Cependant il aurait été bénéfique de réintroduire l'égalité à travers une petite activité en cours.
En effet la méthode de calcul à été acquise comme une "recette",
les élèves ne semblaient pas savoir faire le lien avec les formules de proportionnalités du cours.\\
On aurait pu donc présenter l'égalité des produits en croix comme propriété corollaire de l'égalité des quotients;
à l'aide de la définition du quotient.

\demo{Egalité des produits en croix}{
    Soient $a,b,c,d$ 4 nombres tels que le tableau suivant soit un tableau de proportionnalité:
    \begin{align*}
        &\propTable{a}{c}{b}{d}\\
        \iona \frac{a}{b} &= \frac{a \times d}{b \times d}\\
        \iet \frac{c}{d} &= \frac{c \times b}{d \times b}\\
        \ior \frac{a}{b} &= \frac{c}{d} \textrm{ (égalité des quotients)}
        \ialors \frac{a \times d}{b \times d} &= \frac{c \times b}{d \times b}
        \idou a \times d &= c \times b \textrm{ (définitions du quotient)}
    \end{align*}
}

Cela aurait d'ailleur contribué à renforcer la nécessité de l'égalité des quotients qui semble toujours fragile.\\

Le retour sur l'exercice 8 a permis de relever une confusion,
présente chez de nombreux élèves,
sur les configurations ou l'on peut utiliser une méthode de calcul de 4e proportionnelle.\\
Si les 2 colonnes que l'on étudie de ne sont pas accolées dans le tableau les élèves ont du mal à comprendre que la méthode reste correcte.
Cela semble émaner d'une mauvaise compréhension du tableau de proportionnalité et en particulier sur les opérations entre les colonnes que l'on peut opérer.
En effet si les la propriété multiplicative a bien été comprise, je ne pense pas que ca soit le cas pour les propriété additive;
et en particulier sur les échanges de colonnes.
Il faudra alors jouer sur cette variable didactique à l'avenir pour rappeler ces propriétés essentielles.\\

Pour cette séance il me parrait maintenant essentielles de prévoire de gros outils de différentiations pédagogique,
tant certains élèves sont déjà très alaise avec la méthode de calcul de $4^e$ proportionnelle à l'aide du produit en croix,
alors que d'autres en on à peine entendu parler.