% VARIABLES %%%
\def\authors{PESIN - CADOT - COURTIN}
% \date{\today}
\def\longTitle{Les Puissances}
\def\shortTitle{Puissances}
%%

% \disableAnimation
% \shortAnimation
\firstSlide

\def\imgPath{stage-M2/4e/puissance/}
\def\imgExtension{_myr_4e_2016.png}

\newcommand{\powq}[2]{\iquestion{\pow{#1}{#2}}{#1^{#2}}}
\newcommand{\powiq}[3]{\iquestion{#1^{#2}}{\pow{#1}{#2} = #3}}

\qf{
    {$\pow{10}{3} = $, $1000$},
    {$6^2 = $, $36$},
    {$10^2 = $, $100$},
    {$\pow{2}{4} = $, $16$},
    {$10^2 = $, $100$}%
}[10]

\slide{COURS}{
    \bchap{\longTitle}
    \act{Myriade 1p76}{
        \vspace{-1cm}
        \img{\imgf{activite_1_p76}}[9.7cm]
    }
}

\bsec{Puissance d'un nombre}

\slide{}{
    \ssec
    \df{}{
        Soient $n$ un entier supérieur ou égal à $1$ et $a$ un nombre relatif.
        \begin{align*}
            a^n = \palt{2}{
                \lPowBrace{a}{n}
            }
        \end{align*}
    }
}

\slide{}{
    \expl{}{
        \startQuestions
        Ecrire avec la notation puissance:
        \begin{enumerate}
            \powq{6}{5}
            \powq{(-5)}{3}
            \powq{0,2}{4}
            \iquestion{\pow{3}{2}\times \pow{2}{4}}{3^3 \times 2^4}
        \end{enumerate}
    }
}

\slide{}{
    \expl{}{
        \startQuestions
        % \fsize{13pt}
        Calculer:
        \begin{enumerate}
            \powiq{2}{5}{32}
            \powiq{(-3)}{2}{9}
            \iquestion{-3^2}{-3 \times -3 = -9}
            \iquestion{2000^1}{2000}
            \powiq{0}{12}{0}
            \powiq{1}{6}{1}
            \powiq{\frac{2}{3}}{3}{\frac{8}{27}}
            \powiq{0,5}{2}{0,25}
        \end{enumerate}
    }
}

\slide{}{
    \expl{}{
        \startQuestions
        Donner le signe du resultat:
        \begin{enumerate}
            \item $(-1)^{12} \palt{2}{\geqslant 0}$
            \item $(-1)^{211} \palt{3}{\leqslant 0}$
            \item $(-879)^{75} \palt{4}{\leqslant 0}$
            \item $(-1984,6)^{2024} \palt{5}{\geqslant 0}$
        \end{enumerate}
    }

    \rmk{}{
        Soit $a$ un nombre négatif et $n$ un entier,
        $a^n$ est\palt{2}{positif } si\palt{2}{$n$ est pair },
        et\palt{3}{négatif } si\palt{3}{$n$ est impair}.
    }
}

\exoList{2 p80,4 p80,6 p80}[Myriade p80]

\exoslide{exo_2_p80}
\exoslide{exo_4_p80}
\exoslide{exo_6_p80}

\qf{
    {$\pow{9}{6} = $, $9^6$},
    {$\pow{0,15}{4} = $, $0{,}15^4$},
    {$2^4 = $, $16$},
    {$10^4 = $, $1000$}%
}[10]

\bsec{Puissances de 10}
\bsubsec{Propriétés}
\slide{COURS}{
    \ssec\ssubsec
    \vspace{-0.5cm}
    \act{Découverte}{
        \vspace{-0.7cm}
        \img{\imgf{activite_2_p76_1.a}}[10cm]
        % \vspace{-0.3cm}
        \small \color{BurntOrange} b. \color{black} Que remarque t-on?
        \vspace{-0.3cm}
        \img{\imgf{activite_2_p76_2.a}}[10cm]
        \small \color{BurntOrange} b. \color{black} Que remarque t-on?
    }
}

\slide{}{
    \pr{}{}
}