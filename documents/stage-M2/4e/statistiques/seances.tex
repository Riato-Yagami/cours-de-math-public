% VARIABLES %%%
\def\authors{CADOT - COURTIN - PESIN}
% \date{\today}
\def\longTitle{Statistiques}
\def\shortTitle{Statistiques}
\def\theme{\longTitle}
%%

% \disableAnimation
% \shortAnimation
% \firstSlide

\def\imgPath{stage-M2/4e/statistiques/}
\def\imgExtension{.png}

\bsec{Rappel}

\slide{COURS}{
    % \ssec
    \exo{Myriade 8 p161}{
        \imgp{exo_8_p161}
        A quelle fréquence apparait le tarif de 15$\euro$?
    }
}

\exoslide{activite_1_p156}

\bsec{Définition}

\slide{}{
    \df{Médianne}{
        La médianne est une valeur qui partage la série ordonné par ordre croissant,
        tel que les deux sous-séries est le même effectifs.
    }

    \expl{Série à effectifs total impair}{
        56 ; 67 ; 10 ; 35 ; 43 ; 9 ; 8\\
        \begin{enumerate}
            \item Ordonner les valeur de la série statistique par ordre croissant.
            \item Trouver la position de la médianne.
            \item Donner la médianne.
        \end{enumerate}
    }

    \expl{Série à effectifs total pair}{
        56 ; 67 ; 10 ; 35 ; 43 ; 9 ; 8\\
        \begin{enumerate}
            \item Ordonner les valeur de la série statistique par ordre croissant.
            \item Trouver la position de la médianne.
            \item Donner la médianne.
        \end{enumerate}
    }
}
