% VARIABLES %%%
\def\authors{CADOT - COURTIN - \href{https://juels.dev/}{PESIN}}
% \date{\today}
\def\longTitle{Statistiques}
\def\shortTitle{\MakeUppercase{\longTitle}}
\def\theme{\longTitle}
%%

% \disableAnimation
% \shortAnimation
% \firstSlide

\def\imgPath{stage-M2/4e/statistiques/}
\def\imgExtension{.png}

\bsec{Rappel}

\slide{COURS}{
    \bseq{Statistiques}
    \ssec%
    \vspace*{-0.5cm}
    \exo{Myriade 8 p161}{%
        \vspace*{-0.5cm}
        \imgp{exo_8_p161}[7cm]
        A quelle fréquence apparait le tarif de 15$\euro$?
    }
}

\bsec{Définitions}

\slide{}{
    \ssec \vspace*{-0.25cm}
    \act{1 p 156}{\vspace*{-0.75cm}%
        \imgp{activite_1_p156}[9.5cm]
    }
}

\slide{}{
    \df{médiane}{
        Une médiane est une valeur qui partage une série numérique ordonnée par ordre croissant en deux sous-séries de même effectif.
    }

    \df{étendue}{
        L'étendue d'une série numérique est la différence entre la plus grande valeur et la plus petite valeur de cette série.
    }
}

\slide{}{
    \expl{Pour les deux séries qui suivent}{\vspace*{-0.5cm}
        \begin{enumerate}
            \item Ordonner les valeurs de la série statistique par ordre croissant.
            \item Trouver la position d'une médiane.
            \item Donner une médiane.
            \item Donner l'étendue.
        \end{enumerate}

        \aalt{%
            Série à effectif total impair :
            \begin{align*}
                56 ; 67 ; 10 ; 35 ; 43 ; 9 ; 8
            \end{align*}}
        {%
            Série à effectif total pair :
            \begin{align*}
                3 ; 16 ; 10 ; 500 ; 12 ; 9 ; 60 ; 9
            \end{align*}
        }
    }
}

\section*{Exercices}

\exoList{2 p160, 3p160, 4p160, 5p160}[][4]

\exoslide{exo_2_p160}
\exoslide{exo_3_p160}
\exoslide{exo_4_p160}
\exoslide{exo_5_p160}