% VARIABLES %%%
\def\authors{PESIN - CADOT - COURTIN}
\def\date{(activité originale \href{http://rjorro.free.fr/vil9/4ieme/4e_DOCII4_TICEmoyennes.pdf}{Mme JORRO})}
\def\theme{Activité : Statistiques}
\def\imgPath{libre-office/}
%%
\vspace*{-1cm}

% \vspace*{-0.5cm}

\hint{%
    \begin{itemize}
        \item Amorce : \textbf{\href{https://juels.dev/math/attachment/amorce\_activite\_statistiques\_4e.ods}{amorce\_activite\_statistiques\_4e.ods}}
        \item une formule commence toujours par le signe «=»
        \item NB(cellule1 : cellule2) donne le nombre de valeurs dans la plage cellule1 : cellule2
        \item SOMME(cellule1 : cellule2) donne la somme des valeurs dans la plage cellule1 : cellule2
        \item MEDIANE(cellule1  : cellule2) donne la médiane dans la plage cellule1 : cellule2
        \item MIN(cellule1 : cellule2) donne la plus petite valeur dans la plage cellule1 : cellule2
        \item MAX(cellule1 : cellule2) donne la plus grande valeur dans la plage cellule1 : cellule2
    \end{itemize}
}

\vspace*{-0.5cm}

\section*{Exercice 1}

On trouve,
dans la feuille 1 de l'amorce, les tailles d'élèves d'une classe de $4^e$.

\begin{enumerate}
    \item Remplir les cellules B2, C2 et D2 de sorte que : 
    \begin{enumerate}
        \item en B2 on ait l'effectif total de la série des tailles.
        \item en C2 on ait la somme de toutes les tailles.
        \item en D2 on ait la taille moyenne de cette série.
    \end{enumerate}
    \item Remplir H2, tel qu'elle ait la médiane de la série des tailles.
    \item Remplir les cellules F2, G2 et H2 de sorte que : 
    \begin{enumerate}
        \item en F2 on ait la plus petite taille.
        \item en G2 on ait la plus grande taille.
        \item en H2 on ait la l'étendue de la série des tailles.
    \end{enumerate}
\end{enumerate}

\section*{Exercice 2}

On trouve,
dans la feuille 2 de l'amorce,
la répartition des jeunes inscrits dans un centre de vacances, en fonction de leur âge.

\begin{enumerate}
    \item \begin{enumerate}
        \item Remplir C2, tel qu'elle ait la somme des âges d'enfant de 10ans.
        \item En étirant C2, remplir la plage C3 : C9 de la même manière.
        \item Remplir D2, tel qu'elle ait la somme des âges de tous les jeunes.
    \end{enumerate}
    \item Remplir les cellules E2 et F2 de sorte que :
    \begin{enumerate}
        \item en E2 on ait l'effectif total de la série des âges.
        \item en F2 on ait l'âges moyen des jeunes.
    \end{enumerate}
    \item Construire un diagramme en bâtons représentant les âges des jeunes,
    en suivant les étapes suivantes :
    \begin{enumerate}
        \item sélectionner la plage A2 : B9
        \item choisir : $\rightarrow$ Insertion $\rightarrow$ \icon{diagramme} Diagramme
        \item cocher : Étapes $\rightarrow$ 2. Plage de données $\rightarrow$ Première colonne comme étiquette
    \end{enumerate}
    \item De la même manière construire un diagramme circulaire représentant les âges des jeunes.
    Il faudra en plus des étapes précédentes :
    \begin{itemize}
        \item sélectionner : Étapes $\rightarrow$ 1. Type de diagramme $\rightarrow$ \icon{secteur} Secteur
    \end{itemize}    
\end{enumerate}