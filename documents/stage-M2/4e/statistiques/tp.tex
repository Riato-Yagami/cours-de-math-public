% VARIABLES %%%
\def\authors{PESIN - CADOT - COURTIN}
% \date{\today}
\def\theme{TP - Statistiques}
%%

\section*{Exercice 1}

\begin{enumerate}
    \item 
    \hint{
        \begin{itemize}
            \item une formule commence toujours par le signe «=»
            \item NB(cellule1:cellule2) donne le nombre de valeurs dans la plage cellule1:cellule2
            \item SOMME(cellule1:cellule2) donne la somme des valeurs dans la plage cellule1:cellule2
        \end{itemize}
    }
    Remplir les cellules B2, C2 et D2 de sorte que : 
    \begin{enumerate}
        \item en B2 on ait l’effectif total de la série des tailles
        \item en C2 on ait la somme de toutes les tailles
        \item en D2 on ait la taille moyenne de cette série
    \end{enumerate}
    \item
    \hint{
        \begin{itemize}
            \item MEDIANE(cellule1:cellule2) donne la médiane dans la plage cellule1:cellule2
            \item MIN(cellule1:cellule2) donne la plus petite valeur dans la plage cellule1:cellule2
            \item MAX(cellule1:cellule2) donne la plus grande valeur dans la plage cellule1:cellule2
        \end{itemize}
    }
    Remplir les cellules B2, C2 et D2 de sorte que : 
    \begin{enumerate}
        \item en E2 on ait la médiane de la série des tailles
        \item en F2 on ait la plus petite taille
        \item en G2 on ait la plus grande taille
        \item en H2 on ait la plus grande taille
    \end{enumerate}
\end{enumerate}