% VARIABLES %%%
\def\authors{PESIN - CADOT - COURTIN}
% \date{\today}
\def\theme{Puissances Formules}
\thispagestyle{empty}
%%

\pr{}{
    Pour $x$ un nombre et $a$ et $b$ deux entiers.
    \begin{align*}
        x^a\times x^b = x^{a+b}
        \iet
        \frac{x^a}{x^b} = x^{a-b}
        \iet 
        (x^a)^b = x^{a\times b}
    \end{align*}
}

\rmk{Convention}{
    Pour $x$ un nombre: $x^0 = 1$ 
}

\subsection{Prefixes}

\def\colWidth{1.5cm}
\begin{center}
    \begin{tabular}{|C{2.5cm}|C{\colWidth}|C{\colWidth}|C{\colWidth}|C{\colWidth}|C{\colWidth}|C{\colWidth}|}
        \hline
        Prefixes & giga & méga & kilo & milli & micro & nano \\
        \hline
        Symbole & G & M & k & m & $\mu$ & n \\
        \hline
        Signification & $10^9$ & $10^6$ & $10^3$ & $10^{-3}$ & $10^{-6}$ & $10^{-9}$ \\
        \hline
    \end{tabular}
\end{center}

\expl{}{
    \begin{itemize}
        \item Un gigaoctet, noté Go, correspond à une quantité de données numériques de $10^9$ octets,
        soit un milliard d'octets.
        \item Un microgramme, noté $\mu$g, correspond à $10^{-6}$ grammes,
        soit un millionième de gramme.
    \end{itemize}
}