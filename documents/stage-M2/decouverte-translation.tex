% VARIABLES %%%
\def\theme{Une nouvelle transformation du plan}
\def\date{26/10/2023}
\def\author{\fsize{8pt}
Jules PESIN - Antonin CADOT - Laurent COURTIN
}
%%%%%%%%%%%%%%%

\section{Révisions}
\ex{Reconnaitre et décrire les transformations suivantes}

\begin{multicols}{2}
    \img{stage-M2/decouverte-translation/symetrie_axiale.png}[5cm]
    \img{stage-M2/decouverte-translation/symetrie_centrale.png}[7cm]
\end{multicols}

\ex{Pour chaque cas, dire s'il s'agit de symétrie centrale par rapport à un point ou non}
\img{stage-M2/decouverte-translation/exo2_symetrie_centrale.png}

\ex{Pour chaque cas, dire s'il s'agit de symétrie axiale par rapport à la droite $(d)$ ou non}
\begin{multicols}{4}
    \img{stage-M2/decouverte-translation/exo2_symetrie_axiale_1.png}
    \img{stage-M2/decouverte-translation/exo2_symetrie_axiale_2.png}
    \img{stage-M2/decouverte-translation/exo2_symetrie_axiale_3.png}
    \img{stage-M2/decouverte-translation/exo2_symetrie_axiale_4.png}
\end{multicols}

\newpage
\section{Nouvelle Transformation}

\ex{Décrire une démarche qui a permis de passer de la figure f1 à la f2, de même de la figure f3 à f4}
\img{stage-M2/decouverte-translation/exo3_decrire_demarche.png}

\ex{Décrire une démarche qui a permis passer de la figure $F_1$ à la $F_2$}
\img{stage-M2/decouverte-translation/translation_goose.png}[8cm]