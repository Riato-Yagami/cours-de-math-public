% VARIABLES %%%
\def\theme{Devoir Maison - Translation}
\def\date{20/11/2023}
\def\author{\fsize{8pt}
    PESIN - CADOT - COURTIN
}

\studentinfo

\vspace*{-0.25cm}

\begin{center}
    \it{Toutes les réponses sont à faire sur cet énoncé.}
\end{center}

\vspace*{-0.75cm}

\ex{Construire une translation sur quadrillage}{
    \vspace*{-1cm}
    \img{images/stage-M2/devoir-maison/exo_translation_triangle_quadrillage.png}[11cm]
    \it{\underline{Indication:} Pour répondre à la question 2)
    vous pouvez placer deux points
    l'un image de l'autres par cette transformation.}\\

    \vspace*{-0.25cm}

    2) \awsr{Le triangle $T_3$ est l'image du triangle $T_1$ par la translation qui transforme $F$ en $G$
    (les deux points placés aux extrémités droites des triangles $T_1$ et $T_3$)
    }
}

\ex{Construire une translation sur papier blanc}{
    Construire l'image du quadrilatère $ABCD$ par la translation qui transforme A en B.
    \vspace*{0.45cm}
    \img{stage-M2/devoir-maison/exo_translation_quadrilatere_papier_blanc.png}[4.5cm]
}


