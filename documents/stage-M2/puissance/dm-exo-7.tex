\def\theme{Correction DM}
\def\date{26/10/2023}
\def\authors{\fsize{8pt}
    PESIN - CADOT - COURTIN
}

\ex{7}{
    On cherche à calculer la distance
    $d$ parcourue par la lumière en une année.\\
    On sait que la vitesse de la lumière est:
    \begin{align*}
        v &= 300 000km/h\\
        &= 3 \times 10^5 km/h
    \end{align*}
    On va utiliser la formule: $d=v \times t$ avec $t$ la durée d'une année.\\
    On cherche alors le temps en seconde dans une année:
    \begin{align*}
        t &= 60 \times 60 \times 24 \times 365\\
        &= 31 536 000 s\\
        &= 3,1 536 \times 10^7 s
    \end{align*}
    Car il y a:
    \begin{itemize}
        \item $60$ secondes dans $1$ minute.
        \item $60$ minutes dans $1$ heure.
        \item $24$ heure dans $1$ journée.
        \item $365$ jours dans $1$ année. 
        (on pourait également prendre $365,25$ jours dans $1$ année pour etre plus précis.)
    \end{itemize}

    On a alors:
    \begin{align*}
        d &= v \times t\\
        &= 3 \times 10^5 \times 3,1 536 \times 10^7\\
        &= 3 \times 3,1 536 \times 10^{5+7}\\
        &= 9,4608 \times 10^{12} km
    \end{align*}
    La lumière parcours donc $9,4608 \times 10^{12} km$ en une année.
}