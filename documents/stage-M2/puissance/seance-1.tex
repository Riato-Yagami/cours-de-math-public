% VARIABLES %%%
\def\authors{PESIN - CADOT - COURTIN}
% \date{\today}
\def\longTitle{Les Puissances}
\def\shortTitle{Puissances}
%%

% \maketitle

\slide{Cours - PRENDRE SON CAHIER A L'ENVERS}{
    \bchap{\longTitle}
    \vspace*{0.3cm}
    \bsec{Puissance d'un nombre}
    \bsubsec{Puissance d'exposant positif}

    \df{}{
        Soient $n$ un entier supérieur ou égal à $1$ et $a$ un nombre relatif.
        \begin{align*}
            a^n = \palt{2}{
                \underbrace{a \times a \times ... \times a \times a}_{n \textrm{ fois}}
            }
        \end{align*}
    }
}

\slide{}{
    \expl{}{
        \begin{multicols*}{2}
            \begin{equation*}
                2^5 = 2 \times 2 \times 2 \times 2 \times 2\\
                (-3)^2 = 
            \end{equation*}
        \end{multicols*}
    }
}