\def\theme{Fiche de préparation: Les produits en croix}
\def\date{18/10/2023}

\setboolean{outline}{true}
\setboolean{subsectionInOutline}{true}

\seanceInfo[Lundi 18 décembre 2023]
[\link{Proportionnalité}][2]
[Utiliser le produit en croix pour calculer la 4e proportionnelle]
[\ilink{4e}{07} \ilink{4e}{05}]
[
    \item Comparation de fractions
    \item Compréhension de formule de calcul littéral
    \item Lecture graphique
    \item Egalité / Inégalité des quotients pour justifier une situation de proportionnalité
]
[Activité découverte; représentation graphique d'une situation de Proportionnalité 
(Myriade activité 2 p137)]

%     \section*{Cours}
%     \section{Proportionnalité}
%     \subsection{Représentation graphique}
%     \pr{Représentation graphique}{}
%     \subsection{Produits en croix}
%     \pr{Egalité des produits en croix}{}
%     \mthd{Calcul 4e proportionnelle}{}
%     \rmk{Egalité des quotients}{}

\prepTable{
    \prepRow{
        \section*{Correction}
        \limg{stage-M2/proportionnalite/exercices/ex_20_p140_myriade_4e_2016.png}
        \limg{stage-M2/proportionnalite/exercices/ex_21_p141_myriade_4e_2016.png}
        \limg{stage-M2/proportionnalite/exercices/ex_8_p138_myriade_4e_2016.png}
    }{
        \begin{itemize}[wide=0pt, leftmargin=*]
            \item 20p140: 
            \begin{itemize}[wide=0pt, leftmargin=*]
                \item Type: Reconnaitre situation de proportionnalité par lecture graphique
                \item Attendus: "points alignés avec l'origines alors proportionnalité"
            \end{itemize}
            \item 21p140:
            \begin{itemize}[wide=0pt, leftmargin=*]
                \item Type: Constrution graphique / Reconnaitre situation de proportionnalité par lecture graphique et par le calcul
                \item Attendus: "inégalités et egalités des quotients" / "point alignés"
            \end{itemize}
            \item 8p138
            \begin{itemize}[wide=0pt, leftmargin=*]
                \item Type: Calcul de $4^e$ proportionnelle
                \item Attendus: "coefficient de proportionnalité" / opérations sur les colonnes
            \end{itemize}
        \end{itemize}
    }{
        $15\min$
    }

    \prepRow{
        \section*{Correction}
        % \setcounter{section}{1}
        \color{Red}II - Proportionnalité\\
        \color{Green}1 - Représentation graphique\\
        \color{black}Prop - Représentation graphique\\
        \color{Green}2 - Produits en croix\\
        \color{black}Prop - Egalité des produits en croix\\
        Mthd - Calcul 4e proportionnelle\\
        Rmk - Egalité des quotients\\
        % \section{Proportionnalité}
        % \subsection{Représentation graphique}
        % \pr{Représentation graphique}{}
        % \subsection{Produits en croix}
        % \pr{Egalité des produits en croix}{}
        % \mthd{Calcul 4e proportionnelle}{}
        % \rmk{Egalité des quotients}{}
    }{
        \begin{enumerate}[wide=1pt, leftmargin=*]
            \item Finir Institutionnalisation du cours précédent
            (propriété à noter et graphique à dessiner).
            \item Revennir sur l'exercices 8p138 et demandé si d'autres techniques auraient pu être utilisé?
            Attendus: "produits en croix".
            \item Institutionnalisation de cette technique.
            \item Après 3 minutes de réflexion individuelles,
            démonstration en discussion avec les éleves du passage de l'égalité des produits en croix à la méthode de détermination de 4e proportionnelle.
            \item Résolution d'exemples, travail individuelles,
            avec les 4 configurations possibles dans un tableau de 2 par 2.
            \item Refaire le lien avec l'égalité des quotients car elle ne semblait pas très bien maitrisé lors de la séance précédente.
        \end{enumerate}
    }{
        $30\min$
    }

    \prepRow{
        \section*{Exercices}
        \limg{stage-M2/proportionnalite/exercices/ex_1_p138_myriade_4e_2016.png}
        \limg{stage-M2/proportionnalite/exercices/ex_2_p138_myriade_4e_2016.png}
        \limg{stage-M2/proportionnalite/exercices/ex_5_p138_myriade_4e_2016.png}
        \limg{stage-M2/proportionnalite/exercices/ex_4_p138_myriade_4e_2016.png}
    }{
        Exercices à commencer en classe et à finir à la maison
    }{
        $5\min$
    }
}
