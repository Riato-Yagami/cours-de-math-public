% VARIABLES %%%
\def\authors{PESIN - CADOT - COURTIN}
% \date{\today}
\def\longTitle{La Proportionnalité}
\def\shortTitle{Proportionnalité}
%%

% \disableAnimation
% \shortAnimation
% \firstSlide

\def\colWidth{1.5cm}

\newcommand{\propTable}[4]{
    \begin{tabular}{|C{\colWidth}|C{\colWidth}|}
        \hline
        $#1$ & $#2$ \\ \hline
        $#3$ & $#4$ \\ \hline
    \end{tabular}
}

\bsec{Introduction}

\slide{Cours}{
    \bchap{\longTitle}
    \vspace*{0.3cm}
    \ssec

    \exo{Myriade: Activité 2 p 134}{}
}

\slide{}{
    \vspace*{-0.5cm}
    \img{stage-M2/proportionnalite/exercices/activite_2_p137_myriade_4e_2016.png}[12cm][-0.7cm]
}

\slide{}{
    \rmdr{Coefficient de proportionnalité}{
        Dans un tableau de proportionnalité,
        on passe d'une ligne à l'autre en multipliant par le coefficient de proportionnalité.
    }

    \expl{}{
        Voir tableau 1 de l'Activité d'Introduction.
    }
}

\bsec{Proportionnalité}
\bsubsec{Représentation graphique}

\slide{}{
    \ssec\ssubsec

    \pr{Représentation graphique}{
        Un graphique présente une situation de proportionnalité si 
        \palt{2}{l'ensemble de ces points sont alignés avec l'origine.}
    } 
}

\slide{}{
    \expl{}{
        \img{stage-M2/proportionnalite/graphique_situation_proportionnelle.png}[8cm]
        Le graphique ci-dessus présente une situation de proportionnalité.
    }
}

\slide{Exercice}{
    \exo{Myriade: p138-141}{
        \vspace{-0.25cm}
        \begin{itemize}
            \item 8 p138
            \item 20 p140
            \item 21 p141
        \end{itemize}
        A finir pour lundi 18 décembre.
    }
}

\slide{}{
    \img{stage-M2/proportionnalite/exercices/ex_20_p140_myriade_4e_2016.png}[8cm]
}

\slide{}{
    \img{stage-M2/proportionnalite/exercices/ex_21_p141_myriade_4e_2016.png}[10cm]
}

\slide{}{
    \img{stage-M2/proportionnalite/exercices/ex_8_p138_myriade_4e_2016.png}[10cm]
}

\bsubsec{Produits en croix}
\slide{Cours}{
    \ssubsec
    Soit $a,b,c,d$ quatres nombres.
    
    \pr{Egalité des produits en croix}{
        Dans le tableau de proportionnalité:
        \begin{center}
            \propTable{a}{c}{b}{d}
        \end{center}
        On a l'égalité: $a \times d = b \times c$. 
    }
}

% \slide{}{
%     \expl{}{ Les tableaux suivants sont-ils des tableaux de proportionnalités?
%         \begin{center}
%             \propTable{2}{5}{3}{7,5}
%             \propTable{1,5}{2,5}{3}{6}\\
%         \end{center}
%         \begin{align*}
%             &\palt{2}{2\times7,5 = 15 = 3 \times 5}\\
%             &\palt{3}{1,5\times6 = 9 \ne 10,5 = 3 \times 3,5}
%         \end{align*}
%     }
% }

\slide{}{
    \mthd{Calcul 4e proportionnelle}{
        Si l'on connait 3 valeurs par exemple $b,c,d$.
        On peut utiliser l'égalité des produits en croix pour calculer la 4e proportionnelle $a$.\\
        En effet:
        \begin{align*}
            a \times d &= b \times c\\
            \ialors \palt{2}{\frac{a \times d}{d}} &= \palt{2}{\frac{b \times c}{d}}
            \ialors a &= \palt{3}{\frac{b \times c}{d}}
        \end{align*}
    }
}

\slide{}{
    \expl{}{ 
        Compléter les tableaux de proportionnalités suivant :
        % \vspace{0.2cm}
        \begin{center}
            \propTable{11}{6}{\palt{2}{27,5}}{15}
            \propTable{4}{5}{3,2}{\palt{2}{4}}\\
            \vspace{0.2cm}
            \propTable{x}{6}{\palt{2}{\frac{3.6x}{6}}}{3,6}
            \propTable{\palt{2}{\frac{1}{12}}}{\frac{1}{6}}{3}{6}
        \end{center}
    }

    \rmk{Egalité des quotients}{
        On a aussi
        \begin{align*}
            \frac{a}{b} = \frac{c}{d}
        \end{align*}
    }
}

\slide{Exercices}{
    \exo{Myriade: p138}{
        \vspace{-0.25cm}
        \begin{itemize}
            \item 1 p138
            \item 2 p138
            \item 5 p138
            \item 4 p138
        \end{itemize}
        A finir pour vendredi 22 décembre.
    }
}

\slide{}{
    \img{stage-M2/proportionnalite/exercices/ex_1_p138_myriade_4e_2016.png}[11cm]
}

\slide{}{
    \img{stage-M2/proportionnalite/exercices/ex_2_p138_myriade_4e_2016.png}[11cm]
}

\slide{}{
    \img{stage-M2/proportionnalite/exercices/ex_5_p138_myriade_4e_2016.png}[11cm]
}

\slide{}{
    \img{stage-M2/proportionnalite/exercices/ex_4_p138_myriade_4e_2016.png}[11cm]
}

\bsec{Pourcentages}

\slide{Cours}{
    \ssec

    \df{Pourcentage}{
        Calculer $t\%$ d'un nombre revient à multiplier ce nombre par $\frac{t}{100}$.
    }
    \vspace{-0.5cm}
    \expl{}{ 
        Calculer $30\%$ de $60$.\\
        On fait : \palt{2}{$60 \times \frac{30}{100} = 10$}
    }
    \vspace{-0.7cm}
    \rmk{}{
        Tableau de proportionnalité de la situation précédente:
        \propTable{\palt{2}{10}}{60}{30}{100}
    }
}

\bsec{Ratio}
\bsubsec{Ratio}

\newcommand{\red}[1]{\color{Red}#1\color{black}}
\newcommand{\green}[1]{\color{ForestGreen}#1\color{black}}

\slide{}{
    \ssec\ssubsec

    \df{Ratio}{
        On dit que deux nombres $a$ et $b$ sont dans le \textbf{ratio $2:3$}, si:
        \begin{equation*}
            \frac{a}{2} = \frac{b}{3}
        \end{equation*}
    }

    \expl{}{
        Dans une classe, il y a \red{4 filles } et \green{6 garçons}.\\
        Les filles et les garçons sont dans le ratio $\red{2}:\green{3}$.\\
        En effet, \red{4 } et \green{6 } sont dans le ratio $\red{2}:\red{3}$
        car : $\frac{\red{4}}{\red{2}} = \frac{\green{6}}{\green{3}}$.
    }
}
