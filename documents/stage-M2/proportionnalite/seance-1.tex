% VARIABLES %%%
\def\authors{PESIN - CADOT - COURTIN}
% \date{\today}
\def\longTitle{La Proportionnalité}
\def\shortTitle{Proportionnalité}
%%

% \disableAnimation
% \shortAnimation
% \firstSlide

\bsec{Introduction}

\slide{Cours}{
    \bchap{\longTitle}
    \vspace*{0.3cm}
    \ssec

    \ex{Myriade: Activité 2 p 134}{}
}

\slide{}{
    \vspace*{-0.5cm}
    \img{stage-M2/proportionnalite/seance-1/activite_2_p137_myriade_4e_2016.png}[12cm][-0.7cm]
}

\slide{}{
    \rmdr{Coefficient de proportionnalité}{
        Dans un tableau de proportionnalité,
        on passe d'une ligne à l'autre en multipliant par le coefficient de proportionnalité.
    }

    \expl{}{
        Voir tableau 1 de l'Activité d'Introduction.
    }
}

\bsec{Proportionnalité}
\bsubsec{Représentation graphique}

\slide{}{
    \ssec\ssubsec

    \pr{Représentation graphique}{
        Un graphique présente une situation de proportionnalité si 
        \palt{2}{l'ensemble de ces points sont alignés avec l'origine.}
    } 
}

\slide{}{
    \expl{}{
        \img{stage-M2/proportionnalite/seance-1/graphique_situation_proportionnelle.png}[8cm]
        Le graphique ci-dessus présente une situation de proportionnalité.
    }
}

\slide{Exercices}{
    \ex{Myriade}{
        \vspace{-0.25cm}
        \begin{itemize}
            \item 8 p138
            \item 20 p140
            \item 21 p141
        \end{itemize}

        A finir pour lundi 18 décembre.
    }
}

\slide{}{
    \img{stage-M2/proportionnalite/seance-1/ex_8_p138_myriade_4e_2016.png}[10cm]
}

\slide{}{
    \img{stage-M2/proportionnalite/seance-1/ex_20_p140_myriade_4e_2016.png}[8cm]
}

\slide{}{
    \img{stage-M2/proportionnalite/seance-1/ex_21_p141_myriade_4e_2016.png}[10cm]
}

