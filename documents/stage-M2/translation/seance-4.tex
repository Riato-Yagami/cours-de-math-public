% VARIABLES %%%
\def\authors{PESIN - CADOT - COURTIN}
% \date{\today}
\def\longTitle{Translation}
\def\shortTitle{Translation - Construire}
%%

% \maketitle
% \slide{Exercices}{
%     \img{/stage-M2/seance-4/exo_13_figure_papier_quadrille.png}
% }

\slide{Exercices}{
    \def\tikzScale{0.5cm}
    \def\figShift{(4, 3)}
    \def\pointColor{black}
    \def\fig{(0,0) -- (0,2) -- (1,2) -- (1,1) -- (3,1) -- (3,0) -- cycle}
    \begin{minipage}{0.45\textwidth}
        \img{/stage-M2/seance-4/exo_13_figure_papier_quadrille.png}
    \end{minipage}
    \begin{minipage}{0.45\textwidth}
        \begin{tikzpicture}[x=\tikzScale, y=\tikzScale]
            \fill[green!30, shift= \figShift] \fig;
            \drawGrid{11}{8}{0.01mm};
            % \onslide<2->{
                \draw[line width= 0.24mm, Green, shift= \figShift] \fig;
                \node[shift=\figShift, color = red] at (1.5,0.5) {1};
                \drawPoint{A}{0}{0}{\figShift};
                \drawPoint{B}{3}{0}{\figShift};
                \drawPoint{C}{0}{2}{\figShift};
                % \node[shift=\figShift] at (-0.5,-0.5) {A};
                % \node[shift=\figShift] at (3.5,-0.5) {B};
                % \node[shift=\figShift] at (-0.5,2.5) {C};
            % }
        \end{tikzpicture}
    \end{minipage}
}

\slide{}{
    \img{/stage-M2/seance-4/exo_5_p183_myriade_4e_2016.png}[9cm]
}

% \slide{}{
%     \img{/stage-M2/seance-4/exo_translation_triangle_quadrillage.png}[9cm]
% }

\slide{Cours}{
    \stepcounter{subsec}
    \bsubsec{Sur Papier blanc}
    \expl{Translation d'un point}{
        Construire le point A' image du point A par la translation qui transforme B en C.
    }
    \vspace*{-1cm}
    \img{stage-M2/seance-4/translation_point.png}[6cm]
}


\slide{}{
    \img{stage-M2/seance-4/translation_point.png}[9cm]
}

\slide{}{
    \begin{minipage}{0.35\textwidth}
        \img{stage-M2/seance-4/translation_point.png}
    \end{minipage}
    \begin{minipage}{0.55\textwidth}
        \underline{Etape de construction:}
        \begin{enumerate}
            \pause\item Tracer la droite parallèle à $(BC)$ passant par $A$.
            \pause\item Sur cette nouvelle droite, 
            placer $A'$ tel que $AA' = BC$ et $A$ vers $A'$ soit dans le même sens que $B$ vers $C$.
        \end{enumerate}
    \end{minipage}
    
    \pause\rmk{}{On a $(BC)//(AA')$ et $BC=AA'$ alors $ABCA'$ est un parallélogramme.}
}

\slide{}{
    \pr{}{
        Si $A'$ est l'image du point $A$ par la translation qui transforme $B$ en $C$,
        alors $ABCA'$ est un parallélogramme.
    }

    \pause\rmk{}{
        Pour construire l'image d'une figure par une translation,
        on construit l'image de chacun de ses points par cette même translation.
    }
}

\slide{}{
    \expl{Translation d'une figure}{
        Construire l'image du triangle ABC par la translation qui envoie D sur E.
    }

    \img{stage-M2/seance-4/translation_triangle.png}[6cm]
}