\dm{IsoPolis}
\setGrade{6e}

% \section*{IsoPolis : }

\subsection*{Introduction}
Vous êtes responsable de l'aménagement urbain de la nouvelle ville dont vous aurez également la liberté de choisir le nom.  

\subsection*{Consignes obligatoires :}
Votre plan urbain devra inclure les éléments suivants :
\begin{itemize}
    \item Au moins trois bâtiments.
    \item Au moins un bâtiment composé de deux solides simples ou plus.
    \item Des bâtiments utilisant des cubes et des pavés droits.
    \item Au moins un bâtiment utilisant un prisme droit ou une pyramide.
\end{itemize}

\subsection*{Consignes supplémentaires (facultatives) :}
Pour enrichir votre ville, vous pouvez intégrer :
\begin{itemize}
    \item Des bâtiments utilisant des cylindres.
    \item Des bâtiments utilisant des cônes.
    % \item Des solides ayant une base polygonale non rectangulaire.
    \item Une ou plusieurs îlots adjacents.
\end{itemize}

\subsection*{Travail préliminaire :}
Avant de commencer, réfléchissez aux points suivants :
\begin{enumerate}
    \item Voulez-vous suivre un thème spécifique ? 
    (Par exemple : une ville futuriste, far-west, médiévale, fantastique, etc.)
    \item Quels bâtiments voulez-vous intégrer dans votre ville ? 
    (Par exemple : un hôpital, une école, des maisons, etc.)
    \item Dessinez un brouillon sur une feuille de papier isométrique de quelques inspirations.
\end{enumerate}

\subsection*{Procédure de construction :}
\begin{enumerate}
    \item Tracer l'empreinte de votre ville :
    \begin{itemize}
        \item Sur une première feuille, dessinez l'empreinte de votre ville et nommez-y les batiments placés.
        \item Conseil : réfléchissez à l'usage prévu pour chaque bâtiment et leur emplacement dans la ville.
    \end{itemize}
    \item Construire les bâtiments :
    \begin{itemize}
        \item Sur une deuxième feuille isométrique, reproduisez l'empreinte de votre ville au crayon à papier.
        \item Toujours au crayon à papier, dessinez les solides correspondant à vos bâtiments en respectant les proportions.
        \item Effacez les traits cachés (ceux qui se trouvent derrière un bâtiment ou à l'intérieur d'une structure complexe).
    \end{itemize}
    \item Laissez libre cours à votre imagination : ajoutez des détailles à vos batiments comme vous le souhaitez !
\end{enumerate}
