\setSeq{6bis}{Représentation isometrique}
\setGrade{6e}

\newcommand{\csiso}[1]{
    \begin{center}
        \isometric[0.75][1pt][0.25]{#1}
    \end{center}
}

% \exo{Compléter les grilles isométriques suivantes}{
%     \multiColEnumerate{1}{
%         \item \dividePage{
%             Solides :
%             \sIso{
    \draw [thick] (4,4) -- (4,6) -- (6,8) -- (6,7) -- (7,7) -- (7,8) -- (6,8) -- (4,8) -- (3,7) -- (3,6) -- (4,6);
    \draw [thick] (6,7) -- (5,6) -- (6,6) -- (7,7);
    \draw [thick] (6,6) -- (6,5) -- (5,4) -- (4,4) -- (5,5) -- (5,6);
    \draw [thick] (6,5) -- (5,5);
    \draw [thick] (3,6) -- (4,7) -- (4,8);
    \draw [thick] (4,7) -- (5,7);
}
%         }{
%             Empruntes :
%             \sIso{}
%         }
%         \item \dividePage{
%             \sIso{}
%         }{
%             \sIso{
    \draw [thick] (3,7) -- (3,6) -- (4,6) -- (4,5) -- (5,5) -- (5,7) -- (3,7) -- cycle;
}
%         }
%         \item \dividePage{
%             \sIso{
    % \fill[thick,color=gradeColor,fill=gradeColor,fill opacity=0.10] (5,9) -- (6,8) -- (6,7) -- (7,7) -- (7,9) -- cycle;
    % \fill[thick,color=gradeColor,fill=gradeColor,fill opacity=0.10] (5,6) -- (5,7) -- (4,7) -- cycle;
    \draw [thick] (3,7) -- (4,8) -- (5,9) -- (7,9) -- (7,8) -- (7,7) -- (6,7) -- (6,8) -- (5,7) -- (5,6) -- (4,7) -- (3,6) -- (4,5) -- (6,7);
    \draw [thick] (7,7) -- (5,5) -- (4,5);
    \draw [thick] (4,7) -- (5,7);
    \draw [thick] (3,7) -- (3.50,6.50);
    \draw [thick] (5,9) -- (6,8);
}
%         }{
%             \sIso{}
%         }
%         \item \dividePage{
%             \sIso{}
%         }{
%             \csiso{
    \draw [thick] (3,5) -- (3,7) -- (6,7) -- (3,5) -- cycle;
}
%         }
%     }
% }

% \exo{}{Nommer le plus de solide particulier composant cet assemblages de solides.
%     \begin{center}
    \isometric[0.75][1pt][0.5]{
        \draw [shift={(9,11)},thick]  plot[domain=2.36:5.50,variable=\t]({1*1*cos(\t r)+0*1*sin(\t r)},{0*1*cos(\t r)+1*1*sin(\t r)});
        \draw [shift={(12,14)},thick]  plot[domain=2.36:5.50,variable=\t]({1*1*cos(\t r)+0*1*sin(\t r)},{0*1*cos(\t r)+1*1*sin(\t r)});
        \draw [shift={(13,15)},thick]  plot[domain=2.36:5.50,variable=\t]({1*1*cos(\t r)+0*1*sin(\t r)},{0*1*cos(\t r)+1*1*sin(\t r)});
        \draw [shift={(13,15)},thick]  plot[domain=-0.79:2.36,variable=\t]({1*1*cos(\t r)+0*1*sin(\t r)},{0*1*cos(\t r)+1*1*sin(\t r)});
        \draw [shift={(10,17)},thick]  plot[domain=1.99:5.86,variable=\t]({1*1*cos(\t r)+0*1*sin(\t r)},{0*1*cos(\t r)+1*1*sin(\t r)});
        % \draw [thick,dashed] (0.50,12.50) -- (8.50,4.50) -- (18.50,14.50) -- (10.50,22.50) -- (0.50,12.50) -- cycle;
        % \draw [thick,dashed] (5,11) -- (0.50,6.50) -- (4.50,2.50) -- (9,7) -- (5,11) -- cycle;
        \draw [thick] (13,13) -- (11,13) -- (11,15) -- (11,16) -- (9,16) -- (9,18) -- (9.78,18);
        \draw [thick] (11,13) -- (10,12) -- (10,14) -- (10,15) -- (11,16);
        \draw [thick] (10,14) -- (9,13) -- (8,12) -- (7.50,12);
        \draw [thick] (10,12) -- (12,12) -- (13,13) -- (12.97,13.55);
        \draw [thick] (9,16) -- (8,15) -- (6,13) -- (6,12) -- (8,14) -- (7,14) -- (7,15) -- (5,13) -- (6,13);
        \draw [thick] (6,12) -- (6,11) -- (7,13) -- (8,13) -- (7,11) -- (6,11);
        \draw [thick] (11.41,12) -- (9.71,10.29);
        \draw [thick] (10,13.41) -- (8.29,11.71);
        \draw [thick] (12.29,15.71) -- (11.29,14.71);
        \draw [thick] (13.71,14.29) -- (12.71,13.29);
        \draw [thick] (13,18) -- (13,16);
        \draw [thick] (9.59,17.91) -- (12,19) -- (10.91,16.59);
        \draw [thick] (9,18) -- (6,15) -- (6,14) -- (5,15) -- (4,14) -- (5,13);
        \draw [thick] (11.55,18) -- (13,18);
        \draw [thick] (5,15) -- (6,15);
        \draw [thick] (4,12) -- (4,11) -- (5,11) -- (5.50,12.50) -- (4,12) -- cycle;
        \draw [thick] (5.50,12.50) -- (4,11);
        \draw [thick] (7,10) -- (8,10) -- (8,7) -- (6,9) -- (8,10);
        \draw [thick] (6,9) -- (7,10);
    }
\end{center}
% }

\df{Cube}{
    \sIso{
    % \draw [thick,dashed] (0.50,12.50) -- (8.50,4.50) -- (18.50,14.50) -- (10.50,22.50) -- (0.50,12.50) -- cycle;
    % \draw [thick,dashed] (5,11) -- (0.50,6.50) -- (4.50,2.50) -- (9,7) -- (5,11) -- cycle;
    \draw [thick] (3,7) -- (5,9) -- (5,7) -- (7,7) -- (5,5) -- (3,5) -- (5,7) -- (3,5);
    \draw [thick] (5,9) -- (7,9) -- (7,7);
    \draw [thick] (3,5) -- (3,7);
    \draw [thick,dashed] (5,7) -- (7,9);
    \draw [thick,dashed] (5,5) -- (5,7);
    \draw [thick,dashed] (5,7) -- (3,7);
}  
}

\df{Cône}{
    \csiso{
    \draw [shift={(4,6)},thick]  plot[domain=1.83:6.02,variable=\t]({1*1.41*cos(\t r)+0*1.41*sin(\t r)},{0*1.41*cos(\t r)+1*1.41*sin(\t r)});
    \draw [shift={(4,6)},thick,dashed]  plot[domain=-0.26:1.83,variable=\t]({1*1.41*cos(\t r)+0*1.41*sin(\t r)},{0*1.41*cos(\t r)+1*1.41*sin(\t r)});
    \draw [thick] (3.48,7.32) -- (6.50,8.50) -- (5.32,5.48);
}
}


