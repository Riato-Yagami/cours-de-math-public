% VARIABLES %%%
\def\authors{\jules}
% \date{\today}
\def\longTitle{Rentrée}
\def\shortTitle{RENTREE}
\setcounter{seq}{0}
\def\title{\longTitle}
% \bseq{\longTitle}

\setboolean{showRef}{false}

\def\imgPath{enseignement/special/1er-seance/}
\def\imgExtension{.png}
\def\date{Jeudi 5 Septembre 2024}
\def\grade{4e}
\def\class{E}
%%

\slide{\hfill RENTREE - \date{} - \grade \class \hfill}{%
    \begin{center}
        Professeur de Mathématiques : \\
        \huge \m{M. PESIN}
    \end{center}
}

\slide{Plan de classe}{
    \imgp{plan-de-classe-\grade\class}
}

\slide{Règle de vie de classe - RESPECT et TRAVAIL}{
    \begin{enumerate}
        \item Lorsqu'une personne s'adresse à la classe, on se tait et on l'écoute.
        \item On lève la main pour demander la parole.
        \item On fait des maths.
    \end{enumerate}
}

\slide{Règle de travail}{
    \begin{enumerate}
        \item On prend le cours, les cahiers peuvent être relevés.
        \item On prend la correction des exercices au stylo vert.
        \item On participe, on pose des questions (pas de questions bêtes dans l'apprentissage !) et on y répond (toujours en levant la main)
        \item On fait des maths !
        \item Il y aura des évalutions régulières, notées ou non, cela sera précisé.
    \end{enumerate}
}

\slide{Matériel - apporté à CHAQUE cours}{
    \begin{enumerate}
        \item 2 Cahiers - $24 \times 32$ - grands carreaux : %
        \multiColItemize{2}{
            \item cahier de cours
            \item cahier d'exercices
        }
        \item Réserve de copies - simples et doubles - grands carreaux
        \item Matériel de géométrie : %
        \multiColItemize{3}{
            \item règle graduée
            \item équerre
            \item compas
            \item rapporteur
            \item crayon à papier
        }
        \item Calculatrice collège
    \end{enumerate}
}

\slide{Activité - Course aux nombres}{
    \imgp{can-\grade}[11cm]
}

\slide{Emploi du temps}{
    \imgp{edt}[11cm]
}