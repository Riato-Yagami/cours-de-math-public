% VARIABLES %%%
% \def\authors{Jules Pesin}
% \setGrade{6e}
\setTitle{LibreOffice \Calc - Prise en main}
% \input{header/special/libreoffice.tex}
\colorlet{gradeColor}{calc}
\emptyBackground
\def\amorce{\textit{calc-prise-en-main.ods}}
\def\iconPath{libreOffice/calc/}

\definecolor{cell}{HTML}{CCCCCC} % #CCCCCC

\NewDocumentCommand{\cell}{O{cellule}}{%
    \renewcommand\fbox{\fcolorbox{cell}{white}}%
    \fbox{\ttfamily\bfseries#1}%
}

\NewDocumentCommand{\ccalc}{O{cellule}}{%
    {\ttfamily\bfseries=#1}%
}
%%

LibreOffice \Calc est un tableur puissant qui va t'aider à gérer des données, effectuer des calculs et présenter des informations sous forme de tableaux ou de graphiques.
\\\\
Cette fiche de prise en main est accompagnée d'un fichier \Calc à éditer : \amorce.

\section{Découvrir l'interface}

\begin{enumerate}
    \item Ouvre le logiciel \Calc.
    \item Observe l'interface :
    \begin{enumerate}
    \item En haut, il y a la barre de menus et la barre d'outils pour accéder à diverses fonctions.
    \item Au centre, il y a la grille de \cell, organisée en lignes (numérotées) et colonnes (indiquées par des lettres).
    \item En bas, tu vois la barre de feuilles, où tu peux créer de nouvelles feuilles de donnés.
    \end{enumerate}
    \item Procédure à suivre pour ouvrir le fichier d'accompagnement :
    \begin{itemize}
        \item Télécharger le fichier \amorce.
        \item Cliquer sur :
        Barre de menus $\rightarrow$ \textbf{Fichier} $\rightarrow$ \icon{open-file} \textbf{Ouvrir...}
        \item Localiser le fichier \amorce{} sur votre ordinateur.
    \end{itemize}
    \item Teste le fonctionnement du tableur :
    \begin{itemize}
    \item Clique sur la cellule \cell[C3] et tape ton prénom.
    \item Appuie sur \textbf{Entrée} pour valider la saisie.
    \end{itemize}
\end{enumerate}

% \hint{Si tu fais une erreur, tu peux effacer le contenu d'une cellule en appuyant sur la touche \textbf{Suppr}.}

\section{Effectuer des calculs simples}

\begin{enumerate}
    \item Clique sur \cell[C6].
    \item Tape un calcul en commençant par «$=$» (par exemple : \ccalc[3+5]) et appuie sur \textbf{Entrée}.
    \item Essaie d'autres opérations :
    \multiColItemize{2}{
        \item \textbf{Soustraction} : \cell[D6] \ccalc[50-15]
        \item \textbf{Multiplication} : \cell[E6] \ccalc[4*6]
        \item \textbf{Division} : \cell[F8] \ccalc[100/25]
    }
    \item Tu peux écrire des calcules avec le contenue d'autre cellule; essaie, ces opérations :
    \multiColItemize{2}{
        \item \cell[C8] \ccalc[E8*F9+E9]
        \item \cell[C9] \ccalc[F8/2]
    }
    \item Modifie, la valeur de E8. Observe ce qu'il se passe en C8.
\end{enumerate}

\section{Utiliser des fonctions}

Les tableurs proposes de nombreuses fonctions intégrées qui te permettent d'effectuer divers calculs sur des \cell[plages de cellules], c'est-à-dire l'ensemble des cellules comprises entre deux coordonnées données.

\begin{enumerate} 
        \item Teste les fonctions suivantes sur la plage \cell[D11:G12] :
        \begin{itemize}
            \item \textbf{NBVAL} : \cell[C15] \ccalc[NBVAL(D11:G12)] compte le nombre de cellules non vides dans la plage \cell[D11:G12].
            Le résultat sera affiché dans la cellule \cell[C15].
            \item \textbf{SOMME} : \cell[D15] \ccalc[SOMME(D11:G12)] calcule la somme des valeurs.
            \item \textbf{MOYENNE} : \cell[E15] \ccalc[MOYENNE(D11:G12)] calcule la moyenne des valeurs.
            \item \textbf{MAX} : \cell[F15] \ccalc[MAX(D11:G12)] retourne la valeur la plus élevée.
    \end{itemize}
\end{enumerate}

\section{Créer un graphique}

Les graphiques te permettent de représenter visuellement des données pour mieux les analyser ou les présenter.

\begin{enumerate}
    \item Sélectionne la plage \cell[C17:E18] en cliquant sur la première cellule \cell[C17] et en glissant jusqu'à la dernière \cell[E2].
    \item Accède à l'outil de création de graphiques :\\
    Barre de menus $\rightarrow$ \textbf{Insertion} $\rightarrow$ \icon{diagram} \textbf{Diagramme}.
    \item Suis les étapes de l'assistant de création de graphique :
    \begin{itemize}
        \item Choisis le type de graphique qui correspond à tes données (colonne, ligne, camembert, etc.).
        \item Configure les options nécessaires (titres, légendes, axes, etc.).
        \item Termine en cliquant sur \textbf{Terminer} pour insérer le graphique.
    \end{itemize}
    \item \textbf{Déplacer et redimensionner le graphique} :
    Une fois le graphique inséré, clique une fois à l'extérieur de celui-ci pour valider sa création. Puis, clique sur le graphique pour le sélectionner.
    \begin{itemize}
        \item Glisse le graphique à l'endroit souhaité (par exemple, en haut à droite pour ne pas gêner ta feuille de calcul).
    \item Ajuste sa taille en cliquant et en tirant sur ses poignées de redimensionnement.
    \end{itemize}
\end{enumerate}

\section{Étendre des cellules par glissement}

\Calc te permet de dupliquer ou d'étendre facilement des données ou des formules en utilisant la poignée d'extension. Voici comment procéder :

\begin{enumerate}
    \item \textbf{Dupliquer une valeur individuelle} :
    \begin{itemize}
        \item Clique sur la cellule \cell[C20].
        \item Positionne le curseur sur le coin inférieur droit de la cellule. Le curseur devient une petite croix noire (poignée d'extension).
        \item Maintiens le clic gauche de la souris et glisse vers \cell[C29].
        \item Relâche la souris : les cellules de \cell[C21] à \cell[C29] sont remplies avec la valeur de \cell[C20].
    \end{itemize}

    \item \textbf{Dupliquer plusieurs colonnes} :
    \begin{itemize}
        \item Selectionne la plage \cell[D20:E20]
        \item Positionne le curseur sur la poignée d'extension.
        \item Glisse jusqu'à \cell[E29].
    \end{itemize}

    \item \textbf{Étendre une plage de cellules} :
    \begin{itemize}
        \item Sélectionne la plage de cellules \cell[F20:F21].
        \item Glisse jusqu'à \cell[F29].
    \end{itemize}

    \item \textbf{Étendre une formule} :
    \begin{itemize}
        \item Saisis une formule en \cell[H20] pour calculer, la somme de \cell[I20] et \cell[J20].
        \item Place le curseur sur la poignée d'extension de \cell[H20] et glisse jusqu'à \cell[H29].
    \end{itemize}
\end{enumerate}

\section{Enregistrer et partager ton travail}

\begin{enumerate}
    \item Clique sur : Barre de menus $\rightarrow$ \textbf{Fichier} $\rightarrow$ \icon{save} \textbf{Enregistrer sous...}.
    \item Donne un nom explicite à ton fichier "classe-NOM-calc-prise-en-main.ods".
    \item Choisis un emplacement sur ton ordinateur.
    \item Clique sur \textbf{Enregistrer}.
\end{enumerate}

% \hint{Pour voir les plages de cellules (par exemple A1:A5), clique et glisse pour les sélectionner directement.}

% \section{Formater les cellules}

% \begin{enumerate}
%     \item Sélectionne une ou plusieurs cellules.
%     \item Clique droit et choisis \textbf{Formater les cellules...}.
%     \item Dans la fenêtre qui s'ouvre, explore les options :
%     \begin{itemize}
%     \item \textbf{Nombre} : pour définir le format (texte, numérique, monétaire, etc.).
%     \item \textbf{Police} : pour changer la taille ou le style d'écriture.
%     \item \textbf{Bordures} : pour ajouter des lignes autour des cellules.
%     \item \textbf{Arrière-plan} : pour colorer les cellules.
%     \end{itemize}
%     \item Applique tes modifications et observe les changements.
% \end{enumerate}