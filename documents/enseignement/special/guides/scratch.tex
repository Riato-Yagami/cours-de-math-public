% VARIABLES %%%
% \def\authors{\jules \ et \href{http://www.cellulegeometrie.eu/documents/pub/pub_14.pdf}{la Haute École en Hainaut}}
% \setGrade{6e}
\setTitle{Scratch - Prise en main}
% DOC https://ctan.math.illinois.edu/macros/latex/contrib/scratch3/scratch3-fr.pdf

\setscratch{scale=.75}

\def\block{{\setscratch{scale=.5}\begin{scratch}\blockmove{\Large bloc}\end{scratch} }}

\newcommand{\scr}[1]{\begin{scratch}#1\end{scratch}}

\definecolor{smotion}{HTML}{4C97FF} % #4C97FF
\definecolor{slooks}{HTML}{9966FF} % #9966FF
\definecolor{ssound}{HTML}{D65CD6} % #D65CD6
\definecolor{sevents}{HTML}{FFD500} % #FFD500
\definecolor{scontrol}{HTML}{FFAB19} % #FFAB19
\definecolor{ssensing}{HTML}{4CBFE6} % #4CBFE6
\definecolor{soperators}{HTML}{6DB26E} % #6DB26E
\definecolor{svariables}{HTML}{F28011} % #F28011
\definecolor{smyblocks}{HTML}{FF6680} % #FF6680

\def\smotion{\textcolor{smotion}{\faCircle\,Mouvement}} % Déplacement du lutin
\def\slooks{\textcolor{slooks}{\faCircle\,Apparence}} % Modifier l'apparence du lutin ou de la scène
\def\ssound{\textcolor{ssound}{\faCircle\,Son}} % Jouer des sons ou de la musique
\def\sevents{\textcolor{sevents}{\faCircle\,Événement}} % Déclencher des scripts en réponse à des actions
\def\scontrol{\textcolor{scontrol}{\faCircle\,Contrôle}} % Boucles, conditions, et contrôle du flux
\def\ssensing{\textcolor{ssensing}{\faCircle\,Capteur}} % Réagir à des informations extérieures ou internes
\def\soperators{\textcolor{soperators}{\faCircle\,Opérateur}} % Calculs mathématiques et logiques
\def\svariables{\textcolor{svariables}{\faCircle\,Variable}} % Stockage et manipulation de données
\def\smyblocks{\textcolor{smyblocks}{\faCircle\,Mes blocs}} % Création de blocs personnalisés

\def\spen{{\icon{scratch/pen} Stylo}}
\def\spenExtension{{\icon{scratch/pen-extension} Stylo}}
\def\sextensions{{\icon{scratch/extensions} $\lbrack$ Ajouter une extensions $\rbrack$}}
\def\sflag{{\icon{scratch/flag}%
%  Drapeau
}}

% \setscratch{scale=.75}
% \setscratch{print=true}
% \setscratch{fill blocks=true}
\colorlet{gradeColor}{scratch}
\emptyBackground
%%



% \hint{
%     \begin{itemize}
%         \item Appeler M. Pesin à la fin de chaque partie.
%         \item Enregistrer les productions avec le nom : "NOM.S-tp-polygones-partie-X.ggb"
%     \end{itemize}
% }

\Scratch est un logiciel de programmation par \block, il va t'aider à découvrire l'algorithmique.

\section{Découvrir l'interface}

\begin{enumerate}
    \item Ouvre le logiciel \Scratch.
    \item Observe l'interface :
    \begin{enumerate}
        \item A gauche il y a des \block
        organisés par catégories : \smouvement, \sapparance, \ssound, etc.
        \item Au centre, il y a la \key{zone de scripts}, où tu pourras assembler les \block pour créer des algorithmes.
        \item À droite, tu vois la \key{scène} , où se déplacera ton personnage (appelé \key{lutin} ou sprite).  
    \end{enumerate}
    \item Teste le fonctionnement de  \Scratch : 
    \begin{itemize} 
        \item Glisse un bloc de la catégorie \smouvement
        (par exemple :
        \begin{scratch}\blockmove{avancer de \ovalnum{10}}\end{scratch}
        ) dans la \key{zone de scripts}. 
        \item Clique dessus pour voir le lutin bouger. 
        \item Change la valeur dans le bloc
        (par exemple :
        \begin{scratch}\blockmove{avancer de \ovalnum{100}}\end{scratch}
        ) et clique à nouveau. 
    \end{itemize}
\end{enumerate}

\hint{Si tu fais une erreur, tu peux supprimer un \block en le glissant vers la liste des \block.}

\section{Imbriquer des blocs}

\begin{enumerate}
    \item Places les \block suivants en les connectant dans cet ordre :
        \begin{scratch}
            \blockmove{avancer de \ovalnum{40}}
            \blocklook{dit \ovalnum{Bonjour !}}
            \blockcontrol{attendre \ovalnum{1} secondes}
            \blocklook{dit \ovalnum{Au revoir !}}
            \blockmove{avancer de \ovalnum{60}}
        \end{scratch}
        \hint{La couleur des \block correspond à leur catégorie.}
    \item Clique sur n'importe quel \block placer pour executer ton programme.
\end{enumerate}


\section{Utiliser l'extension Crayon pour tracer des formes}

\begin{enumerate}
    \item Clique sur l'icône \sextensions{} en bas à gauche pour ajouter une \key{extension}.
    \item Sélectionne l'extension \spenExtension.
    De nouveaux blocs, comme
    \begin{scratch}\blockpen{stylo en position d'écriture}\end{scratch}
    et
    \begin{scratch}\blockpen{relever le stylo}\end{scratch}
    seront ajoutés dans la nouvelle catégorie \spen .
    \item Teste les blocs suivants :
    \begin{scratch}
        \blockpen{stylo en position d'écriture}
        \blockmove{avancer de \ovalnum{50}}
        \blockmove{tourner \turnright{} de \ovalnum{45} degrés}
        \blockmove{avancer de \ovalnum{50}}
    \end{scratch}
    % \item Observe ce qui se passe lorsque le crayon est baissé et que le lutin avance. Relève ensuite le crayon pour qu'il cesse de tracer.
\end{enumerate}

\hint{Pour mieux visualiser tes tracés :
tu peux ajuster la taille de ton lutin en utilisant l'option "Taille" située dans l'onglet "Sprite" sous la scène.
}

\section{Automatiser l'exécution}

\begin{enumerate}
    \item Glisse le bloc
    \begin{scratch}\blockinit{quand \greenflag est cliqué}\end{scratch}
    de la catégorie \sevent.
    % \begin{scratch}\blockevent{quand \flag est cliqué}\end{scratch}) dans la \key{zone de scripts}.
    \item Connecte ce bloc à une série d'actions, comme :
    % \begin{center}
        \begin{scratch}
            % \blockcustom{effacer tout}
            \blockmove{aller à x: \ovalnum{0} y: \ovalnum{0}}
            \blockpen{effacer tout}
            \blockpen{stylo en position d'écriture}
            \blockmove{avancer de \ovalnum{100}}
            \blockmove{tourner \turnright{} de \ovalnum{90} degrés}
            \blockmove{avancer de \ovalnum{100}}
        \end{scratch}
    % \end{center}
    \item Clique sur le \sflag{} pour exécuter l'algorithme.
\end{enumerate}