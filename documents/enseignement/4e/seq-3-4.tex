%%%
\setGrade{4e}
\evaluation{2}
%%%
% \def\imgPath{enseignement/6e/}

\seqEvaluation{3}{Nombres Relatifs}{
    Appliquer la règle des signes pour les produits et quotients de nombres relatifs.
    /2,
    Déterminer le signe d'un produit de plusieurs facteurs relatifs.
    /4,
    Trouver les antécédents du carré d'un nombre donné.
    /4,
    Déterminer la somme, differences de nombres relatifs.
    /0%
}

\seqEvaluation{4}{Théorème de pythagore - Contraposé et réciproque}{
    Utiliser et représenter les grands nombres entiers
    /3,
    Utiliser et représenter les nombres décimaux jusqu'à trois décimales.
    /4,
    Utiliser la division euclidienne.
    / 2,
    Résoudre des problèmes relevant des structures additives et multiplicatives en mobilisant une ou plusieurs étapes de raisonnement.
    / 2,
    Organiser un calcul en une seule ligne{,} utilisant si nécessaire des parenthèses.
    / 2%
}

\seqEvaluation{}{Compétences générales}{
    Écrire ses calculs
    /1,
    Rédiger des phrases réponses
    /2
}