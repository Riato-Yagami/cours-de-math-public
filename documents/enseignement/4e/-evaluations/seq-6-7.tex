\setGrade{4e}
\evaluation{3}
[corr]

\seqEvaluation{6}{Calcul Littéral}{
    Manipuler une expression littéral pour raisonner en géometrie.
    /3,
    Développer une expression littérale.
    / 2,
    Réduire une expression littérale.
    / 2,
    Factoriser une expression littérale.
    / 3%
}

\seqEvaluation{7}{Nombres rationnels}{
    Calculer des sommes et  difference de fractions.
    / 4,
    Calculer des produits de fractions.
    / 2,
    Calculer des quotients de fractions.
    / 1,
    Résoudre des problèmes metant en jeux des calculs de fractions
    / 3%
}

\evalutionEnd[1][1]

\noCalculator

\exo{Développer et réduire}{
    Développe et réduis les expressions suivantes :
    \begin{enumerate}
        \item $x(x+4) = \nswr[3]{x \times x + x \times 4 = x^2 + 4x}$
        \item $7(2-9y) = \nswr[3]{7 \times 2 - 7 \times 9y = 14 - 63y}$
        \item $-2y(5-y) = \nswr[3]{-2y \times 5 - (-2y) \times y = -10y + 2y^2}$
        \item $(9-3t) \times 4 = \nswr[3]{9 \times 4 - 3t \times 4 = 36 - 12t}$
    \end{enumerate}
}[\sesa{4}{2011}[15][84][ms]]

\newpage

\exo{Factoriser}{
    Factorise les expressions suivantes :
    \begin{enumerate}
        \item $16 \times \np{4.9} - 6 \times \np{4.9}
        = \nswr[3]{\np{4.9} \times (16 - 6)}$
        \item $3x - 9
        = \nswr[3]{3(x - 3)}$
        \item $15 + 45y
        = \nswr[3]{(1 + 3y)}$
        % \item $31z - 31
        % = \nswr[3]{31(z - 1)}$
    \end{enumerate}
}[\sesa{4}{2021}[3][48]]

\exo{\ttps{} Opérations de fractions}{
    Calcul le résultat sous forme de fractions simplifié.
    \begin{enumerate}
        \item $\dfrac{-9}{2} + \dfrac{7}{5}
        = \nswr[3]{\dfrac{-9 \times 5}{2 \times 5} + \dfrac{7 \times 2}{5 \times 2}
        = \dfrac{-45}{10} + \dfrac{14}{10}
        = \dfrac{-31}{10}}$
        % \item $3 + \dfrac{5}{3}
        % = \nswr[3]{\dfrac{9}{3} + \dfrac{5}{3}
        % = \dfrac{14}{3}}$
        \item $\dfrac{1}{8} - \dfrac{5}{4} + \dfrac{-7}{6}
        = \nswr[5]{\dfrac{3}{24} - \dfrac{30}{24} + \dfrac{-28}{24}
        = \dfrac{3 - 30 - 28}{24}
        = \dfrac{-55}{24}}$
        \item $\dfrac{-8}{3} \times \dfrac{3}{4} \times \dfrac{5}{7}
        = \nswr[4]{\dfrac{-8 \times 3 \times 5}{3 \times 4 \times 7}
        = \dfrac{-120}{84}
        = \dfrac{-10}{7}}$
        \item $\dfrac{16}{21} \times \dfrac{35}{32}
        = \nswr[5]{\dfrac{16 \times 35}{21 \times 32}
        = \dfrac{\cancel{16} \times 5 \times \cancel{7}}{3 \times \cancel{7} \times \cancel{16} \times 2}
        = \dfrac{5}{6}}$
        \item $2 - \dfrac{3}{4} \times \dfrac{16}{9}
        = \nswr[6]{2 - \dfrac{48}{36}
        = 2 - \dfrac{48}{36}
        = \dfrac{72}{36} - \dfrac{48}{36}
        = \dfrac{24}{36}
        = \dfrac{24}{36}
        = \dfrac{2}{3}
        }$
        \item $\dfrac{5}{7} \div \dfrac{15}{2}
        = \nswr[4]{\dfrac{5}{7} \times \dfrac{2}{15}
        = \dfrac{\cancel{5} \times 2}{7 \times 3 \cancel{5}}
        = \dfrac{2}{21}}$
        % \item $\dfrac{2}{\dfrac{3}{5}}
        % = \nswr[3]{2 \times \dfrac{5}{3}
        % = \dfrac{10}{3}}$
    \end{enumerate}
}[\sesa{4}{2021}[][14]]

\def\answerHeight{8}
\exo{Réussite au brevet}{
    Au collège de la Paix, $162$ élèves ont présenté le brevet en $2024$ et sept élèves sur neuf l'ont obtenu.
    En $2025$, M. Pesin prédit que, sur les $150$ élèves inscrits, quatre cinquièmes d'entre eux réussiront.
    \begin{enumerate}
        \item M. Pesin prédit-il un meilleur taux de réussite que l'année précédente ?
        \nswr[\answerHeight]{
            \begin{itemize}
                \item D'une part $\dfrac{7}{9} = \dfrac{7 \times 5}{9 \times 5} = \dfrac{35}{45}$.
                \item D'autre part $\dfrac{4}{5} = \dfrac{4 \times 9}{5 \times 9} = \dfrac{36}{45}$.
                \item Donc, $\dfrac{36}{45} > \dfrac{35}{45}$, M. Pesin prédit un meilleur taux de réussite.
            \end{itemize}
        }
        \item Si ses prédictions sont correctes, quelle année, entre $2024$ et $2025$, comptera le plus d'élèves ayant réussi le brevet ?
        \nswr[\answerHeight]{
            \begin{itemize}
                \item D'une part $162 \times \dfrac{7}{9} = 126$.
                \item D'autre part $150 \times \dfrac{4}{5} = 120$.
                \item Alors $162 \times \dfrac{7}{9} > 150 \times \frac{4}{5}$
                \item Ainsi en $2024$, il y aura plus d'élèves ayant réussi le brevet.
            \end{itemize}
        }
    \end{enumerate}
}
%[\href{https://www.letudiant.fr/college/annuaire-des-colleges/fiche/college-la-paix-92.html#success-rates}{L'Etudiant}]

\exo{Dimension manquante}{
    \begin{enumerate}
        \item Quelle est la longueur du côté d'un carré dont l'aire est $16x^2$ ?\\
        \nswr[\answerHeight]{
            \begin{itemize}
                \item L'aire $A$ d'un carré de côté $c$ est donnée par la formule : $A = c^2$.
                \item Ici, $A = 16x^2$.
                \item On a : $4x \times 4x = 4^2 \times x^2 = 16x^2$.
                \item Donc, la longueur du côté $c$ du carré est $4x$.
            \end{itemize}
        }
        \item Quelle est la longueur d'un rectangle dont la largeur est $3x$ et l'aire est $3x^2 + 6x$ ?\\
        \nswr[\answerHeight]{
            \begin{itemize}
                \item L'aire $A$ d'un rectangle de longueur $L$ et de largeur $l$ est donnée par la formule : $A = l \times L$.
                \item Ici, $A = 3x^2 + 6x$ et $l = 3x$.
                \item On a : $3x^2 + 6x = 3x \times L$.
                \item En factorisant par $3x$, on obtient : $3x^2 + 6x = 3x \times (x + 2)$.
                \item Donc, la longueur $L$ du rectangle est $x + 2$.
            \end{itemize}
        }
    \end{enumerate}
}

\newpage

\exo{Démonstration de la double distributivité}{
    Soient $a$, $b$, $c$ et $d$ des nombres.
    \begin{enumerate}
        \item Dessinez une figure dont l'aire est égale à $(a + b) \times (d + c)$.
        \ctikz{
    \boundingBox[15.7][12.01][0.5pt][1][(-13.92,2.53)]
    \nswr[0]{
        \fill[thick,fill opacity=0.10] (-13.12,2.93) -- (-4.22,3.06) -- (-4.26,6.06) -- (-13.16,5.93) -- cycle;
        \fill[thick,fill opacity=0.10] (-13.28,14.33) -- (-13.16,5.93) -- (-4.26,6.06) -- (-4.38,14.46) -- cycle;
        \fill[thick,fill opacity=0.10] (-4.38,14.46) -- (1.62,14.54) -- (1.74,6.14) -- (-4.26,6.06) -- cycle;
        \fill[thick,fill opacity=0.10] (-4.26,6.06) -- (-4.22,3.06) -- (1.78,3.14) -- (1.74,6.14) -- cycle;
        \draw[] (-7.84,4.65) node {$c \times a$};
        \draw[] (-8.55,2.53) node {a};
        \draw[] (-13.80,4.65) node {c};
        \draw[] (-7.97,10.52) node {$a \times d$};
        \draw[] (-13.92,10.48) node {d};
        \draw[] (-0.46,10.52) node {$d \times b$};
        \draw[] (-0.46,4.69) node {$c \times b$};
        \draw[] (-1.14,2.62) node {b};
        \draw [thick] (-13.12,2.93) -- (-4.22,3.06);
        \draw [thick] (-4.22,3.06) -- (-4.26,6.06);
        \draw [thick] (-4.26,6.06) -- (-13.16,5.93);
        \draw [thick] (-13.16,5.93) -- (-13.12,2.93);
        \draw [thick] (-13.28,14.33) -- (-13.16,5.93);
        \draw [thick] (-4.26,6.06) -- (-4.38,14.46);
        \draw [thick] (-4.38,14.46) -- (-13.28,14.33);
        \draw [thick] (-4.38,14.46) -- (1.62,14.54);
        \draw [thick] (1.62,14.54) -- (1.74,6.14);
        \draw [thick] (1.74,6.14) -- (-4.26,6.06);
        \draw [thick] (-4.22,3.06) -- (1.78,3.14);
        \draw [thick] (1.78,3.14) -- (1.74,6.14);
    }
}
        \item Utilisez cette figure pour exprimer $(a + b) \times (d + c)$ comme une somme de 4 termes.
        \item \nswr[\remaininglines]{
            $(a + b) \times (d + c) = a \times d + a \times c + b \times d + b \times c$
        }
    \end{enumerate}
}[][\cmdGeoGebra[knapz2xg]]
