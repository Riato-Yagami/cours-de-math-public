%%%
\setGrade{4e}
\evaluation{1}
%%%

\seqEvaluation{1}{Divisibilité et nombres premiers}{
    Utiliser les critères de divisibilité
    /3,
    Trouver les diviseurs d'un nombre
    /3,
    Résoudre un problèmes mettant en jeu des notions de divisibilité
    /2
}

\seqEvaluation{2}{Théorème de Pythagore - Sens direct}{
    Utiliser l'égalité de Pythagore
    /2,
    Utiliser le théorème de Pythagore pour calculer une longueur
    /4,
    Utiliser la racine carrée d'un nombre positif
    /1,
    Raisonner sur les aires
    /0%
}

\evalutionEnd

\calculator

% \def\cW{1cm}
% \exo{20p17}{
%     Mettre une croix dans le tableau suivant si le nombre en colonne est divisible par le nombre en ligne.
%     Justification non nécessaires.

%     \begin{tabular}{|C{4cm}|C{\cW}|C{\cW}|C{\cW}|C{\cW}|C{\cW}|C{\cW}|}
%         \hline
%         est divisible par & 2 & 3 & 4 & 5 & 9 & 10 \\\hline
%         \hline
%         360 & & & & & & \\\hline
%     \end{tabular}

% }[\mi]

\vspace{0.25cm}

\exo{Critères de divisibilité}{
    En justifiant à l'aide des critères de divisibilité,
    répondre aux questions suivantes :
    $\np{28 944}$ est divisible par $2$ ? par $3$? par $4$ ? par $5$ ? par $9$ ?
}

\answerFill

\exo{45p19}{
    \imgp{enseignement/4e/divisibilite-et-nombres-premiers/mi-c4/exo-45p19.png}[10cm]
    \vspace{0.5cm}
}[\mi]

\answerSec{11}

\exo{\tiersTemps Diviseurs communs}{
    Docteur Ivo Robotnik doit composer des escouades de robots animaux pour son armée.
    Il dispose pour cela de $246$ robots lapins et $164$ robots crabes.

    En précisant bien les différentes étapes de raisonnement ;
    donner toutes les compositions d'escouades possibles ?
}

\answerFill

\exo{Pythagore à trou}{
    Compléter les deux aires manquantes.
    Répondre sur l'énoncé et bien faire figurer vos calculs.

    \dividePage{
        \imgp{enseignement/4e/theoreme-de-pythagore/sens-direct/pythagore-a-trou-spe}[8cm]
    }{
        \answerSec{10}[Calculs]
    }
    
}

\exo{Théorème de Pythagore}{
    Soit $BFG$ un triangle rectangle en $G$.
    Avec $BG$ = $5\;\centi\meter$ et $BF$ = $12\;\centi\meter$.
    Donner la longueur $FG$ arrondi au millimètre prés en précisant bien les étapes de votre raisonnement.
}

\answerSec{0}[Schéma]

\vspace{5cm}

\answerFill

\exo{\bonus Raisonner avec les aires}{
    \imgp{enseignement/4e/theoreme-de-pythagore/sens-direct/pythagore-rectangle}[11cm]
    \begin{enumerate}
        \item Conjecturer une relation entre les aires des rectangles 1, 2 et 3.
        \item Démontrer cette conjecture.
    \end{enumerate}
}

\answerFill
