%%%
\setGrade{4e}
\evaluation{2}
% [corr]
%%%
% \def\imgPath{enseignement/6e/}

\seqEvaluation{3}{Nombres Relatifs}{
    Déterminer le produit et le quotient de nombres relatifs.
    /2,
    Déterminer la somme et la différence de nombres relatifs.
    /2,
    Déterminer le signe d'un produit comportant plusieurs facteurs relatifs.
    /1,
    Trouver les antécédents du carré d'un nombre donné.
    /1%
}

\seqEvaluation{4}{Théorème de pythagore - Contraposé et réciproque}{
    Montrer qu'un triangle n'est pas rectangle.
    /2,
    Montrer qu'un triangle est rectangle.
    /2,
    Déterminer si un triangle est rectangle.
    /1,
    Maîtriser les notions de contraposée et de réciproque. /0%
    }

\seqEvaluation{5}{Proportionnalité - Tableaux et graphiques}{
    Reconnaître un tableau de proportionnalité.
    /2,
    Compléter un graphique cartésien.
    /1,
    Reconnaître une situation de proportionnalité sur un graphique.
    /1%
}

\evalutionEnd[3][2]

% \newpage

% \exo{Calcul de 4e proportionnelle avec des nombres relatifs \tt}{
%     Trouver les valeurs manquantes des tableaux de proportionnalité ci-dessous :
%     \multiColEnumerate{3}{
%         \item \Propor[Simple,Math,Stretch=1.25,%
%         ]{-18/30,-6/\awsr{\np{10}}}
%         % \item \Propor[Simple,Math,Stretch=1.25,%
%         % ]{\awsr{\np{3}}/22,\np{1.5}/11}
%         \item \Propor[Simple,Math,Stretch=1.25,%
%         ]{\np{2.5}/-5,-60/\awsr{\np{120}},\awsr{\np{-360}}/720}
%     }
% }
    
% \answerFill[Calculs][%
%     \begin{enumerate}
%         \item On peut utiliser l'égalité des produits en croix pour calculer la quatrième proportionnelle :
%         $\frac{22\times\np{1.5}}{11} = \frac{33}{11} = 3$
%         % \item $\frac{30\times(-6)}{-18} = \frac{-180}{-18} = 10$
%         \item On a $\np{2,5} \times (-2) = -5$ alors $-2$ est le coefficient qui permet de passer de la première à la deuxième ligne.
%         On calcule alors : $-60 \times (-2) = 120$ et $720 \div (-2) = -360$
%     \end{enumerate}
% ]

\exo{Manipulation de relatifs}{
    \multiColEnumerate{1}{
        \item $[-3-(-7+5)] \times (-\np{0.5}) = \awsr[4]{
            [-3-2] \times (-\np{1.5})
            = -5 \times (-\np{1.5})
            = \np{7.5}
        }$
        \item Quel est le résultat d'un produit de $\np{4815162342}$ fracteurs égales à $-1$.\\
        \awsr[4]{
            $\np{4815162342}$ est pair donc le résultat est positif
            et multiplié $1$ par lui même donne $1$ donc le résultat est $1$.
        }
        \item Donner deux nombres dont le carré vaut $18$.
        \awsr[4]{
            $\sqrt{18}^2 = 18$ et $(-\sqrt{18})^2 = 18$.
            $18$ et $-18$ sont deux nombres dont le carré vaut $18$.
        }
        % \item $\frac{2-[5-3\times(2-4)]}{2-15\div5} = \awsr[3]{
        %     \frac{2-[5-3\times(-2)]}{2-3}
        %     = \frac{2-[5-6]}{-1}
        %     = \frac{2-[-1]}{-1}
        %     = \frac{3}{-1}
        %     = -3
        % }$
    }
}[\ching{4}{nombres-relatifs-operations}[$26$a et E.$35$a]]

\newpage

\exo{Déterminer si un triangle est rectangle}{
    Déterminer la nature de chacun des triangles ci-dessous.
    \ctikz[0.6]{
        \draw[gray!40] (-6,-4) rectangle (8,5);
        \draw [penciline, thick] (-3.2,3.86)-- (-0.84,-0.1);
        \draw [penciline,thick] (-0.84,-0.1)-- (-4.5,-1.86);
        \draw [penciline,thick] (-4.5,-1.86)-- (-3.2,3.86);
        \draw [penciline,thick] (-0.28,-2.28)-- (2.28,1.94);
        \draw [penciline,thick] (6.14,-2.72)-- (2.28,1.94);
        \draw [penciline,thick] (-0.28,-2.28)-- (6.14,-2.72);
        \draw (-4.86,1.04) node[anchor=north west] {6cm};
        \draw (-2.8,-1.22) node[anchor=north west] {5cm};
        \draw (-1.8,2.04) node[anchor=north west] {3cm};
        \draw (0.26,0.32) node[anchor=north west] {6dm};
        \draw (2.26,-2.78) node[anchor=north west] {8dm};
        \draw (4.46,0.14) node[anchor=north west] {10dm};
        \drawPoint{A}{-3.20}{3.86}
        \drawPoint{B}{-4.50}{-1.86}
        \drawPoint{C}{-0.84}{-0.10}
        \drawPoint{D}{-0.28}{-2.28}
        \drawPoint{E}{2.28}{1.94}
        \drawPoint{F}{6.14}{-2.72}
    }
}[\ching{4}{reciproque-pythagore}[$8$]]


\answerSec{14}[Triangle ABC][
    \Pythagore[Reciproque,Unite=cm]{ABC}{3}{5}{6}
]

\answerFill[Triangle EDF][
    \Pythagore[Reciproque,Unite=dm]{EDF}{10}{8}{6}
]

\exo{}{
    Chez Zoro, des tee-shirts sont en vente. Les prix normaux ainsi que les prix en période de soldes sont indiqués dans le tableau ci-dessous.
    \begin{enumerate}
        \begin{table}[h!]
            \centering
            \renewcommand{\arraystretch}{1.5} % Ajuste la hauteur des lignes
            \setlength{\tabcolsep}{8pt} % Ajuste l'espacement des colonnes
            \begin{tabular}{|>{\bfseries}c|*{7}{c|}} % Colonne en gras pour la première colonne
                \hline
                \rowcolor{gray!15} 
                Tee-shirts vendus & 1 & 2 & 3 & 4 & 5 & 6 & 7 \\ \hline
                Prix normal (en \euro) & 5 & 10 & 15 & 20 & 25 & 30 & 35 \\ \hline
                Prix soldé (en \euro)  & 5 & 10 & 12 & 17 & 22 & 24 & 29 \\ \hline
            \end{tabular}
        \end{table}
        \item Complétez le graphique cartésien ci-dessous en plaçant les points correspondant :
        \begin{itemize}
            \item en \textcolor{Blue}{bleu}, les points représentant les prix normaux ;
            \item en \textcolor{Red}{rouge}, les points représentant les prix en période de soldes.
        \end{itemize}
        \vspace{-1cm}
        \begin{center}
            \begin{tikzpicture}[yscale = 0.2, xscale = 1.5]
                \tkzInit[xmin=0,xmax=7.5,ymin=0,ymax=37]
                \tkzGrid[sub,color=gradeColor!50!white,subxstep=1,subystep=1]        
                \tkzLabelX[step=1]
                \tkzLabelY[step=5]
                \tkzDrawY[label={Prix (en \euro)}, above , step=5]
                \tkzDrawX[label={Tee-shirts vendus}, right, step=1]
                % Tracer les points
                \ifthenelse{\boolean{answer}}{
                    \foreach \x/\y/\z in {1/5/5, 2/10/10, 3/15/12, 4/20/17, 5/25/22, 6/30/24, 7/35/29}{
                        \drawPoint{}{\x}{\y}[Blue];
                        \drawPoint{}{\x}{\z}[Red];
                    }
                    \draw[Blue] (0,0) -- (7,35);
                    \draw[Red] (0,0) -- (2,10) -- (3,12) -- (4,17) -- (5,22) -- (6,24) -- (7,29);
                }{}
            \end{tikzpicture}
        \end{center}
        \item Les prix normaux et soldés sont-ils proportionnels au nombre de tee-shirts vendus ?
        Justifiez votre réponse à l'aide d'un argument graphique pour chaque cas.
    \end{enumerate}
}[\sesa{4}{2021}[5][61]]

\answerFill[Réponse][
    \begin{enumerate}\loadenumi[exo][1]
        \item \begin{itemize}
            \item Les points des prix des t-shirts non soldés sont alignés avec l'origine,
            il y a donc bien proportionnalité en période normale.
            \item Dans le cas des prix soldés,
            les points ne sont pas alignés,
            il n'y a donc pas proportionnalité.
        \end{itemize}
    \end{enumerate}
]

\exo{\tiersTemps Reconnaître un tableau de proportionnalité}{
    Les tableaux suivants présentent-ils des situations de proportionnalités ?
    \vspace{-1cm}
    \multiColEnumerate{2}{
        \item \Propor[Simple,Math,Stretch=1.25,%
        ]{-18/30,-6/10}
        % \item \Propor[Simple,Math,Stretch=1.25,%
        % ]{\awsr{\np{3}}/22,\np{1.5}/11}
        \item \Propor[Simple,Math,Stretch=1.25,%
        ]{\np{2.5}/-5,-60/120,-360/710}
    }
}

\answerSec{11}[Réponse][%
    \begin{enumerate}
        \item $-6 \times 30 = 180 = -18 \times 10$\\
        L'égalité des produits en croix étant vérifier,
        on a bien proportionnalité.
        \item $\frac{22\times\np{1.5}}{11} = \frac{33}{11} = 3$
        % \item $\frac{30\times(-6)}{-18} = \frac{-180}{-18} = 10$
        \item $\frac{-5}{\np{2.5}} = \frac{1}{2} = \frac{-360}{\np{720}} \neq \frac{-360}{\np{710}}$\\
        L'égalité des quotients n'étant pas vérifier, il n'y a pas proportionnalité. 
    \end{enumerate}
]

\exo{\bonus Contraposée et réciproque}{
    On s'intéresse à un quadrilatère.
    \begin{enumerate}  
        \item La proposition suivante est-elle vraie ?  
        «\Sialors{c'est un losange}{ses diagonales sont perpendiculaires.}»
        \item Écrivez la contraposée de cette proposition. Cette contraposée est-elle vraie ?
        \item Écrivez la réciproque de cette proposition. Cette réciproque est-elle vraie ? Justifiez votre réponse.  
    \end{enumerate}  
    
}

\answerFill[Réponse][%
    \begin{enumerate}
        \item La proposition est vraie.
        \item La contraposée de cette proposition est :
        «\Sialors{les diagonales ne sont pas perpendiculaires}{ce n'est pas un losange}».
        Cette contraposée est également vraie.
        \item La réciproque de cette proposition est :
        «\Sialors{les diagonales sont perpendiculaires}{c'est un losange}».
        Cette réciproque est fausse,
        car un cerf-volant est un quadrilatère dont les diagonales sont perpendiculaires sans être nécessairement un losange.
    \end{enumerate}
]