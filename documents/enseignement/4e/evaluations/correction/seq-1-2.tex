%%%
\setGrade{4e}
\setTitle{Evaluation 1 - Correction}
%%%

\seqEvaluation{1}{Divisibilité et nombres premiers}{
    Utiliser les critères de divisibilité
    /3,
    Trouver les diviseurs d'un nombre
    /3,
    Résoudre un problèmes mettant en jeu des notions de divisibilité
    /2
}

\seqEvaluation{2}{Théorème de Pythagore - Sens direct}{
    Utiliser l'égalité de Pythagore
    /2,
    Utiliser le théorème de Pythagore pour calculer une longueur
    /4,
    Utiliser la racine carrée d'un nombre positif
    /1,
    Raisonner sur les aires
    /0%
}

\evalutionEnd

\calculator

% \def\cW{1cm}
% \exo{20p17}{
%     Mettre une croix dans le tableau suivant si le nombre en colonne est divisible par le nombre en ligne.
%     Justification non nécessaires.

%     \begin{tabular}{|C{4cm}|C{\cW}|C{\cW}|C{\cW}|C{\cW}|C{\cW}|C{\cW}|}
%         \hline
%         est divisible par & 2 & 3 & 4 & 5 & 9 & 10 \\\hline
%         \hline
%         360 & & & & & & \\\hline
%     \end{tabular}

% }[\mi]

\vspace{0.25cm}

\exo{Critères de divisibilité}{
    En justifiant à l'aide des critères de divisibilité,
    répondre aux questions suivantes :
    $\num{28 944}$ est divisible par $2$ ? par $3$? par $4$ ? par $5$ ? par $9$ ?
}

\answerFill[Réponse][Le nombre $\num{28 944}$ : \begin{enumerate}
    \item est divisible par 2,
    car il termine par $4$ un nombre pair.
    \item est divisible par 3 et 9,
    car $2+8+9+4+4 = 27$
    et $2+7 = 9$
    qui est dans la table de $3$ et de $9$.
    \item est divisible par 4,
    car il termine par $44$ un et $44 = 4 \times 11$.
    \item n'est pas divisible par $5$,
    car il ne termine pas par $0$ ou $5$.
\end{enumerate}]

\exo{45p19}{
    \imgp{enseignement/4e/divisibilite-et-nombres-premiers/mi-c4/exo-45p19.png}[10cm]
    \vspace{0.5cm}
}[\mi]

\answerSec{11}[Réponse][
    \begin{enumerate}
        \item\begin{itemize}
            \item Pour trouver les diviseurs de $6$, on commence par le décomposer en facteurs premiers.
            \item $6 = 2 \times 3$
            \item Les diviseurs de $6$ sont donc $1;2;3$ et $6$.
            \item Or $1+2+3 = 6$.
            \item $6$ est donc bien un nombre parfait.
        \end{itemize}
        \item\begin{itemize}
            \item Pour trouver les diviseurs de $28$, on commence par le décomposer en facteurs premiers.
            \item $28 = 2 \times 2 \times 7$
            \item Les diviseurs de $6$ sont donc $1;2;7;2\times2 = 4;2\times7 = 14$ et $28$.
            \item Or $1+2+7+4+14 = 28$.
            \item $28$ est donc bien un nombre parfait.
        \end{itemize}
    \end{enumerate}
]

\exo{\tiersTemps Diviseurs communs}{
    Docteur Ivo Robotnik doit composer des escouades de robots animaux pour son armée.
    Il dispose pour cela de $246$ robots lapins et $164$ robots crabes.

    En précisant bien les différentes étapes de raisonnement ;
    donner toutes les compositions d'escouades possibles ?
}

\answerFill[Réponse][
    \begin{itemize}
        \item $246 = 2 \times 123 = 2 \times 3 \times 41$
        \item $164 = 2 \times 82 = 2 \times 2 \times 41$
        \item Les facteurs premiers communs de $246$ et $164$ sont $2$ et $41$.
        \item Les diviseurs commun de $246$ et $164$ sont donc $1;2;41$ et $82$.
        \item Il existe 4 compositions possible:
        \begin{enumerate}
            \item 1 escouade de $246$ robots lapins et $164$ robots crabes.
            \item 2 escouade de $123$ robots lapins et $82$ robots crabes.
            \item 41 escouade de $2 \times 3 = 6$ robots lapins et $2 \times 2 = 4$ robots crabes.
            \item 82 escouade de $3$ robots lapins et $2$ robots crabes.
        \end{enumerate}
    \end{itemize}
]

\exo{Pythagore à trou}{
    Compléter les deux aires manquantes.
    Répondre sur l'énoncé et bien faire figurer vos calculs.

    \dividePage{
        \imgp{enseignement/4e/theoreme-de-pythagore/sens-direct/pythagore-a-trou-spe-corr}[8cm]
    }{
        \answerSec{10}[Calculs][\begin{itemize}
            \item $10+12,5=22,5$
            \item $22,5-8,02 = 14,48$
        \end{itemize}]
    }
    
}

\exo{Théorème de Pythagore}{
    Soit $BFG$ un triangle rectangle en $G$.
    Avec $BG$ = $5\;\centi\meter$ et $BF$ = $12\;\centi\meter$.
    Donner la longueur $FG$ arrondi au millimètre prés en précisant bien les étapes de votre raisonnement.
}

\answerSec{0}[Schéma]

\imgp{enseignement/4e/theoreme-de-pythagore/sens-direct/shema-pythagore-eval}[3cm]

\answerFill[Réponse][\begin{itemize}
    \item On sait que le triangle GBF est rectangle en G.
    \item Alors d'après le théorème de Pythagore :
    \begin{align*} BF^2 &= BG^2 + FG^2\\
        \alors 12^2 &= 5^2 + FG^2\\
        \alors FG^2 &= 12^2 - 5^2 = 144 - 25 = 119\\
        \alors FG &= \sqrt{119} \approx 10,9\\
        \alors FG &= 10,9\;\centi\meter
    \end{align*}
\end{itemize}]

\exo{\bonus Raisonner avec les aires}{
    \imgp{enseignement/4e/theoreme-de-pythagore/sens-direct/pythagore-rectangle}[11cm]
    \begin{enumerate}
        \item Conjecturer une relation entre les aires des rectangles 1, 2 et 3.
        \item Démontrer cette conjecture.
    \end{enumerate}
}

\answerFill[Réponse][\begin{enumerate}
    \item On conjecture que l'aire du rectangle 3
    est égale à la somme des rectangles 1 et 2.
    \item \begin{itemize}
        \item Les rectangles 3, 1 et 2
        on comme largeur respectivement
        la longueur de l'hypothénus du triangle $ABC$
        et des longueurs de ces deux autres cotés.
        \item Et ils ont comme longueur respectivement
        deux fois la longueur l'hypothénus du triangle $ABC$
        et deux fois des longueurs de ces deux autres cotés.
        \item Alors leurs aires sont respectivement les doubles des aires des carrés construit sur l'hypothénus et sur ces deux autres cotés.
        \item Hors d'après le théorème de Pythagore l'aire du carré construit sur l'hypothénus est égale à la somme des aires des carrés construit sur les deux autres cotés.
        \item On peut en conclure notre conjecture en multipliant les deux memebres de notre égalité par deux.
    \end{itemize}
\end{enumerate}]
