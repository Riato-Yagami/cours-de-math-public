\dm{Escargot}[corr]
\setGrade{4e}

% https://ww2.ac-poitiers.fr/math/sites/math/IMG/pdf/dm_escrgot.pdf
% \newcommand{\trgl}[1]{}
\NewDocumentCommand{\trgl}{O{}}{\textcolor{gradeColor}{triangle\ifNotNull{#1}{\cir[gradeColor]{#1}}}}
\newcommand{\trgls}{\textcolor{gradeColor}{triangles}}

\begin{enumerate}
    \item Placer un point $S$ au centre d'une feuille blanche $A4$.
    \item Construisez le \trgl[1] $SAB$, rectangle en $A$, avec $AB = AS = \Lg{1}$.
    \item Donnez, en justifiant, la mesure exacte (laisser le nombre sous forme de raçine) de l'hypoténuse du \trgl[1].
    \item Tracez le \trgl[2] $SBC$, rectangle en $B$, avec $BC = \Lg{1}$ (les \trgls{} \cir[gradeColor]{1} et \cir[gradeColor]{2} ont un côté en commun et ne s'intersectent pas).
    \item Donnez, en justifiant, la mesure exacte de l'hypoténuse du \trgl[2].
    \item Tracez le \trgl[3] $SCD$, rectangle en $C$, avec $CD = \Lg{1}$ (les \trgls{} \cir[gradeColor]{2} et \cir[gradeColor]{3} ont un côté en commun et ne s'intersectent pas).
    \item Construisez de la même manière le \trgl[4]. Sans justifier, quelle est la mesure exacte de son hypoténuse ?
    \item Quelle sera la mesure de l'hypoténuse du \trgl[6] ? Et celle du \trgl[35] ?
    \item Construisez des \trgls{} jusqu'à obtenir un \trgl{} dont l'hypoténuse mesure $\sqrt{15}\;\Lg{}$. De quel \trgl{} s'agit-il ?
    \item Pour $n$ un nombre, quelle sera la mesure de l'hypoténuse du \trgl[n] ?
\end{enumerate}

\nswr[0]{
    \ctikz[2.4]{
    \boundingBox[5.63][5.58][0.5pt][0.5][(0,0)]
    \draw[thick,color=gradeColor,fill=gradeColor,fill opacity=0.10] (3.53,1.67) -- (3.56,1.84) -- (3.38,1.87) -- (3.36,1.70) -- cycle;
    \draw[thick,color=gradeColor,fill=gradeColor,fill opacity=0.10] (4.48,1.65) -- (4.38,1.79) -- (4.24,1.69) -- (4.34,1.54) -- cycle;
    \draw[thick,color=gradeColor,fill=gradeColor,fill opacity=0.10] (5.21,2.30) -- (5.04,2.36) -- (4.98,2.19) -- (5.15,2.14) -- cycle;
    \draw[thick,color=gradeColor,fill=gradeColor,fill opacity=0.10] (5.43,3.26) -- (5.26,3.22) -- (5.30,3.05) -- (5.47,3.08) -- cycle;
    \draw[thick,color=gradeColor,fill=gradeColor,fill opacity=0.10] (5.16,4.20) -- (5.02,4.09) -- (5.13,3.96) -- (5.27,4.06) -- cycle;
    \draw[thick,color=gradeColor,fill=gradeColor,fill opacity=0.10] (4.50,4.93) -- (4.41,4.78) -- (4.57,4.70) -- (4.65,4.85) -- cycle;
    \draw[thick,color=gradeColor,fill=gradeColor,fill opacity=0.10] (3.59,5.33) -- (3.58,5.16) -- (3.75,5.14) -- (3.77,5.32) -- cycle;
    \draw[thick,color=gradeColor,fill=gradeColor,fill opacity=0.10] (2.60,5.37) -- (2.65,5.20) -- (2.82,5.25) -- (2.77,5.42) -- cycle;
    \draw[thick,color=gradeColor,fill=gradeColor,fill opacity=0.10] (1.66,5.06) -- (1.76,4.91) -- (1.91,5.01) -- (1.81,5.16) -- cycle;
    \draw[thick,color=gradeColor,fill=gradeColor,fill opacity=0.10] (0.88,4.45) -- (1.02,4.34) -- (1.12,4.48) -- (0.98,4.59) -- cycle;
    \draw[thick,color=gradeColor,fill=gradeColor,fill opacity=0.10] (0.32,3.63) -- (0.49,3.57) -- (0.55,3.73) -- (0.38,3.79) -- cycle;
    \draw[thick,color=gradeColor,fill=gradeColor,fill opacity=0.10] (0.04,2.67) -- (0.21,2.66) -- (0.22,2.84) -- (0.05,2.85) -- cycle;
    \draw[thick,color=gradeColor,fill=gradeColor,fill opacity=0.10] (0.04,1.68) -- (0.21,1.72) -- (0.17,1.89) -- (0.00,1.85) -- cycle;
    \draw[thick,color=gradeColor,fill=gradeColor,fill opacity=0.10] (0.32,0.72) -- (0.47,0.81) -- (0.39,0.96) -- (0.23,0.88) -- cycle;
    \node at (3.64, 2.11) {\cir[gradeColor]{1}};
    \node at (4.40, 2.13) {\cir[gradeColor]{2}};
    \node at (4.86, 2.68) {\cir[gradeColor]{3}};
    \node at (4.92, 3.48) {\cir[gradeColor]{4}};
    \node at (4.68, 4.28) {\cir[gradeColor]{5}};
    \node at (4.07, 4.80) {\cir[gradeColor]{6}};
    \node at (3.27, 5.04) {\cir[gradeColor]{7}};
    \node at (2.40, 5.00) {\cir[gradeColor]{8}};
    \node at (1.64, 4.61) {\cir[gradeColor]{9}};
    \node at (0.97, 4.04) {\cir[gradeColor]{10}};
    \node at (0.52, 3.28) {\cir[gradeColor]{11}};
    \node at (0.39, 2.44) {\cir[gradeColor]{12}};
    \node at (0.48, 1.60) {\cir[gradeColor]{13}};
    \node at (0.78, 0.80) {\cir[gradeColor]{14}};
    \drawPoint{A}{3.36}{1.70}
    \drawPoint{B}{4.34}{1.54}
    \drawPoint{S}{3.51}{2.68}
    \draw[color=gradeColor] (4.23,2.25) node {$\sqrt{2}$};
    \drawPoint{C}{5.15}{2.14}
    \draw[color=gradeColor] (4.61,2.61) node {$\sqrt{3}$};
    \drawPoint{D}{5.47}{3.08}
    \draw[color=gradeColor] (4.69,3.15) node {$\sqrt{4}$};
    \drawPoint{E}{5.27}{4.06}
    \draw[color=gradeColor] (4.53,3.65) node {$\sqrt{5}$};
    \drawPoint{F}{4.65}{4.85}
    \draw[color=gradeColor] (4.09,4.03) node {$\sqrt{6}$};
    \drawPoint{G}{3.77}{5.32}
    \draw[color=gradeColor] (3.64,4.16) node {$\sqrt{7}$};
    \drawPoint{H}{2.77}{5.42}
    \draw[color=gradeColor] (3.10,4.10) node {$\sqrt{8}$};
    \drawPoint{I}{1.81}{5.16}
    \draw[color=gradeColor] (2.62,3.87) node {$\sqrt{9}$};
    \drawPoint{J}{0.98}{4.59}
    \draw[color=gradeColor] (2.21,3.56) node {$\sqrt{10}$};
    \drawPoint{K}{0.38}{3.79}
    \draw[color=gradeColor] (1.97,3.16) node {$\sqrt{11}$};
    \drawPoint{L}{0.05}{2.85}
    \draw[color=gradeColor] (1.85,2.67) node {$\sqrt{12}$};
    \drawPoint{M}{0.00}{1.85}
    \draw[color=gradeColor] (1.88,2.16) node {$\sqrt{13}$};
    \drawPoint{N}{0.23}{0.88}
    \draw[color=gradeColor] (2.02,1.68) node {$\sqrt{14}$};
    \drawPoint{O}{0.72}{0.00}
    \draw[color=gradeColor] (2.34,1.24) node {$\sqrt{15}$};
    \draw [thick] (3.36,1.70) -- (4.34,1.54) -- (3.51,2.68) -- (3.36,1.70);
    \draw [thick] (5.15,2.14) -- (3.51,2.68) -- (5.47,3.08) -- (5.15,2.14) -- (4.34,1.54);
    \draw [thick] (3.51,2.68) -- (5.27,4.06) -- (5.47,3.08);
    \draw [thick] (4.65,4.85) -- (3.51,2.68) -- (3.77,5.32) -- (4.65,4.85) -- (5.27,4.06);
    \draw [thick] (2.77,5.42) -- (3.51,2.68) -- (1.81,5.16) -- (2.77,5.42) -- (3.77,5.32);
    \draw [thick] (0.98,4.59) -- (3.51,2.68) -- (0.38,3.79) -- (0.98,4.59) -- (1.81,5.16);
    \draw [thick] (3.51,2.68) -- (0.05,2.85) -- (0.38,3.79);
    \draw [thick] (0.00,1.85) -- (3.51,2.68) -- (0.23,0.88) -- (0.00,1.85) -- (0.05,2.85);
    \draw [thick] (3.51,2.68) -- (0.72,0.00) -- (0.23,0.88);
}
    \begin{enumerate}
        \item \begin{itemize}
            \item Considérons le triangle $SAB$, rectangle en $A$.
            \item D'après le théorème de Pythagore :
            \[ SB^2 = SA^2 + AB^2 = 1^2 + 1^2 = 1 + 1 = 2. \]
            \item Alors l'hypoténuse du \trgl[1] est $SB = \sqrt{2}\,\Lg{}$.
        \end{itemize}
        \item \begin{itemize}
            \item On a $SBC$ un triangle rectangle en $C$.
            \item Alors d'après le théorème de pythagore :
            \[ SC^2 = SB^2 + BC^2 = (\sqrt{2})^2 + 1^2 = 2 + 1 = 3. \]
            \item Alors l'hypoténuse du \trgl[2] est $SC = \sqrt{3}\,\Lg{}$.
        \end{itemize}
        \item L'hypoténuse du \trgl[4] est $SC = \sqrt{5}\,\Lg{}$.
        \item Le \trgl[] ayant un hypoténuse de mesure $\sqrt{15}\,\Lg{}$ est le \trgl[14].
        \item Pour $n$ un nombre, le \trgl[n] aura un hypoténuse de mesure $\sqrt{n+1}\,\Lg{}$
    \end{enumerate}
}
