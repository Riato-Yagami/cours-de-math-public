\setGrade{4e}
\tp{Équations}
[corr]
\def\iconPath{libreOffice/calc/}

\definecolor{cell}{HTML}{CCCCCC} % #CCCCCC

\NewDocumentCommand{\cell}{O{cellule}}{%
    \renewcommand\fbox{\fcolorbox{cell}{white}}%
    \fbox{\ttfamily\bfseries#1}%
}

\NewDocumentCommand{\ccalc}{O{cellule}}{%
    {\ttfamily\bfseries=#1}%
}
%%%

\hint{Enregistrez votre travail sous le nom : \file{TP-equations-[classe]-[prenom]-[NOM].odt}}

L'objectif de ce TP est d'utiliser un logiciel de type tableur (ex : \Calc) pour résoudre des \key{équations}, c'est-à-dire trouver une valeur qui rend une égalité vraie.

\section{Résoudre des équations}

\exo{}{
    Nous allons déterminer la valeur de $x$ pour laquelle l'égalité
    $2x+3 = 4x-9$ est vérifiée.
    
    Pour cela :
    \begin{enumerate}
        \item Ouvrez le fichier \file{TP-4e-equation.ods} avec un logiciel de tableur.
        \item \begin{enumerate}
            \item Dans la cellule \cell[B2], entrez une formule pour calculer la valeur de $2x+3$ en fonction de la valeur de $x$ entrée en \cell[A2].
            \item Dans la cellule \cell[C2], entrez une formule pour calculer la valeur de $4x-9$ en fonction de la valeur de $x$ entrée en \cell[A2]. 
        \end{enumerate}
        \item Étendez les formules des cellules \cell[B2:C2] jusqu'à \cell[B12:C12].
        \item Pour quelle valeur de $x$ obtient-on l'égalité $2x+3 = 4x-9$ ? \nswr[0]{L'égalité $2x+3 = 4x-9$ est vraie pour $x=6$}
    \end{enumerate}
}[\href{http://mathematiques.ac-bordeaux.fr/college2010/ressources/progressions/Prog_4_fichiers_en_ligne/4_equation_tableur.pdf}{Académie de Bordeaux}]

\exo{}{
    À l'aide de la feuille : \textbf{Exercice 2} (sélectionnable en bas de l'interface),
    trouvez une solution pour les équations suivantes :
    \multiColEnumerate{2} {
        \item $10x = 5x+90$ \nswr[0]{vraie pour $x=18$}
        \item $2x^2 = 16x$ \nswr[0]{vraie pour $x=8$}
    }
}

\exo{}{
    À l'aide de la feuille : \textbf{Exercice 3}, résolvez les équations suivantes :
    \multiColEnumerate{2} {
        \item $3x+10 = 5x+3$ \nswr[0]{vraie pour $x=\np{3.5}$}
        \item $7x+\np{9,15} = 6x+\np{15,52}$ \nswr[0]{vraie pour $x=\np{6.37}$}
    }
    \hint{Les valeurs solution de $x$ ne sont pas forcément entières}
}

\section{Résoudre un problème à l'aide d'une équation}

\exo{}{
    Marcus cherche à compléter sa collection de mangas.
    Il compare les offres de deux revendeurs : Nil et Mississippi.
    \begin{itemize}
        \item \textbf{Offre Nil} propose un prix de 7,50€ par manga avec des frais de livraison de 15€.
        \item \textbf{Offre Mississippi} propose un prix de 8€ par manga avec des frais de livraison de 2€.
    \end{itemize}
    À l'aide de la feuille : \textbf{Exercice 4}, trouvez à partir de combien de mangas achetés l'offre Nil devient plus avantageuse que l'offre Mississippi.
    \nswr[0]{
        \begin{itemize}
            \item Pour $x$ le nombre de mangas achetés, le coût total chez Nil est de $\np{7.50}x + 15$€ et chez Mississippi de $8x + 2$€.
            \item Nous cherchons la valeur de $x$ pour laquelle $\np{7.50}x + 15 = 8x + 2$.
            \item En utilisant le tableur, nous trouvons que cette valeur est $x = 26$.
            \item Ainsi, il faut acheter au moins 26 mangas pour que l'offre de Nil soit plus avantageuse que celle de Mississippi.
        \end{itemize}
    }
}