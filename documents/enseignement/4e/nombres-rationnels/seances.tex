% VARIABLES %%%
\setSeq{6}{Nombres rationnels}
\setGrade{4e}

\dym{https://www.maths-et-tiques.fr/telech/19Fractions1.pdf}

\obj{
    \item Definition d'un nombre rationnel.
    \item Calculer des sommes, difference, produit et quotient de fractions.
    \item Utiliser la notion d'inverse.
}

\scn{Somme et différence de fraction}

\slide{qf}{
    \exo{Fraction quotient}{
    \noCalculator\\
    Ecrires les nombres suivant sous forme décimale :
    \multiColEnumerate{2}{
        \item $\frac{5}{2} = \nswr{\np{2.5}}$
        \item $\frac{9 \times 6}{3} = \nswr{27}$
        \item $\frac{4 + 6}{6-11} = \nswr{-2}$
        \item $\frac{6 \times (-6)}{-100 \div 10} = \nswr{\np{3.6}}$
    }
}

}

\slide{cr}{
    \sseq
    \section{Définition}
    \df{}{
    On appelle \key{nombre rationnel},
    un nombre qui peut s'exprimer comme le quotient de deux entiers relatifs.
}[\wiki{Nombre_rationnel}]
}

\slide{cr}{
    \section{Fraction égale}
}

\slide{cr}{
    \section{Somme et différence}
    \subsection{De même dénominateur}
    \mthd{Somme ou différence de fractions avec le \key{même dénominateur}}{
    \begin{enumerate}
        \item On prend comme numérateur : la somme ou différence des numérateurs.
        \item On conserve le dénominateur commun.
    \end{enumerate}
}
    Pour $a,b,c,d$ et $n$ des nombres.
    \expl{}{
    \multiColEnumerate{2}{
        \item $\frac{1}{2} + \frac{5}{2}
        = \nswr{\frac{1+5}{2} = \frac{6}{2}}$
        \item $\frac{14}{21} - \frac{12}{21}
        = \nswr{\frac{14-12}{21} = \frac{2}{21}}$
        \item $\frac{a}{d} + \frac{b}{d}
        = \nswr{\frac{a+b}{d}}$
        \item $\frac{a}{d} - \frac{b}{d}
        = \nswr{\frac{a-b}{d}}$
    }
}
}

\slide{cr}{
    \subsection{De dénominateur différents}
    \mthd{Somme et différence de fractions de \key{dénominateurs différents}}{
    \begin{enumerate}
        \item On met les fractions au même dénominateur.
        \item On fait leur somme ou différence comme précedement.
    \end{enumerate}
}
    \expl{}{
    \multiColEnumerate{1}{
        \item $\frac{1}{2} - \frac{1}{4}
        = \nswr{\frac{2}{4} - \frac{1}{4}
        = \frac{2-1}{4} = \frac{1}{4}
        }$
        \item $\frac{2}{12} + \frac{2}{3}
        = \nswr{\frac{2}{12} + \frac{2\times4}{3\times4}
        = \frac{2}{12} + \frac{8}{12}
        = \frac{2+8}{12}
        = \frac{10}{12}
        }$
        \item $\frac{3}{5} + \frac{6}{7}
        = \nswr{\frac{3\times7}{5\times7} + \frac{6\times5}{7\times5}
        = \frac{21}{35} + \frac{30}{35}
        = \frac{21+30}{35} = \frac{51}{35}
        }$
        \item $\frac{12}{3} - \frac{6}{2}
        = \nswr{\frac{12\times2}{3\times2} - \frac{6\times3}{2\times3}
        = \frac{24}{6} - \frac{18}{6}
        = \frac{24-18}{6} = \frac{6}{6}
        }$
        \item $\frac{a}{b} + \frac{c}{d}
        = \nswr{\frac{a\times d}{b\times d} + \frac{c \times b}{d\times b}
        = \frac{ad}{bd} + \frac{cb}{bd}
        = \frac{ad+cb}{bd}
        }$
    }
}
}

\slide{cr}{
    \section{Produit et quotient}
    \subsection{Produit}
    \mthd{Produits de fractions}{
    \begin{enumerate}
        \item On prend comme numérateur : le produit des numérateurs.
        \item On prend comme dénominateur : le produit des dénominateurs.
    \end{enumerate}
}[\cmdGeogebra[azrs82xa]]
}
\slide{cr}{
    \subsection{Inverse}
    \df{}{
    On appelle \key{inverse} d'un nombre $x$, le nombre qui, multiplié par $x$, donne $1$.
}[\wiki{Inverse}]
    \expl{}{
    \multiColEnumerate{1}{
        \item $0.5 \times \nswr{2} = 1$. L'inverse de $0.5$ est donc $\nswr{2}$.
        \item $10 \times \nswr{0.1} = 1$. L'inverse de $10$ est donc $\nswr{0.1}$.
        \item $\dfrac{1}{16} \times \nswr{16} = 1$. L'inverse de $\dfrac{1}{9}$ est donc $\nswr{9}$.
        \item $23 \times \nswr{\dfrac{1}{23}} = 1$. L'inverse de $23$ est donc $\nswr{\dfrac{1}{23}}$.
        \item $\dfrac{4}{5} \times \nswr{\dfrac{5}{4}} = 1$. L'inverse de $\dfrac{4}{5}$ est donc $\nswr{\dfrac{5}{4}}$.
    }
}
    \rmk{}{
    \begin{enumerate}
        \item L'inverse d'un nombre $x$ est $\nswr{\frac{1}{x}}$.
        \item L'inverse d'un rationnel $\frac{n}{d}$, le rationnel $\nswr{\frac{d}{n}}$.
    \end{enumerate}
}
}

\slide{exo}{
    \exo{QCM sur les fractions}{
    Dans chacune des ci-dessous, une seule réponse est correcte. Laquelle ?
    \begin{enumerate}
        \item L'inverse de $\frac{2}{7}$ est :
        \multiColEnumerate{3}{
            \item supérieur à $7$
            \item égale à $\np{3.5}$
            \item inférieur à $2$
        }
        \item $\frac{1}{15}$ est égale à :
        \multiColEnumerate{3}{
            \item $\np{0.0666666667}$
            \item $\frac{2}{5} \div \frac{1}{6}$
            \item $\frac{2}{30}$
        }
        \item Dans un ruisseau, il s'écoule, en moyenne, 120 $\Vol{}$ d'eau en 45 $\minute$.
        Le débit de ce ruisseau, en $\Vol{}/\hour$, est égal à :
        \multiColEnumerate{3}{
            \item $120 \div \frac{3}{4}$
            \item $120 \times \frac{3}{4}$
            \item $120 \times \np{0.75}$
        }
        La fraction $\frac{143}{132}$ est :
        \multiColEnumerate{3}{
            \item irréductible
            \item comprise entre 1 et \np{1.1}
            \item décimale
        }
    \end{enumerate}
}[\href{https://cache.media.education.gouv.fr/file/Fractions/23/2/RA16_C4_MATH_fractions_flash3_sens_quotient_554232.pdf}
{Utiliser les nombres pour comparer, calculer et résoudre des problèmes : les fractions}]
}

\slide{cr}{
    \subsection{Quotient}
    \input{resources/enseignement/4e/nombres-rationnels/mthd-quotion-de-fractions.tex}
}

\slide{exo}{
    \exo{QCM sur les fractions}{
    Dans chacune des ci-dessous, une seule réponse est correcte. Laquelle ?
    \begin{enumerate}
        \item L'inverse de $\frac{2}{7}$ est :
        \multiColEnumerate{3}{
            \item supérieur à $7$
            \item égale à $\np{3.5}$
            \item inférieur à $2$
        }
        \item $\frac{1}{15}$ est égale à :
        \multiColEnumerate{3}{
            \item $\np{0.0666666667}$
            \item $\frac{2}{5} \div \frac{1}{6}$
            \item $\frac{2}{30}$
        }
        \item Dans un ruisseau, il s'écoule, en moyenne, 120 $\Vol{}$ d'eau en 45 $\minute$.
        Le débit de ce ruisseau, en $\Vol{}/\hour$, est égal à :
        \multiColEnumerate{3}{
            \item $120 \div \frac{3}{4}$
            \item $120 \times \frac{3}{4}$
            \item $120 \times \np{0.75}$
        }
        La fraction $\frac{143}{132}$ est :
        \multiColEnumerate{3}{
            \item irréductible
            \item comprise entre 1 et \np{1.1}
            \item décimale
        }
    \end{enumerate}
}[\href{https://cache.media.education.gouv.fr/file/Fractions/23/2/RA16_C4_MATH_fractions_flash3_sens_quotient_554232.pdf}
{Utiliser les nombres pour comparer, calculer et résoudre des problèmes : les fractions}]
}