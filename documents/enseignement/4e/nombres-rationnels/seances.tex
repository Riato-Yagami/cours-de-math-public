% VARIABLES %%%
\setSeq{7}{Nombres rationnels}
\setGrade{4e}

\def\imgPath{enseignement/4e/nombres-rationnels/}

\dym{https://www.maths-et-tiques.fr/telech/19Fractions1.pdf}
\def\caPrefix{4e-mars-2022-}

\forStudents
% \forPrint

\obj{
    \item Definition d'un nombre rationnel.
    \item Calculer des sommes, difference, produit et quotient de fractions.
    \item Utiliser la simplification pour calculer des produits de fractions de manière efficace.
    \item Utiliser la notion d'inverse.
}

\scn{Rappel sur les definitions d'ensembles de nombres}

\slide{qf}{
    \exo{Fraction quotient}{
    \noCalculator\\
    Ecrires les nombres suivant sous forme décimale :
    \multiColEnumerate{2}{
        \item $\frac{5}{2} = \nswr{\np{2.5}}$
        \item $\frac{9 \times 6}{3} = \nswr{27}$
        \item $\frac{4 + 6}{6-11} = \nswr{-2}$
        \item $\frac{6 \times (-6)}{-100 \div 10} = \nswr{\np{3.6}}$
    }
}

}

\slide{exo}{
    \act{Nature des nombres}{
    En maternelle, nous avons appris à compter des objets en utilisant les nombres $1, 2, 3$, etc.
    Ces nombres sont les premiers utilisés « naturellement »,
    et on les appelle les nombres entiers naturels.
    Depuis l'école primaire et au collège, nous avons découvert d'autres nombres.
    Voici une liste de nombres :
    \multiColItemize{3}{
        \item $-\np{27.2}$
        \item $-\sqrt{4}$
        \item $\frac{10371}{100}$
        \item $\frac{27}{13}$
        \item $\frac{3}{2}$
        \item $\frac{-21}{15}$
        \item $\np{0.33333}\ldots$
        \item $\pi$
        \item $\frac{-10}{5}$
        \item $\sqrt{2}$
        \item $\frac{47}{21}$
        \item $-15 + 20$
        \item $\frac{-10}{3}$
        \item $37$
        \item $1 \div 7$
    }
    \begin{enumerate}
        \item Indique par une pastille :
        \begin{itemize}
            \item \textcolor{Blue}{bleue} les nombres entiers naturels.
            \item \textcolor{Red}{rouge} les nombres entiers relatifs.
            \item \textcolor{Green}{verte} les nombres décimaux.
        \end{itemize} 
        \hint{Certains nombres peuvent être indiqués plusieurs fois.}
        \item Quels sont les nombres restants ?
        Essaie de les classer dans deux catégories de nombres différentes.
    \end{enumerate}
}[\href{https://clg-monnet-briis.ac-versailles.fr/IMG/pdf/cours_fractions-3.pdf}{Collège Monnet}]
}

\scn{Définition des nombres rationnels}

\slide{qf}{
    \exo{Repartition de vols}{
    On a représenté sur le diagramme suivant les vols du mois de février d'une compagnie aérienne.
    
    \ctikz[0.45]{
    % \boundingBox[11.76][10.56][0.5pt][1][(-2.41,-2.86)]
    \draw [shift={(3.50,1.18)},thick,color=gradeColor,fill=gradeColor,fill opacity=0.10] (0,0) -- (89.85:1.13) arc (89.85:119.85:1.13) -- cycle;
    \draw [shift={(3.50,1.18)},thick,color=gradeColor,fill=gradeColor,fill opacity=0.10] (0,0) -- (119.85:1.13) arc (119.85:149.85:1.13) -- cycle;
    \draw [shift={(3.50,1.18)},thick,color=gradeColor,fill=gradeColor,fill opacity=0.10] (0,0) -- (149.85:1.13) arc (149.85:179.85:1.13) -- cycle;
    \draw[thick,color=gradeColor,fill=gradeColor,fill opacity=0.10] (2.70,1.18) -- (2.70,0.38) -- (3.50,0.38) -- (3.50,1.18) -- cycle;
    \draw [thick] (3.50,1.18) circle (7.48cm);
    \draw [shift={(3.50,1.18)},thick,color=gradeColor] (89.85:1.13) arc (89.85:119.85:1.13);
    \draw [shift={(3.50,1.18)},thick,color=gradeColor] (119.85:1.13) arc (119.85:149.85:1.13);
    \draw [shift={(3.50,1.18)},thick,color=gradeColor] (149.85:1.13) arc (149.85:179.85:1.13);
    \draw [shift={(3.50,1.18)},thick,color=gradeColor,fill=gradeColor,fill opacity=0.35]  (0,0) --  plot[domain=-1.57:1.57,variable=\t]({1*7.48*cos(\t r)+0*7.48*sin(\t r)},{0*7.48*cos(\t r)+1*7.48*sin(\t r)}) -- cycle ;
    \draw [shift={(3.50,1.18)},thick,color=gradeColor,fill=gradeColor,fill opacity=0.28]  (0,0) --  plot[domain=3.14:4.71,variable=\t]({1*7.48*cos(\t r)+0*7.48*sin(\t r)},{0*7.48*cos(\t r)+1*7.48*sin(\t r)}) -- cycle ;
    \draw [shift={(3.50,1.18)},thick,color=gradeColor,fill=gradeColor,fill opacity=0.21]  (0,0) --  plot[domain=2.62:3.14,variable=\t]({1*7.48*cos(\t r)+0*7.48*sin(\t r)},{0*7.48*cos(\t r)+1*7.48*sin(\t r)}) -- cycle ;
    \draw [shift={(3.50,1.18)},thick,color=gradeColor,fill=gradeColor,fill opacity=0.14]  (0,0) --  plot[domain=2.09:2.62,variable=\t]({1*7.48*cos(\t r)+0*7.48*sin(\t r)},{0*7.48*cos(\t r)+1*7.48*sin(\t r)}) -- cycle ;
    \draw [shift={(3.50,1.18)},thick,color=gradeColor,fill=gradeColor,fill opacity=0.07]  (0,0) --  plot[domain=1.57:2.09,variable=\t]({1*7.48*cos(\t r)+0*7.48*sin(\t r)},{0*7.48*cos(\t r)+1*7.48*sin(\t r)}) -- cycle ;
    \draw[color=gradeColor] (9.2,1.65) node {France};
    \draw[color=gradeColor] (-0.70,-2.86) node {Europe};
    \draw[color=gradeColor] (-1.6,2.2) node {Amérique};
    \draw[color=gradeColor] (-0.8,5.4) node {Afrique};
    \draw[color=gradeColor] (1.88,7.70) node {Asie};
    \draw [thick,gradeColor] (3.25,2.12) -- (3.17,2.42);
    \draw [thick,gradeColor] (2.81,1.87) -- (2.60,2.09);
    \draw [thick,gradeColor] (2.56,1.43) -- (2.26,1.51);
}
    Dans chaque cas, indiquer quelle fraction représentent les vols vers :
    \multiColItemize{3}{\item  la France \item l'Europe \item l'Asie}

    Au mois de février, cette compagnie a affrété 576 vols. Calculer le nombre de vols vers :
    \multiColItemize{3}{\item  la France \item l'Europe \item l'Asie}
}[\href{https://cache.media.education.gouv.fr/file/Fractions/22/7/RA16_C4_MATH_fractions_flash1_part_fractions_554227.pdf}
{Utiliser les nombres pour comparer, calculer et résoudre des problèmes :
Les fractions - Un exemple de question flash - « Vision-partage » de la fraction}]
}

\slide{cr}{
    \sseq

    Dans ce cours $a,b,c \et d$ designent des nombres.
    
    \section{Définition}
    \df{}{
    On appelle \key{nombre rationnel} est, en mathématiques,
    un nombre qui peut s'exprimer comme le quotient de deux entiers relatifs.
}[\wiki{Nombre_rationnel}]
    \expl{}{
    % \hspace{-1.6cm}%
    \Tableau[%
        DoubleEntree,
        Stretch=1.5,
        Couleur=gradeColor!15,
        LegendesH={$\np{0.6}$,$\num{13.2}$,$\frac{1}{10}$,$\frac{1}{2}$,$60$,$\frac{1}{3}$,$\frac{30}{3}$,$\pi$,$0$,$58\div100$},
        LegendesV={Nombre rationnel ?},
        Largeur=1cm
    ]{\nswr{Oui},\nswr{Oui},\nswr{Oui},\nswr{Oui},\nswr{Oui},\nswr{Non},\nswr{Oui},\nswr{Non},\nswr{Oui},\nswr{Oui}}
}
}

\scn{Sommes et différences de fractions}

\slide{qf}{
    \exo{D'accord ou pas d'accord ?}{
    \begin{align*}
        \dfrac{6+2}{6+4}
        = \dfrac{\cancel{6}+2}{\cancel{6}+4}
        = \dfrac{2}{4}
        = \dfrac{1}{2}
    \end{align*}
}[\href{https://cache.media.education.gouv.fr/file/Fractions/23/4/RA16_C4_MATH_fractions_flash4_travail_erreur_554234.pdf}
{Utiliser les nombres pour comparer, calculer et résoudre des problèmes :
les fractions - Un exemple de questions flash - Travail sur l'erreur}]
}

\slide{exo}{
    \act{Somme de fractions}{
    \begin{enumerate}
        \item Vrai ou Faux ?
        \multiColEnumerate{2}{
            \item $\frac{1}{2} + \frac{3}{2} = \frac{1+3}{2+2}$
            \item $\frac{1}{2} + \frac{3}{4} = \frac{1+3}{2+4}$
            \item $\frac{6}{2} + \frac{4}{2} = \frac{6+4}{2}$
            \item $\frac{1}{3} + \frac{1}{3} = \frac{1}{3+3}$
            \item $\frac{15}{5} + \frac{2}{1} = \frac{15+2\times5}{5}$
            \item $\frac{1}{5} + \frac{5}{5} = \frac{1+5}{5}$
        }
        \item Quelle est la méthode pour additionner deux nombres de même dénominateur ?
        \item Quelle est la méthode pour additionner deux nombres de dénominateur différents ? Par exemple $\frac{3}{4}$ et $\frac{5}{2}$.
    \end{enumerate}
}
}

\slide{cr}{
    % \section{Somme et différence}
    \subsection{De même dénominateur}
    \mthd{Somme ou différence de fractions avec le \key{même dénominateur}}{
    \begin{enumerate}
        \item On prend comme numérateur : la somme ou différence des numérateurs.
        \item On conserve le dénominateur commun.
    \end{enumerate}
}
    \expl{}{
    \multiColEnumerate{2}{
        \item $\frac{1}{2} + \frac{5}{2}
        = \nswr{\frac{1+5}{2} = \frac{6}{2}}$
        \item $\frac{14}{21} - \frac{12}{21}
        = \nswr{\frac{14-12}{21} = \frac{2}{21}}$
        \item $\frac{a}{d} + \frac{b}{d}
        = \nswr{\frac{a+b}{d}}$
        \item $\frac{a}{d} - \frac{b}{d}
        = \nswr{\frac{a-b}{d}}$
    }
}
}

\slide{cr}{
    \subsection{De dénominateur différents}
    \mthd{Somme et différence de fractions de \key{dénominateurs différents}}{
    \begin{enumerate}
        \item On met les fractions au même dénominateur.
        \item On fait leur somme ou différence comme précedement.
    \end{enumerate}
}
    \expl{}{
    \multiColEnumerate{1}{
        \item $\dfrac{1}{2} - \dfrac{1}{4}
        = \nswr{\dfrac{2}{4} - \dfrac{1}{4}
        = \dfrac{2-1}{4} = \dfrac{1}{4}
        }$
        \item $\dfrac{2}{12} + \dfrac{2}{3}
        = \nswr{\dfrac{2}{12} + \dfrac{2\times4}{3\times4}
        = \dfrac{2}{12} + \dfrac{8}{12}
        = \dfrac{2+8}{12}
        = \dfrac{10}{12}
        }$
        \item $\dfrac{3}{5} + \dfrac{6}{7}
        = \nswr{\dfrac{3\times7}{5\times7} + \dfrac{6\times5}{7\times5}
        = \dfrac{21}{35} + \dfrac{30}{35}
        = \dfrac{21+30}{35} = \dfrac{51}{35}
        }$
        \item $\dfrac{12}{3} - \dfrac{6}{2}
        = \nswr{\dfrac{12\times2}{3\times2} - \dfrac{6\times3}{2\times3}
        = \dfrac{24}{6} - \dfrac{18}{6}
        = \dfrac{24-18}{6} = \dfrac{6}{6}
        }$
        \item $\dfrac{a}{b} + \dfrac{c}{d}
        = \nswr{\dfrac{a\times d}{b\times d} + \dfrac{c \times b}{d\times b}
        = \dfrac{ad}{bd} + \dfrac{cb}{bd}
        = \dfrac{ad+cb}{bd}
        }$
    }
}
}

\scn{Exercices - Sommes et différences de fractions}

\slide{qf}{
    \exo{QCM sur les fractions 1}{
    \begin{enumerate}
        \item Dans un ruisseau, il s'écoule, en moyenne, $120 \Vol{}$ d'eau en 45 $\minute$.
        Le débit de ce ruisseau, en $\Vol{}/\hour$, est égal à :
        \multiColEnumerate{3}{
            \item $120 \div \frac{3}{4}$
            \item $120 \times \frac{3}{4}$
            \item $120 \times \np{0.75}$
        }
        \item La fraction $\frac{143}{132}$ est :
        \multiColEnumerate{3}{
            \item irréductible
            \item comprise entre $1$ et $\np{1.1}$
            \item décimale
        }
    \end{enumerate}
}
}

\bookSlide{21p121,24p121,27p121,28p121,57p124}[7.5cm][2]

\ifthenelse{\boolean{answer}}{
    \slide{exo}{
        \newcommand{\cf}[2]{\ensuremath{\Capa*{\dfrac{#1}{#2}}}}

\exo{28p121}{
    \nswr[0]{
        \begin{align*}
            \cf{1}{3} + \cf{1}{4} + \cf{5}{12} &= \cf{4}{12} + \cf{3}{12} + \cf{5}{12}\\
            &= \cf{4+3+5}{12}\\
            &= \cf{12}{12}\\
            &= \Capa{1}
        \end{align*}
        En additionnant ces trois quantité, on obtient $\Capa{1}$.
        Achille a tort, la carafe sera remplie à ras bord mais ne débordera pas.
    }
}[\mi]
        \exo{54p124}{
    \nswr[0]{
        On a $\dfrac{32}{4} = 8$. Il y a alors \Octet[Go]{8} de la clé USB utilisés par les photos.\\
        Et $32 - (20 + 8) = 4$. Il reste donc \Octet[Go]{4} de la clé USB.\\
        Or $\Octet[Go]{4} = \Octet[Mo]{4000}$,
        et $4000 \div 4 = 1000$.\\
        Il reste ainsi de la place pour environ 1000 musiques au format MP3 sur la clé USB de Samuel.
    }
}[\mi]
    }
}{}

\scn{Definition - Produits de fractions}

\caSlide{5-6-7}

\slide{cr}{
    \section{Produit et quotient}
    \subsection{Produit}
    \mthd{Produits de fractions}{
    \begin{enumerate}
        \item On prend comme numérateur : le produit des numérateurs.
        \item On prend comme dénominateur : le produit des dénominateurs.
    \end{enumerate}
}[][\cmdGeoGebra[azrs82xa]]
    \expl{}{
    \multiColEnumerate{2}{
        \item $\frac{2}{5} \times \frac{3}{4} = \nswr{\frac{2 \times 3}{5 \times 4} = \frac{6}{20} = \frac{3}{10}}$
        \item $\frac{5}{3} \times \frac{1}{3} = \nswr{\frac{5 \times 1}{3 \times 3} = \frac{5}{9}}$
        \item $\frac{7}{8} \times \frac{-2}{3} = \nswr{\frac{7 \times (-2)}{8 \times 3} = \frac{-14}{24} = \frac{-7}{12}}$
        \item $\frac{a}{c} \times \frac{b}{d} = \nswr{\frac{a \times b}{c \times d}}$
    }
}
}

\scn{Exercices - Produits de fractions}

\caSlide{8-9-10}

\bookSlide{4p132,5p132,7p132,48p135}[7.5cm][2]

\ifthenelse{\boolean{answer}}{
    \slide{exo}{
        \exo{4p132}{
    \multiColItemize{2}{
        \item $A = \dfrac{5}{8} \times \dfrac{-3}{2} =
        \nswr[0]{\dfrac{5 \times (-3)}{8 \times 2} = \dfrac{-15}{16} = - \dfrac{15}{16}}$
        \item $B = \dfrac{7}{8} \times \dfrac{-3}{8} =
        \nswr[0]{\dfrac{7 \times (-3)}{8 \times 8} = \dfrac{-21}{64} = - \dfrac{21}{64}}$
        \item $C = -3 \times \dfrac{4}{5} =
        \nswr[0]{\dfrac{-3 \times 4}{5} = \dfrac{-12}{5} = - \dfrac{12}{5}}$
        \item $D = \dfrac{-2}{11} \times \dfrac{-10}{3} =
        \nswr[0]{\dfrac{-2 \times (-10)}{11 \times 3} = \dfrac{20}{33}}$
    }
}[\mi]
        \exo{5p132}{
    \multiColItemize{1}{
        \item $A = \dfrac{-7}{3} \times \dfrac{5}{4} \times \dfrac{-1}{3}
        = \nswr{\dfrac{-7 \times 5 \times -1}{3 \times 4 \times 3}
        \dfrac{7 \times 5 \times 1}{3 \times 4 \times 3}
        = \dfrac{35}{36}}$
        \item $B = \dfrac{4}{3} \times \dfrac{-5}{7} \times \dfrac{-1}{3} \times (-5)
        = \nswr{\dfrac{4 \times -5 \times -1 \times -5}{3 \times 7 \times 3}
        = -\dfrac{4 \times 5 \times 1 \times 5}{3 \times 7 \times 3}
        = -\dfrac{100}{63}}$
    }
}[\mi]
        \exo{7p132}{
    \multiColItemize{1}{
        \item $A = \dfrac{-2}{21} \times \dfrac{14}{5} - \dfrac{8}{5} =
        \nswr[0]{\dfrac{-28}{105} - \dfrac{8}{5} = \dfrac{-28}{105} - \dfrac{168}{105} = \dfrac{-196}{105} = -\dfrac{28}{15}}$
        \item $B = \dfrac{1}{4} + \dfrac{5}{3} \times \dfrac{-1}{8} =
        \nswr[0]{\dfrac{1}{4} + \dfrac{-5}{24} = \dfrac{6}{24} - \dfrac{5}{24} = \dfrac{1}{24}}$
        \item $C = \dfrac{4}{-3} + \dfrac{-7}{6} \times \dfrac{-2}{5} =
        \nswr[0]{-\dfrac{4}{3} + \dfrac{14}{30} = -\dfrac{4}{3} + \dfrac{7}{15} = -\dfrac{20}{15} + \dfrac{7}{15} = -\dfrac{13}{15}}$
        \item $D = (\dfrac{8}{15} - \dfrac{7}{5}) - (\dfrac{-1}{6} + \dfrac{2}{9}) =
        \nswr[0]{(\dfrac{8}{15} - \dfrac{21}{15}) - (\dfrac{-3}{18} + \dfrac{4}{18}) = \dfrac{-13}{15} - \dfrac{1}{18} = \dfrac{-234}{270} - \dfrac{15}{270} = \dfrac{-249}{270} = -\dfrac{83}{90}}$
    }
}
    }
}{}

\scn{Utiliser la simplification pour calculer des produits de fractions}

\caSlide{11-12-13}

\slide{exo}{
    \exo{Multiplier des fractions de manière efficace}{
    Utiliser la simplification de fractions pour calculer les produits suivants:
    \begin{enumerate}
        \item $\dfrac{15}{8} \times \dfrac{9}{15} 
        = \nswr[0]{\dfrac{\cancel{15} \times 9}{8 \times \cancel{15}} = \dfrac{8}{9}}$
        \item $\dfrac{-6}{36} \times \dfrac{36}{-15} 
        = \nswr[0]{\dfrac{6 \times \cancel{36}}{\cancel{36} \times 5} = \dfrac{6}{5}}$
        \item $\dfrac{7}{-9} \times \dfrac{-9}{-7}
        = \nswr[0]{- \dfrac{\cancel{7} \times \cancel{9}}{\cancel{9} \times \cancel{7}} = - \dfrac{1}{1} = - 1}$
        \item $\dfrac{35}{3} \times \dfrac{4}{70}
        = \nswr[0]{\dfrac{35 \times 4}{3 \times 70}
        = \dfrac{\cancel{35} \times 4}{3 \times \cancel{35} \times 2}
        = \dfrac{4}{3 \times 2}
        = \dfrac{\cancel{2} \times 2}{3 \times \cancel{2}}
        = \dfrac{2}{3}}$
        \item $\dfrac{-3}{30} \times \dfrac{36}{7}
        = \nswr[0]{-\dfrac{3 \times 36}{30 \times 7}
        = -\dfrac{3 \times \cancel{6} \times 6}{5 \times \cancel{6} \times 7}
        = -\dfrac{3 \times 6}{5 \times 7}
        = -\dfrac{18}{35}}$
    \end{enumerate}
}[\met{TFrac}[13]]
}

\bookSlide{6p132}[7.5cm][1]

\scn{Definition - Inverse et Quotients}

\slide{qf}{
    \exo{QCM sur les fractions 2}{
    Dans chacune des ci-dessous, une seule réponse est correcte. Laquelle ?
    \begin{enumerate}
        \item L'inverse de $\dfrac{2}{7}$ est :
        \multiColEnumerate{3}{
            \item supérieur à $7$
            \item égale à $\np{3.5}$
            \item inférieur à $2$
        }
        \item $\dfrac{1}{15}$ est égale à :
        \multiColEnumerate{3}{
            \item $\np{0.0666666667}$
            \item $\dfrac{2}{5} \div \dfrac{1}{6}$
            \item $\dfrac{2}{30}$
        }
    \end{enumerate}
}
% [\href{https://cache.media.education.gouv.fr/file/Fractions/23/2/RA16_C4_MATH_fractions_flash3_sens_quotient_554232.pdf}
% {Utiliser les nombres pour comparer, calculer et résoudre des problèmes : les fractions}]
}

\slide{cr}{
    \subsection{Inverse}
    \df{}{
    On appelle \key{inverse} d'un nombre $x$, le nombre qui, multiplié par $x$, donne $1$.
}[\wiki{Inverse}]
    \expl{}{
    \multiColEnumerate{1}{
        \item $0.5 \times \nswr{2} = 1$. L'inverse de $0.5$ est donc $\nswr{2}$.
        \item $10 \times \nswr{0.1} = 1$. L'inverse de $10$ est donc $\nswr{0.1}$.
        \item $\dfrac{1}{16} \times \nswr{16} = 1$. L'inverse de $\dfrac{1}{9}$ est donc $\nswr{9}$.
        \item $23 \times \nswr{\dfrac{1}{23}} = 1$. L'inverse de $23$ est donc $\nswr{\dfrac{1}{23}}$.
        \item $\dfrac{4}{5} \times \nswr{\dfrac{5}{4}} = 1$. L'inverse de $\dfrac{4}{5}$ est donc $\nswr{\dfrac{5}{4}}$.
    }
}
    \rmk{}{
    \begin{enumerate}
        \item L'inverse d'un nombre $x$ est $\nswr{\dfrac{1}{x}}$.
        \item L'inverse d'un nombre $\dfrac{a}{b}$ est $\nswr{\dfrac{b}{a}}$.
    \end{enumerate}
}
}

\slide{cr}{
    \subsection{Quotient}
    \mthd{Quotient de fractions}{
    \key{Diviser} un nombre par une fraction est équivalent à la \key{multiplier par l'inverse} de cette même fraction.
}
    \expl{}{
    \multiColEnumerate{1}{
        \item $\dfrac{2}{5} \div \dfrac{3}{4} = \nswr{\dfrac{2}{5} \times \dfrac{4}{3} = \dfrac{2 \times 4}{5 \times 3} = \dfrac{8}{15}}$
        \item $\dfrac{5}{3} \div \dfrac{1}{3} = \nswr{\dfrac{5}{3} \times \dfrac{3}{1} = \dfrac{5 \times 3}{3 \times 1} = 5}$
        \item $\dfrac{7}{8} \div \dfrac{-2}{3} = \nswr{\dfrac{7}{8} \times \dfrac{3}{-2} = \dfrac{7 \times 3}{8 \times (-2)} = \dfrac{21}{-16} = -\dfrac{21}{16}}$
        \item $\dfrac{a}{c} \div \dfrac{b}{d} = \nswr{\dfrac{a}{c} \times \dfrac{d}{b} = \dfrac{a \times d}{c \times b}}$
    }
}
}

\scn{Exercices - Inverse et Quotients}

\caSlide{14-15-16}

\bookSlide{22p133,28p133,37p134,42p134}[7.5cm][2]

\scn{Problèmes mettant en jeu des fractions}

\caSlide{17-18-19}

\slide{exo}{
    \exo{Le compte est bon}{
    Voici une liste de nombres : 
    $1$, $2$, $3$, $4$,
    $\dfrac{1}{3}$, $\dfrac{5}{4}$, $\dfrac{1}{4}$, $\dfrac{3}{6}$, $\dfrac{4}{7}$, $\dfrac{5}{8}$, $\dfrac{6}{9}$.
    \begin{enumerate}
        \item Pour obtenir le nombre $\dfrac{9}{8}$,
    vous pouvez effectuer toutes les opérations que vous souhaitez et décrire les résultats intermédiaires.
    Attention,
    chaque nombre ci-dessus est unique et ne peut être utilisé qu'une seule fois,
    en les convertissant si nécessaire en fractions égales.
    \\\bonus
        \item Trouvez le résultat en utilisant le moins d'opérations possible.
        \item Faites de même avec $\dfrac{6}{7}$, $\dfrac{4}{9}$, $\dfrac{9}{24}$, $\dfrac{9}{28}$.
    \end{enumerate}
}[\prbltq{le-compte-est-bon-avec-des-fractions}]

\nswr[0]{
    Pour obtenir $\dfrac{9}{8}$, on peut procéder comme suit :
    \begin{itemize}
        \item $\dfrac{5}{4} + \dfrac{1}{4} = \dfrac{6}{4} = \dfrac{3}{2}$
        \item $\dfrac{3}{2} \div \dfrac{1}{3} = \dfrac{3}{2} \times 3 = \dfrac{9}{2}$
        \item $\dfrac{9}{2} \times \dfrac{1}{4} = \dfrac{9}{8}$
    \end{itemize}
    On peut aussi faire :
    \begin{itemize}
        \item $\dfrac{5}{8} + \dfrac{3}{6} = \dfrac{5}{8} + \dfrac{1}{2} = \dfrac{5}{8} + \dfrac{4}{8} = \dfrac{9}{8}$
    \end{itemize}
}

}

\bookSlide{62p137,60p137}[7.5cm][2]

\scn{Tache complexe mettant en jeu des fractions}

\slide{qf}{
    \exo{Chocolat au lait - Olé}{
    \begin{enumerate}
        \item On prépare une boisson chocolatée en mélangeant du chocolat et du lait.
        \begin{itemize}
            \item La recette $A$ mélange $3$ doses de chocolat pour $2$ doses de lait.
            \item La recette $B$ mélange $2$ doses de chocolat pour $1$ dose de lait.
        \end{itemize}
        Quel est le Mélange qui a le plus le goût du chocolat ?
        \item On remplit deux récipients identiques,
        l'un avec le liquide A,
        l'autre avec le liquide B.
        \begin{itemize}
            \item $5$ litres du liquide $A$ pèsent $3$ kg.
            \item $7$ litres du liquide $B$ pèsent $4$ kg.
        \end{itemize}
        Lequel des deux récipients ainsi remplis est le plus lourd ?
    \end{enumerate}
}[\prbltq{chocolat-au-lait-ole}]
}

\bookSlide{61-1p137,61-2p137}[7.5cm][2]

\caSlide{20-21}

\caSlide{22-23-24}

\caSlide{25-26-27}

\caSlide{28-29-30}