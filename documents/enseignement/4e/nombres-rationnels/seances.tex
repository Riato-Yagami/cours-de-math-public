% VARIABLES %%%
\setSeq{7}{Nombres rationnels}
\setGrade{4e}

% \forStudents
\forPrint

\dym{https://www.maths-et-tiques.fr/telech/19Fractions1.pdf}

\obj{
    \item Definition d'un nombre rationnel.
    \item Calculer des sommes, difference, produit et quotient de fractions.
    \item Utiliser la notion d'inverse.
}

\scn{Définition des nombres rationnels}

\slide{qf}{
    \exo{Fraction quotient}{
    \noCalculator\\
    Ecrires les nombres suivant sous forme décimale :
    \multiColEnumerate{2}{
        \item $\frac{5}{2} = \nswr{\np{2.5}}$
        \item $\frac{9 \times 6}{3} = \nswr{27}$
        \item $\frac{4 + 6}{6-11} = \nswr{-2}$
        \item $\frac{6 \times (-6)}{-100 \div 10} = \nswr{\np{3.6}}$
    }
}

}

\slide{exo}{
    \act{Nature des nombres}{
    En maternelle, nous avons appris à compter des objets en utilisant les nombres $1, 2, 3$, etc.
    Ces nombres sont les premiers utilisés « naturellement »,
    et on les appelle les nombres entiers naturels.
    Depuis l'école primaire et au collège, nous avons découvert d'autres nombres.
    Voici une liste de nombres :
    \multiColItemize{3}{
        \item $-\np{27.2}$
        \item $-\sqrt{4}$
        \item $\frac{10371}{100}$
        \item $\frac{27}{13}$
        \item $\frac{3}{2}$
        \item $\frac{-21}{15}$
        \item $\np{0.33333}\ldots$
        \item $\pi$
        \item $\frac{-10}{5}$
        \item $\sqrt{2}$
        \item $\frac{47}{21}$
        \item $-15 + 20$
        \item $\frac{-10}{3}$
        \item $37$
        \item $1 \div 7$
    }
    \begin{enumerate}
        \item Indique par une pastille :
        \begin{itemize}
            \item \textcolor{Blue}{bleue} les nombres entiers naturels.
            \item \textcolor{Red}{rouge} les nombres entiers relatifs.
            \item \textcolor{Green}{verte} les nombres décimaux.
        \end{itemize} 
        \hint{Certains nombres peuvent être indiqués plusieurs fois.}
        \item Quels sont les nombres restants ?
        Essaie de les classer dans deux catégories de nombres différentes.
    \end{enumerate}
}[\href{https://clg-monnet-briis.ac-versailles.fr/IMG/pdf/cours_fractions-3.pdf}{Collège Monnet}]
}

\slide{cr}{
    \sseq

    Dans se cours $a,b,c \et d$ designes des nombres.
    
    \section{Définition}
    \df{}{
    On appelle \key{nombre rationnel} est, en mathématiques,
    un nombre qui peut s'exprimer comme le quotient de deux entiers relatifs.
}[\wiki{Nombre_rationnel}]
    \expl{}{
    % \hspace{-1.6cm}%
    \Tableau[%
        DoubleEntree,
        Stretch=1.5,
        Couleur=gradeColor!15,
        LegendesH={$\np{0.6}$,$\num{13.2}$,$\frac{1}{10}$,$\frac{1}{2}$,$60$,$\frac{1}{3}$,$\frac{30}{3}$,$\pi$,$0$,$58\div100$},
        LegendesV={Nombre rationnel ?},
        Largeur=1cm
    ]{\nswr{Oui},\nswr{Oui},\nswr{Oui},\nswr{Oui},\nswr{Oui},\nswr{Non},\nswr{Oui},\nswr{Non},\nswr{Oui},\nswr{Oui}}
}
}

\scn{Sommes et différences de fractions}

\slide{qf}{
    \exo{D'accord ou pas d'accord ?}{
    \begin{align*}
        \dfrac{6+2}{6+4}
        = \dfrac{\cancel{6}+2}{\cancel{6}+4}
        = \dfrac{2}{4}
        = \dfrac{1}{2}
    \end{align*}
}[\href{https://cache.media.education.gouv.fr/file/Fractions/23/4/RA16_C4_MATH_fractions_flash4_travail_erreur_554234.pdf}
{Utiliser les nombres pour comparer, calculer et résoudre des problèmes :
les fractions - Un exemple de questions flash - Travail sur l'erreur}]
}

\slide{exo}{
    \act{Somme de fractions}{
    \begin{enumerate}
        \item Vrai ou Faux ?
        \multiColEnumerate{2}{
            \item $\frac{1}{2} + \frac{3}{2} = \frac{1+3}{2+2}$
            \item $\frac{1}{2} + \frac{3}{4} = \frac{1+3}{2+4}$
            \item $\frac{6}{2} + \frac{4}{2} = \frac{6+4}{2}$
            \item $\frac{1}{3} + \frac{1}{3} = \frac{1}{3+3}$
            \item $\frac{15}{5} + \frac{2}{1} = \frac{15+2\times5}{5}$
            \item $\frac{1}{5} + \frac{5}{5} = \frac{1+5}{5}$
        }
        \item Quelle est la méthode pour additionner deux nombres de même dénominateur ?
        \item Quelle est la méthode pour additionner deux nombres de dénominateur différents ? Par exemple $\frac{3}{4}$ et $\frac{5}{2}$.
    \end{enumerate}
}
}

\slide{cr}{
    % \section{Somme et différence}
    \subsection{De même dénominateur}
    \mthd{Somme ou différence de fractions avec le \key{même dénominateur}}{
    \begin{enumerate}
        \item On prend comme numérateur : la somme ou différence des numérateurs.
        \item On conserve le dénominateur commun.
    \end{enumerate}
}
    \expl{}{
    \multiColEnumerate{2}{
        \item $\frac{1}{2} + \frac{5}{2}
        = \nswr{\frac{1+5}{2} = \frac{6}{2}}$
        \item $\frac{14}{21} - \frac{12}{21}
        = \nswr{\frac{14-12}{21} = \frac{2}{21}}$
        \item $\frac{a}{d} + \frac{b}{d}
        = \nswr{\frac{a+b}{d}}$
        \item $\frac{a}{d} - \frac{b}{d}
        = \nswr{\frac{a-b}{d}}$
    }
}
}

\slide{cr}{
    \subsection{De dénominateur différents}
    \mthd{Somme et différence de fractions de \key{dénominateurs différents}}{
    \begin{enumerate}
        \item On met les fractions au même dénominateur.
        \item On fait leur somme ou différence comme précedement.
    \end{enumerate}
}
    \expl{}{
    \multiColEnumerate{1}{
        \item $\dfrac{1}{2} - \dfrac{1}{4}
        = \nswr{\dfrac{2}{4} - \dfrac{1}{4}
        = \dfrac{2-1}{4} = \dfrac{1}{4}
        }$
        \item $\dfrac{2}{12} + \dfrac{2}{3}
        = \nswr{\dfrac{2}{12} + \dfrac{2\times4}{3\times4}
        = \dfrac{2}{12} + \dfrac{8}{12}
        = \dfrac{2+8}{12}
        = \dfrac{10}{12}
        }$
        \item $\dfrac{3}{5} + \dfrac{6}{7}
        = \nswr{\dfrac{3\times7}{5\times7} + \dfrac{6\times5}{7\times5}
        = \dfrac{21}{35} + \dfrac{30}{35}
        = \dfrac{21+30}{35} = \dfrac{51}{35}
        }$
        \item $\dfrac{12}{3} - \dfrac{6}{2}
        = \nswr{\dfrac{12\times2}{3\times2} - \dfrac{6\times3}{2\times3}
        = \dfrac{24}{6} - \dfrac{18}{6}
        = \dfrac{24-18}{6} = \dfrac{6}{6}
        }$
        \item $\dfrac{a}{b} + \dfrac{c}{d}
        = \nswr{\dfrac{a\times d}{b\times d} + \dfrac{c \times b}{d\times b}
        = \dfrac{ad}{bd} + \dfrac{cb}{bd}
        = \dfrac{ad+cb}{bd}
        }$
    }
}
}

\scn{Produits de fractions}

\slide{qf}{
    \exo{QCM sur les fractions 1}{
    \begin{enumerate}
        \item Dans un ruisseau, il s'écoule, en moyenne, $120 \Vol{}$ d'eau en 45 $\minute$.
        Le débit de ce ruisseau, en $\Vol{}/\hour$, est égal à :
        \multiColEnumerate{3}{
            \item $120 \div \frac{3}{4}$
            \item $120 \times \frac{3}{4}$
            \item $120 \times \np{0.75}$
        }
        \item La fraction $\frac{143}{132}$ est :
        \multiColEnumerate{3}{
            \item irréductible
            \item comprise entre $1$ et $\np{1.1}$
            \item décimale
        }
    \end{enumerate}
}
}

\slide{cr}{
    \section{Produit et quotient}
    \subsection{Produit}
    \mthd{Produits de fractions}{
    \begin{enumerate}
        \item On prend comme numérateur : le produit des numérateurs.
        \item On prend comme dénominateur : le produit des dénominateurs.
    \end{enumerate}
}[][\cmdGeoGebra[azrs82xa]]
    \expl{}{
    \multiColEnumerate{2}{
        \item $\frac{2}{5} \times \frac{3}{4} = \nswr{\frac{2 \times 3}{5 \times 4} = \frac{6}{20} = \frac{3}{10}}$
        \item $\frac{5}{3} \times \frac{1}{3} = \nswr{\frac{5 \times 1}{3 \times 3} = \frac{5}{9}}$
        \item $\frac{7}{8} \times \frac{-2}{3} = \nswr{\frac{7 \times (-2)}{8 \times 3} = \frac{-14}{24} = \frac{-7}{12}}$
        \item $\frac{a}{c} \times \frac{b}{d} = \nswr{\frac{a \times b}{c \times d}}$
    }
}
}
\slide{cr}{
    \subsection{Inverse}
    \df{}{
    On appelle \key{inverse} d'un nombre $x$, le nombre qui, multiplié par $x$, donne $1$.
}[\wiki{Inverse}]
    \expl{}{
    \multiColEnumerate{1}{
        \item $0.5 \times \nswr{2} = 1$. L'inverse de $0.5$ est donc $\nswr{2}$.
        \item $10 \times \nswr{0.1} = 1$. L'inverse de $10$ est donc $\nswr{0.1}$.
        \item $\dfrac{1}{16} \times \nswr{16} = 1$. L'inverse de $\dfrac{1}{9}$ est donc $\nswr{9}$.
        \item $23 \times \nswr{\dfrac{1}{23}} = 1$. L'inverse de $23$ est donc $\nswr{\dfrac{1}{23}}$.
        \item $\dfrac{4}{5} \times \nswr{\dfrac{5}{4}} = 1$. L'inverse de $\dfrac{4}{5}$ est donc $\nswr{\dfrac{5}{4}}$.
    }
}
    \rmk{}{
    \begin{enumerate}
        \item L'inverse d'un nombre $x$ est $\nswr{\dfrac{1}{x}}$.
        \item L'inverse d'un nombre $\dfrac{a}{b}$ est $\nswr{\dfrac{b}{a}}$.
    \end{enumerate}
}
}

\slide{qf}{
    \exo{QCM sur les fractions 2}{
    Dans chacune des ci-dessous, une seule réponse est correcte. Laquelle ?
    \begin{enumerate}
        \item L'inverse de $\dfrac{2}{7}$ est :
        \multiColEnumerate{3}{
            \item supérieur à $7$
            \item égale à $\np{3.5}$
            \item inférieur à $2$
        }
        \item $\dfrac{1}{15}$ est égale à :
        \multiColEnumerate{3}{
            \item $\np{0.0666666667}$
            \item $\dfrac{2}{5} \div \dfrac{1}{6}$
            \item $\dfrac{2}{30}$
        }
    \end{enumerate}
}
% [\href{https://cache.media.education.gouv.fr/file/Fractions/23/2/RA16_C4_MATH_fractions_flash3_sens_quotient_554232.pdf}
% {Utiliser les nombres pour comparer, calculer et résoudre des problèmes : les fractions}]
}

\slide{exo}{
    \exo{Le compte est bon}{
    Voici une liste de nombres : 
    $1$, $2$, $3$, $4$,
    $\dfrac{1}{3}$, $\dfrac{5}{4}$, $\dfrac{1}{4}$, $\dfrac{3}{6}$, $\dfrac{4}{7}$, $\dfrac{5}{8}$, $\dfrac{6}{9}$.
    \begin{enumerate}
        \item Pour obtenir le nombre $\dfrac{9}{8}$,
    vous pouvez effectuer toutes les opérations que vous souhaitez et décrire les résultats intermédiaires.
    Attention,
    chaque nombre ci-dessus est unique et ne peut être utilisé qu'une seule fois,
    en les convertissant si nécessaire en fractions égales.
    \\\bonus
        \item Trouvez le résultat en utilisant le moins d'opérations possible.
        \item Faites de même avec $\dfrac{6}{7}$, $\dfrac{4}{9}$, $\dfrac{9}{24}$, $\dfrac{9}{28}$.
    \end{enumerate}
}[\prbltq{le-compte-est-bon-avec-des-fractions}]

\nswr[0]{
    Pour obtenir $\dfrac{9}{8}$, on peut procéder comme suit :
    \begin{itemize}
        \item $\dfrac{5}{4} + \dfrac{1}{4} = \dfrac{6}{4} = \dfrac{3}{2}$
        \item $\dfrac{3}{2} \div \dfrac{1}{3} = \dfrac{3}{2} \times 3 = \dfrac{9}{2}$
        \item $\dfrac{9}{2} \times \dfrac{1}{4} = \dfrac{9}{8}$
    \end{itemize}
    On peut aussi faire :
    \begin{itemize}
        \item $\dfrac{5}{8} + \dfrac{3}{6} = \dfrac{5}{8} + \dfrac{1}{2} = \dfrac{5}{8} + \dfrac{4}{8} = \dfrac{9}{8}$
    \end{itemize}
}

}

\slide{cr}{
    \subsection{Quotient}
    \mthd{Quotient de fractions}{
    \key{Diviser} un nombre par une fraction est équivalent à la \key{multiplier par l'inverse} de cette même fraction.
}
    \expl{}{
    \multiColEnumerate{1}{
        \item $\dfrac{2}{5} \div \dfrac{3}{4} = \nswr{\dfrac{2}{5} \times \dfrac{4}{3} = \dfrac{2 \times 4}{5 \times 3} = \dfrac{8}{15}}$
        \item $\dfrac{5}{3} \div \dfrac{1}{3} = \nswr{\dfrac{5}{3} \times \dfrac{3}{1} = \dfrac{5 \times 3}{3 \times 1} = 5}$
        \item $\dfrac{7}{8} \div \dfrac{-2}{3} = \nswr{\dfrac{7}{8} \times \dfrac{3}{-2} = \dfrac{7 \times 3}{8 \times (-2)} = \dfrac{21}{-16} = -\dfrac{21}{16}}$
        \item $\dfrac{a}{c} \div \dfrac{b}{d} = \nswr{\dfrac{a}{c} \times \dfrac{d}{b} = \dfrac{a \times d}{c \times b}}$
    }
}
}