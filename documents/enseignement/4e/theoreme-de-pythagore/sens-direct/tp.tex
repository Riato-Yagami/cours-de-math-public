% VARIABLES %%%
\def\authors{\jules \ et \href{http://www.cellulegeometrie.eu/documents/pub/pub_14.pdf}{la Haute École en Hainaut}}
\setGrade{4e}
\def\assignmentNameWidth{6cm}
\tp{Théorème de Pythagore}
\def\imgPath{enseignement/4e/divisibilite-et-nombres-premiers/}
%%

\vspace*{-1cm}
\hint{
    \begin{itemize}
        \item Bien lire les indications.
        \item Répondre aux questions sur son cahier d'exercices.
        \item Appeler M. Pesin à la fin de chaque partie.
        \item Enregistrer les productions avec le nom : "NOM.S-tp-pythagore-partie-x.ggb"
    \end{itemize}
}
\vspace*{-0.75cm}

\section{Théorème de Pythagore classique}

\begin{enumerate}
    \item Sur \Geogebra \def\iconPath{geogebra/}
    faire un clique droit et désélectionner \tool{Afficher axes}[axes].
    Dans \tool{Afficher Grille}[grid],
    sélectionner : \tool{Pas de grille}.
    \item Créer deux points $A$ et $B$. Puis créer le segment $[AB]$.\\
    \hint{Outil à utiliser : \tool{Point}[dot] \tool{Segment}[segment]}
    \item Créer une perpendiculaire à la droite $(AB)$ passant par $A$.
    Puis placer $C$ sur cette perpendiculaire. Enfin, créer le segment $[CB]$.\\
    \hint{Nouvel outil à utiliser : \tool{Perpendiculaire}[perpendicular]; il faut d'abord selectionner votre droite puis le point.}
    \item Quelle est la nature du triangle $ABC$ ?
    \item Sur chacun des côtés du triangle, créer des carrés.\\
    \hint{Nouvel outil à utiliser : \tool{Polygone régulier}[regular-polygon],
    après avoir sélectionné l'outil,
    cliquer sur les deux premiers sommets du polygone et indiquer le nombre de sommets souhaité.
    Attention, en fonction du premier sélectionné, le polygone se crée d'un côté du segment ou de l'autre.}
    \item Quelle relation existe entre ces trois carrés ? Utiliser l'outil \tool{Aire}[area] pour le vérifier.
\end{enumerate}

\section{Pythagore avec des triangles équilatéraux}

\begin{enumerate}
    \item Dans un nouveau fichier \Geogebra \def\iconPath{geogebra/} reproduire les trois premières étapes du \textbf{\color{Red}I.} pour obtenir un nouveau triangle $ABC$.
    \item Sur chacun des côtés du triangle, créer des triangles équilatéraux.\\
    \hint{Outil à utiliser : \tool{Polygone régulier}[regular-polygon]}
    \item Mesurer l'aire de ces triangles. Que remarque-t-on ?\\
    \hint{Outil à utiliser : \tool{Aire}[area]}
    \item Émettre une conjecture commençant par :
    «
    \Sialors{un triangle est rectangle}{l'aire du triangle équilatéral construit sur l'hypoténuse \hole}
    »
\end{enumerate}

\section{Pythagore avec des demi-cercles}

\begin{enumerate}
    \item Dans un nouveau fichier \Geogebra \def\iconPath{geogebra/} reproduire les trois premières étapes du \textbf{\color{Red}I.} pour obtenir un nouveau triangle $ABC$.
    \item Placer un point au milieu de chacun des côtés du triangle.\\
    \hint{Outil à utiliser : \tool{Milieu ou centre}[center]}
    \item Sur chacun des côtés du triangle, créer des demi-cercles avec l'outil.\\
    \hint{Outil à utiliser : \tool{Secteur circulaire}[circular-sector],
    après avoir sélectionné l'outil,
    sélectionner d'abord le centre du demi-cercle puis les deux sommets du triangle désirés.
    }
    \item Mesurer l'aire des demi-disques. Que remarque-t-on ?
    \item Émettre une conjecture.
    \item Donner la formule de l'aire d'un disque. D'un demi-disque.
    \item Prouver la conjecture.\\
    \hint{Il faudra utiliser le Théorème de Pythagore et la distributivité.}
\end{enumerate}
