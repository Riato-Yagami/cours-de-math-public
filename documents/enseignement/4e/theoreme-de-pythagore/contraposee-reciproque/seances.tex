% VARIABLES %%%
\setSeq{4}{Théorème de Pythagore - Contraposée et réciproque}
\setGrade{4e}
\def\imgPath{enseignement/4e/theoreme-de-pythagore/contraposee-et-reciproque/}

% \forPrint
% \setboolean{answer}{true}

\def\ym{https://www.maths-et-tiques.fr/telech/19Pyth2.pdf}
%%

\obj{
    \item Comprendre les notions de réciproque et de contraposée.
    \item Utiliser la contraposée du Théorème de Pythagore pour montrer qu'un triangle n'est pas rectangle.
    \item Utiliser la réciproque du Théorème de Pythagore pour montrer qu'un triangle est rectangle.
    \item Déterminer si un triangle est rectangle ou non.
}

\obj{
    \item Reconnaitre sur un graphique une situation de proportionnalité ou de non proportionnalité.
    \item Calcule d'une quatrième proportionnelle.
    \item Utiliser une formule liant deux grandeurs dans une situation de proportionnalité.
    \item Résoudre des problèmes en utilisant la proportionnalité dans le cadre de la géométrie.
}[Flash]

\scn{Découvrir des notions de logiques ; la réciproque}

\slide{qf}{\bvspace{-0.35cm}
    \begin{enumerate}
        \item Les tableaux suivants représentent-ils des situations de proportionnalité ?
        Utilisez une calculatrice pour vérifier vos hypothèses.
        \multiColEnumerate{1}{
            \item \Propor[Simple]{1/2,6/12,3/5,10/20}
            \item \Propor[Simple]{2/3,6/9,30/45}
        }
        \item Essayez ensuite de justifier vos réponses sans calculatrice en expliquant votre raisonnement.
    \end{enumerate}
}

\bsec{Logique}
\bsubsec{Réciproque}

\slide{exo}{\bshrink
    \act{}{
        \begin{enumerate}
            \item « \Sialors{c'est un triangle}{c'est un polygones à trois sommets} » est une \key{proposition}.  
            Est-elle vraie ? Justifie ta réponse.  
            
            \item « \Sialors{c'est un triangle}{c'est un polygones à quatre sommets} » est une autre proposition.  
            Est-elle vraie ? Justifie ta réponse.  
            
            \item La première proposition est composée de deux parties :  
            \multiColItemize{1}{
                \item l'\key{antécédent} : « c'est un triangle »,
                \item le \key{conséquent} : « c'est un polygones à trois sommets ».
            }  
            On appelle \key{réciproque} d'une proposition la phrase qu'on obtient en inversant l'antécédent et le conséquent.  
            Écris la réciproque de la première proposition.
            Est-elle vraie ? \saveenumi
        \end{enumerate}
    }[\href{http://www.mathsaharry.com/aw/52.pdf}{Math à Harry}]
}

\slide{exo}{
    \begin{enumerate} \loadenumi[act]
        \item Les propositions suivantes sont-elles vraies ?
        Écris leurs réciproques et détermine si elles sont vraies.
        \multiColEnumerate{1}{
            \item \Sialors{c'est un carré}{c'est un rectangle avec tous ses côtés égaux}
            \item \Sialors{il peut pondre des œufs}{c'est un oiseau}
            \item \Sialors{c'est un rectangle}{c'est un quadrilatère dont tous les opposés sont parallèles} 
            \item \Sialors{c'est un rectangle}{c'est un carré}
            \item \Sialors{$AB = BC$}{$B$ est le milieu de $[AC]$}
        } 
    \end{enumerate}
}

\slide{cr}{\bsmall
    \ssec
    \ssubsec

    \df{}{
        On appelle \key{réciproque}
        d'une proposition :
        {« \Sialors{$A$}{$B$} »}
        ; la proposition :
        {« \Sialors{$B$}{$A$} »}.
    }
    
    \rmk{}{
        Une proposition peut être vraie sans que sa réciproque le soit, et inversement.
    }

    \expl{}{
        La proposition « \Sialors{$[AB]$ et $[CD]$ ne se coupent pas}{$[AB]$ et $[CD]$ sont parallèles}»
        est \awsr{fausse}.\\
        Sa réciproque: \awsr{« \Sialors{$[AB]$ et $[CD]$ sont parallèle}{$[AB]$ et $[CD]$ ne se coupent pas}»}
        est \awsr{vrai}.
    }
}

\scn{Découvrir des notions de logiques ; la contraposée}

\slide{qf}{\calculator \\ Completer les tableaux suivants :
    \multiColEnumerate{3}{
        \item \begin{center}
            \Propor[Simple,
            Math,
            Stretch=1.25,%
            ]{6/5,\awsr{\np{2.4}}/2}
        \end{center}
        \item \begin{center}
            \Propor[Simple,
            Math,
            Stretch=1.25,%
            ]{\np{237.6}/\awsr{66},\np{46.8}/13}
        \end{center}
        \item \begin{center}
            \Propor[Simple,
            Math,
            Stretch=1.25,%
            ]{\awsr{12}/18,-3/-4.5}
        \end{center}
    }
}

\bsubsec{Contraposée}

\slide{exo}{
    \ssubsec
    \act{}{
        On appelle \key{contraposée} d'une proposition
        {« \Sialors{$A$}{$B$} »}
        la proposition obtenue en écrivant :  
        {« \Sialors{non $B$}{non $A$} »}.
        \begin{enumerate}
            \item La contraposée de la proposition : «\Sialors{c'est un triangle}{il a trois côtés}».
            est donc «\Sialors{il n'a pas trois côtés}{ce n'est pas un triangle}» Est-elle vraie ? \saveenumi
        \end{enumerate}  
    }
}

\slide{exo}{
    \begin{enumerate} \loadenumi[act]
        \item Les propositions suivantes sont-elles vraies ? Écris leurs contraposées et dis si elles sont vraies :  
        \multiColEnumerate{1}{ 
            \item \Sialors{$x=7$}{$x$ est un nombre premier}
            \item \Sialors{c'est un nombre positif}{il est strictement inférieur à zéro}
            \item \Sialors{il est à Issy-les-Moulineaux}{il n'est pas en Espagne}  
            \item \Sialors{$AB=BC$}{$B$ est le milieu de $[AC]$} 
        }
        \item Que remarques-tu ?
    \end{enumerate}
}

\slide{cr}{
    \df{}{
        On appelle \key{contraposée}
        d'une proposition ;
        {« \Sialors{$A$}{$B$} »} ;
        la proposition : 
        {« \Sialors{non $B$}{non $A$} »}.  
    }

    \rmk{}{
        Une proposition et sa contraposée sont toujours soit toutes les deux vraies,
        soit toutes les deux fausses.
    }

    \expl{}{
        La proposition « \Sialors{c'est un triangle est équilatéral}{ses trois côtés sont égaux} »  
        est \awsr{vraie}.  
        Sa contraposée: \awsr{« \Sialors{ses trois côtés ne sont pas égaux}{ce n'est pas un triangle équilatéral} »},
        est \awsr{également vraie}.
    }
}

\scn{Appliquer ses connaissances sur la contraposée et la réciproque au Théorème de Pythagore}

\slide{qf}{
    \calculator
    \exo{}{
        Sachant que huit briques de masse identique pèsent \Masse{13.6},
        calcule la masse de six de ces briques.
    }[\afa{4e}[6]]
}

\bsec{Contraposée du Théorème de Pythagore}

\slide{exo}{\bshrink
    \act{}{\bvspace{-1cm}
        \def\crossSize{0.15}
        \ctikz[1]{
            \draw[gray!40] (-1,-4) rectangle (11,3);
            \draw [penciline,thick] (8.2,1.76)-- (5.28,-1.84);
            \draw [penciline,thick] (5.28,-1.84)-- (9.5,-0.3);
            \draw [penciline,thick] (9.5,-0.3)-- (8.2,1.76);
            \draw [penciline,thick] (2.76,-2.24)-- (3.84,-0.56);
            \draw [penciline,thick] (3.84,-0.56)-- (0.34,1.22);
            \draw [penciline,thick] (0.34,1.22)-- (2.76,-2.24);
            \drawPoint{D}{3.84}{-0.56}
            \drawPoint{E}{0.34}{1.22}
            \drawPoint{F}{2.76}{-2.24}
            \drawPoint{G}{8.20}{1.76}
            \drawPoint{H}{9.50}{-0.30}
            \drawPoint{I}{5.28}{-1.84}
            \draw (3.48,-1.38) node[anchor=north west] {4cm};
            \draw (0.92,-0.74) node[anchor=north west] {5cm};
            \draw (2.06,0.92) node[anchor=north west] {6cm};
            \draw (5.72,0.4) node[anchor=north west] {5cm};
            \draw (7.62,-1.22) node[anchor=north west] {12cm};
            \draw (9,1.3) node[anchor=north west] {13cm};
        }
    }
}

\slide{exo}{
    Pour les triangles $EDF$ et $GHI$, répondre aux questions suivantes :
    \begin{enumerate}\setItemColor{act}
        \item Ce triangle respect-ils l'égalité de Pythagore ?
        \item En utilisant vos connaissances sur le théorème de Pythagore,
        pouvez-vous conclure si le triangle est rectangle ou non ?
        Justifiez votre réponse en précisant l'outil de logique utilisé.
    \end{enumerate}
}

\slide{cr}{
    \ssec
    \bvspace{-0.5cm}
    \ctr{du théorème de Pythagore}{
        Dans un triangle $ABC$.
        \Sialors{$AB^2 \neq AC^2 + BC^2$}{le triangle $ABC$ n'est pas rectangle en $C$}
    }
    \bvspace{-0.75cm}
    \expl{}{
        Soit $NEZ$ est un triangle tel que : $NE = \Lg{8}, EZ = \Lg{16}\et ZN = \Lg{14}$.\\
        Démontrer que le triangle n'est pas rectangle.\\
        \awsr[5]{
            \begin{itemize}
                \item $NE$ est le plus long côté, il sagirait donc de l'hypothénus si le triangle était rectangle.
                \item D'une part :$EZ^2 = 16^2 = 256$
                \item D'autre part : $NE^2 + ZN^2 = 8^2 + 14^2 = 64 + 196= 260$
                \item Alors : $EZ^2 \neq NE^2 + ZN^2$
                \item D'après la contraposée du théorème de Pythagore le triangle $NEZ$ n'est pas rectangle.
            \end{itemize}
        }
    }
}

\bookSlide{29p431}[12cm]

\ifArticle{\awsr[0]{
    \begin{itemize}
        \item On commence par convertir les longueurs en \Lg{}. (On aurait également pu les convertir toutes en \Lg[dm]{}).
        \multiColItemize{3}{
            \item $PU = \Lg[dm]{3.6} = \Lg{36}$ \item $PF = \Lg[dm]{5.5} = \Lg{55}$ \item $UF = \Lg{42}$
        }
        \item \Pythagore[Reciproque,Unite=cm]{PUF}{55}{36}{42} %PF PU UF 
        % \item $NE$ est le plus long côté, il sagirait donc de l'hypothénus si le triangle était rectangle.
        % \item D'une part :$EZ^2 = 16^2 = 256$
        % \item D'autre part : $NE^2 + ZN^2 = 8^2 + 14^2 = 64 + 196= 260$
        % \item Alors : $EZ^2 \neq NE^2 + ZN^2$
        % \item D'après la contraposée du théorème de Pythagore le triangle $NEZ$ n'est pas rectangle.
    \end{itemize}
}}

\scn{Démontrer la réciproque du théorème de Pythagore sur un cas particulier}

\def\repere{%
    \tkzInit[xmin=0,xmax=4,ymin=0,ymax=3]
    \tkzGrid[sub,color=gradeColor!50!white,subxstep=1,subystep=0.5]        
    \tkzLabelX[step=1]
    \tkzLabelY[step=1]
    \tkzDrawY[label={Prix (en \euro)}, above , step=0.5]
}
\slide{qf}{\bsmall
    \hspace{-1.25cm}\wideFrame[4em]{%
        \def\size{0.65}\def\crossWidth{0.25mm}%
        \multiColItemize{3}{
            \item[]\ctikz[\size]{
                \repere
                \tkzDrawX[label={Poire}, below right, step=1]
                \drawPoint{}{1}{1}[Red]
                \drawPoint{}{2}{1.5}[Red]
                \drawPoint{}{3}{2.5}[Red]
                \drawPoint{}{4}{3}[Red]
            }
            \item[]\ctikz[\size]{
                \repere
                \tkzDrawX[label={Orange}, below right, step=1]
                \drawPoint{}{1}{0.5}[Red]
                \drawPoint{}{2}{1}[Red]
                \drawPoint{}{3}{1.5}[Red]
                \drawPoint{}{4}{2}[Red]
            }
            \item[]\ctikz[\size]{
                \repere
                \tkzDrawX[label={Kiwi}, below right, step=1]
                \drawPoint{}{1}{1}[Red]
                \drawPoint{}{2}{1.5}[Red]
                \drawPoint{}{3}{2}[Red]
                \drawPoint{}{4}{2.5}[Red]
            }
        }
        \vspace{-0.5cm}
        \begin{enumerate}
            \item Le prix du lot de chacun des trois fruits est-il proportionnel à la quantité achetée ?
            \item Que remarquez-vous sur la disposition des points dans le cas de proportionnalité ?
        \end{enumerate}
    }
}

\bsec{Réciproque du théorème de Pythagore}

\slide{exo}{
    \act{}{
        On va démontrer pour un exemple que la réciproque du théorème de Pythagore est vraie.
        Soit $ABC$ un triangle tel que : $AB = \Lg{5}; AC = \Lg{4}; BC = \Lg{5}$.
        \begin{enumerate}
            \item Tracer le triangle $ABC$.
            \item Verifier si $AB^2 = AC^2 + BC^2$. Est-ce que le triangle $ABC$ peut être rectangle?
            \item Contruire une droite perpendiculaire à la droite $(BC)$.
            \item Placer un point $D$ sur cette perpendiculaire tel que $DC = AC$
            et $D$ soit placé à «l'opposé» de $A$. \saveenumi
        \end{enumerate}
    }[\href{https://clg-monnet-briis.ac-versailles.fr/La-reciproque-du-theoreme-de-Pythagore}{Collège Jean Monnet}]
}

\slide{exo}{
    \begin{enumerate} \loadenumi[act]
        \item Quelle est la nature du triangle $BCD$?
        \item Calculer la longueur $BD$.
        \item Comparer les triangles $ABC$ et $BCD$.
        \item Conclure sur la nature du triangle $ABC$.
    \end{enumerate}
    De manière similaire on pourait prouver que la réciproque du théorème de Pythagore est toujours vraie.
}

\scn{Démontrer qu'un triangle est rectangle}

\slide{qf}{
    Les grandeurs suivantes sont-elles proportionnelles entre elles ?
    \begin{enumerate}
        \item Le périmètre d'un carré et la longueur de son côté.
        \item L'aire d'un carré et la longueur de son côté.
        \item Le périmètre d'un rectangle et sa largeur (en supposant que la longueur reste fixe).
        \item Le périmètre d'un cercle et la longueur de son rayon.
    \end{enumerate}
    \awsr[0]{%
        Deux grandeurs sont proportionnelles si l'une peut être obtenue en multipliant l'autre par un coefficient constant,
        c'est le coefficient de proportionnalité.
        \begin{enumerate}
            \item Le périmètre d'un carré $P$ et la longueur de son côté $c$ sont proportionnelle car $P = 4 \times c$. Le coefficient de proportionnalité est $4$.  
            \item L'aire $A$ d'un carré n'est pas proportionnelle à la longueur $c$ de son côté, car $A = c^2$.
            La relation entre ces grandeurs n'est pas une multiplication par un coefficient constant.
            \item Le périmètre $P$ d'un rectangle et sa largeur $l$, avec une longueur $L$ fixe :  
            $P = 2L + 2l$. La relation entre $P$ et $l$ n'est pas proportionnelle, car $2L$ est un terme constant qui ne dépend pas de $l$.  
            \item Le périmètre $P$ d'un cercle est proportionnel à son rayon $r$, car $P = 2\pi \times r$. Le coefficient de proportionnalité est $2\pi$.  
        \end{enumerate}
    }
}

\slide{cr}{
    \ssec

    \bvspace{-0.5cm}

    \rcp{du théorème de Pythagore}{
        Dans un triangle ABC.
        \Sialors{$AB^2 = AC^2 + BC^2$}
        {le triangle $ABC$ est rectangle en $C$}
    }

    \bvspace{-0.75cm}

    \expl{}{
        Soit $CGT$ est un triangle tel que : $CG = \Lg{45}, GT = \Lg{28}\et TC = \Lg{53}$.\\
        Démontrer que le triangle $CGT$ est rectangle.\\
        \awsr[6]{
            \begin{itemize}
                \item $TC$ est le plus long côté, il sagirait donc de l'hypothénus si le triangle était rectangle.
                \item D'une part :$TC^2 = 53^2 = 256$
                \item D'autre part : $NE^2 + ZN^2 = 8^2 + 14^2 = 64 + 196= 260$
                \item Alors : $EZ^2 \neq NE^2 + ZN^2$
                \item D'après la réciproque du théorème de Pythagore le triangle $CGT$ est rectangle en $G$.
            \end{itemize}
        }
    }
}

\slide{qf}{
    \begin{enumerate}
        \item Sachant que la longueur $\mathcal{P}$ d'un cercle
        est proportionnelle à son rayon $r$
        avec un coefficient de proportionnalité $2\pi$.
        Donnez la formule permettant de calculer $\mathcal{P}$ en fonction de $r$.
        \item Sachant que la tension $U$ aux bornes d'une résistance
        est proportionnelle à l'intensité $I$ du courant qui la traverse
        avec un coefficient de proportionnalité égal à la valeur de la résistance $R$.
        Donnez la formule permettant de calculer $U$ en fonction de $I$.
    \end{enumerate}
}

\bookSlide{27p431,26p431,36p432}[7cm][2]

\bookSlide{35p432,52p435}[6cm][2]