% VARIABLES %%%
\setSeq{4}{Théorème de Pythagore - Contraposée et réciproque}
\setGrade{4e}
\def\imgPath{enseignement/4e/theoreme-de-pythagore/contraposee-et-reciproque/}

% \forPrint
% \setboolean{answer}{true}
%%

\obj{
    \item Comprendre les notions de réciproque et de contraposée.
    \item Utiliser la contraposée du Théorème de Pythagore pour montrer qu'un triangle n'est pas rectangle.
    \item Utiliser la réciproque du Théorème de Pythagore pour montrer qu'un triangle est rectangle.
    \item Déterminer si un triangle est rectangle ou non.
}

\obj{
    \item Reconnaitre sur un graphique une situation de proportionnalité ou de non proportionnalité.
    \item Calcule d'une quatrième proportionnelle.
    \item Utiliser une formule liant deux grandeurs dans une situation de proportionnalité.
    \item Résoudre des problèmes en utilisant la proportionnalité dans le cadre de la géométrie.
}[Flash]

\scn{Découvrir des notions de logiques; réciproque et contraposée}

\slide{qf}{\bvspace{-0.35cm}
    \begin{enumerate}
        \item Les tableaux suivants représentent-ils des situations de proportionnalité ?
        Utilisez une calculatrice pour vérifier vos hypothèses.
        \multiColEnumerate{1}{
            \item \Propor[Simple]{1/2,6/12,3/5,10/20}
            \item \Propor[Simple]{2/3,6/9,30/45}
        }
        \item Essayez ensuite de justifier vos réponses sans calculatrice en expliquant votre raisonnement.
    \end{enumerate}
}

\bsec{Logique}
\bsubsec{Réciproque}

\slide{exo}{\bshrink
    \act{}{
        \begin{enumerate}
            \item « \Sialors{c'est un triangle}{il a trois côtés} » est une \key{proposition}.  
            Est-elle vraie ? Justifie ta réponse.  
            
            \item « \Sialors{c'est un triangle}{il a quatre côtés} » est une autre proposition.  
            Est-elle vraie ? Justifie ta réponse.  
            
            \item La première proposition est composée de deux parties :  
            \multiColItemize{1}{
                \item l'\key{antécédent} : « c'est un triangle »,
                \item le \key{conséquent} : « il a trois côtés ».
            }  
            On appelle \key{réciproque} d'une proposition la phrase qu'on obtient en inversant l'antécédent et le conséquent.  
            Écris la réciproque de la première proposition.
            Est-elle vraie ? \saveenumi
        \end{enumerate}
    }[\href{http://www.mathsaharry.com/aw/52.pdf}{Math à Harry}]
}

\slide{exo}{
    \begin{enumerate} \loadenumi[act]
        \item Les propositions suivantes sont-elles vraies ?
        Écris leurs réciproques et détermine si elles sont vraies.
        \multiColEnumerate{1}{
            \item \Sialors{c'est un carré}{c'est un rectangle avec tous ses côtés égaux}
            \item \Sialors{il pond des œufs}{c'est un oiseau}
            \item \Sialors{c'est un rectangle}{c'est un quadrilatère dont tous les côtés sont parallèles} 
            \item \Sialors{c'est un rectangle}{c'est un carré}
            \item \Sialors{$AB = BC$}{$B$ est le milieu de $[AC]$}
        } 
    \end{enumerate}
}

\slide{cr}{\bsmall
    \ssec
    \ssubsec

    \df{}{
        On appelle \key{réciproque}
        d'une proposition :
        {« \Sialors{$A$}{$B$} »}
        ; la proposition :
        {« \Sialors{$B$}{$A$} »}.
    }
    
    \rmk{}{
        Une proposition peut être vraie sans que sa réciproque le soit, et inversement.
    }

    \expl{}{
        La proposition « \Sialors{$[AB]$ et $[CD]$ ne se coupent pas}{$[AB]$ et $[CD]$ sont parallèles}»
        est \bawsr{fausse}.\\
        Sa réciproque: \bawsr{« \Sialors{$[AB]$ et $[CD]$ sont parallèle}{$[AB]$ et $[CD]$ ne se coupent pas}»}
        est \bawsr{vrai}.
    }
}

\bsubsec{Contraposée}

\slide{exo}{
    \ssubsec
    \act{}{
        On appelle \key{contraposée} d'une proposition
        {« \Sialors{$A$}{$B$} »}
        la proposition obtenue en écrivant :  
        {« \Sialors{non $B$}{non $A$} »}.
        \begin{enumerate}
            \item La contraposée de la proposition : «\Sialors{c'est un triangle}{il a trois côtés}».
            est donc «\Sialors{il n'a pas trois cotés}{ce n'est pas un triangle}» Est-elle vraie ? \saveenumi
        \end{enumerate}  
    }
}

\slide{exo}{
    \begin{enumerate} \loadenumi[act]
        \item Les propositions suivantes sont-elles vraies ? Écris leurs contraposées et dis si elles sont vraies :  
        \multiColEnumerate{1}{ 
            \item \Sialors{$x=7$}{$x$ est un nombre premier}
            \item \Sialors{c'est un nombre positif}{il est strictement inférieur à zéro}
            \item \Sialors{il est à Issy-les-Moulineaux}{il n'est pas en Espagne}  
            \item \Sialors{$AB=BC$}{$B$ est le milieu de $[AC]$} 
        }
        \item Que remarques-tu ?
    \end{enumerate}
}

\slide{cr}{
    \df{}{
        On appelle \key{contraposée}
        d'une proposition ;
        {« \Sialors{$A$}{$B$} »} ;
        la proposition : 
        {« \Sialors{non $B$}{non $A$} »}.  
    }

    \rmk{}{
        Une proposition et sa contraposée sont toujours soit toutes les deux vraies,
        soit toutes les deux fausses.
    }

    \expl{}{
        La proposition « \Sialors{c'est un triangle est équilatéral}{ses trois côtés sont égaux} »  
        est \bawsr{vraie}.  
        Sa contraposée: \bawsr{« \Sialors{ses trois côtés ne sont pas égaux}{ce n'est pas un triangle équilatéral} »},
        est \bawsr{également vraie}.
    }
}

\bsec{Contraposée du Théorème de Pythagore}

\slide{cr}{
    \ssec
    \ctr{du théorème de Pythagore}{}
}

\bsec{Réciproque du Théorème de Pythagore}

\slide{cr}{
    \ssec
    \rcp{du théorème de Pythagore}{}
}