% VARIABLES %%%
\setSeq{4}{Théorème de Pythagore - Contraposé et réciproque}
\setGrade{4e}
\def\imgPath{enseignement/4e/theoreme-de-pythagore/contraposé-et-reciproque/}
% \setboolean{answer}{true}
%%

\obj{
    \item Utiliser la contraposé du Théorème de Pythagore pour montrer qu'un triangle n'est pas rectangle.
    \item Utiliser la réciproque du Théorème de Pythagore pour montrer qu'un triangle est rectangle.
    \item Déterminer si un triangle est rectangle ou non.
}

\obj{
    \item Reconnaitre sur un graphique une situation de proportionnalité ou de non proportionnalité.
    \item Calcule d'une quatrième proportionnelle.
    \item Utiliser une formule liant deux grandeurs dans une situation de proportionnalité.
    \item Résoudre des problèmes en utilisant la proportionnalité dans le cadre de la géométrie.
}

\bsec{Contraposé}
\bsec{Réciproque}