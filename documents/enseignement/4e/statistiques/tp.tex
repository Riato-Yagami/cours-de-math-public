\setGrade{4e}
\tp{Statistiques}
\def\iconPath{libreOffice/calc/}

\definecolor{cell}{HTML}{CCCCCC} % #CCCCCC

\NewDocumentCommand{\cell}{O{cellule}}{%
    \renewcommand\fbox{\fcolorbox{cell}{white}}%
    \fbox{\ttfamily\bfseries#1}%
}

\NewDocumentCommand{\ccalc}{O{cellule}}{%
    {\ttfamily\bfseries=#1}%
}
%%%

\hint{Enregistrer votre travail sous le nom : "classe-NOM-tp-statistiques.odt"}

\exo{}{
    Dans la feuille $1$, la plage de cellules \cell[B3:B34] contient les tailles des élèves d'une classe de $4^e$.

    \begin{enumerate} \item \textbf{Compléter les cellules} :
        Utilise les \ccalc[FORMULES()] suivantes pour remplir les cellules indiquées :
        \begin{itemize}
            \item \ccalc[NBVAL()] pour \cell[D3]
            \item \ccalc[MOYENNE()] pour \cell[D7]
            \item \ccalc[MIN()] pour \cell[D11]
            \item \ccalc[MAX()] pour \cell[E11]
        \end{itemize}
        \item \textbf{Expliquer les résultats} :
        Indique ce que représentent les valeurs affichées dans les cellules \cell[D3], \cell[D11] et \cell[E11].
        \item \textbf{Calculer l'étendue} :
        Remplis la cellule \cell[D15] avec une formule calculant l'étendue,
        c'est-à-dire la différence entre la plus grande et la plus petite taille.
        \item \textbf{Trouver la médiane} :
        En \cell[D19], utilise la formule \ccalc[MEDIANE()] pour calculer la médiane de la série.
        \item \textbf{Trier les données} :
        Mets les tailles des élèves dans l'ordre croissant en suivant ces étapes :
        \begin{itemize}
            \item Sélectionne la plage \cell[B3:B34].
            \item Accède à la barre de menus et choisis :
        \textbf{Données} $\rightarrow$ \textbf{Trier par ordre croissant} \icon{sort-increasing}.
        \end{itemize}
        \item \textbf{Analyser la médiane} :
        Après le tri, observe les données. Que représente la médiane ?
    \end{enumerate}
}

\exo{}{
    Dans la feuille $2$,
    vous trouverez la répartition des jeunes inscrits dans un centre de vacances,
    en fonction de leur âge. Complétez les étapes suivantes :

    \begin{enumerate}
        \item Complétez la cellule \cell[C13] pour obtenir le nombre total d'inscrits.
        \item Complétez la cellule \cell[E3] pour obtenir la somme des âges des enfants de 10 ans.
        \item Étendez la formule de la cellule \cell[E3] jusqu'à \cell[E10] pour obtenir les sommes des âges cumulés par tranche d'âge.
        \item Complétez la cellule \cell[E12] pour obtenir la somme totale des âges cumulés.
        \item Complétez la cellule \cell[E16] pour calculer l'âge moyen des jeunes inscrits.
        \item Construisez un diagramme en bâtons représentant la répartition des âges des jeunes :
        \begin{enumerate}
            \item Sélectionnez la plage \cell[B3:C10].
            \item Allez dans : Barre de menus $\rightarrow$ Insertion $\rightarrow$ Diagramme \icon{diagram}.
        \end{enumerate}
        % \item De la même manière, construisez un diagramme circulaire représentant les âges des jeunes. En plus des étapes précédentes, suivez les instructions suivantes :
        % \begin{itemize}
        %     \item Sélectionnez : Étapes $\rightarrow$ 1. Type de diagramme $\rightarrow$ Secteur \icon{sector}.
        % \end{itemize}
    \end{enumerate}
    
}
