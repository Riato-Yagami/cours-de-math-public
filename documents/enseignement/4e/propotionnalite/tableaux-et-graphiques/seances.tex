% VARIABLES %%%
\setSeq{5}{Proportionnalité - Tableaux et graphiques}
\setGrade{4e}
\def\imgPath{enseignement/4e/theoreme-de-pythagore/contraposé-et-reciproque/}
% \setboolean{answer}{true}
%%

% \obj{
%     \item Reconnaitre sur un graphique une situation de proportionnalité ou de non proportionnalité.
%     \item Calcule d'une quatrième proportionnelle.
%     \item Utiliser une formule liant deux grandeurs dans une situation de proportionnalité.
%     \item Résoudre des problèmes en utilisant la proportionnalité dans le cadre de la géométrie.
% }

\pr{Coefficient de proportionnalité}{
    Dans un tableau de proportionnalité,
    on passe d'une ligne à l'autre en multipliant par le coefficient de proportionnalité.
}

\pr{}{
    Dans un graphique \sialors{tous les points sont alignés avec l'origine}
    {on une situation de proportionnalité}.
}

\pr{Egalité des produits en croix}{Pour $a,b,c,d$ des nombres.
    Dans le tableau de proportionnalité:
    \begin{center}
        \propTable{a}{c}{b}{d}
    \end{center}
    On a l'égalité: $a \times d = b \times c$. 
}

\rmk{Egalité des quotients}{
    On a aussi
    \begin{align*}
        \frac{a}{b} = \frac{c}{d}
    \end{align*}
}

\mthd{Calcul 4e proportionnelle}{
    Si l'on connait 3 valeurs par exemple $b,c,d$.
    On peut utiliser l'égalité des produits en croix pour calculer la 4e proportionnelle $a$.\\
    En effet:
    \begin{align*}
        a \times d &= b \times c\\
        \ialors \palt{2}{\frac{a \times d}{d}} &= \palt{2}{\frac{b \times c}{d}}
        \ialors a &= \palt{3}{\frac{b \times c}{d}}
    \end{align*}
}