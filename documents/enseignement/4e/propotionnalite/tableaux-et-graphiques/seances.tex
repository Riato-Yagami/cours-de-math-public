% VARIABLES %%%
\setSeq{5}{Proportionnalité - Tableaux et graphiques}
\setGrade{4e}
\def\imgPath{enseignement/4e/theoreme-de-pythagore/contraposé-et-reciproque/}
% \setboolean{answer}{true}
\def\ym{\href{https://www.maths-et-tiques.fr/telech/19Proport1.pdf}{Yvan Monka}}
% \forStudent
% \setboolean{demonstration}{false}
%%

\def\cp{coefficient de proportionnalité}
% \obj{
%     \item Reconnaitre sur un graphique une situation de proportionnalité ou de non proportionnalité.
%     \item Calcule d'une quatrième proportionnelle.
%     \item Utiliser une formule liant deux grandeurs dans une situation de proportionnalité.
%     \item Résoudre des problèmes en utilisant la proportionnalité dans le cadre de la géométrie.
% }

\renewcommand{\arraystretch}{1.5}

\avspace{0.1cm}

\bsec{Grandeurs proportionnelles}

\df{Grandeurs proportionnelles}{%
    Deux grandeurs sont dites \key{proportionnelles}
    lorsque les valeurs de l'une sont obtenues en multipliant les valeurs de l'autre par un même nombre non nul,
    appelé \key{coefficient de proportionnalité}.
}

\expl{}{
    \begin{tabular}{|>{\bfseries}c|*{4}{c|}} % Colonne en gras pour la première colonne
        \hline
        \rowcolor{gray!15} 
        Grandeur 1 & coté & coté & rayon & tension \\ \hline
        Grandeur 2  & périmètre du carré & aire du carré & périmètre du cercle & intensité \\ \hline
        coefficient?  & \awsr{$4$} & \awsr{non} & \awsr{$2\pi$} & \awsr{$R$} \\ \hline
    \end{tabular}
    % \begin{itemize}
    %     \item La longueur du coté d'un carré et sont périmètre avec \cp{} : $4$ car $\mathcal{P} = 4 \times c$.
    %     \item 
    % \end{itemize}
}

\bsec{Tableau de proportionnalité}

% \pr{Coefficient de proportionnalité}{
%     \Sialors{on est dans un tableau de proportionnalité}
%     {on peu passer d'une ligne à l'autre en multipliant par un \key{\cp}}
% }

Pour un tableau de proportionnalité : \propTable{a}{c}{b}{d} avec $a,b,c,d$ des nombres.

\pr{}{
    On peu passer d'une ligne à l'autre en multipliant par un \cp.
}

\expl{}{\vspace{-0.75cm}
    \multiColEnumerate{2}{
        \item \Propor[Stretch=1.5, Simple]{1/2.5,2/5,5/12.5}
        \FlechesPD{1}{2}{$\times\awsr{2}$}
        \FlechesPG{2}{1}{$\div\awsr{2}$}
        \item \Propor[Stretch=1.5, Simple]{120/12,3/0.3}
        \FlechesPD{1}{2}{$\times\awsr{\frac{1}{10}}$}
        \FlechesPG{2}{1}{$\div\awsr{\frac{1}{10}}$}
    }
}

\pr{Egalité des produits en croix}{
    On a l'égalité: $a \times d = b \times c$.
}

\newpage

\expl{}{
    Utiliser l'égalité des produits en croix pour vérifier si on a bien proportionnalité.
    \vspace{-0.75cm}\multiColEnumerate{2}{
        \item \Propor[Stretch=1.5, Simple]{15/36,1.2/2}
        \item \Propor[Stretch=1.5, Simple]{6/1.5,3/0.5}
    }\vspace{-0.75cm}
    \awsr[5]{
        \begin{enumerate}
            \item $15 \times 2 = 30$ et $36 \times \np{1.2} = 30$
            alors on égalité des produits en croix : $15 \times 2 = 36 \times \np{1.2}$,
            il sagit donc d'un tableau de proportionnalité.
            \item $6 \times \np{0.5} = 3$ et $\np{1.5} \times 3 = 4.5$
            alors on n'a pas égalité des produits en croix : $6 \times \np{0.5} \neq \np{1.5} \times 3 = 4.5$,
            il ne sagit donc pas d'un tableau de proportionnalité.
        \end{enumerate}
    }
}

\cor{Egalité des quotients}{
    On a aussi : $\frac{a}{b} = \frac{c}{d}$
}

\demo{}{
    On a : $\frac{a}{b} = \frac{ a\times d}{b \times d} = \frac{ \awsr{b \times c} }{b \times d}$ d'après l'égalité des produits en croix.\\
    Or $\frac{ b \times c }{b \times d} = \awsr{\frac{c}{d}}$.
    Alors $\frac{a}{b} = \awsr{\frac{c}{d}}.$
}[\href{https://pedagogie.ac-toulouse.fr/mathematiques/system/files/2023-03/demonstration_produits_en_croix.pdf}{Académie de Toulouse}]

\expl{}{
    Utiliser l'égalité des quotients pour vérifier si on a bien proportionnalité.
    \vspace{-0.75cm}\multiColEnumerate{2}{
        \item \Propor[Stretch=1.5, Simple]{10/12,5/6,20/23}
        \item \Propor[Stretch=1.5, Simple]{2/3,4/6,6/9}
    }\vspace{-0.75cm}
    \awsr[5]{
        \begin{enumerate}
            \item $\frac{10}{12} = \frac{20}{24} \neq \frac{20}{23}$
            alors on n'a pas égalité des quotients,
            il ne sagit donc pas d'un tableau de proportionnalité.
            \item $\frac{2}{3} = \frac{4}{6} = \frac{6}{9}$
            alors on égalité des quotients,
            il sagit donc d'un tableau de proportionnalité.
        \end{enumerate}
    }
}

% \ctr{}{
%     \Sialors{$\frac{a}{b} \neq \frac{c}{d}$}{l'égalité des quotients n'est pas respécté et on a pas proportionnalité}
% }

\mthd{Calcul 4e proportionnelle}{
    Si l'on connait 3 valeurs par exemple $b,c,d$.\\
    On peu calculer $a$ avec l'égalité $a = \awsr{\frac{b \times c}{d}}$.
}

\demo{}{
    En partant de l'égalité des produits en croix : $a \times d = b \times c$.\\
    Alors $a$ est le nombre qui multiplié par \awsr{$d$} done \awsr{$b \times c$}.\\
    D'après la définition du quotient : $a = \awsr{\frac{b \times c}{d}}$.
}

\expl{Compléter les tableaux de proportionnalité suivants}{
    \def\cW{2.5cm}
    \multiColEnumerate{2}{
        \item \begin{tabular}{|C{\cW}|C{1cm}|}
            \hline
            \np{9.6} & 3 \\ \hline
            \awsr{$\frac{\np{9.6}\times2}{3} = \np{6.4}$} & 2 \\ \hline
        \end{tabular}
        \item \begin{tabular}{|C{1cm}|*{2}{C{\cW}|}}
            \hline
            3 & 5 & \awsr{$\frac{\np{21.7}\times5}{7} = \np{15.5}$} \\ \hline
            \np{4.2} & \awsr{$\frac{\np{4.2}\times6}{3} = 7$} & \np{21.7} \\ \hline
        \end{tabular}
    }
}

\bsec{Représentation graphique}

\pr{Représentation graphique}{
    Sur un graphique, une situation de proportionnalité est représentée par des points alignés
    avec l'origine.
}[\ym]

\expl{}{
    Chaque graphique suivant représente-t-il une situation de proportionnalité ?
    \def\repere{%
        \tkzInit[xmin=0,xmax=8,ymin=0,ymax=8]
        \tkzGrid[sub,color=gradeColor!50!white,subxstep=1,subystep=1]        
        \tkzLabelX[step=2]
        \tkzLabelY[step=2]
        \tkzDrawY[step=1]
        \tkzDrawX[step=1]
    }
    \def\size{0.55}\def\crossWidth{0.25mm}
    \vspace{-0.75cm}
    \multiColItemize{3}{
        \item[]\ctikz[\size]{
            \repere
            \node at (4,9) {\cir[gradeColor]{1}};
            \drawPoint{}{2}{2.4}[Red]
            \drawPoint{}{4}{4.8}[Red]
            \drawPoint{}{6.5}{7.8}[Red]
            \ifthenelse{\boolean{answer}}{\draw[answer] (-1,-1.2) -- (7.5,9);}{}
        }
        \item[]\ctikz[\size]{
            \node at (4,9) {\cir[gradeColor]{2}};
            \repere
            \drawPoint{}{1}{1.3}[Red]
            \drawPoint{}{3}{3}[Red]
            \drawPoint{}{6}{7.8}[Red]
            \ifthenelse{\boolean{answer}}{\draw[answer] (-1,-1.3) -- (9/1.3,9);}{}
        }
        \item[]\ctikz[\size]{
            \repere
            \node at (4,9) {\cir[gradeColor]{3}};
            \drawPoint{}{1}{2}[Red]
            \drawPoint{}{4}{5}[Red]
            \drawPoint{}{6}{7}[Red]
            \drawPoint{}{7}{8}[Red]
            \ifthenelse{\boolean{answer}}{\draw[answer] (-1,0) -- (8,9);}{}
        }
    }
    \awsr[6]{Seuls les points du graphique \cir[gradeColor]{1} sont alignés avec l'origine.
    Ainsi, parmi les trois graphiques,
    c'est le seul qui représente une situation de proportionnalité.}
}



% \slide{qf}{
%     Les situations présentées dans ces tableaux sont-elles proportionnelles ?
%     \multiColEnumerate{2}{
%         \item \begin{center}
%             \Propor[Simple,
%             Math,
%             Stretch=1.25,%
%             ]{12/3,16/4,40/10}
%         \end{center}
%         \item \begin{center}
%             \Propor[Simple,
%             Math,
%             Stretch=1.25,%
%             ]{15/5,9/3,20/6}
%         \end{center}
%     }
% }

% \slide{qf}{\calculator \\ Completer les tableaux suivants :
%     \multiColEnumerate{3}{
%         \item \begin{center}
%             \Propor[Simple,
%             Math,
%             Stretch=1.25,%
%             ]{6/5,\awsr{\np{2.4}}/2}
%         \end{center}
%         \item \begin{center}
%             \Propor[Simple,
%             Math,
%             Stretch=1.25,%
%             ]{\np{237.6}/\awsr{66},\np{46.8}/13}
%         \end{center}
%         \item \begin{center}
%             \Propor[Simple,
%             Math,
%             Stretch=1.25,%
%             ]{\awsr{12}/18,-3/-4.5}
%         \end{center}
%     }
% }

% \slide{qf}{
%     \nullsubsec{}{
%         Sachant que huit briques de masse identique pèsent 13,6 kg, calcule la masse de six de ces
%         briques.
%     }[\afa{4e}[6]]
% }

% \slide{qf}{
%     \nullsubsec{}{
%         \begin{enumerate}
%             \item Sachant que la longueur $\mathcal{P}$ d'un cercle
%             est proportionnelle à son rayon $r$
%             avec un \cp $2\pi$.
%             Donnez la formule permettant de calculer $\mathcal{P}$ en fonction de $r$.
%             \item Sachant que la tension $U$ aux bornes d'une résistance
%             est proportionnelle à l'intensité $I$ du courant qui la traverse
%             avec un \cp égal à la valeur de la résistance $R$.
%             Donnez la formule permettant de calculer $U$ en fonction de $I$.
%         \end{enumerate}
%     }
% }
