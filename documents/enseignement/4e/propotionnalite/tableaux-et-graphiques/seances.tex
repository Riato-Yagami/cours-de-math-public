% VARIABLES %%%
\setSeq{5}{Proportionnalité - Tableaux et graphiques}
\setGrade{4e}
\def\imgPath{enseignement/4e/theoreme-de-pythagore/contraposé-et-reciproque/}
% \setboolean{answer}{true}
\def\ym{\href{https://www.maths-et-tiques.fr/telech/19Proport1.pdf}{Yvan Monka}}
%%

% \obj{
%     \item Reconnaitre sur un graphique une situation de proportionnalité ou de non proportionnalité.
%     \item Calcule d'une quatrième proportionnelle.
%     \item Utiliser une formule liant deux grandeurs dans une situation de proportionnalité.
%     \item Résoudre des problèmes en utilisant la proportionnalité dans le cadre de la géométrie.
% }

\pr{Coefficient de proportionnalité}{
    Dans un tableau de proportionnalité,
    on passe d'une ligne à l'autre en multipliant par le coefficient de proportionnalité.
}

\pr{Représentation graphique}{
    Sur un graphique, une situation de proportionnalité est représentée par des points alignés
    avec l'origine.
}[\ym]

\pr{Egalité des produits en croix}{Pour $a,b,c,d$ des nombres.
    Dans le tableau de proportionnalité:
    \begin{center}
        \propTable{a}{c}{b}{d}
    \end{center}
    On a l'égalité: $a \times d = b \times c$. 
}

\rmk{Egalité des quotients}{
    On a aussi
    \begin{align*}
        \frac{a}{b} = \frac{c}{d}
    \end{align*}
}

\mthd{Calcul 4e proportionnelle}{
    Si l'on connait 3 valeurs par exemple $b,c,d$.
    On peut utiliser l'égalité des produits en croix pour calculer la 4e proportionnelle $a$.\\
    En effet:
    \begin{align*}
        a \times d &= b \times c\\
        \ialors \palt{2}{\frac{a \times d}{d}} &= \palt{2}{\frac{b \times c}{d}}
        \ialors a &= \palt{3}{\frac{b \times c}{d}}
    \end{align*}
}

\slide{qf}{
    Les situations présentées dans ces tableaux sont-elles proportionnelles ?
    \multiColEnumerate{2}{
        \item \begin{center}
            \Propor[Simple,
            Math,
            Stretch=1.25,%
            ]{12/3,16/4,40/10}
        \end{center}
        \item \begin{center}
            \Propor[Simple,
            Math,
            Stretch=1.25,%
            ]{15/5,9/3,20/6}
        \end{center}
    }
}

\slide{qf}{\calculator \\ Completer les tableaux suivants :
    \multiColEnumerate{3}{
        \item \begin{center}
            \Propor[Simple,
            Math,
            Stretch=1.25,%
            ]{6/5,\bawsr{\num{2.4}}/2}
        \end{center}
        \item \begin{center}
            \Propor[Simple,
            Math,
            Stretch=1.25,%
            ]{\num{237.6}/\bawsr{66},\num{46.8}/13}
        \end{center}
        \item \begin{center}
            \Propor[Simple,
            Math,
            Stretch=1.25,%
            ]{\bawsr{12}/18,-3/-4.5}
        \end{center}
    }
}

\slide{qf}{
    \nullsubsec{}{
        Sachant que huit briques de masse identique pèsent 13,6 kg, calcule la masse de six de ces
        briques.
    }[\afa{4e}[6]]
}

\slide{qf}{
    \nullsubsec{}{
        \begin{enumerate}
            \item Sachant que la longueur $\mathcal{P}$ d'un cercle
            est proportionnelle à son rayon $r$
            avec un coefficient de proportionnalité $2\pi$.
            Donnez la formule permettant de calculer $\mathcal{P}$ en fonction de $r$.
            \item Sachant que la tension $U$ aux bornes d'une résistance
            est proportionnelle à l'intensité $I$ du courant qui la traverse
            avec un coefficient de proportionnalité égal à la valeur de la résistance $R$.
            Donnez la formule permettant de calculer $U$ en fonction de $I$.
        \end{enumerate}
    }
}
