% VARIABLES %%%
\def\authors{\jules \ et \href{http://www.cellulegeometrie.eu/documents/pub/pub_14.pdf}{Académie de Reims}}
\date{2024}
\def\theme{TP : Test de divisibilité}

\def\imgPath{enseignement/4e/divisibilite-et-nombres-premiers/}
\def\imgExtension{.png}
\thispagestyle{assignment}

\setscratch{scale=.75}
\setscratch{print=true}
\setscratch{fill blocks=true}
%%

% \disableAnimation
% \shortAnimation
% \firstSlide
\vspace*{-1cm}
\hint{
    \begin{itemize}
        \item Répondre aux questions sur une feuille.
        \item Enregistrer la production avec le nom : "NOM-Prenom-tp-divisibilite.sb3"
    \end{itemize}
}
\vspace*{-0.75cm}

\section{Division euclidienne}
\begin{enumerate}
    \item Poser la division euclidienne de $152$ par $8$.
    \item $152$ est-il divisible par $8$ ? Pourquoi ?
    \item Sur \Scratch ecrire et éxecuter le programme suivant:
    \begin{center}\begin{scratch}
        \blockinit{quand \greenflag est cliqué}
        \blocklook{dire \ovaloperator{\ovalnum{152} modulo \ovalnum{8}}}
    \end{scratch}\end{center}
    \item Qu'est-ce que calcule l'opérateur modulo ?
\end{enumerate}

\vspace*{-0.25cm}
\section{Teste de divisibilité}

\def\nb{\ovalvariable{nombre} }
\def\dv{\ovalvariable{diviseur} }
\begin{enumerate}
    \item Créer deux variables \nb et \dv depuis l'onglet \icon{scratch/variables} Variables.
    \item A quoi doit être égale \ovaloperator{\nb modulo \dv}
    pour que \nb soit divisible par \dv ?
    \item Ecrire un programme qui:
    \begin{itemize}
        \item Donne une valeur à deux variables \nb et \dv.
        \item Dit « \nb est divisible par \dv» ou « \nb n'est pas divisible par \dv» en fonction des cas.
    \end{itemize}
    \hint{Blocs utiles (à utiliser plusieurs fois pour certains):\\  
        \booloperator{\ovalnum{} = \ovalnum{ }}
        \begin{scratch}
            \blockifelse{si \boolempty[3em] alors}{ }{ }
        \end{scratch}
        \begin{scratch}
            \blockvariable{mettre \selectmenu{ma variable} à \ovalnum{ }}
        \end{scratch}
        \ovaloperator{regrouper \ovalnum{} et \ovalnum{}}
        \begin{scratch}
            \blocklook{dire \ovalnum{}}
        \end{scratch}
        \ovaloperator{\ovalnum{} modulo \ovalnum{}}
    }
\end{enumerate}

\vspace*{-0.25cm}
\section{Liste des diviseurs}

\def\dvs{\ovallist{diviseurs} }
\begin{enumerate}
    \item Pour connaître tous les diviseurs d'un nombre,
    quel est le plus petit nombre à tester?
    Et le plus grand?
    \item Créer une nouvelle liste \dvs depuis l'onglet \icon{scratch/variables} Variables.
    \item Ecrire un programme qui:
    \begin{itemize}
        \item Teste tous les diviseurs possibles pour un \nb un par un, en commencant par le plus petit et finissant par le plus grand.
        \item Ajoute les diviseurs de \nb à \dvs.
    \end{itemize}
    \hint{ Blocs utiles (en plus des blocs précédents):\\
        \begin{scratch}
            \blockinfloop{répéter jusqu'à ce que \boolempty[3em]}{ }
        \end{scratch}
        \begin{scratch}
            \blockif{si \boolempty[3em] alors}{ }
        \end{scratch}
        \begin{scratch}
            \blocklist{ajouter \ovalvariable{ma variable} à \selectmenu{ma liste}}
        \end{scratch}
        \booloperator{\ovalnum{} > \ovalnum{ }}
        \begin{scratch}
            \blockvariable{ajouter \ovalnum{1} à \selectmenu{ma variable}}
        \end{scratch}
        \begin{scratch}
            \blocklist{supprimer tous les éléments de la liste \selectmenu{ma liste}}
        \end{scratch}
    }
\end{enumerate}