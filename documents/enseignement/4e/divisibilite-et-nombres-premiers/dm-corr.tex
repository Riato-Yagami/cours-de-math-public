% VARIABLES %%%
\setTitle{Correction - Devoir Maison - Séquence 1}
\setgrade{4e}
\def\imgPath{enseignement/4e/divisibilite-et-nombres-premiers/}
%%

\exo{20p17}{
    \imgp{mi-c4/exo-20p17}[6cm]

    \imgp{dm-20p17}[8cm]
}

\exo{45p19}{
    \imgp{mi-c4/exo-45p19}[6cm]
    \begin{enumerate}
        \item\begin{itemize}
            \item Pour trouver les diviseurs de $6$, on commence par le décomposer en facteurs premiers.
            \item $6 = 2 \times 3$
            \item Les diviseurs de $6$ sont donc $1;2;3$ et $6$.
            \item Or $1+2+3 = 6$.
            \item $6$ est donc bien un nombre parfait.
        \end{itemize}
        \item\begin{itemize}
            \item Pour trouver les diviseurs de $28$, on commence par le décomposer en facteurs premiers.
            \item $28 = 2 \times 2 \times 7$
            \item Les diviseurs de $6$ sont donc $1;2;7;2\times2 = 4;2\times7 = 14$ et $28$.
            \item Or $1+2+7+4+14 = 28$.
            \item $28$ est donc bien un nombre parfait.
        \end{itemize}
    \end{enumerate}
}

\exo{56p20}{
    \imgp{mi-c4/exo-56p20}[6cm]
    \begin{itemize}
        \item Le prénom Annabelle comporte $9$ lettres.
        \item Or la division euclidienne de $1000$ par $9$ donne : $1000 = 9 \times 111 + 1$.
        \item Alors à la $999$ lettre Annabelle aura écrit son prénom pour la $111^e$ fois.
        \item Et la $1000^e$ lettre sera donc un $A$.
    \end{itemize}
}