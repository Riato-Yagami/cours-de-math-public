% VARIABLES %%%
\def\authors{\jules}
% \date{\today}
\def\longTitle{Divisibilité et nombres premiers}
\setcounter{seq}{1}
\bseq{\longTitle}

\setboolean{showRef}{false}

\def\my{Myriade 5e}
% \newcommand{\myl}[1]{\href{#1}{\my}}

\def\imgPath{enseignement/4e/divisibilite-et-nombres-premiers/}
\def\imgExtension{.png}
%%

% Yvan Monka : https://www.maths-et-tiques.fr/telech/19Divi-np.pdf
% Crible d'Ératosthène : https://fr.wikipedia.org/wiki/Crible_d%27%C3%89ratosth%C3%A8ne
% Juniper Green : https://fr.wikipedia.org/wiki/Juniper_Green_(jeu)

\qf{
    {$2$ divise : \choice{$4$} \choice{$5$} \choice{$6$}, \choicea{1} et \choicea{3}},
    {$6$ divise : \choice{$18$} \choice{$12$} \choice{$2$}, \choicea{1} et \choicea{2}},
    {$30$ est divisible par : \choice{$3$} \choice{$10$} \choice{$5$} \choice{$4$}, \choicea{1}{,} \choicea{2} et \choicea{3}},
    {Quels nombres sont premiers ? : \choice{$6$} \choice{$13$} \choice{$2$} \choice{$1$}, \choicea{2} et \choicea{3}}%
}

\bsec{Nombres premiers}
\bsubsec{Définition}

\slide{COURS}{
    \sseq\ssec
    \df{}{
        Un entier est \key{premier} s'il a exactement deux diviseurs différents 1 et lui même.
    }
}

\slide{}{
    \act{Crible d'Ératosthène}{
        \imgp{crible-d-eratosthene}[5cm]
    }
}

\bsec{Décomposition en facteurs premiers}