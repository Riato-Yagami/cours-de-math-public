% VARIABLES %%%
\def\authors{\jules}
% \date{\today}
\def\longTitle{Divisibilité et nombres premiers}
\setcounter{seq}{1}
\bseq{\longTitle}

\setgrade{4e}
% \newcommand{\myl}[1]{\href{#1}{\my}}

\def\imgPath{enseignement/4e/divisibilite-et-nombres-premiers/}
\def\imgExtension{.png}
%%

% Yvan Monka : https://www.maths-et-tiques.fr/telech/19Divi-np.pdf
% Crible d'Ératosthène : https://fr.wikipedia.org/wiki/Crible_d%27%C3%89ratosth%C3%A8ne
% Juniper Green : https://fr.wikipedia.org/wiki/Juniper_Green_(jeu)

% \disableAnimation
% \shortAnimation
% \firstSlide

\avspace{0.1cm}

\obj{
    \item Déterminer la liste des nombres premiers inférieurs à 100.
    \item Décomposition d'un nombre entier en produit de facteurs premiers
    \item Modéliser et résoudre des problèmes simples mettant en jeu les notions de divisibilité et de nombre premier.
}

\scn{Divisibilité}{}

\qf{
    {$2$ divise : \choice{$4$} \choice{$5$} \choice{$6$}, \choicea{1} et \choicea{3}},
    {$30$ est divisible par : \choice{$3$} \choice{$10$} \choice{$5$} \choice{$4$}, \choicea{1}{,} \choicea{2} et \choicea{3}},
    {Donner le résultat de la division euclienne de $31$ par $4$, $31 = 4 \times 7 + 3$},
    {Quels nombres sont premiers ? : \choice{$6$} \choice{$13$} \choice{$2$} \choice{$1$}, \choicea{2} et \choicea{3}}%
}

\bsec{Divisibilité}
\slide{COURS}{
    \sseq\ssec
    \df{}{
        On dit qu'un entier $a$ est \key{divisible} par un entier $b$ s'il existe un entier $k$ tel que $a = \palt{2}{b \times k}  $.\\
        On dit alors que $a$ est un \key{multiple} de $b$, et que $b$ \key{divise} $a$ ou est un \key{diviseur} de $a$.}[\href{https://fr.wikipedia.org/wiki/Divisibilité}{Wikipédia}]
    \bvspace{-1cm}
}

\slide{}{
    \expl{}{\bvspace{-0.25cm}
        \begin{enumerate}
            \item $14 = \palt{3}{2} \times \palt{3}{7} \palt{4}{= 1 \times 14}$ \\
            alors 14 est divisible par $\palt{3}{2 ; 7} \palt{4}{; 1 \et 14}$
            \item $124 = \palt{5}{1 \times 124 = 2 \times 62 = 4 \times 31}$\\
            $\palt{5}{1;2;4;31;62;124$}\; sont les diviseurs de 124.
            \item $57 = 8 \times \palt{6}{7+1}$
            alors le reste de la division euclienne de $57$ par $8$ est $\palt{6}{1}$.
            57 est donc \palt{6}{non divisible} par $8$.
        \end{enumerate}
    }
}

\def\imgPrefix{mi-c4/exo-}

\scn{Critères de divisibilité}{}

\qf{
    {$6$ divise : \choice{$18$} \choice{$12$} \choice{$2$}, \choicea{1} et \choicea{2}},
    {Donner la liste des diviseurs de 12 : , 1;2;6;12},
    {Donner la liste des diviseurs de 39 : , 1;3;13;39}%
}

\exoSlide{21p17,22p17,23p17}[7cm][2][\mi]

\slide{COURS}{
    \pr{Critères de divisibilité}{
        Un nombre entier est divisible par :
        \begin{itemize}
            \setlength\itemsep{-0.1em}
            \item $2$ si son chiffre des unités est pair.
            \item $5$ si son chiffre des unités est $0$ ou $5$.
            \item $10$ si son chiffre des unités est $0$.
            \item $3$ si la somme de ces chiffres est un multiple de $3$.
            \item $9$ si la somme de ces chiffres est un multiple de $9$.
            \item $4$ si le nombre formé par ses deux derniers chiffres est multiple de $4$.
        \end{itemize}
    }
}

\slide{}{
    \expl{}{
        Donner les diviseurs de $1944$ inférieurs à $10$.
        \palt{2}{
            \begin{itemize}
                \item $1$ divise tous les entiers.
                \item $4$ est pair donc 1944 est divisible par $2$
                \item $44 = 4 \times 11$ donc 1944 est divisible par $4$.
                \item $1+9+4+4 = \pcalc{1+9+4+4}$ et $1+8 = 9$ donc $1944$ est divisible par $9$ et $3$.
                \item $1944 = 277 \times 7 + \ncalc{1944-277*7}$ donc $1944$ n'est pas divisible par $7$.
                \item $1944 \div 8 = \ncalc{1944/2} \div 4 = \ncalc{1944/4} \div 2 $
                donc $1944$ est divisible par $8$, car $1944$ est divisible par $2$, $3$ fois de suite.
            \end{itemize}
            Les diviseurs de $1944$ inférieurs à $10$ sont donc $1;2;3;4;8;9$.
        }
    }
}

\scn{Problème de divisibilité}{travail de groupes}

\def\imgPrefix{mi-c4/qf-}
\qfSlide{
    \imgp{2abcdp12}[9cm]
    \imgp{5p12}[9cm]
    \imgp{6p12}[9cm]
}

\slide{EXERCICES}{
    \exo{}{
        Établir la liste des diviseurs communs de $189$ et $126$.
    }
}

\def\imgPrefix{mi-c4/exo-}
\exoSlide{42p19,48p19,65p21}[7cm][2][\mi]

% \avspace{0.25cm}
% \ifArticle{\TODO{? Scéance critères de divisibilité}}

\scn{Nombres premiers}{}

\def\imgPrefix{mi-c4/qf-}
\qfSlide{
    \imgp{3p12}[9cm]
    \imgp{7p12}[9cm]
}

\bsec{Nombres premiers}
\bsubsec{Définition}

\def\imgPrefix{}
\slide{}{
    \bvspace{-0.4cm}
    \act{4 p13 - Crible d'Ératosthène}{
        \ifBeamer{\vspace{-0.5cm}\small}
        \dividePage{%
            \imgp{crible-d-eratosthene}[5cm]%
        }{%
            \begin{enumerate}
                \item Barrer le 1.
                \item Entourer 2,
                puis barrer tous les multiples de 2 autres que 2.
                \item Entourer le premier nombre ni entouré ni barré,
                puis barrer tous ses multiples autres que lui-même.
                \item Répéter la consigne jusqu'à atteindre le premier nombre premier plus grand que 10.
                \item Quelle particularité possèdent les vingt-cinq nombres entourés ?
            \end{enumerate}
        }[0.35]%
        % \avspace{0.25cm}
        % \bvspace{-0.25cm}
        % \begin{enumerate}
        %     \setcounter{enumi}{3}
        %     \item Quelle particularité possèdent les vingt-cinq nombres entourés ?
        % \end{enumerate}
    }[\mi]
}

\slide{COURS}{
    \ssec
    \df{}{
        Un nombre entier est dit \key{premier} s'il a exactement deux diviseurs différents: 1 et lui-même.
    }
    % \bvspace{-1cm}
    \ifBeamer{\small}
    \bvspace{-1cm}
    \rmk{}{%
    \bvspace{-0.25cm}
        \begin{itemize}
            \item $1$ n'est pas premier. Il n'a qu'un seul diviseur, lui-même.
            \item $2$, le plus petit nombre premier est le seul nombre premier pair.
        \end{itemize}
    }
    \pr{}{%
        Il existe une infinité de nombres premiers.
    }
}

% \slide{EXERCICES}{
%     \exo{27;28;29}
% }

\def\imgPrefix{mi-c4/exo-}
\exoSlide{27p17,28p17,29p17}[7cm][2][\mi]

\scn{Décomposition en facteurs premiers}{}

\def\imgPrefix{mi-c4/qf-}
\qfSlide{
    \imgp{5p17}[7cm]
}

\bsubsec{Décomposition en facteurs premiers}

\def\imgPrefix{}

\slide{EXERCICES}{
    \bvspace{-0.75cm}
    \act{Representation chromatique des nombres}{\bvspace{-1cm}
    \wideFrame[6em]{
        \dividePage{\imgp{representation-chromatique-des-nombres}[5.75cm]}{
            \begin{enumerate}
                \item Écrire sous la forme d'un produit de facteurs premiers.
                \begin{align*}
                    4 &= \hole \qquad 10 = \hole \qquad 14 = \hole \\
                    15 &= \hole \qquad 20 = \hole
                \end{align*}
                \item Expliquer comment les nombres sont représentés sur ce tableau.
            \end{enumerate}
        }[0.35]
    }
    }[\href{https://www.monclasseurdemaths.fr/tc/représentation-chromatique-des-nombres/}
    {Mon classeur de maths}]
}

\slide{COURS}{
    \ssubsec
    \pr{}{
        Tout nombre entier strictement supérieur à 1 peut se décomposer en produit de facteurs premiers.
    }
    \bvspace{-0.5cm}
    \expl{}{Donner la décomposition en facteurs premiers de :\bvspace{-1cm}%
        \begin{align*}%
            1100 &= \palt{2}{%
                2 \times 550\\
                &=2 \times 2 \times 275\\
                &=2 \times 2 \times 5 \times 275\\
                &=2 \times 2 \times 5 \times 55\\
                &=2 \times 2 \times 5 \times 5 \times 11
            }
        \end{align*}%
    }
}

\slide{cr}{
    \mthd{Décomposition en facteurs premiers}{
        On prend la liste des nombres premiers dans l'ordre,
        puis tester la division par chaque nombre premier,
        un par un.
    }
}

\slide{qf}{
    Donner la décomposition en facteurs premiers de :
    \multiColEnumerate{3}{
        \item 6 \item 28 \item 5 \item 24 \item 66 \item 52
    }
}

\def\imgPrefix{mi-c4/exo-}

\slide{EXERCICES}{
    \exo{}{
        \begin{enumerate}
            \item Trouver la décomposition en facteurs premiers de $70$ et $105$.
            \item En déduire les diviseurs communs de $70$ et $105$.
        \end{enumerate}
    }
    \bvspace{-0.5cm}
    \exo{}{
        Un fleuriste doit réaliser des bouquets tous identiques.
        Il dispose pour cela de 434 roses et 620 tulipes. Quelles sont toutes les compositions de bouquets possibles ?
    }[\href{https://eduscol.education.fr/document/14056/download}
    {Attendues de fin d'année de 4e}]
}

\exoSlide{20p17,45p19,56p20}[7cm][2][\mi][dm]

\scn{Simplification de fractions}{}

\slide{qf}{
    Trouver les diviseurs communs de $60$ et $42$
}

\slide{cr}{
    \app{Simplification de fractions}{}
    \bvspace{-1cm}
    \expl{}{
        Simplifier la fraction :
        \begin{align*}
            \frac{140}{294} &= \palt{2}{
                \frac{2 \times 2 \times 5 \times 7}{2 \times 3 \times 7 \times 7}\\
                &= \frac{2 \times 5}{3 \times 7}\\
                &= \frac{10}{21}
            }
        \end{align*}
    }[\href{https://math-coaching.com/fiche/simplifier-fraction-decomposition-facteurs-premiers-130}{Math Coaching}]
}

\slide{cr}{
    \mthd{Simplification de fractions}{
        \begin{enumerate}
            \item On décompose le numérateur et dénominateurs en facteurs premiers.
            \item On supprime les facteurs communs.
            \item On multiplie ensemble les facteurs restants au numérateur et dénominateurs.
        \end{enumerate}
    }
}

\exoSlide{58p54}[10cm][1][\mi]