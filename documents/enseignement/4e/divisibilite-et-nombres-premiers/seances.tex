% VARIABLES %%%
\def\authors{\jules}
% \date{\today}
\def\longTitle{Divisibilité et nombres premiers}
\setcounter{seq}{1}
\bseq{\longTitle}

\setgrade{4e}
% \newcommand{\myl}[1]{\href{#1}{\my}}

\def\imgPath{enseignement/4e/divisibilite-et-nombres-premiers/}
\def\imgExtension{.png}
%%

% Yvan Monka : https://www.maths-et-tiques.fr/telech/19Divi-np.pdf
% Crible d'Ératosthène : https://fr.wikipedia.org/wiki/Crible_d%27%C3%89ratosth%C3%A8ne
% Juniper Green : https://fr.wikipedia.org/wiki/Juniper_Green_(jeu)

% \disableAnimation
% \shortAnimation
% \firstSlide

\avspace{0.1cm}
\scn{Divisibilité}{}

\qf{
    {$2$ divise : \choice{$4$} \choice{$5$} \choice{$6$}, \choicea{1} et \choicea{3}},
    {$30$ est divisible par : \choice{$3$} \choice{$10$} \choice{$5$} \choice{$4$}, \choicea{1}{,} \choicea{2} et \choicea{3}},
    {Donner le résultat de la division euclienne de $31$ par $4$, $31 = 4 \times 7 + 3$},
    {Quels nombres sont premiers ? : \choice{$6$} \choice{$13$} \choice{$2$} \choice{$1$}, \choicea{2} et \choicea{3}}%
}

\bsec{Divisibilité}
\slide{COURS}{
    \sseq\ssec
    \df{}{
        On dit qu'un entier $a$ est \key{divisible} par un entier $b$ s'il existe un entier $k$ tel que $a = \palt{2}{b \times k}  $.\\
        On dit alors que $a$ est un \key{multiple} de $b$, et que $b$ \key{divise} $a$ ou est un \key{diviseur} de $a$.}[\href{https://fr.wikipedia.org/wiki/Divisibilité}{Wikipédia}]
    \bvspace{-1cm}
}

\slide{}{
    \expl{}{\bvspace{-0.25cm}
        \begin{enumerate}
            \item $14 = \palt{3}{2} \times \palt{3}{7} \palt{4}{= 1 \times 14}$ \\
            alors 14 est divisible par $\palt{3}{2 ; 7} \palt{4}{; 1 \et 14}$
            \item $124 = \palt{5}{1 \times 124 = 2 \times 62 = 4 \times 31}$\\
            $\palt{5}{1;2;4;31;62;124$}\; sont les diviseurs de 124.
            \item $57 = 8 \times \palt{6}{7+1}$
            alors le reste de la division euclienne de $57$ par $8$ est $\palt{6}{1}$.
            57 est donc \palt{6}{non divisible} par $8$.
        \end{enumerate}
    }
}

\def\imgPrefix{mi-c4/exo-}

\scn{Critères de divisibilité}{}

\qf{
    {$6$ divise : \choice{$18$} \choice{$12$} \choice{$2$}, \choicea{1} et \choicea{2}},
    {Donner la liste des diviseurs de 12 : , 1;2;6;12},
    {Donner la liste des diviseurs de 39 : , 1;3;13;39}%
}

\exoSlide{21p17,22p17,23p17}[7cm][2][\mi]

\slide{COURS}{
    \pr{Critères de divisibilité}{
        Un nombre entier est divisible par :
        \begin{itemize}
            \setlength\itemsep{-0.1em}
            \item $2$ si son chiffre des unités est paire.
            \item $5$ si son chiffre des unités est $0$ ou $5$.
            \item $10$ si son chiffre des unités est $0$.
            \item $3$ si la somme de ces chiffres est un multiple de $3$.
            \item $9$ si la somme de ces chiffres est un multiple de $9$.
            \item $4$ si le nombre formé par ses deux derniers chiffres est multiple de $4$.
        \end{itemize}
    }
}

\slide{}{
    \expl{}{
        Donner les diviseurs de $1944$ inférieurs à $10$.
        \palt{2}{
            \begin{itemize}
                \item $1$ divises tous les entiers.
                \item $4$ est paire donc 1944 est divisible par $2$
                \item $44 = 4 \times 11$ donc 1944 est divisible par $4$.
                \item $1+9+4+4 = \pcalc{1+9+4+4}$ et $1+8 = 9$ donc $1944$ est divisible par $9$ et $3$.
                \item $1944 = 277 \times 7 + \ncalc{1944-277*7}$ donc $1944$ n'est pas divisible par $7$.
                \item $1944 \div 8 = \ncalc{1944/2} \div 4 = \ncalc{1944/4} \div 2 $
                donc $1944$ est divisible par $8$, car $1944$ est divisible par $2$, $3$ fois de suite.
            \end{itemize}
            Les diviseurs de $1944$ inférieurs à $10$ sont donc $1;2;3;4;8;9$.
        }
    }
}

\scn{Problème de divisibilité}{travail de groupes}

\def\imgPrefix{mi-c4/qf-}
\qfSlide{
    \imgp{2abcdp12}[9cm]
    \imgp{5p12}[9cm]
    \imgp{6p12}[9cm]
}

\slide{EXERCICES}{
    \exo{}{
        Établir la liste des diviseurs communs de $189$ et $126$.
    }
}

\def\imgPrefix{mi-c4/exo-}
\exoSlide{42p19,48p19,65p21}[7cm][2][\mi]

% \avspace{0.25cm}
% \ifArticle{\TODO{? Scéance critères de divisibilité}}

\scn{Nombres premiers}{}

\def\imgPrefix{mi-c4/qf-}
\qfSlide{
    \imgp{3p12}[9cm]
    \imgp{7p12}[9cm]
}

\bsec{Nombres premiers}
\bsubsec{Définition}

\def\imgPrefix{}
\slide{}{
    \bvspace{-0.4cm}
    \act{4 p13 - Crible d'Ératosthène}{
        \bvspace{-1.10cm}
        \dividePage{%
            \imgp{crible-d-eratosthene}[5cm]%
        }{%
            \begin{enumerate}
                \item Entourer 2,
                puis barrer tous les multiples de 2 autres que 2.
                \item Entourer le premier nombre ni entouré ni barré,
                puis barrer tous ses multiples autres que lui-même.
                \item Répéter la consigne de la question précédente jusqu'à avoir barré ou entouré tous les nombres.
            \end{enumerate}
        }[0.35]%
        \avspace{0.25cm}
        \bvspace{-0.25cm}
        \begin{enumerate}
            \setcounter{enumi}{3}
            \item Quelle particularité possèdent les vingt-cinq nombres entourés ?
        \end{enumerate}
    }[\mi]
}

\slide{COURS}{
    \ssec
    \df{}{
        Un entier est \key{premier} s'il a exactement deux diviseurs différents 1 et lui-même.
    }
    % \bvspace{-1cm}
    \ifBeamer{\small}
    \rmk{}{%
    \bvspace{-0.25cm}
        \begin{itemize}
            \item $1$ n'est pas premier. Il n'a qu'un seul diviseur, lui-même.
            \item $2$, le plus petit nombre premier est le seul nombre premier paire.
            \item Il existe une infinité de nombres premiers.
        \end{itemize}
    }
}

% \slide{EXERCICES}{
%     \exo{27;28;29}
% }

\def\imgPrefix{mi-c4/exo-}
\exoSlide{27p17,28p17,29p17}[7cm][2][\mi]

\bsec{Décomposition en facteurs premiers}

\slide{}{
    \pr{}{
        Tout nombre non premier peut se décomposer en produit de facteurs premiers.
    }

    % \rmk{HP}{
    %     Cette décomposition est enfaite unique.
    % }
}