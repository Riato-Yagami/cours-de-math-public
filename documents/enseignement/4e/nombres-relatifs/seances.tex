% VARIABLES %%%
% \date{\today}
\setSeq{3}{Nombres Relatifs}
\setGrade{4e}
\def\imgPath{enseignement/4e/nombres-relatifs}

\def\ym{\href{https://www.maths-et-tiques.fr/telech/19Nomb_rela.pdf}{Yvan Monka}}
%%

\obj{
    \item Produits et des quotients de nombres relatifs
}

\scn{Découvrir les produits de nombres relatifs}

\slide{qf}{
    \begin{enumerate}
        \item Développer les expressions suivantes :
        \multiColEnumerate{2}{
            \item $(x + 2) \times 6 =$
            \item $y \times (a + (-6)) =$
            \item $k \times (a + b) =$
            \item $8,1 \times (10 + 2) =$
        }
        \item Réduire les expressions suivantes :
        \multiColEnumerate{2}{
            \item $4 \times x =$
            \item $y \times 3 =$
            \item $a \times b =$
            \item $- q \times l =$
        }
    \end{enumerate}
}

\slide{exo}{
    \act{}{
        Dans cette activité on cherche à trouvé,
        pour deux nombres positifs $x$ et $y$,
        à quoi est égale $x \times  (- y)$ et $ (- x) \times (- y)$.
        \begin{enumerate}
            \item On commence par s'interesser au résultat de $3 \times (-5)$.
            \begin{enumerate}
                \item Combien donne $3 \times (5 + (-5))$
                \item Développer le produit $3 \times (5 + (-5))$
                \item D'après le ${\color{\currentColor}a)}$, à combien est égale la forme développer de l'expression précédente ?
                \item Combien donne $3 \times 5$ ?
                \item Combien donne alors $3 \times (-5)$?
            \end{enumerate}
            \item On s'interesse maintenant au résultat de $x \times (-y)$.\\
            Reproduit le raisonnement de ${\color{\currentColor}a)}$ à ${\color{\currentColor}e)}$ en substituant $3$ par $x$ et $5$ par $y$.
            \item Que donne le produit d'un nombre positif par un nombre négatif?
            \item On s'interesse maintenant au résultat de $-3 \times (-5)$.\\
            Reproduit le raisonnement de ${\color{\currentColor}a)}$ à ${\color{\currentColor}e)}$ en substituant $3$ par $-3$.
            \item On s'interesse maintenant au résultat de $-x \times (-y)$.\\
            Reproduit le raisonnement de ${\color{\currentColor}a)}$ à ${\color{\currentColor}e)}$ en substituant $3$ par $-x$ et $5$ par $y$.
            \item Que donne le produit d'un nombre négatif par un nombre négatif?
        \end{enumerate}
    }
}

\slide{cr}{
    \pr{}{
        Le \key{produit} de deux nombres :
        \begin{itemize}
            \item de \key{même signes} est \palt{2}{\key{positif}}.
            \item de \key{signe opposés} est \palt{2}{\key{négatifs}}.
        \end{itemize}
    }

    \expl{}{
        \multiColEnumerate{2}{
            \item $2 \times 7 = \palt{2}{14}$
            \item $2 \times (-7) = \palt{2}{-14}$
            \item $-2 \times 7 = \palt{2}{-14}$
            \item $-2 \times (-7) = \palt{2}{14}$
        }
    }
}