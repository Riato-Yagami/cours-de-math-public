% VARIABLES %%%
% \date{\today}
\setSeq{3}{Nombres Relatifs}
\setGrade{4e}
\def\imgPath{enseignement/4e/nombres-relatifs/}

% \print

\def\ym{\href{https://www.maths-et-tiques.fr/telech/19Nomb_rela.pdf}{Yvan Monka}}
\def\caPrefix{4e-entrainement.2-}
%%

\obj{
    \item Appliquer la règle des signes pour les produits et quotients de nombres relatifs.
    \item Déterminer le signe d'un produit de plusieurs facteurs relatifs.
    \item Déterminer le carré et le cube d'un nombre relatif.
    \item Trouver les antécédents du carré d'un nombre donné.
}

\scn{Tournois \icon{RELATIvs/logo}}

\bsec{Produits et quotients de nombres relatifs}
\bsubsec{Règle des signes}

\scn{Démontration des propriétés sur les produits de nombres relatifs}

\slide{qf}{
    \begin{enumerate}
        \item Développer les expressions suivantes :
        \multiColEnumerate{2}{
            \item $(x + 2) \times 6 =$
            \item $y \times (a + (-6)) =$
            \item $k \times (a + b) =$
            \item $8,1 \times (10 + 2) =$
        }
        \item Réduire les expressions suivantes :
        \multiColEnumerate{2}{
            \item $4 \times x =$
            \item $y \times 3 =$
            \item $a \times b =$
            \item $- q \times l =$
        }
    \end{enumerate}
}

\slide{exo}{\bsmall\bvspace{-0.7cm}
    \act{}{
        Dans cette activité on cherche à trouvé,
        pour deux nombres positifs $x$ et $y$,
        à quoi est égale $x \times  (- y)$ et $ (- x) \times (- y)$.
        \begin{enumerate}
            \item On commence par s'interesser au résultat de $3 \times (-5)$.
            \begin{enumerate}
                \item Combien donne $3 \times (5 + (-5))$
                \item Développer le produit $3 \times (5 + (-5))$
                \item D'après le ${\color{\currentColor}a)}$, à combien est égale la forme développer de l'expression précédente ?
                \item Combien donne $3 \times 5$ ?
                \item Combien donne alors $3 \times (-5)$?
            \end{enumerate}\saveenumi{2}
        \end{enumerate}
    }
}

\slide{exo}{\bsmall
    \begin{enumerate}\setItemColor{RoyalBlue}\loadenumi
        \item On s'interesse maintenant au résultat de $x \times (-y)$.\\
        Reproduit le raisonnement de ${\color{\currentColor}a)}$ à ${\color{\currentColor}e)}$ en substituant $3$ par $x$ et $5$ par $y$.
        \item Que donne le produit d'un nombre positif par un nombre négatif?
        \item On s'interesse maintenant au résultat de $-3 \times (-5)$.\\
        Reproduit le raisonnement de ${\color{\currentColor}a)}$ à ${\color{\currentColor}e)}$ en substituant $3$ par $-3$.
        \item On s'interesse maintenant au résultat de $-x \times (-y)$.\\
        Reproduit le raisonnement de ${\color{\currentColor}a)}$ à ${\color{\currentColor}e)}$ en substituant $3$ par $-x$ et $5$ par $y$.
        \item Que donne le produit d'un nombre négatif par un nombre négatif?
    \end{enumerate}
    
}

\slide{cr}{
    \sseq\ssec\ssubsec
    \pr{}{
        Le \key{produit} ou \key{quotient} de deux nombres :
        \begin{itemize}
            \item de \key{même signes} est \bawsr{\key{positif}}.
            \item de \key{signe opposés} est \bawsr{\key{négatifs}}.
        \end{itemize}
    }
}

\slide{cr}{
    \expl{}{
        \multiColEnumerate{2}{
            \item $-3 \times (-8) = \bawsr{24}$
            \item $-8 \div (-6) = \bawsr{\frac{-8}{-6} = \frac{8}{6} = \frac{4}{3} \approx \numprint{1.3333}}$
            \item $-3 \times 5 = \bawsr{-15}$
            \item $5 \times (-4) = \bawsr{-20}$
            \item $-4 \div 2 = \bawsr{\frac{-4}{2} = -\frac{4}{2} = -\frac{2}{1} = -2}$
            \item $4 \div (-5) = \bawsr{-\frac{4}{5} = -0.8}$
        }
    }

    \bvspace{-0.5cm}

    \rmk{}{
        On a pour $a$ est $b$ deux nombres.
        \begin{enumerate}
            \item $a \times (-b) = (-a) \times b = - (a \times b)$
            \item $\frac{-a}{b} = \frac{a}{-b} = -\frac{a}{b}$
        \end{enumerate}
    }
}

\bookSlide{29p87,27p87,35p88,38p88}[6.25cm][2]

\scn{Déterminer le signe d'un produits de relatifs à plusieurs facteurs}

% \def\caPrefix{4e-entrainement.2-}
% \caSlide{22-23-24}

% \slide{qf}{
%     \renewcommand{\arraystretch}{1.75}
%     \begin{center}
%         \begin{tabular}{| M{0.1\linewidth} | M{0.5\linewidth} | M{0.4\linewidth} |}
%             \hline
%             22 & Le périmètre d'un rectangle de longueur $14,6$ et de largeur $5,5$ est : &
%             \\\hline
%             23 & $7-9-(-4) =$ &
%             \\\hline
%             24 & $-13+10\times0,5 =$ &
%             \\\hline
%         \end{tabular}
%     \end{center}
% }

\slide{qf}{
    \nullsubsec{}{
        Soit $y$ un nombre négatif différent de 0.\\
        Indiquer le signe de chacune de ces expressions:
        \multiColEnumerate{2}{
            \item $A = -4 - y$
            \item $B = y \times y$
            \item $C = y + y + y$
            \item $D = y ^ 3$
            \item $E = -3 \times y$
            \item $F = -3 + y$
        }
    }[\rpmc[184]]
}

\bsubsec{Corollaires}


\slide{exo}{\bshrink
    \act{}{\noCalculator
        \begin{enumerate}
            \item Calculer le \key{résultat} des produits suivants :
            \multiColEnumerate{2}{
                \item $3 \times (-6) \times (-2)$
                \item $-1 \times (-3) \times (-1)$
                \item $-2 \times 1 \times (-1) \times (-6)$
                \item $-1 \times (-1) \times (-1) \times (-1) \times (-1)$
            }
            \item Déterminer le \key{signe} des produits suivants :
            \multiColEnumerate{1}{
                \item $-15198 \times 1231 \times (-1,6)$
                \item $-1,465884 \times (-3) \times (-1) \times (-3) \times (-\frac{1}{654})$
                \item $51,65465 \times (-1) \times (-\pi) \times (-6)$
            }\saveenumi{2}
        \end{enumerate}
    }
}

\slide{exo}{
    \begin{enumerate}\setItemColor{RoyalBlue}\loadenumi
        \item Déterminer le signe d'un produit de :
        \multiColEnumerate{1}{
            \item $10$ facteurs égaux à $-1$.
            \item $11$ facteurs égaux à $-1$.
            \item $1541$ facteurs égaux à $-1$.
            \item $8$ facteurs négatifs.
            \item $5$ facteurs négatifs et $31$ facteurs positifs.
            \item $5$ facteurs négatifs et $32$ facteurs positifs.
        }
        \item Émettre une conjecture sur le signe d'un produit de nombres relatifs.
    \end{enumerate}
}

\slide{cr}{
    \ssubsec
    \pr{}{
        Le résultat d'un \key{produit de relatifs} est :
        \begin{enumerate}
            \item \key{positif} s'il y a un nombre \key{pair} de facteurs négatifs dans le produit.
            \item \key{négatif} s'il y a un nombre \key{impair} de facteurs négatifs dans le produit.
        \end{enumerate}
    }
}

\slide{cr}{
    \expl{}{
        Déterminer le signe des produits suivants :
        \multiColEnumerate{1}{
            \item $-1 \times (-1,2) \times 2 \times (-3,36954) \times 2$
            \item $2 \times \frac{1}{2}  \times (-\pi) \times 1221 \times (-2)$
            \item $(-1,58)^2$
            \item $(-1654)^3$
            \item Un produit de $2024$ fracteurs égales à $-1$.
        }
    }
}

\slide{cr}{
    \rmk{}{
    \begin{enumerate}
        \item Le \key{carré} d'un nombre est toujours \key{positif}.
        \item Le \key{cube} d'un nombre \key{conserve le signe} de ce nombre.
    \end{enumerate}
}
}

\bookSlide{20p87,47p90}[6.25cm][2]

\scn{Résoudre une équation du second degré sans terme linéaire}

\caSlide{28-29-30}

\bsec{Antécédents du carré}

\slide{exo}{\bsmall
    \act{}{%
        \begin{enumerate}
            \item Trouver deux nombres distincts dont le carré est égal à $4$.
            \item Trouver deux nombres distincts dont le carré est égal à $1$.
            \item Trouver deux nombres distincts dont le carré est égal à $2$.
            \item Trouver deux nombres distincts dont le carré est égal à $x$.
        \end{enumerate}
    }
}

\slide{cr}{
    \pr{}{
        Pour $x$ un nombre positif.
        Les nombres dont le carré est égal à $x$ sont $\sqrt{x}$ ou $-\sqrt{x}$.
    }
}

\slide{cr}{
    \expl{}{
        \begin{enumerate}
            \item \sialors{$a^2 = 16$}{$a = \bawsr{4}$ ou $a = \bawsr{-4}$}
            \item \sialors{$b^2 = 1$}{$b = \bawsr{1}$ ou $b = \bawsr{-1}$}
            \item \sialors{$c^2 = 21$}{$c = \bawsr{\sqrt{21}}$ ou $c = \bawsr{-\sqrt{21}}$}
            \item Soit $x$ un nombre positif, \sialors{$d^2 = x$}{$d = \bawsr{\sqrt{x}}$ ou $d = \bawsr{-\sqrt{x}}$}
        \end{enumerate}
    }
}

\bookSlide{50p91}[5cm][1]
