%%%
\setGrade{6e}
\evaluation{2}

\setboolean{answer}{false}

\def\cwr{\href{https://college-willy-ronis.fr/maths/wp-content/uploads/2020/11/Chap-1-exercices-corriges-6eme.pdf}{College Willy Ronis}}
%%%
% \def\imgPath{enseignement/6e/}

\seqEvaluation{3}{Géométrie plane - distance}{
    Tracer un segment de longueur donnée.
    /2,
    Placer le milieu d'un segment de longueur donnée.
    /4,
    Reconnaître{,} nommer{,} décrire{,} reproduire un cercle.
    /4,
    Déterminer le plus court chemin entre un point et une droite.
    /0%
}

\seqEvaluation{4}{Nombres}{
    Utiliser et représenter les grands nombres entiers.
    /3,
    Utiliser la division euclidienne.
    / 2,
    Résoudre des problèmes relevant des structures additives et multiplicatives en mobilisant une ou plusieurs étapes de raisonnement.
    / 2,
    Organiser un calcul en une seule ligne{,} utilisant si nécessaire des parenthèses.
    / 2%
}

\seqEvaluation{}{Compétences générales}{
    Écrire ses calculs
    /1,
    Rédiger des phrases réponses
    /2
}

\newpage
\def\length{10}
\def\crossWidth{0.3mm}
\def\crossSize{0.15}
\exo{Triplet de cercles}{
    \begin{enumerate}
        \item Trace la demi droite $[AC)$.
        \item Place $B$ sur la demi-droite $[AC)$ tel que $AB = \Lg{\length}$.
        \item Marque le point $O$, milieu du segment $[AB]$.
        \item Trace le cercle de centre $O$ et de rayon $\Lg{\directlua{tex.print(math.floor(\length / 2))}}$.
        \item Trace les cercles de diamètres $[AO]$ et $[OB]$.
    \end{enumerate}
    \ctikz[1]{
        \draw[gradeColor!40] (-7,-3) rectangle (10,12);
        \ifthenelse{\boolean{answer}}{%
            \draw [thick,domain=-3.84:9] plot(\x,{(--7.6992--0.52*\x)/1.76});
            \draw [thick] (0.9550885402945024,4.6567307050870115) circle (5cm);
            \draw [thick] (-1.4424557298527487,3.948365352543506) circle (2.5cm);
            \draw [thick] (3.3526328104417535,5.365096057630517) circle (2.5cm);
            \drawPoint{B}{5.75}{6.07}
            \drawPoint{D}{0.96}{4.66}
            \drawPoint{E}{3.35}{5.37}
            \drawPoint{F}{-1.44}{3.95}
        }{}%
        \drawPoint{A}{-3.84}{3.24}
        \drawPoint{C}{-2.08}{3.76}
    }
}[\sesa{6}{2021}[9][85]]

\exo{}{Écrire en chiffres :
    \multiColEnumerate{1}{
        \item Neuf-mille-quatre-vingt-quinze. \bawsr{\num{9095}}
        \item $8\times1000+2\times10+1\times10000+7$ \bawsr{\num{10827}}
        \item 15 milliers et 1 234 unités. \bawsr{\num{16234}}
        \item Quatre-millions-sept-cent-quatre-vingts. \bawsr{\num{4 000 780}}
        \item 2 millions d'unités et 2 soixantaines. \bawsr{\num{2 000 120}}
    }
}[\cwr]

\exo{}{
    \begin{enumerate}
        \item Le 17 juin 2345 sera une date très particulière car elle s'écrire : 17 06 2345, c'est à dire avec huit chiffres tous
        différents. Quelle a été la dernière date à posséder cette propriété, c'est à dire à s'écrire sous la forme d'un nombre à
        huit chiffres tous différents ?
    \end{enumerate}
}[\cwr]

\answerFill[Réponses][
    \begin{enumerate}
        \item le 26 août 1987 qui donne : 26 07 1987
    \end{enumerate}
]