% VARIABLES %%%
% \date{\today}

\setSeq{2}{Organisation et gestion de données}

\setGrade{6e}
% \setImgPath
\def\imgPath{enseignement/6e/organisation-et-gestion-de-donnees/}
%%

\def\ym{\href{https://www.maths-et-tiques.fr/telech/19Tab_Graph.pdf}{Yvan Monka}}

\avspace{0.1cm}

\obj{
    \item Prélever des données numériques à partir de supports variés.
    \item Produire des tableaux, diagrammes et graphiques organisant des données numériques.
    % \item Exploiter et communiquer des résultats de mesures.
    \item Reconnaitre une situations de proportionnalité/situations qui ne sont pas de proportionnalité
}

\scn{Tableaux}{}

\bsec{Tableaux}
% \bsubsec{Tableaux de données}

\def\imgPrefix{dim-6e/qf-}
\exoSlide{1-2p96}[10cm][1][\dim][qf]
\def\imgPrefix{}

\def\imgPrefix{dim-6e/act-}
\slide{exo}{
    \bvspace{-0.75cm}
    \act{1p98}{\bvspace{-0.75cm}\imgp{1p98}[11.5cm]}[\dim]
}

\slide{cr}{
    \sseq\ssec
    \vc{}{
        Un \key{tableau} permet de rassembler et d'organiser des données pour les lire plus facilement.
    }
}

\slide{cr}{
    \expl{}{
        Au collège de la Paix,
        les enfants ont le choix entre 3 LV2 :
        italien, allemand ou espagnol.\\

        En 6eA, il y a 25 élèves. 12 ont choisi espagnol, 6 allemand et les autres italien.

        En 6eB, 13 élèves ont choisi espagnol et 5 élèves allemand.

        Dans ces deux classes, 12 élèves ont choisi italien.\\

        Présenter ces données dans un tableau à double entrée.
    }[\href{https://eduscol.education.fr/document/14014/download}{Attendues 6e}]
}

\newcommand{\last}[1]{\color{red}\textbf{#1}}

\def\imgPrefix{}
\slide{cr}{\bvspace{-0.5cm}
    \mthd{Construire un tableau}{\bvspace{-0.75cm}
        \begin{enumerate}
            \item On réalise un tableau à double entrée avec les données de l'énoncé et on ajoute une colonne et une ligne «total». 
            \item On le complète avec les données de l'énoncé.
            \item On finit de compléter le tableau en effectuant les calculs.
        \end{enumerate}
        \begin{center}
            \begin{tabular}{|c|c|c|c|c|}
                \hline
                & Espagnol & Allemand & Italien & Total\\
                \hline
                5eA    & 12       & 6        & \last{7}       & 25        \\
                \hline
                5eB    & 13       & 5        & \last{5}       & \last{23}   \\
                \hline
                Total  & \last{25}       & \last{11}       & 12    & \last{48}      \\
                \hline
            \end{tabular}
        \end{center}
        % \imgp{tableau-exemple}[8cm]
    }[\ym]
}

\def\imgPrefix{dim-6e/exo-}
\exoSlide{17p104,19p104}[6cm][2][\dim]

\scn{Diagrammes en bâton}{}

\bsec{Diagrammes}
\bsubsec{Diagrammes en bâton}

\def\imgPrefix{dim-6e/qf-}
\exoSlide{3-4p96}[8cm][1][\dim][qf]

\def\imgPrefix{dim-6e/exo-}
\exoSlide{6p101}[8cm][1][\dim]

\slide{}{
    \act{}{
        \ifBeamer{\imgp{6p101}[10cm]}
        \begin{itemize}
            \item Représente les données de l'exercice 6 page 101 sous la forme d'un diagramme en bâtons.
        \end{itemize}
        \def\imgPrefix{}
        \imgp{diagramme-baton-fond-exo-6p102}[6cm]
    }
}

\def\imgPrefix{}
\slide{cr}{
    \ssec\ssubsec

    \vc{}{
        Un \key{diagramme en bâton} (ou à barres) permet de comparer visuellement des données.
    }[\dim]
    
    \expl{}{
        \begin{itemize}
            \item Notes (sur 5) de la dernière évaluation des 6eA; sous forme de \key{liste} :
            \begin{align*}
                0;4;5;5;5;1;3;4;2;3;4;1;0;4;3;4;5;4;3;2;4;2;5;5;4
            \end{align*}
            \item On peut représenter ces données dans un \key{tableau} :
            \begin{center}
                \begin{tabular}{|c|c|c|c|c|c|c|c|}
                    \hline
                    Notes & 0 & 1 & 2 & 3 & 4 & 5 & Total \\
                    \hline
                    Effectifs & 2 & 2 & 3 & 4 & 7 & 6 & 24 \\
                    \hline
                \end{tabular}
            \end{center}
            \item Ou avec un \key{diagramme en bâton}.
            \imgp{diagramme-baton-exo-6p102}[8cm]
        \end{itemize}
    }
}
% \def\imgPrefix{dim-6e/act-}
% \slide{exo}{
%     \bvspace{-0.75cm}
%     \act{2p98}{\bvspace{-0.75cm}\imgp{2p98}[9cm]}[\dim]
% }

% \def\imgPrefix{}
% \slide{exo}{
%     \imgp{tableau-activite-2p98}[11cm]
% }

% \slide{cr}{
% }

\slide{exo}{
    \bvspace{-0.5cm}
    \exo{Vrai ou faux?}{
        Le nombre de tablettes vendues de la marque B est trois fois plus important que le nombre de tablettes vendues de la marque A.
        \imgp{diagramme-en-baton-attendus-6e}[7cm]
    }
}

\slide{cr}{
    \pr{}{
        Si un diagramme en bâtons a pour origine 0,
        alors la hauteur des barres est proportionnelle aux effectifs.
    }
}


% \newpage

\def\imgPrefix{dim-6e/exo-}
\exoSlide{27p105}[3.8cm][1][\dim]

\scn{Diagrammes circulaires}{}

\bsubsec{Diagrammes circulaires}

\def\imgPrefix{dim-6e/qf-}
\exoSlide{5-6p96}[8cm][1][\dim][qf]

\def\imgPrefix{dim-6e/act-}
\slide{exo}{
    \bvspace{-0.75cm}
    \act{4p99}{\bvspace{-0.75cm}\imgp{4p99}[11cm]}[\dim]
}

\slide{cr}{
    \ssubsec
    \vc{}{
        Un \key{diagramme circulaire} permet d'observer une répartition.
    }[\dim]
    \pr{}{
        La valeur de chaque angle au centre est proportionnelle à la portion représentée.
    }
}

\scn{Utiliser différents modes de représentation}
\slide{qf}{
    \multiColEnumerate{3}{
        \item $7 \times 2 + 2$
        \item $15 + 28 \div 7$
        \item $12 \times (15-13)$
        \item $2 \times 3 - 1$
        \item $30 \div (15-13)$
        \item $19 + 2 \times 6$
    }
}

\def\imgPrefix{}
\slide{exo}{
    \bvspace{-0.5cm}
    \exo{}{\ifBeamer{\small}
        \bvspace{-0.25cm}
        \begin{enumerate}
            \dividePage{%
                Dans un collège,
                112 élèves viennent en voiture,
                autant viennent à vélo,
                56 viennent en bus et 280 viennent à pied.
                \item Un seul de ces diagrammes circulaires représente le mode de déplacement des élèves de ce collège.
                Lequel?
            }{\bvspace{-1cm}\imgp{vers-de-mobilites-douces-1}[6.5cm]}[0.5]
            
            \ifArticle{
                \dividePage{\imgp{vers-de-mobilites-douces-2}[8cm]}{%
                \item Compléter le tableau ci-contre, puis choisir les nombres appropriés pour graduer
                le diagramme en bâtons qui représente ces données.
                }
            }
            \saveenumi{1}
        \end{enumerate}
    }[\rpmc[34]]
}

\ifBeamer{
    \slide{exo}{\bvspace{-0.25cm}
        \begin{enumerate}
            \loadenumi
            \item Compléter le tableau ci-contre, puis choisir les nombres appropriés pour graduer
        le diagramme en bâtons qui représente ces données.
        \bvspace{-0.45cm}\imgp{vers-de-mobilites-douces-2}[8cm]
        \end{enumerate}
    }
}

\def\imgPrefix{dim-6e/exo-}

\scn{Situations de proportionnalité}{}

\exoSlide{36p107}[5.25cm][1][\dim][qf]

\bsec{Situations de proportionnalité}

\slide{exo}{
    \bvspace{-0.5cm}
    \act{}{
        \begin{itemize}
            \item Pour préparer une recette de dahl,
            Christopher a besoin de 4 gousses d'ail pour 6 personnes.
            Comme il déjeune avec ses amis Sarah et Jean.

            Peut-il prévoir le nombre de gousses d'ail nécessaires ?
            Pourquoi ? Si oui, combien ?

            \item Jean l'a félicité 2 fois pour sa cuisine en 20 minutes.

            Peut-il prévoir combien de félicitations il recevra de Jean en 40 minutes?
            Pourquoi ? Si oui, combien ?
        \end{itemize}
        
    }
}

\slide{cr}{
    \ssec
    % \df{}{
    %     Deux grandeurs sont dites \key{proportionnelles} si les valeurs de l'une s'obtiennent en multipliant les valeurs de l'autre par un même nombre non nul,
    %     appelé le \key{coefficient de proportionnalité}.
    % }
    \vc{}{
        Deux \key{grandeurs proportionnelles} sont deux grandeurs qui varient dans les mêmes proportions.
    }
}

\scn{Grandeurs proportionnelles}

\def\imgPrefix{}
\slide{qf}{%
    \begin{enumerate}
        \item \imgp{cn-CM2-juin-2023-exo-18}[8cm]
        \item \imgp{cn-CM2-juin-2023-exo-21-22}[8cm]
    \end{enumerate}
}

\slide{cr}{
    \expl{}{
        Les $2kg$ de lentilles coutent $5\EUR$.
        Combien en coutent $10kg$ de lentilles ? $12kg$ de lentilles?
    }
}

% \slide{cr}{
%     \rmk{}{
%         On peut présenter une situation de proportionnalité dans un \key{tableau de proportionnalite}.
%     }
% }


\def\imgPrefix{dim-6e/exo-}
\exoSlide{2p83}[6cm][1][\dim]

\scn{Decouvrir les graphiques cartésiens}{}

\def\imgPrefix{dim-6e/qf-}
\exoSlide{54-55-56p112}[10cm][1][\dim][qf]
\def\imgPrefix{}

\bsec{Graphiques cartésiens}

\def\imgPrefix{dim-6e/act-}
\slide{exo}{
    \bvspace{-0.70cm}
    \act{3p99}{\bvspace{-0.75cm}\imgp{3p99}[9cm]}[\dim]
}

\def\imgPrefix{}

\slide{exo}{
    \bvspace{-0.25cm}
    \begin{center}
        \begin{tikzpicture}[yscale = 0.75, xscale = 0.75]
            \tkzInit[xmin=0,xmax=10,ymin=0,ymax=10]
            \tkzGrid[sub,color=gradeColor!50!white,subxstep=5,subystep=0.1]        
            % \tkzLabelX[step=10]
            % \tkzLabelY[step=2]
            \tkzDrawY[label={}]
            \tkzDrawX[label= {}]
        \end{tikzpicture}
    \end{center}
    % \imgp{graph-bg-alcoolemie}[7cm]
}

\slide{cr}{
    \ssec
    \vc{}{
        Un \key{graphique cartésien} permet de présenter l'évolution d'une grandeur en fonction d'une autre.
    }
}

\scn{Dessiner un graphique cartésien}

\def\imgPrefix{dim-6e/qf-}
\exoSlide{57p112}[10cm][1][\dim][qf]
\def\imgPrefix{}

\def\graph{
    \begin{center}
        \begin{tikzpicture}[yscale = 0.35, xscale = 0.75]
            \tkzInit[xmin=0,xmax=12,ymin=0,ymax=20]
            \tkzGrid[sub,color=gradeColor!50!white,subxstep=5,subystep=0.2]        
            % \tkzLabelX[step=10]
            % \tkzLabelY[step=2]
            \tkzDrawY[label={}]
            \tkzDrawX[label= {}]
        \end{tikzpicture}
    \end{center}
}

\slide{cr}{
    \bvspace{-0.6cm}
    \ifBeamer{\small}
    \expl{}{%
    Le tableau ci-dessous représente les températures moyennes à Issy-les-Moulineaux pour chaque mois de 2023.
        
        \vspace{0.1cm}
        \wideFrame[9.2em]{
            \def\cW{0.55cm}
            \begin{tabular}{|c|C{\cW}|C{\cW}|C{\cW}|C{\cW}|C{\cW}|C{\cW}|C{\cW}|C{\cW}|C{\cW}|C{\cW}|C{\cW}|C{\cW}|C{\cW}|}
                \hline
                Mois & J & F & M & A & M & J & J & A & S & O & N & D \\
                \hline
                Température (\degres C) & 4,0 & 4,3 & 7,0 & 10,4 & 14,2 & 17,4 & 19,1 & 19,1 & 15,9 & 12,4 & 7,6 & 4,3 \\
                \hline
            \end{tabular}
        }

        \vspace{0.25cm}

        Représenter les données du tableau dans un graphique cartésien.
    }[\href{https://www.annuaire-mairie.fr/ensoleillement-issy-les-moulineaux.html}{https://www.annuaire-mairie.fr/}]
    \bvspace{-0.8cm}

    \ifArticle{\graph}
    % \imgp{graph-bg-meteo}[7cm]
}

\ifBeamer{\slide{cr}{\graph}}

\slide{exo}{
    \exo{}{
        Que pourrait représenter ce graphique à propos d'une salle de classe?
        Le décrire avec le plus de précision possible.
        Justifier et compléter le graphique ci-dessous.
        \imgp{l-allure-de-la-courbe}[10cm]
    }[\rpmc[31]]
}

\slide{cr}{
    \rmk{}{
        Les points sont :
        \begin{enumerate}
            \item représentés par des «+».
            \item reliés par des segments.
        \end{enumerate}
    }
}

% \bsec{Reconnaître une situation de proportionnalité}

\slide{exo}{%
    \bvspace{-0.6cm}
    \exo{}{
        On a demandé aux élèves des trois classes de 6e du collège Anatole France combien d'animaux de compagnie vivaient avec eux.
        On a représenté les résultats dans le diagramme suivant.
        \bvspace{-0.25cm}
        \imgp{nos-amis-les-betes}[9cm]
    }[\rpmc[28]]
}

% \exoSlide{41p109,25p105}[10cm][1][\dim][dm]

\slide{exo}{
    Les affirmations suivantes sont-elles vraies ou fausses? Justifier.
    \begin{enumerate}
        \setItemColor{\currentColor}
        \item 21 élèves ont un seul animal de compagnie.
        \item Il y a 75 élèves en 6e au collège Anatole France.
        \item Les élèves qui ont deux animaux de compagnie sont trois fois plus nombreux que les élèves qui ont trois animaux de compagnie.
        \item 70 élèves ont moins de trois animaux de compagnie.
        \item Plus de la moitié des élèves ont au moins un animal de compagnie.
    \end{enumerate}
}