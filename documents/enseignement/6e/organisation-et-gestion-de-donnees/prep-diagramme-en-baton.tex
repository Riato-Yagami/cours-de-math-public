% VARIABLES %%%
\setTitle{Fiche de Prep - Séance 3: Lire et construire un diagrammes en bâton}
\date{27/09/2024}
\setGrade{6e}
\def\imgPath{enseignement/6e/organisation-et-gestion-de-donnees/}
%%

\def\imgPrefix{dim-6e/}
\prepTable{
    \prepRow{
        \slide{SEANCES PRECEDENTES}{}
    }{
        \begin{itemize}
            \item Lire un tableau
            \item Construire un tableau
        \end{itemize}
    }{-}
    \prepRow{
        \slide{qf}{}
        \imgp{qf-3-4p96}
    }{
        \begin{itemize}[wide=0pt, leftmargin=*]
            \item 3p96: 
            \begin{itemize}[wide=0pt, leftmargin=*]
                \item Tâche: Lire un diagramme en bâton.
                \item Modalité de correction : correction d'un élève au tableau qui entourera les données recherché dans le document.
            \end{itemize}
        \end{itemize}
    }{
        $5\min$
    }
    \prepRow{
        \slide{CORRECTION}{}
        \imgp{exo-6p101.png}
    }{
        \begin{itemize}[wide=0pt, leftmargin=*]
            \item 6p101: 
            \begin{itemize}[wide=0pt, leftmargin=*]
                \item Tâche: Constrution de tableau a partir de série statique.
                \item Difficultés attendues : 
                identification des données à organiser,
                classement des données,
                différence entre modalité et effectif,
                utilisation de la ligne ou de la colonne
                \item Modalité de correction : création d'un tableau sur libreOffice Calc.
            \end{itemize}
        \end{itemize}
    }{
        $7\min$
    }
    \prepRow{
        \slide{ACTIVITE}{}
        \imgp{exo-6p101.png}
    }{
        \begin{itemize}[wide=0pt, leftmargin=*]
            \item A partir des données de l'exercice corrigé contruire un diagramme en bâton: 
            \begin{itemize}[wide=0pt, leftmargin=*]
                \item Tâche: Constrution d'un diagramme en bâton a partir d'un tableau.
                \item Difficultés attendues :
                gestion des axes,
                présentation graphique,
                respect de la proportionnalité,
                compréhension de la consigne,
                choix des unités et des échelles
                \item Modalité de correction :
                Sur un fond de graphique,
                un élève dessine un diagramme en bâton.
                Un éleve dessine au tableau le diagramme en bâton.
            \end{itemize}
        \end{itemize}
    }{
        $15\min$
    }
    \prepRow{
        \slide{cr}{}
        \setcounter{section}{1}
        \section{Diagrammes}
        \subsection{Diagrammes en bâton}
    }{
        \begin{itemize}
            \item \vc{Diagrammes en bâton}{à recopier}
            \item \expl{Differentes représentation}{à coller (correction de l'exercice/activité)}
        \end{itemize}
    }{
        $5\min$
    }

    \def\imgPrefix{}
    \prepRow{
        \slide{exo}
        \def\imgPrefix{}
        \imgp{diagramme-en-baton-attendus-6e}
    }{
        \begin{itemize}
            \item Vrai ou Faux?
            \begin{itemize}
                \item Tâche : Porter un regard critique sur une représentation graphique.
                \item Difficultés attendues:
                confusion entre les nombres réels et la représentation visuelle,
                compréhension de l'échelle
            \end{itemize}
        \end{itemize}
    }{10min}

    \prepRow{
        \slide{COURS (si le temps)}{}
    }{
        \begin{itemize}
            \item \pr{bâtons proportionnels}{à copier}
        \end{itemize}
    }{6min}

    \prepRow{
        \slide{EXERCICES A LA MAISON}{}
        \begin{itemize}
            \item 27p105
        \end{itemize}
    }{\imgp{exo-27p105}[5cm]}{3min}
    \prepRow{
        \slide{SEANCES SUIVANTES}{}
    }{
        \begin{itemize}
            \item Diagrammes circulaires
            \item Situations de proportionnalité
            \item Graphiques cartésiens
        \end{itemize}
    }{-}
}