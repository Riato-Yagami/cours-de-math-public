% VARIABLES %%%
\def\authors{\href{https://juels.dev/}{Jules PESIN}}
% \date{\today}
\def\longTitle{Relations géométriques}
\def\shortTitle{\MakeUppercase{\longTitle}}
\def\theme{\longTitle}

\setboolean{showRef}{false}

\def\my{Myriade 6e}
\newcommand{\myl}[1]{\href{#1}{\my}}

\def\imgPath{enseignement/6e/relations-geometriques/}
\def\imgExtension{.png}
%%

% Yvan Monka : https://www.maths-et-tiques.fr/telech/19Para_Perp.pdf

% \disableAnimation
% \shortAnimation
% \firstSlide

% \qf{
%     {$70\euro$ diminué de $10\%$, $63\euro$},
%     {$50\euro$ augmenté de $20\%$, $\pcalc{50*120/100}\euro$}%
% }[15]



% \qfSUB{}{}
\slide{QUESTIONS FLASH}{%
    \sqf Comment peut-on nommer la droite ci-dessous?
    \vspace*{-0.5cm}
    \imgp{qf1}[5cm]
    \qfs $(AB)$ \hfill \qfs $[AB]$ \hfill \qfs $(BA)$ \hfill \qfs $(d)$ \hfill \qfs $AB$ \hfill
    \vspace*{0.5cm}
    \\ \hint{plusieurs réponses sont attendus}
}

\slide{}{
    \sqf Quels points semblent alignés?
    \vspace*{-0.5cm}
    \imgp{qf2}[5cm]
    \qfs $A$ et $F$ \hfill \qfs $A$,$D$ et $F$ \hfill \qfs $A$,$C$ et $D$ \hfill \qfs $C$,$A$ et $D$ \hfill
    \vspace*{0.5cm}
    \\ \hint{plusieurs réponses sont attendus}
}

\slide{}{
    \sqf Les droites $(f)$ et $(g)$ semblent être:
    \vspace*{-0.5cm}
    \imgp{qf3}[5cm]
    \qfs Séquentes \hfill \qfs Parallèles \hfill \qfs Perpendiculaires \hfill
    \vspace*{0.5cm}
    \\ \hint{plusieurs réponses sont attendus}
}

\slide{}{
    \sqf Quelles droites semblent parallèles:
    \vspace*{-0.5cm}
    \imgp{qf4}[5cm]
    \qfs $(AB) \et (CE)$ \hfill \qfs $(AB) \et (DC)$ \hfill \qfs $(CD) \et (EB)$ \hfill
    \vspace*{0.5cm}
    % \\ \hint{plusieurs réponses sont attendus}
}

\bsec{Définitions}
\bsubsec{Alignement}
\slide{COURS}{
    \bseq{\longTitle}
    \ssec\ssubsec%

    \df{}{
        Des points sont \key{alignés} s'il existe une droite passant par tous ces points.
    }
}

\slide{}{
    \expl{}{
        \dividePage{
            \imgp{alignement}[5cm]
        }{
            Les points $A$, $C$ et $B$ sont alignés.\\
            Les points $A$, $D$ et $B$ ne sont pas alignés.
        }[0.35]
    }
    \rmk{}{
        Deux points peuvent toujours être relié par une droite et sont donc toujours alignés.
    }
}

\bsubsec{Relations entre des droites}

\slide{}{
    \ssubsec%
    %
    \df{}{
        Deux droites sont \key{sécantes} si elles se coupe en un unique point.
    }
}

\slide{}{
    \expl{}{
        \dividePage{
            \imgp{secantes}[5cm]
        }{
            Les droites $(AE)$ et$(BD)$ sont sécantes.\\ $C$ est leur point d'intersection.
        }[0.45]
    }[\myl{https://biblio.manuel-numerique.com?openBook=9782047392935\%3FY29udGV4dGVSZXNvdXJjZT17InR5cGUiOiJhcnRpY2xlIiwiaWRyZWYiOiJpZF9DaGFwdGVyXzAxMl9ab29tX0dyYXBoaWNfNTUxX1NDUl94aHRtbCIsImFydGljbGVUeXBlIjoiem9vbSJ9}]
}

\slide{}{
    \df{}{
        Deux droites sont \key{perpendiculaires} si elles sont sécantes et leur intersection forme un angle droit.
    }
    \expl{}{
        \dividePage{
            \imgp{perpendiculaires}[4cm]
        }{
            Les droites $(EF)$ et$(GF)$ sont perpendiculaires.\\ On note $(EF) \perp (GF)$.
        }[0.35]
    }[\myl{https://biblio.manuel-numerique.com?openBook=9782047392935\%3FY29udGV4dGVSZXNvdXJjZT17InR5cGUiOiJhcnRpY2xlIiwiaWRyZWYiOiJpZF9DaGFwdGVyXzAxMl9ab29tX0dyYXBoaWNfNTUyX1NDUl94aHRtbCIsImFydGljbGVUeXBlIjoiem9vbSJ9}]
}

\slide{}{
    \df{}{
        Deux droites sont \key{parallèles} si elles ne sont pas sécantes.
    }
    \expl{}{
        \dividePage{
            \imgp{paralleles}[4cm]
        }{
            Les droites $(d)$ et$(d')$ sont parallèles.\\ On note $(d) \parallel (d')$.
            \rmk{}{Deux droites parallèles conservent le même écartement}
        }[0.3]
    }[\myl{https://biblio.manuel-numerique.com?openBook=9782047392935\%3FY29udGV4dGVSZXNvdXJjZT17InR5cGUiOiJhcnRpY2xlIiwiaWRyZWYiOiJpZF9DaGFwdGVyXzAxMl9ab29tX0dyYXBoaWNfNTUzX1NDUl94aHRtbCIsImFydGljbGVUeXBlIjoiem9vbSJ9}]
}

\bsec{Construction}
\bsubsec{Perpendiculaires}
\slide{}{
    \ssec\ssubsec
    \dividePage{
        \exo{}{Construire la droite perpendiculaires à $(d)$ passant par $A$.
        \imgp{construction}[5cm]}
    }{
        \mthd{}{
            \imgp{construction-perpendiculaire}[5cm]
        }
    }
}

\bsubsec{Parallèles}
\slide{}{
    \ssubsec
    \dividePage{
        \exo{}{Construire la droite parallèles à $(d)$ passant par $A$.
        \imgp{construction}[5cm]}
    }{
        \mthd{}{
            \imgp{construction-parallele}[5cm]
        }
    }
}

\bsec{Propriété du parallélisme}
\slide{}{
    \ssec
    \pr{}{}
}

\slide{}{
    \pr{}{}
}

\slide{}{
    \pr{}{}
}