\setGrade{6e}
\evaluation{3}
% [corr]

\seqEvaluation{5}{Géométrie plane - Polygones}{
    Coder des figures simples.
    /1,
    Raisonner sur des figures simples.
    /4%
}

\seqEvaluation{6}{Nombres - Decimaux}{
    Utiliser les nombres décimaux jusqu'à trois décimales.
    /2,
    Ajouter{,} soustraire et multiplier des nombres décimaux.
    /2,
    Comparer des nombres décimaux.
    /1,
    Placer des nombres décimaux sur une droite graduée.
    /3%
}

\seqEvaluation{7}{Géometrie dans l'espace - Solide}{
    Reconnaître des solides particuliers et leurs caractéristiques
    /1,
    Raisonner en 3 dimensions.
    /3,
    Représenter des ensembles
    /0%
}

\evalutionEnd[1][2]

\noCalculator
\vspace{-0.5cm}
% % % DOC https://ctan.math.illinois.edu/macros/latex/contrib/scratch3/scratch3-fr.pdf

\setscratch{scale=.75}

\def\block{{\setscratch{scale=.5}\begin{scratch}\blockmove{\Large bloc}\end{scratch} }}

\newcommand{\scr}[1]{\begin{scratch}#1\end{scratch}}

\definecolor{smotion}{HTML}{4C97FF} % #4C97FF
\definecolor{slooks}{HTML}{9966FF} % #9966FF
\definecolor{ssound}{HTML}{D65CD6} % #D65CD6
\definecolor{sevents}{HTML}{FFD500} % #FFD500
\definecolor{scontrol}{HTML}{FFAB19} % #FFAB19
\definecolor{ssensing}{HTML}{4CBFE6} % #4CBFE6
\definecolor{soperators}{HTML}{6DB26E} % #6DB26E
\definecolor{svariables}{HTML}{F28011} % #F28011
\definecolor{smyblocks}{HTML}{FF6680} % #FF6680

\def\smotion{\textcolor{smotion}{\faCircle\,Mouvement}} % Déplacement du lutin
\def\slooks{\textcolor{slooks}{\faCircle\,Apparence}} % Modifier l'apparence du lutin ou de la scène
\def\ssound{\textcolor{ssound}{\faCircle\,Son}} % Jouer des sons ou de la musique
\def\sevents{\textcolor{sevents}{\faCircle\,Événement}} % Déclencher des scripts en réponse à des actions
\def\scontrol{\textcolor{scontrol}{\faCircle\,Contrôle}} % Boucles, conditions, et contrôle du flux
\def\ssensing{\textcolor{ssensing}{\faCircle\,Capteur}} % Réagir à des informations extérieures ou internes
\def\soperators{\textcolor{soperators}{\faCircle\,Opérateur}} % Calculs mathématiques et logiques
\def\svariables{\textcolor{svariables}{\faCircle\,Variable}} % Stockage et manipulation de données
\def\smyblocks{\textcolor{smyblocks}{\faCircle\,Mes blocs}} % Création de blocs personnalisés

\def\spen{{\icon{scratch/pen} Stylo}}
\def\spenExtension{{\icon{scratch/pen-extension} Stylo}}
\def\sextensions{{\icon{scratch/extensions} $\lbrack$ Ajouter une extensions $\rbrack$}}
\def\sflag{{\icon{scratch/flag}%
%  Drapeau
}}

% \setscratch{scale=.75}
% \setscratch{print=true}
% \setscratch{fill blocks=true}

\dividePage{
    \exo{}{
        \begin{center}
            \begin{scratch}
                \blockinit{quand \greenflag est cliqué}
                \blockpen{effacer tout}
                \blockpen{stylo en position d'écriture}
                \blockrepeat{répéter \ovalnum{2} fois}{
                    \blockmove{avancer de \ovalnum{40} pas}
                    \blockmove{tourner \turnright{} de \ovalnum{90} degrés}
                    \blockmove{avancer de \ovalnum{60} pas}
                    \blockmove{tourner \turnright{} de \ovalnum{90} degrés}
                }
            \end{scratch}
        \end{center}
    }
}{
    \begin{enumerate}
        \item Tracer un schéma codé de ce qui est construit par ce programme \Scratch dans l'encadré.
        \ctikz[1]{
    \boundingBox[8][6][0.5pt][1][(-1,-1)][dot]
    \nswr[0]{
        \draw[thick,color=gradeColor,fill=gradeColor,fill opacity=0.10] (0,3.76) -- (0.24,3.76) -- (0.24,4) -- (0,4) -- cycle;
        \draw[thick,color=gradeColor,fill=gradeColor,fill opacity=0.10] (5.76,4) -- (5.76,3.76) -- (6,3.76) -- (6,4) -- cycle;
        \draw[thick,color=gradeColor,fill=gradeColor,fill opacity=0.10] (6,0.24) -- (5.76,0.24) -- (5.76,0) -- (6,0) -- cycle;
        \draw[thick,color=gradeColor,fill=gradeColor,fill opacity=0.10] (0.24,0) -- (0.24,0.24) -- (0,0.24) -- (0,0) -- cycle;
        \draw [thick] (0,4) -- (6,4) -- (6,0) -- (0,0) -- cycle;
        \draw [thick] (3,4.11) -- (3,3.89);
        \draw [thick] (6.11,2.05) -- (5.89,2.05);
        \draw [thick] (6.11,1.95) -- (5.89,1.95);
        \draw [thick] (3,-0.11) -- (3,0.11);
        \draw [thick] (-0.11,1.95) -- (0.11,1.95);
        \draw [thick] (-0.11,2.05) -- (0.11,2.05);
    }
    \draw [thick, ->] (0,4) -- (1,4);
    \draw [thick, |-|] (6,-1) -- (7,-1) ;
    \draw[color=gradeColor] (6.48,-.6) node {$10$ pas};
    \draw[color=gradeColor] (0.50,4.39) node {Départ};
}
        \item De quelle figure sagit-il ?
        \nswr[3]{Ce programme \Scratch trace un rectangle.}
    \end{enumerate}
    % \answerSec{5}[\textcolor{exo}{2.}][Ce programme \Scratch trace un rectangle.]
}[0.4]
\exo{Altitudes}{
    On trouve dans ce tableau les montagnes les plus hautes de quelques pays d'Europe du monde et leur altitudes.
    \begin{center} \def\cW{3cm}
        \begin{tabular}{|*{3}{C{\cW}|}}\hline
            % Allemagne & Zugspitze & \Lg[km]{2.963} \\\hline
            Autriche & Großglockner & \Lg[km]{3.8} \\\hline
            Pologne & Mont Rysy & \Lg[km]{2.5} \\\hline
            Espagne & Mulhacén & \Lg[km]{3.482} \\ \hline
            Irlande & Carrauntoohil & \Lg[km]{1.05} \\\hline
            France & Mont Blanc & \Lg[km]{4.81} \\\hline
            Roumanie & Moldoveanu & \Lg[km]{2.54} \\\hline
            Hongrie & Mont Kékes & \Lg[km]{1.014} \\\hline
        \end{tabular}
    \end{center}
    Classe ces montagnes de la moins élevée à la plus élevée.
}[\dmeepc{6}]

\answerFill[Réponse
(sous forme d'une liste dans l'odre croissant des première lettre des pays correspondant à la montagnes séparé par des «;»)]
[H;I;P;R;E;A;F]

\exo{Ecrire sous forme décimale}{
    \multiColItemize{2}{
        % \item $ 2 + \dfrac{9}{100} + \dfrac{1}{10000} + \dfrac{2}{10} = \nswr{\np{2.2901}}$
        \item soixante douze millièmes = $\nswr{\np{0.072}}$
        \item $\dfrac{7598}{100} = \nswr{\np{75.98}}$
        \item $\dfrac{101}{1000} + \dfrac{12}{1000} = \nswr{\np{0.113}}$
        \item $\np{39045}$ cent-millièmes = $\nswr{\np{0.39045}}$
    }
}

\exo{Construire un carré}{
    \begin{enumerate}
        \item Construisez le carré ABCD de centre O en utilisant une règle non graduée, une équerre et un compas.
        \ctikz[\ifBA{0.6}{1}]{
    \draw[gray!40] (0.5,-7.5) rectangle (12.5,3.5);
    \drawPoint{A}{11.22}{-3.7}
    \drawPoint{O}{7.82}{-2.02}
    \nswr{
        \draw [thick,dashed] (7.82,-2.02) circle (3.79cm);
        \draw [thick,dashed] (10.19,2.77)-- (5.57,-6.58);
        \draw [thick,dashed] (11.22,-3.70)-- (1.88,0.91);
        \draw [thick,answer] (4.42,-0.34)-- (9.50,1.38);
        \draw [thick,answer] (9.50,1.38)-- (11.22,-3.70);
        \draw [thick,answer] (11.22,-3.70)-- (6.14,-5.42);
        \draw [thick,answer] (6.14,-5.42)-- (4.42,-0.34);
        \drawPoint{D}{9.5}{1.38}
        \drawPoint{B}{6.14}{-5.42}
        \drawPoint{C}{4.42}{-0.34}
    }
    \draw [thick] (11.22,-3.70) -- (7.82,-2.02);
}
        \item Quelles propriétés du carré avez-vous utilisées pour votre tracé ?\\
        \nswr[5]{
            Les diagonales du carré :
            \multiColItemize{1}{
                \item se coupent en leur milieu.
                \item sont de même longueur.
                \item sont perpendiculaires.
            }
        }
        \item Codez la figure.
    \end{enumerate}
}[\prbltq{construire-un-carre}]

\exo{Etape de calcul}{
    Calcul en faisant apparaitre les toutes les étapes de calculs :
    \multiColEnumerate{1}{
        % \item $\np{1.5} + 8 \times 9
        % = \nswr[1]{\np{1.5} + 72
        % = \np{73.5}}$
        \item $2 \times (\np{10} - \np{6.9}) - 5
        = \nswr[2]{2 \times \np{3.1} - \np{5}
        = \np{6.2} - \np{5}
        = \np{1.2}}$
        \item $3 \times \np{0.5} + 12 \times \np{0.25}
        = \nswr[2]{\np{1.5} + \np{3} = \np{4.5}}$
    }
}

\pagebreak

\exo{Décrire des solides}{
    \ctikz[0.7]{
    \boundingBox[18][14][0.75pt][1][(-1,-1)][iso]
    \node at (4.41, 4.43) {\cir[gradeColor]{1}};
    \node at (9.61, 8.95) {\cir[gradeColor]{2}};
    \node at (13.07, 3.44) {\cir[gradeColor]{3}};
    \drawPoint{A}{1.73}{0}
    \drawPoint{B}{0}{1}
    \drawPoint{C}{3.46}{1}
    \drawPoint{D}{5.20}{0}
    \drawPoint{E}{7.79}{1.50}
    \drawPoint{F}{3.46}{4}
    \drawPoint{G}{0}{4}
    \drawPoint{H}{2.60}{5.50}
    \drawPoint{I}{7.79}{4.50}
    \drawPoint{J}{5.20}{3}
    \drawPoint{K}{1.73}{3}
    \drawPoint{L}{2.60}{2.50}
    \draw [thick] (0,4) -- (2.60,5.50) -- (7.79,4.50) -- (7.79,1.50) -- (5.20,0) -- (3.46,1) -- (1.73,0) -- (0,1) -- (0,4) -- (1.73,3) -- (3.46,4) -- (5.20,3) -- (7.79,4.50);
    \draw [thick] (5.20,0) -- (5.20,3);
    \draw [thick] (3.46,1) -- (3.46,4);
    \draw [thick] (1.73,0) -- (1.73,3);
    \draw [thick] (9.53,2.50) -- (12.99,6.50) -- (11.26,1.50) -- (9.53,2.50);
    \draw [thick] (12.99,6.50) -- (14.72,1.50) -- (11.26,1.50);
    \draw [thick] (12.99,6.50) -- (16.45,2.50) -- (14.72,1.50);
    \draw [thick,dashed] (12.99,6.50) -- (12.99,4.50);
    \draw [thick,dashed] (16.45,2.50) -- (12.99,4.50);
    \draw [thick,dashed] (12.99,4.50) -- (9.53,2.50);
    \draw [thick] (6.06,7.50) -- (6.06,9.50) -- (8.66,6) -- (12.99,8.50) -- (10.39,12) -- (6.06,9.50);
    \draw [thick,dashed] (10.39,12) -- (10.39,10);
    \draw [thick,dashed] (10.39,10) -- (6.06,7.50);
    \draw [thick] (6.06,7.50) -- (8.66,6);
    \draw [thick,dashed] (10.39,10) -- (12.99,8.50);
    \draw [thick,dashed] (0,1) -- (2.60,2.50);
    \draw [thick,dashed] (2.60,2.50) -- (2.60,5.50);
    \draw [thick,dashed] (2.60,2.50) -- (7.79,1.50);
}
    \begin{enumerate}
        \item À quelle famille commune appartiennent tous ces solides ?
        \\\nswr[2]{Il s'agit de polyèdres.}
        \item À quelle famille particulière appartient chacun de ces solides ?
        \\\nswr[4]{Les solides \cir[gradeColor]{1} et \cir[gradeColor]{2} sont des prismes droits
        et le solide \cir[gradeColor]{3} est une pyramide.}
        \item Combien de faces, arêtes et sommets compte le solide \cir[gradeColor]{2} ?
        \\\nswr[3]{Le solide \cir[gradeColor]{2} a $5$ faces, $9$ arêtes et $6$ sommets.}
        \item Dans le solide \cir[gradeColor]{1}, nommez :
        \begin{enumerate}
            \item Une face parallèle à la face $BGKA$.
            \\\nswr[3]{La face $KACF$ semble perpendiculaire à la face $BGKA$.}
            \item Deux arêtes adjacentes et perpendiculaires à l'arête $[HI]$.
            \\\nswr[3]{Les arêtes $[IE]$ et $[HL]$ sont perpendiculaires à l'arête $[HI]$.}
        \end{enumerate}
    \end{enumerate}
}

\pagebreak

\vspace{-1cm}

\def\lines{%
    0/1/20,
    10/11/10,
    0/5/10,
    1/1.1/10,
    623/623.01/5
}

\def\points{%
    1/A/6,1/B/15,
    2/C/2,2/D/3,2/E/4,
    3/F/2,3/G/3,3/H/4,3/I/9,
    4/J/3,4/K/5,4/L/8,4/M/9,
    5/N/3,5/O/4
}

\def\paths{%
    chemin(E,D,G,H,J,F,C,A,B,I,M,O,N,K,E)
    §chemin(L,M) 
}

\exo{\ttps Dessin Gradué}{%
    Place les points celon les indications du tableau ci-dessous.\def\cW{1.1cm}%
    \vspace{-0.5cm}
    \begin{center}
        \begin{tabular}{|*{3}{c|c|c|}}
            \hline
            \textbf{Ligne} & Point & Abscisse
            & \textbf{Ligne} & Point & Abscisse
            & \textbf{Ligne} & Point & Abscisse \\
            \hline
            \textbf{1} & A & $\np{0.3}$   & \textbf{3} & G & $\np{0.5} + 1$                 & \textbf{4} & M & $\np{1.1} - \np{0.01}$ \\\hline
            \textbf{1} & B & $\np{0.75}$  & \textbf{3} & H & $\np{5} \times \np{0.3}$        & \textbf{5} & N & $\np{623} + \np{0.003} \times 2$ \\\hline
            \textbf{1} & C & $\np{10.2}$  & \textbf{3} & I & $\np{4.5}$  & \textbf{5} & O & $\np{623.008}$ \\\hline
            \textbf{2} & D & $\np{10} + \frac{3}{10}$  & \textbf{4} & J & $\np{1.03}$                    & & & \\\hline
            \textbf{2} & E & $\np{10.4}$  & \textbf{4} & K & $\np{1.05}$                    & & & \\\hline
            \textbf{3} & F & $\np{1}$     & \textbf{4} & L & $\np{1} + 8 \div \np{100}$  & & & \\\hline
        \end{tabular}

        \vspace{0.5cm}

        \ifthenelse{\boolean{answer}}{
            \DessinGradue[LignesIdentiques=false,Echelle=1.25,EcartVertical=1,
            Solution
            ]{\lines}{\points}{\paths}
        }{
            \DessinGradue[LignesIdentiques=false,Echelle=1.25,EcartVertical=1
            ]{\lines}{\points}{\paths}
        }

    \end{center}
    Tracer le polygone EDGHJFCABIMONKE et le segment [LM].
}


\exo{\bonus Patates de solides}{
    De la même manière que ce qui a été fait en cours pour les polygones et les nombres :
    constituez des « patates » (ou ensembles) avec les solides vus en cours.
    Pensez à inclure des schémas illustrant vos différentes « patates ».
    \nswr[0]{
        \ctikz[0.6]{
    % \boundingBox[18][14][0.5pt][1][(-6,-5)]
    \draw [rotate around={0:(-2.60,-0.00)},thick] (-2.60,-0.00) ellipse (0.50cm and 0.29cm);
    \draw [rotate around={0:(-0.64,-0.21)},thick] (-0.64,-0.21) ellipse (0.50cm and 0.29cm);
    \draw [rotate around={0:(-0.64,0.79)},thick] (-0.64,0.79) ellipse (0.50cm and 0.29cm);
    \draw (-2.79,-0.25) node[anchor=north west] {cône};
    \draw (-1.90,-0.59) node[anchor=north west] {cylindre de révolution};
    \draw (-0.29,-2.31) node[anchor=north west] {sphère};
    \draw [thick] (-0.07,-1.82) circle (0.50cm);
    \draw [rotate around={0:(-0.08,-1.82)},thick] (-0.08,-1.82) ellipse (0.50cm and 0.29cm);
    \draw [rotate around={122.67:(-1.61,0.45)},thick] (-1.61,0.45) ellipse (4.48cm and 1.78cm);
    \draw (-4.47,3.61) node[anchor=north west] {Solides de révolution};
    \draw (6.49,2.85) node[anchor=north west] {pavés droits};
    \draw (6.00,-0.39) node[anchor=north west] {cube};
    \draw [rotate around={-115.63:(6.87,1.09)},thick] (6.87,1.09) ellipse (2.33cm and 1.78cm);
    \draw (6.97,-2.57) node[anchor=north west] {pyramide};
    \draw (5.43,7.02) node[anchor=north west] {prismes droits};
    \draw [rotate around={98.24:(6.77,2.86)},thick] (6.77,2.86) ellipse (4.53cm and 2.96cm);
    \draw [rotate around={141.45:(5.51,2.77)},thick] (5.51,2.77) ellipse (7.11cm and 5.02cm);
    \draw (1.99,8.28) node[anchor=north west] {polyèdres};
    \draw [rotate around={164.57:(3.28,2.24)},thick] (3.28,2.24) ellipse (9.23cm and 6.91cm);
    \draw [shift={(-2.60,6.50)},thick]  plot[domain=0:3.14,variable=\t]({1*0.50*cos(\t r)+0*0.50*sin(\t r)},{0*0.50*cos(\t r)+1*0.50*sin(\t r)});
    \draw [rotate around={2.60:(-2.60,6.50)},thick] (-2.60,6.50) ellipse (0.50cm and 0.25cm);
    \draw (-1.48,8.45) node[anchor=north west] {Solides};
    \draw [rotate around={0:(-2.96,2)},thick] (-2.96,2) ellipse (0.50cm and 0.29cm);
    \draw [rotate around={0.27:(-2.96,3.03)},thick] (-2.96,3.03) ellipse (0.25cm and 0.10cm);
    \draw [thick] (-1.14,0.79) -- (-1.14,-0.21);
    \draw [thick] (-0.14,0.79) -- (-0.14,-0.21);
    \draw [thick] (5.63,-0.25) -- (6.06,-0.50) -- (6.50,-0.25) -- (6.50,0.25) -- (6.06,0) -- (6.06,-0.50);
    \draw [thick] (5.63,-0.25) -- (5.63,0.25) -- (6.06,0);
    \draw [thick] (5.63,0.25) -- (6.06,0.50) -- (6.50,0.25);
    \draw [thick] (6.05,1.28) -- (7.35,0.53) -- (7.35,1.53) -- (6.05,2.28) -- (6.05,1.28);
    \draw [thick] (7.35,1.53) -- (7.78,1.78) -- (7.78,0.78) -- (7.35,0.53);
    \draw [thick] (7.78,1.78) -- (6.48,2.53) -- (6.05,2.28);
    \draw [thick] (7.99,-2.31) -- (8.42,-1.56) -- (8.86,-2.31) -- (8.86,-2.81) -- (8.42,-1.56);
    \draw [thick] (7.99,-2.31) -- (8.86,-2.81);
    \draw [thick] (6.48,5.45) -- (7.35,3.95) -- (8.21,3.45) -- (8.65,4.70) -- (6.48,5.45) -- (6.52,6.36) -- (7.35,4.95) -- (7.35,3.95);
    \draw [thick] (7.35,4.95) -- (8.21,4.45) -- (8.65,5.70) -- (6.52,6.36);
    \draw [thick] (8.21,4.45) -- (8.21,3.45);
    \draw [thick] (8.65,5.70) -- (8.65,4.70);
    \draw [thick] (1.21,4.75) -- (1.21,5.75) -- (1.65,5.50) -- (1.65,4.50) -- (3.38,4.50) -- (3.38,5.50) -- (1.65,5.50);
    \draw [thick] (2.08,6.25) -- (3.38,5.50) -- (2.08,5.75) -- (2.08,6.25) -- (1.65,6.50) -- (1.65,6.00) -- (2.08,5.75);
    \draw [thick] (2.08,6.25) -- (2.51,6.50) -- (2.08,6.75) -- (1.65,6.50);
    \draw [thick] (3.38,5.50) -- (2.51,6.50);
    \draw [thick] (1.65,4.50) -- (1.21,4.75);
    \draw [thick] (1.21,5.75) -- (1.65,6.00);
    \draw [thick] (-2.60,1) -- (-3.08,0.08);
    \draw [thick] (-2.60,1) -- (-2.12,0.08);
    \draw [thick] (-2.60,5) -- (-1.73,5.50) -- (-1.73,6.50) -- (-2.60,6) -- (-3.46,6.50) -- (-3.46,6) -- (-3.90,5.25) -- (-3.46,5) -- (-3.03,5.75) -- (-3.46,6);
    \draw [thick] (-2.60,6) -- (-2.60,5) -- (-3.03,5.25) -- (-3.46,5) -- (-3.46,5);
    \draw [thick] (-3.03,5.25) -- (-3.03,5.75);
    \draw [thick] (-3.46,6.50) -- (-3.03,6.75);
    \draw [thick] (-1.73,6.50) -- (-2.17,6.75);
    \draw [thick] (-3.46,2) -- (-3.21,3.03);
    \draw [thick] (-2.72,3.01) -- (-2.46,2);
}
    }
}
