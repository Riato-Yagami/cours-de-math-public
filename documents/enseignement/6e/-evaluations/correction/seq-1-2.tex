\setGrade{6e}
\setTitle{Evaluation 1 - Correction}

\seqEvaluation{1}{Géométrie plane - points et droites}{
    Suivre un programme de construction
    /3,
    Construire une perpendiculaire passant par un point donné
    /4,
}

\seqEvaluation{2}{Organisation et gestion de données}{
    Prélever des données à partir de supports variés
    /2,
    Produire un graphique cartésien
    /4,
    Compléter un tableau à double entrée
    /4,
    Reconnaître une situation de proportionnalité /0%
}

\seqEvaluation{}{Compétences générales}{
    Écrire ses calculs
    /1,
    Rédiger des phrases réponses
    /2
}

\evalutionTotal
\vspace{-1.5cm}
\noCalculator
% images/enseignement/6e/geometrie-plane/points-et-droites/dim-6e/exo-44p217.png

\exo{\ttps Tableau à double entrée}{
    Lors des Jeux Olympiques, des athlètes amateurs doivent choisir entre trois sports :
    l'athlétisme, la natation ou l'escalade.
    Ils sont répartis en deux groupes : les juniors et les seniors.
    \begin{itemize}
        \item Dans la catégorie juniors, il y a 20 athlètes.
        8 ont choisi l'athlétisme, 7 la natation et les autres l'escalade.
        \item Dans la catégorie seniors, 15 athlètes ont choisi la natation et 6 l'athlétisme.
        \item Dans ces deux catégories, 12 athlètes ont choisi l'escalade.
    \end{itemize}
    
    Compléter le tableau ci-dessous, en faisant figurer vos calculs en dessous.\\
    \vspace{-0.5cm}
    \renewcommand{\arraystretch}{1}
    \def\cW{3cm}
    \begin{center}
        \begin{tabular}{|C{2cm}|C{\cW}|C{\cW}|C{\cW}|C{\cW}|}
            \hline
            & \textbf{Athlétisme} & \textbf{Natation} & \textbf{Escalade} & \textbf{Total} \\ \hline
            \textbf{Juniors} & {\color{Blue}$8$} & {\color{Blue}$7$} & {\color{Green}$5$} & {\color{Blue}$20$}\\ \hline
            \textbf{Seniors} & {\color{Blue}$6$} & {\color{Blue}$15$} & {\color{Green}$7$} & {\color{Green}$28$}\\ \hline
            \textbf{Total}   & {\color{Green}$14$} & {\color{Green}$22$} & {\color{Blue}$12$} & {\color{Green}$48$} \\ \hline
        \end{tabular}
    \end{center}
}

\answerFill[Calculs effectués][
\multiColItemize{3}{
    \item $20 - 8 - 7 = {\color{Green}5}$
    \item $12 - 5 = {\color{Green}7}$
    \item $6 + 8 = {\color{Green}14}$
    \item $7 + 15 = {\color{Green}22}$
    \item $6 + 15 + 7 = {\color{Green}28}$
    \item $20 + 28 = {\color{Green}48}$
}]

\exo{44p217}{
    \vspace{-0.5cm}
    \imgp{enseignement/6e/geometrie-plane/points-et-droites/dim-6e/exo-44p217.png}[9cm]

    Dessiner la figure sur la grille ci-dessous en conservant l'emplacement du point $A$.

    \imgp{enseignement/6e/geometrie-plane/points-et-droites/dm-44p217}[8cm]
}[\dim]

\vspace{-0.5cm}

\answerFill[\textbf{\color{Orange}d.}][La droite $(AC)$ semble passer par le point $P$.]

\newpage

\def\crossWidth{0.4mm}
\exo{Graphique cartésien}{
    Le tableau ci-dessous présente le nombre de consoles PlayStation 3 vendues (en millions) par Sony chaque année, de 2006 à 2016.

    \vspace{0.25cm}
    \begin{tabular}{|C{2.5cm}|c|c|c|c|c|c|c|c|c|c|c|}
        \hline
        Année & 2006 & 2007 & 2008 & 2009 & 2010 & 2011 & 2012 & 2013 & 2014 & 2015 & 2016 \\
        \hline
        PS3 (en millions)  & 1,3 & 7,9 & 10,2 & 13 & 14 & 14,7 & 11,5 & 8,3 & 4,1  & 1,3 & 1,2 \\
        \hline
    \end{tabular}

    \vspace{0.25cm}
    Compléter de la manière la plus precise possible le graphique cartésien ci-dessous avec les donnés du tableau.
    
    % \vspace{0.5cm}
    
    \begin{center}
        \begin{tikzpicture}[yscale = 1, xscale = 1.2]
            \tkzInit[xmin=2005,xmax=2017,ymin=0,ymax=15]
            % \tkzGrid[sub]
            \tkzGrid[sub,color=gradeColor!50!white,subxstep=1,subystep=0.2]        
            \tkzLabelX[step=5]
            \tkzLabelY[step=2]
            \tkzDrawY[label={PS3 (en millions)}, above , step=2]
            \tkzDrawX[label={Années}]
            % Tracer les points
            \foreach \x/\y in {2006/1.3, 2007/7.9, 2008/10.2, 2009/13.0, 2010/14.0, 2011/14.7, 2012/11.5, 2013/8.3, 2014/4.1, 2015/1.3, 2016/1.2} {
                % \tkzPlotPoint(\x,\y)
                \drawPoint{}{\calc{\x-2005}}{\y};
            }
            % Relier les points
            \draw[Green](1,1.3)--(2,7.9)--(3,10.2)--(4,13.0)--(5,14.0)--(6,14.7)--(7,11.5)--(8,8.3)--(9,4.1)--(10,1.3)--(11,1.2);
        \end{tikzpicture}
    \end{center}

}[\wiki{PlayStation_3}]

\newpage

\exo{\bonus}{
    Les deux diagrammes ci-dessous fournissent des informations sur les trois premières sagas du manga One Piece :
    East Blue, Alabasta et Skypiea.

    Le diagramme en bâton donne le nombre de chapitres dans chacune de ces sagas.
    Et et le diagramme circulaire indique comment sont répartis les tomes du manga parmi ces trois sagas.
    
    \imgp{enseignement/6e/organisation-et-gestion-de-donnees/diagramme-one-piece.png}[15cm]

    \begin{enumerate}
        \item Combien y a-t de chapitres dans la $1^{\textrm{re}}$ saga?
        \item Quelle saga à le moins de tomes?
        \item Le nombre de tomes est-il proportionnel au nombre de chapitres dans ces trois premières saga ?
    \end{enumerate}
}[\href{https://onepiece.fandom.com/fr/wiki/Mer_de_la_Survie_:_Saga_Super_Rookies}{One Piece Encyclopédie}]

\answerFill[Réponse][
    \begin{enumerate}
        \item D'après le diagramme en baton la $1^{\textrm{re}}$ saga : East Blue comprend 100 chapitres
        \item D'après le diagramme circulaire, la $3^{\textrm{e}}$ saga : Skypiea a le moins de tomes.
        \item D'après le diagramme circulaire, il y a autant de tomes dans la $1^{\textrm{re}}$ et la $2^{\textrm{e}}$ saga.
        Or il y a plus de chapitres dans la $2^{\textrm{re}}$ saga que dans la $1^{\textrm{re}}$.
        Le nombre de tomes dans une saga n'est donc pas proportionnel à son nombre de chapitres.
    \end{enumerate}
]