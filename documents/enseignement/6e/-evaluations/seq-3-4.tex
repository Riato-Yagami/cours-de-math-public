%%%
\setGrade{6e}
\evaluation{2}
% [corr]

\def\cwr{\href{https://college-willy-ronis.fr/maths/wp-content/uploads/2020/11/Chap-1-exercices-corriges-6eme.pdf}{College Willy Ronis}}
%%%
% \def\imgPath{enseignement/6e/}

\seqEvaluation{3}{Géométrie plane - distance}{
    Tracer un segment de longueur donnée.
    /2,
    Placer le milieu d'un segment de longueur donnée.
    /1,
    Construire un cercle.
    /3%
    % Déterminer le plus court chemin entre un point et une droite.
    % /0%
}

\seqEvaluation{4}{Nombres - Entiers}{
    Utiliser et représenter les grands nombres entiers.
    / 1,
    Utiliser la division euclidienne.
    / 3,
    Résoudre des problèmes relevant des structures additives et multiplicatives en mobilisant une ou plusieurs étapes de raisonnement.
    / 3,
    % Organiser un calcul en une seule ligne{,} utilisant si nécessaire des parenthèses.
    % / 2,
    Résoudre des calculs en lignes en respectant les priorités opératoires
    /2,
    Analyser un système de numération
    /0%
}

\evalutionEnd[3][2]

\noCalculator

\exo{Utiliser les chiffres arabes}{Écrire les nombres suivant à l'aide de chiffres arabes:
    \multiColEnumerate{1}{
        % \item Neuf-mille-quatre-vingt-quinze : \nswr{$\np{9095}$}
        \item $8\times\np{1000}+2\times10+1\times\np{10000}+7$ : \nswr{$\np{18027}$}
        % \item \Ecriture{4 000 780}
        \item Quatre-milliards-sept-cent-quatre-vingts :
        \nswr{$\np{4 000 000 780}$}
        \item $2$ millions et $2$ soixantaines : \nswr{$\np{2 000 120}$}
        \item $15$ milliers et $\np{3 234}$ unités : \nswr{$\np{18234}$}
    }
}[\cwr]

\exo{Calculs en lignes}{
    Résoudre les opérations suivantes :
    \begin{enumerate}
        \item $211 \times 2 + 12 \times 4 = \nswr[3]{422 + 48 = 470}$ 
        \item $2 \times \np{10 000} + (14 - 9) = \nswr[3]{\np{20 000} + 5 = \np{20 005}}$
        \item $(24 + 2 \times 6) \div 2 = \nswr[3]{(24 + 12) \div 2 = 36 \div 2 = 18}$
    \end{enumerate}
}

\newpage

\def\length{10}
\def\crossWidth{0.3mm}
\def\crossSize{0.15}
\exo{Triplet de cercles}{
    \begin{enumerate}
        \item Trace la demi-droite $[AC)$.
        \item Place $B$ sur la demi-droite $[AC)$ tel que : $AB = \Lg{\length}$.
        \item Marque le point $O$, milieu du segment $[AB]$.
        \item Trace le cercle de centre $O$ et de rayon $\Lg{\directlua{tex.print(math.floor(\length / 2))}}$.
        \item Trace deux cercles de diamètres $[AO]$ et $[OB]$.
    \end{enumerate}
    \ctikz[1]{
        \draw[gradeColor!40] (-7,-3) rectangle (10,12);
        \ifthenelse{\boolean{answer}}{%
            \draw [thick,domain=-3.84:9] plot(\x,{(--7.6992--0.52*\x)/1.76});
            \draw [thick] (0.9550885402945024,4.6567307050870115) circle (5cm);
            \draw [thick] (-1.4424557298527487,3.948365352543506) circle (2.5cm);
            \draw [thick] (3.3526328104417535,5.365096057630517) circle (2.5cm);
            \drawPoint{B}{5.75}{6.07}
            \drawPoint{D}{0.96}{4.66}
            \drawPoint{E}{3.35}{5.37}
            \drawPoint{F}{-1.44}{3.95}
        }{}%
        \drawPoint{A}{-3.84}{3.24}
        \drawPoint{C}{-2.08}{3.76}
    }
}[\sesa{6}{2021}[9][85]]

\newpage

\exo{Problème 1}{
    Salma range ses 8 000 timbres dans un classeur, elle colle 52 timbres par pages.
    Combien de timbres contient la dernière page non remplie?
}[\iP{6}{2021}[1][16]]

\answerSec{12}[Réponses][
    Pour trouver la réponse on peut poser la division euclidienne de 8000 par 52.
    \begin{center}
        \longDivision{8000}{52}
    \end{center}
    La dernière page non remplie contient alors 44 timbres.
]

% \exo{BONUS}{
%     \begin{enumerate}
%         \item Le 17 juin 2345 sera une date très particulière car elle s'écrire : $17\;06\;2345$, c'est à dire avec huit chiffres tous
%         différents. Quelle a été la dernière date à posséder cette propriété, c'est à dire à s'écrire sous la forme d'un nombre à
%         huit chiffres tous différents ?
%     \end{enumerate}
% }[\cwr]

% \answerFill[Réponses][
%     \begin{enumerate}
%         \item le 26 août 1987 qui donne : $26\;07\;1987$
%     \end{enumerate}
% ]

\exo{\tiersTemps Problème 2}{
    Un boulanger achète 5 sacs de \Masse[kg]{25} de farine.  
    Chaque sac lui coûte \Prix{50}.  
    Pour chaque kilo de farine, il peut faire 8 baguettes qu'il vend à \Prix{2} l'unité.
    Si on considère qu'il ne dépense rien pour les autres ingrédients de sa baguette quel bénéfice réalisera-t-il?
}

\answerFill[Réponse][
    \begin{itemize}
        \item $5 \times 25 = 125$ Le boulanger achete \Masse[kg]{125} de farine.
        \item $5 \times 50 = 250$ Le coût total des sacs est de \Prix{250}.
        \item $125 \times 8 = 1000$ Il peut faire 1000 baguettes avec sa farine.
        \item $1000 \times 2 = 2000$ \Prix{2000} Il vendra pour \Prix{2000} de baguettes.
        \item $2000 - 250 = 1750$ \Prix{1750} Il fera un bénéfice total de \Prix{1750}.
    \end{itemize}
]

\def\aspc{\ifthenelse{\boolean{answer}}{}{\vspace*{1cm}}}

\exo{\bonus Numération Maya}{
    Cet exercice explore la numération utilisée par la civilisation mésoaméricaine maya.
    \begin{enumerate}
        \item Quelques exemples de nombres écrits à l'aide de chiffres maya :
        \multiColItemize{3}{
            \item \Maya{4} = 4
            \item \Maya{8} = 8	
            \item \Maya{17} = 17
        }
        Que représentent les symboles \Maya{1} et \Maya{5} ?\\
        \nswr[4]{Le symbole \Maya{1} correspond à une unité, tandis que \Maya{5} représente cinq unités.}
        %
        \item Complète les égalités suivantes :
        \multiColEnumerate{3}{
            \item \Maya{3} = \nswr{3}
            \item \Maya{12} = \nswr{12}
            \item \nswr{\Maya{19}} = 19
        }
        %
        \item Observe les nombres ci-dessous et explique leur composition :
        \multiColItemize{3}{
            \item \Maya{21} = 21
            \item \Maya{46} = 46	
            \item \Maya{60} = 60
        }
        \nswr[5]{Un espace horizontal entre deux symboles indique un changement de classe : on passe du comptage des unités à celui des vingtaines.}
        %
        \item Quelle est la base utilisée dans cette numération ?\\
        \nswr[3]{Cette numération est en base 20.}
        %
        \item Que signifie le symbole \Maya{0} ?\\
        \nswr[4]{Le symbole \Maya{0} indique qu'il n'y a aucune unité dans la classe où il est placé.}
        %
        \item Complète les égalités suivantes :
        \multiColEnumerate{3}{
            \item \Maya{33} = \nswr{33}
            \item \Maya{452} = \nswr{452}
            \item \aspc\nswr{\Maya{156}} = 156
        }
    \end{enumerate}
}
