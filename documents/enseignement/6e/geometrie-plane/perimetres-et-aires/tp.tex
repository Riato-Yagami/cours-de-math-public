\setGrade{6e}
\tp{\GeoGebra - Périmètre et aires}
% [corr]

\def\iconPath{geogebra/}

\def\gdot{\tool{Point}[dot]}
\def\gcirclePoint{\tool{Cercle (centre-point)}[circle-point]}
\def\gcircleRadius{\tool{Cercle (centre-rayon)}[circle-radius]}
\def\gsegment{\tool{Segment}[segment]}
\def\gselect{\tool{Selection}[select]}
\def\gcenter{\tool{Milieu}[center]}
\def\gperpendicular{\tool{Perpendiculaire}[perpendicular]}
\def\gparallel{\tool{Parallèle}[parallel]}
\def\gpolygon{\tool{Polygone}[polygon]}

\def\glength{\tool{Distance et longueur}[length]}
\def\garea{\tool{Aire}[area]}

\def\ghideShow{\tool{Montrer/Cacher l'objet}[hide-show]}
\def\gzoom{\tool{Agrandissement}[zoom]}
\def\gunzoom{\tool{Réduction}[unzoom]}


\def\gmenu{\tool{Menu}[menu]}
\def\gproperties{\tool{Propriétés}[properties]}

\dividePage{
    Dans la figure ci-contre, $AB = \Lg{10}$.\\ \\ 
    Nous allons chercher où placer le point $C$ sur le segment $[AB]$ pour que le périmètre du triangle équilatéral $CBD$ soit égal au périmètre du carré $ACEF$.\\ \\ 
    Pour cela, nous allons utiliser \GeoGebra{}. 
}{
    \ctikz[0.70]{
    % \boundingBox[10.1][6.6][0.5pt][1][(-8,2)]
    \fill[thick,color=gradeColor,fill=gradeColor,fill opacity=0.10] (-5.33,2) -- (-1.67,8.35) -- (2,2) -- cycle;
    \fill[thick,color=gradeColor,fill=gradeColor,fill opacity=0.10] (-5.33,2) -- (-8,2) -- (-8,4.67) -- (-5.33,4.67) -- cycle;
    \drawPoint{A}{-8}{2}
    \drawPoint{B}{2}{2}
    \drawPoint{C}{-5.33}{2}
    \drawPoint{D}{-1.67}{8.35}
    \drawPoint{F}{-8}{4.67}
    \drawPoint{E}{-5.33}{4.67}
    \draw [thick,gradeColor] (-5.33,2) -- (-1.67,8.35);
    \draw [thick,gradeColor] (-1.67,8.35) -- (2,2);
    \draw [thick,gradeColor] (2,2) -- (-5.33,2);
    \draw [thick,gradeColor] (-5.33,2) -- (-8,2);
    \draw [thick,gradeColor] (-8,2) -- (-8,4.67);
    \draw [thick,gradeColor] (-8,4.67) -- (-5.33,4.67);
    \draw [thick,gradeColor] (-5.33,4.67) -- (-5.33,2);
}
}

\begin{enumerate}
    \item Ouvrir \capytale{2436-5483042}.
    \item \begin{enumerate}
        \item Placer un point $A$ avec l'outil \gdot.
        \item Placer un point $B$ à \Lg{10} de $A$ à l'aide de \gcircleRadius.
        \hint{Vous pouvez cacher vos traits de construction grâce à l'outil \ghideShow.}
        \item Tracer le segment $[AB]$ avec \gsegment{} et placer dessus le point $C$.
        \item Vérifier que $B$ et $C$ sont bien placés.
        Pour cela, utiliser l'outil \gselect{} pour déplacer les points $B$ et $C$.
        Si le point $B$ reste toujours à \Lg{10} de $A$ et que le point $C$ reste toujours sur le segment $[AB]$, alors les points sont bien placés et on peut dire que votre figure est \key{solide}.
    \end{enumerate}
    \item \begin{enumerate}
        \item À l'aide de l'outil \gcirclePoint{} que vous utiliserez comme un compas,
        placer le point $D$ pour former le triangle équilatéral $CBD$.
        Pensez à toujours vérifier que vos figures sont solides.
        \item À l'aide des outils \gcirclePoint{} et \gperpendicular{}, placer les points $E$ et $F$ pour former le carré $ACEF$.
    \end{enumerate}
    \item \begin{enumerate}
        \item Dessiner les polygones $CBD$ et $ACEF$ à l'aide de l'outil \gpolygon{}.
        \item À l'aide de l'outil \glength{}, mesurer le périmètre du triangle $CBD$ et du carré $ACEF$ en cliquant sur les polygones.
        \item Toujours avec l'outil \glength{}, mesurer le segment $[AC]$.
    \end{enumerate}
    \item \begin{enumerate}
        \item En déplaçant le point $C$, chercher une longueur $AC$ pour laquelle les périmètres des deux polygones sont les mêmes.
        \item Cette mesure est-elle entière ?
        \item Cette mesure semble-t-elle décimale ? \hint{Vous pouvez modifier le nombre de décimales affichées depuis l'onglet \gmenu{} $\rightarrow$ \gproperties.}
        \item Donner la mesure de $AC$ au millième près.
    \end{enumerate}
    \item \begin{enumerate}
        \item À l'aide de l'outil \garea{}, mesurer les aires du triangle $CBD$ et du carré $ACEF$ en cliquant sur les polygones.
        \item Donner une mesure de $AC$ au millième près pour que les aires des deux polygones soient égales.
        \item Cette mesure est-elle la même que celle pour le périmètre ?
        \item Peut-on dire que deux polygones qui ont la même aire ont le même périmètre ?
    \end{enumerate}
\end{enumerate}