\setGrade{5e}
\tp{\GeoGebra - Polygones}
% \setTitle{TP \GeoGebra - Polygones}
\def\imgPath{enseignement/6e/geometrie-plane/polygones/}
\def\iconPath{geogebra/}

Lancer l'ativité \capytale{2b0d-4880399}

\begin{enumerate}
    \item Construire un triangle équilatéral.  
    \hint{
        \begin{itemize}
            \item Outils nécessaires : \tool{Point}[dot], \tool{Cercle (centre-point)}[circle-point], \tool{Segment}[segment].
            \item Réfléchissez à la méthode utilisée pour construire un triangle équilatéral à la règle et au compas afin de reporter les longueurs.
            Dans \GeoGebra{}, le compas est remplacé par l'outil \tool{Cercle (centre-point)} :
            cliquez d'abord sur le point où vous placeriez la pointe du compas,
            puis sur un point par lequel le cercle devrait passer.
        \end{itemize}
    }

    \item \begin{itemize}
        \item Dans \GeoGebra, une figure est dite \key{solide} si,
        en déplaçant n'importe quel point ou segment,
        elle conserve ses propriétés géométriques.
        \item Sélectionnez un point de votre triangle et vérifiez qu'il est bien solide.
        Si ce n'est pas le cas, reconstruisez-le correctement.
        \item Toutes les figures que vous construirez doivent être solide.
    \end{itemize}
    
    \def\iconPath{geogebra/}\hint{Outil à utiliser : \tool{Sélection}[select].}

    \item Construire un carré.
    \\\hint{
        Outil à utiliser : \tool{Perpendiculaire}[perpendicular].  
        (Sélectionnez une droite existante, puis le point par lequel doit passer la perpendiculaire.)
    }

    \item Construire un losange.  
    Utilisez la propriété des diagonales du losange pour le construire.
    \\\hint{Outil à utiliser : \tool{Milieu}[center].}

    \item Construire un quadrilatère dont les côtés sont parallèles.
    \\\hint{
        Outil à utiliser : \tool{Parallèle}[parallel].  
        (Sélectionnez une droite de référence, puis le point par lequel doit passer la parallèle.)
    }
\end{enumerate}
