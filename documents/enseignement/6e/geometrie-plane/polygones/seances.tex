%%% VARIABLES %%%
\setSeq{5}{Géométrie plane - Polygones}
\setGrade{6e}
\def\imgPath{enseignement/6e/geometrie-plane/polygones/}
\def\ym{\href{https://www.maths-et-tiques.fr/telech/19Quad.pdf}{Yvan Monka}}
% Triangle https://www.maths-et-tiques.fr/telech/19Triang.pdf
% \forPrint
\def\caPrefix{6e-juin-2022-}
%%

\obj{
    \item Savoir coder des figures simples :
    \begin{itemize}
        \item triangles (et particuliers : rectangle, isocèle, équilatéral)
        \item quadrilatères (et particuliers : carré, rectangle, losange)
    \end{itemize}
    \item Connaître et utiliser le vocabulaire associé à ces figures et leurs propriétés.
    \item Reconnaître, nommer et décrire des figures complexes (assemblages de figures simples).
    \item Représenter, reproduire, tracer ou construire des figures simples.
    \item Représenter, reproduire, tracer ou construire des figures complexes (assemblages de figures simples).
    \item Réaliser, compléter ou rédiger un programme de construction d'une figure plane.
}

\scn{Définir un polygone}

\caSlide{6-7-8}

\bsec{Définitions}

\slide{cr}{\bsmall
    \sseq\ssec
    \df{}{
        On appelle \key{polygone} une ligne brisée fermée.
    }[\wiki{Polygone}]

    \df{}{
        On appelle :
        \begin{itemize}
            \item \awsr{\key{côtés}} les segments qui compose un polygone.
            \item \awsr{\key{sommets}} les extremités des côtés.
            \item \awsr{\key{diagonale}} un segment reliant deux sommets non consécutifs.
        \end{itemize}
    }[\href{https://www.desmaths.fr/cours/index.php?n=6e&c=09}{Desmaths.fr}]
}

\slide{cr}{
    \expl{}{
        \ctikz[1]{
    \draw[gray!40] (-7.5,-4.5) rectangle (1.5,3.5);
    % \draw[color=gradeColor] (-4.02,-3.19) node {$sommet$};
    \draw [thick] (-6.50,1.92) -- (-3.44,-2.78) -- (0.46,-0.58) -- (-0.04,1.72) -- cycle;
    \draw [thick] (-3.44,-2.78) -- (0.46,-0.58);
    \awsr[0]{
        \draw [thick,dashed] (-6.50,1.92)-- (0.46,-0.58);
        \draw [thick,dashed] (-3.44,-2.78)-- (-0.04,1.72);
        \drawPoint{sommet}{-3.44}{-2.78}[answer]
        \draw[color=answer] (-3.48,0.3) node {diagonale};
        \draw[color=answer] (-3.16,2.29) node {côté};
    }
}
        Ce polygone à :
        \multiColEnumerate{3}{
            \item \awsr{5} sommets
            \item \awsr{5} cotés
            \item \awsr{4} diagonales
        }
    }
}

% \newpage

\slide{cr}{\bsmall
    \rl{}{
        Le nom d'un polygone est donné par ses sommets.
        On part d'un sommet et on fait le tour du polygone,
        dans un sens ou dans l'autre.
    }

    \expl{}{
        \ctikz[\ifBA{0.6}{1}]{
    \draw[gray!40] (-0.5,-9) rectangle (9.5,-2);
    \draw [thick] (2.52,-4.74)-- (3.34,-2.54);
    \draw [thick] (3.34,-2.54)-- (5.54,-4.46);
    \draw [thick] (5.54,-4.46)-- (3.1,-7.46);
    \draw [thick] (3.1,-7.46)-- (1.18,-7.04);
    \draw [thick] (1.18,-7.04)-- (0.24,-3.1);
    \draw [thick] (0.24,-3.1)-- (2.52,-4.74);
    \draw [thick] (4.92,-7.44)-- (8.04,-4.48);
    \draw [thick] (8.04,-4.48)-- (8.74,-6.56);
    \draw [thick] (8.74,-6.56)-- (7.38,-8);
    \draw [thick] (7.38,-8)-- (4.92,-7.44);
    \node at (2.76, -5.42) {\cir[gradeColor]{1}};
    \node at (7.14, -6.34) {\cir[gradeColor]{2}};
    \drawPoint{A}{2.52}{-4.74}
    \drawPoint{B}{3.34}{-2.54}
    \drawPoint{C}{5.54}{-4.46}
    \drawPoint{D}{3.10}{-7.46}
    \drawPoint{E}{1.18}{-7.04}
    \drawPoint{F}{0.24}{-3.10}
    \drawPoint{G}{4.92}{-7.44}
    \drawPoint{H}{8.04}{-4.48}
    \drawPoint{I}{8.74}{-6.56}
    \drawPoint{J}{7.38}{-8.00}
}
        Nom du polygone : \renewcommand{\theenumi}{\cir[gradeColor]{\arabic{enumi}}}
        \multiColEnumerate{1}{
            \item \awsr{$ABCDEF$ ou $FEDCB$ ou $CDEFAB$, etc.}
            \item \awsr{$GHIJ$ ou $IJGH$ ou $HIJG$, etc.}
        }
    }
}

\bookSlide{3p191}[10cm]

\bsec{Triangles}
\bsubsec{Définitions}
\slide{cr}{\bsmall
    \ssec\ssubsec
    \df{}{
    On appelle \key{triangle} un polygone à \key{trois cotés}.
    }[\wiki{Triangle}]

    \df{Triangles particuliers}{
        On appelle un triangle :
        \begin{itemize}
            \item qui a \key{un angle droit}, un \awsr{\key{triangle rectangle}}.
            \item qui a \key{deux côtés de même longueur}, un \awsr{\key{triangle isocèle}}.
            \item qui a \key{trois côtés de même longueur}, un \awsr{\key{triangle équilatéral}}.
        \end{itemize}
    }
}

\slide{cr}{\bshrink
    \expl{}{\bvspace{-1cm}
        \dividePage{
            % \ctikz[0.75]{
%     \node at (-9.25,2) {Main levée :};
%     \draw[gray!40] (-11,-4.5) rectangle (-1.5,2.5);
%     \draw [thick] (-7.732276223770383,1.4133499018531208)-- (-8.633853984852276,-3.695590744277629);
%     \draw [thick] (-8.035088611626442,-1.1672339199213526) -- (-8.331041596996219,-1.1150069225031567);
%     \draw [thick] (-10.452035803034093,-0.13435858800413536)-- (-7.732276223770383,1.4133499018531208);
%     \draw [thick] (-10.452035803034093,-0.13435858800413536)-- (-8.633853984852276,-3.695590744277629);
%     \draw [thick] (-9.409115044617078,-1.8466480342064568) -- (-9.676774743269291,-1.9833012980753089);
%     \draw [thick] (-2.4880989134773723,1.4283761978711529)-- (-6.878745987295947,-0.5982064066661978);
%     \draw [thick] (-4.559056188315157,0.30699128388583014) -- (-4.685000698583784,0.5798535369400059);
%     \draw [thick] (-4.681844202189537,0.25031625426494836) -- (-4.807788712458164,0.5231785073191241);
%     \draw [thick] (-6.878745987295947,-0.5982064066661978)-- (-3.4197292665953287,-0.8405945008516216);
%     \draw [thick] (-5.206186740444952,-0.5647783514991452) -- (-5.227194362095813,-0.864569126275788);
%     \draw [thick] (-5.071280891795464,-0.574231781242032) -- (-5.0922885134463245,-0.8740225560186748);
%     \draw [thick] (-3.4197292665953287,-0.8405945008516216)-- (-2.4880989134773723,1.4283761978711529);
%     \draw [thick] (-3.11859923198327,0.28841361768481283) -- (-2.8405952472545413,0.17426628620679063);
%     \draw [thick] (-3.0672329328181602,0.4135154108127409) -- (-2.7892289480894314,0.29936807933471865);
%     \draw [thick] (-7.582013263590067,-4.116327032782513)-- (-6.274725510021322,-1.6219618937892655);
%     \draw [thick] (-6.274725510021322,-1.6219618937892655)-- (-3.164282234288792,-2.914223351339985);
%     \draw [thick] (-7.582013263590067,-4.116327032782513)-- (-3.164282234288792,-2.914223351339985);
%     \draw [thick] (-6.413987417963006,-1.8876800169883443)-- (-6.130526271658642,-2.020522498657643);
%     \draw [thick] (-5.997683789989346,-1.7370613523532756)-- (-6.130526271658642,-2.020522498657643);
%     \node at (-5.96, -2.60) {\cir[gradeColor]{1}};
%     \node at (-9.3, -0.5) {\cir[gradeColor]{2}};
%     \node at (-4.25, 0.09) {\cir[gradeColor]{3}};
% }

\ctikz[\ifBA{0.75}{0.5}]{
    \node at (-9.25,2) {Main levée :};
    \draw[gray!40] (-11.5,-5.5) rectangle (-1.5,2.5);
    \node at (-5.96, -2.6) {\cir[gradeColor]{1}};
    \node at (-9.49, -0.18) {\cir[gradeColor]{2}};
    \node at (-4.25, 0.09) {\cir[gradeColor]{3}};
    \draw [penthick] (-7.73,1.41) -- (-8.63,-3.70) -- (-10.45,-0.13) -- (-7.73,1.41) -- cycle;
    \draw [thick] (-8.04,-1.17) -- (-8.33,-1.12);
    \draw [thick] (-9.41,-1.85) -- (-9.68,-1.98);
    \draw [penthick] (-2.49,1.43) -- (-6.88,-0.60) -- (-3.42,-0.84) -- (-2.49,1.43) -- cycle;
    \draw [thick] (-4.56,0.31) -- (-4.69,0.58);
    \draw [thick] (-4.68,0.25) -- (-4.81,0.52);
    \draw [thick] (-5.21,-0.56) -- (-5.23,-0.86);
    \draw [thick] (-5.07,-0.57) -- (-5.09,-0.87);
    \draw [thick] (-3.12,0.29) -- (-2.84,0.17);
    \draw [thick] (-3.07,0.41) -- (-2.79,0.30);
    \draw [penthick] (-7.58,-4.12) -- (-6.27,-1.62) -- (-3.16,-2.91) -- (-7.58,-4.12) -- cycle;
    \draw [thick] (-6.41,-1.89) -- (-6.13,-2.02) -- (-6.00,-1.74);
}
        }{
            \ctikz[\ifBA{0.75}{0.5}]{
    \node at (3.5,0) {Construction :};
    \draw[gray!40] (1.5,-7) rectangle (11,0.5);
    \draw [thick] (2.32,-1.32)-- (4.88,-2.22);
    \draw [thick] (2.32,-1.32)-- (3.3797065257830927,-6.487501437772536);
    \draw [thick] (3.3797065257830927,-6.487501437772536)-- (4.88,-2.22);
    \draw [thick] (5.96,-0.32)-- (10.08,-1.12);
    \draw [thick] (10.08,-1.12)-- (10.442377394648803,-5.30127763056312);
    \draw [thick] (10.442377394648803,-5.30127763056312)-- (5.96,-0.32);
    \draw [thick] (4.86,-4.22)-- (8.24,-3.62);
    \draw [thick] (8.24,-3.62)-- (7.069615242270663,-6.847165864791403);
    \draw [thick] (7.069615242270663,-6.847165864791403)-- (4.86,-4.22);
    \ifthenelse{\boolean{answer}}{
        \node at (6.62, -4.78) {\cir[answer]{3}};
        \node at (8.92, -1.94) {\cir[answer]{2}};
        \node at (3.50, -3.00) {\cir[answer]{1}};
    }{}
}
        }
        \renewcommand{\theenumi}{\cir[gradeColor]{\arabic{enumi}}}
        \multiColEnumerate{3}{
            \item \awsr{triangle rectangle}
            \item \awsr{triangle isocèle}
            \item \awsr{triangle equilatéral}
        }
    }
}

\scn{Triangles}
\caSlide{9-10-11}

\slide{exo}{
    \exo{}{
        Construire sur papier blanc un:
        \multiColEnumerate{1}{
            \item triangle rectangle (qui n'est pas un triangle isocèle)
            \item triangle isocèle (qui n'est ni un triangle rectangle, ni un triangle equilatéral)
            \item triangle equilatéral
            \item triangle rectangle isocèle
        }
    }
}

\scn{Quadrilatères}
\caSlide{13-14-15}

\bsec{Quadrilatères}
\bsubsec{Définitions}

\slide{cr}{\bsmall
    \ssec\ssubsec
    \df{}{
    On appelle \key{quadrilatère} un polygone à \key{quatre cotés}.
    }[\wiki{Quadrilatère}]
    \df{Quadrilatères particuliers}{
        On appelle un quadrilatère :
        \begin{itemize}
            \item qui a \key{4 côtés de même longueur}, un \awsr{\key{losange}}.
            \item qui a \key{4 angles droits}, un \awsr{\key{rectangle}}.
            \item qui est un \key{losange et un rectangle}, un \awsr{\key{carré}}.
        \end{itemize}
    }
}

\slide{cr}{\bshrink
    \expl{}{\bvspace{-1cm}
        \dividePage{
            % \ctikz[0.75]{
%     \node at (-6.25,3.5) {Main levée :};
%     \draw[gray!40] (-8,-3.5) rectangle (0,4);
%     \draw [thick] (-7.778284270200122,1.6478437265214079)-- (-4.394815927873779,3.0278437265214073);
%     \draw [thick] (-6.1432985052072855,2.4769788248944566) -- (-6.029801692866616,2.198708628148358);
%     \draw [thick] (-4.394815927873779,3.0278437265214073)-- (-5.071172734102863,-1.1925783757940112);
%     \draw [thick] (-4.584624557356605,0.893855218085332) -- (-4.881364104620036,0.9414101326420652);
%     \draw [thick] (-5.071172734102863,-1.1925783757940112)-- (-6.694641076429205,-1.7725783757940112);
%     \draw [thick] (-5.832353325081154,-1.6240820603136805) -- (-5.9334604854509125,-1.341074691274342);
%     \draw [thick] (-6.694641076429205,-1.7725783757940112)-- (-7.778284270200122,1.6478437265214079);
%     \draw [thick] (-7.379708555403346,-0.10774985945737921) -- (-7.093216791225981,-0.016984789815224242);
%     \draw [thick] (-2.8588484393142535,1.1979673519568315)-- (-3.3521741028902223,3.750406260534805);
%     \draw [thick] (-3.240212359237956,2.3792825761470513) -- (-2.94514709241024,2.436311666272112);
%     \draw [thick] (-3.265875449794234,2.512061946219524) -- (-2.9708101829665177,2.569091036344585);
%     \draw [thick] (-3.3521741028902223,3.750406260534805)-- (-0.6240366095212064,2.445287890171434);
%     \draw [thick] (-1.9842568465253285,3.2625783262098618) -- (-2.113949274909395,2.991477417269209);
%     \draw [thick] (-1.8622614375020343,3.2042167334370313) -- (-1.991953865886101,2.9331158244963778);
%     \draw [thick] (-0.6240366095212064,2.445287890171434)-- (-0.4783225189536222,0.8950249299911179);
%     \draw [thick] (-0.40790377211464945,1.7515397218053703) -- (-0.7071108946071335,1.7234163034788006);
%     \draw [thick] (-0.39524823386769464,1.6168965166837517) -- (-0.6944553563601786,1.588773098357182);
%     \draw [thick] (-0.4783225189536222,0.8950249299911179)-- (-2.8588484393142535,1.1979673519568315);
%     \draw [thick] (-1.6204773874786649,0.8888991627260577) -- (-1.5825388451813023,1.1870207751880801);
%     \draw [thick] (-1.7546321130865739,0.90597150675987) -- (-1.7166935707892113,1.2040931192218924);
%     \draw [thick] (-1.5443835803565311,0.45954921111946806)-- (-1.331499214534525,-1.1785274229902294);
%     \draw [thick] (-1.331499214534525,-1.1785274229902294)-- (-3.6514992145345255,-3.2785274229902317);
%     \draw [thick] (-3.6514992145345255,-3.2785274229902317)-- (-3.904383580356532,-0.8204507888805322);
%     \draw [thick] (-1.5443835803565311,0.45954921111946806)-- (-3.904383580356532,-0.8204507888805322);
%     \draw [thick] (-2.56124853967404,1.1600952021084898)-- (-2.5723026078732616,1.5035703227668247);
%     \draw [thick] (-2.5723026078732616,1.5035703227668247)-- (-2.915777728531596,1.4925162545676025);
%     \draw [thick] (-0.775922418593835,0.9328970798394597)-- (-0.7715652500358677,1.1980656085772077);
%     \draw [thick] (-0.7715652500358677,1.1980656085772077)-- (-0.5063967212981197,1.1937084400192413);
%     \draw [thick] (-0.5959624071767092,2.1466043801433137)-- (-0.9593872378552182,2.2113285261273194);
%     \draw [thick] (-0.9593872378552182,2.2113285261273194)-- (-0.8946630918712115,2.574753356805829);
%     \draw [thick] (-3.1058544991183616,3.4315504793458906)-- (-3.081547620540222,3.6209407939004112);
%     \draw [thick] (-3.2952448136728796,3.455857357924029)-- (-3.1058544991183616,3.4315504793458906);
%     \draw [thick] (-1.7341415797600948,0.08809944585310951)-- (-1.5057207266071018,0.16205100315856263);
%     \draw [thick] (-1.7341415797600948,0.08809944585310951)-- (-1.808093137065548,0.3165202990061031);
%     \draw [thick] (-1.711449496197932,-1.0385641926768407)-- (-1.553914518550412,-1.3798516205908173);
%     \draw [thick] (-1.3701620682839544,-0.8810292150293236)-- (-1.711449496197932,-1.0385641926768407);
%     \draw [thick] (-3.8736819392543795,-1.1188756692669284)-- (-3.5364510852010596,-1.0146527308204838);
%     \draw [thick] (-3.5364510852010596,-1.0146527308204838)-- (-3.6406740236475086,-0.6774218767671636);
%     \draw [thick] (-3.6822008556366774,-2.9801025426038352)-- (-3.5070920416847535,-2.9020944114377185);
%     \draw [thick] (-3.5070920416847535,-2.9020944114377185)-- (-3.4290839105186346,-3.077203225389641);
%     \node at (-6.00, 0.84) {\cir[gradeColor]{1}};
%     \node at (-2.74, -1.01) {\cir[gradeColor]{2}};
%     \node at (-1.93, 2.30) {\cir[gradeColor]{3}};
% }

\ctikz[0.75]{
    \node at (-6.25,4) {Main levée :};
    \draw[gray!40] (-8.5,-4.5) rectangle (0.5,4.5);
    \node at (-6, 0.84) {\cir[gradeColor]{1}};
    \node at (-2.74, -1.01) {\cir[gradeColor]{2}};
    \node at (-1.93, 2.3) {\cir[gradeColor]{3}};
    \draw [penthick] (-7.78,1.65) -- (-4.39,3.03) -- (-5.07,-1.19) -- (-6.69,-1.77) -- (-7.78,1.65) -- cycle;
    \draw [thick] (-6.14,2.48) -- (-6.03,2.20);
    \draw [thick] (-4.58,0.89) -- (-4.88,0.94);
    \draw [thick] (-5.83,-1.62) -- (-5.93,-1.34);
    \draw [thick] (-7.38,-0.11) -- (-7.09,-0.02);
    \draw [penthick] (-2.86,1.20) -- (-3.35,3.75) -- (-0.62,2.45) -- (-0.48,0.90) -- (-2.86,1.20) -- cycle;
    \draw [thick] (-3.24,2.38) -- (-2.95,2.44);
    \draw [thick] (-3.27,2.51) -- (-2.97,2.57);
    \draw [thick] (-1.98,3.26) -- (-2.11,2.99);
    \draw [thick] (-1.86,3.20) -- (-1.99,2.93);
    \draw [thick] (-0.41,1.75) -- (-0.71,1.72);
    \draw [thick] (-0.40,1.62) -- (-0.69,1.59);
    \draw [thick] (-1.62,0.89) -- (-1.58,1.19);
    \draw [thick] (-1.75,0.91) -- (-1.72,1.20);
    \draw [penthick] (-1.54,0.46) -- (-1.33,-1.18) -- (-3.65,-3.28) -- (-3.90,-0.82) -- (-1.54,0.46) -- cycle;
    \draw [thick] (-2.56,1.16) -- (-2.57,1.50) -- (-2.92,1.49);
    \draw [thick] (-0.78,0.93) -- (-0.77,1.20) -- (-0.51,1.19);
    \draw [thick] (-0.60,2.15) -- (-0.96,2.21) -- (-0.89,2.57);
    \draw [thick] (-3.11,3.43) -- (-3.08,3.62);
    \draw [thick] (-3.30,3.46) -- (-3.11,3.43);
    \draw [thick] (-1.73,0.09) -- (-1.51,0.16);
    \draw [thick] (-1.73,0.09) -- (-1.81,0.32);
    \draw [thick] (-1.71,-1.04) -- (-1.55,-1.38);
    \draw [thick] (-1.37,-0.88) -- (-1.71,-1.04);
    \draw [thick] (-3.87,-1.12) -- (-3.54,-1.01) -- (-3.64,-0.68);
    \draw [thick] (-3.68,-2.98) -- (-3.51,-2.90) -- (-3.43,-3.08);
}
        }{
            \ctikz[0.6]{
    \node at (1,0.5) {Construction :};
    \draw[gray!40] (-1,-9) rectangle (9,1);
    \draw [thick] (1.88,-2.04)-- (3.6,0.5);
    \draw [thick] (6.24,-3.1)-- (8.5,-4.54);
    \draw [thick] (0.1,-2.36)-- (2.62,-4.24);
    \draw [thick] (2.62,-4.24)-- (2.987350071573866,-7.362475608377858);
    \draw [thick] (0.1,-2.36)-- (0.46735007157386704,-5.482475608377857);
    \draw [thick] (0.46735007157386704,-5.482475608377857)-- (2.987350071573866,-7.362475608377858);
    \draw [thick] (3.6,0.5)-- (6.14,-1.22);
    \draw [thick] (6.14,-1.22)-- (4.42,-3.76);
    \draw [thick] (4.42,-3.76)-- (1.88,-2.04);
    \draw [thick] (3.7636940901242135,-6.986424552999498)-- (6.24,-3.1);
    \draw [thick] (8.5,-4.54)-- (6.023694090124214,-8.4264245529995);
    \draw [thick] (6.023694090124214,-8.4264245529995)-- (3.7636940901242135,-6.986424552999498);
    \ifthenelse{\boolean{answer}}{
        \node at (3.90, -1.36) {\cir[answer]{3}};
        \node at (6.08, -5.50) {\cir[answer]{2}};
        \node at (1.44, -4.68) {\cir[answer]{1}};
    }{}
}
        }
        \renewcommand{\theenumi}{\cir[gradeColor]{\arabic{enumi}}}
        \multiColEnumerate{3}{
            \item \awsr{losange}
            \item \awsr{rectangle}
            \item \awsr{carré}
        }
    }
}

\slide{exo}{
    \exo{}{
        Construire sur papier blanc un:
        \multiColEnumerate{1}{
            \item rectangle (qui n'est pas un carré)
            \item losange (qui n'est pas un carré)
            \item carré
        }
    }
}

% \slide{exo}{
%     \expl{}{
%         \ctikz[1]{
    \node at (-9.5,4.5) {Dessins à main levée :};
    \draw[gray!40] (-12,-5.5) rectangle (1,5);
    \draw [thick] (-7.883468342326343,2.88)-- (-4.5,4.26);
    \draw [thick] (-6.243323631318021,3.6964864530664077) -- (-6.140144711008322,3.443513546933592);
    \draw [thick] (-4.5,4.26)-- (-5.071172734102863,-0.4562898709104622);
    \draw [thick] (-4.649974548455078,1.8854316098114006) -- (-4.9211981856477856,1.9182785192781378);
    \draw [thick] (-5.071172734102863,-0.4562898709104622)-- (-6.694641076429205,-1.036289870910462);
    \draw [thick] (-5.83694910509796,-0.8749295841101609) -- (-5.928864705434106,-0.6176501577107623);
    \draw [thick] (-6.694641076429205,-1.036289870910462)-- (-7.883468342326343,2.88);
    \draw [thick] (-7.419767602056759,0.8821759151602174) -- (-7.1583418166987896,0.9615342139293213);
    \draw [thick] (-2.8588484393142535,1.1979673519568315)-- (-3.3521741028902223,3.750406260534805);
    \draw [thick] (-3.2279668057710738,2.3879102334287574) -- (-2.9597256541095134,2.4397548608151762);
    \draw [thick] (-3.251296888094963,2.50861875167646) -- (-2.9830557364334025,2.560463379062879);
    \draw [thick] (-3.3521741028902223,3.750406260534805)-- (-0.6240366095212064,2.445287890171434);
    \draw [thick] (-1.9846067110417263,3.247602757950158) -- (-2.102508918663605,3.001147386185928);
    \draw [thick] (-1.8737017937478224,3.194546764520312) -- (-1.991604001369701,2.9480913927560817);
    \draw [thick] (-0.6240366095212064,2.445287890171434)-- (-0.4783225189536222,0.8950249299911179);
    \draw [thick] (-0.42092884412580966,1.7441412389213617) -- (-0.6929353191189769,1.7185744949881168);
    \draw [thick] (-0.4094238093558507,1.6217383251744357) -- (-0.6814302843490179,1.5961715812411905);
    \draw [thick] (-0.4783225189536222,0.8950249299911179)-- (-2.8588484393142535,1.1979673519568315);
    \draw [thick] (-1.6248508503564172,0.9032261607485953) -- (-1.590361266449724,1.1742458084413427);
    \draw [thick] (-1.7468096918181528,0.9187464735066064) -- (-1.7123201079114594,1.189766121199354);
    \draw [thick] (0.16861416569906512,0.26420736288505703)-- (0.6219192678095761,-1.5692111194590517);
    \draw [thick] (0.6219192678095761,-1.5692111194590517)-- (-1.6980807321904232,-3.669211119459054);
    \draw [thick] (-1.6980807321904232,-3.669211119459054)-- (-2.1913858343009354,-1.0157926371149433);
    \draw [thick] (-9.120625640325118,2.9442783962844237)-- (-8.329666006420327,-2.391556587664774);
    \draw [thick] (-8.60804710844596,0.4180049133242753) -- (-8.878299399097637,0.37794395688188354);
    \draw [thick] (-8.590019678046884,0.29639138253102026) -- (-8.86027196869856,0.2563304260886285);
    \draw [thick] (-8.571992247647808,0.17477785173776522) -- (-8.842244538299484,0.13471689529537342);
    \draw [thick] (-11.150959633904789,-1.5441650160508096)-- (-9.120625640325118,2.9442783962844237);
    \draw [thick] (-11.150959633904789,-1.5441650160508096)-- (-8.329666006420327,-2.391556587664774);
    \draw [thick] (-9.818763586982225,-1.8016663028134772) -- (-9.897353954186883,-2.063323970417915);
    \draw [thick] (-9.701017636560227,-1.8370319680555733) -- (-9.779608003764887,-2.098689635660011);
    \draw [thick] (-9.58327168613823,-1.8723976332976695) -- (-9.66186205334289,-2.1340553009021073);
    \draw [thick] (-2.377299364797488,-0.0803128201625537)-- (-3.7532764155453844,-4.98588484393143);
    \draw [thick] (-2.9005582191659562,-2.4516171174257875) -- (-3.1636114944859153,-2.377832598880233);
    \draw [thick] (-2.9337612525114563,-2.569991091319769) -- (-3.1968145278314153,-2.4962065727742146);
    \draw [thick] (-2.9669642858569563,-2.6883650652137505) -- (-3.2300175611769153,-2.614580546668196);
    \draw [thick] (-3.7532764155453844,-4.98588484393143)-- (-4.181677481046373,-1.9460555972952576);
    \draw [thick] (-4.085586324685623,-3.6067725982069443) -- (-3.8150542625596757,-3.5686466989330987);
    \draw [thick] (-4.102742979358853,-3.4850331702502664) -- (-3.832210917232905,-3.446907270976421);
    \draw [thick] (-4.119899634032082,-3.3632937422935885) -- (-3.849367571906134,-3.325167843019743);
    \draw [thick] (-4.181677481046373,-1.9460555972952576)-- (-2.377299364797488,-0.0803128201625537);
    \draw [thick] (-3.4631503654563787,-1.006594429395902) -- (-3.266762364202518,-1.1965231891903854);
    \draw [thick] (-3.377682423548862,-0.9182198288316636) -- (-3.181294422295001,-1.1081485886261468);
    \draw [thick] (-3.292214481641345,-0.8298452282674252) -- (-3.095826480387484,-1.0197739880619086);
    \draw [thick] (-8.116510145152326,-1.3852244314524182)-- (-4.377831093175001,-2.430644826234202);
    \draw [thick] (-4.377831093175001,-2.430644826234202)-- (-7.349454272249164,-4.50812786011886);
    \draw [thick] (-8.116510145152326,-1.3852244314524182)-- (-7.349454272249164,-4.50812786011886);
    \draw [thick] (0.16861416569906512,0.26420736288505703)-- (-2.1913858343009354,-1.0157926371149433);
    \draw [thick] (-2.56124853967404,1.1600952021084898)-- (-2.5723026078732616,1.5035703227668247);
    \draw [thick] (-2.5723026078732616,1.5035703227668247)-- (-2.915777728531596,1.4925162545676025);
    \draw [thick] (-0.775922418593835,0.9328970798394597)-- (-0.7715652500358677,1.1980656085772077);
    \draw [thick] (-0.7715652500358677,1.1980656085772077)-- (-0.5063967212981197,1.1937084400192413);
    \draw [thick] (-0.5959624071767092,2.1466043801433137)-- (-0.9593872378552182,2.2113285261273194);
    \draw [thick] (-0.9593872378552182,2.2113285261273194)-- (-0.8946630918712115,2.574753356805829);
    \draw [thick] (-3.1058544991183616,3.4315504793458906)-- (-3.081547620540222,3.6209407939004112);
    \draw [thick] (-3.2952448136728796,3.455857357924029)-- (-3.1058544991183616,3.4315504793458906);
    \draw [thick] (-0.0013386406625495312,-0.12077988642929871)-- (0.24061969653844031,-0.027023136081894893);
    \draw [thick] (-0.0013386406625495312,-0.12077988642929871)-- (-0.09509539100995336,0.12117845077169109);
    \draw [thick] (0.2284315020981753,-1.4490530821876138)-- (0.3995039637936888,-1.7705353170596394);
    \draw [thick] (0.5499137369702007,-1.2779806204920994)-- (0.2284315020981753,-1.4490530821876138);
    \draw [thick] (-2.1365515091960465,-1.3107387284440042)-- (-1.8131263916727698,-1.196188842524855);
    \draw [thick] (-1.8131263916727698,-1.196188842524855)-- (-1.927676277591915,-0.8727637250015763);
    \draw [thick] (-1.7529150572953127,-3.3742650281299937)-- (-1.567479295870688,-3.282451160433842);
    \draw [thick] (-1.567479295870688,-3.282451160433842)-- (-1.4756654281745345,-3.4678869218584643);
    \draw [thick] (-4.666748567503779,-2.3498568752298485)-- (-4.771565808147926,-2.497718925392548);
    \draw [thick] (-4.6237037579852345,-2.6025361660367183)-- (-4.771565808147926,-2.497718925392548);
    \node at (-9.70, -0.27) {\cir[gradeColor]{1}};
    \node at (-6.11, 1.82) {\cir[gradeColor]{2}};
    \node at (-1.93, 2.28) {\cir[gradeColor]{3}};
    \node at (-6.72, -2.67) {\cir[gradeColor]{4}};
    \node at (-3.61, -2.07) {\cir[gradeColor]{5}};
    \node at (-0.78, -1.37) {\cir[gradeColor]{6}};
}
%         \renewcommand{\theenumi}{\cir[gradeColor]{\arabic{enumi}}}
%         \multiColEnumerate{3}{
%             \item \awsr{triangle isocèle}
%             \item \awsr{losange}
%             \item \awsr{carré}
%             \item \awsr{triangle rectangle}
%             \item \awsr{triangle isocèle}
%             \item \awsr{rectangle}
%         }
%         \ctikz[1]{
    \node at (2,2) {Construction représentative :};
    \draw[gray!40] (-1.5,-9.5) rectangle (12,2.5);
    \draw [thick] (-0.26,-0.98)-- (1.46,1.56);
    \draw [thick] (6.02,-3)-- (8.7,-3.64);
    \draw [thick] (2.12,-5.12)-- (4.44,-2.92);
    \draw [thick] (4.44,-2.92)-- (4.351711889479146,-6.116029600854889);
    \draw [thick] (2.12,-5.12)-- (2.031711889479145,-8.316029600854888);
    \draw [thick] (2.031711889479145,-8.316029600854888)-- (4.351711889479146,-6.116029600854889);
    \draw [thick] (-0.88,-2.42)-- (1.64,-3.88);
    \draw [thick] (-0.88,-2.42)-- (-0.6501246882793013,-7.832817955112219);
    \draw [thick] (-0.6501246882793013,-7.832817955112219)-- (1.64,-3.88);
    \draw [thick] (8.54,-0.98)-- (11.42,-3.54);
    \draw [thick] (11.42,-3.54)-- (10.96179990249823,-7.365970814139775);
    \draw [thick] (10.96179990249823,-7.365970814139775)-- (8.54,-0.98);
    \draw [thick] (4.9,0.36)-- (8.28,0.96);
    \draw [thick] (8.28,0.96)-- (7.109615242270663,-2.267165864791401);
    \draw [thick] (7.109615242270663,-2.267165864791401)-- (4.9,0.36);
    \draw [thick] (1.46,1.56)-- (4,-0.16);
    \draw [thick] (4,-0.16)-- (2.28,-2.7);
    \draw [thick] (2.28,-2.7)-- (-0.26,-0.98);
    \draw [thick] (4.731700711498737,-8.394753270599033)-- (6.02,-3);
    \draw [thick] (8.7,-3.64)-- (7.411700711498736,-9.034753270599033);
    \draw [thick] (7.411700711498736,-9.034753270599033)-- (4.731700711498737,-8.394753270599033);
    \ifthenelse{\boolean{answer}}{
        \node at (-0.04, -4.42) {\cir[answer]{4}};
        \node at (1.72, -0.34) {\cir[answer]{3}};
        \node at (6.72, -0.12) {\cir[answer]{4}};
        \node at (10.34, -3.70) {\cir[answer]{5}};
        \node at (6.66, -5.82) {\cir[answer]{3}};
        \node at (3.18, -5.54) {\cir[answer]{2}};
    }{}
}
%     }
% }

% \slide{exo}{
%     \exo{}{
%         Construire sur papier blanc un:
%         \multiColEnumerate{2}{
%             \item rectangle (qui n'est pas un carré)
%             \item losange (qui n'est pas un carré)
%             \item carré
%             \item triangle rectangle (qui n'est pas un triangle isocèle)
%             \item triangle isocèle (qui n'est ni un triangle rectangle, ni un triangle equilatéral)
%             \item triangle equilatéral
%             \item triangle rectangle isocèle
%         }
%     }
% }

\slide{exo}{
    \exo{}{
        Un triangle equilatéral a le même périmètre qu'un carré de \Lg{6} de côté.
        Quelle est la longueur d'un côté de ce triangle ?
    }[\dmeepc{6}[161]]
}