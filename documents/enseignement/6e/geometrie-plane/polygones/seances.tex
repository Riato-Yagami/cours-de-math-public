%%% VARIABLES %%%
\setSeq{5}{Géométrie plane - Polygones}
\setGrade{6e}
\def\imgPath{enseignement/6e/geometrie-plane/polygones/}
\def\ym{\href{https://www.maths-et-tiques.fr/telech/19Quad.pdf}{Yvan Monka}}
% Triangle https://www.maths-et-tiques.fr/telech/19Triang.pdf
% \forPrint
\def\caPrefix{6e-juin-2022-}
%%

\obj{
    \item Savoir coder des figures simples :
    \begin{itemize}
        \item triangles (et particuliers : rectangle, isocèle, équilatéral)
        \item quadrilatères (et particuliers : carré, rectangle, losange)
    \end{itemize}
    \item Connaître et utiliser le vocabulaire associé à ces figures et leurs propriétés.
    \item Reconnaître, nommer et décrire des figures complexes (assemblages de figures simples).
    \item Représenter, reproduire, tracer ou construire des figures simples.
    \item Représenter, reproduire, tracer ou construire des figures complexes (assemblages de figures simples).
    \item Réaliser, compléter ou rédiger un programme de construction d'une figure plane.
}

\scn{Définir un polygone}

\caSlide{6-7-8}

\bsec{Définitions}

\slide{cr}{\bsmall
    \sseq\ssec
    \df{}{
        On appelle \key{polygone} une ligne brisée fermée.
    }[\wiki{Polygone}]

    \df{}{
        On appelle :
        \begin{itemize}
            \item \nswr{\key{côtés}} les segments qui compose un polygone.
            \item \nswr{\key{sommets}} les extremités des côtés.
            \item \nswr{\key{diagonale}} un segment reliant deux sommets non consécutifs.
        \end{itemize}
    }[\href{https://www.desmaths.fr/cours/index.php?n=6e&c=09}{Desmaths.fr}]
}

\slide{cr}{
    \expl{}{
        % TODO quadrilatere plutot que pentagone
        \ctikz[0.5]{
    \draw[gray!40] (0,-6.5) rectangle (13,1);
    \draw [thick] (2.52,-2.06)-- (8.82,0.24);
    \draw [thick] (8.82,0.24)-- (12.48,-2.42);
    \draw [thick] (12.48,-2.42)-- (6.82,-5.86);
    \draw [thick] (6.82,-5.86)-- (0.82,-5.32);
    \draw [thick] (0.82,-5.32)-- (2.52,-2.06);
}
        Ce polygone à :
        \multiColEnumerate{3}{
            \item \nswr{4} sommets
            \item \nswr{4} côtés
            \item \nswr{2} diagonales
        }
    }
}

\slide{cr}{
    \rmk{}{Un polygone possède autant de sommets que de côtés,
    mais en général,
    il ne possède pas autant de diagonale
    }
}

% \newpage

\slide{cr}{\bshrink
    \rl{}{
        Le nom d'un polygone est donné par ses sommets.
        On part d'un sommet et on fait le tour du polygone,
        dans un sens ou dans l'autre.
    }\bvspace{-0.5cm}

    \expl{}{\bvspace{-1cm}
        \ctikz[\ifBA{0.6}{1}]{
    \draw[gray!40] (-0.5,-9) rectangle (9.5,-2);
    \draw [thick] (2.52,-4.74)-- (3.34,-2.54);
    \draw [thick] (3.34,-2.54)-- (5.54,-4.46);
    \draw [thick] (5.54,-4.46)-- (3.1,-7.46);
    \draw [thick] (3.1,-7.46)-- (1.18,-7.04);
    \draw [thick] (1.18,-7.04)-- (0.24,-3.1);
    \draw [thick] (0.24,-3.1)-- (2.52,-4.74);
    \draw [thick] (4.92,-7.44)-- (8.04,-4.48);
    \draw [thick] (8.04,-4.48)-- (8.74,-6.56);
    \draw [thick] (8.74,-6.56)-- (7.38,-8);
    \draw [thick] (7.38,-8)-- (4.92,-7.44);
    \node at (2.76, -5.42) {\cir[gradeColor]{1}};
    \node at (7.14, -6.34) {\cir[gradeColor]{2}};
    \drawPoint{A}{2.52}{-4.74}
    \drawPoint{B}{3.34}{-2.54}
    \drawPoint{C}{5.54}{-4.46}
    \drawPoint{D}{3.10}{-7.46}
    \drawPoint{E}{1.18}{-7.04}
    \drawPoint{F}{0.24}{-3.10}
    \drawPoint{G}{4.92}{-7.44}
    \drawPoint{H}{8.04}{-4.48}
    \drawPoint{I}{8.74}{-6.56}
    \drawPoint{J}{7.38}{-8.00}
}
        \bvspace{-0.5cm}Nom du polygone : \renewcommand{\theenumi}{\cir[gradeColor]{\arabic{enumi}}}
        \bvspace{-0.25cm}\multiColEnumerate{1}{
            \item \nswr{$ABCDEF$ ou $FEDCB$ ou $CDEFAB$, etc.}
            \item \nswr{$GHIJ$ ou $IJGH$ ou $HIJG$, etc.}
        }
    }
}

\bookSlide{3p191}[6cm]

\scn{Définitions et codage de triangles particulier}

\slide{qf}{
    En ecrivant toutes les étapes;
    resoudre ce calcul :\\
    $(24 + 2 \times 6) \div 2 = \nswr[0]{(24 + 12) \div 2 = 36 \div 2 = 18}$
}

\bsec{Triangles}
\bsubsec{Définitions}
\slide{cr}{\bsmall
    \ssec\ssubsec
    \df{}{
    On appelle \key{triangle} un polygone à \key{trois côtés}.
    }[\wiki{Triangle}]

    \df{Triangles particuliers}{
        On appelle un triangle :
        \begin{itemize}
            \item qui a \key{un angle droit}, un \nswr{\key{triangle rectangle}}.
            \item qui a \key{deux côtés de même longueur}, un \nswr{\key{triangle isocèle}}.
            \item qui a \key{trois côtés de même longueur}, un \nswr{\key{triangle équilatéral}}.
        \end{itemize}
    }
}

\slide{cr}{\bshrink
    \expl{}{\bvspace{-1cm}
        \dividePage{
            % \ctikz[0.75]{
%     \node at (-9.25,2) {Main levée :};
%     \draw[gray!40] (-11,-4.5) rectangle (-1.5,2.5);
%     \draw [thick] (-7.732276223770383,1.4133499018531208)-- (-8.633853984852276,-3.695590744277629);
%     \draw [thick] (-8.035088611626442,-1.1672339199213526) -- (-8.331041596996219,-1.1150069225031567);
%     \draw [thick] (-10.452035803034093,-0.13435858800413536)-- (-7.732276223770383,1.4133499018531208);
%     \draw [thick] (-10.452035803034093,-0.13435858800413536)-- (-8.633853984852276,-3.695590744277629);
%     \draw [thick] (-9.409115044617078,-1.8466480342064568) -- (-9.676774743269291,-1.9833012980753089);
%     \draw [thick] (-2.4880989134773723,1.4283761978711529)-- (-6.878745987295947,-0.5982064066661978);
%     \draw [thick] (-4.559056188315157,0.30699128388583014) -- (-4.685000698583784,0.5798535369400059);
%     \draw [thick] (-4.681844202189537,0.25031625426494836) -- (-4.807788712458164,0.5231785073191241);
%     \draw [thick] (-6.878745987295947,-0.5982064066661978)-- (-3.4197292665953287,-0.8405945008516216);
%     \draw [thick] (-5.206186740444952,-0.5647783514991452) -- (-5.227194362095813,-0.864569126275788);
%     \draw [thick] (-5.071280891795464,-0.574231781242032) -- (-5.0922885134463245,-0.8740225560186748);
%     \draw [thick] (-3.4197292665953287,-0.8405945008516216)-- (-2.4880989134773723,1.4283761978711529);
%     \draw [thick] (-3.11859923198327,0.28841361768481283) -- (-2.8405952472545413,0.17426628620679063);
%     \draw [thick] (-3.0672329328181602,0.4135154108127409) -- (-2.7892289480894314,0.29936807933471865);
%     \draw [thick] (-7.582013263590067,-4.116327032782513)-- (-6.274725510021322,-1.6219618937892655);
%     \draw [thick] (-6.274725510021322,-1.6219618937892655)-- (-3.164282234288792,-2.914223351339985);
%     \draw [thick] (-7.582013263590067,-4.116327032782513)-- (-3.164282234288792,-2.914223351339985);
%     \draw [thick] (-6.413987417963006,-1.8876800169883443)-- (-6.130526271658642,-2.020522498657643);
%     \draw [thick] (-5.997683789989346,-1.7370613523532756)-- (-6.130526271658642,-2.020522498657643);
%     \node at (-5.96, -2.60) {\cir[gradeColor]{1}};
%     \node at (-9.3, -0.5) {\cir[gradeColor]{2}};
%     \node at (-4.25, 0.09) {\cir[gradeColor]{3}};
% }

\ctikz[0.75]{
    \node at (-9.25,2) {Main levée :};
    \draw[gray!40] (-11.5,-5.5) rectangle (-1.5,2.5);
    \node at (-5.96, -2.6) {\cir[gradeColor]{1}};
    \node at (-9.49, -0.18) {\cir[gradeColor]{2}};
    \node at (-4.25, 0.09) {\cir[gradeColor]{3}};
    \draw [penthick] (-7.73,1.41) -- (-8.63,-3.70) -- (-10.45,-0.13) -- (-7.73,1.41) -- cycle;
    \draw [thick] (-8.04,-1.17) -- (-8.33,-1.12);
    \draw [thick] (-9.41,-1.85) -- (-9.68,-1.98);
    \draw [penthick] (-2.49,1.43) -- (-6.88,-0.60) -- (-3.42,-0.84) -- (-2.49,1.43) -- cycle;
    \draw [thick] (-4.56,0.31) -- (-4.69,0.58);
    \draw [thick] (-4.68,0.25) -- (-4.81,0.52);
    \draw [thick] (-5.21,-0.56) -- (-5.23,-0.86);
    \draw [thick] (-5.07,-0.57) -- (-5.09,-0.87);
    \draw [thick] (-3.12,0.29) -- (-2.84,0.17);
    \draw [thick] (-3.07,0.41) -- (-2.79,0.30);
    \draw [penthick] (-7.58,-4.12) -- (-6.27,-1.62) -- (-3.16,-2.91) -- (-7.58,-4.12) -- cycle;
    \draw [thick] (-6.41,-1.89) -- (-6.13,-2.02) -- (-6.00,-1.74);
}
        }{
            \ctikz[\ifBA{0.75}{0.5}]{
    \node at (3.5,0) {Construction :};
    \draw[gray!40] (1.5,-7) rectangle (11,0.5);
    \draw [thick] (2.32,-1.32)-- (4.88,-2.22);
    \draw [thick] (2.32,-1.32)-- (3.3797065257830927,-6.487501437772536);
    \draw [thick] (3.3797065257830927,-6.487501437772536)-- (4.88,-2.22);
    \draw [thick] (5.96,-0.32)-- (10.08,-1.12);
    \draw [thick] (10.08,-1.12)-- (10.442377394648803,-5.30127763056312);
    \draw [thick] (10.442377394648803,-5.30127763056312)-- (5.96,-0.32);
    \draw [thick] (4.86,-4.22)-- (8.24,-3.62);
    \draw [thick] (8.24,-3.62)-- (7.069615242270663,-6.847165864791403);
    \draw [thick] (7.069615242270663,-6.847165864791403)-- (4.86,-4.22);
    \ifthenelse{\boolean{answer}}{
        \node at (6.62, -4.78) {\cir[answer]{3}};
        \node at (8.92, -1.94) {\cir[answer]{2}};
        \node at (3.50, -3.00) {\cir[answer]{1}};
    }{}
}
        }
        \renewcommand{\theenumi}{\cir[gradeColor]{\arabic{enumi}}}
        \multiColEnumerate{3}{
            \item \nswr{triangle rectangle}
            \item \nswr{triangle isocèle}
            \item \nswr{triangle equilatéral}
        }
    }
}

\scn{Construction de triangles particuliers}
\caSlide{9-10-11}

\slide{exo}{\bshrink
    \act{}{
        \begin{enumerate}
            \item Construisez, sur une feuille de papier blanc,
            les figures suivantes à l'aide d'une règle non graduée,
            d'un compas et d'une équerre :
            \multiColEnumerate{1}{
                \item triangle rectangle (qui n'est pas un triangle isocèle)
                \item triangle isocèle (qui n'est ni un triangle rectangle, ni un triangle equilatéral)
                \item triangle equilatéral
                \item triangle rectangle isocèle
            }
            \item À l'aide d'un rapporteur, mesurez les angles de vos figures.
            \item Codez les figures en fonction de vos observations.
            Codez les figures en fonction de vos observations.
            \hint{deux angles égaux sont représentés par deux arcs de cercle portant les mêmes symboles}
        \end{enumerate}
    }
}

\bookSlide{27p195}

\scn{Propriétés des triangles particuliers}
\caSlide{13-14-15}

\bsubsec{Propriétés}

\slide{cr}{

\ssubsec
    \pr{}{
        \begin{enumerate}
            \item Les angles à la base d'un triangle isocèle sont égaux.
            \item Tous les angles d'un triangle équilatéral valent \ang{60}.
        \end{enumerate}
    }[\wiki{Triangle_isocèle}[Propriétés] et \wiki{Triangle_équilatéral}[Propriétés]]

    \rmk{}{
        \Sialors{un  triangle est équilatéral}{il est isocèle}
    }
}

\bsubsec{Hauteur d'un triangle}

\slide{cr}{
    \ssubsec

    \df{}{
    On appelle \key{hauteur} d'un triangle,
    \key{une droite} passant par un sommet et \key{coupant perpendiculairement} la droite porté par le côté opposé à ce sommet.
}[\wiki{Hauteur_d'un_triangle}]
}

\slide{cr}{\bshrink
    \expl{}{
        Tracer les hauteurs des triangles suivants : \vspace{-0.5cm}
        \ctikz[\ifBA{0.5}{1}]{
    \draw[gray!40] (-10.5,-2.5) rectangle (7.5,9.5);
    \draw [thick] (-6.10,5) -- (-8.46,0.96) -- (-3.32,2.64) -- (-6.10,5) -- cycle;
    \draw [thick] (2.78,7.20) -- (2.16,2.02) -- (-1.08,1.36) -- (2.78,7.20) -- cycle;
    \nswr[0]{
        \draw[thick,color=gradeColor,fill=gradeColor,fill opacity=0.10] (4.18,2.43) -- (4.10,2.85) -- (3.68,2.76) -- (3.77,2.35) -- cycle;
        \draw[thick,color=gradeColor,fill=gradeColor,fill opacity=0.10] (0.56,3.08) -- (0.80,3.43) -- (0.44,3.66) -- (0.21,3.31) -- cycle;
        \draw[thick,color=gradeColor,fill=gradeColor,fill opacity=0.10] (-5.20,4.80) -- (-5.53,5.07) -- (-5.80,4.75) -- (-5.48,4.47) -- cycle;
        \draw[thick,color=gradeColor,fill=gradeColor,fill opacity=0.10] (-6.63,4.08) -- (-6.27,3.87) -- (-6.05,4.24) -- (-6.42,4.45) -- cycle;
        \draw[thick,color=gradeColor,fill=gradeColor,fill opacity=0.10] (-4.73,2.18) -- (-4.86,2.58) -- (-5.27,2.45) -- (-5.13,2.05) -- cycle;
        \draw[thick,color=gradeColor,fill=gradeColor,fill opacity=0.10] (1.62,1.04) -- (1.56,0.62) -- (1.99,0.57) -- (2.04,0.99) -- cycle;
        \draw [thick,answer] (-8.96,5.93)-- (-1.97,1.85);
        \draw [thick,answer] (-4.11,-1.08)-- (-6.51,6.25);
        \draw [thick,answer] (-9.92,-0.76)-- (-4.22,5.96);
        \draw [thick,answer] (-1.76,4.61)-- (6.68,-0.97);
        \draw [thick,answer] (4.49,-1.21)-- (2.56,8.26);
        \draw [thick,answer] (-2.99,1.59)-- (6.88,0.41);
        \draw [thick,dashed] (2.94,8.58)-- (1.72,-1.69);
        \draw [thick,dashed] (6.05,2.81)-- (-3.19,0.93);
    }
}
    }
    \rmk{}{
    Un triangle possède \nswr{trois} hauteurs qui sont concourantes en un point appelé \key{orthocentre}.
}[][\cmdGeoGebra[fjehbrn3]]
}

\scn{Définitions et codage des quadrilatères}
\caSlide{16-17-18}

\bsec{Quadrilatères}
\bsubsec{Définitions}

\slide{cr}{\bsmall
    \ssec\ssubsec
    \df{}{
    On appelle \key{quadrilatère} un polygone à \key{quatre côtés}.
    }[\wiki{Quadrilatère}]
    \df{Quadrilatères particuliers}{
        On appelle un quadrilatère :
        \begin{itemize}
            \item qui a \key{4 côtés de même longueur}, un \nswr{\key{losange}}.
            \item qui a \key{4 angles droits}, un \nswr{\key{rectangle}}.
            \item qui est un \key{losange et un rectangle}, un \nswr{\key{carré}}.
        \end{itemize}
    }
}

\slide{cr}{\bshrink
    \expl{}{\bvspace{-1cm}
        \dividePage{
            % \ctikz[0.75]{
%     \node at (-6.25,3.5) {Main levée :};
%     \draw[gray!40] (-8,-3.5) rectangle (0,4);
%     \draw [thick] (-7.778284270200122,1.6478437265214079)-- (-4.394815927873779,3.0278437265214073);
%     \draw [thick] (-6.1432985052072855,2.4769788248944566) -- (-6.029801692866616,2.198708628148358);
%     \draw [thick] (-4.394815927873779,3.0278437265214073)-- (-5.071172734102863,-1.1925783757940112);
%     \draw [thick] (-4.584624557356605,0.893855218085332) -- (-4.881364104620036,0.9414101326420652);
%     \draw [thick] (-5.071172734102863,-1.1925783757940112)-- (-6.694641076429205,-1.7725783757940112);
%     \draw [thick] (-5.832353325081154,-1.6240820603136805) -- (-5.9334604854509125,-1.341074691274342);
%     \draw [thick] (-6.694641076429205,-1.7725783757940112)-- (-7.778284270200122,1.6478437265214079);
%     \draw [thick] (-7.379708555403346,-0.10774985945737921) -- (-7.093216791225981,-0.016984789815224242);
%     \draw [thick] (-2.8588484393142535,1.1979673519568315)-- (-3.3521741028902223,3.750406260534805);
%     \draw [thick] (-3.240212359237956,2.3792825761470513) -- (-2.94514709241024,2.436311666272112);
%     \draw [thick] (-3.265875449794234,2.512061946219524) -- (-2.9708101829665177,2.569091036344585);
%     \draw [thick] (-3.3521741028902223,3.750406260534805)-- (-0.6240366095212064,2.445287890171434);
%     \draw [thick] (-1.9842568465253285,3.2625783262098618) -- (-2.113949274909395,2.991477417269209);
%     \draw [thick] (-1.8622614375020343,3.2042167334370313) -- (-1.991953865886101,2.9331158244963778);
%     \draw [thick] (-0.6240366095212064,2.445287890171434)-- (-0.4783225189536222,0.8950249299911179);
%     \draw [thick] (-0.40790377211464945,1.7515397218053703) -- (-0.7071108946071335,1.7234163034788006);
%     \draw [thick] (-0.39524823386769464,1.6168965166837517) -- (-0.6944553563601786,1.588773098357182);
%     \draw [thick] (-0.4783225189536222,0.8950249299911179)-- (-2.8588484393142535,1.1979673519568315);
%     \draw [thick] (-1.6204773874786649,0.8888991627260577) -- (-1.5825388451813023,1.1870207751880801);
%     \draw [thick] (-1.7546321130865739,0.90597150675987) -- (-1.7166935707892113,1.2040931192218924);
%     \draw [thick] (-1.5443835803565311,0.45954921111946806)-- (-1.331499214534525,-1.1785274229902294);
%     \draw [thick] (-1.331499214534525,-1.1785274229902294)-- (-3.6514992145345255,-3.2785274229902317);
%     \draw [thick] (-3.6514992145345255,-3.2785274229902317)-- (-3.904383580356532,-0.8204507888805322);
%     \draw [thick] (-1.5443835803565311,0.45954921111946806)-- (-3.904383580356532,-0.8204507888805322);
%     \draw [thick] (-2.56124853967404,1.1600952021084898)-- (-2.5723026078732616,1.5035703227668247);
%     \draw [thick] (-2.5723026078732616,1.5035703227668247)-- (-2.915777728531596,1.4925162545676025);
%     \draw [thick] (-0.775922418593835,0.9328970798394597)-- (-0.7715652500358677,1.1980656085772077);
%     \draw [thick] (-0.7715652500358677,1.1980656085772077)-- (-0.5063967212981197,1.1937084400192413);
%     \draw [thick] (-0.5959624071767092,2.1466043801433137)-- (-0.9593872378552182,2.2113285261273194);
%     \draw [thick] (-0.9593872378552182,2.2113285261273194)-- (-0.8946630918712115,2.574753356805829);
%     \draw [thick] (-3.1058544991183616,3.4315504793458906)-- (-3.081547620540222,3.6209407939004112);
%     \draw [thick] (-3.2952448136728796,3.455857357924029)-- (-3.1058544991183616,3.4315504793458906);
%     \draw [thick] (-1.7341415797600948,0.08809944585310951)-- (-1.5057207266071018,0.16205100315856263);
%     \draw [thick] (-1.7341415797600948,0.08809944585310951)-- (-1.808093137065548,0.3165202990061031);
%     \draw [thick] (-1.711449496197932,-1.0385641926768407)-- (-1.553914518550412,-1.3798516205908173);
%     \draw [thick] (-1.3701620682839544,-0.8810292150293236)-- (-1.711449496197932,-1.0385641926768407);
%     \draw [thick] (-3.8736819392543795,-1.1188756692669284)-- (-3.5364510852010596,-1.0146527308204838);
%     \draw [thick] (-3.5364510852010596,-1.0146527308204838)-- (-3.6406740236475086,-0.6774218767671636);
%     \draw [thick] (-3.6822008556366774,-2.9801025426038352)-- (-3.5070920416847535,-2.9020944114377185);
%     \draw [thick] (-3.5070920416847535,-2.9020944114377185)-- (-3.4290839105186346,-3.077203225389641);
%     \node at (-6.00, 0.84) {\cir[gradeColor]{1}};
%     \node at (-2.74, -1.01) {\cir[gradeColor]{2}};
%     \node at (-1.93, 2.30) {\cir[gradeColor]{3}};
% }

\ctikz[0.75]{
    \node at (-6.25,4) {Main levée :};
    \draw[gray!40] (-8.5,-4.5) rectangle (0.5,4.5);
    \node at (-6, 0.84) {\cir[gradeColor]{1}};
    \node at (-2.74, -1.01) {\cir[gradeColor]{2}};
    \node at (-1.93, 2.3) {\cir[gradeColor]{3}};
    \draw [penthick] (-7.78,1.65) -- (-4.39,3.03) -- (-5.07,-1.19) -- (-6.69,-1.77) -- (-7.78,1.65) -- cycle;
    \draw [thick] (-6.14,2.48) -- (-6.03,2.20);
    \draw [thick] (-4.58,0.89) -- (-4.88,0.94);
    \draw [thick] (-5.83,-1.62) -- (-5.93,-1.34);
    \draw [thick] (-7.38,-0.11) -- (-7.09,-0.02);
    \draw [penthick] (-2.86,1.20) -- (-3.35,3.75) -- (-0.62,2.45) -- (-0.48,0.90) -- (-2.86,1.20) -- cycle;
    \draw [thick] (-3.24,2.38) -- (-2.95,2.44);
    \draw [thick] (-3.27,2.51) -- (-2.97,2.57);
    \draw [thick] (-1.98,3.26) -- (-2.11,2.99);
    \draw [thick] (-1.86,3.20) -- (-1.99,2.93);
    \draw [thick] (-0.41,1.75) -- (-0.71,1.72);
    \draw [thick] (-0.40,1.62) -- (-0.69,1.59);
    \draw [thick] (-1.62,0.89) -- (-1.58,1.19);
    \draw [thick] (-1.75,0.91) -- (-1.72,1.20);
    \draw [penthick] (-1.54,0.46) -- (-1.33,-1.18) -- (-3.65,-3.28) -- (-3.90,-0.82) -- (-1.54,0.46) -- cycle;
    \draw [thick] (-2.56,1.16) -- (-2.57,1.50) -- (-2.92,1.49);
    \draw [thick] (-0.78,0.93) -- (-0.77,1.20) -- (-0.51,1.19);
    \draw [thick] (-0.60,2.15) -- (-0.96,2.21) -- (-0.89,2.57);
    \draw [thick] (-3.11,3.43) -- (-3.08,3.62);
    \draw [thick] (-3.30,3.46) -- (-3.11,3.43);
    \draw [thick] (-1.73,0.09) -- (-1.51,0.16);
    \draw [thick] (-1.73,0.09) -- (-1.81,0.32);
    \draw [thick] (-1.71,-1.04) -- (-1.55,-1.38);
    \draw [thick] (-1.37,-0.88) -- (-1.71,-1.04);
    \draw [thick] (-3.87,-1.12) -- (-3.54,-1.01) -- (-3.64,-0.68);
    \draw [thick] (-3.68,-2.98) -- (-3.51,-2.90) -- (-3.43,-3.08);
}
        }{
            \ctikz[\ifBA{0.75}{0.5}]{
    \node at (1,0.5) {Construction :};
    \draw[gray!40] (-1,-9) rectangle (9,1);
    \draw [thick] (1.88,-2.04)-- (3.6,0.5);
    \draw [thick] (6.24,-3.1)-- (8.5,-4.54);
    \draw [thick] (0.1,-2.36)-- (2.62,-4.24);
    \draw [thick] (2.62,-4.24)-- (2.987350071573866,-7.362475608377858);
    \draw [thick] (0.1,-2.36)-- (0.46735007157386704,-5.482475608377857);
    \draw [thick] (0.46735007157386704,-5.482475608377857)-- (2.987350071573866,-7.362475608377858);
    \draw [thick] (3.6,0.5)-- (6.14,-1.22);
    \draw [thick] (6.14,-1.22)-- (4.42,-3.76);
    \draw [thick] (4.42,-3.76)-- (1.88,-2.04);
    \draw [thick] (3.7636940901242135,-6.986424552999498)-- (6.24,-3.1);
    \draw [thick] (8.5,-4.54)-- (6.023694090124214,-8.4264245529995);
    \draw [thick] (6.023694090124214,-8.4264245529995)-- (3.7636940901242135,-6.986424552999498);
    \ifthenelse{\boolean{answer}}{
        \node at (3.90, -1.36) {\cir[answer]{3}};
        \node at (6.08, -5.50) {\cir[answer]{2}};
        \node at (1.44, -4.68) {\cir[answer]{1}};
    }{}
}
        }
        \renewcommand{\theenumi}{\cir[gradeColor]{\arabic{enumi}}}
        \multiColEnumerate{3}{
            \item \nswr{losange}
            \item \nswr{rectangle}
            \item \nswr{carré}
        }
    }
}

\scn{Construction de quadrilatères particuliers}
\caSlide{19-20-21}
\slide{exo}{\bshrink
    \act{}{
        \begin{enumerate}
            \item Sur une feuille de papier blanc,
            construisez les quadrilatères suivants à l'aide d'une règle non graduée,
            d'un compas et d'une équerre :
            \bvspace{-0.25cm}\begin{enumerate}
                \item un rectangle (qui n'est pas un carré) ;
                \item un losange (qui n'est pas un carré) ;
                \item un carré.
            \end{enumerate}
            \item Codez ces quadrilatères selon vos observations.
            \item Tracez les diagonales de ces quadrilatères et codez-les selon vos observations.
            \item Comment sont les côtés opposés du losange ?
        \end{enumerate}
    }
}


\scn{Propriétés des quadrilatères particuliers}
\caSlide{23-24-25}

\bsubsec{Propriétés}

\slide{cr}{
    \ssec
    \pr{}{
    \SialorsS{un quadrilatère est un losange}{
        \begin{enumerate}
            \item Ses côtés opposés sont \nswr{parallèles}.
            \item Ses diagonales sont \nswr{perpendiculaires} et \nswr{se coupent en leur milieu}.
        \end{enumerate}
    }
    }[\ym]
}

\slide{cr}{
    \pr{}{
        \SialorsS{un quadrilatère est un rectangle}{
            \begin{enumerate}
                \item Ses côtés opposés sont \nswr{parallèles} et \nswr{de même longueur}.
                \item Ses diagonales ont \nswr{la même longueur} et \nswr{se coupent en leur milieu}.
            \end{enumerate}
        }
    }[\ym]

    \rmk{}{
        \Sialors{un quadrilatère est un carré}{il possède les propriétés du rectangle et du losange}
    }
}
% https://www.problematheque-csen.fr/fiche-probleme/construire-un-carre/

\scn{Rédiger un programme de construction}
\caSlide{28-29-30}

\bookSlide{70p201,75p201}[6cm][1]

\ifthenelse{answer}{
    \exo{75p201}{
    Tracer un pentagone $FHLOC$.
    Placer les points :
    \begin{itemize}
        \item $A$ au milieu du segment $[FH]$
        \item $S$ au milieu du segment $[OL]$
        \item $K$ au milieu du segment $[OC]$
    \end{itemize}
    Tracer le triangle $AKS$.
}[\dim]
}{}

\scn{Raisonner à l'aide des propriétés de polygones particuliers}
\slide{qf}{
    \begin{enumerate}
        \item $\np{1.2} + \np{6.03} - \np{1.1} = $
        \item $\np{6.1} + \np{7,3} \times 2 = $
        \item $\np{0.06398} \times \np{1000} = $
    \end{enumerate}
}

\slide{exo}{
    \exo{}{
        \begin{itemize}
        \item On considère deux triangles équilatéraux $ABC$ et $ABD$, avec $C$ et $D$ deux points distincts.
        \item Déterminer la nature du quadrilatère $ACBD$. Justifier votre réponse à l'aide d'un raisonnement écrit.
        \end{itemize}
    }
}

\nswr[0]{
    \ctikz[0.5]{
    \boundingBox[6][9][0.5pt][0.5][(-1,-1)]
    \drawPoint{A}{0.00}{3.68}
    \drawPoint{C}{2.40}{7.46}
    \drawPoint{B}{4.70}{3.44}
    \drawPoint{D}{2.36}{0.00}
    \draw [penthick] (2.40,7.46) -- (0.00,3.68) -- (4.70,3.44) -- (2.40,7.46);
    \draw [penthick] (0.00,3.68) -- (2.36,0.00) -- (4.70,3.44);
    \draw [thick] (1.37,5.46) -- (1.03,5.68);
    \draw [thick] (2.36,3.76) -- (2.34,3.36);
    \draw [thick] (3.38,5.35) -- (3.72,5.55);
    \draw [thick] (1.35,1.95) -- (1.01,1.73);
    \draw [thick] (3.36,1.83) -- (3.70,1.61);
}
    \begin{itemize}
        \item Les triangles $ABC$ et $ABD$ sont equilatéraux et partagent un côté, ils sont donc égaux.
        \item Alors $ACBD$ possède $4$ côtés égaux est un losange.
    \end{itemize}
}