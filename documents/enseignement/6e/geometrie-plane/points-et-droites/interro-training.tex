% VARIABLES %%%
\setTitle{Interrogation - Entrainement - Séquence 1}
\setGrade{6e}
%%

\exo{}{
    \begin{enumerate}
        \item La droite $(GE)$ est parallèle à la droite $(CD)$
        et la droite $(CD)$ est parallèle à la droite $(TV)$. 
        Quelle est la relation entre les droites $(GE)$ et $(TV)$ ?
        \item La droite $(d)$ est perpendiculaire à la droite $(d')$
        et la droite $(d')$ est parallèle à la droite $(d'')$. 
        Quelle est la relation entre les droites $(d)$ et $(d'')$ ?
        \item La droite $(d_1)$ est perpendiculaire à la droite $(d_2)$
        et la droite $(d_3)$ est parallèle à la droite $(d_2)$. 
        Quelle est la relation entre les droites $(d_3)$ et $(d_1)$ ?
    \end{enumerate}
}

\newpage

\corr{}{
    \begin{enumerate}
        \item \Ona $(TV) \prll (CD)$ et $(GE) \prll (CD)$.\\
        \Or \sialors{deux droites sont parallèles à une même droite}
        {alors elles sont parallèles entre elles.} \\ 
        \Donc $(GE) \prll (TV)$.
        \item \Ona $(d') \prll (d)$ et $(d'') \perp (d')$.\\
        \Or \sialors{deux droites sont parallèles}
        {toute perpendiculaire à l'une est perpendiculaire à l'autre.} \\ 
        \Donc $(d'') \perp (d)$.
        \item \Ona $(d_1) \perp (d_2)$ et $(d_3) \perp (d_1)$.\\
        \Or \sialors{deux droites sont perpendiculaires à une même droite}
        {elles sont parallèles entre elles.} \\ 
        \Donc $(d_1) \perp (d_3)$.
    \end{enumerate}
}