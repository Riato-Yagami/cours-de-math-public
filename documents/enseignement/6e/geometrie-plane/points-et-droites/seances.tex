% VARIABLES %%%
% \date{\today}
\def\longTitle{Géométrie plane - points et droites}
\def\shortTitle{\MakeUppercase{\longTitle}}
% \def\theme{\longTitle}

\setgrade{6e}
\def\imgPath{enseignement/6e/geometrie-plane/points-et-droites/}
%%

\def\ym{\href{https://www.maths-et-tiques.fr/telech/19Para_Perp.pdf}{Yvan Monka}}

\avspace{0.1cm}

\obj{
    \item Connaître la définition de l'alignement de trois points ainsi que de l'appartenance à une droite et reconnaître ces situations.
    \item Savoir tracer avec l'équerre la droite perpendiculaire à une droite donnée passant par un point donné.
    \item Savoir tracer avec la règle et l'équerre la droite parallèle à une droite donnée passant par un point donné.
    \item Connaître les propriétés entre perpendicularité et parallélisme et savoir s'en servir pour raisonner.
}

\scn{Représenter un point et une droite}{}

\bseq{\longTitle}
\bsec{Objet géométrique}
\bsubsec{Le point}

\slide{exo}{
    \act{}{Représenter un point.}
}

\slide{cr}{
    \sseq\ssec\ssubsec
    \vc{}{
        On nomme \key{point} le plus petit élément géométrique.
        Il est infiniment petit,
        tant qu'il n'a pas de dimension.
    }
}

\slide{cr}{
    \rmk{}{
        On représente le point par une croix.
    }

    \df{}{
        Deux points sont dits distincts s'ils ne sont pas confondus.
    }

    \expl{}{
        Les points $A$ et $B$ sont confondus et les points $A$ et $C$ sont distincts.
    }
}

\slide{exo}{
    \act{}{Représenter une droite passant par deux points $A$ et $B$.}
}

\bsubsec{La droite}
\slide{cr}{
    \ssubsec

    \vc{}{On nomme \key{droite} un tracé rectiligne infini.}
}

\slide{cr}{
    \rmk{}{On ne peut pas représenter une droite dans son intégralité.}

    \axio{}{
        Il existe une unique droite passant par deux points distincts.
    }

    \expl{}{
        Pour deux points $D$ et $E$ distincts.
        On peut noter la droite qui passe par $D$ et $E$, $(DE)$ ou $(ED)$.
    }
}

\scn{Représenter un segment et une demi-droite}{}

\slide{exo}{
    \act{}{Représenter un segment}
}


\bsubsec{Le segment et la demi-droite}

\slide{cr}{
    \ssubsec

    \df{}{Un \key{segment} est une portion de droite,
    limitée par deux \key{extrémités}.
    }

    \expl{}{
        Pour deux points $D$ et $E$ distincts.
        Le segment reliant $D$ et $E$ est le chemin le plus court entre ces deux points.
        On le note $[DE]$.
    }
}


\slide{cr}{
    \ssubsec

    \df{}{Une \key{demi-droite} est une portion de droite,
    limitée par une seule extrémité, son \key{origine}.
    }

    \expl{}{
        Pour deux points $D$ et $E$ distincts.
        On peut noter, la demi-droite d'origine $D$ passant par $E$, $[DE)$.
    }
}

\scn{Activité alignement}{}

\slide{qf}{%
    \sqf Comment peut-on nommer la droite ci-dessous?
    \bvspace{-0.5cm}
    \imgp{qf1}[5cm]
    \qfs $(AB)$ \hfill \qfs $[AB]$ \hfill \qfs $(BA)$ \hfill \qfs $(d)$ \hfill \qfs $AB$ \hfill
    \bvspace{0.5cm}
    \\ \hint{plusieurs réponses sont attendues}
}

\slide{qf}{
    \sqf Quels points semblent alignés?
    \bvspace{-0.5cm}
    \imgp{qf2}[5cm]
    \qfs $A$ et $F$ \hfill \qfs $A$,$D$ et $F$ \hfill \qfs $A$,$C$ et $D$ \hfill \qfs $C$,$A$ et $D$ \hfill
    \bvspace{0.5cm}
    \\ \hint{plusieurs réponses sont attendues}
}

\bsec{Alignement}

% \slide{EXERCICES}{
%     \bvspace{-0.5cm}
%     \act{1 p208}{%
%         \bvspace{-1cm}
%         \imgp{dim-6e-act-1-p208}[10cm]
%     }[\dim]
% }
\def\imgPrefix{dim-6e/act-}
\slide{exo}{\bvspace{-0.65cm}\act{}{\bvspace{-0.45cm}\imgp{1p208}[11cm]}
}
\def\imgPrefix{}

% \bsubsec{Alignement}
\slide{cr}{
    \ssec%

    \df{}{
        Des points sont dits \key{alignés} s'il existe une droite passant par tous ces points.
    }
}

\slide{cr}{
    \bvspace{-0.45cm}
    \expl{}{
        \dividePage{
            \imgp{alignement}[4cm]
        }{
            Les points $A$, $C$ et $B$ sont alignés.\\
            Les points $A$, $D$ et $B$ ne sont pas alignés.
        }[0.35]
    }
    \bvspace{-1.2cm}
    \rmk{}{
        \begin{enumerate}
            \item Deux points peuvent toujours être reliés par une droite et sont donc toujours alignés.
            \item Dans l'exemple précédent ; C est placé sur la droite $(AB)$. On dit que $C$ appartient à la droite $(AB)$ et on note : $C\in(AB)$.
        \end{enumerate}
        
    }
}

% \exoList{5 p211,7 p211}[][3]
% \slide{EXERCICES}{
%     \exo{p211}{}
%     \bvspace{-1cm}
%     \imgp{dim-6e-exo-5-p211}[10cm]
%     \imgp{dim-6e-exo-7-p211}[10cm]
% }

\scn{Restauration de figures}{}

\setcounter{qf}{0}
\slide{qf}{
    \sqf Les droites $(f)$ et $(g)$ semblent être:
    \bvspace{-0.5cm}
    \imgp{qf3}[4cm]
    \qfs Sécantes \hfill \qfs Parallèles \hfill \qfs Perpendiculaires \hfill
    \bvspace{0.5cm}
    \\ \hint{plusieurs réponses sont attendues}
}

\slide{qf}{
    \sqf Quelles droites semblent parallèles :
    \bvspace{-0.5cm}
    \imgp{qf4}[5cm]
    \qfs $(AB) \et (CE)$ \hfill \qfs $(AB) \et (DC)$ \hfill \qfs $(CD) \et (EB)$ \hfill
    \bvspace{0.5cm}
    % \\ \hint{plusieurs réponses sont attendus}
}

\slide{exo}{
    \exo{Restaurer à droite la figure de gauche}{
        \imgp{restauration-1}[8cm]
        \imgp{restauration-2}[8cm]
    }[\href{https://pedagogie.ac-orleans-tours.fr/fileadmin/_migrated/content_uploads/Activites_geom_C3__6eme.pdf}
    {Académie d'Orléans}]
}

\def\imgPrefix{dim-6e/exo-}
\exoSlide{5p211,7p211}[8cm][1][\dim]


\scn{Perpendicularité}{}

\slide{qf}{
    \imgp{32p216}[7cm]
}
\def\imgPrefix{}

\bsec{Positions relatives de plusieurs droites}
\bsubsec{Perpendicularité}

\slide{cr}{
    \ssec\ssubsec%
    %
    \df{}{
        Deux droites sont dites \key{sécantes} si elles se coupent en un unique point.
    }
    \bvspace{-1cm}
    % [\myl{https://biblio.manuel-numerique.com?openBook=9782047392935\%3FY29udGV4dGVSZXNvdXJjZT17InR5cGUiOiJhcnRpY2xlIiwiaWRyZWYiOiJpZF9DaGFwdGVyXzAxMl9ab29tX0dyYXBoaWNfNTUxX1NDUl94aHRtbCIsImFydGljbGVUeXBlIjoiem9vbSJ9}]
    % 
}

\slide{cr}{
    \expl{}{
        \dividePage{
            \imgp{secantes}[4cm]
        }{
            Les droites $(AE)$ et$(BD)$ sont sécantes.\\ $F$ est leur point d'intersection.
        }[0.45]
    }[\my]
}

\slide{exo}{
    \act{}{
        \begin{enumerate}
            \item Placer trois points $A,B$ et $C$.
            \item Tracer $(AB)$.
            \item Tracer la droite perpendiculaire à la droite $(AB)$ passant par $C$.
        \end{enumerate}
    }
}

\slide{cr}{
    \df{}{
        Deux droites sont dites \key{perpendiculaires} si elles sont sécantes et leur intersection forme un angle droit.
    }
    \bvspace{-1cm}
    \expl{}{
        \dividePage{
            \imgp{perpendiculaires}[4cm]
        }{
            Les droites $(EF)$ et$(GF)$ sont perpendiculaires.\\ On note $(EF) \perp (GF)$.
        }[0.35]
    }[\my]
}

\slide{cr}{
    \dividePage{
        \mthd{Construction de droites perpendiculaires}{
            \imgp{construction-perpendiculaire}[5cm]
        }
    }{
        \expl{}{Construire la droite perpendiculaire à la droite $(d)$ passant par $A$.
        \imgp{construction}[5cm]}
    }
}

\slide{exo}{
    \bvspace{-0.5cm}
    \exo{Reproduire la figure en bas à droite en respectant sa nouvelle échelle.}{
        \bvspace{-1cm}
        \imgp{restauration-3}[6cm]
    }[\href{https://ienetaples.etab.ac-lille.fr/la-pause-des-pauses/la-pause-procedure/4-la-restauration-de-figure-en-geometrie/}
    {Académie de Lilles}]
}

\def\imgPrefix{dim-6e/exo-}
\exoSlide{14p213,55p218}[7cm][2][\dim]
\def\imgPrefix{}

\scn{Parallélisme}

\qfSlide{
    \dividePage{\imgp{dim-6e/img-12p213}[5cm]}{
        Citer deux couples de droites qui semblent :
        \begin{enumerate}
            \item perpendiculaires
            \item sécantes mais non perpendiculaires
            \item parallèles
        \end{enumerate}
    }
}

\bsubsec{Parallèlisme}

\slide{exo}{
    \act{}{
        \begin{enumerate}
            \item Placer trois points $A,B$ et $C$.
            \item Tracer $(AB)$.
            \item Tracer la droite parallèle à la droite $(AB)$ passant par $C$.
        \end{enumerate}
    }
}

\slide{cr}{
    \ssubsec
    \df{}{
        Deux droites sont dites \key{parallèles} si elles ne sont pas sécantes.
    }
    \bvspace{-1cm}
    \expl{}{
        \dividePage{
            \imgp{paralleles}[3.5cm]
        }{
            Les droites $(d)$ et $(d')$ sont parallèles.\\ On note $(d) \prll (d')$.
            \bvspace{-0.5cm}\rmk{}{Deux droites parallèles conservent le même écartement}
        }[0.3]
    }[\my]
    % [\myl{https://biblio.manuel-numerique.com?openBook=9782047392935\%3FY29udGV4dGVSZXNvdXJjZT17InR5cGUiOiJhcnRpY2xlIiwiaWRyZWYiOiJpZF9DaGFwdGVyXzAxMl9ab29tX0dyYXBoaWNfNTUzX1NDUl94aHRtbCIsImFydGljbGVUeXBlIjoiem9vbSJ9}]
}

\slide{cr}{
    \dividePage{
        \mthd{Construction de droites parallèles}{
            \imgp{construction-parallele}[4.5cm]
        }
    }{
        \expl{}{Construire la droite parallèle à la droite $(d)$ passant par $A$.
        \imgp{construction}[5cm]}
    }
}

% \def\imgPrefix{dim-6e/exo-}
% \exoSlide{11p213}[6.25cm][1][\dim]
% \def\imgPrefix{}
% \slide{EXERCICES}{
%     \bvspace{-0.5cm}
%     \exo{11 p213}{
%         \bvspace{-0.75cm}
%         \imgp{dim-6e-exo-11-p213}[6.25cm]
%     }
% }

% \scn{Construction}{}

% \ifArticle{\newpage}

\def\imgPrefix{dim-6e/exo-}
\exoSlide{68p221,11p213}[4.5cm][2][\dim]
% \exoSlide{}[5.5cm][1][\dim]
\def\imgPrefix{}

\scn{Propriétés 1}{}

% \qf{{$3+10-7$,$=6$},
%     {$A{,}B{,}C$ sont alignés dans cet ordre.
%     Compléter avec $\in$ ou $\notin$.
%     \begin{align*}
%         \qfs \; B \hole [AC]\\
%         \qfs \; A \hole [BC]\\
%         \qfs \; A \hole (BC)\\
%         \qfs \; C \hole [AB)\\
%     \end{align*},}%
%     {}
% }

\qfSlide{
    \begin{enumerate}
        \item Les points $A,B,C$ sont alignés dans cet ordre.
        Compléter avec $\in$ ou $\notin$.
        \begin{align*}
            \qfs \; B \hole [AC]\qquad
            &\qfs \; A \hole [BC]\\
            \qfs \; A \hole (BC)\qquad
            &\qfs \; C \hole [AB)\\
        \end{align*}
        \item Ou peut-on ajouter un point $D$ pour que : $A \notin [CD]$ et $A \in [CD) $ ?
    \end{enumerate}

}

\bsubsec{Propriétés}

\slide{exo}{
    \act{Illustrer la proposition 30 d'Euclide}{
        \imgp{euclide-proposition-30}[9cm]
    }[\href{http://promenadesmaths.free.fr/telecharger/euclide_elements_1804.pdf}{Elements d'Euclide - 1804}]
}

\slide{cr}{
    \ssubsec
    \pr{}{
        \Si deux droites sont parallèles à une même droite,
        \Alors elles sont parallèles entre elles
        \imgp{thm-1}[3.5cm]
    }[\ym]
}

\slide{cr}{
    \rmk{}{Cette propriété sert à montrer que 2 droites sont \palt{2}{parallèles}.}
}

\scn{Propriétés 2}{}

\def\imgPrefix{dim-6e/exo-}
\exoSlide{54p218}[8cm][1][\dim][qf]
\def\imgPrefix{}

\slide{exo}{
    \act{}{%
        \begin{enumerate}
            \item Tracer deux droites $(d_1)$ et $(d_2)$ parallèles.
            \item Tracer une droite $(d_3)$ perpendiculaire à la droite $(d_1)$
            \item Comment semblent être les droites $(d_2)$ et $(d_3)$ ?
        \end{enumerate}
    }
}

\slide{cr}{
    \pr{}{\Si deux droites sont parallèles, \Alors toute perpendiculaire à l'une
    est perpendiculaire à l'autre.
    \imgp{thm-2}[3.5cm]}[\ym]
    \bvspace{-0.75cm}
    \rmk{}{Cette propriété sert à montrer que 2 droites sont \palt{2}{perpendiculaires}.}
}

\slide{exo}{
    \act{}{%
        \begin{enumerate}
            \item Tracer deux droites $(d_1)$ et $(d_2)$ perpendiculaires.
            \item Tracer une droite $(d_3)$ perpendiculaire à la droite $(d_2)$
            \item Comment semblent être les droites $(d_1)$ et $(d_3)$ ?
        \end{enumerate}
    }
}

\slide{cr}{
    \pr{}{\Si deux droites sont perpendiculaires à une même droite,
    \Alors elles sont parallèles entre elles.
    \imgp{thm-3}[3.5cm]}[\ym]
    \bvspace{-0.75cm}
    \rmk{}{Cette propriété sert à montrer que 2 droites sont \palt{2}{parallèles}.}
}

\def\imgPrefix{dim-6e/exo-}

\scn{Démontrations en géométrie}{}

\slide{cr}{
    \expl{}{
        \begin{enumerate}
            \item Tracer un triangle quelconque $ABC$ et placer un point $M$ sur le côté $[BC]$.
            \item Tracer la perpendiculaire à la droite $(AB)$ passant par le point $C$. Elle coupe $(AB)$ en $H$.
            \item Tracer la perpendiculaire à la droite $(CH)$ passant par le point $M$. Elle coupe $(CH)$ en $K$.
            \item  Prouver que les droites $(AB)$ et $(MK)$ sont parallèles.
        \end{enumerate}
    }[\ym]
}

\exoSlide{42p217,72p221}[7cm][2][\dim]

\exoSlide{44p217,71p221}[6cm][2][\dim][dm]