% VARIABLES %%%
\def\authors{\jules}
% \date{\today}
\def\longTitle{Géometrie plane - points et droites}
\def\shortTitle{\MakeUppercase{\longTitle}}
% \bseq{\longTitle}
% \def\theme{\longTitle}

\setboolean{showRef}{false}

\def\dim{Dimension 6e 2016}

\def\imgPath{enseignement/6e/geometrie-plane/points-et-droites/}
\def\imgExtension{.png}

%%

% Yvan Monka : https://www.maths-et-tiques.fr/telech/19Para_Perp.pdf

% \disableAnimation
% \shortAnimation
% \firstSlide

% \qf{
%     {$70\euro$ diminué de $10\%$, $63\euro$},
%     {$50\euro$ augmenté de $20\%$, $\pcalc{50*120/100}\euro$}%
% }[15]



% \qfSUB{}{}

% \vspace*{0.05cm}

\scn{Alignement}{}

\slide{QUESTIONS FLASH}{%
    \sqf Comment peut-on nommer la droite ci-dessous?
    \vspace*{-0.5cm}
    \imgp{qf1}[5cm]
    \qfs $(AB)$ \hfill \qfs $[AB]$ \hfill \qfs $(BA)$ \hfill \qfs $(d)$ \hfill \qfs $AB$ \hfill
    \vspace*{0.5cm}
    \\ \hint{plusieurs réponses sont attendues}
}

\slide{}{
    \sqf Quels points semblent alignés?
    \vspace*{-0.5cm}
    \imgp{qf2}[5cm]
    \qfs $A$ et $F$ \hfill \qfs $A$,$D$ et $F$ \hfill \qfs $A$,$C$ et $D$ \hfill \qfs $C$,$A$ et $D$ \hfill
    \vspace*{0.5cm}
    \\ \hint{plusieurs réponses sont attendues}
}

\bseq{\longTitle}
\bsec{Alignement}

\slide{EXERCICES}{
    \vspace*{-0.5cm}
    \act{1 p208}{%
        \vspace*{-1cm}
        \imgp{dim-6e-act-1-p208}[10cm]
    }[\dim]
}

% \bsubsec{Alignement}
\slide{COURS}{
    \sseq\ssec%

    \df{}{
        Des points sont \key{alignés} s'il existe une droite passant par tous ces points.
    }
}

\slide{}{
    \vspace*{-0.45cm}
    \expl{}{
        \dividePage{
            \imgp{alignement}[4cm]
        }{
            Les points $A$, $C$ et $B$ sont alignés.\\
            Les points $A$, $D$ et $B$ ne sont pas alignés.
        }[0.35]
    }
    \vspace*{-1.2cm}
    \rmk{}{
        \begin{enumerate}
            \item Deux points peuvent toujours être reliés par une droite et sont donc toujours alignés.
            \item Dans l'exemple précédent ; C est placé sur la droite $(AB)$. On dit que $C$ appartient à la droite $(AB)$ et on note : $C\in(AB)$.
        \end{enumerate}
        
    }
}

% \exoList{5 p211,7 p211}[][3]

\slide{EXERCICES}{
    \exo{p211}{}
    \vspace*{-1cm}
    \imgp{dim-6e-exo-5-p211}[10cm]
    \imgp{dim-6e-exo-7-p211}[10cm]
}

\scn{Perpendicularité et parallèlisme}{}

\bsec{Droites}
\bsubsec{Definitions}

\setcounter{qf}{0}
\slide{QUESTIONS FLASH}{
    \sqf Les droites $(f)$ et $(g)$ semblent être:
    \vspace*{-0.5cm}
    \imgp{qf3}[5cm]
    \qfs Sécantes \hfill \qfs Parallèles \hfill \qfs Perpendiculaires \hfill
    \vspace*{0.5cm}
    \\ \hint{plusieurs réponses sont attendues}
}

\slide{}{
    \sqf Quelles droites semblent parallèles :
    \vspace*{-0.5cm}
    \imgp{qf4}[5cm]
    \qfs $(AB) \et (CE)$ \hfill \qfs $(AB) \et (DC)$ \hfill \qfs $(CD) \et (EB)$ \hfill
    \vspace*{0.5cm}
    % \\ \hint{plusieurs réponses sont attendus}
}

\slide{COURS}{
    \ssec\ssubsec%
    %
    \df{}{
        Deux droites sont \key{sécantes} si elles se coupent en un unique point.
    }
    \vspace*{-1cm}
    \expl{}{
        \dividePage{
            \imgp{secantes}[4cm]
        }{
            Les droites $(AE)$ et$(BD)$ sont sécantes.\\ $C$ est leur point d'intersection.
        }[0.45]
    }[\myl{https://biblio.manuel-numerique.com?openBook=9782047392935\%3FY29udGV4dGVSZXNvdXJjZT17InR5cGUiOiJhcnRpY2xlIiwiaWRyZWYiOiJpZF9DaGFwdGVyXzAxMl9ab29tX0dyYXBoaWNfNTUxX1NDUl94aHRtbCIsImFydGljbGVUeXBlIjoiem9vbSJ9}]
}

\slide{}{
    \df{}{
        Deux droites sont \key{perpendiculaires} si elles sont sécantes et leur intersection forme un angle droit.
    }
    \vspace*{-1cm}
    \expl{}{
        \dividePage{
            \imgp{perpendiculaires}[4cm]
        }{
            Les droites $(EF)$ et$(GF)$ sont perpendiculaires.\\ On note $(EF) \perp (GF)$.
        }[0.35]
    }[\myl{https://biblio.manuel-numerique.com?openBook=9782047392935\%3FY29udGV4dGVSZXNvdXJjZT17InR5cGUiOiJhcnRpY2xlIiwiaWRyZWYiOiJpZF9DaGFwdGVyXzAxMl9ab29tX0dyYXBoaWNfNTUyX1NDUl94aHRtbCIsImFydGljbGVUeXBlIjoiem9vbSJ9}]
}

\slide{}{
    \df{}{
        Deux droites sont \key{parallèles} si elles ne sont pas sécantes.
    }
    \vspace*{-1cm}
    \expl{}{
        \dividePage{
            \imgp{paralleles}[4cm]
        }{
            Les droites $(d)$ et$(d')$ sont parallèles.\\ On note $(d) \parallel (d')$.
            \rmk{}{Deux droites parallèles conservent le même écartement}
        }[0.3]
    }[\myl{https://biblio.manuel-numerique.com?openBook=9782047392935\%3FY29udGV4dGVSZXNvdXJjZT17InR5cGUiOiJhcnRpY2xlIiwiaWRyZWYiOiJpZF9DaGFwdGVyXzAxMl9ab29tX0dyYXBoaWNfNTUzX1NDUl94aHRtbCIsImFydGljbGVUeXBlIjoiem9vbSJ9}]
}

\slide{EXERCICES}{
    \vspace*{-0.5cm}
    \exo{11 p213}{
        \vspace*{-0.75cm}
        \imgp{dim-6e-exo-11-p213}[6.25cm]
    }
}

\scn{Construction}{}

\bsubsec{Construction}

\slide{}{
    \act{}{
        \begin{enumerate}
            \item Placer trois points $A,B$ et $C$.
            \item Tracer $(AB)$.
            \item Tracer une droite perpendiculaire à $(AB)$ passant par $C$.
        \end{enumerate}
    }
}

\slide{COURS}{
    \dividePage{
        \mthd{}{
            \imgp{construction-perpendiculaire}[5cm]
        }
    }{
        \expl{}{Construire la droite perpendiculaires à $(d)$ passant par $A$.
        \imgp{construction}[5cm]}
    }
}

\slide{EXERCICES}{
    \act{}{
        \begin{enumerate}
            \item Placer trois points $A,B$ et $C$.
            \item Tracer $(AB)$.
            \item Tracer une droite parallèle à $(AB)$ passant par $C$.
        \end{enumerate}
    }
}

\slide{COURS}{
    \dividePage{
        \mthd{}{
            \imgp{construction-parallele}[5cm]
        }
    }{
        \expl{}{Construire la droite parallèles à $(d)$ passant par $A$.
        \imgp{construction}[5cm]}
    }
}

\slide{EXERCICES}{
    \vspace*{-0.5cm}
    \exo{14 p213}{
        \vspace*{-0.75cm}
        \imgp{dim-6e-exo-14-p213}[10cm]
    }
}

\scn{Propriétés}{}

\bsec{Propriétés}
\slide{}{
    \ssec
    \pr{}{}
}

\slide{}{
    \pr{}{}
}

\slide{}{
    \pr{}{}
}