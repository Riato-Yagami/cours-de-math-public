\setGrade{6e}
\setTitle{Devoir de vacances - Cercle chromatique}
% \emptyBackground

\def\imgPath{enseignement/6e/geometrie-plane/distances/ddv/}

\def\Co{\ensuremath{\mathscr{C}_1} }
\def\Cd{\ensuremath{\mathscr{C}_2} }

\def\ankinluu{\href{https://x.com/Ankinluu\_}{@Ankinluu\_}}
\def\authors{\jules{} et \ankinluu}

\begin{itemize}
    \item \hint{
    \begin{itemize}
        \item La construction doit être faite sur une feuille blanche de format A4.
        \item Tracez des traits et placez des points de façon légère, afin de pouvoir les effacer facilement si besoin.
        \item Certains choix de longueurs vous seront laissés au cours de la construction. Faites comme bon vous semble, il n'y a pas de mauvais choix.
        \item Vous pouvez ajouter des tracés supplémentaires pour vous aider à la construction de la figure. C'est même encouragé.
    \end{itemize}
    }
\end{itemize}

\section{Construction du cercle chromatique}

\dividePage{
    \begin{itemize}
        \item Nous allons construire une figure ressemblant au schéma de droite.
        \item Pour cela, suivez les étapes suivantes :
    \end{itemize}
}{\vspace{-2.75cm}
\imgp{cercle-chromatique-schema}[5cm]
}[0.5]


\begin{enumerate}
    \item \textbf{Construction du cercle extérieur :} 
    \begin{enumerate}
        \item Placez un point $O$ au centre de votre feuille.
        \item Tracez un cercle \Co de centre $O$ avec un rayon compris entre $6$ et $10\;\centi\meter$.
    \end{enumerate}

    \item \textbf{Construction dans le cercle :}
    \begin{enumerate}    
        \item Placez un point $A$ sur le cercle \Co.
        \item Tracez une perpendiculaire à la droite $(OA)$ passant par le point $O$.
        \item Placez un point $G$ à l'une des intersections entre la perpendiculaire et le cercle \Co.
        \item Placez le point $M$ au milieu du segment $[AG]$.
        \item Tracez la demi-droite $[OM)$.
        \item Placez le point $H$ à l'intersection de la demi-droite $[OM)$ et du cercle \Co.
    \end{enumerate}

    \item \textbf{Construction des points du cercle extérieur :}
    \begin{enumerate}
        \item Tracez le segment $[AH]$, qui est une corde du cercle \Co.
        \item Placez un point $B$, distinct du point $H$, sur le cercle \Co de sorte que le segment $[AB]$ ait la même longueur que le segment $[AH]$.
        \item Placez les points $C$, $D$, $E$ et $F$ sur le cercle \Co de façon à ce que les cordes $[AB], [BC], [CD], [DE], [EF]$ et $[FG]$ aient toutes la même longueur.
    \end{enumerate}

    \item \textbf{Construction du cercle intérieur :}
    \begin{enumerate}
        \item Tracez un cercle \Cd de centre $O$ avec un diamètre compris entre $2$ et $6\;\centi\meter$.
        \item Placez le point $A'$ à l'intersection du segment $[OA]$ et du cercle \Cd.
        \item Placez les points $B', C', D', E', F'$ et $G'$ à l'intersection des segments :
        $[OB]$, $[OC]$, $[OD]$, $[OE]$, $[OF]$, et $[OG]$ respectivement et du cercle \Cd.
        \item Tracez les segments $[AA'], [BB'], [CC'], [DD'], [EE'], [FF']$ et $[GG']$.
    \end{enumerate}
    \item Effacer les traits de construction et les points pour garder que ce qui constitue le cercle chromatique comme sur le schéma.
\end{enumerate}

\section{Compléter le cercle chromatique}

\dividePage{
    \begin{itemize}
        \item Pour cette partie,
        vous allez personnaliser votre cercle chromatique en vous inspirant du
        \href{https://knowyourmeme.com/memes/color-wheel-character-challenge}{Color Wheel Character Challenge}.
        Défi artistique lancé sur Twitter en 2023 avec le cercle chromatique du dessinateur \ankinluu ci-dessous.
        \begin{center}
            \imgp{cercle-chromatique-Ankinluu}[7cm]
        \end{center}
    \end{itemize}
}{\vspace{-2.5cm}
    \begin{center}
        \imgp{cercle-chromatique-couleurs}[6.1cm]
    \end{center}
    \vspace{-0.75cm}
    \begin{itemize}
        \item Ce challenge invite les artistes à remplir chaque section d'un cercle chromatique
        en dessinant des personnages qui respectent les couleurs associées à chaque portion du cercle :
        % \begin{tcolorbox}[
        %     enhanced,
        %     colback=Black!45,                % Background color
        %     borderline={2.5mm}{-0.25mm}{black}, % Outer black border with thickness 0.5mm
        %     borderline={1mm}{0.5mm}{white},  % Inner white border with thickness 1mm
        %     arc=5mm,                      % Corner roundness
        %     width=0.95\linewidth,          % Adjust box width
        %     % sharp corners=all,            % Apply rounded corners to all sides
        % ]
        \newcommand{\col}[2]{\contour{black}{\textbf{\textcolor{#1}{#2}}}}
        \contourlength{0.4mm}
        \multiColItemize{2}{
            \item \col{orange}{Orange}
            \item \col{yellow}{Jaune}
            \item \col{green}{Vert}
            \item \col{cyan}{Cyan}
            \item \col{blue}{Bleu}
            \item \col{violet}{Violet}
            \item \col{magenta}{Rose}
            \item \col{red}{Rouge}
        }
        % \end{tcolorbox}
    \end{itemize}
}[0.5]

\begin{itemize}
    \item \hint{
        \begin{itemize}
            \item Utilisez des crayons de couleur pour compléter votre travail.
            \item Laissez libre cours à votre imagination tout en respectant les teintes définies.
            Amusez-vous et exprimez votre créativité !
            \item Tout comme dans l'exemple de \ankinluu n'hésitez pas si vous le souhaitez à dépacer des cadres pour donner du relief à vos dessins. 
            \item Ne vous inquiétez pas pour la qualité du dessin ou du coloriage :
            vous ne serez évidemment pas notés sur cet aspect.
        \end{itemize}    
    }
    \item Voici vos options :
    \begin{itemize}
        \item \textbf{Option 1 : Dessin} : Vous pouvez dessiner des personnages ou animaux ou fleurs ou tout autre sujet de votre choix, en suivant les couleurs du cercle chromatique.
        \item \textbf{Option 2 : Coloriage} : Si vous ne souhaitez pas dessiner, vous pouvez simplement colorier les différentes arches du disque en respectant les couleurs.
    \end{itemize}
\end{itemize}



