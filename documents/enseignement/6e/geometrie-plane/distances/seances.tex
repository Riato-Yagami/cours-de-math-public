% VARIABLES %%%
% \setTitle{Géométrie plane - distances}
\setSeq{3}{Géométrie plane - distances}
\setGrade{6e}

\def\imgPath{enseignement/6e/geometrie-plane/distances/}

\def\ym{\href{https://www.maths-et-tiques.fr/telech/19Long.pdf}{Yvan Monka}}

% \firstSlide
% Cercle https://www.maths-et-tiques.fr/telech/19Cercle.pdf
%%

\obj{
    \item Savoir tracer un segment de longueur donnée.
    \item Savoir placer le milieu d'un segment de longueur donnée.
    \item Savoir déterminer le plus court chemin entre un point et une droite.
    \item Connaître et savoir estimer la distance entre un point et une droite.
    \item Reconnaître, nommer, décrire, reproduire un cercle
    \item Utiliser le vocabulaire associé : rayon, diamètre, centre, arc de cercle et corde.
    \item Différentier le cercle et le disque.
}

\scn{Chemin le plus court}

\bsec{Distance entre deux points}

\slide{qf}{\nullsubsec{}{\imgp{qf/ca-cm2-mars-2023-1-2-3}[10cm]}[\ca]}

\bsubsec{Segment}

\def\actDistance{
    \begin{tikzpicture}[scale = 0.7]
        % Grand rectangle extérieur
        \draw[gradeColor, thick] (0,0) rectangle (10,8);
        % Obstacles
        \draw[gradeColor, fill=gradeColor!20] (3,5) rectangle (4,7.5);
        \draw[gradeColor, fill=gradeColor!20] (5,1) rectangle (7,6); 
        \draw[gradeColor, fill=gradeColor!20] (5,1) rectangle (7,6); 
        % Points A et B
        \drawPoint{E}{2}{7}
        \drawPoint{S}{8}{2}
    \end{tikzpicture}
    \hspace{0.5cm}
    \begin{tikzpicture}[scale = 0.75]
        % Grand rectangle extérieur
        \draw[gradeColor, thick] (0,0) rectangle (10,8);
        
        % Obstacles
        \draw[gradeColor, fill=gradeColor!20] (1,3) rectangle (4,4);
        \draw[gradeColor, fill=gradeColor!20] (4,2) rectangle (4.5,7.5);
        \draw[gradeColor, fill=gradeColor!20] (5,1.5) rectangle (6,7);
        \draw[gradeColor, fill=gradeColor!20] (6,3.5) rectangle (8.8,4.8);
    
        % Points S et E
        \drawPoint{S}{3}{5}
        \drawPoint{E}{7.5}{3}
    \end{tikzpicture}
}

\slide{exo}{
    \act{}{
        L'agent Solid Snake se trouve au point $S$ et doit rejoindre les escaliers de sortie situés au point $E$ le plus rapidement possible.
        Aide-le en traçant sur les deux plans de la base militaire le chemin le plus court entre les points $S$ et $E$,
        tout en évitant les obstacles.
        
        % \vspace{0.25cm}
        \begin{center}
            \ifArticle{\actDistance}
        \end{center}
    }
}

\ifBeamer{
    \slide{exo}{
        \wideFrame[7em]{\actDistance}
    }
}

\slide{cr}{
    \sseq\ssec

    \pr{}{
        Le \key{chemin le plus court} entre deux points est la \key{ligne droite}.
    }

    \bvspace{-0.5cm}

    \nt{}{
        On note $AB$ la \key{distance} entre les points $A$ et $B$.
    }

    \bvspace{-0.5cm}

    \pr{}{
        Un segment $[AB]$ à pour longueur $AB$.
    }
}

\slide{cr}{
    \nt{}{%
        On code les segments de même longueur par un même symbole dessiné sur chacun de ces segments.
    }
    \expl{}{%
        \imgp{expl-codage}[8cm]
    }
}

\bookSlide{6p191,7p191}[9cm]

\scn{Trouver le milieu entre deux points}

\slide{qf}{\nullsubsec{}{\imgp{qf/ca-cm2-mars-2023-4-5-6}[10cm]}[\ca]}

\bsubsec{Milieu d'un segment}

\slide{exo}{
    \bvspace{-0.5cm}
    \ifBeamer{\small}
    \act{}{
        Gandalf le Gris doit retrouver trois anneaux de pouvoir avant que Sauron ne mette la main dessus.
        Les anneaux sont cachés entre trois lieux importants :
        La Comté au point $C$,
        Le Gondor au point $G$,
        et Le Mordor au point $M$.\\
        Marquer sur la carte, en suivant les indications ci-dessous,
        les positions des anneaux Les points $A_1$, $A_2$ et $A_3$.
        \begin{enumerate}
            \item Le premier anneau est caché exactement à mi-chemin entre La Comté et Le Gondor.
            \item Le deuxième anneau se trouve à égale distance entre La Comté et le Mordor.
            \item Le troisième anneau est caché au milieu du segment qui relie les positions des deux premiers anneaux.
        \end{enumerate}
        \imgp{activite-middle-earth}[14cm]
    }[\href{https://paulinebaynes.com/?what=artifacts&cat=79}{Pauline Baynes (illustratrice)}]
}

\ifBeamer{
    \slide{exo}{
        \imgp{activite-middle-earth}[11cm]
    }
}

\slide{cr}{
    \df{}{
        Un point est dit \key{milieu d'un segment} si il est situé à égale distance de ses extrémités.
    }[\wiki{Milieu_d'un_segment}]
}

\bookSlide{8p191,9p191}

\scn{Utiliser le milieu d'un segment}

\slide{qf}{\nullsubsec{}{\imgp{qf/ca-cm2-mars-2023-7-8-9}[10cm]}[\ca]}

\def\exoGrid{
    \def\crossWidth{0.04cm}
    \def\gridColor{gradeColor!50}
    % \begin{center}
        \begin{tikzpicture}[scale = 0.5]
            \tkzInit[xmax=15, ymax=11]
            \tkzGrid[color=\gridColor,subxstep=0.5,subystep=0.5]
            \drawPoint{A}{6}{7}
            \drawPoint{B}{10}{5}
        \end{tikzpicture}
        \ifArticle{\hspace{0.5cm}}
        \begin{tikzpicture}[scale = 0.5]
            \tkzInit[xmax=15, ymax=11]
            \tkzGrid[color=\gridColor,subxstep=0.5,subystep=0.5]
            \drawPoint{A}{5}{4}
            \drawPoint{B}{11}{8}
        \end{tikzpicture}
        % \vspace{0.5cm}
        % \begin{tikzpicture}[scale = 0.5]
        %     \tkzInit[xmax=15, ymax=11]
        %     \draw [draw=\gridColor] (0,0) rectangle (15,11);
        %     % \tkzGrid[color=gradeColor,subxstep=0.5,subystep=0.5]
        %     \drawPoint{A}{3}{9}
        %     \drawPoint{B}{13}{3}
        % \end{tikzpicture}
    % \end{center}
}

\slide{exo}{
    \exo{}{
        Chacune des figures suivantes est constituée de deux points $A$ et $B$.
        Chacune d'elles doit être complétée par un point $J$ qui respecte les conditions suivantes :
        \begin{enumerate}
            \item $I$, $C$ et $D$ sont trois points tels que $I$ est le milieu des segments $[AC]$ et $[BD]$;
            \item $E$ est le point tel que $A$ est le milieu du segment $[DE]$;
            \item $J$ est le milieu du segment $[CE]$.
        \end{enumerate}
        \ifArticle{\begin{center}\exoGrid\end{center}}
        \begin{itemize}
            \item Placer deux points $A$ et $B$ sur une feuille blanche et de même trouver l'emplacement de $J$.
        \end{itemize}
    }[\rpmc[149]]
}

\ifBeamer{
    \slide{exo}{
        \wideFrame[7em]{
            \exoGrid
        }
    }
}

\scn{Distance d'un point à une droite}

\slide{qf}{\nullsubsec{}{\imgp{qf/ca-cm2-mars-2023-10-11-12}[10cm]}[\ca]}

\bsec{Distance d'un point à une droite}

\slide{exo}{
    \act{}{Achille (représenté par le point A sur la carte)
    est fatigué et s'est trop éloigné de la côte de la plage du Peulx.
    \begin{enumerate}
        \item Tracez le plus court chemin qu'il doit suivre pour rejoindre la côte.
        \item Quelle est la distance qu'il devra nager ?
    \end{enumerate}
        \ctikz[0.75]{
            \draw[very thick, YellowOrange] (3,0) -- (6,9);
            \filldraw[YellowOrange, opacity=0.1] (0,0) -- (3,0) -- (6,9) -- (0,9) -- cycle;
            \draw[gray!40] (0,0) rectangle (17,9);
            \node[YellowOrange] at (2.5,4.5) {Plage};
            \drawPoint{$A$}{13}{4}
            \node[above left] at (17,0) {Echelle : $1\centi\meter$ = $10\meter$};
        }   
    }[\href{https://www.maths-et-tiques.fr/telech/DISTANCES_GR.pdf}{Yvan Monka}]
}

\slide{cr}{
    \ssec
    \df{}{
        La \key{distance d'un point à une droite} est la plus courte distance séparant ce point et un point de la droite.
    }[\wiki{Distance_d'un_point_à_une_droite}]

    \pr{}{
        Le plus petit segment ayant comme extrémité un point $A$ et un point d'une droite $(d)$
        est celui qui est perpendiculaire à la droite $(d)$.
        \imgp{expl-distance-droite}[8cm]
    }
}

\scn{Le cercle}

\slide{qf}{\nullsubsec{}{\imgp{qf/ca-cm2-mars-2023-13-14-15}[10cm]}[\ca]}

\bsec{Le cercle}

\bsubsec{Definitions}

\slide{exo}{
    \act{}{\def\dist{$6 \centi\meter$ }
        \begin{enumerate}
            \item Placer un point $O$ à plus de \dist des bords de votre page.
            \item Placer un point $C_1$ à \dist du point $O$.
            \item Placer un autre point $C_2$ à \dist du point $O$.
            \item Répétez ce processus jusqu'à placer le point $C_{10}$
            \item Combien de points distincts peut-on placer à \dist du point $O$?
            \item Sur quelle figure seront-ils tous situés?
        \end{enumerate}
    }
}

\slide{cr}{
    \df{}{
        On appelle \key{cercle}, l'ensemble des points situé à \key{égale distance} d'un point nommé \key{centre}.
    }
}

\slide{cr}{
    \ifBeamer{}
    \df{}{
        Dans un cercle, on appelle:
        \begin{itemize}
            \item \palt{2}{\key{rayon}}, un segment joignant le centre à un point du cercle.
            \item \palt{2}{\key{corde}}, un segment dont les extrémités se trouvent sur le cercle.
            \item \palt{2}{\key{diamètre}}, une corde passant par le centre.
            \item \palt{2}{\key{arc}}, une portion de cercle délimitée par deux points.
        \end{itemize}
        \ifArticle{\imgp{cercle-definitions}[6cm]}
    }[\wiki{cercle}]
}

\slide{cr}{
    \ifBeamer{\imgp{cercle-definitions-vide}[6cm]}
}

\slide{cr}{
    \df{}{
        On appelle \key{disque}, l'ensemble des points situé à une \key{distance inférieure ou égale} d'un point nommé \key{centre}.
    }
}