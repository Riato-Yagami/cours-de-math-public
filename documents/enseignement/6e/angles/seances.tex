% VARIABLES %%%
\setSeq{8}{Angles}
\setGrade{6e}

\def\imgPath{enseignement/6e/angles/}

\dym{https://www.maths-et-tiques.fr/telech/19Angles.pdf}
\dcwr{https://college-willy-ronis.fr/maths/chapitre-9-les-angles/}

% \forPrint
% \forStudents

\obj{
    \item Identifier des angles dans une figure géométrique.
    \item Utiliser la notation « angle ABC » pour les angles.
    \item Comparer des angles, avec recours ou non à leur mesure (gabarit, superposition, calque).
    \item Reproduire un angle donné en utilisant un gabarit.
    \item Estimer qu'un angle est droit, aigu ou obtus.
    \item Utiliser un rapporteur pour mesurer un angle en degrés.
    \item Construire un angle de mesure donnée en degrés, à l'aide du rapporteur.
    \item Construire un diagramme circulaire ou semi-circulaire (proportionnalité).
}

\scn{Utiliser un gabarit pour mesurer un angle}

\slide{qf}{
    \exo{Etape de calcul}{
    Calcul en faisant apparaitre les toutes les étapes de calculs :
    \multiColEnumerate{1}{
        % \item $\np{1.5} + 8 \times 9
        % = \nswr[1]{\np{1.5} + 72
        % = \np{73.5}}$
        \item $2 \times (\np{10} - \np{6.9}) - 5
        = \nswr[2]{2 \times \np{3.1} - \np{5}
        = \np{6.2} - \np{5}
        = \np{1.2}}$
        \item $3 \times \np{0.5} + 12 \times \np{0.25}
        = \nswr[2]{\np{1.5} + \np{3} = \np{4.5}}$
    }
}
}

\slide{exo}{
    \newcommand{\gabarit}[1]{\item[]\input{resources/enseignement/6e/angles/tikz/gabarits/#1.tex}}
\def\figScale{0.6}

\act{Gabarits cassés}{
    \begin{enumerate}
        \item Découpe les 4 gabarits dont la pointe est brisée.
        \multiColItemize{4}{
            \gabarit{90} \gabarit{60} \gabarit{30} \gabarit{15}
        }
        \item Utilise ces gabarits pour mesurer les angles des figures suivantes :
        \def\nswrSkip{1cm}
\ctikz{
    \boundingBox[19.97][18.3][0.5pt][1][(-5.18,-12.86)]
    \draw [shift={(-1.48,0.92)},thick,color=gradeColor,fill=gradeColor,fill opacity=0.10] (0,0) -- (38.07:0.88) arc (38.07:68.07:0.88) -- cycle;
    \draw [shift={(-1.48,0.92)},thick,color=gradeColor,fill=gradeColor,fill opacity=0.10] (0,0) -- (-111.93:0.88) arc (-111.93:38.07:0.88) -- cycle;
    \draw [shift={(-1.48,0.92)},thick,color=gradeColor,fill=gradeColor,fill opacity=0.10] (0,0) -- (-141.93:0.88) arc (-141.93:-111.93:0.88) -- cycle;
    \draw [shift={(-1.48,0.92)},thick,color=gradeColor,fill=gradeColor,fill opacity=0.10] (0,0) -- (68.07:0.88) arc (68.07:218.07:0.88) -- cycle;
    \draw [shift={(10.89,-4.26)},thick,color=gradeColor,fill=gradeColor,fill opacity=0.10] (0,0) -- (66.80:0.88) arc (66.80:111.80:0.88) -- cycle;
    \draw [shift={(14.58,4.35)},thick,color=gradeColor,fill=gradeColor,fill opacity=0.10] (0,0) -- (171.80:0.88) arc (171.80:246.80:0.88) -- cycle;
    \draw [shift={(7.01,5.44)},thick,color=gradeColor,fill=gradeColor,fill opacity=0.10] (0,0) -- (-68.20:0.88) arc (-68.20:-8.20:0.88) -- cycle;
    \draw [shift={(6.63,-0.32)},thick,color=gradeColor,fill=gradeColor,fill opacity=0.10] (0,0) -- (-152.86:0.88) arc (-152.86:-107.86:0.88) -- cycle;
    \draw [shift={(-3.31,-12.01)},thick,color=gradeColor,fill=gradeColor,fill opacity=0.10] (0,0) -- (27.14:0.88) arc (27.14:72.14:0.88) -- cycle;
    \draw [shift={(-0.77,-4.11)},thick,color=gradeColor,fill=gradeColor,fill opacity=0.10] (0,0) -- (-107.86:0.88) arc (-107.86:27.14:0.88) -- cycle;
    \draw [shift={(4.08,-8.23)},thick,color=gradeColor,fill=gradeColor,fill opacity=0.10] (0,0) -- (72.14:0.88) arc (72.14:207.14:0.88) -- cycle;
    \draw [shift={(11.77,-8.27)},thick,color=gradeColor,fill=gradeColor,fill opacity=0.10] (0,0) -- (-133.50:0.88) arc (-133.50:61.50:0.88) -- cycle;
    \draw [shift={(11.77,-8.27)},thick,color=gradeColor,fill=gradeColor,fill opacity=0.10] (0,0) -- (136.50:0.88) arc (136.50:226.50:0.88) -- cycle;
    \draw [shift={(11.77,-8.27)},thick,color=gradeColor,fill=gradeColor,fill opacity=0.10] (0,0) -- (61.50:0.88) arc (61.50:136.50:0.88) -- cycle;
    \node at (-4.13, 1.31) {\cir[gradeColor]{1}};
    \node at (10.86, 2.33) {\cir[gradeColor]{2}};
    \node at (1.40, -5.37) {\cir[gradeColor]{3}};
    \node at (8.40, -7.56) {\cir[gradeColor]{4}};
    \draw[color=gradeColor] (-0.33,2.49) node {\nswr[0]{$30\textrm{\degre}$}};
    \draw[color=gradeColor] (-0.21,0.15) node {\nswr[0]{$150\textrm{\degre}$}};
    \draw[color=gradeColor] (-2.46,-0.32) node {\nswr[0]{$30\textrm{\degre}$}};
    \draw[color=gradeColor] (-2.38,2.20) node {\nswr[0]{$150\textrm{\degre}$}};
    \draw[color=gradeColor] (11.06,-2.57) node {\nswr[0]{$45\textrm{\degre}$}};
    \draw[color=gradeColor] (13.05,3.78) node {\nswr[0]{$75\textrm{\degre}$}};
    \draw[color=gradeColor] (8.22,4.63) node {\nswr[0]{$60\textrm{\degre}$}};
    \draw[color=gradeColor] (5.65,-1.37) node {\nswr[0]{$45\textrm{\degre}$}};
    \draw[color=gradeColor] (-2.26,-10.62) node {\nswr[0]{$45\textrm{\degre}$}};
    \draw[color=gradeColor] (0.11,-5.21) node {\nswr[0]{$135\textrm{\degre}$}};
    \draw[color=gradeColor] (3.13,-7.11) node {\nswr[0]{$135\textrm{\degre}$}};
    \draw[color=gradeColor] (11.80,-9.51) node {\nswr[0]{$195\textrm{\degre}$}};
    \draw[color=gradeColor] (10.51,-7.87) node {\nswr[0]{$90\textrm{\degre}$}};
    \draw[color=gradeColor] (11.53,-6.76) node {\nswr[0]{$75\textrm{\degre}$}};
    \draw [thick] (-5.18,-1.97) -- (3.01,4.44);
    \draw [thick] (0.17,5.03) -- (-3.04,-2.93);
    \draw [thick] (10.89,-4.26) -- (14.58,4.35) -- (7.01,5.44) -- (10.89,-4.26);
    \draw [thick] (-3.31,-12.01) -- (-0.77,-4.11) -- (6.63,-0.32) -- (4.08,-8.23) -- (-3.31,-12.01);
    \draw [thick] (7.40,-12.86) -- (11.77,-8.27) -- (14.79,-2.70);
    \draw [thick] (7.17,-3.90) -- (11.77,-8.27);
}
    \end{enumerate}
}[\href{http://clg-blois-begon-blois.tice.ac-orleans-tours.fr/eva/sites/clg-blois-begon-blois/IMG/pdf/6g-e_maths_miteul_012_activite-2.pdf}{Collège Michel Bégon}]
}
% \slide{exo}{
%     \exo{Billard}{
    \begin{enumerate}
        \item La balles rebondis avec un angle de \ang{90}
        \begin{center}
            \Billard[Longueur=6cm,Largeur=4cm]{"NEIGE"}
        \end{center}

        \item La balles rebondis avec le même angle qu'elle est arrivé.
        \begin{center}
            \Billard[Vrai,Longueur=7cm,Largeur=4cm]{"MONTAGNE"}
        \end{center}
    \end{enumerate}
}
% }

\scn{Nommer un angle}

\def\caPrefix{6e-juin-2023-}
\caSlide{10-11-12}

\slide{cr}{
    \sseq
    \section{Definition}
    \df{}{
    On appelle \key{angle} l'ouverture limitée par deux demi-droites de même origine.
}[\ym]
    \expl{}{
    \ctikz[0.75]{
    % \boundingBox[8.38][5.62][0.5pt][1][(-8.38,-4.9)]
    \draw [shift={(-7.98,-4.90)},thick,color=gradeColor,fill=gradeColor,fill opacity=0.10] (0,0) -- (2.47:1) arc (2.47:50.21:1) -- cycle;
    \drawPoint{A}{-7.98}{-4.90}
    \drawPoint{C}{-4.15}{-0.30}
    \drawPoint{B}{-3.66}{-4.71}
    \draw [thick] (-7.98,-4.90) -- (-1.96,-4.64);
    \draw [thick] (-7.98,-4.90) -- (-3.30,0.72);
}
    L'angle formé par les demi-droite $[AB)$ et $[AC)$ est noté $\widehat{BAC}$
}
}

\bookSlide{2p145,25p148}[7cm][2]

\scn{Mesurer un angle à l'aide d'un rapporteur}
\caSlide{13-14-15}

\slide{cr}{
    \section{Mesure}
    \mthd{Mesurer un angle}{
    Pour mesurer un angle, on le compare au \key{degré},
    défini comme $\frac{1}{360}$ d'un tour complet.
    Pour cela, on utilise un rapporteur.
    \ctikz[0.6]{
    % \boundingBox[5.96][6.02][0.5pt][1][(0,0)]
    \drawPoint{A}{0}{0}
    \drawPoint{B}{5.80}{1.55}
    \drawPoint{C}{4.53}{5.60}
    \draw [thick] (0,0) -- (5.80,1.55);
    \draw [thick] (0,0) -- (4.53,5.60);
    \tkzRapporteur[Origine={(0,0)},Rotation=15,Couleur=gradeColor,Echelle=1.25]
}
    \begin{enumerate}
        \item Placez le centre du rapporteur sur le sommet de l'angle.
        \item Alignez un côté de l'angle avec le zéro du rapporteur.
        \item Lisez la mesure de l'angle (inférieure ou supérieure à $\ang{90}$) sur le rapporteur.
    \end{enumerate}
}


}

\bookSlide{27p148}[10cm][1]

\slide{exo}{
    \exo{Mesurer des angles}{
    Mesure les angles ci-dessous :
    \ctikz[1.25]{
    \boundingBox[9.93][9.29][0.5pt][1][(-7.35,-2.84)]
    \draw [shift={(0.64,0.73)},thick,color=gradeColor,fill=gradeColor,fill opacity=0.10] (0,0) -- (179.53:0.60) arc (179.53:227.53:0.60) -- cycle;
    \draw [shift={(-3.96,2.07)},thick,color=gradeColor,fill=gradeColor,fill opacity=0.10] (0,0) -- (9.38:0.60) arc (9.38:105.38:0.60) -- cycle;
    \draw [shift={(-6.18,0.99)},thick,color=gradeColor,fill=gradeColor,fill opacity=0.10] (0,0) -- (100.37:0.60) arc (100.37:232.37:0.60) -- cycle;
    \draw [shift={(2.58,3.15)},thick,color=gradeColor,fill=gradeColor,fill opacity=0.10] (0,0) -- (129.38:0.60) arc (129.38:189.38:0.60) -- cycle;
    \draw[color=gradeColor] (-0.24,0.26) node {\nswr[0]{$48\textrm{\degre}$}};
    \draw[color=gradeColor] (-3.60,2.50) node {\nswr[0]{$96\textrm{\degre}$}};
    \draw[color=gradeColor] (-7.20,1.66) node {\nswr[0]{$132\textrm{\degre}$}};
    \draw[color=gradeColor] (1.56,3.58) node {\nswr[0]{$60\textrm{\degre}$}};
    \draw [thick] (-4.20,0.77) -- (0.64,0.73) -- (-2.63,-2.84);
    \draw [thick] (-3.96,2.07) -- (2.58,3.15) -- (-0.13,6.45);
    \draw [thick] (-7.04,5.69) -- (-6.18,0.99) -- (-7.35,-0.53);
    \draw [thick] (-4.96,5.71) -- (-3.96,2.07);
}
}
}

\scn{Construire un angle à l'aide d'un rapporteur}
\caSlide{16-17-18}

\bookSlide{8p145,33p149}[10cm][1]

\scn{Nature d'un angle}
\caSlide{23-24-25}

\slide{cr}{
    \vc{Nature d'un angle}{
    On classe les angles par catégories selon leur mesure :
    \begin{tabular}{|*{6}{C{\cW}|}
        Nul & Aigu & Droit & Obtus & Plat & Plein
    \end{tabular}
    }
}[\cwr]
}

\bookSlide{3p145,26p148}[7cm][2]

\scn{Ecrire un programme de constructions}

\bookSlide{67p153,68p153}[7cm][2]

\ifthenelse{\boolean{answer}}{
    \slide{exo}{
        \exo{67p153}{
    Tracer un triangle \nswr{$CES$ rectangle en $E$}.\\
    Placer un point V sur \nswr{le segment [CS]}\\
    tel que \nswr{$\widehat{CEV} = \ang{30}$
    et tracer la demi-droite $[EV)$}.
}[\dim]

        \exo{68p153}{
        \begin{itemize}
            \item Tracer un carré $BAST$.
            \nswr[0]{
                \item Tracer la diagonale $[BS]$.
                \item Placer le point $I$ sur le segment $[BA]$ de sorte que l'angle $\widehat{BSI}$ soit égal à l'angle $\widehat{BSI}$.
                \item Tracer la demi-droite $[SI)$.
            } 
        \end{itemize}
}[\dim]

    }
}{}

\scn{Construire un diagramme circulaire}
\caSlide{19-20-21-22}

\slide{exo}{\exo{Repartition de vols}{
    On a représenté sur le diagramme suivant les vols du mois de février d'une compagnie aérienne.
    
    \ctikz[0.45]{
    % \boundingBox[11.76][10.56][0.5pt][1][(-2.41,-2.86)]
    \draw [shift={(3.50,1.18)},thick,color=gradeColor,fill=gradeColor,fill opacity=0.10] (0,0) -- (89.85:1.13) arc (89.85:119.85:1.13) -- cycle;
    \draw [shift={(3.50,1.18)},thick,color=gradeColor,fill=gradeColor,fill opacity=0.10] (0,0) -- (119.85:1.13) arc (119.85:149.85:1.13) -- cycle;
    \draw [shift={(3.50,1.18)},thick,color=gradeColor,fill=gradeColor,fill opacity=0.10] (0,0) -- (149.85:1.13) arc (149.85:179.85:1.13) -- cycle;
    \draw[thick,color=gradeColor,fill=gradeColor,fill opacity=0.10] (2.70,1.18) -- (2.70,0.38) -- (3.50,0.38) -- (3.50,1.18) -- cycle;
    \draw [thick] (3.50,1.18) circle (7.48cm);
    \draw [shift={(3.50,1.18)},thick,color=gradeColor] (89.85:1.13) arc (89.85:119.85:1.13);
    \draw [shift={(3.50,1.18)},thick,color=gradeColor] (119.85:1.13) arc (119.85:149.85:1.13);
    \draw [shift={(3.50,1.18)},thick,color=gradeColor] (149.85:1.13) arc (149.85:179.85:1.13);
    \draw [shift={(3.50,1.18)},thick,color=gradeColor,fill=gradeColor,fill opacity=0.35]  (0,0) --  plot[domain=-1.57:1.57,variable=\t]({1*7.48*cos(\t r)+0*7.48*sin(\t r)},{0*7.48*cos(\t r)+1*7.48*sin(\t r)}) -- cycle ;
    \draw [shift={(3.50,1.18)},thick,color=gradeColor,fill=gradeColor,fill opacity=0.28]  (0,0) --  plot[domain=3.14:4.71,variable=\t]({1*7.48*cos(\t r)+0*7.48*sin(\t r)},{0*7.48*cos(\t r)+1*7.48*sin(\t r)}) -- cycle ;
    \draw [shift={(3.50,1.18)},thick,color=gradeColor,fill=gradeColor,fill opacity=0.21]  (0,0) --  plot[domain=2.62:3.14,variable=\t]({1*7.48*cos(\t r)+0*7.48*sin(\t r)},{0*7.48*cos(\t r)+1*7.48*sin(\t r)}) -- cycle ;
    \draw [shift={(3.50,1.18)},thick,color=gradeColor,fill=gradeColor,fill opacity=0.14]  (0,0) --  plot[domain=2.09:2.62,variable=\t]({1*7.48*cos(\t r)+0*7.48*sin(\t r)},{0*7.48*cos(\t r)+1*7.48*sin(\t r)}) -- cycle ;
    \draw [shift={(3.50,1.18)},thick,color=gradeColor,fill=gradeColor,fill opacity=0.07]  (0,0) --  plot[domain=1.57:2.09,variable=\t]({1*7.48*cos(\t r)+0*7.48*sin(\t r)},{0*7.48*cos(\t r)+1*7.48*sin(\t r)}) -- cycle ;
    \draw[color=gradeColor] (9.2,1.65) node {France};
    \draw[color=gradeColor] (-0.70,-2.86) node {Europe};
    \draw[color=gradeColor] (-1.6,2.2) node {Amérique};
    \draw[color=gradeColor] (-0.8,5.4) node {Afrique};
    \draw[color=gradeColor] (1.88,7.70) node {Asie};
    \draw [thick,gradeColor] (3.25,2.12) -- (3.17,2.42);
    \draw [thick,gradeColor] (2.81,1.87) -- (2.60,2.09);
    \draw [thick,gradeColor] (2.56,1.43) -- (2.26,1.51);
}
    Dans chaque cas, indiquer quelle fraction représentent les vols vers :
    \multiColItemize{3}{\item  la France \item l'Europe \item l'Asie}

    Au mois de février, cette compagnie a affrété 576 vols. Calculer le nombre de vols vers :
    \multiColItemize{3}{\item  la France \item l'Europe \item l'Asie}
}[\href{https://cache.media.education.gouv.fr/file/Fractions/22/7/RA16_C4_MATH_fractions_flash1_part_fractions_554227.pdf}
{Utiliser les nombres pour comparer, calculer et résoudre des problèmes :
Les fractions - Un exemple de question flash - « Vision-partage » de la fraction}]}

\slide{exo}{\exo{Démographie du Royaume d'Hyrule}{
    Le royaume d'Hyrule est habité par différents peuples. Nous allons \key{construire un diagramme circulaire} pour représenter sa démographie.
    Voici une estimation de leur répartition dans le royaume :
    \begin{center}
        \begin{tabular}{|*{7}{C{1.75cm}|}}
            \hline
            Peuple & Hylien & Gerudo & Goron  & Zora   & Autres & Totals\\ \hline
            Part   & $51\%$ & $18\%$ & $8\%$ & $7\%$ & $\nswr[0]{16\%}$ & $\nswr[0]{100\%}$ \\ \hline
            Angle  & $\nswr[0]{\ang{193.8}}$ & $\nswr[0]{\ang{68.4}}$ & $\nswr[0]{\ang{30.4}}$ & $\nswr[0]{\ang{26.6}}$ & $\nswr[0]{\ang{60.8}}$ & $\nswr[0]{\ang{360}}$\\ \hline
        \end{tabular} 
    \end{center}
}
[\href{https://www.reddit.com/r/Breath_of_the_Wild/comments/11gzi0f/the_demographic_of_hyrule_in_breath_of_the_wild/}{Reddit (u/\_pe5e\_)}]
}
