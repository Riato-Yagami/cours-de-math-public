% VARIABLES %%%
\setSeq{6}{Angles}
\setGrade{6e}

\def\imgPath{enseignement/6e/nombres/decimaux/}

\dym{\href{https://www.maths-et-tiques.fr/telech/19Angles.pdf}}

% \forPrint
% \forStudents

\obj{
    \item Identifier des angles dans une figure géométrique.
    \item Utiliser la notation « angle ABC » pour les angles.
    \item Comparer des angles, avec recours ou non à leur mesure (gabarit, superposition, calque).
    \item Reproduire un angle donné en utilisant un gabarit.
    \item Estimer qu'un angle est droit, aigu ou obtus.
    \item Utiliser un rapporteur pour mesurer un angle en degrés.
    \item Construire un angle de mesure donnée en degrés, à l'aide du rapporteur.
    \item Construire un diagramme circulaire ou semi-circulaire (proportionnalité).
}

\slide{exo}{
    \exo{Billard}{
    \begin{enumerate}
        \item La balles rebondis avec un angle de \ang{90}
        \begin{center}
            \Billard[Longueur=6cm,Largeur=4cm]{"NEIGE"}
        \end{center}

        \item La balles rebondis avec le même angle qu'elle est arrivé.
        \begin{center}
            \Billard[Vrai,Longueur=7cm,Largeur=4cm]{"MONTAGNE"}
        \end{center}
    \end{enumerate}
}
}