% VARIABLES %%%
\setSeq{6}{Angles}
\setGrade{6e}

\def\imgPath{enseignement/6e/nombres/decimaux/}

\dym{https://www.maths-et-tiques.fr/telech/19Angles.pdf}
\dcwr{https://college-willy-ronis.fr/maths/chapitre-9-les-angles/}

% \forPrint
% \forStudents

\obj{
    \item Identifier des angles dans une figure géométrique.
    \item Utiliser la notation « angle ABC » pour les angles.
    \item Comparer des angles, avec recours ou non à leur mesure (gabarit, superposition, calque).
    \item Reproduire un angle donné en utilisant un gabarit.
    \item Estimer qu'un angle est droit, aigu ou obtus.
    \item Utiliser un rapporteur pour mesurer un angle en degrés.
    \item Construire un angle de mesure donnée en degrés, à l'aide du rapporteur.
    \item Construire un diagramme circulaire ou semi-circulaire (proportionnalité).
}

\slide{cr}{
    \sseq
    \section{Definition}
    \df{}{
    On appelle \key{angle} l'ouverture limitée par deux demi-droites de même origine.
}[\ym]
}

% \slide{exo}{
%     \exo{Billard}{
    \begin{enumerate}
        \item La balles rebondis avec un angle de \ang{90}
        \begin{center}
            \Billard[Longueur=6cm,Largeur=4cm]{"NEIGE"}
        \end{center}

        \item La balles rebondis avec le même angle qu'elle est arrivé.
        \begin{center}
            \Billard[Vrai,Longueur=7cm,Largeur=4cm]{"MONTAGNE"}
        \end{center}
    \end{enumerate}
}
% }

\slide{cr}{
    \section{Mesure}
    \mthd{Mesure d'un angle}{
    Pour mesurer un angle on le compare au \key{degré},
    defini comme $\frac{1}{360}$ de tour complet.
    On utilise pour celà un rapporteur.
    % \rapporteur(A,B,1)
}

\begin{Geometrie}
    pair A,B,C;
    A=u*(1,1);
    B-A=u*(4,0);
    C=cercles(A,3u) intersectionpoint cercles(B,2u);
    trace polygone(A,B,C);
    trace rapporteur(A,C,1);
    label.lft(btex A etex,A);
    label.rt(btex B etex,B);
    label.top(btex C etex,C);
\end{Geometrie}
}

\slide{cr}{
    \def\figScale{0.2}
\def\crossWidth{0.1mm}
\def\crossSize{0.1}

\vc{Nature d'un angle}{
    On classe les angles par catégories selon leur mesure :
    \begin{center}
        \begin{tabular}{|*{3}{C{3.25cm}|}}
            \hline
            Nul & Aigu & Droit
            \\ \hline
            \ctikz{
    % \draw [shift={(0,0)},thick,color=gradeColor,fill=gradeColor,fill opacity=0.10] (0,0) -- (0:2.48) arc (0:0:2.48) -- cycle;
    \draw [thick] (0,0) -- (5,0);
    \drawPoint{}{0}{0}
}&
            \input{resources/enseignement/6e/angles/tikz/angles/aigu.tex}&
            \ctikz{
    \draw[thick,color=gradeColor,fill=gradeColor,fill opacity=0.10] (1.41,0) -- (1.41,1.41) -- (0,1.41) -- (0,0) -- cycle;
    \draw [thick] (0,0) -- (5,0);
    \draw [thick] (0,0) -- (0,5);
    \drawPoint{}{0}{0}
}
            \\ \hline
            $\nswr[0]{0°}$ & $\nswr[0]{0°}<x<\nswr[0]{90°}$ & $\nswr[0]{90°}$
            \\ \hline
            %%
            Obtus & Plat & Plein
            \\ \hline
            \ctikz{
    \draw [shift={(0,0)},thick,color=gradeColor,fill=gradeColor,fill opacity=0.10] (0,0) -- (0:2) arc (0:135:2) -- cycle;
    \draw [thick] (0,0) -- (5,0);
    \draw [thick] (0,0) -- (-3.54,3.54);
    \drawPoint{}{0}{0}
}&%
            \input{resources/enseignement/6e/angles/tikz/angles/plat.tex}&%
            \ctikz{
    \draw [shift={(0,0)},thick,color=gradeColor,fill=gradeColor,fill opacity=0.10] (0,0) -- (0:2) arc (0:360.00:2) -- cycle;
    \draw [thick] (0,0) -- (5,0);
    \drawPoint{}{0}{0}
}
            \\ \hline
            $\nswr[0]{90°}<x<\nswr[0]{180°}$ & $\nswr[0]{180°}$ & $\nswr[0]{360°}$
            \\ \hline
        \end{tabular}
    \end{center}
}[\cwr]
}