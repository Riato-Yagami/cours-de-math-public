% VARIABLES %%%
\setSeq{4}{Nombres}
\setGrade{6e}

\def\imgPath{enseignement/6e/nombres/}

\def\ym{\href{https://www.maths-et-tiques.fr/telech/19Long.pdf}{Yvan Monka}}

% \firstSlide
% \setboolean{answer}{false}
% \setboolean{debugMode}{true}
% Cercle https://www.maths-et-tiques.fr/telech/19Cercle.pdf
%%

\obj{
    \item Utiliser et représenter les grands nombres entiers (en chiffres et en lettres).
    \item Utiliser une fraction et en donner progressivement le statut de nombre.
    \item Utiliser et représenter les nombres décimaux jusqu'à trois décimales.
    \item Ajouter, soustraire et multiplier des nombres décimaux.
    \item Utiliser la division euclidienne.
    \item Résoudre des problèmes relevant des structures additives et multiplicatives en mobilisant une ou plusieurs étapes de raisonnement (proportionnalité).
    \item Organiser un calcul en une seule ligne, utilisant si nécessaire des parenthèses.
    \item Savoir ce qu'est un ordre de grandeur et savoir l'utiliser pour prévoir un résultat.
}

\def\iconPath{prehistoric-numbers/}
\def\iconSize{25pt}
\slide{exo}{
    \act{Numération préhistorique}{
        Certains hommes préhistorique utilisaient leurs main pour communiquer sur des nombres.
        \imgp{historic/chiffres}
        % \imgp{historic/hands}
        \begin{enumerate}
            \item \multiColEnumerate{3}{
                \item $\icon{6} = \bawsr{6}$
                \item $\icon{8} = \bawsr{8}$
                \item $\icon{2} = \bawsr{2}$
            }
        \end{enumerate}
        \item Quel est le nombre le plus grand pouvant etre dégné
    }
}
