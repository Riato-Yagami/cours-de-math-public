% VARIABLES %%%
\setSeq{6}{Nombres - Décimaux}
\setGrade{6e}

\def\imgPath{enseignement/6e/nombres/decimaux/}

\def\ym{\href{https://www.maths-et-tiques.fr/telech/19Nombres1.pdf}{Yvan Monka}}

% \forPrint

% https://www.maths-et-tiques.fr/telech/19Nombres2.pdf

\obj{
    \item Utiliser une fraction et en donner progressivement le statut de nombre.
    \item Utiliser et représenter les nombres décimaux jusqu'à trois décimales.
    \item Ajouter, soustraire et multiplier des nombres décimaux.
}

\scn{Découvrir les fractions décimales}

\def\caPrefix{6e-mars-2023-}
\caSlide{23-24-25}

\bsec{Puissances de dix}

\slide{exo}{
    \act{}{
        \multiColEnumerate{2}{
            \item $\pow{10}{2} = \nswr{\num{\powTenPositive{2}}}$
            \item $\pow{10}{3} = \nswr{\num{\powTenPositive{3}}}$
            \item $\pow{10}{6} = \nswr{\num{\powTenPositive{6}}}$
            \item $\powBrace{10}{15} = \nswr{\num{\powTenPositive{15}}}$
            \item $\powBrace{10}{100} = \nswr{\powTenBrace{100}}$
        }
    }
}

\slide{cr}{
    \sseq\ssec
    \vc{}{
        Une \key{puissance de 10} est le résultat d'un produit dont tous les facteurs sont $10$.
    }

    \expl{}{\ifBeamer{\\}
        \Tableau[%
            DoubleEntree,
            Stretch=1.5,
            Couleur=gradeColor!15,
            LegendesH={$100$,$2$,$\num{1001}$,$\num{100000}$,$\num{200}$,$\num{10}$},
            LegendesV={Puissance de 10 ?},
            Largeur=2cm
        ]{\nswr{Oui},\nswr{Non},\nswr{Non},\nswr{Oui},\nswr{Non},\nswr{Oui}}
    }
}

\caSlide{26-27-28}

\slide{cr}{
    \pr{}{
        Pour $n$ un nombre entier, on a :\vspace*{-0.25cm}
        \begin{align*}
            \powBrace{10}{n} = \nswr{\powTenBrace{n}}
        \end{align*}
    }

    \expl{}{
        \multiColEnumerate{2}{
            \item $\pow{10}{3} = \nswr{\powTenPositive{3}}$
            \item $\pow{10}{6} = \nswr{\powTenPositive{6}}$
        }
    }
}

\bsec{Fraction décimale}

\slide{cr}{
    \ssec
    \df{}{
        On appelle \key{fraction décimale} une fraction dont le \key{dénominateur est une puissance de $10$}.
    }

    \expl{}{\ifBeamer{\\}
        \Tableau[%
            DoubleEntree,
            Stretch=1.5,
            Couleur=gradeColor!15,
            LegendesH={$\frac{1}{10}$,$\frac{1}{2}$,$\frac{5}{10}$,$\frac{1}{\num{1000}}$,$\frac{1}{\num{30000}}$,$\frac{546985}{\num{10000000}}$},
            LegendesV={Fraction décimale ?},
            Largeur=2cm
        ]{\nswr{Oui},\nswr{Oui},\nswr{Non},\nswr{Oui},\nswr{Non},\nswr{Oui}}
    }
}

\slide{cr}{
    \expl{}{
        Ecrire les nombres suivants sous forme de fractions décimales.
        \multiColEnumerate{2}{
            \item $\num{3.2} = \nswr{\frac{32}{10}}$
            \item $\num{10.2} = \nswr{\frac{102}{10}}$
            \item $\num{0.03} = \nswr{\frac{3}{100}}$
            \item $\num{0.0001} = \nswr{\frac{1}{1000}}$
            \item $6 \div 10 \div 10 = \nswr{\frac{6}{100}}$
            \item $32 \div 10 \div 10 \div 10 \div 10 = \nswr{\frac{32}{10000}}$
        }
    }
}

\caSlide{29-30}

\bsec{Nombre décimal}

\slide{cr}{
    \ssec
    \df{}{
        On appelle \key{nombre décimal}, un nombre pouvant s'écrire sous forme de fraction décimale.
    }

    \expl{}{\ifBeamer{\\}
        \Tableau[%
            DoubleEntree,
            Stretch=1.5,
            Couleur=gradeColor!15,
            LegendesH={$\num{0.6}$,$\num{13.2}$,$\frac{1}{10}$,$\frac{1}{2}$,$60$,$\frac{1}{3}$,$\frac{30}{3}$,$\pi$,$0$,$58\div100$},
            LegendesV={Nombre décimal ?},
            Largeur=2cm
        ]{\nswr{Oui},\nswr{Oui},\nswr{Oui},\nswr{Oui},\nswr{Oui},\nswr{Non},\nswr{Oui},\nswr{Non},\nswr{Oui},\nswr{Oui}}
    }
}

\slide{cr}{
    \expl{}{
        \multiColEnumerate{1}{
            \item $2 \div 10 \div 10 = \frac{\quad2\quad}{\nswr{100}} = \nswr{\num{0.02}}$
            \item $\nswr{36 \div 10 \div 10 \div 10 \div 10} = \frac{36}{10000} = \nswr{\num{0.0036}}$
            \item $\nswr{32 \div 10 \div 10 \div 10} = \frac{\quad32\quad}{\nswr{1000}} = \np{0.032}$
        }
    }
}

\slide{cr}{
    \pr{}{Pour $n$ un nombre entier, on a :\vspace{-0.25cm}
    \begin{align*}
        1 \repeatBrace{\div 10}{n}[quotients]
        = \frac{\quad\quad1\quad\quad}{\nswr{\powTenBrace{n}}}
        = \nswr{\underbrace{0, 0 ... 0}_{n \textrm{ zéros}} 1}
    \end{align*}
    }
}

\slide{exo}{
    \exo{Fraction de segments}{
    \begin{enumerate}
        \item Un segment a pour longueur $\frac{3}{10}$ de $1$ mètres.
        Quelle est cette longueur en mètres ? en décimètres ? en centimètres ?
        \item Un segment a pour longueur $\frac{18}{1000}$ de $1$ mètres.
        Quelle est cette longueur en mètres ? en millimètre ? en centimètres ?
    \end{enumerate}
}[\dmeepc{6}[218 et 233]]
}

\slide{exo}{
    \exo{Consommation d'un scooter}{
    Pierrick fait plusieurs fois le même trajet de \Lg[km]{30} avec son scooter.
    
    Il calcul à chaque fois sa vitesse moyenne et note sa consommation en carburant.
    
    Voici ces relevés :
    \multiColItemize{6}{
        \item \Capa{2} \item \Capa{1.10} \item \Capa{2.3} \item \Capa{1.13} \item \Capa{2.03} \item \Capa{1.2}
    }

    La consommation de carburant augmente lorsque la vitesse moyenne augmente.
    
    Place les consommations dans le tableau.

    \begin{center}
        \begin{tabular}{|C{3cm}|*{6}{C{\cW}|}}
            \hline
            Vitesse en \Vitesse[kmh]{} & $20$ & $22$ & $25$ & $35$ & $40$ & $45$ \\
            \hline
            Consommation en \Capa[l]{}
            & \nswr[0]{\np{1.10}}
            & \nswr[0]{\np{1.13}}
            & \nswr[0]{\np{1.2}}
            & \nswr[0]{\np{2}}
            & \nswr[0]{\np{2.03}}
            & \nswr[0]{\np{2.3}} \\
            \hline
        \end{tabular}
    \end{center}

}[\dmeepc{6}[234]]
}

\scn{Décomposer un nombre décimal en fractions décimales}
