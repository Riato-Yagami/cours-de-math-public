% VARIABLES %%%
\setSeq{5}{Nombres - Decimaux}
\setGrade{6e}

\def\imgPath{enseignement/6e/nombres/decimaux/}

\def\ym{\href{https://www.maths-et-tiques.fr/telech/19Nombres1.pdf}{Yvan Monka}}

% https://www.maths-et-tiques.fr/telech/19Nombres2.pdf

\obj{
    \item Utiliser une fraction et en donner progressivement le statut de nombre.
    \item Utiliser et représenter les nombres décimaux jusqu'à trois décimales.
    \item Ajouter, soustraire et multiplier des nombres décimaux.
    \item Résoudre des problèmes relevant des structures additives et multiplicatives en mobilisant une ou plusieurs étapes de raisonnement.
}

\scn{Découvrir les fractions décimales}

\bsubsec{Nombres décimaux}

\slide{exo}{
    \act{}{
        \multiColEnumerate{1}{
            \item $\pow{10}{2} = \awsr{\np{\powTenPositive{2}}}$
            \item $\pow{10}{3} = \awsr{\np{\powTenPositive{3}}}$
            \item $\pow{10}{6} = \awsr{\np{\powTenPositive{6}}}$
            \item $\powBrace{10}{15} = \awsr{\np{\powTenPositive{15}}}$
            \item $\powBrace{10}{100} = \awsr{\powTenBrace{100}}$
        }
    }
}

\slide{cr}{
    \ssubsec

    \vc{}{
        Une \key{puissance de 10} est le résultat d'un produit dont tous les facteurs sont $10$.
    }

    \expl{}{
        \Tableau[%
            DoubleEntree,
            Stretch=1.5,
            Couleur=gradeColor!15,
            LegendesH={$100$,$2$,$\np{1001}$,$\np{100000}$,$\np{200}$,$\np{10}$},
            LegendesV={Puissance de 10 ?},
            Largeur=2cm
        ]{\awsr{Oui},\awsr{Non},\awsr{Non},\awsr{Oui},\awsr{Non},\awsr{Oui}}
    }
}

\slide{cr}{
    \pr{}{
        Pour $n$ un nombre entier, on a :
        \begin{align*}
            \powBrace{10}{n} = \awsr{\powTenBrace{n}}
        \end{align*}
    }

    \expl{}{
        \multiColEnumerate{2}{
            \item $\pow{10}{3} = \powTenPositive{3}$
            \item $\pow{10}{6} = \powTenPositive{6}$
        }
    }
}

\slide{cr}{
    \df{}{
        On appelle \key{fraction décimale} une fraction dont le \key{dénominateur est une puissance de $10$}.
    }

    \expl{}{
        \Tableau[%
            DoubleEntree,
            Stretch=1.5,
            Couleur=gradeColor!15,
            LegendesH={$\frac{1}{10}$,$\frac{1}{2}$,$\frac{5}{10}$,$\frac{1}{\np{1000}}$,$\frac{1}{\np{30000}}$,$\frac{546985}{\np{10000000}}$},
            LegendesV={Fraction décimale ?},
            Largeur=2cm
        ]{\awsr{Oui},\awsr{Oui},\awsr{Non},\awsr{Oui},\awsr{Non},\awsr{Oui}}
    }
}

\slide{exo}{
    \act{}{
        Ecrire les nombres suivants sous forme de fractions décimales.
        \multiColEnumerate{2}{
            \item $\np{3.2} = \awsr{\frac{32}{10}}$
            \item $\np{10.2} = \awsr{\frac{102}{10}}$
            \item $\np{0.03} = \awsr{\frac{3}{100}}$
            \item $\np{0.0001} = \awsr{\frac{1}{1000}}$
            \item $6 \div 10 \div 10 = \awsr{\frac{6}{100}}$
            \item $32 \div 10 \div 10 \div 10 \div 10 = \awsr{\frac{32}{10000}}$
        }
    }
}

\slide{cr}{
    \df{}{
        On appelle \key{nombre décimal}, un nombre pouvant s'écrire sous forme de fraction décimale.
    }

    \expl{}{
        \Tableau[%
            DoubleEntree,
            Stretch=1.5,
            Couleur=gradeColor!15,
            LegendesH={$\np{0.6}$,$\np{13.2}$,$\frac{1}{10}$,$\frac{1}{2}$,$60$,$\frac{1}{3}$,$\frac{30}{3}$,$\pi$,$0$},
            LegendesV={Nombre décimal ?},
            Largeur=2cm
        ]{\awsr{Oui},\awsr{Oui},\awsr{Oui},\awsr{Oui},\awsr{Oui},\awsr{Non},\awsr{Oui},\awsr{Non},\awsr{Oui}}
    }
}

\slide{cr}{
    \pr{}{Pour $n$ un nombre entier, on a :
    \begin{align*}
        1 \repeatBrace{\div 10}{n}[quotients]
        = \frac{1}{\powTenBrace{n}}
        = \underbrace{0, 0 ... 0}_{n \textrm{ zéros}} 1
    \end{align*}
    }

    \expl{}{
        \multiColEnumerate{1}{
            \item $1 \div 10 \div 10 = \frac{1}{\awsr{100}} = \awsr{\np{0.01}}$
            \item $\awsr{1 \div 10 \div 10 \div 10 \div 10} = \frac{1}{10000} = \awsr{\np{0.0001}}$
            \item $1 \awsr{\div 10 \div 10 \div 10} = \frac{1}{\awsr{1000}} = \awsr{\np{0.001}}$
        }
    }
}

\scn{Décomposer un nombre décimal en fractions décimales}
