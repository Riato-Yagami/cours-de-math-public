% VARIABLES %%%
\setSeq{6}{Nombres - Décimaux}
\setGrade{6e}

\def\imgPath{enseignement/6e/nombres/decimaux/}

\def\ym{\href{https://www.maths-et-tiques.fr/telech/19Nombres1.pdf}{Yvan Monka}}

% \setfonts{Papyrus}

% \forPrint
\forStudents

% https://www.maths-et-tiques.fr/telech/19Nombres2.pdf

\obj{
    \item Utiliser une fraction et en donner progressivement le statut de nombre.
    \item Utiliser et représenter les nombres décimaux jusqu'à trois décimales.
    \item Ajouter, soustraire et multiplier des nombres décimaux.
    \item Repérer des nombres décimaux sur une droite graduée.
    \item Ordonner des nombres décimaux.
}

\scn{Découvrir les puissances de 10}

\def\caPrefix{6e-mars-2023-}
\caSlide{23-24-25}

% \bsec{Puissances de dix}

\slide{exo}{
    \act{Produits de facteurs 10}{
    \multiColEnumerate{2}{
        \item $\pow{10}{2} = \nswr{\num{\powTenPositive{2}}}$
        \item $\pow{10}{3} = \nswr{\num{\powTenPositive{3}}}$
        \item $\pow{10}{6} = \nswr{\num{\powTenPositive{6}}}$
        \item $\powBrace{10}{15} = \nswr{\num{\powTenPositive{15}}}$
        \item $\powBrace{10}{100} = \nswr{\powTenBrace{100}}$
    }
}
}

\slide{cr}{
    \sseq\section{Puissances de dix}
    \vc{}{
    Une \key{puissance de 10} est le résultat d'un produit dont tous les facteurs sont $10$.
}
    \expl{}{
    \begin{center}
        \Tableau[%
        DoubleEntree,
        Stretch=1.5,
        Couleur=gradeColor!15,
        LegendesH={$100$,$2$,$\num{1001}$,$\num{100000}$,$\num{200}$,$\num{10}$},
        LegendesV={Puissance de 10 ?},
        Largeur=2cm
    ]{\nswr{Oui},\nswr{Non},\nswr{Non},\nswr{Oui},\nswr{Non},\nswr{Oui}} 
    \end{center}
}
}

\slide{cr}{
    \pr{}{
    Pour $n$ un nombre entier, on a :\vspace*{-0.25cm}
    \begin{align*}
        \powBrace{10}{n} = \nswr{\powTenBrace{n}}
    \end{align*}
}
    \expl{}{
    \multiColEnumerate{2}{
        \item $\pow{10}{3} = \nswr{\powTenPositive{3}}$
        \item $\pow{10}{6} = \nswr{\powTenPositive{6}}$
    }
}
}

\scn{Utiliser la notion de puissance de 10 introduire les fractions décimales}

\caSlide{26-27-28}

% \bsec{Fraction décimale}

\slide{cr}{
    \section{Fraction décimale}
    \df{}{
    On appelle \key{fraction décimale} une fraction dont le \key{dénominateur est une puissance de $10$},
    et le \key{numérateur est un entier}.
}[\wiki{Fraction_(mathématiques)}[Écriture_décimale,_écriture_fractionnaire]]
    \expl{}{
    \begin{center}
        \Tableau[%
        DoubleEntree,
        Stretch=1.5,
        Couleur=gradeColor!15,
        LegendesH={$\frac{1}{10}$,$\frac{1}{2}$,$\frac{5}{10}$,$\frac{1}{\num{1000}}$,$\frac{1}{\num{30000}}$,$\frac{546985}{\num{10000000}}$},
        LegendesV={Fraction décimale ?},
        Largeur=2cm
    ]{\nswr{Oui},\nswr{Oui},\nswr{Non},\nswr{Oui},\nswr{Non},\nswr{Oui}}
    \end{center}
}
}

\slide{cr}{
    \expl{}{
    Ecrire les nombres suivants sous forme de fractions décimales.
    \multiColEnumerate{2}{
        \item $\num{3.2} = \nswr{\dfrac{32}{10}}$
        \item $\num{10.2} = \nswr{\dfrac{102}{10}}$
        \item $\num{0.03} = \nswr{\dfrac{3}{100}}$
        \item $\num{0.0001} = \nswr{\dfrac{1}{1000}}$
        \item $6 \div 10 \div 10 = \nswr{\dfrac{6}{100}}$
        \item $32 \div 10 \div 10 \div 10 \div 10 = \nswr{\dfrac{32}{10000}}$
    }
}
}

\scn{Definir la notion de nombre décimal}
\caSlide{29-30}

% \bsec{Nombre décimal}

\slide{cr}{
    \section{Nombre décimal}
    \df{}{
    On appelle \key{nombre décimal}, un nombre pouvant s'écrire sous forme de fraction décimale.
}[\wiki{Nombre_décimal}]
    \expl{}{
    % \hspace{-1.6cm}%
    \Tableau[%
        DoubleEntree,
        Stretch=1.5,
        Couleur=gradeColor!15,
        LegendesH={$\num{0.6}$,$\num{13.2}$,$\frac{1}{10}$,$\frac{1}{2}$,$60$,$\frac{1}{3}$,$\frac{30}{3}$,$\pi$,$0$,$58\div100$},
        LegendesV={Nombre décimal ?},
        Largeur=1cm
    ]{\nswr{Oui},\nswr{Oui},\nswr{Oui},\nswr{Oui},\nswr{Oui},\nswr{Non},\nswr{Oui},\nswr{Non},\nswr{Oui},\nswr{Oui}}
}
}

\slide{cr}{
    \expl{}{
    \multiColEnumerate{1}{
        \item $2 \div 10 \div 10 = \frac{\quad2\quad}{\nswr{100}} = \nswr{\num{0.02}}$
        \item $\nswr{36 \div 10 \div 10 \div 10 \div 10} = \frac{36}{10000} = \nswr{\num{0.0036}}$
        \item $\nswr{32 \div 10 \div 10 \div 10} = \frac{\quad32\quad}{\nswr{1000}} = \np{0.032}$
    }
}
    \rmk{}{
    Un nombre est décimal si sa partie décimale est \nswr{finie}.
}
    \pr{}{Pour $n$ un nombre entier, on a :\vspace{-0.25cm}
\begin{align*}
    1 \repeatBrace{\div 10}{n}[quotients]
    = \frac{\quad\quad1\quad\quad}{\nswr{\powTenBrace{n}}}
    = \nswr{\underbrace{0, 0 ... 0}_{n \textrm{ zéros}} 1}
\end{align*}
}
}

\scn{Comparaison de nombres décimaux}
\def\caPrefix{6e-mars-2022-}
\caSlide{10-11-12}

\slide{exo}{
    \exo{Fraction de segments}{
    \begin{enumerate}
        \item Un segment a pour longueur $\frac{3}{10}$ de $1$ mètres.
        Quelle est cette longueur en mètres ? en décimètres ? en centimètres ?
        \item Un segment a pour longueur $\frac{18}{1000}$ de $1$ mètres.
        Quelle est cette longueur en mètres ? en millimètre ? en centimètres ?
    \end{enumerate}
}[\dmeepc{6}[218 et 233]]
    \exo{Consommation d'un scooter}{
    Pierrick fait plusieurs fois le même trajet de \Lg[km]{30} avec son scooter.
    
    Il calcul à chaque fois sa vitesse moyenne et note sa consommation en carburant.
    
    Voici ces relevés :
    \multiColItemize{6}{
        \item \Capa{2} \item \Capa{1.10} \item \Capa{2.3} \item \Capa{1.13} \item \Capa{2.03} \item \Capa{1.2}
    }

    La consommation de carburant augmente lorsque la vitesse moyenne augmente.
    
    Place les consommations dans le tableau.

    \begin{center}
        \begin{tabular}{|C{3cm}|*{6}{C{\cW}|}}
            \hline
            Vitesse en \Vitesse[kmh]{} & $20$ & $22$ & $25$ & $35$ & $40$ & $45$ \\
            \hline
            Consommation en \Capa[l]{}
            & \nswr[0]{\np{1.10}}
            & \nswr[0]{\np{1.13}}
            & \nswr[0]{\np{1.2}}
            & \nswr[0]{\np{2}}
            & \nswr[0]{\np{2.03}}
            & \nswr[0]{\np{2.3}} \\
            \hline
        \end{tabular}
    \end{center}

}[\dmeepc{6}[234]]
}

\scn{Décomposer un nombre décimal en fractions décimales}
\caSlide{13-14-15}

\slide{exo}{
    \exo{Comparaison de décimaux}{
    \begin{enumerate}
        \item Ranger dans l'odre croissant les nombres décimaux suivants :
        \multiColItemize{2}{
            \item $\dfrac{6}{10} + \dfrac{1}{100} + \dfrac{1}{10000}$
            \item six cent onze millièmes
            \item $\np{6.1111}$
            \item $6+ \dfrac{101}{1000}$
            \item $6111$ dix-millièmes
            \item $\dfrac{6101}{10000}$
        }
        \item Combien de nombres différents sont écrits dans la liste ci-dessous :
        \multiColItemize{5}{
            \item $\dfrac{1284}{10000}$
            \item $\dfrac{1}{4}$
            \item $\np{1.4}$
            \item $\np{0.25}$
            \item $\dfrac{25}{100}$
        }
    \end{enumerate}
}[\afa{6}[3]]
    \exo{Encadrement de décimaux}{
    \begin{enumerate}
        \item Intercaler un nombre décimal entre : \multiColEnumerate{2}{
            \item $\num{6}$ et $\num{7}$
            \item $\num{10.3}$ et $\num{10.5}$
            \item $\num{9.1}$ et $\num{9.2}$
            \item $\num{0.600}$ et $\num{0.601}$
            \item $\num{3.451}$ et $\num{3.452}$
        }
        \item Encadrer le nombre $\np{28.4597}$ par : \multiColEnumerate{1}{
            \item par deux nombres entiers consécutifs ;
            \item par deux nombres décimaux, au dixième près ;
            \item par deux nombres décimaux, au centième près ;
            \item par deux nombres décimaux, au millième près.
        }
    \end{enumerate}
}[\afa{6}[3]]
}

\scn{Somme de fractions décimales}
\caSlide{16-17-18}

\slide{exo}{
    \exo{Somme de fractions de même dénominateurs}{
    \begin{enumerate}
        \item Calculer en laissant sous forme de fractions décimales: \multiColEnumerate{3}{
            \item $\dfrac{3}{10}+\dfrac{4}{10}$
            \item $\dfrac{26}{100}+\dfrac{31}{100}+\dfrac{43}{100}$
            \item $\dfrac{7}{10}+\dfrac{3}{10}$
        }
        \item Calculer en laissant sous forme de fractions : \multiColEnumerate{3}{
            \item $\dfrac{3}{5}+\dfrac{4}{5}$
            \item $\dfrac{26}{25}+\dfrac{31}{25}+\dfrac{43}{25}$
            \item $\dfrac{7}{2}+\dfrac{7}{2}$
        }
    \end{enumerate}
}[\afa{6}[3]]
    \exo{Nombre décimal caché}{
    À partir des renseignements qui suivent, il trouve le nombre caché :
    \begin{enumerate}
        \item C'est un nombre décimal de 5 chiffres.
        \item Son chiffre des dixièmes est le même que celui de 17,54.
        \item Son chiffre des centièmes est le chiffre des unités de millions de 738 214 006.
        \item Son chiffre des unités est le chiffre des dizaines de mille de 120 008.
        \item Son chiffre des millièmes est la moitié de celui des centièmes.
        \item Son chiffre des dix-millièmes est égal au chiffre des unités.
    \end{enumerate}
}[\afa{6}[2]]
}

\caSlide{19-20-21-22}