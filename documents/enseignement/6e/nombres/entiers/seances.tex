% VARIABLES %%%
\setSeq{4}{Nombres - Entiers}
\setGrade{6e}

\def\imgPath{enseignement/6e/nombres/entiers/}

\def\ym{\href{https://www.maths-et-tiques.fr/telech/19Nombres1.pdf}{Yvan Monka}}

% https://www.maths-et-tiques.fr/telech/19Nombres2.pdf

% \setboolean{newPageOnSlide}{true}
% \setboolean{answer}{false}
% \setboolean{debugMode}{true}
%%

\obj{
    \item Decomposer un nombre dans plusieurs bases.
    \item Conversion de durée.
    \item Utiliser et représenter les grands nombres entiers (en chiffres et en lettres).
    \item Utiliser la division euclidienne.
    \item Organiser un calcul en une seule ligne, utilisant si nécessaire des parenthèses.
    \item Savoir ce qu'est un ordre de grandeur et savoir l'utiliser pour prévoir un résultat.
}

\scn{Découvrir une numération préhistorique}

\qfSlide{
    \begin{enumerate}
        \item $2 + 3 \time 5 = $
        \item $6 - 2 + 3 - 1 =$
        \item $4 \times (6 + 3) =$
    \end{enumerate}
}

\bsec{Compter}
\bsubsec{Base 12}

\slide{exo}{\bvspace{-0.75cm}
    \act{Numération préhistorique}{
        Certains hommes préhistorique utilisaient leurs main pour communiquer sur des nombres.
        \imgp{historic/numbers}[6.5cm]
    }[\dmeepcS]
}

% \endinput

\slide{exo}{
    \begin{enumerate}\setItemColor{act}
        \item \multiColEnumerate{3}{
            \item $\Prehistoric{6} = \awsr{6}$
            \item $\Prehistoric{10} = \awsr{10}$
            \item $\icon{prehistoric-numbers/u/0}[60pt] = 7$
        }
        \item Quel est le nombre le plus grand pouvant etre communiquer de cette manière ?
        \item Comment pourait-on communiquer des nombres plus grands ?
        \saveenumi
    \end{enumerate}
}

\slide{exo}{
    \begin{enumerate}\loadenumi[act]
        \item \multiColEnumerate{2}{
            \item $\Prehistoric{12} = \awsr{12}$
            \item $\Prehistoric{15} = \awsr{15}$
            \item $\Prehistoric{50} = \awsr{50}$
            \item $\Prehistoric{0} = 58$
            \item $\Prehistoric{0} = 100$
            \item $\Prehistoric{135} = \awsr{135}$
        }
        \saveenumi
    \end{enumerate}
}

\slide{exo}{
    \begin{enumerate}\loadenumi[act]
        \item Quel est le plus grand nombre pouvant etre communiqué avec cette méthode ?
        \item Certains hommes préhistoriques comptaient donc par 12 car ils avaient 12 phalanges?
        De la même manière, par combien comptons-nous ? Pourquoi selon vous ?
    \end{enumerate}
}

\scn{Utiliser une numération en base 12}

\slide{exo}{\cdp{Table de 12}{\Table{12}{12}}}

\slide{cr}{
    \sseq\ssec\ssubsec \bvspace{-0.5cm}
    \vc{}{
        Le \key{système duodécimal},
        ou de \key{base $12$}.
        Est une méthode de \key{comptage par douzaines}.
    }[\wiki{Système_duodécimal}]
}

\slide{cr}{
    \expl{}{
        \multiColEnumerate{2}{
            \item $15 =$ \baseDecomposition{15}{12}["hide"]
            \item $56 =$ \baseDecomposition{56}{12}["hide"] 
            \item $\awsr{27} =$ \baseDecomposition{27}{12}
            \item $\awsr{112} =$ \baseDecomposition{112}{12} \saveenumi
        }\multiColEnumerate{1}{\loadenumi
            \item $145 =$ \awsr{\baseDecomposition{145}{12}}
        }
    }
}

\def\scale{1.15}
\slide{exo}{\small
    \exo{Justifiez, en détaillant vos calculs, le nombre de cubes présents dans chacune des figures.}{\calculator
        \multiColEnumerate{2}{
            \item \RepresenterEntier[Base=12,Echelle=\scale]{28}
            \item \RepresenterEntier[Base=12,Echelle=\scale]{100} \saveenumi
        }
    }
}

\slide{exo}{
    \multiColEnumerate{1}{ \loadenumi[exo]
        \item \RepresenterEntier[Base=12,Echelle=\scale]{332}
        \item \RepresenterEntier[Base=12,Echelle=\scale]{1942}
    }
}


\scn{Découvrir la numération Babylonienne}

\qfSlide{
    \multiColEnumerate{2}{
        \item \awsr{6425} = \baseDecomposition{6425}{10}
        \item \awsr{1237} = \baseDecomposition{1237}{12}
        \item 9233 = \baseDecomposition{9233}{10}["hide"]
        \item 232 = \baseDecomposition{232}{12}["hide"]
    }
}

\bsubsec{Base 60}

\def\aspc{\ifbool{answer}{}{\vspace{1cm}}}

\slide{exo}{\bshrink
    \act{Numération Babylonienne}{
        \begin{enumerate}\bvspace{-1cm}
            \item \multiColEnumerate{3}{
                \item \Babylone{24} = 24
                \item \Babylone{6} = 6
                \item \Babylone{50} = 50
            }
            Que représentent le clou \Babylone{1} et le chevron \Babylone{10} ?
            \item \multiColEnumerate{2}{
                \item \Babylone{12} = \awsr{12}
                \item \Babylone{34} = \awsr{34}
                \item \aspc \awsr{\Babylone{23}} = 23
            } \saveenumi
        \end{enumerate}
    }[\wiki{Numération_mésopotamienne}[Numération_sexagésimale_de_position]]
}

\slide{exo}{
    \begin{enumerate} \loadenumi[act]
        \item \multiColEnumerate{2}{
            \item \Babylone{70} = 70
            \item \Babylone{61} = 61
            \item \Babylone{190} = 190
            \item \Babylone{1380} = 1380
            \item \Babylone{3865} = 3865
        }
        Comment ces nombres sont-ils composés ?
        \saveenumi
    \end{enumerate}
}

\slide{exo}{
    \begin{enumerate} \loadenumi[act]
        \item \multiColEnumerate{2}{
            \item \Babylone{86} = \awsr{86}
            \item \aspc \awsr{\Babylone{132}} = 132
            \item \Babylone{325} = \awsr{325}
            \item \Babylone{7271} = \awsr{7271}
            \item \aspc \awsr{\Babylone{10872}} = 10872
        } \saveenumi
    \end{enumerate}
}

\slide{exo}{
    \begin{enumerate} \loadenumi[act]
        \item Les chiffres babyloniens changent de signification selon leur position : on parle donc de \key{numération de position}.
        Notre système de numération actuel est-il aussi un système de numération de position ?
        \item Existe-t-il des éléments que nous comptons encore aujourd'hui de manière similaire aux Babyloniens ?
    \end{enumerate}
}

\scn{Utiliser une numération en base 60}

\qfSlide{
    \exo{Donner l'heure}{
        \multiColEnumerate{3}{
            \item \Horloge[Secondes=false]{7:30}
            \item \Horloge[Secondes=false]{15:24}
            \item \Horloge[Secondes=false]{12:46}
        }
    }
}

\slide{cr}{
    \ssubsec
    \vc{}{
        Le \key{système sexagésimal},
        ou de \key{base $60$}.
        Est une méthode de \key{comptage par soixantaines}.
    }[\wiki{Système_sexagésimal}]
}

\slide{cr}{
    \expl{}{
        \multiColEnumerate{2}{
            \item \awsr{186} = \baseDecomposition{186}{60}
            \item \awsr{1200} = \baseDecomposition{1200}{60}
            \item 720 = \baseDecomposition{75}{60}["hide"]
            \item 1842 = \baseDecomposition{500}{60}["hide"] \saveenumi
        }\multiColEnumerate{1}{\loadenumi
            \item 8350 = \awsr{\baseDecomposition{8350}{60}}
        }
    }
}

\slide{cr}{
    \rmk{}{
        L'usage moderne du sexagésimal est assez proche de celui de la mesure du temps.
        \multiColItemize{2}{
            \item $\Horaire{1} = \awsr{60}\textrm{min}$
            \item $\Horaire{;1} = \awsr{60}\textrm{s}$
        }
    }

    \expl{}{
        \multiColItemize{1}{
            \item $\Horaire{1} = \awsr{3600}\sec$
            \item $\Horaire{;;602} = \parseSeconds{602}["hide"]$
            \item $\Horaire{;;7623} = \parseSeconds{7623}["hide"]$
        }
    }
}

\bookSlide{36p149,38p149,39p149}[7cm][2]

\def\scale{2}

\scn{Utiliser une numération en base 10}

\slide{qf}{\bsmall
    \nullsubsec{}{
        \begin{itemize}
            \item La Lune est à \Ecriture{384 000} de kilomètre de la Terre.
            \item Jupiter est à \Ecriture{91 000 000} de kilomètre de la Terre.
            \item Pluton est à \Ecriture{4 297 000 000} de kilomètre de la Terre.
        \end{itemize}

        Complète le tableau ci-dessous avec ces nombres écrits en chiffres.

        \begin{center}
            \Tableau[%
            DoubleEntree,
            Couleur=gradeColor!15,
            LegendesH={Lune,Jupiter,Pluton},
            LegendesV={Distance à la Terre (km)},
            Largeur=135pt
            ]{\awsr{384 000},\awsr{91 000 000},\awsr{4 297 000 000}}
        \end{center}
    }[\dmeepcS]
}

\bsec{Numération décimales}

\bsubsec{Nombres entiers}

\slide{cr}{
    \ssec\ssubsec
    \vc{}{
        Le \key{système décimal},
        ou de \key{base $10$}.
        Est une méthode de \key{comptage par dizaines}.
    }[\wiki{Système_décimal}]

    \hist{}{
        L'utilisation actuelle des \key{chiffres arabes} repose sur un système de numération \key{décimal et positionnel}.
        Leur diffusion au Moyen-Orient et en Europe serait due à un ouvrage du mathématicien persan d'\key{Al-Khwârizmî} (780-850 ap. J.-C.).
    }[\wiki{Al-Khwârizmî} \wiki{Système_de_numération_indo-arabe}]
}

\slide{cr}{
    \mthd{}{
        Pour écrire un nombre on utilise \key{10 symboles} appelé \key{chiffres}.
        La \key{position} des chiffres dans l'écriture d'un nombre détermine sa valeure.
    }

    \expl{}{
        \begin{enumerate}
            \item $\np{110} =$ \baseDecomposition{110}{10}["hide"] 
            \item $\np{5841} =$ \baseDecomposition{5841}{10}["hide"]
            \item $\awsr{{\np{1010}}} =$ \baseDecomposition{1010}{10}
        \end{enumerate}
    }
}

\slide{cr}{
    \rmk{}{
        On regroupe les chiffres de nombres par groupes de trois afin d'en améliorer la lisibilité.
    }
}

\slide{cr}{\bshrink
    \expl{}{
        Classer les chiffres des nombres suivants,
        puis les réécrire correctement :
        \multiColEnumerate{3}{
            \item $54 454$
            \item $36 119 312$
            \item $3 300 001 200$
        }
        \bvspace{-0.5cm}
        \decimalTable{{54454,36119312,3300001200}}["hide"]
        \multiColEnumerate{3}{
            \item $\awsr{\np{54454}}$
            \item $\awsr{\np{36 119 312}}$
            \item $\awsr{\np{3 300 001 200}}$
        }
    }
}

\scn{Manipuler des nombres entiers dans le système décimal}

\slide{qf}{
    Donner le chiffres des :
    \multiColEnumerate{2}{
        \item milliers de $\np{56165453}$
        \item milliards de $\np{546160006546521}$
        \item dizaines de milliers de $\np{346805235}$
        \item centaines de milliards de $\np{340045235}$
        \item centaines de $\np{65465,654654}$
        \item dizaines de millions de $\np{211010100,001}$
    }
}

\slide{exo}{
    \exo{Nombres mystères}{
        \begin{enumerate}
            \item Donne un exemple de nombre inférieur à $400$ pour lequel :
            \begin{itemize}
                \item le chiffre des dizaines est la moitié du chiffre des centaines.
                \item la somme des chiffres est $11$.
            \end{itemize}\saveenumi
        \end{enumerate}
    }[\dmeepcS]
}

\slide{exo}{
    \begin{enumerate}\loadenumi[exo]
        \item Donne un exemple de nombre à quatre chiffres tel que :
        \begin{itemize}
            \item le chiffre des dizaines est la moitié du chiffre des centaines.
            \item la somme des chiffres est $11$.
        \end{itemize}
        \item Donne un exemple de nombres à trois chiffres pour lequel :
        \begin{itemize}
            \item il est inférieur à $\np{2000}$ ;
            \item il a trois chiffres identiques ;
            \item la somme de ses chiffres est 10.
        \end{itemize}
    \end{enumerate}
}

\scn{Formaliser la notion de la division euclidienne}

\slide{qf}{Compléter les divisions suivantes
    \multiColEnumerate{2}{
        \item \longDivision{155}{5}
        \item \longDivision{700}{49}
    }
}

\bsubsec{Division euclidienne}

\slide{exo}{\bshrink
    \act{}{%
    Le roi de Divisia possède $27$ pièces d'or et souhaite les partager équitablement entre $4$ chevaliers.
    Les pièces restantes seront données à son écuyer.
    
    \begin{enumerate}
        \item Combien de pièces chaque chevalier recevra-t-il ?
        \item Combien de pièces resteront pour l'écuyer ?
        \item Supposons maintenant que le roi possède $100$ pièces et qu'il les partage entre $6$ chevaliers :
        \begin{enumerate}
            \item Combien de pièces chaque chevalier recevra-t-il ?
            \item Combien de pièces resteront pour l'écuyer ?
        \end{enumerate}\saveenumi
    \end{enumerate}
    }
}

\slide{exo}{
    \begin{enumerate}\loadenumi[act]
        \item Toujours avec $100$ pièces,
        existe-t-il un nombre de chevaliers tel que; l'écuyer reçoive:
        \multiColEnumerate{1}{
            \item $0$ pièce ?
            \item plus de pièces qu'un chevalier ?
            \item autant de pièces qu'un chevalier ?
            \item autant de pièces qu'il y a de chevalier ?
        }
    \end{enumerate}
}

\slide{cr}{\bsmall
    \ssubsec

    \df{}{
        Pour deux entiers $a$ et $b$,
        on appelle \key{division euclidienne}
        du \textcolor{Green}{\key{dividende} $a$} par le \textcolor{Red}{\key{diviseur} $b$}
        l'expression :
        \begin{align*}
            \textcolor{Green}{a}
            = \textcolor{Red}{b}
            \times \textcolor{Blue}{q}
            + \textcolor{Violet}{r}
        \end{align*}
        Où le \textcolor{Blue}{\key{quotient} $q$} et \textcolor{Violet}{\key{reste} $r$} sont deux entiers avec
        $\textcolor{Violet}{r} < \textcolor{Red}{b}$.
    }

    \expl{}{
        \multiColEnumerate{3}{
            \item $100 = 6 \times \awsr{16} + \awsr{4}$
            \item $246 = 3 \times \awsr{82} + \awsr{0}$
            \item $360 = 23 \times \awsr{15} + \awsr{15}$
        }
    }
}

\scn{Notion de divisibilité}

\slide{qf}{
    \exo{}{
        Quel est le nombre de:
        \multiColEnumerate{2}{
            \item dizaines dans $750$.
            \item milliers dans $\np{665454}$.
            \item millions dans $\np{9876502300}$.
            \item dizaines de milliers dans $\np{121321}$.
            \item centaines de millions dans $\np{2313251}$.
            \item centaines dans $\np{352154.16}$.
        }
    }
}

\slide{cr}{\bsmall
    \df{}{
        Pour $a$ et $b$ deux entiers,
        on dit que $b$ \key{divise} $a$ si le reste de la division euclidienne de $a$ par $b$ est $0$.
    }

    \pr{}{
        \Sialors{$b$ \key{divise} $a$}{$a$ est un \key{multiple} de $b$}
    }

    \begin{center}
        \expl{}{
            \Tableau[%
                DoubleEntree,
                Stretch=1.5,
                Couleur=gradeColor!15,
                LegendesH={\qquad$4$\qquad,\qquad$6$\qquad,\qquad$12$\qquad,\qquad$31$\qquad},
                LegendesV={Divise 62 ?, Multiple de 4 ?},
                Largeur=2cm
            ]{
                \awsr{Oui},\awsr{Non},\awsr{Oui},\awsr{Oui},
                \awsr{Oui},\awsr{Non},\awsr{Oui},\awsr{Non}
            }
        }
    \end{center}
}

\slide{exo}{
    \exo{}{
        Une fleuriste dispose de 1815 fleurs.
        Doit-elle réaliser des bouquets de 16 fleurs ou de 17 fleurs pour en utiliser le plus possible ?
    }[\iP{6}{2021}[2][16]]
}
\bookSlide{21p45,23p45,25p45}[7cm][2]

\scn{Priorités opératoires}

\bsec{Opérations}
\bsubsec{Règles opératoires}

\slide{qf}{\bvspace{-0.5cm}
    \exo{}{
        \multiColEnumerate{1}{
            \item $\np{44420} \times 100 = $
            \item $981 \times \np{100000} = $
            \item $\np{685540000} \div \np{10} = $
            \item $\np{230020000000} \div \np{10000} = $
            \item $\np{10001} \times \np{20000} = $
            \item $\np{3090300} \div \np{300} = $
        }
    }
}

\slide{cr}{
    \ssec\ssubsec

    \rl{}{
        Les calculs se font dans l'ordre des priorités suivant:%
        \begin{enumerate}
            \item La multiplication et la division
            \item L'addition et la soustraction
        \end{enumerate}
    }
}

\slide{cr}{
    \rl{}{
        En cas d'opérations de mêmes priorités, on effectue les opérations de gauche à droite.
    }

    \expl{}{
        \multiColEnumerate{1}{
            \item $3 - 2 + 3 = \awsr{1 + 3 = 4}$ 
            \item $19 - 6\times 3 = \awsr{19 - 18 =  1}$
            \item $3 + \np{3.2} \times 2 - 4 = \awsr{3 + \np{6.4} -4 = \np{9.4} - 4 = \np{5.4}}$
        }
    }
}

\slide{cr}{
    \rl{}{
        On commence par effectuer les calculs entre parenthèses.
    }

    \expl{}{
        \multiColEnumerate{1}{
            \item $(1 + 2) \times 21 = \awsr{3 \times 21 = 63}$
            \item $(11 \times 3) + (15 \div 2) = \awsr{33 + 7.5 = 40.5}$
            \item $((13 - (3 - 2)) + 2) = \awsr{(13 - 1) + 2 = 12 + 2 = 14}$
        }
    }
}

\bsubsec{Vocabulaire opératoires}

% \newpage

\slide{cr}{
    \ssubsec
    \bvspace{-0.5cm}
    \vc{}{
        On connait quatres types d'opérations :
        \begin{itemize}
            \item L'\key{addition} permet de calculer la \key{somme} de deux \key{termes}.
            \item La \key{soustraction}  permet de calculer la \key{différence} entre deux \key{termes}.
            \item La \key{multiplication} permet de calculer la \key{produit} de deux \key{facteurs}.
            \item La \key{division} permet de calculer la \key{quotient} de deux \key{nombres}.
        \end{itemize}
    }
}

\slide{cr}{
    \vc{}{Dans un calcul,
    le type de la dernière opération effectuée détermine le nom donné au calcul dans son ensemble.}
    \bvspace{-0.5cm}
    \expl{Nommer les calculs suivants}{
        \bvspace{-0.5cm}
        \multiColEnumerate{1}{
            \item $1,6 + 4$ est \awsr{la somme} de \awsr{$1,6$ et $4$}.
            \item $(\frac{2}{6} + 3) \times 9$ est \awsr{le produit } de \awsr{$\frac{2}{6} + 3$ et $9$}.
            \item $6,6 + 1 \times 8$ est \awsr{la somme de $6,6$ par $1 \times 8$}.
            \item $\frac{2}{6} + 3 - 9$ est \awsr{la différence entre $\frac{2}{6}$ et $9$}.
            \item $\pi \div (3 - 9)$ est \awsr{le quotient de $\pi$ par $(3 - 9)$}.
        }
    }
}

\slide{exo}{\bshrink
    \exo{Les bons comptes}{
        Pour résoudre les problèmes suivants :  
        \begin{itemize}
            \item Présenter chaque calcul séparément, en précisant ce que représente la valeur obtenue à chaque étape.  
            \item Présenter tous les calculs en une seule expression permettant d'obtenir le résultat final.  
        \end{itemize}
        
        \begin{enumerate}
        \item Xavier possède \Prix{28} dans sa tirelire.  
                Son grand-père lui donne \Prix{75}. Il a désormais \Prix{53} de plus que sa sœur Christine.  
                Quelle somme d'argent possède Christine ?\saveenumi
        \end{enumerate}
    }
}

\slide{exo}{
    \begin{enumerate}\loadenumi[exo]
        \item Un commerçant achète sept rouleaux de \Lg[m]{50} de tissu.  
        Chaque rouleau coûte \Prix{392}.  
        Il revend le tissu au prix de \Prix{12} par mètre.  
        Quel bénéfice réalise-t-il après avoir revendu la totalité du tissu ?
        \item Julien et Georges possèdent à eux deux un total de \Prix{47}.  
        Julien dépense \Prix{12} et Georges dépense \Prix{7}.  
        Après ces dépenses, ils ont chacun la même somme.  
        Quelle somme Julien possédait-il avant sa dépense ?
    \end{enumerate}
    \awsr[0]{
        \begin{enumerate}
            \item 
            \begin{itemize} 
                \item \begin{itemize}
                    \item $28 + 75 = 103$ : Xavier possède maintenant \Prix{103}.
                    \item $103 - 53 = 50$ : Christine possède \Prix{50}.
                \end{itemize}
                \item $28 + 75 - 53 = 50$
            \end{itemize}
            \item Le bénéfice est la différence entre le prix de vente total et le prix d'achat total.
            \begin{itemize}
                \item \begin{itemize}
                    \item $392 \times 7 = 2744$ : prix d'achat total des 7 rouleaux, soit \Prix{2744}.
                    \item $7 \times 50 = 350$ : le commerçant dispose de \Lg[m]{350} de tissu.
                    \item $350 \times 12 = 4200$ : prix de vente total du tissu, soit \Prix{4200}.
                    \item $4200 - 2744 = 1456$ : bénéfice réalisé après la revente de la totalité du tissu, soit \Prix{1456}.
                \end{itemize}
                \item $7 \times 50 \times 12 - 392 \times 7 = 1456$
            \end{itemize}
            \item 
            \begin{itemize}
                \item \begin{itemize}
                    \item $47 - 12 = 35$ : après la dépense de Julien, il reste \Prix{35}.
                    \item $35 - 7 = 28$ : après la dépense de Georges, il reste \Prix{28}.
                    \item $28 \div 2 = 14$ : chacun possède maintenant \Prix{14}.
                    \item $14 + 12 = 26$ : Julien possédait donc \Prix{26} avant sa dépense.
                \end{itemize}
                \item $(47 - 12 - 7) \div 2 + 12 = 26$
            \end{itemize}
        \end{enumerate}        
    }
}