\setSeq{7}{Géométrie dans l'espace - Solides}
\setGrade{6e}
\def\imgPath{enseignement/6e/geometrie-dans-l-espace/solides/}
\dym{https://www.maths-et-tiques.fr/telech/19Para_Cube.pdf}

% \forPrint
% \forStudents

\obj{
    \item Reconnaître des solides (pavé droit, cube, cône et cylindre).
    \item Identifier les caractéristiques de différents solides :
    sommets, faces et arêtes.
    \item Construire le patron d'un pavé droit.
    \item Reproduire des solide.
}

\scn{Associer des empreintes et solides}

\slide{qf}{
    \exo{Associer empreintes et solides}{
    Former le maximum de paires entre les empreintes et les solides.
    \ctikz[0.75]{
    \boundingBox[17][12][0.5pt][1][(0,1.5)][iso]
    \node at (3.82, 11.48) {\cir[gradeColor]{1}};
    \node at (7.90, 12.12) {\cir[gradeColor]{2}};
    \node at (13.88, 8.66) {\cir[gradeColor]{3}};
    \node at (14.30, 5.88) {\cir[gradeColor]{4}};
    \node at (12.96, 3.32) {\cir[gradeColor]{5}};
    \draw (9.40,8.24) node[anchor=north west] {\cir[gradeColor]{A}};
    \draw (7.18,4.48) node[anchor=north west] {\cir[gradeColor]{B}};
    \draw (1.62,8.72) node[anchor=north west] {\cir[gradeColor]{C}};
    \draw [thick] (14.72,7.50) -- (15.59,8) -- (14.72,8.50) -- (15.59,9) -- (14.72,9.50) -- (13.86,9) -- (12.12,10) -- (11.26,9.50) -- (14.72,7.50);
    \draw [thick] (5.20,4) -- (8.66,2) -- (9.53,2.50) -- (9.53,4.50) -- (8.66,4) -- (8.66,2);
    \draw [thick] (6.06,6.50) -- (5.20,6) -- (4.33,5.50) -- (4.33,4.50) -- (5.20,4) -- (6.06,4.50) -- (6.06,5.50) -- (5.20,5) -- (4.33,5.50);
    \draw [thick] (0,8) -- (2.60,6.50) -- (3.46,7) -- (2.60,8.50) -- (3.46,9) -- (2.60,9.50) -- (1.73,9) -- (0.87,9.50) -- (0,9) -- (0,8);
    \draw [thick] (3.46,7) -- (3.46,7);
    \draw [thick] (2.60,6.50) -- (2.60,6.50) -- (1.73,8) -- (0,9);
    \draw [thick] (6.93,8) -- (10.39,6) -- (11.26,6.50) -- (11.26,7.50) -- (10.39,8) -- (11.26,8.50) -- (10.39,9) -- (9.53,8.50) -- (8.66,9) -- (7.79,8.50) -- (7.79,9.50) -- (8.66,10) -- (8.66,9);
    \draw [thick] (10.39,6) -- (10.39,7) -- (11.26,7.50) -- (11.26,8.50);
    \draw [thick] (6.93,9) -- (6.93,8);
    \draw [thick] (7.79,9.50) -- (6.93,10) -- (6.93,9);
    \draw [thick] (6.93,10) -- (7.79,10.50) -- (8.66,10);
    \draw [thick] (11.26,3.50) -- (13.86,2) -- (14.72,2.50) -- (13.86,3) -- (14.72,3.50) -- (13.86,4) -- (12.99,3.50) -- (12.12,4) -- (11.26,3.50);
    \draw [thick] (5.20,10) -- (1.73,12) -- (2.60,12.50) -- (6.06,10.50) -- (5.20,10);
    \draw [thick] (6.93,5) -- (6.93,6) -- (7.79,6.50) -- (8.66,6) -- (7.79,5.50) -- (6.93,6) -- (6.06,6.50);
    \draw [thick] (7.79,5.50) -- (7.79,4.50) -- (8.66,4);
    \draw [thick] (8.66,6) -- (8.66,5) -- (9.53,4.50);
    \draw [thick] (6.06,5.50) -- (6.93,5);
    \draw [thick] (5.20,5) -- (5.20,4);
    \draw [thick] (5.20,12) -- (6.93,13) -- (10.39,11) -- (9.53,10.50) -- (6.93,12) -- (6.06,11.50) -- (5.20,12);
    \draw [thick] (3.46,8) -- (3.46,9);
    \draw [thick] (3.46,8) -- (3.04,7.73);
    \draw [thick] (1.73,8) -- (2.60,8.50);
    \draw [thick] (12.99,6.50) -- (13.86,6) -- (12.99,5.50) -- (13.86,5) -- (14.72,5.50) -- (15.59,5) -- (16.45,5.50) -- (13.86,7) -- (12.99,6.50);
    \draw [thick] (7.79,8.50) -- (10.39,7);
}
}

}

\scn{Reconnaître des polyèdres particuliers}

\slide{cr}{
    \sseq
    \def\figScale{0.6}
% \newcommand{\sld}[2]{\item \dividePage{#1}
% {\input{resources/enseignement/6e/geometrie-dans-l-espace/solides/tikz/#2.tex}}[0.5]
% }

\newcommand{\sld}[2]{\item #1
    \input{resources/enseignement/6e/geometrie-dans-l-espace/solides/tikz/#2.tex}
}

\df{Polyèdres}{%
    On appelle :%
    \multiColItemize{2}{%
        \sld{%
            \key{polyhèdre}, un solide constitué de \key{faces} polygonales.\\
            Les côtés de ces polygones sont appelés  \key{arêtes}.\\
            Les extrémités des arêtes sont des points appelés  \key{sommets}.
        }{polyhedre}
        \sld{%
            \nswr{\key{prisme droit}}[5cm], un polyhèdre constitué de deux \key{bases} polygonales superposables,
            reliées entre elles par des faces \nswr{\key{rectangulaires}}[5cm].
        }{prisme-droit}
        \sld{%
            \nswr{\key{pavé droit}}[5cm], un \nswr{prisme droit}[5cm] dont les bases sont \nswr{\key{rectangulaires}}.
        }{pave-droit}
        \sld{%
            \nswr{\key{cube}}[5cm], un \nswr{pavé droit}[5cm] dont toutes les faces sont des \nswr{\key{carrés}}[5cm].
        }{cube}
        \sld{%
            \nswr{\key{pyramide}}[5cm], un polyhèdre formé d'une base polygonale reliée à un \key{sommet} par des faces \nswr{\key{triangulaires}}[5cm].
        }{pyramide}
    }
}[\wiki{Polyèdre}[Polyèdres_simples]]
}

\bookSlide{27p254}[7cm]

\slide{exo}{
    \exo{Classer des polyèdres}{\def\cW{5cm}
    Ranger les polyèdres suivants dans les catégories correspondantes ci-dessous :
    \ctikz[0.75]{
    \boundingBox[19.92][21][0.5pt][1][(0,-0.5)][iso]
    \node at (2.87, 17.07) {\cir[gradeColor]{1}};
    \node at (10.04, 18.83) {\cir[gradeColor]{2}};
    \node at (6.55, 13.56) {\cir[gradeColor]{3}};
    \node at (14.53, 15.61) {\cir[gradeColor]{4}};
    \node at (3.22, 9.41) {\cir[gradeColor]{5}};
    \node at (10.01, 8.66) {\cir[gradeColor]{6}};
    \node at (18.18, 8.79) {\cir[gradeColor]{7}};
    \node at (2.29, 2.88) {\cir[gradeColor]{8}};
    \node at (10.46, 2.09) {\cir[gradeColor]{9}};
    \node at (15.17, 4.59) {\cir[gradeColor]{10}};
    \draw [thick] (0.00,3.00) -- (4.33,5.50) -- (6.06,4.50) -- (6.06,2.50) -- (1.73,0.00) -- (0.00,1.00) -- (0.00,3.00) -- (1.73,2.00) -- (1.73,0.00);
    \draw [thick] (1.73,2.00) -- (4.33,3.50);
    \draw [thick,dashed] (4.33,3.50) -- (6.06,2.50);
    \draw [thick,dashed] (4.33,3.50) -- (4.33,5.50);
    \draw [thick] (6.06,4.50) -- (1.73,2.00);
    \draw [thick] (7.79,7.50) -- (10.39,6.00) -- (12.99,7.50) -- (10.39,11.00) -- (7.79,7.50);
    \draw [thick,dashed] (7.79,7.50) -- (10.39,9.00);
    \draw [thick,dashed] (10.39,9.00) -- (12.99,7.50);
    \draw [thick] (10.39,11.00) -- (10.39,6.00);
    \draw [thick,dashed] (10.39,9.00) -- (10.39,11.00);
    \draw [thick] (14.72,5.50) -- (15.59,6.00) -- (16.45,5.50) -- (15.59,5.00) -- (14.72,5.50) -- (14.72,2.50) -- (15.59,2.00) -- (15.59,5.00);
    \draw [thick] (16.45,5.50) -- (16.45,2.50) -- (15.59,2.00);
    \draw [thick,dashed] (14.72,2.50) -- (15.59,3.00);
    \draw [thick,dashed] (15.59,3.00) -- (16.45,2.50);
    \draw [thick,dashed] (0.00,1.00) -- (1.73,2.00);
    \draw [thick] (1.73,11.00) -- (1.73,8.00) -- (4.33,6.50) -- (4.33,9.50) -- (1.73,11.00);
    \draw [thick,dashed] (1.73,8.00) -- (3.46,9.00);
    \draw [thick,dashed] (3.46,9.00) -- (1.73,11.00);
    \draw [thick] (4.33,9.50) -- (6.06,7.50) -- (4.33,6.50);
    \draw [thick,dashed] (6.06,7.50) -- (3.46,9.00);
    \draw [thick] (15.59,8.00) -- (15.59,10.00) -- (17.32,9.00) -- (17.32,7.00) -- (18.19,6.50) -- (19.05,7.00) -- (19.92,8.50) -- (19.92,10.50) -- (17.32,9.00);
    \draw [thick] (15.59,10.00) -- (18.19,11.50) -- (19.92,10.50);
    \draw [thick] (15.59,8.00) -- (17.32,7.00);
    \draw [thick,dashed] (19.05,7.00) -- (17.32,8.00);
    \draw [thick,dashed] (17.32,8.00) -- (16.45,7.50);
    \draw [thick,dashed] (17.32,8.00) -- (18.19,9.50);
    \draw [thick,dashed] (18.19,9.50) -- (18.19,11.50);
    \draw [thick,dashed] (19.92,8.50) -- (18.19,9.50);
    \draw [thick] (6.93,12.00) -- (8.66,13.00) -- (6.93,15.00) -- (5.20,13.00) -- (6.93,12.00) -- (6.93,11.00) -- (5.20,13.00);
    \draw [thick,dashed] (5.20,13.00) -- (6.93,14.00);
    \draw [thick,dashed] (6.93,15.00) -- (6.93,14.00);
    \draw [thick] (6.93,11.00) -- (8.66,13.00);
    \draw [thick,dashed] (6.93,14.00) -- (6.93,11.00);
    \draw [thick,dashed] (6.93,14.00) -- (8.66,13.00);
    \draw [thick] (12.99,13.50) -- (14.72,12.50) -- (16.45,13.50) -- (16.45,15.50) -- (14.72,14.50) -- (12.99,15.50) -- (12.99,13.50);
    \draw [thick] (12.99,15.50) -- (14.72,16.50) -- (16.45,15.50);
    \draw [thick] (14.72,14.50) -- (14.72,12.50);
    \draw [thick,dashed] (14.72,14.50) -- (12.99,13.50);
    \draw [thick,dashed] (14.72,14.50) -- (16.45,13.50);
    \draw [thick,dashed] (14.72,14.50) -- (14.72,16.50);
    \draw [thick] (7.79,1.50) -- (10.39,0.00) -- (12.12,1.00) -- (9.53,3.50) -- (10.39,0.00);
    \draw [thick] (9.53,3.50) -- (7.79,1.50);
    \draw [thick] (9.53,3.50) -- (11.26,2.50) -- (12.12,1.00);
    \draw [thick,dashed] (9.53,3.50) -- (9.53,1.50);
    \draw [thick,dashed] (7.79,1.50) -- (9.53,1.50);
    \draw [thick,dashed] (9.53,1.50) -- (11.26,2.50);
    \draw [thick] (1.73,16.00) -- (3.46,15.00) -- (5.20,16.00) -- (5.20,18.00) -- (3.46,17.00) -- (2.60,17.50) -- (1.73,18.00) -- (1.73,16.00);
    \draw [thick] (2.60,17.50) -- (3.46,18.00) -- (2.60,19.50) -- (1.73,18.00);
    \draw [thick] (2.60,19.50) -- (2.60,18.50);
    \draw [thick,dashed] (2.60,18.50) -- (3.46,18.00);
    \draw [thick,dashed] (2.60,18.50) -- (1.73,18.00);
    \draw [thick] (2.60,17.50) -- (2.60,19.50);
    \draw [thick,dashed] (2.60,18.50) -- (3.46,19.00);
    \draw [thick] (3.46,19.00) -- (5.20,18.00);
    \draw [thick,dashed] (3.46,19.00) -- (3.46,17.00);
    \draw [thick,dashed] (3.46,17.00) -- (5.20,16.00);
    \draw [thick,dashed] (1.73,16.00) -- (3.46,17.00);
    \draw [thick] (3.46,17.00) -- (3.46,15.00);
    \draw [thick] (3.03,18.75) -- (3.46,19.00);
    \draw [thick] (7.79,16.50) -- (9.53,15.50) -- (11.26,15.50) -- (12.99,17.50) -- (12.99,19.50) -- (7.79,18.50) -- (7.79,16.50);
    \draw [thick,dashed] (7.79,16.50) -- (12.99,17.50);
    \draw [thick] (7.79,18.50) -- (9.53,17.50) -- (9.53,15.50);
    \draw [thick] (9.53,17.50) -- (11.26,17.50) -- (11.26,15.50);
    \draw [thick] (11.26,17.50) -- (12.99,19.50);
    \draw [thick] (6.93,12.00) -- (6.93,15.00);
}
    \multiColItemize{1}{
        \item Polyèdres «quelconques» : \nswr{\cir[answer]{1} \cir[answer]{3}}
        \item Prismes droits «quelconques» : \nswr{\cir[answer]{2} \cir[answer]{5} \cir[answer]{7}}
        \item Pavés droits : \nswr{\cir[answer]{4} \cir[answer]{8} \cir[answer]{10}}
        \item Pyramides : \nswr{\cir[answer]{6} \cir[answer]{9}}
    }
}
}

\scn{Construire des solides en perspective isométrique}

\slide{qf}{
    \exo{Comptage dans un solide}{\bshrink
    \dividePage{\ctikz[1]{
    \boundingBox[2.5][3.5][0.5pt][0.5][(0,0)][cavalier]
    \draw [thick] (0.00,1.50) -- (0.00,0.00) -- (1.50,0.00) -- (2.50,1.00) -- (2.50,2.50) -- (1.50,1.50) -- (0.00,2.50) -- (1.00,3.50) -- (2.50,2.50);
    \draw [thick,dashed] (1.00,2.50) -- (1.00,3.50);
    \draw [thick] (0.00,2.50) -- (0.00,1.50);
    \draw [thick,dashed] (1.00,2.50) -- (1.00,1.00) -- (0.00,0.00);
    \draw [thick,dashed] (1.00,1.00) -- (2.50,1.00);
    \draw [thick] (1.50,1.50) -- (1.50,0.00);
}}{
        \begin{enumerate}
            \item Combien d'arêtes comporte ce solide ?
            \item de faces ?
            \item de sommets ?
        \end{enumerate}
    }[0.3]
}

}

\slide{exo}{
    \newcommand{\se}[1]{
    \item \dividePage{
        \input{resources/enseignement/6e/geometrie-dans-l-espace/solides/empreintes-manquantes/solide-#1.tex}
    }{
        \input{resources/enseignement/6e/geometrie-dans-l-espace/solides/empreintes-manquantes/empreinte-#1.tex}
    }
}

\def\figScale{0.65}
\def\dotSize{1pt}

\exo{Construction de polyèdres en perspective isométrique}{
    Compléter les empreintes ou solides manquants :

    \dividePage{Solides :}{Empreintes :}
    \multiColEnumerate{1}{
        \se{1} \se{2} \se{3} \se{4} \se{5}
    }
}
}

\scn{Construire le patron d'un pavé droit}

\slide{exo}{
    \act{Patron d'un pavé droit}{
    On considère le pavé droit ci-dessous :

    \ctikz[1]{
    \boundingBox[8][7][0.5pt][1][(0,-0.5)][iso]
    \drawPoint{A}{3.46}{0}
    \drawPoint{B}{0.87}{1.50}
    \drawPoint{C}{0.87}{3.50}
    \drawPoint{G}{4.33}{5.50}
    \drawPoint{E}{6.93}{2}
    \drawPoint{D}{3.46}{2}
    \drawPoint{H}{6.93}{4}
    \drawPoint{F}{4.33}{3.50}
    \draw[color=gradeColor] (1.96,0.64) node {3};
    \draw[color=gradeColor] (3.26,1.15) node {2};
    \draw[color=gradeColor] (5.34,0.87) node {4};
    \draw [thick] (0.87,1.50) -- (3.46,0) -- (3.46,2) -- (6.93,4) -- (4.33,5.50) -- (0.87,3.50) -- (0.87,1.50);
    \draw [thick] (6.93,4) -- (6.93,2) -- (3.46,0);
    \draw [thick] (3.46,2) -- (0.87,3.50);
    \draw [thick,dashed] (4.33,3.50) -- (0.87,1.50);
    \draw [thick,dashed] (4.33,3.50) -- (6.93,2);
    \draw [thick,dashed] (4.33,3.50) -- (4.33,5.50);
}

    \begin{enumerate}
        \item Combien de faces compose ce pavé droit ?
        \item Sans considérer les noms des sommets, combien de polygones différents composent ces faces ?
        Préciser leurs natures.
        \item Dessinez ces polygones. (Les unités de mesure sont données en \Lg[cm]{})
        \item Combien de fois chacun de ces polygones apparaît-il dans le pavé droit?
        \item Dessinez à main levée tous les polygones avec les noms des sommets.
        \item Sur une feuille blanche, dessinez tous les polygones avec les noms de leurs sommets.
        Chaque polygone doit partager au moins un côté avec un autre polygone.
        Les sommets doivent être nommés et peuvent apparaître plusieurs fois.
        \item Découpez le patron du pavé droit obtenu à l'étape précédente et pliez-le pour obtenir un pavé droit.
    \end{enumerate}
}
}

\bookSlide{53p259}[7cm]

\scn{Reconnaître des solides de révolution particuliers}

\slide{qf}{
    \exo{Compléter un patron}{
    \ctikz[1]{
    \boundingBox[4][2][0.5pt][0.5][(0,0)][cavalier]
    \draw[color=gradeColor] (3.73,0.61) node {4cm};
    \draw[color=gradeColor] (2.64,0.72) node {2cm};
    \draw[color=gradeColor] (1.61,0.00) node {6cm};
    \draw [thick] (1.00,2.14) -- (4.00,2.14) -- (4.00,1.14) -- (3.00,0.14) -- (3.00,1.14) -- (4.00,2.14);
    \draw [thick] (3.00,1.14) -- (0.00,1.14) -- (1.00,2.14);
    \draw [thick] (0.00,1.14) -- (0.00,0.14) -- (3.00,0.14);
}

    Recopiez à main levée ce patron de pavé droit, puis complétez les longueurs et ajoutez les codages manquants.
    \ctikz[\ifBA{0.3}{0.5}]{
    \boundingBox[16.8][8][0.5pt][0.5][(0,0)][dot]
    \draw[color=gradeColor] (11.39,5.56) node {... cm};
    \draw[color=gradeColor] (7.21,7.16) node {... cm};
    \draw[color=gradeColor] (1.08,4.25) node {...cm};
    \draw [thick,gradeColor] (8.03,2.00) -- (8.00,6.00) -- (14.00,6.04) -- (14.03,2.04) -- (8.03,2.00) -- (6.03,1.99) -- (6.00,5.99) -- (8.00,6.00) -- (7.99,8.00) -- (13.99,8.04) -- (14.00,6.04) -- (16.00,6.05) -- (16.03,2.05) -- (14.03,2.04) -- (14.04,0.04) -- (8.04,0.00) -- (8.03,2.00);
    \draw [thick,gradeColor] (6.00,5.99) -- (0.00,5.95) -- (0.03,1.95) -- (6.03,1.99);
}
}
}

\slide{cr}{
    \def\figScale{2.25}
% \newcommand{\solide}[2]{\item \dividePage{#1}{\input{resources/enseignement/5e/geometrie-dans-l-espace/solides/tikz/#2.tex}}}%

\newcommand{\solide}[2]{\item #1
    \input{resources/enseignement/5e/geometrie-dans-l-espace/solides/tikz/#2.tex}
}

\df{Solides de révolution}{%
    On appelle :
    \multiColItemize{1}{
        \item \key{solide de révolution} un solide obtenu en faisant tourner une figure plane autour d'un axe.
        \solide{\nswr{\key{cylindre de révolution}}[5cm] un solide de révolution dont les bases sont des cercles superposables}{cylindre}        
        \solide{\nswr{\key{cône}}[5cm] un solide de révolution dont la base est un cercle et dont les faces latérales sont des triangles.}{cone}
        \solide{\nswr{\key{sphère}}[5cm] une surface constituée de tous les points situés à une même distance d'un point appelé \key{centre}.}{sphere}
    }
}[\wiki{Solide_de_révolution}]
}

\bookSlide{30p255}[7cm][1][\gradeBook][qf]