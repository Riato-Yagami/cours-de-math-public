\slide{exo}{
    \act{cruic}{
    }[\href{https://www.facebook.com/share/p/Vqwr8YVPZZkraPh4/}{Olivier Po}]
}

\slide{exo}{
    \tice{\Geogebra}{La danse des parallélogrammes}{
        \begin{enumerate}
            \item Tracer un parallélogramme ABCD.
            Tracer la diagonale $[BD]$ et y placer un point E.
            Tracer la droite $(d1)$ passant par $E$ parallèle à $(BC)$.
            Elle coupe $(AB)$ en $I$ et $(DC)$ en $L$.
            Tracer la droite $(d2)$ passant par $E$ parallèle à $(AB)$.
            Elle coupe $(AD)$ en $J$ et $(BC)$ en $K$.
            \item Colorier en bleu les aires des quadrilatères EIAJ et EKCL respectivement.
            \item Prouver que les aires bleues sont égales quelque que soit la position du point E sur [BD].
        \end{enumerate}
    }[\prbltq{la-danse-des-parallelogrammes}]
}
