\setSeq{8}{Ecriture fractionnaire - Comparaison}
\setGrade{5e}
\def\imgPath{enseignement/5e/ecriture-fractionnaire/comparaison/}
\dym{https://www.maths-et-tiques.fr/telech/19Fract1.pdf}
% https://www.maths-et-tiques.fr/telech/19Fract2.pdf

% \forStudents
% \forPrint

\def\a{\ensuremath{\textcolor{Red}{a}}}
\def\b{\ensuremath{\textcolor{Blue}{b}}}
\def\q{\ensuremath{\textcolor{Green}{q}}}
\def\fab{\ensuremath{\dfrac{\a}{\b}}}

\obj{
    \item Décomposer une fraction comme une somme (ou d'une différence) d'un entier et d'une fraction
    \item Reconnaître et produire des fractions égales.
    \item Calculer des sommes de fractions de même dénominateur ou multiples l'un de l'autre.
    \item Relier fractions, proportions et pourcentages.
}

\scn{Découvrir la définition du quotient}

\slide{qf}{
    \exo{Repartition de vols}{
    On a représenté sur le diagramme suivant les vols du mois de février d'une compagnie aérienne.
    
    \ctikz[0.45]{
    % \boundingBox[11.76][10.56][0.5pt][1][(-2.41,-2.86)]
    \draw [shift={(3.50,1.18)},thick,color=gradeColor,fill=gradeColor,fill opacity=0.10] (0,0) -- (89.85:1.13) arc (89.85:119.85:1.13) -- cycle;
    \draw [shift={(3.50,1.18)},thick,color=gradeColor,fill=gradeColor,fill opacity=0.10] (0,0) -- (119.85:1.13) arc (119.85:149.85:1.13) -- cycle;
    \draw [shift={(3.50,1.18)},thick,color=gradeColor,fill=gradeColor,fill opacity=0.10] (0,0) -- (149.85:1.13) arc (149.85:179.85:1.13) -- cycle;
    \draw[thick,color=gradeColor,fill=gradeColor,fill opacity=0.10] (2.70,1.18) -- (2.70,0.38) -- (3.50,0.38) -- (3.50,1.18) -- cycle;
    \draw [thick] (3.50,1.18) circle (7.48cm);
    \draw [shift={(3.50,1.18)},thick,color=gradeColor] (89.85:1.13) arc (89.85:119.85:1.13);
    \draw [shift={(3.50,1.18)},thick,color=gradeColor] (119.85:1.13) arc (119.85:149.85:1.13);
    \draw [shift={(3.50,1.18)},thick,color=gradeColor] (149.85:1.13) arc (149.85:179.85:1.13);
    \draw [shift={(3.50,1.18)},thick,color=gradeColor,fill=gradeColor,fill opacity=0.35]  (0,0) --  plot[domain=-1.57:1.57,variable=\t]({1*7.48*cos(\t r)+0*7.48*sin(\t r)},{0*7.48*cos(\t r)+1*7.48*sin(\t r)}) -- cycle ;
    \draw [shift={(3.50,1.18)},thick,color=gradeColor,fill=gradeColor,fill opacity=0.28]  (0,0) --  plot[domain=3.14:4.71,variable=\t]({1*7.48*cos(\t r)+0*7.48*sin(\t r)},{0*7.48*cos(\t r)+1*7.48*sin(\t r)}) -- cycle ;
    \draw [shift={(3.50,1.18)},thick,color=gradeColor,fill=gradeColor,fill opacity=0.21]  (0,0) --  plot[domain=2.62:3.14,variable=\t]({1*7.48*cos(\t r)+0*7.48*sin(\t r)},{0*7.48*cos(\t r)+1*7.48*sin(\t r)}) -- cycle ;
    \draw [shift={(3.50,1.18)},thick,color=gradeColor,fill=gradeColor,fill opacity=0.14]  (0,0) --  plot[domain=2.09:2.62,variable=\t]({1*7.48*cos(\t r)+0*7.48*sin(\t r)},{0*7.48*cos(\t r)+1*7.48*sin(\t r)}) -- cycle ;
    \draw [shift={(3.50,1.18)},thick,color=gradeColor,fill=gradeColor,fill opacity=0.07]  (0,0) --  plot[domain=1.57:2.09,variable=\t]({1*7.48*cos(\t r)+0*7.48*sin(\t r)},{0*7.48*cos(\t r)+1*7.48*sin(\t r)}) -- cycle ;
    \draw[color=gradeColor] (9.2,1.65) node {France};
    \draw[color=gradeColor] (-0.70,-2.86) node {Europe};
    \draw[color=gradeColor] (-1.6,2.2) node {Amérique};
    \draw[color=gradeColor] (-0.8,5.4) node {Afrique};
    \draw[color=gradeColor] (1.88,7.70) node {Asie};
    \draw [thick,gradeColor] (3.25,2.12) -- (3.17,2.42);
    \draw [thick,gradeColor] (2.81,1.87) -- (2.60,2.09);
    \draw [thick,gradeColor] (2.56,1.43) -- (2.26,1.51);
}
    Dans chaque cas, indiquer quelle fraction représentent les vols vers :
    \multiColItemize{3}{\item  la France \item l'Europe \item l'Asie}

    Au mois de février, cette compagnie a affrété 576 vols. Calculer le nombre de vols vers :
    \multiColItemize{3}{\item  la France \item l'Europe \item l'Asie}
}[\href{https://cache.media.education.gouv.fr/file/Fractions/22/7/RA16_C4_MATH_fractions_flash1_part_fractions_554227.pdf}
{Utiliser les nombres pour comparer, calculer et résoudre des problèmes :
Les fractions - Un exemple de question flash - « Vision-partage » de la fraction}]
}

\slide{exo}{
    \act{}{\noCalculator
    Trouver, si possible,
    les nombres manquants dans les égalités suivantes :
    \multiColEnumerate{3}{
        \item $8 \times \nswr{12} = 96$
        \item $2 \times \nswr{\np{2.5}} = 5$
        \item $5 \times \nswr{\dfrac{4}{5}} = 4$
        \item $0 \times \nswr{NaN} = 6$
        \item $8 \times \nswr{\dfrac{17}{8}} = 17$
        \item $7 \times \nswr{\dfrac{\np{4.2}}{7}} = \np{4.2}$
    }
}[\href{https://cache.media.education.gouv.fr/file/Fractions/23/2/RA16_C4_MATH_fractions_flash3_sens_quotient_554232.pdf}
{Utiliser les nombres pour comparer, calculer et résoudre des problèmes : les fractions}]
}

\slide{cr}{
    \sseq
    \df{}{
    Pour $a$ et $b$ deux nombres, avec $b$ non nul.
    On appelle \key{quotient} de $a$ par $b$ le nombre $q$ tel que $b \times q = a$.\\
    On le note $\frac{a}{b}$.
}[\wiki{Quotient}]
    \expl{}{
    \multiColEnumerate{2}{
        \item $5 \times \nswr{\dfrac{6}{5}} = 6$
        \item $-3 \times \nswr{\dfrac{10}{-3}} = 10$
        \item $\pi \times \nswr{\dfrac{1}{\pi}} = 1$
        \item $\np{19.6} \times \dfrac{12}{\np{19.6}} = \nswr{12}$
    }
}
}

\scn{Utiliser la définition du quotient}

\slide{qf}{
    \exo{Chocolat au lait - Olé}{
    \begin{enumerate}
        \item On prépare une boisson chocolatée en mélangeant du chocolat et du lait.
        \begin{itemize}
            \item La recette $A$ mélange $3$ doses de chocolat pour $2$ doses de lait.
            \item La recette $B$ mélange $2$ doses de chocolat pour $1$ dose de lait.
        \end{itemize}
        Quel est le Mélange qui a le plus le goût du chocolat ?
        \item On remplit deux récipients identiques,
        l'un avec le liquide A,
        l'autre avec le liquide B.
        \begin{itemize}
            \item $5$ litres du liquide $A$ pèsent $3$ kg.
            \item $7$ litres du liquide $B$ pèsent $4$ kg.
        \end{itemize}
        Lequel des deux récipients ainsi remplis est le plus lourd ?
    \end{enumerate}
}[\prbltq{chocolat-au-lait-ole}]
}

\bookSlide{40p145,42p145,43p145,48p145}[7.5cm][2]

\scn{Exercice de recherche sur les fractions}

\slide{exo}{
    \exo{Aller-retour}{
    \calculator
    \begin{enumerate}
        \item Complète les égalités suivantes
        \multiColEnumerate{3}{
            \item $\frac{\nswr[0]{3}[1cm]}{\nswr[0]{1}[1cm]} = 3$
            \item $\frac{\nswr[0]{7}[1cm]}{5} = \np{1.4}$
            \item $\frac{\nswr[0]{233}[1cm]}{100} = \np{2.33}$
        }
        \item On sait que $\frac{10}{3} = 3,3333333...$
        \begin{enumerate}
            \item Donne une fraction de deux nombres entiers dont l'écriture décimale est :
            $2,333333333...$
            \item Même demande pour : $1,888888888...$
        \end{enumerate}
        \item Même consigne pour :
        \multiColEnumerate{3}{
            \item $0,45\,45\,45\,45\,45...$
            \item $3,57\,57\,57\,57\,57...$
            \item $5,475\,475\,475\,475...$
        }
    \end{enumerate}
}[\prbltq{aller-retour}]
}

\slide{exo}{
    \newcommand{\engrenages}[2]{
    \begin{center}
        #1 dents et #2 dents\\
        \Engrenages[ListeCouleurs={Cornsilk,LightGreen},Unite=0.6mm]{1/#1,1/#2}
    \end{center}
}

\exo{Engrenages}{
    Pour chacun des quatre engrenages ci-dessous, lorsque la roue droite $(D)$ effectue un tour,
    combien de tour(s) effectue la roue gauche $(G)$?\\
    \dividePage{
        \multiColEnumerate{1}{
            \item \engrenages{10}{30}
            \item \engrenages{20}{30}\saveenumi
        }
    }{
        \multiColEnumerate{1}{\loadenumi
            \item \engrenages{30}{40}
            \item \engrenages{30}{20}
        }
    }
}[\prbltq{engrenages}]
}

\scn{Fractions égales}

\bookSlide{24p144,25p144}[10cm][1][\gradeBook][qf]

\slide{qf}{
    \exo{D'accord ou pas d'accord ?}{
    \begin{align*}
        \dfrac{6+2}{6+4}
        = \dfrac{\cancel{6}+2}{\cancel{6}+4}
        = \dfrac{2}{4}
        = \dfrac{1}{2}
    \end{align*}
}[\href{https://cache.media.education.gouv.fr/file/Fractions/23/4/RA16_C4_MATH_fractions_flash4_travail_erreur_554234.pdf}
{Utiliser les nombres pour comparer, calculer et résoudre des problèmes :
les fractions - Un exemple de questions flash - Travail sur l'erreur}]
}

\slide{cr}{
    \pr{Fractions égales}{
    On ne change pas une fraction lorsqu'on multiplie son numérateur et son dénominateur par un même nombre.
    C'est-à-dire que pour toute fraction $\frac{a}{b}$ et pour tout nombre non nul $k$, on a :
    \[
    \frac{a}{b} = \frac{a \times k}{b \times k}
    \]
}
    \def\nswrWidth{0.4cm}

\expl{}{
    \multiColEnumerate{3}{
        \item $\dfrac{2}{3}
        = \dfrac{2\times\nswr{6}[\nswrWidth]}{3\times\nswr{6}[\nswrWidth]}
        = \dfrac{12}{18}$
        \item $\dfrac{4}{5}
        = \dfrac{4\times\nswr{7}[\nswrWidth]}{5\times\nswr{7}[\nswrWidth]}
        = \dfrac{28}{35}$
        \item $\dfrac{8}{16}
        = \dfrac{8\times\nswr{\dfrac{1}{8}}}{16\times\nswr{\dfrac{1}{8}}[\nswrWidth]}
        = \dfrac{1}{2}$
        \item $\dfrac{\np{1.5}}{2}
        = \dfrac{\np{1.5}\times\nswr{10}[\nswrWidth]}{2\times\nswr{10}[\nswrWidth]}
        = \dfrac{15}{20}$
        \item $\dfrac{5}{6}
        = \dfrac{\nswr{5\times9}[0.7cm]}{\nswr{6\times9}[0.7cm]}
        = \dfrac{45}{54}$
    }
}
}

\scn{Comparer des fractions}

\bookSlide{27p144,63p146}[10cm][1][\gradeBook][qf]

\bookSlide{61p146,57p146,58p146,64p146}[7.5cm][2]