%%% VARIABLES %%%
\setSeq{5}{Angles et parallélisme}
\setGrade{5e}
\def\imgPath{enseignement/5e/angles-et-parallelisme/}
\def\ym{\href{https://www.maths-et-tiques.fr/telech/19Angles5e.pdf}{Yvan Monka}}
% \forPrint
% \def\caPrefix{6e-juin-2022-}
%%
% [\href{https://myriade.editions-bordas.fr/Myriade4e/assets/cherchons-ensemble-chapitre-10-angles-et-parallelisme-triangles-semblables/preview}{Myriade 4e 2016}]

\obj{
    \item À partir des connaissances suivantes :
    codage des figures, caractérisations angulaires du parallélisme (angles alternes internes, angles correspondants),
    mener des raisonnements en utilisant des propriétés des figures, des configurations et des symétries.
    \item Démontrer que la somme des angles dans un triangle est de \ang{180}
}

\scn{Définir la notion d'angles alternes-internes et d'angles corrépondants}

\slide{qf}{\bshrink
    \ctikz[\ifBA{0.55}{0.9}]{
        \draw[gray!40] (-4.5,-7.5) rectangle (12.5,2.5);
        \draw[color=gradeColor] (5.42,-1.85) node {(d)};
        \draw[color=gradeColor] (4.08,1.09) node {(a)};
        \draw[color=gradeColor] (10.30,-4.29) node {(c)};
        \draw[color=gradeColor] (6.14,-5.57) node {(d)};
        \draw[color=gradeColor] (1.34,-3.65) node {(f)};
        \draw[color=gradeColor] (-2.24,-2.53) node {(g)};
        \draw[color=gradeColor] (1.96,-5.15) node {(e)};
        \draw [thick] (-0.40,-2.44) -- (7.48,-1.18);
        \draw [thick] (5.54,1.50) -- (0.96,-1.48);
        \draw [thick] (5.56,-3.44) -- (11.54,-4.20);
        \draw [thick] (10,-2.70) -- (3.74,-6.66);
        \draw [thick] (-3.22,-4.18) -- (2.18,-3.32);
        \draw [thick] (-2.06,-1.54) -- (-1.20,-6.95);
        \draw [thick] (-2.90,-5.64) -- (4.17,-4.51);
        \draw [thick] (-1.28,-3.87) -- (-1.21,-4.31) -- (-1.61,-4.37);
        \draw [thick] (-1.04,-5.34) -- (-0.97,-5.78) -- (-1.37,-5.84);
    }%
    \bvspace{-1.25cm}Les droites suivantes sont-elles sécantes ?
    \bvspace{-0.15cm}
    \multiColEnumerate{2}{
        \item $(a)$ et $(b)$
        \item $(d)$ et $(c)$
        \item $(f)$ et $(e)$
        \item $(g)$ et $(e)$
    }
}

\bsec{Angles alternes-internes et angles correspondants}

\slide{exo}{\bshrink
    \act{}{
        \ctikz[\ifBeamer{0.3}\ifArticle{0.5}]{
    \draw[gray!40] (-8,-15) rectangle (21,10);
    \draw [shift={(-2.34,3.99)},thick,color=gradeColor,fill=gradeColor,fill opacity=0.10] (0,0) -- (-75.47:1.06) arc (-75.47:19.12:1.06) -- cycle;
    \draw [shift={(-1.90,2.28)},thick,color=gradeColor,fill=gradeColor,fill opacity=0.10] (0,0) -- (-160.36:1.06) arc (-160.36:-75.47:1.06) -- cycle;
    \draw [shift={(10.34,0.90)},thick,color=gradeColor,fill=gradeColor,fill opacity=0.10] (0,0) -- (58.10:1.06) arc (58.10:173.22:1.06) -- cycle;
    \draw [shift={(10.34,0.90)},thick,color=gradeColor,fill=gradeColor,fill opacity=0.10] (0,0) -- (-121.90:1.06) arc (-121.90:-6.78:1.06) -- cycle;
    \draw [shift={(17.90,-0.95)},thick,color=gradeColor,fill=gradeColor,fill opacity=0.10] (0,0) -- (149.79:1.06) arc (149.79:240.46:1.06) -- cycle;
    \draw [shift={(16.45,-3.50)},thick,color=gradeColor,fill=gradeColor,fill opacity=0.10] (0,0) -- (-29.07:1.06) arc (-29.07:60.46:1.06) -- cycle;
    \draw [shift={(4.38,-5.55)},thick,color=gradeColor,fill=gradeColor,fill opacity=0.10] (0,0) -- (11.92:1.06) arc (11.92:80.48:1.06) -- cycle;
    \draw [shift={(4.93,-2.30)},thick,color=gradeColor,fill=gradeColor,fill opacity=0.10] (0,0) -- (140.82:1.06) arc (140.82:260.48:1.06) -- cycle;
    \draw [shift={(12.47,-9.50)},thick,color=gradeColor,fill=gradeColor,fill opacity=0.10] (0,0) -- (23.28:1.06) arc (23.28:160.72:1.06) -- cycle;
    \draw [shift={(15.60,-10.60)},thick,color=gradeColor,fill=gradeColor,fill opacity=0.10] (0,0) -- (-107.08:1.06) arc (-107.08:-19.28:1.06) -- cycle;
    \draw [shift={(-4.32,-1.70)},thick,color=gradeColor,fill=gradeColor,fill opacity=0.10] (0,0) -- (0.82:1.06) arc (0.82:111.40:1.06) -- cycle;
    \draw [shift={(-4.32,-1.70)},thick,color=gradeColor,fill=gradeColor,fill opacity=0.10] (0,0) -- (-179.18:1.06) arc (-179.18:-68.60:1.06) -- cycle;
    \draw [shift={(8.20,6.30)},thick,color=gradeColor,fill=gradeColor,fill opacity=0.10] (0,0) -- (51.41:1.06) arc (51.41:167.24:1.06) -- cycle;
    \draw [shift={(11.74,5.50)},thick,color=gradeColor,fill=gradeColor,fill opacity=0.10] (0,0) -- (27.08:1.06) arc (27.08:167.24:1.06) -- cycle;
    \draw [shift={(6.72,-8.78)},thick,color=gradeColor,fill=gradeColor,fill opacity=0.10] (0,0) -- (-156.16:1.06) arc (-156.16:-91.33:1.06) -- cycle;
    \draw [shift={(6.64,-12.11)},thick,color=gradeColor,fill=gradeColor,fill opacity=0.10] (0,0) -- (88.67:1.06) arc (88.67:182.92:1.06) -- cycle;
    \draw [shift={(-3.51,-8.93)},thick,color=gradeColor,fill=gradeColor,fill opacity=0.10] (0,0) -- (48.98:1.06) arc (48.98:133.40:1.06) -- cycle;
    \draw [shift={(-2.43,-10.08)},thick,color=gradeColor,fill=gradeColor,fill opacity=0.10] (0,0) -- (-46.60:1.06) arc (-46.60:0.28:1.06) -- cycle;
    \draw [thick] (1.19,0.74)-- (9.24,-5.81);
    \draw [thick] (0.45,-6.38)-- (12.03,-3.93);
    \draw [thick] (5.66,2.08)-- (4.03,-7.65);
    \draw [thick] (15.65,0.36)-- (19.47,-1.87);
    \draw [thick] (14.67,-2.51)-- (18.31,-4.53);
    \draw [thick] (19.28,1.48)-- (15.79,-4.68);
    \draw [thick] (6.78,1.32)-- (13.29,0.55);
    \draw [thick] (11.87,3.35)-- (8.84,-1.52);
    \draw [thick] (-3.01,6.57)-- (-1.43,0.48);
    \draw [thick] (-4.75,3.15)-- (3.33,5.96);
    \draw [thick] (-3.43,1.74)-- (3.43,4.18);
    \draw [thick] (10.36,-8.76)-- (17.54,-11.28);
    \draw [thick] (18.31,-6.99)-- (9.45,-10.81);
    \draw [thick] (-5.29,0.78)-- (-2.92,-5.28);
    \draw [thick] (-6.71,-1.74)-- (-1.78,-1.66);
    \draw [thick] (-5.61,-4.92)-- (-1.11,-3.47);
    \draw [thick] (14.80,-13.21)-- (17.06,-5.86);
    \draw [thick] (5.34,6.95)-- (14.73,4.82);
    \draw [thick] (9.73,8.22)-- (7.07,4.89);
    \draw [thick] (9.66,4.43)-- (15.90,7.62);
    \node at (0.17, 4.47) {\cir[gradeColor]{1}};
    \node at (8.95, 5.63) {\cir[gradeColor]{2}};
    \node at (11.96, 2.13) {\cir[gradeColor]{3}};
    \node at (15.47, -1.17) {\cir[gradeColor]{4}};
    \node at (-3.13, -2.76) {\cir[gradeColor]{5}};
    \node at (5.37, -3.79) {\cir[gradeColor]{6}};
    \node at (14.48, -9.10) {\cir[gradeColor]{7}};
    \draw [thick] (2.93,-10.45)-- (9.34,-7.62);
    \draw [thick] (3.74,-12.26)-- (8.60,-12.01);
    \draw [thick] (6.76,-7.09)-- (6.61,-13.18);
    \node at (5.73, -10.17) {\cir[gradeColor]{8}};
    \draw [thick] (-5.26,-7.09)-- (-0.93,-11.66);
    \draw [thick] (-5.86,-11.62)-- (-1.36,-6.45);
    \draw [thick] (-6.32,-10.10)-- (0.84,-10.06);
    \node at (-2.32, -8.43) {\cir[gradeColor]{9}};
}
    }
}

\slide{exo}{\bsmall
    \begin{enumerate}\setItemColor{act}
        \item Dans les figures \cir[gradeColor]{4} et \cir[gradeColor]{6}, les angles représentés sont dits {\key{alternes-internes}}.
        Ce n'est pas le cas pour les autres figures. 
        À partir de ces observations :
        \begin{itemize}
            \item Dessine à main levée deux couples d'angles alternes-internes et deux couples d'angles qui ne sont pas alternes-internes.
            \item Propose une définition pour expliquer dans quelles conditions deux angles sont alternes-internes.
        \end{itemize}
        \item De même, seuls les angles de la figure \cir[gradeColor]{2} sont dits {\key{correspondants}}. 
        \begin{itemize}
            \item Dessine à main levée deux couples d'angles correspondants et deux couples d'angles qui ne sont pas correspondants.
            \item Propose une définition pour expliquer dans quelles conditions deux angles sont correspondants.
        \end{itemize}
    \end{enumerate}
}

\slide{cr}{\bsmall
    \sseq\ssec
    \df{}{
        Deux angles formés par deux droites coupées par une sécante sont dits \key{alternes-internes} si :
        \begin{itemize}
            \item ils sont situés de part et d'autre de la sécante (\key{alternes});
            \item ils sont situés entre les deux droites (\key{internes});
            \item ils ne sont pas sur le même sommet.
        \end{itemize}
    }[\wiki{Angles_alternes-internes}]
}

\slide{cr}{
    \expl{}{
        \ctikz[0.75]{
    \draw[gray!40] (-0.5,-5.5) rectangle (9.5,2.5);
    \draw [shift={(5.95,-0.35)},thick,color=gradeColor,fill=gradeColor,fill opacity=0.10] (0,0) -- (-135:0.60) arc (-135:-43.69:0.60) -- cycle;
    \draw [shift={(3.24,-3.06)},thick,color=gradeColor,fill=gradeColor,fill opacity=0.10] (0,0) -- (45:0.60) arc (45:174.56:0.60) -- cycle;
    \draw [thick] (3.62,1.88) -- (8.08,-2.38);
    \draw [thick] (0.48,-2.80) -- (7.62,-3.48);
    \draw [thick] (7.92,1.62) -- (1.52,-4.78);
}
    }
}

\slide{cr}{
    \df{}{
        Deux angles formés par deux droites coupées par une sécante sont dits \key{correspondants} si :
        \begin{itemize}
            \item ils sont du même côté de la sécante;
            \item l'un est situé entre les deux droites et l'autre hors des deux droites;
            \item ils ne sont pas sur le même sommet.
        \end{itemize}
    }[\wiki{Angles_alternes-internes}]
}

\slide{cr}{
    \expl{}{
        \ctikz[0.55]{
    \draw[gray!40] (-0.5,-6.5) rectangle (13.5,4.5);
    \draw [shift={(6.93,0.45)},thick,color=gradeColor,fill=gradeColor,fill opacity=0.10] (0,0) -- (78.66:0.60) arc (78.66:175.20:0.60) -- cycle;
    \draw [shift={(6.46,-1.92)},thick,color=gradeColor,fill=gradeColor,fill opacity=0.10] (0,0) -- (78.66:0.60) arc (78.66:180.57:0.60) -- cycle;
    \draw [thick] (1.08,0.94) -- (11.32,0.08);
    \draw [thick] (2.58,-1.96) -- (12.58,-1.86);
    \draw [thick] (7.48,3.18) -- (5.78,-5.30);
}
    }
}

\bookSlide{16p296,27p297}[7cm]

\scn{Découvrire la propriétée sur les angles alternes égaux}
\slide{qf}{\bshrink
    \ctikz[\ifBA{0.65}{1}]{
    \draw[gray!40] (-7.5,1.8) rectangle (1.5,9);
    \draw [shift={(-6.60,6.88)},thick,color=gradeColor,fill=gradeColor,fill opacity=0.10] (0,0) -- (-57.62:0.60) arc (-57.62:-2.62:0.60) -- cycle;
    \draw [shift={(-0.58,4.52)},thick,color=gradeColor,fill=gradeColor,fill opacity=0.10] (0,0) -- (-66.57:0.60) arc (-66.57:83.43:0.60) -- cycle;
    \draw[thick,color=gradeColor,fill=gradeColor,fill opacity=0.10] (-0.53,4.94) -- (-0.95,4.99) -- (-1,4.56) -- (-0.58,4.52) -- cycle;
    \drawPoint{A}{-6.6}{6.88}
    \draw[color=gradeColor] (-5.94,7.33) node {$55\textrm{\degre}$};
    \drawPoint{C}{-0.58}{4.52}
    \draw[color=gradeColor] (0.52,4.65) node {$150\textrm{\degre}$};
    \drawPoint{E}{-0.34}{6.59}
    \drawPoint{G}{0.15}{2.84}
    \drawPoint{H}{-2.99}{4.8}
    \drawPoint{I}{-4.26}{3.19}
    \drawPoint{J}{-5.06}{4.45}
    % \draw [thick] (-0.53,4.94) -- (-0.95,4.99);
    \draw [thick] (-0.13,8.42) -- (-0.58,4.52) -- (0.15,2.84);
    \draw [thick] (-6.60,6.88) -- (-0.34,6.59);
    \draw [thick] (-0.58,4.52) -- (-2.99,4.80);
    \draw [thick] (-6.60,6.88) -- (-4.26,3.19);
}
    Compléter avec les bons angles :
    \multiColEnumerate{2}{
        \item $\awsr{GCE} = \ang{150}$
        \item $\awsr{GCE} = \ang{55}$
        \item $\awsr{GCE} = \ang{90} $
        \item $\awsr{GCE} = \ang{260}$
    }
}
\bsec{Propriétés de parallélisme}

\slide{exo}{\bvspace{-0.5cm}
    \act{Eléments d'Euclide : proposition 27 et 28}{
        \begin{enumerate}
            \item \imgp{thm-18-pr-27}[10cm]
            \begin{enumerate}
                \item Réaliser un schéma de la situation qui illustre la proposition.
                \item Colorier de la même couleur les angles « oppofez alternatiuement » qui seraient égaux.
                \item Comment nommes-t-on ces angles aujourd'hui?
            \end{enumerate}\saveenumi
        \end{enumerate}
    }[\eucl[1632][58] et \href{https://mathix.org/linux/archives/19990}{Mathix}]
}

\slide{exo}{
    \begin{enumerate}\loadenumi[act]
        \item \imgp{thm-19-pr-28-alt}[10cm]
        \begin{enumerate}
            \item Réaliser un schéma de la situation qui illustre la proposition.
            \item Colorier de la même couleur « l'angle extérieur et son opposé intérieur du même côté » qui seraient égaux.
            \item Comment nommes-t-on ces angles aujourd'hui?
        \end{enumerate}
    \end{enumerate}
}

\scn{}
\slide{qf}{
    En ecrivant toutes les étapes;
    resoudre ce calcul :\\
    $39 + (7 - 18 + (-1)) = \awsr[3]{39 + (-11 + (-1)) = 39 + (-12) = 27}$
}
