\setSeq{6}{Géometrie dans l'espace - Solides}
\setGrade{5e}
\def\imgPath{enseignement/5e/geometrie-dans-l-espace/solides/}
\def\ym{https://www.maths-et-tiques.fr/telech/19Solides5e.pdf}

% \forPrint
\forStudents

\obj{
    \item Reconnaître des solides (pavé droit, cube, cylindre, prisme droit, pyramide, cône, boule) à partir
    d'un objet réel, d'une image, d'une représentation en perspective cavalière.
    \item Construire et mettre en relation une représentation en perspective cavalière et un patron d'un pavé droit,
    d'un cylindre.
}

\scn{Découvrir la perspective cavalière}

\slide{qf}{
    \exo{Comptage dans un solide}{\bshrink
    \dividePage{\ctikz[1]{
    \boundingBox[2.5][3.5][0.5pt][0.5][(0,0)][cavalier]
    \draw [thick] (0.00,1.50) -- (0.00,0.00) -- (1.50,0.00) -- (2.50,1.00) -- (2.50,2.50) -- (1.50,1.50) -- (0.00,2.50) -- (1.00,3.50) -- (2.50,2.50);
    \draw [thick,dashed] (1.00,2.50) -- (1.00,3.50);
    \draw [thick] (0.00,2.50) -- (0.00,1.50);
    \draw [thick,dashed] (1.00,2.50) -- (1.00,1.00) -- (0.00,0.00);
    \draw [thick,dashed] (1.00,1.00) -- (2.50,1.00);
    \draw [thick] (1.50,1.50) -- (1.50,0.00);
}}{
        \begin{enumerate}
            \item Combien d'arêtes comporte ce solide ?
            \item de faces ?
            \item de sommets ?
        \end{enumerate}
    }[0.3]
}

}

\slide{exo}{
    \exo{Empreintes de solides}{Associe à chaque volume son empreinte
    \ctikz[\ifBA{0.6}{1}]{
    \boundingBox[13][9][0.5pt][0.5][(0,0)][cavalier]
    \node at (2.37, 6.51) {\cir[gradeColor]{1}};
    \node at (6.42, 7.27) {\cir[gradeColor]{2}};
    \node at (9.52, 6.24) {\cir[gradeColor]{3}};
    \node at (0.93, 2.72) {\cir[gradeColor]{4}};
    \node at (5.40, 1.18) {\cir[gradeColor]{5}};
    \node at (10.42, 2.17) {\cir[gradeColor]{6}};
    \draw [thick] (1.00,0.50) -- (2.00,0.50) -- (2.50,1.00) -- (2.50,2.00) -- (2.00,1.50) -- (1.00,1.50) -- (0.00,1.50) -- (0.00,0.50) -- (1.00,0.50);
    \draw [thick] (2.00,0.50) -- (2.00,1.50);
    \draw [thick] (0.00,1.50) -- (0.50,2.00) -- (1.50,2.00) -- (2.00,2.50) -- (3.00,2.50) -- (4.00,2.50) -- (4.00,1.50) -- (3.50,1.00) -- (3.50,2.00) -- (2.50,2.00);
    \draw [thick] (2.00,2.50) -- (2.00,3.50) -- (1.50,3.00) -- (0.50,3.00) -- (1.00,3.50) -- (2.00,3.50);
    \draw [thick] (0.50,3.00) -- (0.50,2.00);
    \draw [thick] (1.50,3.00) -- (1.50,2.00);
    \draw [thick] (3.50,1.00) -- (2.50,1.00);
    \draw [thick] (4.00,2.50) -- (3.50,2.00);
    \draw [thick] (6.00,0.50) -- (7.00,0.50) -- (7.50,1.00) -- (7.50,2.00) -- (6.50,2.00) -- (7.00,2.50) -- (6.00,2.50) -- (5.50,2.00) -- (5.00,1.50) -- (6.00,1.50) -- (7.00,1.50) -- (7.50,2.00);
    \draw [thick] (5.00,1.50) -- (5.00,0.50) -- (6.00,0.50);
    \draw [thick] (7.00,2.50) -- (7.00,2.00);
    \draw [thick] (7.00,1.50) -- (7.00,0.50);
    \draw [thick] (10.00,2.50) -- (12.00,2.50) -- (13.00,3.50) -- (13.00,1.50) -- (12.00,0.50) -- (11.50,0.00) -- (11.50,1.00) -- (12.00,1.50) -- (11.00,1.50) -- (10.50,1.00) -- (11.50,1.00);
    \draw [thick] (10.00,2.50) -- (11.00,3.50) -- (13.00,3.50);
    \draw [thick] (10.00,0.50) -- (9.00,0.50) -- (9.00,1.50) -- (10.00,1.51) -- (10.00,2.50);
    \draw [thick] (9.00,1.50) -- (9.50,2.00) -- (10.00,2.01);
    \draw [thick] (11.50,0.00) -- (10.50,0.00) -- (10.50,1.00);
    \draw [thick] (10.00,0.50) -- (10.50,0.50);
    \draw [thick] (12.00,2.50) -- (12.00,1.50);
    \draw [thick] (6.00,5.50) -- (7.00,5.50) -- (7.50,6.00) -- (7.50,9.00) -- (7.00,8.50) -- (6.00,8.50) -- (6.00,5.50);
    \draw [thick] (7.00,5.50) -- (7.00,8.50);
    \draw [thick] (7.50,9.00) -- (6.50,9.00) -- (6.00,8.50);
    \draw [thick] (8.50,5.50) -- (10.50,5.50) -- (10.00,7.50) -- (8.50,5.50);
    \draw [thick] (11.01,6.00) -- (11.50,6.00) -- (12.00,6.50) -- (12.00,7.50) -- (11.50,7.50) -- (11.00,7.00) -- (11.01,6.00) -- (10.50,5.50);
    \draw [thick] (11.50,6.00) -- (11.50,7.00) -- (11.00,7.00);
    \draw [thick] (10.00,7.50) -- (11.00,6.83);
    \draw [thick] (1.00,5.50) -- (4.00,5.50) -- (5.00,6.50) -- (5.00,8.00) -- (4.00,7.00) -- (4.00,5.50);
    \draw [thick] (1.00,7.00) -- (1.00,5.50);
    \draw [thick] (4.00,7.00) -- (1.00,7.00) -- (1.51,7.50) -- (2.50,7.51) -- (3.00,7.51) -- (3.50,8.00) -- (3.00,9.00) -- (2.00,9.00) -- (1.50,8.50) -- (1.51,7.50);
    \draw [thick] (2.50,8.50) -- (1.50,8.50);
    \draw [thick] (3.00,9.00) -- (2.50,8.50) -- (3.00,7.51);
    \draw [thick] (3.50,8.00) -- (5.00,8.00);
    \draw [thick] (11.50,7.00) -- (12.00,7.50);
}
    \ctikz[\ifBA{0.5}{1}]{
    \boundingBox[11][6][0.5pt][0.5][(0,0)][sdot]
    \fill[thick,color=gradeColor,fill=gradeColor,fill opacity=0.05] (1.00,4.00) -- (1.00,6.00) -- (4.00,6.00) -- (4.00,5.00) -- (3.00,5.00) -- (3.00,4.00) -- cycle;
    \fill[thick,color=gradeColor,fill=gradeColor,fill opacity=0.05] (5.00,4.00) -- (5.00,6.00) -- (8.00,6.00) -- (8.00,4.00) -- cycle;
    \fill[thick,color=gradeColor,fill=gradeColor,fill opacity=0.05] (10.00,6.00) -- (10.00,5.00) -- (11.00,5.00) -- (11.00,6.00) -- cycle;
    \fill[thick,color=gradeColor,fill=gradeColor,fill opacity=0.05] (1.00,2.00) -- (1.00,3.00) -- (3.00,3.00) -- (3.00,0.00) -- (2.00,0.00) -- (2.00,1.00) -- (0.00,1.00) -- (0.00,2.00) -- cycle;
    \fill[thick,color=gradeColor,fill=gradeColor,fill opacity=0.05] (5.00,0.00) -- (5.00,2.00) -- (7.50,2.00) -- (7.50,1.00) -- (7.00,1.00) -- (7.00,0.00) -- cycle;
    \fill[thick,color=gradeColor,fill=gradeColor,fill opacity=0.05] (9.00,2.00) -- (9.00,0.00) -- (11.00,0.00) -- (11.00,1.00) -- (10.00,1.00) -- (10.00,2.00) -- cycle;
    \draw [thick,gradeColor] (1.00,4.00) -- (1.00,6.00) -- (4.00,6.00) -- (4.00,5.00) -- (3.00,5.00) -- (3.00,4.00) -- (1.00,4.00);
    \draw [thick,gradeColor] (5.00,4.00) -- (5.00,6.00) -- (8.00,6.00) -- (8.00,4.00) -- (5.00,4.00);
    \draw [thick,gradeColor] (10.00,6.00) -- (10.00,5.00) -- (11.00,5.00) -- (11.00,6.00) -- (10.00,6.00);
    \draw [thick,gradeColor] (1.00,2.00) -- (1.00,3.00) -- (3.00,3.00) -- (3.00,0.00) -- (2.00,0.00) -- (2.00,1.00) -- (0.00,1.00) -- (0.00,2.00) -- (1.00,2.00);
    \draw [thick,gradeColor] (5.00,0.00) -- (5.00,2.00) -- (7.50,2.00) -- (7.50,1.00) -- (7.00,1.00) -- (7.00,0.00) -- (5.00,0.00);
    \draw [thick,gradeColor] (9.00,2.00) -- (9.00,0.00) -- (11.00,0.00) -- (11.00,1.00) -- (10.00,1.00) -- (10.00,2.00) -- (9.00,2.00);
}
}

% \exo{Empreintes de solides}{Associe à chaque volume son empreinte
%     \ifBA{\vspace{-0.75cm}
%         \dividePage{\ctikz[\ifBA{0.6}{1}]{
    \boundingBox[13][9][0.5pt][0.5][(0,0)][cavalier]
    \node at (2.37, 6.51) {\cir[gradeColor]{1}};
    \node at (6.42, 7.27) {\cir[gradeColor]{2}};
    \node at (9.52, 6.24) {\cir[gradeColor]{3}};
    \node at (0.93, 2.72) {\cir[gradeColor]{4}};
    \node at (5.40, 1.18) {\cir[gradeColor]{5}};
    \node at (10.42, 2.17) {\cir[gradeColor]{6}};
    \draw [thick] (1.00,0.50) -- (2.00,0.50) -- (2.50,1.00) -- (2.50,2.00) -- (2.00,1.50) -- (1.00,1.50) -- (0.00,1.50) -- (0.00,0.50) -- (1.00,0.50);
    \draw [thick] (2.00,0.50) -- (2.00,1.50);
    \draw [thick] (0.00,1.50) -- (0.50,2.00) -- (1.50,2.00) -- (2.00,2.50) -- (3.00,2.50) -- (4.00,2.50) -- (4.00,1.50) -- (3.50,1.00) -- (3.50,2.00) -- (2.50,2.00);
    \draw [thick] (2.00,2.50) -- (2.00,3.50) -- (1.50,3.00) -- (0.50,3.00) -- (1.00,3.50) -- (2.00,3.50);
    \draw [thick] (0.50,3.00) -- (0.50,2.00);
    \draw [thick] (1.50,3.00) -- (1.50,2.00);
    \draw [thick] (3.50,1.00) -- (2.50,1.00);
    \draw [thick] (4.00,2.50) -- (3.50,2.00);
    \draw [thick] (6.00,0.50) -- (7.00,0.50) -- (7.50,1.00) -- (7.50,2.00) -- (6.50,2.00) -- (7.00,2.50) -- (6.00,2.50) -- (5.50,2.00) -- (5.00,1.50) -- (6.00,1.50) -- (7.00,1.50) -- (7.50,2.00);
    \draw [thick] (5.00,1.50) -- (5.00,0.50) -- (6.00,0.50);
    \draw [thick] (7.00,2.50) -- (7.00,2.00);
    \draw [thick] (7.00,1.50) -- (7.00,0.50);
    \draw [thick] (10.00,2.50) -- (12.00,2.50) -- (13.00,3.50) -- (13.00,1.50) -- (12.00,0.50) -- (11.50,0.00) -- (11.50,1.00) -- (12.00,1.50) -- (11.00,1.50) -- (10.50,1.00) -- (11.50,1.00);
    \draw [thick] (10.00,2.50) -- (11.00,3.50) -- (13.00,3.50);
    \draw [thick] (10.00,0.50) -- (9.00,0.50) -- (9.00,1.50) -- (10.00,1.51) -- (10.00,2.50);
    \draw [thick] (9.00,1.50) -- (9.50,2.00) -- (10.00,2.01);
    \draw [thick] (11.50,0.00) -- (10.50,0.00) -- (10.50,1.00);
    \draw [thick] (10.00,0.50) -- (10.50,0.50);
    \draw [thick] (12.00,2.50) -- (12.00,1.50);
    \draw [thick] (6.00,5.50) -- (7.00,5.50) -- (7.50,6.00) -- (7.50,9.00) -- (7.00,8.50) -- (6.00,8.50) -- (6.00,5.50);
    \draw [thick] (7.00,5.50) -- (7.00,8.50);
    \draw [thick] (7.50,9.00) -- (6.50,9.00) -- (6.00,8.50);
    \draw [thick] (8.50,5.50) -- (10.50,5.50) -- (10.00,7.50) -- (8.50,5.50);
    \draw [thick] (11.01,6.00) -- (11.50,6.00) -- (12.00,6.50) -- (12.00,7.50) -- (11.50,7.50) -- (11.00,7.00) -- (11.01,6.00) -- (10.50,5.50);
    \draw [thick] (11.50,6.00) -- (11.50,7.00) -- (11.00,7.00);
    \draw [thick] (10.00,7.50) -- (11.00,6.83);
    \draw [thick] (1.00,5.50) -- (4.00,5.50) -- (5.00,6.50) -- (5.00,8.00) -- (4.00,7.00) -- (4.00,5.50);
    \draw [thick] (1.00,7.00) -- (1.00,5.50);
    \draw [thick] (4.00,7.00) -- (1.00,7.00) -- (1.51,7.50) -- (2.50,7.51) -- (3.00,7.51) -- (3.50,8.00) -- (3.00,9.00) -- (2.00,9.00) -- (1.50,8.50) -- (1.51,7.50);
    \draw [thick] (2.50,8.50) -- (1.50,8.50);
    \draw [thick] (3.00,9.00) -- (2.50,8.50) -- (3.00,7.51);
    \draw [thick] (3.50,8.00) -- (5.00,8.00);
    \draw [thick] (11.50,7.00) -- (12.00,7.50);
}}
%         {\ctikz[\ifBA{0.5}{1}]{
    \boundingBox[11][6][0.5pt][0.5][(0,0)][sdot]
    \fill[thick,color=gradeColor,fill=gradeColor,fill opacity=0.05] (1.00,4.00) -- (1.00,6.00) -- (4.00,6.00) -- (4.00,5.00) -- (3.00,5.00) -- (3.00,4.00) -- cycle;
    \fill[thick,color=gradeColor,fill=gradeColor,fill opacity=0.05] (5.00,4.00) -- (5.00,6.00) -- (8.00,6.00) -- (8.00,4.00) -- cycle;
    \fill[thick,color=gradeColor,fill=gradeColor,fill opacity=0.05] (10.00,6.00) -- (10.00,5.00) -- (11.00,5.00) -- (11.00,6.00) -- cycle;
    \fill[thick,color=gradeColor,fill=gradeColor,fill opacity=0.05] (1.00,2.00) -- (1.00,3.00) -- (3.00,3.00) -- (3.00,0.00) -- (2.00,0.00) -- (2.00,1.00) -- (0.00,1.00) -- (0.00,2.00) -- cycle;
    \fill[thick,color=gradeColor,fill=gradeColor,fill opacity=0.05] (5.00,0.00) -- (5.00,2.00) -- (7.50,2.00) -- (7.50,1.00) -- (7.00,1.00) -- (7.00,0.00) -- cycle;
    \fill[thick,color=gradeColor,fill=gradeColor,fill opacity=0.05] (9.00,2.00) -- (9.00,0.00) -- (11.00,0.00) -- (11.00,1.00) -- (10.00,1.00) -- (10.00,2.00) -- cycle;
    \draw [thick,gradeColor] (1.00,4.00) -- (1.00,6.00) -- (4.00,6.00) -- (4.00,5.00) -- (3.00,5.00) -- (3.00,4.00) -- (1.00,4.00);
    \draw [thick,gradeColor] (5.00,4.00) -- (5.00,6.00) -- (8.00,6.00) -- (8.00,4.00) -- (5.00,4.00);
    \draw [thick,gradeColor] (10.00,6.00) -- (10.00,5.00) -- (11.00,5.00) -- (11.00,6.00) -- (10.00,6.00);
    \draw [thick,gradeColor] (1.00,2.00) -- (1.00,3.00) -- (3.00,3.00) -- (3.00,0.00) -- (2.00,0.00) -- (2.00,1.00) -- (0.00,1.00) -- (0.00,2.00) -- (1.00,2.00);
    \draw [thick,gradeColor] (5.00,0.00) -- (5.00,2.00) -- (7.50,2.00) -- (7.50,1.00) -- (7.00,1.00) -- (7.00,0.00) -- (5.00,0.00);
    \draw [thick,gradeColor] (9.00,2.00) -- (9.00,0.00) -- (11.00,0.00) -- (11.00,1.00) -- (10.00,1.00) -- (10.00,2.00) -- (9.00,2.00);
}}
%     }{
%         \ctikz[\ifBA{0.6}{1}]{
    \boundingBox[13][9][0.5pt][0.5][(0,0)][cavalier]
    \node at (2.37, 6.51) {\cir[gradeColor]{1}};
    \node at (6.42, 7.27) {\cir[gradeColor]{2}};
    \node at (9.52, 6.24) {\cir[gradeColor]{3}};
    \node at (0.93, 2.72) {\cir[gradeColor]{4}};
    \node at (5.40, 1.18) {\cir[gradeColor]{5}};
    \node at (10.42, 2.17) {\cir[gradeColor]{6}};
    \draw [thick] (1.00,0.50) -- (2.00,0.50) -- (2.50,1.00) -- (2.50,2.00) -- (2.00,1.50) -- (1.00,1.50) -- (0.00,1.50) -- (0.00,0.50) -- (1.00,0.50);
    \draw [thick] (2.00,0.50) -- (2.00,1.50);
    \draw [thick] (0.00,1.50) -- (0.50,2.00) -- (1.50,2.00) -- (2.00,2.50) -- (3.00,2.50) -- (4.00,2.50) -- (4.00,1.50) -- (3.50,1.00) -- (3.50,2.00) -- (2.50,2.00);
    \draw [thick] (2.00,2.50) -- (2.00,3.50) -- (1.50,3.00) -- (0.50,3.00) -- (1.00,3.50) -- (2.00,3.50);
    \draw [thick] (0.50,3.00) -- (0.50,2.00);
    \draw [thick] (1.50,3.00) -- (1.50,2.00);
    \draw [thick] (3.50,1.00) -- (2.50,1.00);
    \draw [thick] (4.00,2.50) -- (3.50,2.00);
    \draw [thick] (6.00,0.50) -- (7.00,0.50) -- (7.50,1.00) -- (7.50,2.00) -- (6.50,2.00) -- (7.00,2.50) -- (6.00,2.50) -- (5.50,2.00) -- (5.00,1.50) -- (6.00,1.50) -- (7.00,1.50) -- (7.50,2.00);
    \draw [thick] (5.00,1.50) -- (5.00,0.50) -- (6.00,0.50);
    \draw [thick] (7.00,2.50) -- (7.00,2.00);
    \draw [thick] (7.00,1.50) -- (7.00,0.50);
    \draw [thick] (10.00,2.50) -- (12.00,2.50) -- (13.00,3.50) -- (13.00,1.50) -- (12.00,0.50) -- (11.50,0.00) -- (11.50,1.00) -- (12.00,1.50) -- (11.00,1.50) -- (10.50,1.00) -- (11.50,1.00);
    \draw [thick] (10.00,2.50) -- (11.00,3.50) -- (13.00,3.50);
    \draw [thick] (10.00,0.50) -- (9.00,0.50) -- (9.00,1.50) -- (10.00,1.51) -- (10.00,2.50);
    \draw [thick] (9.00,1.50) -- (9.50,2.00) -- (10.00,2.01);
    \draw [thick] (11.50,0.00) -- (10.50,0.00) -- (10.50,1.00);
    \draw [thick] (10.00,0.50) -- (10.50,0.50);
    \draw [thick] (12.00,2.50) -- (12.00,1.50);
    \draw [thick] (6.00,5.50) -- (7.00,5.50) -- (7.50,6.00) -- (7.50,9.00) -- (7.00,8.50) -- (6.00,8.50) -- (6.00,5.50);
    \draw [thick] (7.00,5.50) -- (7.00,8.50);
    \draw [thick] (7.50,9.00) -- (6.50,9.00) -- (6.00,8.50);
    \draw [thick] (8.50,5.50) -- (10.50,5.50) -- (10.00,7.50) -- (8.50,5.50);
    \draw [thick] (11.01,6.00) -- (11.50,6.00) -- (12.00,6.50) -- (12.00,7.50) -- (11.50,7.50) -- (11.00,7.00) -- (11.01,6.00) -- (10.50,5.50);
    \draw [thick] (11.50,6.00) -- (11.50,7.00) -- (11.00,7.00);
    \draw [thick] (10.00,7.50) -- (11.00,6.83);
    \draw [thick] (1.00,5.50) -- (4.00,5.50) -- (5.00,6.50) -- (5.00,8.00) -- (4.00,7.00) -- (4.00,5.50);
    \draw [thick] (1.00,7.00) -- (1.00,5.50);
    \draw [thick] (4.00,7.00) -- (1.00,7.00) -- (1.51,7.50) -- (2.50,7.51) -- (3.00,7.51) -- (3.50,8.00) -- (3.00,9.00) -- (2.00,9.00) -- (1.50,8.50) -- (1.51,7.50);
    \draw [thick] (2.50,8.50) -- (1.50,8.50);
    \draw [thick] (3.00,9.00) -- (2.50,8.50) -- (3.00,7.51);
    \draw [thick] (3.50,8.00) -- (5.00,8.00);
    \draw [thick] (11.50,7.00) -- (12.00,7.50);
}
%         \ctikz[\ifBA{0.5}{1}]{
    \boundingBox[11][6][0.5pt][0.5][(0,0)][sdot]
    \fill[thick,color=gradeColor,fill=gradeColor,fill opacity=0.05] (1.00,4.00) -- (1.00,6.00) -- (4.00,6.00) -- (4.00,5.00) -- (3.00,5.00) -- (3.00,4.00) -- cycle;
    \fill[thick,color=gradeColor,fill=gradeColor,fill opacity=0.05] (5.00,4.00) -- (5.00,6.00) -- (8.00,6.00) -- (8.00,4.00) -- cycle;
    \fill[thick,color=gradeColor,fill=gradeColor,fill opacity=0.05] (10.00,6.00) -- (10.00,5.00) -- (11.00,5.00) -- (11.00,6.00) -- cycle;
    \fill[thick,color=gradeColor,fill=gradeColor,fill opacity=0.05] (1.00,2.00) -- (1.00,3.00) -- (3.00,3.00) -- (3.00,0.00) -- (2.00,0.00) -- (2.00,1.00) -- (0.00,1.00) -- (0.00,2.00) -- cycle;
    \fill[thick,color=gradeColor,fill=gradeColor,fill opacity=0.05] (5.00,0.00) -- (5.00,2.00) -- (7.50,2.00) -- (7.50,1.00) -- (7.00,1.00) -- (7.00,0.00) -- cycle;
    \fill[thick,color=gradeColor,fill=gradeColor,fill opacity=0.05] (9.00,2.00) -- (9.00,0.00) -- (11.00,0.00) -- (11.00,1.00) -- (10.00,1.00) -- (10.00,2.00) -- cycle;
    \draw [thick,gradeColor] (1.00,4.00) -- (1.00,6.00) -- (4.00,6.00) -- (4.00,5.00) -- (3.00,5.00) -- (3.00,4.00) -- (1.00,4.00);
    \draw [thick,gradeColor] (5.00,4.00) -- (5.00,6.00) -- (8.00,6.00) -- (8.00,4.00) -- (5.00,4.00);
    \draw [thick,gradeColor] (10.00,6.00) -- (10.00,5.00) -- (11.00,5.00) -- (11.00,6.00) -- (10.00,6.00);
    \draw [thick,gradeColor] (1.00,2.00) -- (1.00,3.00) -- (3.00,3.00) -- (3.00,0.00) -- (2.00,0.00) -- (2.00,1.00) -- (0.00,1.00) -- (0.00,2.00) -- (1.00,2.00);
    \draw [thick,gradeColor] (5.00,0.00) -- (5.00,2.00) -- (7.50,2.00) -- (7.50,1.00) -- (7.00,1.00) -- (7.00,0.00) -- (5.00,0.00);
    \draw [thick,gradeColor] (9.00,2.00) -- (9.00,0.00) -- (11.00,0.00) -- (11.00,1.00) -- (10.00,1.00) -- (10.00,2.00) -- (9.00,2.00);
}
%     }
% }
}

\def\figScale{\ifBA{0.75}{0.85}}

\scn{Construction perspective cavalière}

\def\caPrefix{5e-mars-2023-}
\caSlide{27-28-29}

\slide{exo}{
    \newcommand{\se}[1]{
    \item \dividePage{
        \input{resources/enseignement/5e/geometrie-dans-l-espace/solides/empreintes-manquantes/solide-#1.tex}
    }{
        \input{resources/enseignement/5e/geometrie-dans-l-espace/solides/empreintes-manquantes/empreinte-#1.tex}
    }
}

\def\figScale{0.65}

\exo{Construction de polyèdres en perspective cavalière}{
    Compléter les empreintes ou solides manquants :

    \dividePage{Solides :}{Empreintes :}
    \multiColEnumerate{1}{
        \se{1} \se{2} \se{3} \se{4}
    }
}

% \slide{exo}{\bshrink
%     % \newpage
%     \exo{Construction de polyèdres en perspective cavalière}{Compléter les empreintes ou solides manquants
%     \multiColEnumerate{1}{
%         \item \dividePage{
%             Solides :
%             \ctikz[\figScale]{
    \boundingBox[5][5][0.5pt][0.5][(0.5,0)][cavalier]
    \draw [thick] (1.00,2.00) -- (1.00,1.00) -- (2.00,1.00) -- (2.50,1.50) -- (2.50,2.50) -- (2.00,2.00) -- (2.00,1.00);
    \draw [thick] (1.00,2.00) -- (2.00,2.00);
    \draw [thick] (2.50,1.50) -- (4.50,1.50) -- (5.00,2.00) -- (5.00,3.00) -- (4.50,2.50) -- (3.50,2.50) -- (4.00,3.00) -- (5.00,3.00);
    \draw [thick] (4.00,3.00) -- (4.00,4.00) -- (3.00,4.00) -- (2.50,3.50) -- (2.50,2.50);
    \draw [thick] (4.00,4.00) -- (3.50,3.50) -- (3.50,2.50);
    \draw [thick] (3.50,3.50) -- (2.50,3.50);
    \draw [thick] (4.50,2.50) -- (4.50,1.50);
    \draw [thick] (2.00,3.00) -- (2.50,3.01);
    \draw [thick] (2.00,3.00) -- (1.00,2.00);
}
%         }{
%             Empreintes :
%             \ctikz[\figScale]{
    \boundingBox[5][5][0.5pt][0.5][(-1,-1)][sdot]
    \nswr{
        % \draw [thick,gradeColor] (0.00,5.00) -- (6.00,5.00) -- (6.00,0.00) -- (0.00,0.00) -- (0.00,5.00);
        \fill[thick,fill opacity=0.10] (0.00,0.00) -- (0.00,2.00) -- (3.00,2.00) -- (3.00,1.00) -- (1.00,1.00) -- (1.00,0.00) -- cycle;
        \draw [thick] (0.00,0.00) -- (0.00,2.00) -- (3.00,2.00) -- (3.00,1.00) -- (1.00,1.00) -- (1.00,0.00) -- (0.00,0.00);
    }
}

%         }\saveenumi
%         }
%     }
% }

% \slide{exo}{
%     \multiColEnumerate{1}{\loadenumi[exo]
%         \item \dividePage{\ctikz[\figScale]{
    \boundingBox[5][5][0.5pt][0.5][(0,-0.5)][cavalier]
    \nswr{
        \draw [thick] (1.00,1.00) -- (1.00,2.00) -- (1.50,2.50) -- (2.50,2.50) -- (3.00,3.00) -- (4.00,3.00) -- (3.50,2.50) -- (4.50,2.50) -- (4.50,1.50) -- (4.00,1.00) -- (4.00,2.00) -- (4.50,2.50);
        \draw [thick] (2.00,2.00) -- (1.00,2.00);
        \draw [thick] (4.00,2.00) -- (2.00,2.00);
        \draw [thick] (1.00,1.00) -- (4.00,1.00);
        \draw [thick] (4.00,3.00) -- (4.00,2.50);
    }
}}{
%             \ctikz[\figScale]{
    \boundingBox[7.789999999999999][6][\dotSize][1][(0,0)][iso]
    \draw [thick] (1.73,1.50) -- (3.46,0.50) -- (5.19,1.50) -- (3.46,2.50) -- (1.73,1.50);
}
%         }\saveenumi
%     }
% }


% \slide{exo}{
%     \multiColEnumerate{1}{\loadenumi[exo]
%         \item \dividePage{\ctikz[\figScale]{
    \boundingBox[5][5][0.5pt][0.5][(0,0)][cavalier]
    \draw [thick] (2.00,3.50) -- (1.50,3.00) -- (3.00,3.50) -- (3.00,2.50) -- (2.50,2.00) -- (4.00,2.50) -- (3.00,2.50);
    \draw [thick] (4.00,3.50) -- (4.00,1.50) -- (4.50,2.00) -- (4.50,4.00) -- (2.50,4.00) -- (2.00,3.50);
    \draw [thick] (4.00,3.50) -- (4.50,4.00);
    \draw [thick] (4.00,1.50) -- (2.50,1.00) -- (2.50,2.00);
    \draw [thick] (1.50,2.00) -- (1.50,1.00) -- (2.50,1.33);
    \draw [thick] (1.50,2.00) -- (1.50,3.00);
    \draw [thick] (3.00,3.50) -- (4.00,3.50);
}}{
%             \ctikz[\figScale]{
    \boundingBox[5][5][0.5pt][0.5][(-1,-1)][sdot]
    \nswr{
        \fill[thick,fill opacity=0.10] (1.00,2.00) -- (3.00,2.00) -- (3.00,1.00) -- (2.00,1.00) -- (1.00,0.00) -- (1.00,1.00) -- (0.00,0.00) -- (0.00,2.00) -- cycle;
        \draw [thick] (1.00,2.00) -- (3.00,2.00) -- (3.00,1.00) -- (2.00,1.00) -- (1.00,0.00) -- (1.00,1.00) -- (0.00,0.00) -- (0.00,2.00) -- (1.00,2.00);
    }
}
%         }\saveenumi
%     }
% }

% \slide{exo}{
%     \multiColEnumerate{1}{\loadenumi[exo]
%         \item \dividePage{\ctikz[\figScale]{
    \boundingBox[5][5][0.5pt][0.5][(-1.5,-1)][cavalier]
    \nswr{
        % \draw [thick] (1.50,1.00) -- (3.50,1.00) -- (4.50,2.50) -- (4.50,4.00) -- (3.50,2.50) -- (1.50,2.50) -- (4.50,4.00);
        % \draw [thick] (3.50,2.50) -- (3.50,1.00);
        % \draw [thick] (1.50,2.50) -- (1.50,1.00);
        \draw [thick] (0.00,0.00) -- (2.00,0.00) -- (2.75,0.75) -- (2.75,2.75) -- (2.00,2.00) -- (0.00,2.00) -- (2.75,2.75);
        \draw [thick] (0.00,2.00) -- (0.00,0.00);
        \draw [thick] (2.00,2.00) -- (2.00,0.00);
    }
}}{
%             \ctikz[\figScale]{
    \boundingBox[5][5][0.5pt][0.5][(-1,-1)][sdot]
    \fill[thick,color=gradeColor,fill=gradeColor,fill opacity=0.10] (0.00,0.00) -- (2.00,0.00) -- (2.00,1.50) -- cycle;
    \draw [thick,gradeColor] (0.00,0.00) -- (2.00,0.00) -- (2.00,1.50) -- (0.00,0.00);
}
%         }
%     }
% }

}

\scn{Définir différents des polyedres}

\slide{qf}{\bshrink
    \exo{Compléter un patron}{
    \ctikz[1]{
    \boundingBox[4][2][0.5pt][0.5][(0,0)][cavalier]
    \draw[color=gradeColor] (3.73,0.61) node {4cm};
    \draw[color=gradeColor] (2.64,0.72) node {2cm};
    \draw[color=gradeColor] (1.61,0.00) node {6cm};
    \draw [thick] (1.00,2.14) -- (4.00,2.14) -- (4.00,1.14) -- (3.00,0.14) -- (3.00,1.14) -- (4.00,2.14);
    \draw [thick] (3.00,1.14) -- (0.00,1.14) -- (1.00,2.14);
    \draw [thick] (0.00,1.14) -- (0.00,0.14) -- (3.00,0.14);
}

    Recopiez à main levée ce patron de pavé droit, puis complétez les longueurs et ajoutez les codages manquants.
    \ctikz[\ifBA{0.3}{0.5}]{
    \boundingBox[16.8][8][0.5pt][0.5][(0,0)][dot]
    \draw[color=gradeColor] (11.39,5.56) node {... cm};
    \draw[color=gradeColor] (7.21,7.16) node {... cm};
    \draw[color=gradeColor] (1.08,4.25) node {...cm};
    \draw [thick,gradeColor] (8.03,2.00) -- (8.00,6.00) -- (14.00,6.04) -- (14.03,2.04) -- (8.03,2.00) -- (6.03,1.99) -- (6.00,5.99) -- (8.00,6.00) -- (7.99,8.00) -- (13.99,8.04) -- (14.00,6.04) -- (16.00,6.05) -- (16.03,2.05) -- (14.03,2.04) -- (14.04,0.04) -- (8.04,0.00) -- (8.03,2.00);
    \draw [thick,gradeColor] (6.00,5.99) -- (0.00,5.95) -- (0.03,1.95) -- (6.03,1.99);
}
}[][\cmdGeoGebra[hrn3vruh]]
}

\def\figScale{0.75}
\slide{cr}{
    \sseq%
    \section{Polyèdres}
    \df{}{
    On appelle \key{solide} l'ensemble des points situés à l'intérieur d'une région fermée de l'espace.
}[\wiki{Solide_géométrique}]
}

\slide{cr}{
    \newcommand{\sld}[2]{
    \item \dividePage{#1}{\input{resources/enseignement/5e/geometrie-dans-l-espace/solides/tikz/#2.tex}}
}

\def\figScale{1}

\df{Polyèdres}{On appelle :%
    \multiColEnumerate{1}{%
        \sld{%
            \key{polyèdre}, un solide constitué de \key{faces} polygonales.
        }{polyedre}
        \sld{%
            \nswr{\key{prisme droit}}[5cm], un polyèdre constitué de deux \key{bases} polygonales superposables,
            reliées entre elles par des faces \nswr{\key{rectangulaires}}.
        }{prisme-droit}
        \sld{%
            \nswr{\key{pavé droit}}[5cm], un \nswr{prisme droit}[5cm] dont les bases sont \nswr{\key{rectangulaires}}.
        }{pave-droit}
        \sld{%
            \nswr{\key{cube}}[5cm], un \nswr{pavé droit}[5cm] dont toutes les faces sont des \nswr{\key{carrés}}[5cm].
        }{cube}
        \sld{%
            \nswr{\key{pyramide}}[5cm], un polyèdre formé d'une base polygonale reliée à un \key{sommet} par des faces \nswr{\key{triangulaires}}[5cm].
        }{pyramide}
    }
}[\wiki{Polyèdre}[Polyèdres_simples]]

% \df{Polyèdres}{On appelle :
% \multiColEnumerate{1}{ 
%     \item \dividePage{
%         \key{polyèdre} un solide constitué de \key{faces} polygonales.
%     }{\ctikz{
    \boundingBox[7.789999999999999][6][0.5pt][1][(0,0)][iso]
    \draw [thick] (1.73,1.50) -- (3.46,0.50) -- (5.19,0.50) -- (6.93,1.50) -- (5.19,3.50) -- (2.60,4.00) -- (1.73,1.50);
    \draw [thick,dashed] (6.93,1.50) -- (5.19,2.50);
    \draw [thick,dashed] (1.73,1.50) -- (5.19,2.50);
    \draw [thick,dashed] (5.19,2.50) -- (5.19,3.50);
    \draw [thick] (2.60,4.00) -- (3.46,3.50) -- (5.19,3.50);
    \draw [thick] (3.46,3.50) -- (5.19,0.50);
    \draw [thick] (3.46,3.50) -- (3.46,0.50);
}}
%     \item \dividePage{
%         \nswr{\key{prisme droit}}[5cm], un polyèdre constitué de deux \key{bases} polygonales superposables,
%         reliées entre elles par des faces \nswr{\key{rectangulaires}}.
%     }{\ctikz[\figScale]{
    \boundingBox[2.2][2.8][0.5pt][0.5][(0,0)][cavalier]
    \draw [thick] (0.00,2.00) -- (0.00,0.40) -- (1.40,0.00) -- (1.60,0.60) -- (2.20,1.20);
    \draw [thick,dashed] (2.20,1.20) -- (1.00,1.20) -- (0.00,0.40);
    \draw [thick,dashed] (1.00,1.20) -- (1.00,2.80);
    \draw [thick] (0.00,2.00) -- (1.00,2.80) -- (2.20,2.80) -- (1.60,2.20) -- (1.40,1.60) -- (0.00,2.00);
    \draw [thick] (1.40,1.60) -- (1.40,0.00);
    \draw [thick] (2.20,2.80) -- (2.20,1.20);
    \draw [thick] (1.60,0.60) -- (1.60,2.20);
}}\saveenumi
%     \ifArticle{
%         \item \dividePage{
%             \nswr{\key{pavé droit}}[5cm] un \nswr{prisme droit}[5cm] dont les bases sont \nswr{\key{rectangulaires}}.
%         }{\ctikz[\figScale]{
    \boundingBox[4.5][2.5][0.5pt][0.5][(0,0)][cavalier]
    \draw [thick] (0.00,2.00) -- (0.50,2.50) -- (4.50,2.50) -- (4.00,2.00) -- (0.00,2.00) -- (0.00,0.00) -- (4.00,0.00) -- (4.00,2.00);
    \draw [thick,dashed] (0.00,0.00) -- (0.50,0.50) -- (4.50,0.50);
    \draw [thick,dashed] (0.50,0.50) -- (0.50,2.50);
    \draw [thick] (4.00,0.00) -- (4.50,0.50) -- (4.50,2.50);
}}
%         \item \dividePage{
%             \nswr{\key{cube}}[5cm] un \nswr{pavé droit}[5cm] dont les toutes les faces sont des \nswr{\key{carrés}}[5cm].
%         }{\ctikz[\figScale]{
    \boundingBox[3][3][0.5pt][0.5][(0,0)][cavalier]
    \draw [thick] (0.00,2.00) -- (1.00,3.00) -- (3.00,3.00) -- (2.00,2.00) -- (0.00,2.00) -- (0.00,0.00) -- (2.00,0.00) -- (3.00,1.00) -- (3.00,3.00);
    \draw [thick,dashed] (0.00,0.00) -- (1.00,1.00) -- (3.00,1.00);
    \draw [thick,dashed] (1.00,1.00) -- (1.00,3.00);
    \draw [thick] (2.00,2.00) -- (2.00,0.00);
}}
%         \item \dividePage{
%             \nswr{\key{pyramide}}[5cm] un polyèdre formé d'une base polygoneale relié à un \key{sommet} par des faces \nswr{\key{triangulaires}}[5cm].
%         }{\ctikz[\figScale]{
    \boundingBox[2.5][1.5][0.5pt][0.5][(0,0)][cavalier]
    \draw [thick,dashed] (0.50,0.50) -- (0.50,1.50) -- (1.30,0.80) -- (2.50,0.50);
    \draw [thick] (0.50,1.50) -- (2.00,0.00) -- (2.50,0.50) -- (0.50,1.50) -- (0.00,0.00) -- (2.00,0.00);
    \draw [thick,dashed] (0.00,0.00) -- (0.50,0.50) -- (1.30,0.80);
}}
%     }
% }
% }[\wiki{Polyèdre}[Polyèdres_simples]]
}

\scn{Construire le patron d'un cylindre}

\slide{qf}{
    \exo{Calculer des aires}{
    Quels sont les aires des deux figures ci dessous ?
    \ctikz[1]{
    \boundingBox[12][4][0.5pt][0.5][(0,0)]
    \draw[thick,color=gradeColor,fill=gradeColor,fill opacity=0.10] (2.37,0.39) -- (2.39,0.81) -- (1.97,0.84) -- (1.95,0.41) -- cycle;
    \draw [thick] (9.80,2.03) circle (2.27cm);
    \node at (3.96, 1.37) {\cir[gradeColor]{1}};
    \node at (9.92, 3.71) {\cir[gradeColor]{2}};
    \draw[color=gradeColor] (2.52,2.02) node {4cm};
    \draw[color=gradeColor] (0.76,0.20) node {2cm};
    \draw[color=gradeColor] (4.18,0.00) node {8cm};
    \drawPoint{O}{9.80}{2.03}
    \draw[color=gradeColor] (11.00,1.94) node {2dm};
    \draw [thick] (6.78,0.17) -- (2.10,3.49) -- (0.00,0.51) -- (1.95,0.41) -- (2.10,3.49);
    \draw [thick] (1.95,0.41) -- (6.78,0.17);
    \draw [thick] (9.80,2.03) -- (11.62,0.67);
    \draw [thick] (10.78,1.45) -- (10.64,1.25);
    \draw [thick] (9.80,2.03) -- (7.98,3.39);
    \draw [thick] (8.82,2.61) -- (8.96,2.81);
}
    \hint{On prendra $\pi\approx3$}
}
}

\slide{exo}{
    \act{Patron d'un cylindre}{
    Voici deux représentations en perspective cavalière d'un même cylindre :
    \ctikz[\ifBA{1.5}{3}]{
    \boundingBox[2.6][1.35][0.5pt][0.5][(0,0)]
    \draw [rotate around={-0.12:(0.50,0.25)},thick,dashed] (0.50,0.25) ellipse (0.50cm and 0.29cm);
    \draw [rotate around={0:(0.50,1.25)},thick] (0.50,1.25) ellipse (0.50cm and 0.29cm);
    \draw [thick] (1.75,0.50) circle (0.50cm);
    \draw [shift={(2.25,1.00)},thick]  plot[domain=-0.79:2.36,variable=\t]({1*0.50*cos(\t r)+0*0.50*sin(\t r)},{0*0.50*cos(\t r)+1*0.50*sin(\t r)});
    \draw [shift={(2.25,1.00)},thick,dashed]  plot[domain=2.36:5.50,variable=\t]({1*0.50*cos(\t r)+0*0.50*sin(\t r)},{0*0.50*cos(\t r)+1*0.50*sin(\t r)});
    \draw [rotate around={0:(0.50,0.25)},thick]  plot[domain=3.14:6.28,variable=\t]({0.50 + 0.50*cos(\t r)}, {0.25 + 0.29*sin(\t r)});
    \draw[color=gradeColor] (1.09,0.79) node {6cm};
    \draw[color=gradeColor] (2.43,0.37) node {6cm};
    \draw[color=gradeColor] (1.61,0.29) node {3cm};
    \draw[color=gradeColor] (0.74,1.35) node {3cm};
    \draw [thick] (0.00,1.25) -- (0.00,0.25);
    \draw [thick] (1.00,1.25) -- (1.00,0.25);
    \draw [thick] (1.40,0.85) -- (1.90,1.35);
    \draw [thick] (2.10,0.15) -- (2.60,0.65);
    \draw [thick] (1.75,0.50) -- (1.75,0.00);
    \draw [thick] (0.50,1.25) -- (1.00,1.25);
}
    \bvspace{-0.5cm}\begin{enumerate}
        \item Combien de figures géométriques composent ce cylindre ?
        \item Quelle est la nature de chacune de ces figures ?
        \item Dessine à main levée le patron de ce cylindre en ajoutant les mesures.
        \item Construis le patron en respectant les proportions.
        \item Découpe le patron pour vérifier sa cohérence.
    \end{enumerate}
}

% \slide{exo}{\bshrink
%     \act{Patron d'un cylindre}{
%         Voici deux représentations en perspective cavalière d'un même cylindre :
%         \ctikz[\ifBA{1.5}{3}]{
    \boundingBox[2.6][1.35][0.5pt][0.5][(0,0)]
    \draw [rotate around={-0.12:(0.50,0.25)},thick,dashed] (0.50,0.25) ellipse (0.50cm and 0.29cm);
    \draw [rotate around={0:(0.50,1.25)},thick] (0.50,1.25) ellipse (0.50cm and 0.29cm);
    \draw [thick] (1.75,0.50) circle (0.50cm);
    \draw [shift={(2.25,1.00)},thick]  plot[domain=-0.79:2.36,variable=\t]({1*0.50*cos(\t r)+0*0.50*sin(\t r)},{0*0.50*cos(\t r)+1*0.50*sin(\t r)});
    \draw [shift={(2.25,1.00)},thick,dashed]  plot[domain=2.36:5.50,variable=\t]({1*0.50*cos(\t r)+0*0.50*sin(\t r)},{0*0.50*cos(\t r)+1*0.50*sin(\t r)});
    \draw [rotate around={0:(0.50,0.25)},thick]  plot[domain=3.14:6.28,variable=\t]({0.50 + 0.50*cos(\t r)}, {0.25 + 0.29*sin(\t r)});
    \draw[color=gradeColor] (1.09,0.79) node {6cm};
    \draw[color=gradeColor] (2.43,0.37) node {6cm};
    \draw[color=gradeColor] (1.61,0.29) node {3cm};
    \draw[color=gradeColor] (0.74,1.35) node {3cm};
    \draw [thick] (0.00,1.25) -- (0.00,0.25);
    \draw [thick] (1.00,1.25) -- (1.00,0.25);
    \draw [thick] (1.40,0.85) -- (1.90,1.35);
    \draw [thick] (2.10,0.15) -- (2.60,0.65);
    \draw [thick] (1.75,0.50) -- (1.75,0.00);
    \draw [thick] (0.50,1.25) -- (1.00,1.25);
}
%         \bvspace{-0.5cm}\begin{enumerate}
%             \item Combien de figures géométriques composent ce cylindre ?
%             \item Quelle est la nature de chacune de ces figures ? \saveenumi
%         \end{enumerate}
%     }
% }

% \slide{exo}{
%     \begin{enumerate}\loadenumi[act]
%         \item Dessine à main levée le patron de ce cylindre en ajoutant les mesures.
%         \item Construis le patron en respectant les proportions.
%         \item Découpe le patron pour vérifier sa cohérence.
%     \end{enumerate}
% }
}

\scn{Définir différents des solides de révolution}
\caSlide{13-14-15}

\slide{cr}{
    \section{Solides de révolution}
    \def\figScale{2.25}
% \newcommand{\solide}[2]{\item \dividePage{#1}{\input{resources/enseignement/5e/geometrie-dans-l-espace/solides/tikz/#2.tex}}}%

\newcommand{\solide}[2]{\item #1
    \input{resources/enseignement/5e/geometrie-dans-l-espace/solides/tikz/#2.tex}
}

\df{Solides de révolution}{%
    On appelle :
    \multiColItemize{1}{
        \item \key{solide de révolution}, un solide obtenu en faisant tourner une figure plane autour d'un axe.
        \solide{\nswr{\key{cylindre de révolution}}[5cm], un solide de révolution dont les bases sont des cercles superposables}{cylindre}        
        \solide{\nswr{\key{cône}}[5cm], un solide de révolution obtenu par rotation d'un triangle rectangle autour d'un côté adjacent à l'angle droit.}{cone}
        \solide{\nswr{\key{sphère}}[5cm], une surface constituée de tous les points situés à une même distance d'un point appelé \key{centre}.}{sphere}
    }
}[\wiki{Solide_de_révolution}][\cmdGeogebra[agzsnw5r]]
}
