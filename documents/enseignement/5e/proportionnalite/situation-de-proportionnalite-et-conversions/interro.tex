% VARIABLES %%%
\setTitle{Interrogation - Séquence 2}
\setGrade{5e}
\thispagestyle{assignment}
%%

% \hint{Calculatrice autorisée}
\calculator

\def\colWidth{1.5cm}
\exo{Les tableaux ci-dessous représentent-ils une situation de proportionnalité ?}{\vspace{-0.5cm}%
    \multiColEnumerate{2}{
        % \item \propTable{5}{8}{7,5}{12}
        \item \begin{tabular}{|C{\colWidth}|C{\colWidth}|C{\colWidth}|}
            \hline
            $36$ & $13,95$ & $40,5$\\
            \hline
            $8$ & $3,1$ & $9$\\
            \hline
        \end{tabular}
        \item \begin{tabular}{|C{\colWidth}|C{\colWidth}|}
            \hline
            $651,3$ & $128$\\
            \hline
            $100,2$ & $20$\\
            \hline
        \end{tabular}
    }
}

\corr{}{
    \begin{enumerate}
        \item \begin{align*}
            \frac{36}{8} = 4,5\qquad
            \frac{13,95}{3,1} &= 4,5\qquad
            \frac{40,5}{9} = 4,5\\
            \ialors \frac{36}{8} = \frac{13,95}{3,1} &= \frac{40,5}{9}
        \end{align*}
        L'\key{égalité} des quotients indique qu'\key{il s'agit bien} d'une situation de proportionnalité.
        \item \begin{align*}
            \frac{655,36}{128} = 5,12\qquad
            \frac{100,2}{20} &= 5,01\qquad
            \ialors \frac{651,3}{128} \neq \frac{100,2}{20}
        \end{align*}
        L'\key{inégalité} des quotients indique qu'\key{il ne sagit pas} d'une situation de proportionnalité.
    \end{enumerate}
}
