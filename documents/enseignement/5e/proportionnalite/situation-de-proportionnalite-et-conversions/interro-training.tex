% VARIABLES %%%
\setTitle{Interrogation - Entrainement - Séquence 2}
\setGrade{5e}
%%

\calculator

\exo{Les tableaux ci-dessous représentent-ils une situation de proportionnalité ?}{%
\multiColEnumerate{2}{
\item\begin{tabular}{|C{1.5cm}|C{1.5cm}|C{1.5cm}|C{1.5cm}|}
    \hline
    89.08 & 147.39 & 18.53 & 23.46\\
    \hline
    52.4 & 86.7 & 10.9 & 13.8\\
    \hline
\end{tabular}

\item\begin{tabular}{|C{1.5cm}|C{1.5cm}|C{1.5cm}|}
    \hline
    277.5 & 7.83 & 139.2\\
    \hline
    92.5 & 2.7 & 46.4\\
    \hline
\end{tabular}

\item\begin{tabular}{|C{1.5cm}|C{1.5cm}|C{1.5cm}|}
    \hline
    124 & 19.26 & 172.44\\
    \hline
    77.5 & 10.7 & 95.8\\
    \hline
\end{tabular}

\item\begin{tabular}{|C{1.5cm}|C{1.5cm}|}
    \hline
    469.44 & 171.36\\
    \hline
    97.8 & 35.7\\
    \hline
\end{tabular}

\item\begin{tabular}{|C{1.5cm}|C{1.5cm}|C{1.5cm}|C{1.5cm}|}
    \hline
    18.16 & 6.32 & 64.48 & 38.56\\
    \hline
    22.7 & 7.9 & 80.6 & 48.2\\
    \hline
\end{tabular}

\item\begin{tabular}{|C{1.5cm}|C{1.5cm}|C{1.5cm}|C{1.5cm}|}
    \hline
    56.84 & 281.06 & 389.76 & 331.73\\
    \hline
    11.6 & 61.1 & 81.2 & 67.7\\
    \hline
\end{tabular}

\item\begin{tabular}{|C{1.5cm}|C{1.5cm}|C{1.5cm}|C{1.5cm}|}
    \hline
    101.83 & 29.07 & 80.07 & 109.65\\
    \hline
    59.9 & 17.1 & 47.1 & 64.5\\
    \hline
\end{tabular}

\item\begin{tabular}{|C{1.5cm}|C{1.5cm}|}
    \hline
    214.5 & 123.25\\
    \hline
    85.8 & 49.3\\
    \hline
\end{tabular}

\item\begin{tabular}{|C{1.5cm}|C{1.5cm}|C{1.5cm}|}
    \hline
    261.8 & 108.16 & 150.04\\
    \hline
    77 & 33.8 & 48.4\\
    \hline
\end{tabular}

\item\begin{tabular}{|C{1.5cm}|C{1.5cm}|C{1.5cm}|C{1.5cm}|}
    \hline
    58 & 101.4 & 33.2 & 118.2\\
    \hline
    29 & 50.7 & 16.6 & 59.1\\
    \hline
\end{tabular}

\item\begin{tabular}{|C{1.5cm}|C{1.5cm}|C{1.5cm}|}
    \hline
    72.1 & 197.28 & 336.6\\
    \hline
    20.6 & 54.8 & 93.5\\
    \hline
\end{tabular}

\item\begin{tabular}{|C{1.5cm}|C{1.5cm}|C{1.5cm}|}
    \hline
    29.25 & 66.13 & 127.36\\
    \hline
    19.5 & 38.9 & 79.6\\
    \hline
\end{tabular}

\item\begin{tabular}{|C{1.5cm}|C{1.5cm}|}
    \hline
    71.25 & 36.9\\
    \hline
    47.5 & 24.6\\
    \hline
\end{tabular}

\item\begin{tabular}{|C{1.5cm}|C{1.5cm}|C{1.5cm}|C{1.5cm}|}
    \hline
    4.86 & 8.67 & 20.28 & 20.55\\
    \hline
    8.1 & 28.9 & 67.6 & 41.1\\
    \hline
\end{tabular}

\item\begin{tabular}{|C{1.5cm}|C{1.5cm}|}
    \hline
    10.92 & 5.07\\
    \hline
    36.4 & 16.9\\
    \hline
\end{tabular}

\item\begin{tabular}{|C{1.5cm}|C{1.5cm}|C{1.5cm}|C{1.5cm}|}
    \hline
    123.28 & 152.55 & 222.18 & 280.14\\
    \hline
    26.8 & 33.9 & 48.3 & 60.9\\
    \hline
\end{tabular}

\item\begin{tabular}{|C{1.5cm}|C{1.5cm}|}
    \hline
    35.28 & 58.14\\
    \hline
    19.6 & 32.3\\
    \hline
\end{tabular}

\item\begin{tabular}{|C{1.5cm}|C{1.5cm}|C{1.5cm}|C{1.5cm}|}
    \hline
    51 & 121.75 & 178.36 & 125.58\\
    \hline
    20.4 & 48.7 & 68.6 & 54.6\\
    \hline
\end{tabular}

\item\begin{tabular}{|C{1.5cm}|C{1.5cm}|}
    \hline
    209.96 & 40.23\\
    \hline
    72.4 & 14.9\\
    \hline
\end{tabular}

\item\begin{tabular}{|C{1.5cm}|C{1.5cm}|C{1.5cm}|C{1.5cm}|}
    \hline
    94.64 & 13.77 & 279 & 378.93\\
    \hline
    18.2 & 2.7 & 55.8 & 74.3\\
    \hline
\end{tabular}

\item\begin{tabular}{|C{1.5cm}|C{1.5cm}|C{1.5cm}|}
    \hline
    21.76 & 7.48 & 17.6\\
    \hline
    12.8 & 4.4 & 11\\
    \hline
\end{tabular}

\item\begin{tabular}{|C{1.5cm}|C{1.5cm}|}
    \hline
    369 & 43\\
    \hline
    90 & 10\\
    \hline
\end{tabular}

\item\begin{tabular}{|C{1.5cm}|C{1.5cm}|C{1.5cm}|}
    \hline
    267.43 & 339.02 & 401.58\\
    \hline
    56.9 & 73.7 & 87.3\\
    \hline
\end{tabular}

\item\begin{tabular}{|C{1.5cm}|C{1.5cm}|}
    \hline
    16.53 & 178.6\\
    \hline
    8.7 & 94\\
    \hline
\end{tabular}

\item\begin{tabular}{|C{1.5cm}|C{1.5cm}|C{1.5cm}|C{1.5cm}|}
    \hline
    36.96 & 40.59 & 26.4 & 22.77\\
    \hline
    11.2 & 12.3 & 8 & 6.9\\
    \hline
\end{tabular}

\item\begin{tabular}{|C{1.5cm}|C{1.5cm}|C{1.5cm}|}
    \hline
    167.67 & 86.02 & 24.61\\
    \hline
    72.9 & 37.4 & 10.7\\
    \hline
\end{tabular}

\item\begin{tabular}{|C{1.5cm}|C{1.5cm}|C{1.5cm}|C{1.5cm}|}
    \hline
    159.16 & 95.76 & 220.56 & 128.64\\
    \hline
    69.2 & 39.9 & 91.9 & 53.6\\
    \hline
\end{tabular}

\item\begin{tabular}{|C{1.5cm}|C{1.5cm}|C{1.5cm}|}
    \hline
    327.32 & 328.8 & 210.21\\
    \hline
    66.8 & 68.5 & 42.9\\
    \hline
\end{tabular}

\item\begin{tabular}{|C{1.5cm}|C{1.5cm}|C{1.5cm}|C{1.5cm}|}
    \hline
    252.48 & 92.8 & 205.76 & 48\\
    \hline
    78.9 & 29 & 64.3 & 15\\
    \hline
\end{tabular}

\item\begin{tabular}{|C{1.5cm}|C{1.5cm}|C{1.5cm}|}
    \hline
    130.41 & 20.46 & 169.4\\
    \hline
    62.1 & 9.3 & 77\\
    \hline
\end{tabular}

}}

\newpage

\corr{}{%
\begin{enumerate}\item\begin{align*}
\dfrac{89.08}{52.4} = 1.7\qquad \dfrac{147.39}{86.7} = 1.7\qquad \dfrac{18.53}{10.9} = 1.7\qquad \dfrac{23.46}{13.8} = 1.7\qquad 
\end{align*}
L'\key{égalité} des quotients indique qu'\key{il s'agit bien} d'une situation de proportionnalité.

\item\begin{align*}
\dfrac{277.5}{92.5} = 3\qquad \dfrac{7.83}{2.7} = 2.9\qquad \dfrac{139.2}{46.4} = 3\qquad 
\end{align*}
L'\key{inégalité} des quotients indique qu'\key{il ne sagit pas} d'une situation de proportionnalité.

\item\begin{align*}
\dfrac{124}{77.5} = 1.6\qquad \dfrac{19.26}{10.7} = 1.8\qquad \dfrac{172.44}{95.8} = 1.8\qquad 
\end{align*}
L'\key{inégalité} des quotients indique qu'\key{il ne sagit pas} d'une situation de proportionnalité.

\item\begin{align*}
\dfrac{469.44}{97.8} = 4.8\qquad \dfrac{171.36}{35.7} = 4.8\qquad 
\end{align*}
L'\key{égalité} des quotients indique qu'\key{il s'agit bien} d'une situation de proportionnalité.

\item\begin{align*}
\dfrac{18.16}{22.7} = 0.8\qquad \dfrac{6.32}{7.9} = 0.8\qquad \dfrac{64.48}{80.6} = 0.8\qquad \dfrac{38.56}{48.2} = 0.8\qquad 
\end{align*}
L'\key{égalité} des quotients indique qu'\key{il s'agit bien} d'une situation de proportionnalité.

\item\begin{align*}
\dfrac{56.84}{11.6} = 4.9\qquad \dfrac{281.06}{61.1} = 4.6\qquad \dfrac{389.76}{81.2} = 4.8\qquad \dfrac{331.73}{67.7} = 4.9\qquad 
\end{align*}
L'\key{inégalité} des quotients indique qu'\key{il ne sagit pas} d'une situation de proportionnalité.

\item\begin{align*}
\dfrac{101.83}{59.9} = 1.7\qquad \dfrac{29.07}{17.1} = 1.7\qquad \dfrac{80.07}{47.1} = 1.7\qquad \dfrac{109.65}{64.5} = 1.7\qquad 
\end{align*}
L'\key{égalité} des quotients indique qu'\key{il s'agit bien} d'une situation de proportionnalité.

\item\begin{align*}
\dfrac{214.5}{85.8} = 2.5\qquad \dfrac{123.25}{49.3} = 2.5\qquad 
\end{align*}
L'\key{égalité} des quotients indique qu'\key{il s'agit bien} d'une situation de proportionnalité.

\item\begin{align*}
\dfrac{261.8}{77} = 3.4\qquad \dfrac{108.16}{33.8} = 3.2\qquad \dfrac{150.04}{48.4} = 3.1\qquad 
\end{align*}
L'\key{inégalité} des quotients indique qu'\key{il ne sagit pas} d'une situation de proportionnalité.

\item\begin{align*}
\dfrac{58}{29} = 2\qquad \dfrac{101.4}{50.7} = 2\qquad \dfrac{33.2}{16.6} = 2\qquad \dfrac{118.2}{59.1} = 2\qquad 
\end{align*}
L'\key{égalité} des quotients indique qu'\key{il s'agit bien} d'une situation de proportionnalité.

\item\begin{align*}
\dfrac{72.1}{20.6} = 3.5\qquad \dfrac{197.28}{54.8} = 3.6\qquad \dfrac{336.6}{93.5} = 3.6\qquad 
\end{align*}
L'\key{inégalité} des quotients indique qu'\key{il ne sagit pas} d'une situation de proportionnalité.

\item\begin{align*}
\dfrac{29.25}{19.5} = 1.5\qquad \dfrac{66.13}{38.9} = 1.7\qquad \dfrac{127.36}{79.6} = 1.6\qquad 
\end{align*}
L'\key{inégalité} des quotients indique qu'\key{il ne sagit pas} d'une situation de proportionnalité.

\item\begin{align*}
\dfrac{71.25}{47.5} = 1.5\qquad \dfrac{36.9}{24.6} = 1.5\qquad 
\end{align*}
L'\key{inégalité} des quotients indique qu'\key{il ne sagit pas} d'une situation de proportionnalité.

\item\begin{align*}
\dfrac{4.86}{8.1} = 0.6\qquad \dfrac{8.67}{28.9} = 0.3\qquad \dfrac{20.28}{67.6} = 0.3\qquad \dfrac{20.55}{41.1} = 0.5\qquad 
\end{align*}
L'\key{inégalité} des quotients indique qu'\key{il ne sagit pas} d'une situation de proportionnalité.

\item\begin{align*}
\dfrac{10.92}{36.4} = 0.3\qquad \dfrac{5.07}{16.9} = 0.3\qquad 
\end{align*}
L'\key{égalité} des quotients indique qu'\key{il s'agit bien} d'une situation de proportionnalité.

\item\begin{align*}
\dfrac{123.28}{26.8} = 4.6\qquad \dfrac{152.55}{33.9} = 4.5\qquad \dfrac{222.18}{48.3} = 4.6\qquad \dfrac{280.14}{60.9} = 4.6\qquad 
\end{align*}
L'\key{inégalité} des quotients indique qu'\key{il ne sagit pas} d'une situation de proportionnalité.

\item\begin{align*}
\dfrac{35.28}{19.6} = 1.8\qquad \dfrac{58.14}{32.3} = 1.8\qquad 
\end{align*}
L'\key{égalité} des quotients indique qu'\key{il s'agit bien} d'une situation de proportionnalité.

\item\begin{align*}
\dfrac{51}{20.4} = 2.5\qquad \dfrac{121.75}{48.7} = 2.5\qquad \dfrac{178.36}{68.6} = 2.6\qquad \dfrac{125.58}{54.6} = 2.3\qquad 
\end{align*}
L'\key{inégalité} des quotients indique qu'\key{il ne sagit pas} d'une situation de proportionnalité.

\item\begin{align*}
\dfrac{209.96}{72.4} = 2.9\qquad \dfrac{40.23}{14.9} = 2.7\qquad 
\end{align*}
L'\key{inégalité} des quotients indique qu'\key{il ne sagit pas} d'une situation de proportionnalité.

\item\begin{align*}
\dfrac{94.64}{18.2} = 5.2\qquad \dfrac{13.77}{2.7} = 5.1\qquad \dfrac{279}{55.8} = 5\qquad \dfrac{378.93}{74.3} = 5.1\qquad 
\end{align*}
L'\key{inégalité} des quotients indique qu'\key{il ne sagit pas} d'une situation de proportionnalité.

\item\begin{align*}
\dfrac{21.76}{12.8} = 1.7\qquad \dfrac{7.48}{4.4} = 1.7\qquad \dfrac{17.6}{11} = 1.6\qquad 
\end{align*}
L'\key{inégalité} des quotients indique qu'\key{il ne sagit pas} d'une situation de proportionnalité.

\item\begin{align*}
\dfrac{369}{90} = 4.1\qquad \dfrac{43}{10} = 4.3\qquad 
\end{align*}
L'\key{inégalité} des quotients indique qu'\key{il ne sagit pas} d'une situation de proportionnalité.

\item\begin{align*}
\dfrac{267.43}{56.9} = 4.7\qquad \dfrac{339.02}{73.7} = 4.6\qquad \dfrac{401.58}{87.3} = 4.6\qquad 
\end{align*}
L'\key{inégalité} des quotients indique qu'\key{il ne sagit pas} d'une situation de proportionnalité.

\item\begin{align*}
\dfrac{16.53}{8.7} = 1.9\qquad \dfrac{178.6}{94} = 1.9\qquad 
\end{align*}
L'\key{égalité} des quotients indique qu'\key{il s'agit bien} d'une situation de proportionnalité.

\item\begin{align*}
\dfrac{36.96}{11.2} = 3.3\qquad \dfrac{40.59}{12.3} = 3.3\qquad \dfrac{26.4}{8} = 3.3\qquad \dfrac{22.77}{6.9} = 3.3\qquad 
\end{align*}
L'\key{égalité} des quotients indique qu'\key{il s'agit bien} d'une situation de proportionnalité.

\item\begin{align*}
\dfrac{167.67}{72.9} = 2.3\qquad \dfrac{86.02}{37.4} = 2.3\qquad \dfrac{24.61}{10.7} = 2.3\qquad 
\end{align*}
L'\key{égalité} des quotients indique qu'\key{il s'agit bien} d'une situation de proportionnalité.

\item\begin{align*}
\dfrac{159.16}{69.2} = 2.3\qquad \dfrac{95.76}{39.9} = 2.4\qquad \dfrac{220.56}{91.9} = 2.4\qquad \dfrac{128.64}{53.6} = 2.4\qquad 
\end{align*}
L'\key{inégalité} des quotients indique qu'\key{il ne sagit pas} d'une situation de proportionnalité.

\item\begin{align*}
\dfrac{327.32}{66.8} = 4.9\qquad \dfrac{328.8}{68.5} = 4.8\qquad \dfrac{210.21}{42.9} = 4.9\qquad 
\end{align*}
L'\key{inégalité} des quotients indique qu'\key{il ne sagit pas} d'une situation de proportionnalité.

\item\begin{align*}
\dfrac{252.48}{78.9} = 3.2\qquad \dfrac{92.8}{29} = 3.2\qquad \dfrac{205.76}{64.3} = 3.2\qquad \dfrac{48}{15} = 3.2\qquad 
\end{align*}
L'\key{égalité} des quotients indique qu'\key{il s'agit bien} d'une situation de proportionnalité.

\item\begin{align*}
\dfrac{130.41}{62.1} = 2.1\qquad \dfrac{20.46}{9.3} = 2.2\qquad \dfrac{169.4}{77} = 2.2\qquad 
\end{align*}
L'\key{inégalité} des quotients indique qu'\key{il ne sagit pas} d'une situation de proportionnalité.

\end{enumerate}}