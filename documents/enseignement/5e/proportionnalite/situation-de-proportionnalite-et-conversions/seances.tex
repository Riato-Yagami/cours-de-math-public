% VARIABLES %%%
\def\authors{\jules}
% \date{\today}
\def\longTitle{Proportionnalité - Situation de proportionnalité et conversions}
\def\shortTitle{\MakeUppercase{\longTitle}}

\setcounter{seq}{2}
\bseq{\longTitle}
% \def\theme{\longTitle}

\setgrade{5e}

\def\imgPath{enseignement/5e/proportionnalite/situation-de-proportionnalite-et-conversions/}
\def\imgExtension{.png}
%%

% Yvan Monka : https://www.maths-et-tiques.fr/telech/19Prop1.pdf
\ifArticle{\vspace*{0.1cm}}

\obj{
    \item Reconnaître une situation de proportionnalité ou de non proportionnalité entre deux grandeurs.
    \item Résoudre des problèmes de proportionnalité par passage à l'unité.
    \item Effectuer des calculs de durées et d'horaires.
    \item Effectuer des conversions d'unités de longueurs, et de durées.
    \item Partager une quantité en deux ou trois parts selon un ratio donné.
}

\scn{Passage à l'unité}{}

\qfSlide{
    Donner le resultat sous la forme décimale de :
    \multiColEnumerate{2}{
        \item $14 \times 3 = \palt{2}{\ncalc{14*3}} $
        \item $3 \div 2 = \palt{2}{1,5} $
        \item $6 \div 4 = \palt{2}{1,5} $
        \item $5,6 \times 2 = \palt{2}{11,2} $
        \item $\frac{1}{4} = \palt{2}{0.25}$
        \item $\frac{3}{4} = \palt{2}{0.75}$
    }
}

\bsec{Proportionnalité}
\bsubsec{Définition}

\slide{exo}{
    \bvspace{-0.5cm}
    \act{}{
        \begin{itemize}
            \item Pour une recette de dahl,
            Christopher a besoin de 8 gousses d'ail pour 4 personnes.
            Il déjeune avec ses amis Sarah et Jean.
            Peut-il prévoir combien de gousses d'ail il aura besoin ?\\
            Pourquoi ? et si oui, combien ?
            \item Jean l'a félicité 2 fois pour sa cuisine en 20 minutes.
            Peut-il prévoir combien de félicitations il recevra de Jean en 40 minutes?\\
            Pourquoi ? et si oui, combien ?
        \end{itemize}
        
    }
}

\slide{cr}{
    \sseq\ssec\ssubsec
    \df{}{
        Deux grandeurs sont dites \key{proportionnelles} si les valeurs de l'une s'obtiennent en multipliant les valeurs de l'autre par un même nombre non nul,
        appelé le \key{coefficient de proportionnalité}.
    }
}

\slide{cr}{
    \expl{}{
        Les $2kg$ de lentilles $5\EUR$.
        Combien en coutent $13kg$ de lentilles ?
    }

    \mthd{Passage à l'unité}{
        \palt{2}{
            \begin{tabular}{|c|c|c|c|}
                \hline
                Masse (en $\kilo\gram$)  & 2 & 1 & 13\\
                \hline
                Prix (en \EUR)    & 5 & 2,5 & 32,5\\
                \hline
            \end{tabular}
        }
        \palt{3}{Le coefficient de proportionnalité est 2,5.}
    }
}

\slide{cr}{
    \vc{}{
        Un \key{tableau de proportionnalité} permet de présenter une situation de proportionnalité.
        Sa deuxième ligne s'obtient en multipliant la première par le coefficient de proportionnalité.
    }
}

\def\imgPrefix{mm-c4/exo-}
\exoSlide{32p27,51p28,53p28}[7cm][2][\mm]

\scn{Reconnaître une situation de proportionnalité}{}

\def\imgPrefix{mm-c4/qf-}
\exoSlide{15p26,17p26}[7cm][2][\mm][qf]

\bsubsec{Reconnaître une situation de proportionnalité}

\slide{cr}{
    \bvspace{-0.25cm}
    \ssubsec
    \bvspace{-0.5cm}
    \expl{}{
        Une marque d'épices vend différentes tailles de pots de curcuma et de cumin.
        Présenter avec leur prix dans les deux tableaux ci-dessous.\\

        \begin{tabular}{|c|c|c|c|}
            \hline
            Masse de curcuma (en $\gram$)  & 45 & 60 & 90\\
            \hline
            Prix (en \EUR)    & 2.5 & 4 & 5\\
            \hline
        \end{tabular}\ifArticle{\quad}\ifBeamer{\\ \\}
        \begin{tabular}{|c|c|c|c|}
            \hline
            Masse de cumin (en $\gram$)  & 45 & 60 & 90\\
            \hline
            Prix (en \EUR)    & 2.7 & 4.05 & 5.4\\
            \hline
        \end{tabular}\\

        Les prix de ces deux épices sont-ils proportionnels à leur masse ?
    }
}

\slide{cr}{
    \mthd{Reconnaître une situation de proportionnalité}{
        \palt{2}{
            On divise chaque nombre de la première grandeur par ceux de la 2e grandeurs correspondant.
            Si on obtient le même résultat, il y a proportionnalité.
        }
    }
}

% \exoSlide{}[][][][]

\bsec{Grandeurs}

\slide{cr}{
    \df{}{
        Une \key{grandeur} est une caractéristique d'un objet qui se mesure ou se calcule.
    }

    \rmk{}{
        On donne la valeur d'une grandeur en comparaison à une \key{unité de mesure}
    }

    \expl{}{
        \begin{tabular}{|c|c|c|c|c|c|c|c|}
            \hline
            Grandeurs & Masse & Longueurs & Volume & Temps & Stockage\\
            \hline
            Mesure & gramme & mètre & litre et mètre cube & seconde & octet\\
            \hline
        \end{tabular}
    }

    \df{}{
        Deux \key{grandeurs proportionnelles} sont deux grandeurs qui varient dans les mêmes proportions.
    }
}


\def\imgPrefix{}
\slide{qf}{
    \begin{enumerate}
        \item \imgp{qf-cn-3-5e-mars-2023}[6cm]
        \item \imgp{qf-cn-8-5e-mars-2023}[6cm]
    \end{enumerate}
}

\bsubsec{Longueurs}

\slide{cr}{
    \expl{}{
        \begin{enumerate}
            \item Le périmètre d'un cercle est proportionnel à son rayon.
            \item L'aire d'un carré n'est pas proportionnelle à la longueur des côtés.
        \end{enumerate}
    }
}

\bsubsec{Durées}

\bsec{Ratio}