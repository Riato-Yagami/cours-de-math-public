% VARIABLES %%%
\def\authors{\jules}
\setSeq{2}{Proportionnalité - Situation de proportionnalité et conversions}
\setGrade{5e}

\def\imgPath{enseignement/5e/proportionnalite/situation-de-proportionnalite-et-conversions/}
\def\imgExtension{.png}

% \firstSlide
%%

% Yvan Monka : https://www.maths-et-tiques.fr/telech/19Prop1.pdf
\ifArticle{\vspace*{0.1cm}}

\obj{
    \item Reconnaître une situation de proportionnalité ou de non proportionnalité entre deux grandeurs.
    \item Résoudre des problèmes de proportionnalité par passage à l'unité.
    \item Effectuer des calculs de durées et d'horaires.
    \item Effectuer des conversions d'unités de longueurs, et de durées.
    \item Partager une quantité en deux ou trois parts selon un ratio donné.
}

\scn{Passage à l'unité}{}

\qfSlide{
    Donner le resultat sous la forme décimale de :
    \multiColEnumerate{2}{
        \item $14 \times 3 = \palt{2}{\ncalc{14*3}} $
        \item $3 \div 2 = \palt{2}{1,5} $
        \item $6 \div 4 = \palt{2}{1,5} $
        \item $5,6 \times 2 = \palt{2}{11,2} $
        \item $\frac{1}{4} = \palt{2}{0.25}$
        \item $\frac{3}{4} = \palt{2}{0.75}$
    }
}

\bsec{Proportionnalité}
\bsubsec{Définition}

\slide{exo}{
    \bvspace{-0.5cm}
    \act{}{
        \begin{itemize}
            \item Pour une recette de dahl,
            Christopher a besoin de 8 gousses d'ail pour 4 personnes.
            Il déjeune avec ses amis Sarah et Jean.
            Peut-il prévoir combien de gousses d'ail il aura besoin ?\\
            Pourquoi ? et si oui, combien ?
            \item Jean l'a félicité 2 fois pour sa cuisine en 20 minutes.
            Peut-il prévoir combien de félicitations il recevra de Jean en 40 minutes?\\
            Pourquoi ? et si oui, combien ?
        \end{itemize}
        
    }
}

\slide{cr}{
    \sseq\ssec\ssubsec
    \df{}{
        Deux grandeurs sont dites \key{proportionnelles} si les valeurs de l'une s'obtiennent en multipliant les valeurs de l'autre par un même nombre non nul,
        appelé le \key{coefficient de proportionnalité}.
    }
}

\slide{cr}{
    \expl{}{
        Les $2kg$ de lentilles $5\EUR$.
        Combien en coutent $13kg$ de lentilles ?
    }

    \mthd{Passage à l'unité}{
        \palt{2}{
            \begin{tabular}{|c|c|c|c|}
                \hline
                Masse (en $\kilo\gram$)  & 2 & 1 & 13\\
                \hline
                Prix (en \EUR)    & 5 & 2,5 & 32,5\\
                \hline
            \end{tabular}
        }
        \palt{3}{Le coefficient de proportionnalité est 2,5.}
    }
}

\slide{cr}{
    \vc{}{
        Un \key{tableau de proportionnalité} permet de présenter une situation de proportionnalité.
        Sa deuxième ligne s'obtient en multipliant la première par le coefficient de proportionnalité.
    }
}

\def\imgPrefix{mm-c4/exo-}
\exoSlide{32p27,51p28,53p28}[7cm][2][\mm]

\scn{Reconnaître une situation de proportionnalité}{}

\def\imgPrefix{mm-c4/qf-}
\exoSlide{15p26,17p26}[7cm][2][\mm][qf]

\bsubsec{Reconnaître une situation de proportionnalité}

\slide{cr}{
    \bvspace{-0.25cm}
    \ssubsec
    \bvspace{-0.5cm}
    \expl{}{
        Une marque d'épices vend différentes tailles de pots de curcuma et de cumin.
        Présenter avec leur prix dans les deux tableaux ci-dessous.\\

        \begin{tabular}{|c|c|c|c|}
            \hline
            Masse de curcuma (en $\gram$)  & 45 & 60 & 90\\
            \hline
            Prix (en \EUR)    & 2.5 & 4 & 5\\
            \hline
        \end{tabular}\ifArticle{\quad}\ifBeamer{\\ \\}
        \begin{tabular}{|c|c|c|c|}
            \hline
            Masse de cumin (en $\gram$)  & 45 & 60 & 90\\
            \hline
            Prix (en \EUR)    & 2.7 & 4.05 & 5.4\\
            \hline
        \end{tabular}\\

        Les prix de ces deux épices sont-ils proportionnels à leur masse ?
    }
}

\slide{cr}{
    \mthd{Reconnaître une situation de proportionnalité}{
        \palt{2}{
            On divise chaque nombre de la première grandeur par ceux de la 2e grandeurs correspondant.
            Si on obtient le même résultat, il y a proportionnalité.
        }
    }
}

% \exoSlide{}[][][][]
\scn{Conversions}

\def\imgPrefix{}
\slide{qf}{
    \begin{enumerate}
        \item \imgp{qf-cn-3-5e-mars-2023}[6cm]
        \item \imgp{qf-cn-8-5e-mars-2023}[6cm]
    \end{enumerate}
}

\scn{Distinction grandeurs et mesures}

\bsec{Grandeurs}

\bsubsec{Definition}

% \newpage
\slide{exo}{
    \act{Trier ces mots en deux catégories que vous nommerez}{
        \avspace{-0.5cm}\multiColItemize{2}{
            \item volume
            \item durée
            \item centimètre
            \item prix
            \item euro
            \item gramme
            \item litre
            \item surface
        }
    }
}

\slide{cr}{
    \ssec\ssubsec
    \df{}{
        Une \key{grandeur} est une caractéristique d'un objet qui se mesure ou se calcule.
    }
}

\slide{VIDEO}{
    \bvspace{-1cm}
    \hist{}{
        \imgp{l-histoire-du-metre}[9cm]
        \begin{center}
            \hypersetup{urlcolor = gradeColor!90}
            \href{https://youtu.be/PvlsXcOzNd0?si=P0qko7XnxAFT-KCm}
            {L'histoire du mètre (J't'explique)}
        \end{center}
    }[\href{https://youtu.be/PvlsXcOzNd0?si=P0qko7XnxAFT-KCm}{J't'explique}]
}

\slide{}{
    \vc{}{
        On donne la valeur d'une grandeur en comparaison à une \key{unité de mesure} de référence.
    }
    
    \bvspace{-1cm}
    \expl{}{
        \ifBeamer{\\}
        \def\cW{1.8cm}
        \wideFrame[7em]{
            \begin{center}
                \begin{tabular}{|C{\cW}|C{\cW}|C{\cW}|C{\cW}|C{\cW}|C{\cW}|C{\cW}|C{\cW}|}
                    \hline
                    \color{Red}Grandeur & \palt{2}{Masse} & \palt{2}{Longueurs} & \palt{2}{Volume} & \palt{2}{Temps} & \palt{2}{Stockage}\\
                    \hline
                    \color{Red}Unité de mesure & \palt{2}{gramme} & \palt{2}{mètre} & \palt{2}{litre et mètre cube} & \palt{2}{seconde} & \palt{2}{octet}\\
                    \hline
                \end{tabular}
            \end{center}
        }
    }
}

\scn{Utiliser les préfixes de mesures}

\def\imgPrefix{mm-c4/}
\exoSlide{13p58,14p58}[6cm][2][\mm][qf]

\bsubsec{Conversions}

\slide{exo}{
    \act{Compléte avec les unités manquantes}{\avspace{-0.5cm}
        \multiColEnumerate{2}{
            \item Une bouteille de $1,5 \palt{2}{ \liter}$
            \item Une règle de $30 \palt{2}{ \centi\meter}$
            \item Une feuille d'une épaisseur de $0,2 \palt{2}{ \milli\meter}$
            \item Une bébé de $3 \palt{2}{ \kilo\gram}$
            \item Un verre de $24 \palt{2}{ \centi\liter}$
            \item Une carte mémoire de $64 \palt{2}{ \giga\octet}$
            \item Un pont de $1,5 \palt{2}{ \kilo\meter}$
        }
    }
}

\def\imgPrefix{mm-c4/}
\exoSlide{25p384,38p385}[7cm][2][\mm]

\slide{cr}{\def\cW{1cm}
    % \wideFrame[6em]{
    \bvspace{-0.5cm}
        \vc{}{
            \ifBeamer{\\ \\}
            \begin{tabular}{|C{3cm}|C{\cW}|C{\cW}|C{\cW}|C{\cW}|C{\cW}|C{\cW}|C{\cW}|}
                \hline
                \key{Préfixe} & kilo & hecto & deca & \_ & déci & centi & milli \\
                \hline
                \key{Symbole} & \kilo & \hecto & \deca & \_ & \deci & \centi & \milli \\
                \hline
                \key{Signification} & $1000$ & $100$ & $10$ & $1$ & $0,1$ & $0,01$ & $0,001$ \\
                \hline
            \end{tabular}
        }
        \bvspace{-1cm}
        \expl{}{
            \multiColEnumerate{2}{
                \item $1\deci\meter = \palt{2}{0,1} \meter$
                \item $3,6\hecto\liter = \palt{2}{360} \liter$
                \item $1\meter = \palt{2}{100} \centi\meter$
                \item $1\kilo\octet = \palt{2}{100} \deca\octet$
                \item $1\hecto\gram = \palt{2}{100 000} \milli\gram$
                \item $12\centi\meter = \palt{2}{0.012} \deca\meter$ 
            }
        }
    % }
}

\exoSlide{43p385}[9cm][1][\mm]

\scn{Convertir des durées}

\exoSlide{26p384,30p384}[9cm][1][\mm][qf]

\bsubsec{Durées}

\slide{exo}{
    \act{}{
        Un pilotte de kart parcours 90 km en trois-quarts d'heure.
        En imaginant qu'il soit capable de maintenir cette vitesse,
        combien de kilomètres aura-t-il parcourut en 1 heure ?
    }
}

\slide{cr}{
    \ssubsec
    \pr{}{
        \multiColItemize{3}{
            \item $1\hour = \palt{2}{60}\minute$
            \item $1\minute = \palt{2}{60}\second$
            \item $1\textrm{ jour} = \palt{2}{24}\hour$
        }
    }

    \expl{}{
        \begin{enumerate}
            \item Combien y'a-t-il de seconds dans une journée ?
            \item Convertir $78\min$ en heures.
            \item Convertir $7292\sec$ en heures, minutes, secondes.
        \end{enumerate}
    }
}

% \slide{exo}{
%     \exo{}{
%         On dispose de deux robinets.
%         Le premier est capable de remplir un réservoir d'eau de 24 L en 1 minute,
%         le second peut remplir ce même réservoir en 2 minutes.
%         Ouvrant les deux robinets au même moment,
%         combien de temps faudrait-il pour remplir un jacuzzi avec 1 080 L d'eau?
%         \multiColEnumerate{4}{
%             \item 15 min
%             \item 67,5 min
%             \item 135 min
%             \item 30 min
%         }
%     }[\rpmc[162]]
% }

% \bsubsec{Grandeurs proportionnelles}

\scn{Problème de proportionnalité}

\exoSlide{27p384}[9cm][1][\mm][qf]

\slide{exo}{
    \exo{}{
        \begin{enumerate}
            \item Le périmètre d'un cercle est-il proportionnel à son rayon?
            \item L'aire d'un carré est-elle proportionnelle à la longueur de ces côtés?
        \end{enumerate}
    }
}

\slide{exo}{
    \exo{}{
        On dispose de deux robinets.
        Le premier est capable de remplir un réservoir d'eau de 24 L en 1 minute,
        le second peut remplir ce même réservoir en 2 minutes.
        Ouvrant les deux robinets au même moment,
        combien de temps faudrait-il pour remplir un jacuzzi avec 1 080 L d'eau?
        \multiColEnumerate{4}{
            \item 15 min
            \item 67,5 min
            \item 135 min
            \item 30 min
        }
    }[\rpmc[162]]
}

\scn{Decouvrir les ratios}

\bsec{Ratio}

\exoSlide{28p384}[9cm][1][\mm][qf]

\slide{exo}{
    \act{}{
        \begin{enumerate}
            \item Pour faire un dahl formidable,
            il faut mettre du lait de coco et de l'eau dans le ratio 3 pour 2 (noté $3 : 2$).
            Combien faut-il ajouter d'eau pour $150\milli\liter$ de lait de coco?
            \item Afin d'améliorer encore les saveurs,
            on peut ajouter un peu de sauces piquante.
            La recette indique le ratio $1:9:6$. A quel nombre du ratio peut correspondre la sauce piquante?
        \end{enumerate}
    }
}
\slide{cr}{
    \ssec

    \df{}{
        Un \key{ratio} exprime une comparaison entre plusieurs quantités.
    }[\href{https://pedagogie.ac-montpellier.fr/sites/default/files/ressources/Les\%20ratios\%20au\%20cycle\%204.pdf\#page=6}{Académie de Montpellier}]
}

\slide{}{
    \expl{}{
        Assan et Clara se partagent les 140 bonbons qu'ils ont obtenus à Halloween.
        Clara en récupère 60.
        \begin{enumerate}
            \item Dans quel ratio se sont-il partagé ces bonbons?
            \item Quel fraction des bonbons récupère Assan?
            \item Si Clara récupère 9 bonbons et que l'on reste dans le même ratio,
            combien Assan v'a-t-il en récupèrer?
        \end{enumerate}
    }
}

\scn{Utiliser les ratios}

\slide{qf}{
    \exo{}{
        Recette de cocktail sans alcool :
        \begin{itemize}
            \item Jus d'orange, grenadine dans le ratio 3:2
            \item Cola, limonade, jus de citron, dans le ratio 4:5:2
        \end{itemize}
        \begin{enumerate}
            \item Pour $15\centi\liter$ combien faut-il de grenadine.
            \item Pour $20\centi\liter$ de limonade combien faut-il de cola et jus de citron.
        \end{enumerate}
    }
}

\slide{exo}{
    \bvspace{-0.75cm}
    \exo{}{
        \begin{enumerate}
            \item Comment partager 48 macarons entre Simon et Mandy dans le ratio $5:11$?
            \item Ahmed, Simon et Mandy se partagent des macarons dans le ratio $4:3:2$.
            Simon en a 9, combien en ont Ahmed et Mandy?
            \item Simon et Mandy ont réalisé un certain nombre de macarons dans le ratio $5:8$.
            Sachant que Mandy,
            plus expérimentée,
            a fait 66 macarons de plus que Simon,
            combien Mandy en a préparé?
        \end{enumerate}
    }[\rpmc[63]]
}