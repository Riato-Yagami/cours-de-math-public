%%%
\setGrade{5e}
\setTitle{Evaluation 1 - Correction}
\setboolean{answer}{true}
%%%

\seqEvaluation{1}{Nombres Relatifs - Repérage et comparaison}{
    Se repérer sur une droite graduée
    /2,
    Se repérer dans un plan
    /3,
    Comparer des nombres relatifs
    /2,
    Utiliser la notion d'opposé
    /1
}

\seqEvaluation{2}{Proportionnalité - Situation de proportionnalité et conversions}{
    % Reconnaître une situation de proportionnalité
    % /3,
    Résoudre des problèmes de proportionnalité
    /4,
    Effectuer des conversions d'unités
    /3,
    Partager une quantité selon un ratio donné
    /0%
}

\evalutionEnd

\noCalculator

\exo{Droite graduée}{
    Place sur la droite graduée les points suivants :
    \multiColItemize{2}{
        \item $A$ d'abscisse $-3$
        % \item $B$ d'abscisse $-0,7$
        \item $B$ d'abscisse $2,3$
        \item $C$ d'abscisse opposé à l'abscisse de $B$.
        \item $D$ d'abscisse opposé à l'abscisse de $A$.
    }

    \vspace{0.75cm}

    \begin{tikzpicture}[scale = 2]
        % Dessin de la droite graduée
        \draw[-{Latex[length=5mm, width=5mm]}] (-4,0) -- (3.5,0);
    
        % Graduation de la droite avec 10 sous-gradations entre chaque unité
        \foreach \x in {-3,-2,-1,0,1,2,3} {
            \draw (\x,0.1) -- (\x,-0.1);
        }
        
        % Sous-gradations (10 parties entre chaque unité)
        \foreach \x in {-4,-3,-2,-1,0,1,2} {
            \foreach \i in {1,2,...,9} {
                \draw (\x + \i*0.1, 0.05) -- (\x + \i*0.1, -0.05);
            }
        }
    
        % Notation seulement sur 0 et 1
        \node at (0, -0.3) {0};
        \node at (1, -0.3) {1};
        \drawTick{1.8}[E]
        \drawTick{-3.6}[F]

        \drawTick{-3}[A][resultColor]
        \drawTick{2.3}[B][resultColor]

        \drawTick{-2.3}[C][resultColor]
        \drawTick{3}[D][resultColor]
    \end{tikzpicture}

    \begin{itemize}
        \item Donner l'abscisse des points $E$, $F$ et $C$.
    \end{itemize}
}

\answerFill[Réponse][\begin{itemize}
    \item $E$ est d'abscisse $1,8$
    \item $F$ est d'abscisse $-3,6$
    \item $C$ est d'abscisse $-2,3$
\end{itemize}]

\exo{70p198}{\vspace{-0.5cm}
    \imgp{enseignement/5e/nombres-relatifs/reperage-et-comparaison/mm-c4/exo-70p198-no-bonus.png}[10cm]
}[\mm]

\answerSec{4}[Réponse][Il y a 7 chemins possibles :
\multiColEnumerate{2}{
    \item $-7,8 < -5,2 < -3 < 2$
    \item $-7,8 < -5,2 < -3 < 0 < 2$
    \item $-7,8 < -5,2 < -3 < -2 < 0 < 2$
    \item $-7,8 < -6,5 < -5,2 < -3 < 2$
    \item $-7,8 < -6,5 < -5,2 < -3 < 0 < 2$
    \item $-7,8 < -6,5 < -5,2 < -3 < -2 < 0 < 2$
    \item $-7,8 < -6,5 < -2 < 0 < 2$
}]

\exo{Proportionnalité}{
    Chez le primeur, pour les fraises, le prix payé est proportionnel au nombre de barquettes achetées.
    Cinq barquettes coûtent 5,75 €. Combien coûtent quatre barquettes de fraises ?

}[\href{https://manuel.sesamath.net/coll_docs/cmep/valide/kidimath_DS_5N5.pdf}{Sésamath}]

\answerFill[Réponse, calculs et justifications][
    Le prix des fraises est proportionnel au nombre de barquettes achetées.
    On peut utiliser la méthode du \textbf{retour à l'unité} pour résoudre ce problème,
    et le présenter sous la forme d'un tableau de proportionnalité.
    \begin{center}
        \begin{tabular}{|c|c|c|c|}
            \hline
            Nombre de barquettes & $5$ & $\frac{5}{5} = 1$ & $1 \times 4 = 4$\\\hline
            Prix (en euros) & $5,75$ & $\frac{5,75}{5} = 1,15$ & $1,15 \times 4 = 4,60$\\
            \hline
        \end{tabular}
    \end{center}
    Ainsi, le prix de quatre barquettes est de $4,60 \euro$.
]

\exo{\tiersTemps Conversion}{
    \multiColEnumerate{2}{
        % \item $5\deci\gram = \qqqquad \gram$  
        \item $0,75\;\kilo\meter = \awsr{750}\;\meter$  
        \item $150\;\centi\liter = \awsr{0,15}\;\deca\liter$  
        \item $2\;\hour = \awsr{7200}\;\second$
        \item $4221\;\second = \awsr{1}\;\hour\;\awsr{10}\;\minute\;\awsr{21}\;\second$
        % \item $144\min = \qqqquad \hour$
    }
    \answerSec{6}[Calculs][\begin{enumerate}
        \item $0,75 \times 1000 = 750$
        \item $150 \div 1000 = 0,15$
        \item $2 \times 60 \times 60 = 2 \times 3600 = 7200$
        \item $4221 = 3600 + 621 = 60 \times 60 + 10 \times 60 + 21$
        \\or $60 \times 60\;\second = 1\;\hour$
        \\et $10 \times 60\;\second = 10\;\minute$
    \end{enumerate}]
}

\exo{Repérage dans le plan}{
    \begin{enumerate}
        \item Placer les points $A(-2;-2)$ , $B(-2;1)$ et $C(3;1)$ dans le repère ci-dessous.
        \item Placer le point $D$ pour former le rectangle $ABCD$.
        \item Donner les coordonnées du point $D$.
    \end{enumerate}

    \begin{center}
        \begin{tikzpicture}
            \tkzInit[xmin=-4,xmax=4,ymin=-4,ymax=4]
            \tkzGrid[color=gradeColor!75]
            % \tkzDrawX[label={}] \tkzDrawY[label={}]
            \tkzAxeXY[label={}]
            \drawPoint{A}{-2}{-2}[resultColor]
            \drawPoint{B}{-2}{1}[resultColor]
            \drawPoint{C}{3}{1}[resultColor]
            \drawPoint{D}{3}{-2}[resultColor]
        \end{tikzpicture}
    \end{center}

}[\href{https://www.sunudaara.com/mathematiques/exercices-repérage-dans-le-plan-5e}{Sunudaara}]

\answerFill[Réponse {\color{Gray}3.}][On trouve comme coordonnées $D(3;-2)$]

\exo{\bonus Ratio}{
    Pour commencer un jeu, le premier joueur doit recevoir deux fois plus de cartes que le second,
    qui lui-même doit recevoir quatre fois plus de cartes que le troisième.
    \begin{enumerate}
        \item Selon quel ratio a lieu le partage des cartes ?
        \item Peut-on partager ainsi un jeu de 54 cartes ?
    \end{enumerate}
}[\href{https://college-willy-ronis.fr/maths/wp-content/uploads/2022/06/chap20-exercices_corriges-5eme.pdf}{Collège Willy Ronis}]

\answerFill[Réponse, calculs et justifications][
    \begin{enumerate}
        \item \begin{itemize}
            \item Si le troisième joueur reçoit une carte.
            \item On sait que le deuxième joueur en reçoit quatre fois plus.
            \item $4\times1 = 4$
            \item Alors le deuxième joueur reçoit $4$ cartes.
            \item On sait que le premier joueur en reçoit deux fois plus.
            \item $2\times4 = 8$
            \item Alors le premier joueur reçoit $8$ cartes.
            \item Alors le premier, deuxième et troisième joueur,
            se partagent les cartes dans un ratio $8 : 4 : 1$.
        \end{itemize}
        \item \begin{itemize}
            \item $8 + 4 + 1 = 13$
            \item Alors dans un ratio $8 : 4 : 1$ il y a $13$ parts.
            \item Or $54 = 4 \times 13 + 2$.
            \item Alors on ne peut pas diviser $54$ cartes en $13$ parts.
            \item Alors on ne peut pas partager un jeu de $54$ cartes dans un ratio $8:4:1$.
        \end{itemize}
    \end{enumerate}
]