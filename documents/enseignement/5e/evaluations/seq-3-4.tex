%%%
\setGrade{5e}
\evaluation{2}
[corr]
%%%
% \def\imgPath{enseignement/6e/}

\seqEvaluation{3}{Symétrie}{
    Comprendre l'effet des symétries (axiale et centrale) :
    conservation du parallélisme{,} des longueurs et des angles.
    /2,
    Identifier des symétries dans des frises{,} des pavages{,} des rosaces.
    /2%
}

\seqEvaluation{4}{Nombres Relatifs - Sommes et differences}{
    Traduire un enchaînement d'opérations à l'aide d'une expression avec des parenthèses.
    /3,
    Additionner et soustraire des nombres décimaux relatifs.
    /4,
    Résoudre des problèmes faisant intervenir des nombres décimaux relatifs et des fractions simples.
    / 2,
}

\seqEvaluation{}{Compétences générales}{
    Écrire ses calculs
    /1,
    Rédiger des phrases réponses
    /2
}

\exo{\tiersTemps Reconnaître des symétries}{
    \def\crossWidth{0.3mm}\def\crossSize{0.15}\def\nodeShift{0.25}
    \ctikz[1.25]{
        \draw[gray!40] (0,0) grid (6,6);
        % Nested loops to generate points with names
        \newcounter{i}\setcounter{i}{1}%
        \foreach \x in {1,...,5} {%
            \foreach \y in {1,...,5} {%
                % Generate names programmatically
                \pgfmathtruncatemacro{\charCode}{64+\thei} % Convert x to letter (A=1, B=2, etc.)
                \edef\pointName{\char\charCode} % Get the letter name
                \drawPoint{\pointName}{\x}{\y}%
                \stepcounter{i}%
            }
        }
    }
    \begin{enumerate}
        \item L'image du segment $[HR]$ par la symétrie de centre $N$ est : \awsr{le segment $[JT]$}.
        \item Le triangle $QUV$ est l'image de du triangle $SON$ par \awsr{la symetries de centre $R$}.
        \item Le point \awsr{$D$} est l'image de point P par la symetrie d'axe $(AG)$.
        \item L'image du quadrilatère $NXQL$ par la symétrie de centre $M$ est : \awsr{le quadrilatère $LBIN$}.
    \end{enumerate}
}[\sesa{5}{2024}[3][89]]

\exo{Utiliser les propriétés de la symétrie centrale}{
    On considère quatre points $P$, $O$, $K$ et $E$ tels que :  
    \begin{align*}
        KE = \Lg{10.1}, \quad PO = \Lg{3.6}, \quad EP = \Lg{2.6}, \quad PK = \Lg{5}.
    \end{align*}
    Les points $A$,$S$ et $H$ sont respectivement les images des points $P$,$E$ et $K$ par la symétrie centrale de centre $O$.
    Calculez le périmètre du triangle $ASH$.
}

% \begin{Geometrie}[TypeTrace="Schema"]
%     pair A,B,C;
%     A=u*(1,1);
%     B-A=u*(4,1);
%     C=rotation(A,B,-70);
%     trace polygone(A,B,C);
% \end{Geometrie}


\ctikz[0.75]{
    \draw[gray!40] (-1,-10) rectangle (18,2);
    \node at (3,1) {Schéma à main levée :};
    \ifthenelse{\boolean{answer}}{%
        \draw [penciline,thick] (0.48,-7.04)-- (7.4,-8.06);
        \draw [penciline,thick] (7.4,-8.06)-- (5.18,-4.68);
        \draw [penciline,thick] (5.18,-4.68)-- (0.48,-7.04);
        \draw [penciline,thick] (11.06,-3.16)-- (15.683121999999992,1.1780524000000023);
        \draw [penciline,thick] (15.683121999999992,1.1780524000000023)-- (9.047820799999997,0.5338484000000019);
        \draw [penciline,thick] (9.047820799999997,0.5338484000000019)-- (11.06,-3.16);
        \draw (6.825316999999991,-6.262503800000003) node[anchor=north west] {\Lg{2.6}};
        \draw (3.2821949999999944,-8.098485200000004) node[anchor=north west] {\Lg{10.1}};
        \draw (2.090417599999996,-5.038516200000002) node[anchor=north west] {\Lg{5}};
        \drawPoint{P}{5.18}{-4.68}
        \drawPoint{O}{8.16}{-4.22}
        \drawPoint{K}{0.48}{-7.04}
        \drawPoint{E}{7.40}{-8.06}
        \drawPoint{A}{11.06}{-3.16}
        \drawPoint{S}{9.05}{0.53}
        \drawPoint{H}{15.68}{1.18};
    }{}%
}


\answerFill[Réponse][
    \begin{itemize}
        \item On a : $PE + KE + KP = \Lg{2,6} + \Lg{10,1} + \Lg{5} = \Lg{17.7}$.
        \item Alors, le périmètre du triangle $PEK$ est de \Lg{17.7}.
        \item On sait que les points $A$, $S$ et $H$ sont respectivement les images des points $P$, $E$ et $K$ par la symétrie centrale de centre $O$.
        \item Alors, le triangle $ASH$ est l'image du triangle $PEK$ par cette symétrie.
        \item Or, la symétrie centrale conserve les longueurs.
        \item Alors, le périmètre du triangle $ASH$ est égal à celui du triangle $PEK$.
        \item Donc le périmètre du triangle $ASH$ est de \Lg{17.7}.
    \end{itemize}
]

\exo{Calculs}{
    \multiColEnumerate{1}{
        \item $\frac{81}{9} \times 5 -1
        = \awsr[2]{9\times 5 - 1 = 45 -1 = 44}$
        \item $\frac{45,5}{2\times3-1}
        = \awsr[2]{\frac{45,5}{6-1} = \frac{45,5}{5} = 9,1}$
        % \item $\frac{27}{2\times3}-1
        % = \awsr[3]{\frac{27}{6} -1 = 4,5 - 1}$
        % \item $\frac{17-5}{3}+2
        % = \awsr[3]{\frac{12}{3}+2 = 4 + 2 = 6}$
        \item $7\times\frac{15\times4}{3-2}+2\times8
        = \awsr[2]{7 \times \frac{60}{6}+ 16 = 7 \times 10 + 16 = 70 + 16 = 86}$
        \item $7 - (-\np{12}) = \awsr[2]{19}$
        \item $-9 + 6 - 78 = \awsr[2]{-3 - 78 = -81}$
        % \item $12 + (-\np{22,6}) = \awsr[2]{-10,6}$
        \item $39 + (7 - 18 + (-1)) = \awsr[2]{39 + (-11 + (-1)) = 39 + (-12) = 27}$
    }
}[\sesa{5}{2024}[10][41]]

% \exo{}{
%     \begin{enumerate}

%     \end{enumerate}
% }

\exo{La randonnée d'Élodie}{
    Élodie effectue une randonnée en montagne qui se déroule en plusieurs étapes :  
    \begin{itemize}
        \item Depuis son point de départ situé à \Lg{1800} d'altitude, elle grimpe de \Lg{560} pour atteindre un premier refuge.  
        \item Après une pause, elle descend de \Lg[km]{1,3} pour visiter un lac de montagne.  
        \item Ensuite, elle remonte de \Lg{230} pour reprendre le chemin principal.  
        \item Elle rejoint un second refuge situé à \Lg[km]{1,1} plus haut.  
        \item Enfin, le dernier jour, elle redescend de \Lg{860} jusqu'à un arrêt de bus.  
    \end{itemize}

    \begin{enumerate}
        \item Quelle est l'altitude maximale atteinte par Élodie pendant sa randonnée ? Et l'altitude minimale ?  
        \item Quel est le dénivelé total de sa randonnée, c'est-à-dire la différence d'altitude entre son point de départ et son point d’arrivée ?  
    \end{enumerate}
}

\answerFill[Réponse][]