%%%
\setGrade{5e}
\evaluation{2}
% [corr]
%%%
% \def\imgPath{enseignement/6e/}

\seqEvaluation{3}{Symétrie}{
    Comprendre l'effet des symétries (axiale et centrale) :
    conservation du parallélisme{,} des longueurs et des angles.
    /3,
    Reconnaître deux figures image l'une de l'autre par symetrie.
    /3,
    % Identifier des symétries dans des frises{,} des pavages{,} des rosaces.
    % /2%
}

\seqEvaluation{4}{Nombres Relatifs - Sommes et differences}{
    Traduire un enchaînement d'opérations à l'aide d'une expression avec des parenthèses.
    /3,
    Additionner et soustraire des nombres décimaux relatifs.
    /3,
    Résoudre des problèmes faisant intervenir des nombres relatifs.
    /2,
    Respecter l'ordre des priorités opératoires.
    /2,
    Raisonner avec des nombres relatifs.
    /0,
}

% \seqEvaluation{}{Compétences générales}{
%     Écrire ses calculs
%     /1,
%     Rédiger des phrases réponses
%     /2
% }

\evalutionEnd[2][2]

\exo{Reconnaître des symétries}{
    \def\crossWidth{0.3mm}\def\crossSize{0.15}\def\nodeShift{0.25}
    \ctikz[1.1]{
        \draw[gray!40] (0,0) grid (6,6);
        % Nested loops to generate points with names
        \newcounter{i}\setcounter{i}{1}%
        \foreach \x in {1,...,5} {%
            \foreach \y in {1,...,5} {%
                % Generate names programmatically
                \pgfmathtruncatemacro{\charCode}{64+\thei} % Convert x to letter (A=1, B=2, etc.)
                \edef\pointName{\char\charCode} % Get the letter name
                \drawPoint{\pointName}{\x}{\y}%
                \stepcounter{i}%
            }
        }
    }
    \begin{enumerate}
        \item L'image du segment $[HR]$ par la symétrie de centre $N$ est : \awsr[2]{le segment $[JT]$}.
        \item Le triangle $QUV$ est l'image de du triangle $SON$ par \awsr[2]{la symetries de centre $R$}.
        \item Le point \awsr{$D$} est l'image de point P par la symetrie d'axe $(AG)$.
        \item L'image du quadrilatère $NXQL$ par la symétrie de centre $M$ est : \awsr[2]{le quadrilatère $LBIN$}.
    \end{enumerate}
}[\sesa{5}{2024}[3][89]]

\exo{Utiliser les propriétés de la symétrie centrale}{
    On considère quatre points $P$, $O$, $K$ et $E$ tels que :  
    \begin{align*}
        KE = \Lg{10.1}, \quad PO = \Lg{3.6}, \quad EP = \Lg{2.6}, \quad PK = \Lg{5}.
    \end{align*}
    Les points $A$,$S$ et $H$ sont respectivement les images des points $P$,$E$ et $K$ par la symétrie centrale de centre $O$.
    Calculez le périmètre du triangle $ASH$.
}

% \begin{Geometrie}[TypeTrace="Schema"]
%     pair A,B,C;
%     A=u*(1,1);
%     B-A=u*(4,1);
%     C=rotation(A,B,-70);
%     trace polygone(A,B,C);
% \end{Geometrie}


\ctikz[0.75]{
    \draw[gray!40] (-1,-10) rectangle (18,2);
    \node at (3,1) {Schéma à main levée :};
    \ifthenelse{\boolean{answer}}{%
        \draw [penciline,thick] (0.48,-7.04)-- (7.4,-8.06);
        \draw [penciline,thick] (7.4,-8.06)-- (5.18,-4.68);
        \draw [penciline,thick] (5.18,-4.68)-- (0.48,-7.04);
        \draw [penciline,thick] (11.06,-3.16)-- (15.683121999999992,1.1780524000000023);
        \draw [penciline,thick] (15.683121999999992,1.1780524000000023)-- (9.047820799999997,0.5338484000000019);
        \draw [penciline,thick] (9.047820799999997,0.5338484000000019)-- (11.06,-3.16);
        \draw (6.825316999999991,-6.262503800000003) node[anchor=north west] {\Lg{2.6}};
        \draw (3.2821949999999944,-8.098485200000004) node[anchor=north west] {\Lg{10.1}};
        \draw (2.090417599999996,-5.038516200000002) node[anchor=north west] {\Lg{5}};
        \drawPoint{P}{5.18}{-4.68}
        \drawPoint{O}{8.16}{-4.22}
        \drawPoint{K}{0.48}{-7.04}
        \drawPoint{E}{7.40}{-8.06}
        \drawPoint{A}{11.06}{-3.16}
        \drawPoint{S}{9.05}{0.53}
        \drawPoint{H}{15.68}{1.18};
    }{}%
}


\answerFill[Réponse][
    \begin{itemize}
        \item On a : $PE + KE + KP = \Lg{2,6} + \Lg{10,1} + \Lg{5} = \Lg{17.7}$.
        \item Alors, le périmètre du triangle $PEK$ est de \Lg{17.7}.
        \item On sait que les points $A$, $S$ et $H$ sont respectivement les images des points $P$, $E$ et $K$ par la symétrie centrale de centre $O$.
        \item Alors, le triangle $ASH$ est l'image du triangle $PEK$ par cette symétrie.
        \item Or, la symétrie centrale conserve les longueurs.
        \item Alors, le périmètre du triangle $ASH$ est égal à celui du triangle $PEK$.
        \item Donc le périmètre du triangle $ASH$ est de \Lg{17.7}.
    \end{itemize}
]

\exo{Calculs}{
    \multiColEnumerate{1}{
        \item $\frac{81}{9} \times 5 -1
        = \awsr[3]{9\times 5 - 1 = 45 -1 = 44}$
        % \item $\frac{45,5}{2\times3-1}
        % = \awsr[2]{\frac{45,5}{6-1} = \frac{45,5}{5} = 9,1}$
        % \item $\frac{27}{2\times3}-1
        % = \awsr[3]{\frac{27}{6} -1 = 4,5 - 1}$
        % \item $\frac{17-5}{3}+2
        % = \awsr[3]{\frac{12}{3}+2 = 4 + 2 = 6}$
        \item $7\times\frac{15\times4}{3-2}+2\times8
        = \awsr[2]{7 \times \frac{60}{6}+ 16 = 7 \times 10 + 16 = 70 + 16 = 86}$
        \item $7 - (-\np{12}) = \awsr[2]{19}$
        \item $-9 + \np{6.3} - 78 = \awsr[2]{- \np{3.7} - 78 = - \np{81.7}}$
        % \item $12 + (-\np{22,6}) = \awsr[2]{-10,6}$
        \item $39 + (7 - 18 + (-1)) = \awsr[3]{39 + (-11 + (-1)) = 39 + (-12) = 27}$
    }
}[\sesa{5}{2024}[10][41]]

% \exo{}{
%     \begin{enumerate}

%     \end{enumerate}
% }

\exo{\tiersTemps La randonnée d'Élodie}{
    Élodie effectue une randonnée en montagne qui se déroule en plusieurs étapes :  
    \begin{itemize}
        \item Depuis son point de départ situé à \Lg[m]{1800} d'altitude, elle grimpe de \Lg[m]{340} pour atteindre un premier refuge.  
        \item Après une pause, elle descend de \Lg[km]{1,3} pour visiter un lac de montagne.  
        \item Ensuite, elle remonte de \Lg[m]{230} pour reprendre le chemin principal.  
        \item Elle rejoint un second refuge situé à \Lg[km]{0,9} plus haut.  
        \item Enfin, le dernier jour, elle redescend de \Lg[m]{860} jusqu'à un arrêt de bus.  
    \end{itemize}
    Quel est le dénivelé total de sa randonnée, c'est-à-dire la différence d'altitude entre son point de point d'arrivée et son point de départ ?
    \\\hint{Un dénivelé peut être positif ou négatif.}
    % \begin{enumerate}
    %     % \item Quelle est l'altitude maximale atteinte par Élodie pendant sa randonnée ? Et l'altitude minimale ?  
    %     \item 
    % \end{enumerate}
}

\answerFill[Réponse][
    \multiColItemize{1}{
        \item $1800 + 340 - 1300 + 230 + 900 - 860 = 1110$
        \item $1110 - 1800 = -690$
        Le dénivelé total de sa randonnée est de \Lg[m]{-690}
    }
]

\def\a{\cir[Red]{a}} \def\b{\cir[Green]{b}} \def\c{\cir[Blue]{c}} \def\d{\cir[violet]{d}} \def\e{\cir[Orange]{e}}
\def\f{\cir[violet]{5}} \def\t{\cir[Orange]{-3}}
\exo{\bonus Jeu de jetons}{
    Un sac contient des jetons,
    chacun portant une lettre sur une face
    (\a, \b, \c, \d ou \e)
    et un nombre relatif sur l'autre face.
    Chaque lettre correspond à un unique nombre relatif.
    Le jeu consiste à tirer plusieurs jetons,
    et ajouter les nombres associés,
    Le joueur ayant le score le plus élevé remporte la partie.
    Les tirages des joueurs sont donnés ci-dessous avec les valeurs des jetons \d et \e déjà visibles :
    \multiColItemize{2}{
        \item Lili : \a \b \c
        \item Mattéo : \a \f \b \f \c
        \item Loane : \b \t \c \a \b
        \item Antoine :\a \c \a \b \c \b
        \item Nina : \a \f \c \t \b \t \a \c
    }
    On sait que le score de Lili est -6.

    \begin{enumerate}
        \item Pour chaque joueur, déterminez si leur score est calculable.
        Si oui, donnez leur score.
        Sinon, justifiez pourquoi leur score reste indéterminable.
        \item Sachant que les valeurs des lettres sont comprises entre -9 et 9,
        est-il possible d'identifier le vainqueur avec certitude ?
        \item Quelle est la place la plus haute qu'Antoine pourrait atteindre ?
    \end{enumerate}
}[Inspiré de \mi]

\answerFill[Réponse][
    \begin{enumerate}
        \item L'adition est commutative (c'est à dire $\a+\b=\b+\a$),
        on peut alors ajouter les points des jetons dans n'importe quel ordre et on obtiendra toujours la meme chose.
        On va s'en servire pour essayer de determiner les scores de chacuns des joueurs.
        \multiColItemize{1}{
            \item Lili : $\a + \b + \c
            = -6$
            \item Mattéo : $\a + \f + \b + \f + \c
            = (\a  + \b + \c) + \f + \f
            = -6 + 10
            = 4$
            \item Loane : $\b + \t + \c + \a + \b
            = (\a  + \b + \c) + \t + \b
            = -6 + \t + \a
            = -9 + \a$
            \item Antoine : $\a + \c + \a + \b + \c + \b
            = (\a  + \b + \c) + (\a  + \b + \c)
            = -6 + -6 = -12$
            \item Nina : $\a + \f + \c + \t + \b + \t + \a + \c
            = (\a  + \b + \c) + \f + \t + \t + \a + \c
            = -6 + \f + \t + \t + \a + \c
            = -7 + \a + \c$
        }
        On ne connait pas les valeurs de \a et \c et on ne peut donc pas connaitre le score de Loane et Nina.
        \item \multiColItemize{1}{
            \item Max(Loane) $= -9 + 9 = 0$ si $\a = 9$
            \item Max(Nina) $= -7 + -6 + 9 = -4$ si $\b = -9$, $\a = 9$ et $\c = -6$ (on a bien $9 + -9 + -6 = -6$)
        }
        Mattéo est forcément le vainceur car aucun autre joeur ne peut atteindre un score de $4$.
        \item \multiColItemize{1}{
            \item Min(Loane) $= -9 + -9 = -18$ si $a = -9$
            \item Max(Nina) $= -7 + -9 + -6 = -22$ si $\b = 9$ $\a = -9 et \c = -6$ (on a bien $-9 + 9 + -6 = -6$)
        }
        Antoine pourrait être troisième si Loane et Nina atteigne un score inferieur a $-12$ avec par exemple $\a = -9$ et $\c = -6$.
    \end{enumerate}
]

