\setSeq{6}{Espace - Solides}
\setGrade{5e}

\def\ym{https://www.maths-et-tiques.fr/telech/19Solides5e.pdf}

\obj{
    \item Reconnaître des solides (pavé droit, cube, cylindre, prisme droit, pyramide, cône, boule) à partir
    d'un objet réel, d'une image, d'une représentation en perspective cavalière.
    \item Construire et mettre en relation une représentation en perspective cavalière et un patron d'un pavé droit,
    d'un cylindre.
    \item Calculer le volume d'un pavé droit, d'un prisme droit, d'un cylindre.
    \item Calculer le volume d'un assemblage de ces solides.
    \item Effectuer des conversions d'unités de longueurs, d'aires, de volumes.
}

\df{}{
    On appel \key{polyhèdre} un solide qui possède des \key{faces} polygonales
}{\wiki{Polyèdre}}

\df{}{
    On appel \awsr{\key{prisme droit}} un polyhèdre délimité par 2 \key{bases} superposable reliés entrelles par des rectangles.
}{\wiki{Prisme_(solide)}}

\df{}{
    On appel \awsr{\key{pavé droit}} un \awsr{prisme droit} dont les bases sont rectangulaires.
}{\wiki{Pavé_droit}}

\df{}{
    On appel \awsr{\key{cube}} un \awsr{pavé droit} dont les bases sont carré.
}{\wiki{Cube}}

\df{}{
    On appel \awsr{\key{pyramide}} un polyhèdre formé d'une base polygoneale relié à un \key{sommet} par des faces \awsr{\key{triangulaires}}.
}{\wiki{Pyramide}}

\df{}{
    On appel \awsr{\key{cylindre}} un solide formé de deux base circulaire superposable relié par une surface courbe.
}{\wiki{Cylindre}}

\df{}{
    On appel \awsr{\key{cône}} un solide formé d'une base circulaire rélié à un point superposable au centre de la base par un sommet.
}{\wiki{Cône}}



% \slide{cr}{
%     \ctikz[1]{
%         \dotGrid[10][16][0.5pt][0.5][(10,-5)]
%         \fill[thick,color=gradeColor,fill=gradeColor,fill opacity=0.10] (6,-4) -- (9,-4) -- (8.50,-4.50) -- (7.50,-4.50) -- (7,-5) -- (5,-5) -- cycle;
%         \fill[thick,color=gradeColor,fill=gradeColor,fill opacity=0.10] (14,-6) -- (13,-7) -- (15,-7) -- (15.50,-6.50) -- (14.50,-6.50) -- (15,-6) -- cycle;
%         \fill[thick,color=gradeColor,fill=gradeColor,fill opacity=0.10] (7,-6) -- (9,-6) -- (7.50,-7.50) -- (6.50,-7.50) -- (7,-7) -- (5,-7) -- (5.50,-6.50) -- (6.50,-6.50) -- cycle;
%         \fill[thick,color=gradeColor,fill=gradeColor,fill opacity=0.10] (10,-6) -- (12.51,-6.01) -- (12.01,-6.51) -- (11.50,-6.50) -- (11,-7) -- (9,-7) -- cycle;
%         \fill[thick,color=gradeColor,fill=gradeColor,fill opacity=0.10] (10,-4) -- (13,-4) -- (12,-5) -- (9,-5) -- cycle;
%         \fill[thick,color=gradeColor,fill=gradeColor,fill opacity=0.10] (15,-4) -- (16,-4) -- (15.50,-4.50) -- (14.50,-4.50) -- cycle;
%         \node at (6.38, 3.99) {\cir[gradeColor]{1}};
%         \node at (10.42, 4.73) {\cir[gradeColor]{2}};
%         \node at (13.53, 3.7) {\cir[gradeColor]{3}};
%         \node at (4.93, 0.19) {\cir[gradeColor]{4}};
%         \node at (9.4, -1.35) {\cir[gradeColor]{5}};
%         \node at (14.41, -0.36) {\cir[gradeColor]{6}};
%         \draw [thick] (5,-2) -- (6,-2) -- (6.50,-1.50) -- (6.50,-0.50) -- (6,-1) -- (5,-1) -- (4,-1) -- (4,-2) -- (5,-2) -- cycle;
%         \draw [thick] (6,-2) -- (6,-1);
%         \draw [thick] (6.50,-0.50) -- (7.50,-0.50) -- (7.50,-1.50) -- (6.50,-1.50);
%         \draw [thick] (4,-1) -- (4.50,-0.50) -- (5.50,-0.50) -- (6,0) -- (7,0) -- (8,0) -- (8,-1) -- (7.50,-1.50);
%         \draw [thick] (4.50,-0.50) -- (4.50,0.50) -- (5.50,0.50) -- (6,1) -- (6,0);
%         \draw [thick] (5.50,-0.50) -- (5.50,0.50);
%         \draw [thick] (6,1) -- (5,1) -- (4.50,0.50);
%         \draw [thick] (7.50,-0.50) -- (8,0);
%         \draw [thick] (10,-2) -- (11,-2) -- (11.50,-1.50) -- (11.50,-0.50) -- (10.50,-0.50) -- (11,0) -- (10,0) -- (9.50,-0.50) -- (9,-1) -- (10,-1) -- (11,-1) -- (11.50,-0.50);
%         \draw [thick] (11,-1) -- (11,-2);
%         \draw [thick] (9,-1) -- (9,-2) -- (10,-2);
%         \draw [thick] (11,0) -- (11,-0.50);
%         \draw [thick,gradeColor] (6,-4) -- (9,-4) -- (8.50,-4.50) -- (7.50,-4.50) -- (7,-5) -- (5,-5) -- (6,-4) -- cycle;
%         \draw [thick,gradeColor] (14,-6) -- (13,-7) -- (15,-7) -- (15.50,-6.50) -- (14.50,-6.50) -- (15,-6) -- (14,-6) -- cycle;
%         \draw [thick] (14,0) -- (16,0) -- (17,1) -- (17,-1) -- (16,-2) -- (15.50,-2.50) -- (15.50,-1.50) -- (16,-1) -- (15,-1) -- (14.50,-1.50) -- (15.50,-1.50);
%         \draw [thick] (16,0) -- (16,-1);
%         \draw [thick] (17,1) -- (15,1) -- (14,0) -- (14,-1) -- (13,-1) -- (13,-2) -- (14,-2) -- (14.50,-2);
%         \draw [thick] (13,-1) -- (13.50,-0.50) -- (14,-0.50);
%         \draw [thick] (14.50,-1.50) -- (14.50,-2.50) -- (15.50,-2.50);
%         \draw [thick,gradeColor] (7,-6) -- (9,-6) -- (7.50,-7.50) -- (6.50,-7.50) -- (7,-7) -- (5,-7) -- (5.50,-6.50) -- (6.50,-6.50) -- (7,-6) -- cycle;
%         \draw [thick] (10,3) -- (11,3) -- (11.50,3.50) -- (11.50,6.50) -- (11,6) -- (10,6) -- (10,3) -- cycle;
%         \draw [thick] (10,6) -- (10.50,6.50) -- (11.50,6.50);
%         \draw [thick] (11,3) -- (11,6);
%         \draw [thick] (12.50,3) -- (14.50,3) -- (14,5) -- (12.50,3) -- cycle;
%         \draw [thick] (14,5) -- (15,4.33);
%     }
% }