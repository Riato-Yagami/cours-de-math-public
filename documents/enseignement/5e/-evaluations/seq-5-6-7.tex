\setGrade{5e}
\evaluation{3}
[corr]
\seqEvaluation{5}{Angles et parallélisme}{
    Identifier des configurations d'angles particuliers.
    /2,
    Démontrer que deux droites sont parallèles.
    /2,
    Utiliser les propriétés des triangles.
    /2,
    Effectuer des raisonnements basés sur les caractéristiques angulaires du parallélisme.
    /2%
}

\seqEvaluation{6}{Géométrie dans l'espace - solides}{
    Identifier des solides particuliers.
    /2,
    Dessiner un solide en perspective cavalière.
    /1,
    Construire le patron d'un cône.
    /0%
}

\seqEvaluation{7}{Gestion de données - moyenne et médiane}{
    Calculer la moyenne d'une série de données.
    /2,
    Calculer la médiane d'une série de données.
    /2,
    Receuillir des données sous forme de graphiques.
    /1%
}

\evalutionEnd[2][2]

\calculator

\exo{Reconnaître et tracer des solides}{%
    \dividePage{%
        \ctikz[0.75]{
    \boundingBox[9][10][1pt][1][(0,0)][cavalier]
    \node at (1.34, 7.92) {\cir[gradeColor]{1}};
    \node at (1.36, 1.90) {\cir[gradeColor]{2}};
    \node at (6.72, 2.52) {\cir[gradeColor]{3}};
    \drawPoint{A}{5}{6}
    \drawPoint{B}{5}{8}
    \draw [thick] (0,3) -- (0,0) -- (3,0) -- (3,2) -- (2,2) -- (3,3) -- (3,4) -- (1,4) -- (0,3) -- (2,3) -- (2,2);
    \draw [thick] (3,4) -- (2,3);
    \draw [thick] (3,3) -- (4,3) -- (3,2);
    \draw [thick] (4,3) -- (4,1) -- (3,0);
    \draw [thick,dashed] (1,4) -- (1,1);
    \draw [thick,dashed] (1,1) -- (0,0);
    \draw [thick,dashed] (1,1) -- (4,1);
    \draw [thick] (6,5) -- (5,4) -- (8,4) -- (9,5) -- (6,5);
    \draw [thick] (5,4) -- (5,0) -- (8,0) -- (9,1) -- (9,5);
    \draw [thick] (8,0) -- (8,4);
    \draw [thick,dashed] (6,5) -- (6,1);
    \draw [thick,dashed] (6,1) -- (5,0);
    \draw [thick,dashed] (6,1) -- (9,1);
    \draw [thick] (1,6) -- (3,6) -- (1,10) -- (1,6) -- (0,7) -- (1,10) -- (4,7) -- (3,6);
    \draw [thick,dashed] (0,7) -- (4,7);
    \nswr[0]{
        \draw [thick] (5,6) -- (5,8);
        \draw [thick] (5,8) -- (6,9);
        \draw [thick,dashed] (6,9) -- (6,7);
        \draw [thick,dashed] (6,7) -- (8,7);
        \draw [thick,dashed] (5,6) -- (6,7);
        \draw [thick] (8,7) -- (7,6);
        \draw [thick] (7,6) -- (5,6);
        \draw [thick] (8,7) -- (8,9);
        \draw [thick] (8,9) -- (6,9);
        \draw [thick] (5,8) -- (7,8);
        \draw [thick] (7,8) -- (7,6);
        \draw [thick] (7,8) -- (8,9);
    }
}
    }{%
        \begin{enumerate}
            \item À quelle famille commune appartiennent tous ces solides ?
            \\\nswr[2]{Ils appartiennent à la famille des polyèdres.}
            \item À quelle famille spécifique appartient chaque solide ?
            \\\nswr[5]{Le solide \cir[gradeColor]{1} est une pyramide, le \cir[gradeColor]{2} est un prisme droit,
            et le solide \cir[gradeColor]{3} est un pavé droit.}
            \item Tracez un cube en perspective cavalière dont l'une des arêtes est le segment $[AB]$ (avec les arêtes cachées en pointillés).
        \end{enumerate}
    }[0.45]
}

\exo{\ttps Étude des ventes de Minecraft}{
    Le graphique ci-dessous présente les ventes de Minecraft de 2011 à 2023 :
    \begin{center}
    \Stat[% 
        Qualitatif,
        Graphique,
        Donnee=Année,
        Effectif=Ventes (en millions),
        Unitex=1,AngleRotationAbscisse=60,
        Unitey=0.15,Pasy=4,
        Grille,PasGrilley=2,LectureFine,
        CouleurDefaut = ForestGreen,
        EpaisseurBatons=2.5,
        Origine=2010,
    ]{%
        2011/4,2012/5,2013/24,2014/21,2015/18,2016/28,2017/22,2018/32,2019/22,2020/24,2021/42,2022/33,2023/25%
    }
    \end{center}
    \begin{enumerate}
        \item Calculez la moyenne annuelle des ventes de Minecraft entre 2011 et 2023, arrondie au million près.
        \nswr[12]{
            \begin{itemize}
                \item Les ventes annuelles sont : 4, 5, 24, 21, 18, 28, 22, 32, 22, 24, 42, 33, 25.
                \item La somme des ventes est :
                \\ $4 + 5 + 24 + 21 + 18 + 28 + 22 + 32 + 22 + 24 + 42 + 33 + 25 = 300$.
                \item Le nombre d'années est : 13.
                \item La moyenne est donc : $\text{Moyenne} = \frac{\text{Somme des valeurs}}{\text{Effectif total}} = \frac{300}{13} \approx 23$
                \item La moyenne annuelle des ventes est donc de 23 millions.
            \end{itemize}
        }
        \item Déterminez la médiane des ventes de Minecraft sur la période donnée.
        \nswr[\remaininglines]{
        \begin{itemize}
            \item On classe les ventes (en millions) par ordre croissant :
            \\ $4 < 5 < 18 < 21 < 22 < 22 < 24 < 24 < 25 < 28 < 32 < 33 < 42$.
            \item L'effectif total étant de 13 (nombre impair), la médiane est la valeur centrale, soit la $7^e$ valeur : 24 millions.
        \end{itemize}
        }
    \end{enumerate}
}[\href{https://www.businessofapps.com/data/minecraft-statistics/}{Business of Apps}]

\exo{Trouver des angles}{
    \ctikz[0.75]{
    \boundingBox[15.81][12.24][0.5pt][1][(-14.26,-6.16)]
    \draw [shift={(-10.90,0.74)},thick,color=gradeColor,fill=gradeColor,fill opacity=0.10] (0,0) -- (-61.47:0.60) arc (-61.47:-12.53:0.60) -- cycle;
    \draw [shift={(-8.48,3.92)},thick,color=gradeColor,fill=gradeColor,fill opacity=0.10] (0,0) -- (-12.53:0.60) arc (-12.53:117.41:0.60) -- cycle;
    \draw[thick,color=gradeColor,fill=gradeColor,fill opacity=0.10] (-2.42,2.14) -- (-2.00,2.05) -- (-1.91,2.46) -- (-2.32,2.55) -- cycle;
    \draw[thick,color=gradeColor,fill=gradeColor,fill opacity=0.10] (-3.20,-1.41) -- (-2.79,-1.50) -- (-2.70,-1.09) -- (-3.11,-0.99) -- cycle;
    \drawPoint{J}{-2.32}{2.55}
    \drawPoint{I}{-3.11}{-0.99}
    \drawPoint{A}{-13.20}{4.97}
    \drawPoint{B}{-8.48}{3.92}
    \drawPoint{C}{-6.29}{-0.29}
    \drawPoint{D}{-10.90}{0.74}
    \draw[color=gradeColor] (-10.04,0.27) node {$49\textrm{\degre}$};
    \draw[color=gradeColor] (-7.72,4.71) node {$130\textrm{\degre}$};
    \drawPoint{H}{-3.96}{-4.79}
    \drawPoint{E}{-9.11}{5.14}
    \drawPoint{F}{-12.74}{1.15}
    \drawPoint{G}{-9.34}{-2.13}
    \draw [thick] (-1.70,5.36) -- (-4.26,-6.16);
    \draw [thick] (-9.16,5.24) -- (-3.30,-6.06);
    \draw [thick] (-13.80,6.08) -- (-9.18,-2.42);
    \draw [thick] (-14.26,5.21) -- (1.55,1.69);
    \draw [thick] (-0.87,-1.49) -- (-12.99,1.20);
}
    \begin{enumerate}
        \item Trouver la mesure de l'angle $\widehat{ICH}$.\\
        \nswr[\remaininglines]{%
            \begin{itemize}
                \item L'angle $\widehat{EBC}$ est un angle plat.
                \item On a $\widehat{EBC} = \widehat{EBJ} + \widehat{JBC}$.
                \item Donc, $\widehat{JBC} = \widehat{EBC} - \widehat{EBJ} = \ang{180} - \ang{130} = \ang{50}$.
                \item Les angles $\widehat{JBC}$ et $\widehat{ICH}$ sont correspondants.
                \item Les droites $(AJ)$ et $(FI)$ sont parallèles car elles sont toutes deux perpendiculaires à la même droite $(JI)$.
                \item Deux angles correspondants formés par deux droites parallèles sont égaux.
                \item Donc, $\widehat{ICH} = \widehat{JBC} = \ang{50}$.
            \end{itemize}
        }
        \item Les droites $(AD)$ et $(BC)$ sont-elles parallèles?
        \nswr[13]{%
            \begin{itemize}
                \item Les angles $\widehat{ICH}$ et $\widehat{GDC}$ sont correspondants.
                \item Cependant, $\widehat{ICH} \neq \widehat{GDC}$.
                \item Deux angles correspondants formés par deux droites parallèles sont égaux.
                \item Donc, les droites $(AD)$ et $(BC)$ ne peuvent pas être parallèles, sinon $\widehat{ICH}$ et $\widehat{GDC}$ seraient égaux.
            \end{itemize}
        }
        \item Trouver la mesure de l'angle $\widehat{CHI}$.
        \nswr[13]{%
            \begin{itemize}
                \item La somme des angles dans un triangle est de $\ang{180}$.
                \item Donc, $\widehat{ICH} + \widehat{CHI} + \widehat{CIH} = \ang{180}$.
                \item Ainsi, $\ang{50} + \widehat{CHI} + \ang{90} = \ang{180}$.
                \item Donc, $\ang{140} + \widehat{CHI} = \ang{180}$.
                \item Par conséquent, $\widehat{CHI} = \ang{180} - \ang{140} = \ang{40}$.
            \end{itemize}
        }
    \end{enumerate}
}

\exo{\bonus Patron du cône}{
    On souhaite construire un patron d'un cône dont la génératrice mesure \Lg{5} et le rayon de la base \Lg{1.5}.
    
    \dividePage{
        \newcommand*{\ArcAngle}{108}%
\newcommand*{\ArcRadius}{5.0}%
\ctikz[0.75]{
    \draw [thick] (-0.11,-0.98) circle (1.50cm);
    \draw [rotate around={-178.73:(-8.72,-3.26)},thick] (-8.72,-3.26) ellipse (1.50cm and 0.75cm);
    % \draw [rotate around={-178.73:(-8.72,-3.26)},thick,dashed] (-8.72,-3.26) ellipse (1.50cm and 0.75cm);
    \draw [thick,gradeColor] ($(-5.99,-3.74)+(\ArcRadius,0)$) 
    arc (0:\ArcAngle:\ArcRadius);
    \drawPoint{O}{-5.99}{-3.74}
    \drawPoint{A}{-0.99}{-3.75}
    \drawPoint{B}{-7.53}{1.02}
    \draw[color=gradeColor] (-9.72,-0.89) node {$\Lg{5}$};
    \draw[color=gradeColor] (-9.44,-3.03) node {$\Lg{1.50}$};
    \draw [thick,gradeColor] (-8.72,1.51) -- (-10.22,-3.26);
    \draw [thick] (-8.72,1.51) -- (-7.22,-3.26);
    \draw [thick,dashed,gradeColor] (-8.72,-3.26) -- (-10.22,-3.26);
    \draw [thick,dashed] (-5.99,-3.74) -- (-0.11,-0.98);
    \draw [thick] (-7.53,1.02) -- (-5.99,-3.74) -- (-0.99,-3.75);
}
    }{
        \begin{enumerate}
            \item Calculer le périmètre de la base du cône.
            \item Quelle est la longueur de l'arc de cercle $\overset{\frown}{AB}$ ?
            \item Calculer le périmètre du cercle ayant pour rayon la génératrice.
            \item Déterminer la mesure de l'angle $\widehat{AOB}$.
            \item Construire un patron du cône.
        \end{enumerate}
    }
}[\href{http://chica.chevere.free.fr/quatrieme/14/a14.pdf}{B. TRUCHETET}][\cmdGeoGebra[pstm9puc]]

\nswr[0]{
    \begin{enumerate}
        \item Le périmètre $P$ de la base du cône est donné par la formule du périmètre d'un cercle : $P = 2 \pi r$.
        Ici, $r = \Lg{1.5}$,
        donc $P = 2 \pi \times 1,5\Lg{} = 3 \pi\Lg{}$.
        \item La longueur de l'arc de cercle $\overset{\frown}{AB}$ correspondant à la surface conique est égale au périmètre de la base du cône, soit $3 \pi\Lg{}$.
        \item Le périmètre $P'$ du cercle ayant pour rayon la génératrice est donné par la formule du périmètre d'un cercle :
        $P' = 2 \pi R$. Ici, $R = \Lg{5}$, donc $P' = 2 \pi \times 5 = 10 \pi$.
        \item La mesure de l'angle $\widehat{AOB}$ est déterminée par la proportion entre le périmètre de la base du cône et le périmètre du cercle ayant pour rayon la génératrice.
        Donc, $\widehat{AOB} 
        = \dfrac{P}{P'} \times 360^\circ
        = \dfrac{3 \pi}{10 \pi} \times 360^\circ
        = \dfrac{3}{10} \times 360^\circ
        = 108^\circ$.
        \item Pour construire un patron du cône, on dessine un secteur circulaire de rayon \Lg{5} et d'angle $108^\circ$, puis on découpe et on assemble ce secteur avec un cercle de rayon 1,5 cm pour former le cône.
    \end{enumerate}
}

\newpage