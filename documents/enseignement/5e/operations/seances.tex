% VARIABLES %%%
\def\authors{\href{https://juels.dev/}{Jules PESIN}}
% \date{\today}
\def\longTitle{Opérations}
\setcounter{seq}{1}
\bseq{\longTitle}

\setboolean{showRef}{false}

\def\my{Myriade 5e}
\newcommand{\myl}[1]{\href{#1}{\my}}

\def\imgPath{enseignement/5e/operations/}
\def\imgExtension{.png}
%%

% Yvan Monka : https://www.maths-et-tiques.fr/telech/19Calcul_num.pdf

\qf{
    {$3+10-7$,$=6$},
    {$5{,}5-3+2$,$=4{,}5$},
    {$4+6\times2$,$= 16$},
    {$8\div2+8$,$= 12$},
    {$2\times(5+3)$,$= 16$}%
}[10]

\bsec{Ordre opératoires}
\bsubsec{Sans parenthèse}

\slide{COURS}{
    \sseq\ssec\ssubsec

    \rl{}{
        Les calculs se font dans l’ordre des priorités suivant:%
        \begin{enumerate}
            \item La multiplication et la division
            \item L'addition et la soustraction
        \end{enumerate}
    }
}

\slide{}{
    \rl{}{
        En cas d’opérations de mêmes priorités, on effectue les opérations de gauche à droite.
    }
}

\bsubsec{Avec parenthèse}
\slide{}{
    \ssubsec
    \rl{}{
        On commence par effectuer les calculs entre parenthèses.
    }
}

\bsec{Vocabulaire opératoires}
\slide{}{
    \ssec
    \vc{}{
        On connait quatres types d'opérations:
        \begin{itemize}
            \item L'addition permet de calculer la \key{somme} de deux \key{termes}.
            \item La soustraction  permet de calculer la \key{différence} entre deux \key{termes}.
            \item La multiplication permet de calculer la \key{produit} de deux \key{facteurs}.
            \item La division permet de calculer la \key{quotient} de deux \key{nombres}.
        \end{itemize}
    }
}

\slide{}{
    \pr{}{Dans un calculs,
    la dernière opérations détermine le nom pour désigner le calcul en entier.}
}