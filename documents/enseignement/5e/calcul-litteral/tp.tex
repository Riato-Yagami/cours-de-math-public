% https://clg-monnet-briis.ac-versailles.fr/IMG/pdf/tp_programme_de_calcul_1_1_.pdf

\setGrade{5e}
\tp{\Scratch - Programmes de calcul}
% [corr]
% DOC https://ctan.math.illinois.edu/macros/latex/contrib/scratch3/scratch3-fr.pdf

\setscratch{scale=.75}

\def\block{{\setscratch{scale=.5}\begin{scratch}\blockmove{\Large bloc}\end{scratch} }}

\newcommand{\scr}[1]{\begin{scratch}#1\end{scratch}}

\definecolor{smotion}{HTML}{4C97FF} % #4C97FF
\definecolor{slooks}{HTML}{9966FF} % #9966FF
\definecolor{ssound}{HTML}{D65CD6} % #D65CD6
\definecolor{sevents}{HTML}{FFD500} % #FFD500
\definecolor{scontrol}{HTML}{FFAB19} % #FFAB19
\definecolor{ssensing}{HTML}{4CBFE6} % #4CBFE6
\definecolor{soperators}{HTML}{6DB26E} % #6DB26E
\definecolor{svariables}{HTML}{F28011} % #F28011
\definecolor{smyblocks}{HTML}{FF6680} % #FF6680

\def\smotion{\textcolor{smotion}{\faCircle\,Mouvement}} % Déplacement du lutin
\def\slooks{\textcolor{slooks}{\faCircle\,Apparence}} % Modifier l'apparence du lutin ou de la scène
\def\ssound{\textcolor{ssound}{\faCircle\,Son}} % Jouer des sons ou de la musique
\def\sevents{\textcolor{sevents}{\faCircle\,Événement}} % Déclencher des scripts en réponse à des actions
\def\scontrol{\textcolor{scontrol}{\faCircle\,Contrôle}} % Boucles, conditions, et contrôle du flux
\def\ssensing{\textcolor{ssensing}{\faCircle\,Capteur}} % Réagir à des informations extérieures ou internes
\def\soperators{\textcolor{soperators}{\faCircle\,Opérateur}} % Calculs mathématiques et logiques
\def\svariables{\textcolor{svariables}{\faCircle\,Variable}} % Stockage et manipulation de données
\def\smyblocks{\textcolor{smyblocks}{\faCircle\,Mes blocs}} % Création de blocs personnalisés

\def\spen{{\icon{scratch/pen} Stylo}}
\def\spenExtension{{\icon{scratch/pen-extension} Stylo}}
\def\sextensions{{\icon{scratch/extensions} $\lbrack$ Ajouter une extensions $\rbrack$}}
\def\sflag{{\icon{scratch/flag}%
%  Drapeau
}}

% \setscratch{scale=.75}
% \setscratch{print=true}
% \setscratch{fill blocks=true}

\exo{Découvrir les variables}{
    \begin{enumerate}
        \item Ouvrir \capytale{3d2d-5557235}.
        \item Utilisez les blocs des onglets \slooks{} et \sevents{} pour créer un programme \Scratch qui fait dire « Salut » au lutin pendant 2 secondes lorsque l'on clique sur le \sflag.
        \item Depuis l'onglet \svariables{} :
        \begin{itemize}
            \item Créez une variable \ovalvariable{nombre}.
            \item Ajoutez \scr{\blockvariable{mettre \selectmenu{nombre} à \ovalnum{$1$}}} dans votre programme pour initialiser \ovalvariable{nombre} à la valeur $1$.
            \item Faites dire au lutin la valeur de \ovalvariable{nombre} pendant 2 secondes lorsque l'on clique sur le \sflag.
            \item Testez le programme pour plusieurs valeurs de \ovalvariable{nombre} que vous pouvez modifier dans \scr{\blockvariable{mettre \selectmenu{nombre} à \ovalnum{$1$}}}.
        \end{itemize}
        \item \begin{itemize}
            \item Utilisez à nouveau \scr{\blockvariable{mettre \selectmenu{nombre} à \ovalnum{ }}} et \ovaloperator{\ovalnum{ } + \ovalnum{ }} pour ajouter $5$ à \ovalvariable{nombre}.
            \item Faites dire au lutin la valeur de \ovalvariable{nombre} pendant 2 secondes après cette opération.
            \item Testez le programme pour plusieurs valeurs de \ovalvariable{nombre}.
        \end{itemize}
        \item Depuis l'onglet \smyblocks{} :
        \begin{itemize}
            \item Créez un bloc \scr{\blockmoreblocks{Exercice 1}}.
            \item Sous \scr{\initmoreblocks{définir \namemoreblocks{Exercice 1}}}, placez les blocs que vous avez utilisés dans cet exercice.
            \item Placez \scr{\blockmoreblocks{Exercice 1}} de sorte que l'exercice s'exécute lorsque vous appuyez sur le \sflag.
        \end{itemize}
    \end{enumerate}
}
% https://clg-monnet-briis.ac-versailles.fr/IMG/pdf/tp_programme_de_calcul_1_1_.pdf

\exo{Implémenter un programme de calcul dans Scratch}{
    \begin{enumerate}
        \item Créez un bloc \scr{\blockmoreblocks{Exercice 2}} pour composer le nouveau programme sous \scr{\initmoreblocks{définir \namemoreblocks{Exercice 2}}}.
        \item Créez une variable \ovalvariable{résultat} pour stocker le résultat des calculs.
        \item Initialisez la variable \ovalvariable{nombre} avec un nombre de votre choix.
        \item Utilisez des blocs de l'onglet \soperators{} pour créer un programme correspondant au programme de calcul suivant :
        \begin{itemize}
            \item Choisissez un nombre.
            \item Ajoutez-lui $3$.
            \item Multipliez le résultat par $5$.
        \end{itemize}
        \hint{Le nombre de départ ne doit pas être modifié,
        il faudra donc utiliser \scr{\blockvariable{mettre \selectmenu{nombre} à \ovalnum{ }}}.}
        \item Faites dire au lutin le \ovalvariable{résultat} de ce calcul.
        \item Quel est le résultat pour \ovalvariable{nombre} = $2$ ? \ovalvariable{nombre} = $10$ ? \ovalvariable{nombre} = $4654$ ?
    \end{enumerate}
}

\exo{Deux programmes de calcul}{
    \begin{enumerate}
        \item Sous \scr{\initmoreblocks{définir \namemoreblocks{Exercice 3}}}, créez deux programmes de calcul différents sous deux blocs \scr{\initmoreblocks{définir \namemoreblocks{Programme 1}}} et \scr{\initmoreblocks{définir \namemoreblocks{Programme 2}}} correspondant aux programmes de calcul suivants :
        \dividePage{
            \begin{itemize}
                \item Choisissez un nombre.
                \item Enlevez-lui $6$.
                \item Multipliez le résultat par $3$.
            \end{itemize}
        }{
            \begin{itemize}
                \item Choisissez un nombre.
                \item Multipliez-le par $3$.
                \item Enlevez-lui $18$.
            \end{itemize}
        }
        \hint{
            \begin{itemize}
                \item Les deux programmes devront s'appliquer sur le même nombre.
                \item Il faudra stocker les deux résultats dans deux variables \ovalvariable{résultat 1} et \ovalvariable{résultat 2} différentes.
            \end{itemize}
        }
        \item Faites dire au lutin les deux résultats à l'aide de l'\soperators{} \ovaloperator{regrouper \ovalnum{ } et \ovalnum{ }}.
        \item Testez les deux programmes pour plusieurs valeurs de \ovalvariable{nombre}. Que remarquez-vous ?
    \end{enumerate}
}
\exo{Stocker les résultats dans une liste}{
    \begin{enumerate}
        \item Depuis l'onglet \svariables{}, créez une liste \ovallist{résultats} pour stocker les résultats du \scr{\blockmoreblocks{Programme 1}}.
        \item Sous \scr{\initmoreblocks{définir \namemoreblocks{Exercice 4}}}, initialisez la variable \ovalvariable{nombre} à $1$.
        \item À l'aide d'une boucle, répétez le \scr{\blockmoreblocks{Programme 1}} pour $10$ valeurs de \ovalvariable{nombre} allant de $1$ à $10$.
        \item À chaque itération de la boucle, ajoutez le \ovalvariable{résultat 1} du \scr{\blockmoreblocks{Programme 1}} à la liste \ovallist{résultats} à l'aide du bloc \scr{\blocklist{ajouter \ovalnum{ } à \selectmenu{résultats}}}.
        \hint{Pensez à vider la liste \ovallist{résultats} avant de l'utiliser à l'aide du bloc \scr{\blocklist{supprimer tous les éléments de la liste \selectmenu{résultats}}}.}
        \item Testez le programme.
    \end{enumerate}
}
