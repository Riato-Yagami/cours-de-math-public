% VARIABLES %%%
\setSeq{3}{Symétries}
\setGrade{5e}

\def\imgPath{enseignement/5e/symetries/}

\def\ym{\href{https://www.maths-et-tiques.fr/telech/19Sym.pdf}{Yvan Monka}}
%%

\obj{
    \item Transformer une figure par une symétrie centrale.
    \item Identifier des symétries dans des frises, des pavages, des rosaces.
    \item Comprendre l'effet des symétries (axiale et centrale) :
    conservation du parallélisme, des longueurs et des angles.
    \item Mobiliser les connaissances des figures,
    des configurations et des symétries pour déterminer des grandeurs géométriques.
    \item Mener des raisonnements en utilisant des propriétés des figures,
    des configurations et des symétries.
}

\def\grid{\draw[gray!40] (0,0) grid (14,8);}
\def\arrow{\draw [ultra thick] (1,4)--(4,4)--(4,3)--(6,5)--(4,7)--(4,6)--(1,6)--cycle;}

\NewDocumentCommand{\mushroom}{O{(0,0)}}{
    \draw[thick, shift={#1}]
    (0,2) -- (2,4) -- (4,4) -- (6,2) -- (6,0) -- (0,0) -- cycle;
    
    % Spot
    % \draw[ultra thick, shift={#1}] 
    %     (1.5,3.5) -- (2,2) -- (4,2) -- (4.5,3.5);  % Center spot
        
    % Mushroom Stem
    \draw[thick, shift={#1}] 
        (1,0) -- (2,-2) -- (4,-2) -- (5,0) -- cycle; % Stem (rectangle)

    % Eyes
    \draw[thick, shift={#1}] 
        (2.5,-1) -- (2.5,0);

    \draw[thick, shift={#1}] 
        (3.5,-1) -- (3.5,0);
}

\def\figInit{
    \grid
    \arrow
    \drawPoint{$O$}{7}{4}
}

\newcommand{\tikzc}[1]{
    \begin{center}
        \begin{tikzpicture}[scale=0.5]
            #1
        \end{tikzpicture}
    \end{center}
}

\scn{Rappels sur la symétrie axiale}

\slide{qf}{
    \exo{}{
        Combien existe-t-il d'axes de symétrie pour chacun de ces panneaux?
        
        \imgp{panneaux-de-signalisation}[8cm]
    }[\href{https://clairelommeblog.fr/wp-content/uploads/2020/03/panneaux_routiers.pdf}
    {Claire Lommé}]
}

\bsec{Les symétries}
\bsubsec{Symétries Axiale}

% \newpage
\slide{exo}{
    \bvspace{-0.5cm}
    \act{}{
        Construire l'image de la figure par la symétrie d'axe $(d)$.
        \bvspace{-0.75cm}
        \tikzc{
            \draw[gray!40] (0,-3) grid (14,8);
            \arrow
            \draw[thick, gradeColor] (0,-3)--(11,8);
            \node[gradeColor, above left] at (2,-1) {$(d)$};
        }
    }
}

\slide{cr}{
    \sseq\ssec\ssubsec
    \df{}{
        Deux figures sont dites \key{symétriques par rapport à une droite} si elles se \key{superposent par pliage} le long de cette droite.
    }[\wiki{Symétrie_axiale}]
}

\slide{cr}{\bvspace{-0.6cm}
    \expl{}{\bvspace{-0.5cm}
        \imgp{expl-symetrie-axiale}[7.5cm]
    }
}

\slide{cr}{
    \pr{}{
        \Sialors{le point $M'$ est l'image du point $M$ par la symétrie d'axe $(d)$}{
            \begin{enumerate}
                \item la droite $(MM')$ est \bawsr{perpendiculaire} à la droite $(d)$
                \item le milieu du segment $[MM']$ est sur la droite $(d)$.
            \end{enumerate}
        }
        \rmk{}{%
            La droite $(d)$ est alors la \bawsr{\key{médiatrice} } du segment $[MM']$.
        }
    }
}

\slide{exo}{
    \bvspace{-0.6cm}
    \exo{}{
        Construire l'image de la figure par la symétrie d'axe $(d)$.
        \bvspace{-0.75cm}
        \tikzc{
            \draw[gray!40] (0,-4) rectangle (17,9);
            \mushroom[(10,-1)]
            \draw[thick, gradeColor] (7,-3)--(10,8);
            \node[gradeColor, above left] at (7,-3) {$(d)$};
        }
    }
}

\bsubsec{Symétries Centrale}

\def\one{%
    On va construire l'image de la figure par la symétrie de centre $O$.

    \tikzc{\figInit}
}

\def\two{%
    On regarde le «\textit{chemin}» du point $A$ au point $O$.
    
    \tikzc{
        \figInit
        \drawPoint{$A$}{4}{3}
        \draw[dashed, thick, Red] (4,3)--(7,4); 
    }
}

\def\three{%
    On exécute le même «\textit{chemin}», cette fois en partant du point $O$. On trouve alors le point $A'$,
    image du point $A$ par la symétrie de centre $O$.

    \tikzc{
        \figInit
        \drawPoint{$A$}{4}{3}
        \draw[dashed, thick, Red] (7,4)--(10,5);
        \drawPoint{$A'$}{10}{5}
    }
}

\def\four{%
    Fait de même avec les autres points de la figure, puis les reliers, de façon à obtenir l'image de la flèche par la symétrie de centre $O$.

    \tikzc{
        \figInit
        \drawPoint{$A$}{4}{3}
        \drawPoint{$B$}{4}{4}
        \drawPoint{$C$}{1}{4}
        \drawPoint{$D$}{1}{6}
        \drawPoint{$E$}{4}{6}
        \drawPoint{$F$}{4}{7}
        \drawPoint{$G$}{6}{5}
        \drawPoint{$A'$}{10}{5}
    }
}

\scn{Découvrir la symétrie centrale}

\ifArticle{%
    \slide{exo}{
        \act{}{\vspace{-0.5cm}
            \multiColEnumerate{2}{
                \item[] \one
                \item \two
                \item \three
                \item \four
            }
        }[\href{https://drive.google.com/drive/folders/1Itzq0ZPj1sHwSIv9SgbUlIIuHdG28A5I}{Jérôme Potel}]
    }
}


\ifBeamer{%
    \slide{exo}{\act{}{\one}}
    \slide{exo}{\two}
    \slide{exo}{\three}
    \slide{exo}{\four}
}

\slide{cr}{
    \ssubsec
    
    \df{}{
        Deux figures sont dites \key{symétriques par rapport à un point}
        si l'on peut obtenir l'une en effectuant un \key{demi-tour} de l'autre \key{autour de ce point}.
    }[\wiki{Symétrie_centrale}]
}

\slide{cr}{\bvspace{-0.6cm}
    \expl{}{\bvspace{-0.5cm}
        \imgp{expl-symetrie-centrale}[9cm]
    }
}

\slide{cr}{
    \pr{}{
        \Sialors{le point $M'$ est le symétrique du point $M$ par la symétrie de centre $O$}{
            le point $O$ est \bawsr{le milieu du segment} $[MM']$.
        }
    }
}
