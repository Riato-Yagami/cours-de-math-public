% VARIABLES %%%
\setSeq{3}{Symétries}
\setGrade{5e}

% \setboolean{answer}{false}
% \forPrint
\def\imgPath{enseignement/5e/symetries/}

\def\ym{\href{https://www.maths-et-tiques.fr/telech/19Sym.pdf}{Yvan Monka}}
\def\jerome{\href{https://drive.google.com/drive/folders/1Itzq0ZPj1sHwSIv9SgbUlIIuHdG28A5I}{Jérôme Potel}}
%%

\obj{
    \item Transformer une figure par une symétrie centrale.
    \item Identifier des symétries dans des frises, des pavages, des rosaces.
    \item Comprendre l'effet des symétries (axiale et centrale) :
    conservation du parallélisme, des longueurs et des angles.
    \item Mobiliser les connaissances des figures,
    des configurations et des symétries pour déterminer des grandeurs géométriques.
    \item Mener des raisonnements en utilisant des propriétés des figures,
    des configurations et des symétries.
}

\def\grid{\draw[gray!40] (0,0) grid (14,8);}
\def\arrow{\draw [ultra thick] (1,4)--(4,4)--(4,3)--(6,5)--(4,7)--(4,6)--(1,6)--cycle;}

\NewDocumentCommand{\mushroom}{O{(0,0)}}{
    \draw[thick, shift={#1}]
    (0,2) -- (2,4) -- (4,4) -- (6,2) -- (6,0) -- (0,0) -- cycle;
    
    % Spot
    % \draw[ultra thick, shift={#1}] 
    %     (1.5,3.5) -- (2,2) -- (4,2) -- (4.5,3.5);  % Center spot
        
    % Mushroom Stem
    \draw[thick, shift={#1}] 
        (1,0) -- (2,-2) -- (4,-2) -- (5,0) -- cycle; % Stem (rectangle)

    % Eyes
    \draw[thick, shift={#1}] 
        (2.5,-1) -- (2.5,0);

    \draw[thick, shift={#1}] 
        (3.5,-1) -- (3.5,0);
}

\def\figInit{
    \grid
    \arrow
    \drawPoint{$O$}{7}{4}
}

\scn{Rappels sur la symétrie axiales}

\slide{qf}{\bvspace{-1cm}
    \exo{}{
        Est-ce que deux dinosaures identiques peuvent être superposés l'un à l'autre grâce à une symétrie axiale ?
        Si ce n'est pas possible, pourquoi ?
        \imgp{qf/dinos-axiales}[9cm]
    }[\yuyu]
}

\bsec{Les symétries}
\bsubsec{Symétrie Axiale}

\slide{exo}{
    \bvspace{-0.5cm}
    \act{}{
        Construire l'image de la figure par la symétrie d'axe $(d)$.
        \bvspace{-0.75cm}
        \ctikz{
            \draw[gray!40] (0,-3) grid (14,8);
            \arrow
            \draw[thick, gradeColor] (0,-3)--(11,8);
            \node[gradeColor, above left] at (2,-1) {$(d)$};
        }
    }
}

\slide{cr}{
    \sseq\ssec\ssubsec
    \df{}{
        Deux figures sont dites \key{symétriques par rapport à une droite} si elles se \key{superposent par pliage} le long de cette droite.
    }[\wiki{Symétrie_axiale}]
}

\slide{cr}{\bvspace{-0.6cm}
    \expl{}{\bvspace{-0.5cm}
        \imgp{expl-symetrie-axiale}[7.5cm]
    }
}

\slide{cr}{
    \pr{}{
        \Sialors{le point $M'$ est l'image du point $M$ par la symétrie d'axe $(d)$}{
            \begin{enumerate}
                \item la droite $(MM')$ est \awsr{perpendiculaire} à la droite $(d)$
                \item le milieu du segment $[MM']$ est sur la droite $(d)$.
            \end{enumerate}
        }
        \rmk{}{%
            La droite $(d)$ est alors la \awsr{\key{médiatrice} } du segment $[MM']$.
        }
    }
}

\slide{exo}{
    \bvspace{-0.6cm}
    \exo{}{
        Construire l'image de la figure par la symétrie d'axe $(d)$.
        \bvspace{-0.75cm}
        \ctikz{
            \draw[gray!40] (0,-4) rectangle (17,9);
            \mushroom[(10,-1)]
            \draw[thick, gradeColor] (7,-3)--(10,8);
            \node[gradeColor, above left] at (7,-3) {$(d)$};
        }
    }
}

\scn{Découvrir la symétrie centrale}

\slide{qf}{
    \exo{}{
        Combien existe-t-il d'axes de symétrie pour chacun de ces panneaux?
        
        \imgp{panneaux-de-signalisation}[8cm]
    }[\href{https://clairelommeblog.fr/wp-content/uploads/2020/03/panneaux_routiers.pdf}
    {Claire Lommé}]
}

\bsubsec{Symétrie Centrale}

\def\one{%
    On va construire l'image de la figure par la symétrie de centre $O$.

    \ctikz{\figInit}
}

\def\two{%
    On regarde le «\textit{chemin}» du point $A$ au point $O$.
    
    \ctikz{
        \figInit
        \drawPoint{$A$}{4}{3}
        \draw[dashed, thick, Red] (4,3)--(7,4); 
    }
}

\def\three{%
    On exécute le même «\textit{chemin}», cette fois en partant du point $O$. On trouve alors le point $A'$,
    image du point $A$ par la symétrie de centre $O$.

    \ctikz{
        \figInit
        \drawPoint{$A$}{4}{3}
        \draw[dashed, thick, Red] (7,4)--(10,5);
        \drawPoint{$A'$}{10}{5}
    }
}

\def\four{%
    Fait de même avec les autres points de la figure, puis les reliers, de façon à obtenir l'image de la flèche par la symétrie de centre $O$.

    \ctikz[0.5]{
        \figInit
        \drawPoint{$A$}{4}{3}
        \drawPoint{$B$}{4}{4}
        \drawPoint{$C$}{1}{4}
        \drawPoint{$D$}{1}{6}
        \drawPoint{$E$}{4}{6}
        \drawPoint{$F$}{4}{7}
        \drawPoint{$G$}{6}{5}
        \drawPoint{$A'$}{10}{5}
    }
}

\ifArticle{%
    \slide{exo}{
        \act{}{\vspace{-0.5cm}
            \multiColEnumerate{2}{
                \item[] \one
                \item \two
                \item \three
                \item \four
            }
        }[\jerome]
    }
}


\ifBeamer{%
    \slide{exo}{\act{}{\one}}
    \slide{exo}{\two}
    \slide{exo}{\three}
    \slide{exo}{\four}
}

\slide{cr}{
    \ssubsec
    
    \df{}{
        Deux figures sont dites \key{symétriques par rapport à un point}
        si l'on peut obtenir l'une en effectuant un \key{demi-tour} de l'autre \key{autour de ce point}.
    }[\wiki{Symétrie_centrale}]
}

\slide{cr}{\bvspace{-0.6cm}
    \expl{}{\bvspace{-0.5cm}
        \imgp{expl-symetrie-centrale}[9cm]
    }
}

\slide{cr}{
    \pr{}{
        \Sialors{le point $M'$ est le symétrique du point $M$ par la symétrie de centre $O$}{
            le point $O$ est \awsr{le milieu du segment } $[MM']$.
        }
    }
}

\scn{Manipuler la symétrie centrale}

\slide{qf}{\bvspace{-1cm}
    \exo{}{
        Est-ce que deux dinosaures identiques peuvent être superposés l'un à l'autre grâce à une symétrie centrale ?
        Si ce n'est pas possible, pourquoi ?
        \bvspace{-0.5cm}
        \imgp{qf/dinos-centrales}[9cm]
    }[\yuyu]
}

\def\konoha{%
    % \draw [thick] (9,3)-- (8,2)-- (7,3)-- (8,4)-- (10,3)-- (9,1)-- (7,1)-- (5,1)-- (6,3)-- (7,5)-- (9,5)-- (10,4)-- (11,5);
    % \draw [thick] (7,1)-- (6,3);
    \draw [thick] (3,5)-- (4,6)-- (5,5)-- (4,4)-- (2,5)-- (3,7)-- (5,7)-- (7,7)-- (6,5)-- (5,3)-- (3,3)-- (2,4)-- (1,3);
    \draw [thick] (5,7)-- (6,5);
    \drawPoint{$O$}{6}{4}[gradeColor]
}

\slide{exo}{\bvspace{-0.75cm}
    \exo{}{
        Construis l'image de la figure par la symétrie de centre $O$.
        \ctikz[0.6]{
            \draw[gray!40] (0,-1) grid (13,8);
            \konoha
        }
    }
}

% Define a command to place numbered nodes at specific coordinates
\def\hexNode#1#2{
    \draw [color=gradeColor!75, fill opacity=1] #1 node[anchor=center, scale=1] {$#2$};
}

% Define a command for drawing quadrilateral shapes based on four coordinates
\def\quadShape#1#2#3#4{
    \draw[color=black] #1--#2--#3--#4--cycle;
}

% Coordinates for nodes and quadrilateral shapes
\newcommand{\hexagones}{
    % Place nodes with loop (coordinates, label)
    \foreach \coord/\label in {
        (2.75,-1.3)/1, (5,0)/2, (2.75,1.3)/3, (9.5,2.6)/4,
        (7.25,1.3)/5, (7.25,3.9)/6, (7.25,-3.9)/7, (9.5,-2.6)/8,
        (7.25,-1.3)/9, (11.75,-1.3)/10, (14,0)/11, (11.75,1.3)/12,
        (18.5,2.6)/13, (16.25,1.3)/14, (16.25,3.9)/15, (16.25,-3.9)/16,
        (18.5,-2.6)/17, (16.25,-1.3)/18, (7.25,-9.09)/19, (9.5,-7.79)/20,
        (7.25,-6.5)/21, (14,-5.2)/22, (11.75,-6.5)/23, (11.75,-3.9)/24,
        (11.75,-11.69)/25, (14,-10.39)/26, (11.75,-9.09)/27, (16.25,-9.09)/28,
        (18.5,-7.79)/29, (16.25,-6.5)/30, (23,-5.2)/31, (20.75,-6.5)/32,
        (20.75,-3.9)/33, (20.75,-11.69)/34, (23,-10.39)/35, (20.75,-9.09)/36
    }{%
        \hexNode{\coord}{\label}
    }
    % Draw quadrilateral shapes
    \quadShape{(0,0)}{(3,0)}{(4.5,-2.6)}{(1.5,-2.6)} \quadShape{(4.5,2.6)}{(6,0)}{(4.5,-2.6)}{(3,0)}
    \quadShape{(0,0)}{(1.5,2.6)}{(4.5,2.6)}{(3,0)} \quadShape{(9,5.2)}{(10.5,2.6)}{(9,0)}{(7.5,2.6)}
    \quadShape{(4.5,2.6)}{(7.5,2.6)}{(9,0)}{(6,0)} \quadShape{(4.5,2.6)}{(6,5.2)}{(9,5.2)}{(7.5,2.6)}
    \quadShape{(4.5,-2.6)}{(7.5,-2.6)}{(9,-5.2)}{(6,-5.2)} \quadShape{(9,0)}{(10.5,-2.6)}{(9,-5.2)}{(7.5,-2.6)}
    \quadShape{(4.5,-2.6)}{(6,0)}{(9,0)}{(7.5,-2.6)} \quadShape{(9,0)}{(12,0)}{(13.5,-2.6)}{(10.5,-2.6)}
    \quadShape{(13.5,2.6)}{(15,0)}{(13.5,-2.6)}{(12,0)} \quadShape{(9,0)}{(10.5,2.6)}{(13.5,2.6)}{(12,0)}
    \quadShape{(18,5.2)}{(19.5,2.6)}{(18,0)}{(16.5,2.6)} \quadShape{(13.5,2.6)}{(16.5,2.6)}{(18,0)}{(15,0)}
    \quadShape{(13.5,2.6)}{(15,5.2)}{(18,5.2)}{(16.5,2.6)} \quadShape{(13.5,-2.6)}{(16.5,-2.6)}{(18,-5.2)}{(15,-5.2)}
    \quadShape{(18,0)}{(19.5,-2.6)}{(18,-5.2)}{(16.5,-2.6)} \quadShape{(13.5,-2.6)}{(15,0)}{(18,0)}{(16.5,-2.6)}
    \quadShape{(4.5,-7.79)}{(7.5,-7.79)}{(9,-10.39)}{(6,-10.39)} \quadShape{(9,-5.2)}{(10.5,-7.79)}{(9,-10.39)}{(7.5,-7.79)}
    \quadShape{(4.5,-7.79)}{(6,-5.2)}{(9,-5.2)}{(7.5,-7.79)} \quadShape{(13.5,-2.6)}{(15,-5.2)}{(13.5,-7.79)}{(12,-5.2)}
    \quadShape{(9,-5.2)}{(12,-5.2)}{(13.5,-7.79)}{(10.5,-7.79)} \quadShape{(9,-5.2)}{(10.5,-2.6)}{(13.5,-2.6)}{(12,-5.2)}
    \quadShape{(9,-10.39)}{(12,-10.39)}{(13.5,-12.99)}{(10.5,-12.99)} \quadShape{(13.5,-7.79)}{(15,-10.39)}{(13.5,-12.99)}{(12,-10.39)}
    \quadShape{(9,-10.39)}{(10.5,-7.79)}{(13.5,-7.79)}{(12,-10.39)} \quadShape{(13.5,-7.79)}{(16.5,-7.79)}{(18,-10.39)}{(15,-10.39)}
    \quadShape{(18,-5.2)}{(19.5,-7.79)}{(18,-10.39)}{(16.5,-7.79)} \quadShape{(13.5,-7.79)}{(15,-5.2)}{(18,-5.2)}{(16.5,-7.79)}
    \quadShape{(22.5,-2.6)}{(24,-5.2)}{(22.5,-7.79)}{(21,-5.2)} \quadShape{(18,-5.2)}{(21,-5.2)}{(22.5,-7.79)}{(19.5,-7.79)}
    \quadShape{(18,-5.2)}{(19.5,-2.6)}{(22.5,-2.6)}{(21,-5.2)} \quadShape{(18,-10.39)}{(21,-10.39)}{(22.5,-12.99)}{(19.5,-12.99)}
    \quadShape{(22.5,-7.79)}{(24,-10.39)}{(22.5,-12.99)}{(21,-10.39)} \quadShape{(18,-10.39)}{(19.5,-7.79)}{(22.5,-7.79)}{(21,-10.39)}    
}

\slide{exo}{\bvspace{-0.75cm}
    \exo{Pavage}{
        Le pavage ci-dessous est réalisé à l'aide de 36 pièces identiques.
        \bvspace{-0.5cm}
        \ctikz[0.35]{
            \hexagones
            \drawPoint{A}{13.5}{-2.6}
            \drawPoint{B}{12}{-5.2}
            \drawPoint{C}{13.5}{-7.79}
            \drawPoint{D}{18}{-5.2}
            \drawPoint{E}{9}{-2.6}
        }
    }[\jerome{} et \href{https://www.iparcours.fr/ouvrages/ouvrages.php?ouvrage=Cahier52022}{iParcours}]
}

\slide{exo}{\bsmall
    Par la symetrie de centre $A$, quelle est l'image de la figure :
    \begin{enumerate}
        \item Observe le pavage, puis complete le tableau
        \begin{center}
            \def\cW{1cm}\renewcommand{\arraystretch}{1.75}%
            \begin{tabular}{|C{5cm}|C{\cW}|C{\cW}|C{\cW}|C{\cW}|}
                \hline La pièce &
                16 & \awsr{8}  & 36 & 8\\
                \hline est l'image de la pièce&
                \awsr{10} & 17 & 18 & \awsr{34}\\
                \hline par rapport au point &
                A & A & \awsr{D} & C\\
                \hline
            \end{tabular}
        \end{center}
        \item La pièce 6 et 24 sont symétrique par rapport au point $F$.
        Place le point $F$ sur la figure. \saveenumi
    \end{enumerate}
}

\slide{exo}{
    \begin{enumerate}\loadenumi[exo]
        \item Ahmed dit :
        \guillemetleft J'ai transformé la pièce 11 par symétrie de centre $A$ puis par symétrie d'axe $(AB)$.\guillemetright
        Quelle pièce à t-il trouvé.
        \item Comme Ahmed rédige un programme de construction qui permet de transformer: 
        \begin{enumerate}
            \item la pièce 32 en la pièce 20.
            \item la pièce 10 en la pièce 28.
        \end{enumerate}
    \end{enumerate}
}

\scn{Construire une image par symétrie centrale sur papier blanc}

\def\caPrefix{5e-mars-2023-}
\caSlide{23-24-25}

\slide{cr}{
    \mthd{Construire l'image d'un point par une symétrie centrale}{
        Pour construire l'image du point $A$ par une symétrie de centre $O$ :
        \begin{enumerate}
            \item Tracez la demi-droite $[AO)$.
            \item Reportez la mesure $AO$ de l'autre côté du point $O$ sur la droite $(AO)$.
            \item Vous obtenez l'image du point $A$, que l'on peut nommer $A'$.
        \end{enumerate}
    }
}

\slide{cr}{
    \expl{}{Construisez l'image du point $A$ par la symétrie de centre $O$.
    \ctikz{
        \draw[gray!40] (-5,-2) rectangle (10,9);
        \drawPoint{$A$}{-3}{0}
        \drawPoint{$O$}{2}{3}[gradeColor]
    }
    }
}

\slide{cr}{
    \mthd{Construire l'image d'une figure par une symétrie centrale}{
        Pour construire l'image d'une figure par une symétrie de centre $O$ :
        \begin{enumerate}
            \item Construisez l'image de chacun des points de cette figure par la symétrie de centre $O$.
            \item Reliez chacun de ces points comme sur la figure d'origine.
        \end{enumerate}
    }
}

\slide{cr}{\bvspace{-0.5cm}
    \expl{}{Construisez l'image de la figure par la symétrie de centre $O$.
    \ctikz[0.4]{
        \draw[gray!40] (-10,-8) rectangle (11,9);
        \draw [line width=1pt] (0,5)-- (10,6)-- (5,1) -- cycle;
        \drawPoint{$O$}{1}{1}[gradeColor]
    }
}
}

\scn{Découvrir les propriétés de la symétrie centrale}

\def\caPrefix{5e-mars-2022-}
\caSlide{14-15}

\bsec{Propriétés des symétries}

\slide{exo}{\bvspace{-0.75cm}\bsmall
    \act{}{
        \begin{enumerate}
            \item \begin{enumerate}
                \item Sur feuille blanche, construire un rectangle $ABCD$.
                \item Placer un point $P$ sur la droite $(BD)$.
                \item Placer un point $E$ n'importe où sur votre feuille.
                \item Construire les points $A'$, $B'$, $C'$, $D'$ et $P'$, images des points $A$, $B$, $C$ et $D$ par la symétrie de centre $E$.
            \end{enumerate}
            \item \begin{enumerate}
                \item Comment semblent les distances $BD$ et $B'D'$ ?
                \item Comment semblent les aires des triangles $BCD$ et $B'C'D'$ ?
                \item Comment semblent les angles $\widehat{ADB}$ et $\widehat{A'D'B'}$ ?
                \item Sur quelle droite semble être située le point $P$.
                \item Comment semblent les droites $CD$ et $C'D'$ ?
            \end{enumerate}\saveenumi
        \end{enumerate}
    }
}

\slide{exo}{
    \begin{enumerate}\loadenumi[act]
        \item \begin{enumerate}
            \item A partir des questions précédentes;
            formulez des conjectures sur les propriétés de la symétrie centrale.
            \item En déduire la nature du quadrilatère $A'B'C'D'$.
        \end{enumerate}
    \end{enumerate}
}

\slide{cr}{
    \ssec
    \pr{}{
        Les symétries conservent les mesures:
        \multiColItemize{2}{
            \item de distances.
            \item d'angles.
        }
    }

    \cor{}{
        Les symétries conservent:
        \multiColItemize{2}{
            \item les aires.
            \item les alignements.
        }
    }
}

\slide{cr}{
    \pr{}{
        \Sialors{$(d_1)$ est l'image de $(d_2)$ par une symétrie centrale}
        {$(d_1)$ est parallèle à $(d_2)$.}
    }

    \expl{}{On considère un triangle $OAB$.
    \begin{enumerate}
        \item Le point $A'$ est le symétrique de $A$ par rapport à $O$.
        \item Le point $B'$ est le symétrique de $B$ par rapport à $O$.
    \end{enumerate}
        Prouver que les droites $(AB)$ et $(A'B')$ sont parallèles. 
    }[\ym]

    \awsr[0]{
        \begin{itemize}
            \item Les points $A'$ et $B'$ sont les symétriques respectifs des points $A$ et $B$ par rapport au point $O$.
            \item Par conséquent, la droite $(A'B')$ est le symétrique de la droite $(AB)$ par rapport à $O$.
            \item Or, deux droites symétriques par rapport à un point sont parallèles.
            \item Ainsi, les droites $(AB)$ et $(A'B')$ sont parallèles.
        \end{itemize}        
    }
}

\scn{Utiliser les propriétés de la symétrie centrale}

\def\caPrefix{5e-juin-2022-}
\caSlide{14-15-16}

\slide{exo}{\bvspace{-0.75cm}\bsmall
    \exo{Sans le centre}{
        $[A'B']$ est le symétrique du segment $[AB]$ par rapport à un point $O$.
        En utilisant uniquement une règle non graduée et un rapporteur, construis la figure symétrique par rapport à $O$ de la ligne brisée $ABCD$.
        Comment as-tu raisonné ?\bvspace{-0.25cm}
        \ctikz[0.65]{
            \draw[gray!40] (-7,-9) rectangle (11,7);
            \draw [thick] (-6.06,1.06)-- (-2.3,2.88);
            \draw [thick] (7.34,-1.38)-- (3.58,-3.2);
            \draw [thick] (-2.3,2.88)-- (-2.42,-1.68);
            \draw [thick] (-2.42,-1.68)-- (1.34,-3.52);    
            \drawPoint{A}{-6.06}{1.06}
            \drawPoint{B}{-2.30}{2.88}
            \drawPoint{C}{-2.42}{-1.68}
            \drawPoint{D}{1.34}{-3.52}
            \drawPoint{O}{0.64}{-0.16}
            \drawPoint{A'}{7.34}{-1.38}
            \drawPoint{B'}{3.58}{-3.20}
            \awsr[0]{
                \draw [thick, color = answer] (-2.260860768328644,0.597391072850754) -- (-2.4591392316713563,0.6026089271492464);
                \draw [thick, color = answer] (3.58,-3.2) -- (3.7,1.36);
                \draw [thick, color = answer] (3.540860768328643,-0.9173910728507542) -- (3.7391392316713574,-0.9226089271492466);
                \draw [thick, color = answer] (3.7,1.36) -- (-0.06,3.2);
                \draw [thick, color = answer] (1.8135243739047,2.1727573141778724) -- (1.9007084130413938,2.350916002848505);
                \draw [thick, color = answer] (1.7392915869586045,2.2090839971514953) -- (1.8264756260952983,2.3872426858221285);
                \draw [thick, color = answer] (-0.5335243739047016,-2.4927573141778727) -- (-0.6207084130413955,-2.6709160028485055);
                \draw [thick, color = answer] (-0.4592915869586044,-2.529083997151494) -- (-0.5464756260952982,-2.7072426858221266);
                \draw [shift={(-2.3,2.88)},thick,color=answer] (-154.1710425344797:0.49586776859504095) arc (-154.1710425344797:-91.50743575877497:0.49586776859504095);
                \draw[thick,color=answer] (-2.5352884027004894,2.515452402024578) -- (-2.6025136606149144,2.411295945460171);
                \draw [shift={(3.58,-3.2)},thick,color=answer] (25.828957465520286:0.49586776859504095) arc (25.828957465520286:88.49256424122504:0.49586776859504095);
                \draw[thick,color=answer] (3.8152884027004883,-2.8354524020245777) -- (3.882513660614915,-2.731295945460172);
                \draw [shift={(3.7,1.36)},thick,color=answer] (153.92464441605125:0.49586776859504095) arc (153.92464441605125:268.49256424122507:0.49586776859504095);
                \draw[thick,color=answer] (3.3027345735488174,1.1855415104342648) -- (3.189230165991338,1.1356962277011982);
                \draw[thick,color=answer] (3.3614242443784947,1.088666180986734) -- (3.2646883142009218,1.01114223269723);
                \draw [shift={(-2.42,-1.68)},thick,color=answer] (-26.07535558394877:0.49586776859504095) arc (-26.07535558394877:88.49256424122504:0.49586776859504095);
                \draw[thick,color=answer] (-2.022734573548817,-1.505541510434264) -- (-1.9092301659913369,-1.4556962277011973);
                \draw[thick,color=answer] (-2.081424244378494,-1.4086661809867342) -- (-1.9846883142009206,-1.3311422326972282);
                \drawPoint{C'}{3.70}{1.36}[answer]
                \drawPoint{D'}{-0.06}{3.20}[answer]
            }
        }
    }[\dmeepcC]

    \awsr[0]{On utilise les propriétés conservations des mesures d'angles et longueurs de la symétrie centrale,
    pour construire le symétrique de cette figure}
}

\slide{exo}{\bvspace{-0.7cm}\bsmall
    \exo{Figure incomplète}{
        $ABC$ est un triangle, mais le point $C$ est en dehors de la feuille.
        Construis le symétrique du triangle $ABC$ par rapport au point $O$,
        avec seulement une règle et un compas.
        Comment as-tu raisonné ?\bvspace{-0.25cm}
        \ctikz[0.45]{
            \draw[gray!40] (-5,-10) rectangle (10,3);
            \draw [thick] (6.18,-5.08)-- (5,-1.14);
            \draw [thick] (5,-1.14)-- (9.86,-2.845909090909091);
            \draw [thick] (6.18,-5.08)-- (9.86,-4.4466468842729965);
            \drawPoint{B}{5}{-1.14}
            \drawPoint{A}{6.18}{-5.08}
            \drawPoint{O}{4.36}{-4.14}
        }
}[\dmeepcC]
}

\scn{Découvrir la notion de centre de symétrie dans une figure}

\def\caPrefix{5e-entrainement.4-}
\caSlide{29-30}

\bsec{Centre de symétrie}

\NewDocumentCommand{\mimg}{m O{1}}{%
    \listSize{#1}%
    \ifBeamer{\def\imgswidht{0.9\linewidth*#2/\thesize}}
    \ifArticle{\def\imgswidht{\linewidth*#2/\thesize}}
    \foreach \sport in {#1}{%
        \includegraphics[width=\imgswidht]{\imgf{\sport}}
    }
}

\slide{cr}{
    \vc{}{
        Un point est un \key{centre de symétrie} d'une figure,
        lorsqu'en effectuant un \key{demi-tour}
        \key{autour du point},
        la figure se \key{superpose avec elle-même}.
    }[\ym]

    \expl{Pictogrammes JO 2024}{
        \def\imgPrefix{jo/}
        \mimg{athletisme,aviron,basketball,VTT,escalade,natation}[1]
        Lesquels de ces pictogrammes possèdent un centre de symétrie?
    }
}

\slide{exo}{
    \exo{}{
        \begin{enumerate}
            \item Construire un rectangle $ABCD$.
            \item Déterminer $O$, le centre de symétrie du rectangle.
            \item Comparer les aires des triangles $AOD$ et $BOC$ ? Justifier votre réponse.
        \end{enumerate}
    }
    \awsr{
        \ctikz[1]{
            \draw[gray!40] (-8.5,-2.5) rectangle (2.5,5.5);
            \drawPoint{A}{-7.86}{2.4}
            \drawPoint{B}{0.74}{4.22}
            \drawPoint{C}{1.53}{0.47}
            \drawPoint{D}{-7.07}{-1.35}
            \drawPoint{O}{-3.16}{1.43}
            \draw [thick] (-7.86,2.40) -- (0.74,4.22) -- (-7.07,-1.35) -- (-7.86,2.40) -- (1.53,0.47) -- (-7.07,-1.35);
            \draw [thick] (1.53,0.47) -- (0.74,4.22);
        }
        \begin{enumerate} \loadenumi[exo][2]
            \item Les triangles $AOD$ et $BOC$ sont image l'un de l'autre par la symetrie de centre $O$.
            \item Or la symetries centrale conserve les mesures d'aires.
            Donc les aires des triangles $AOD$ et $BOC$ sont égales.
        \end{enumerate}
    }
}

% \slide{exo}{}