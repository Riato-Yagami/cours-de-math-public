% VARIABLES %%%
\def\authors{\jules}
% \date{\today}
\def\longTitle{Nombres Relatifs : Repérage et comparaison}
\def\shortTitle{\MakeUppercase{\longTitle}}
% \bseq{\longTitle}
% \def\theme{\longTitle}

\setboolean{showRef}{false}

% \def\my{Myriade 6e}
% \newcommand{\myl}[1]{\href{#1}{\my}}

\def\imgPath{enseignement/5e/nombres-relatifs/reperage-et-comparaison/}
\def\imgExtension{.png}
%%

% Yvan Monka : https://www.maths-et-tiques.fr/telech/19Nomb_rel1.pdf
% Euler : https://euler-ressources.ac-versailles.fr/wims/wims.cgi?module=help%2Fteacher%2Fprogram%2F&+cmd=new&+job=math.cycle4#chapitre000

% \disableAnimation
% \shortAnimation
% \firstSlide

\def\pv{\; ; \;}
\scn{Repérage}{}

\slide{QUESTIONS FLASH}{%
    \sqf Comparer à l'aide du signe $>$ ou $<$ ou $=$ les nombres :
    \begin{align*}
        &\qfs \; 10 \et 10,075\\
        &\qfs \; 0,5 \et \frac{1}{2}\\
        &\qfs \; \frac{6}{10} \et \frac{6}{9}
    \end{align*}
}

\slide{}{%
    \sqf Ranger les nombres suivant dans l'ordre croissant les nombres :
    \begin{align*}
        1,2 \pv 6 \pv 1,15 \pv 2 \pv 100 \pv 0,584
    \end{align*}
}

\slide{}{%
    \sqf Ranger les nombres suivant dans l'ordre croissant les nombres :
    \begin{align*}
        1,2 \pv 6 \pv 1,15 \pv 2 \pv 100 \pv 0,584
    \end{align*}
}


\slide{EXERCICES}{
    \act{}{
        \dividePage{
            \imgp{carte}[6cm]
        }{
            \begin{enumerate}
                \item Quelles informations nous apporte ce document?
                \item Classer les nombres en deux catégories et donner un nom à chaque catégorie.
                \item Ranger les températures par ordre croissant.
            \end{enumerate}
        }
    }
    % [\href{https://www.facebook.com/groups/994675223903586/search/?q=activite\%20m%C3\%A9t\%C3\%A9o\%20nombres\%20relatifs\&locale=fr\_FR}{Vanessa Cazier}]
}