% VARIABLES %%%
\def\authors{\jules}
% \date{\today}
\def\longTitle{Nombres Relatifs : Repérage et comparaison}
\def\shortTitle{\MakeUppercase{\longTitle}}
\bseq{\longTitle}
% \def\theme{\longTitle}

\setgrade{5e}

\def\imgPath{enseignement/5e/nombres-relatifs/reperage-et-comparaison/}
\def\imgExtension{.png}
%%

% Yvan Monka : https://www.maths-et-tiques.fr/telech/19Nomb_rel1.pdf
% Euler : https://euler-ressources.ac-versailles.fr/wims/wims.cgi?module=help%2Fteacher%2Fprogram%2F&+cmd=new&+job=math.cycle4#chapitre000

% \disableAnimation
% \shortAnimation
% \firstSlide

\def\pv{\; ; \;}

% \article{\vskip}
\ifArticle{\vspace*{0.1cm}}

\scn{Introduction aux nombres négatifs}{}

\bsec{Définitions}

\slide{EXERCICES}{
    \vspace*{-0.5cm}
    \act{}{
        \dividePage{
            \imgp{carte}[6cm]
        }{
            \begin{enumerate}
                \item Quelles informations nous apporte ce document?
                \item Classer les nombres en deux catégories et donner un nom à chaque catégorie.
                \item Ranger les températures par ordre croissant.
            \end{enumerate}
        }
    }
    % [\href{https://www.facebook.com/groups/994675223903586/search/?q=activite\%20m%C3\%A9t\%C3\%A9o\%20nombres\%20relatifs\&locale=fr\_FR}{Vanessa Cazier}]
}

\scn{Institutionnalisation des nombres relatifs}{}

\slide{COURS}{
    \sseq\ssec
    \df{}{
        Un nombre est dit :
        \begin{itemize}
            \item \key{positif} si il est supérieur ou égal à zéro
            \item \key{négatif} si il est supérieur ou égal à zéro
        \end{itemize}
    }
}

\slide{}{
    \rmk{}{
        \begin{itemize}
            \item $O$ est à la fois positif et négatif
            \item On nomme nombres relatifs l'ensemble des nombres positifs et négatifs.
            \item On parle de nombres relatifs car on les considère relativement à zéro.
        \end{itemize}
    }
}

\def\imgPrefix{mm-c4/exo-}
\exoSlide{28p196,29p196,30p196}[9cm][1][\mm]
\def\imgPrefix{}

\scn{Résution d'équations grâce aux nombres relatifs}{}

\slide{QUESTIONS FLASH}{%
    \sqf Comparer à l'aide du signe $>$ ou $<$ ou $=$ les nombres :
    \begin{align*}
        &\qfs \; 10 \hole 10,075\\
        &\qfs \; 0,5 \hole \frac{1}{2}\\
        &\qfs \; \frac{6}{10} \hole \frac{6}{9}
    \end{align*}
}

\slide{}{
    \sqf Completer les opérations à trous :
    \begin{align*}
        &\qfs \; 6 + \hole = 18\\
        &\qfs \; 710,5 + \hole = 770\\
        &\qfs \; 30 - \hole = 22,2
    \end{align*}
}

\slide{EXERCICES}{
    \act{}{
        \begin{enumerate}
            \item $ 5 + \textrm{?} = 13$ : Trouve le nombre représenté par « ? »
            \begin{itemize}
                \item On pourrait écrire : $5 + \icon{sun} = 13$ \\
                La question deviendrait alors :
                Trouve le nombre représenté par « \icon{sun} ».
                \item On pourrait encore écrire : $5 + x = 13$ \\
                La question deviendrait alors :
                Trouve le nombre représenté par « $x$ ».
            \end{itemize}
        \end{enumerate}
    }[\href{https://clairelommeblog.fr/2020/10/11/introduire-les-relatifs-en-5e/}{Claire Lomné}]
}


\slide{}{
    \begin{enumerate}
        \setcounter{enumi}{1}
        \item Ces égalités sont des équations car elles contiennent une \key{inconnue}.
        \begin{itemize}
            \item Tu les appelais aussi des « opérations à trou ».
            \item Pour les resoudres tu te demande combien vaut $13-5$ ?
        \end{itemize}
    \end{enumerate}

}

\slide{}{
    \begin{enumerate}
        \setcounter{enumi}{2}
        \item Résout les équations suivantes :
        \multiColEnumerate{3}{
            \item $10 + x = 21$ \item $x + 6 = 13$\item $97 + y = 100$
            \item $19 = 3 + t$ \item $13+x = 13$ \item $10 + x = 4$
        }
        \item De même :
        \multiColEnumerate{2}{
            \item $x + 15 = 10$ \item $50 + y = 40$ \item $t + 19 = 0$
            \item $a + 7 = 1$
        }
    \end{enumerate}
}

% \slide{EXERCICES}{
%     \act{}{
%         Exprimer un résultat aux opérations suivante :
%         \multiColEnumerate{3}{
%             \item $9 - 3$ \item $0 - 5$ \item $10 - 12$
%             \item $0 - 9,3$ \item $100 - 250$ \item $\frac{1}{5} - \frac{3}{5}$
%         }
%     }
% }

\slide{COURS}{
    \hist{}{
        La découverte des nombres négatifs est souvant attribuée à Brahmagupta,
        Mathématicien et astronome indien du VIe siècle.\\
        Il les introduit dans le but de résoudre des oppositions autrement impossibles.
        Dans les cadres de calculs dettes notamment.
    }
    \expl{}{
        Brahmagupta possède $2\euro$ et achète $4$ baguettes à $1,5\euro$.
        Quelle quantité d'argent a-t-il après son achat ?
    }
}

\slide{}{
    \df{}{
        L'\key{opposé} d'un nombre $n$ est le nombre qui, lorsqu'il est ajouté à $n$, donne $0$.
    }[\href{https://fr.wikipedia.org/wiki/Opposé}{Wikipédia}]

    \expl{}{
        \begin{itemize}
            \item L'opposé de $7$ est \palt{2}{$-7$}.
            \item L'opposé de $-18,6$ est \palt{2}{$18,6$}.
            \item L'opposé d'un nombre $b$ est \palt{2}{$-b$}.
        \end{itemize}
    }
}

\def\imgPrefix{mm-c4/exo-}
\exoSlide{46p197,47p197,48p197}[9cm][1][\mm]
\def\imgPrefix{}

\scn{Repérage sur une droite graduée}{}

\slide{QUESTIONS FLASH}{%
    \setcounter{qf}{0}
    \sqf Ranger les nombres suivant dans l'ordre croissant les nombres :
    \begin{align*}
        1,2 \pv 6 \pv 1,15 \pv 2 \pv 100 \pv 0,584 \pv 2
    \end{align*}
}

\slide{}{%
    \sqf Donner dans chaque cas l'abscisse du point $P$:\\
    \noindent \qfs \imgp{qf-reperage-droite-a}[6cm]
    \noindent \qfs \imgp{qf-reperage-droite-b}[9cm]
    \noindent \qfs \imgp{qf-reperage-droite-c}[6cm]
}

\bsec{Repérage}
\bsubsec{Repérage sur une droite}

\slide{EXERCICES}{
    \act{}{
        \begin{enumerate}
            \item Tracer une droite numérique sur laquelle placer les points :
            $O$ d'abscisse 0, $A$ d'abscisse $2,5$
            \item Placer $B$ d'abscisse $-1$
            \item Placer $C$ d'abscisse $-2,5$
            \item Comment sont $A$ et $C$ relativement à $O$ ?
        \end{enumerate}
    }
}

\def\imgPrefix{mm-c4/expl-}
\slide{COURS}{
    \ssec\ssubsec
    \df{}{L'\key{abscisse} d'un point sur une droite graduée est sa position relativement à $0$.}
    \expl{}{
        \imgp{5p190}[10cm]
        Le point $A$ est d'abscisse \palt{2}{$-3$} et $B$ d'abscisse \palt{2}{$3$}.
        Ils ont alors la même \key{distance à $0$}, mais sont de signes différents. 
    }[\mi]
    \rmk{}{
        Sur une droite gradué, deux points qui ont des abscisses opposées sont symétrique par rapport à l'origine.
    }[\mi]
}

\def\imgPrefix{mm-c4/exo-}
\exoSlide{51p197,53p197,55p197}[7cm][2][\mi]
\def\imgPrefix{}

\scn{Repérage dans un plan}{}

\def\imgPrefix{mm-c4/exo-}
\slide{QUESTIONS FLASH}{
    % \imgp{25p196}[7cm]
    \imgp{26p196}[8cm]
    \imgp{31p196}[8cm]
}
\def\imgPrefix{}

\slide{EXERCICES}{
    \small
    \bvspace{-0.75cm}
    \act{}{
        Guybrush Threepwood est à la recherche du trésor de l'île des singes.
        Il est localisé au point $G$ donné sur le plan ci-dessous.
        
        \bvspace{-0.1cm}
        \wideFrame[6.8em]{
            \dividePage{\imgp{reperage-plan-activite}[6.7cm]}{
                \begin{enumerate}
                    \item Combien de nombres sont nécessaires pour repérer sa position ? Donner ces deux nombres pour Guybrush.
                    \item Que doit-on faire pour qu'il n'y ait pas plusieurs positions possibles repérées par ces nombres?
                    \item Repérer ainsi Guybrush et son bateau accosté au point $B$.
                    \item Demandez de l'aide au capitaine PESIN pour qu'il vous donne l'emplacement du trésor.  
                \end{enumerate}
            }[0.35]
        }
        \begin{enumerate}
            \setcounter{enumi}{4}
            \item Placer un point $T$ à l'emplacement du trésor.
        \end{enumerate}
    }
}

\bsubsec{Repérage dans le plan}

\slide{COURS}{
    \vc{}{%
        Les points du plan sont repérés par deux nombres qui forme ces \key{coordonnées}:
        \begin{itemize}
            \item L'\key{abscisse} qui ce lit sur l'axe horizontal.
            \item L'\key{ordonnée} qui ce lit sur l'axe vertical.
        \end{itemize}
    }[\mi]
}

\slide{}{
    \expl{}{
        \imgp{axes-plan}[5cm]
        $A$ est d'abscisse $3$ et d'ordonnée $2$. On note ces coordonnées $A(3;2)$.
    }[\href{https://www.maths-et-tiques.fr/telech/19Nomb_rel1.pdf}{Yvan Monka}]
}


\def\imgPrefix{mm-c4/exo-}
\exoSlide{57p197,58p198}[5.5cm][2]
\exoSlide{59p198}[6cm]