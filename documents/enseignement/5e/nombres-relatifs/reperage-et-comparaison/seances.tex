% VARIABLES %%%
\def\authors{\jules}
% \date{\today}
\def\longTitle{Nombres Relatifs : Repérage et comparaison}
\def\shortTitle{\MakeUppercase{\longTitle}}
\bseq{\longTitle}
% \def\theme{\longTitle}

\setboolean{showRef}{false}

% \def\my{Myriade 6e}
% \newcommand{\myl}[1]{\href{#1}{\my}}

\def\imgPath{enseignement/5e/nombres-relatifs/reperage-et-comparaison/}
\def\imgExtension{.png}
%%

% Yvan Monka : https://www.maths-et-tiques.fr/telech/19Nomb_rel1.pdf
% Euler : https://euler-ressources.ac-versailles.fr/wims/wims.cgi?module=help%2Fteacher%2Fprogram%2F&+cmd=new&+job=math.cycle4#chapitre000

% \disableAnimation
% \shortAnimation
% \firstSlide

\def\pv{\; ; \;}

% \article{\vskip}
\ifArticle{\vspace*{0.1cm}}

\scn{Introduction aux nombres négatifs}{}

\slide{QUESTIONS FLASH}{%
    \sqf Comparer à l'aide du signe $>$ ou $<$ ou $=$ les nombres :
    \begin{align*}
        &\qfs \; 10 \et 10,075\\
        &\qfs \; 0,5 \et \frac{1}{2}\\
        &\qfs \; \frac{6}{10} \et \frac{6}{9}
    \end{align*}
}

\slide{}{%
    \sqf Ranger les nombres suivant dans l'ordre croissant les nombres :
    \begin{align*}
        1,2 \pv 6 \pv 1,15 \pv 2 \pv 100 \pv 0,584
    \end{align*}
}

\slide{}{%
    \sqf Donner dans chaque cas l'abscisse du point $P$ :\\
    \noindent \qfs \imgp{qf-reperage-droite-a}[6cm]
    \qfs \imgp{qf-reperage-droite-b}[9cm]
    \qfs \imgp{qf-reperage-droite-c}[6cm]
}

\bsec{Définitions}

\slide{EXERCICES}{
    \vspace*{-0.5cm}
    \act{}{
        \dividePage{
            \imgp{carte}[6cm]
        }{
            \begin{enumerate}
                \item Quelles informations nous apporte ce document?
                \item Classer les nombres en deux catégories et donner un nom à chaque catégorie.
                \item Ranger les températures par ordre croissant.
            \end{enumerate}
        }
    }
    % [\href{https://www.facebook.com/groups/994675223903586/search/?q=activite\%20m%C3\%A9t\%C3\%A9o\%20nombres\%20relatifs\&locale=fr\_FR}{Vanessa Cazier}]
}

\slide{COURS}{
    \sseq\ssec
    \df{}{
        Un nombre est dit :
        \begin{itemize}
            \item \key{positif} si il est supérieur ou égal à zéro
            \item \key{négatif} si il est supérieur ou égal à zéro
        \end{itemize}
    }
}

\slide{}{
    \rmk{}{
        \begin{itemize}
            \item $O$ est à la fois positif et négatif
            \item On nomme nombres relatifs l'ensemble des nombres positifs et négatifs.
        \end{itemize}
    }
}

\slide{EXERCICES}{
    \act{}{
        Exprimer un résultat aux opérations suivante :
        \multiColEnumerate{3}{
            \item $9 - 3$ \item $0 - 5$ \item $10 - 12$
            \item $0 - 9,3$ \item $100 - 250$ \item $\frac{1}{5} - \frac{3}{5}$
        }
    }
}

\slide{COURS}{
    \hist{}{
        La découverte des nombres négatifs est souvant attribuée à Brahmagupta,
        Mathématicien et astronome indien du VIe siècle.\\
        Il les introduit dans le but de résoudre des oppositions autrement impossibles.
        Dans les cadres de calculs dettes notamment.
    }
    \expl{}{
        Brahmagupta possède $2\euro$ et achète $4$ baguettes à $1,5\euro$.
        Quelle quantité d'argent a-t-il après son achat ?
    }
}

\scn{Repérage}{}

\bsec{Repérage}

\slide{EXERCICES}{
    \act{}{
        \begin{enumerate}
            \item Tracer une droite numérique sur laquelle placer les points :
            $A$ d'abscisse 1, $B$ d'abscisse $1,5$
            \item Placer $C$ d'abscisse $0$
            \item Placer $D$ d'abscisse $-1$
            \item Placer $E$ d'abscisse $-3,5$
        \end{enumerate}
    }
}

\slide{COURS}{
    \ssec
}

\slide{EXERCICES}{
    \exo{Maths Monde 51 p197}{}
}