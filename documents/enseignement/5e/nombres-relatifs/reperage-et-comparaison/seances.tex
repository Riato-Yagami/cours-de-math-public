% VARIABLES %%%
\def\authors{\jules}
% \date{\today}
\def\longTitle{Nombres Relatifs - Repérage et comparaison}
\def\shortTitle{\MakeUppercase{\longTitle}}
\bseq{\longTitle}
% \def\theme{\longTitle}

\setGrade{5e}

\def\imgPath{enseignement/5e/nombres-relatifs/reperage-et-comparaison/}
\def\imgExtension{.png}

\setboolean{showAllPhases}{true}
% \def\showPhase{EXERCICES}
%%

% Yvan Monka : https://www.maths-et-tiques.fr/telech/19Nomb_rel1.pdf

% \article{\vskip}
\ifArticle{\vspace*{0.1cm}}

\obj{
    \item Introduire les nombres relatifs.
    \item Repérer sur une droite graduée les nombres décimaux relatifs.
    \item Repérer sur une droite graduée et dans le plan muni d'un repère orthogonal.
    \item Utiliser la notion d'opposé.
}

\scn{Introduction aux nombres négatifs}

\bsec{Définitions}

\slide{exo}{
    \vspace*{-0.5cm}
    \act{}{
        \dividePage{
            \imgp{carte}[6cm]
        }{
            \begin{enumerate}
                \item Quelles informations nous apporte ce document?
                \item Classer les nombres en deux catégories et donner un nom à chaque catégorie.
                \item Ranger les températures par ordre croissant.
            \end{enumerate}
        }
    }
    % [\href{https://www.facebook.com/groups/994675223903586/search/?q=activite\%20m%C3\%A9t\%C3\%A9o\%20nombres\%20relatifs\&locale=fr\_FR}{Vanessa Cazier}]
}

\scn{Institutionnalisation des nombres relatifs}{}

\slide{cr}{
    \sseq\ssec
    \df{}{
        Un nombre est dit :
        \begin{itemize}
            \item \key{positif} si il est supérieur ou égal à zéro
            \item \key{négatif} si il est supérieur ou égal à zéro
        \end{itemize}
    }
}

\slide{cr}{
    \rmk{}{
        \begin{itemize}
            \item $O$ est à la fois positif et négatif
            \item On nomme nombres relatifs l'ensemble des nombres positifs et négatifs.
            \item On parle de nombres relatifs car on les considère relativement à zéro.
        \end{itemize}
    }
}

\def\imgPrefix{mm-c4/exo-}
\exoSlide{28p196,29p196,30p196}[9cm][1][\mm]
\def\imgPrefix{}

\scn{Résolution d'équations grâce aux nombres relatifs}{}

\slide{qf}{%
    \sqf Comparer à l'aide du signe $>$ ou $<$ ou $=$ les nombres :
    \begin{align*}
        &\qfs \; 10 \hole 10,075\\
        &\qfs \; 0,5 \hole \dfrac{1}{2}\\
        &\qfs \; \dfrac{6}{10} \hole \dfrac{6}{9}
    \end{align*}
}

\slide{cr}{
    \sqf Completer les opérations à trous :
    \begin{align*}
        &\qfs \; 6 + \hole = 18\\
        &\qfs \; 710,5 + \hole = 770\\
        &\qfs \; 30 - \hole = 22,2
    \end{align*}
}

\slide{exo}{
    \act{}{
        \begin{enumerate}
            \item $ 5 + \textrm{?} = 13$ : Trouve le nombre représenté par « ? »
            \begin{itemize}
                \item On pourrait écrire : $5 + \icon{sun} = 13$ \\
                La question deviendrait alors :
                Trouve le nombre représenté par « \icon{sun} ».
                \item On pourrait encore écrire : $5 + x = 13$ \\
                La question deviendrait alors :
                Trouve le nombre représenté par « $x$ ».
            \end{itemize}
        \end{enumerate}
    }[\href{https://clairelommeblog.fr/2020/10/11/introduire-les-relatifs-en-5e/}{Claire Lomné}]
}


\slide{exo}{
    \begin{enumerate}
        \setcounter{enumi}{1}
        \item Ces égalités sont des équations car elles contiennent une \key{inconnue}.
        \begin{itemize}
            \item Tu les appelais aussi des « opérations à trou ».
            \item Pour les résoudre, tu te demandes combien vaut $13-5$ ?
        \end{itemize}
    \end{enumerate}

}

\slide{exo}{
    \begin{enumerate}
        \setcounter{enumi}{2}
        \item Résout les équations suivantes :
        \multiColEnumerate{3}{
            \item $10 + x = 21$ \item $x + 6 = 13$\item $97 + y = 100$
            \item $19 = 3 + t$ \item $13+x = 13$ \item $10 + x = 4$
        }
        \item De même :
        \multiColEnumerate{2}{
            \item $x + 15 = 10$ \item $50 + y = 40$ \item $t + 19 = 0$
            \item $a + 7 = 1$
        }
    \end{enumerate}
}

% \slide{EXERCICES}{
%     \act{}{
%         Exprimer un résultat aux opérations suivante :
%         \multiColEnumerate{3}{
%             \item $9 - 3$ \item $0 - 5$ \item $10 - 12$
%             \item $0 - 9,3$ \item $100 - 250$ \item $\dfrac{1}{5} - \dfrac{3}{5}$
%         }
%     }
% }

\slide{cr}{
    \hist{}{
        L'invention des nombres négatifs est souvent attribuée à Brahmagupta,
        mathématicien et astronome indien du VIe siècle.\\
        Il les introduit dans le but de résoudre des opérations autrement impossibles.
        Dans les cadres de calculs dettes notamment.
    }
    \expl{}{
        Brahmagupta possède $2\euro$ et achète $4$ baguettes à $1,5\euro$ l'unité.
        Quelle quantité d'argent a-t-il après son achat ?
    }
}

\slide{cr}{
    % \df{}{
    %     L'\key{opposé} d'un nombre $n$ est le nombre qui, lorsqu'il est ajouté à $n$, donne $0$.
    % }[\href{https://fr.wikipedia.org/wiki/Opposé}{Wikipédia}]

    \df{}{
        Deux nombres dont la somme est égale à 0 sont dits \key{opposé}.
    }

    \expl{}{
        \begin{itemize}
            \item L'opposé de $7$ est \palt{2}{$-7$}.
            \item L'opposé de $-18,6$ est \palt{2}{$18,6$}.
            \item L'opposé d'un nombre relatif $b$ est \palt{2}{$-b$}.
        \end{itemize}
    }
}

\def\imgPrefix{mm-c4/exo-}
\exoSlide{46p197,47p197,48p197}[9cm][1][\mm]
\def\imgPrefix{}

\scn{Repérage sur une droite graduée}{}

\slide{qf}{%
    \setcounter{qf}{0}
    \sqf Ranger les nombres suivant dans l'ordre croissant les nombres :
    \begin{align*}
        1,2 \pv 6 \pv 1,15 \pv 2 \pv 100 \pv 0,584 \pv 2
    \end{align*}
}

\slide{qf}{%
    \sqf Donner dans chaque cas l'abscisse du point $P$:\\
    \noindent \qfs \imgp{qf-reperage-droite-a}[6cm]
    \noindent \qfs \imgp{qf-reperage-droite-b}[9cm]
    \noindent \qfs \imgp{qf-reperage-droite-c}[6cm]
}

\bsec{Repérage}
\bsubsec{Repérage sur une droite}

\slide{exo}{
    \act{}{
        \begin{enumerate}
            \item Tracer une droite numérique sur laquelle placer les points :
            $O$ d'abscisse 0, $A$ d'abscisse $2,5$
            \item Placer $B$ d'abscisse $-1$
            \item Placer $C$ d'abscisse $-2,5$
            \item Comment sont $A$ et $C$ relativement à $O$ ?
        \end{enumerate}
    }
}

\def\imgPrefix{mm-c4/expl-}
\slide{cr}{
    \ssec\ssubsec
    \df{}{L'\key{abscisse} d'un point est le nombre qui permet de repérer ce point sur la droite graduée.}
    \bvspace{-1cm}
    \expl{}{\bvspace{-0.5cm}\imgp{5p190}[10cm]
        Le point $A$ est d'abscisse \palt{2}{$-3$} et $B$ d'abscisse \palt{2}{$3$}.
        Ils ont alors la même \key{distance à $0$}, mais sont de signes différents. 
    }[\mi]
}

\slide{}{
    \rmk{}{
        Sur une droite graduée, deux points qui ont des abscisses opposées sont symétrique par rapport à l'origine.
    }[\mi]
}

\def\imgPrefix{mm-c4/exo-}
\exoSlide{51p197,53p197,55p197}[7cm][2][\mi]
\def\imgPrefix{}

\scn{Découverte repérage dans un plan}{}

\bsubsec{Repérage dans le plan}

\def\imgPrefix{mm-c4/exo-}
\exoSlide{26p196,31p196}[8cm][1][\mi][qf]

\def\imgPrefix{}

\slide{exo}{
    \small
    \bvspace{-0.75cm}
    \act{}{Guybrush, repéré par le point $G$. est à la recherche du trésor de l'île des singes.

        \bvspace{-0.1cm}
        \wideFrame[6.8em]{
            \dividePage{\imgp{reperage-plan-activite}[6.7cm]}{
                \begin{enumerate}
                    \item Combien de nombres sont nécessaires pour repérer sa position ? Donnez ces nombres pour Guybrush.
                    \item Que doit-on faire pour qu'il n'y ait pas plusieurs positions possibles repérées par ces nombres?
                    \item Repérer ainsi Guybrush et son bateau accosté au point $B$.
                    \item Demandez de l'aide au capitaine PESIN pour qu'il vous donne l'emplacement du trésor.  
                \end{enumerate}
            }[0.35]
        }
        \begin{enumerate}
            \setcounter{enumi}{4}
            \item Placer un point $T$ à l'emplacement du trésor.
        \end{enumerate}
    }
}

\scn{Institutionnalisation repérage dans un plan}{}

\def\imgPrefix{mm-c4/exo-}
\exoSlide{64p198}[6cm][1][\mm][qf]
\def\imgPrefix{}

\slide{cr}{
    \vc{}{%
        Les points du plan sont repérés par deux nombres qui forment ces \key{coordonnées}:
        \begin{itemize}
            \item L'\key{abscisse} qui se lit sur l'axe horizontal.
            \item L'\key{ordonnée} qui se lit sur l'axe vertical.
        \end{itemize}
    }[\mi]
}

\slide{cr}{
    \expl{}{
        \imgp{axes-plan}[5cm]
        $A$ est d'abscisse $3$ et d'ordonnée $2$. On note ces coordonnées $A(3;2)$.
    }[\href{https://www.maths-et-tiques.fr/telech/19Nomb_rel1.pdf}{Yvan Monka}]
}

\def\imgPrefix{mm-c4/exo-}
\exoSlide{57p197,58p198}[5.5cm][2][\mi]
\exoSlide{59p198}[6cm][1][\mi]

\scn{Comparaison de nombres relatifs}{}

\bsec{Comparaison de nombres relatifs}
\slide{cr}{
    \ssec
    \mthd{}{
        Lorsque l'on compare  deux nombres relatifs,
        si les deux nombres sont :
        \begin{itemize}
            \item positifs, le plus grand est celui avec la plus grande distance à 0.
            \item négatifs, le plus grand est celui avec la plus petite distance à 0.
            \item de signes opposées, le nombre positif est toujours plus grand.
        \end{itemize}
    }
}

\slide{cr}{
    \expl{Comparer :}{
        \multiColEnumerate{3}{
            \item $10$ et $ 10,09$
            \item $-1$ et $-1,1$
            \item $-100$ et $50$
        }
    }
    \rmk{Comparaison avec droite graduée}{Le point le plus à droite correspond au nombre le plus grand.}
}

\exoSlide{66p198}[6cm][1][\mi]

\exoSlide{70p198,87p200}[6cm][2][\mi][dm]