% VARIABLES %%%
\setTitle{Correction - Devoir Maison - Séquence 1}
\setgrade{5e}
\def\imgPath{enseignement/5e/nombres-relatifs/reperage-et-comparaison/}
%%

\exo{70p198}{
    \imgp{mm-c4/exo-70p198}[6cm]
    Il y a \key{7 chemins} possible :
    \multiColEnumerate{2}{
        \item $-7,8 < -5,2 < -3 < 2$
        \item $-7,8 < -5,2 < -3 < 0 < 2$
        \item $-7,8 < -5,2 < -3 < -2 < 0 < 2$
        \item $-7,8 < -6,5 < -5,2 < -3 < 2$
        \item $-7,8 < -6,5 < -5,2 < -3 < 0 < 2$
        \item $-7,8 < -6,5 < -5,2 < -3 < -2 < 0 < 2$
        \item $-7,8 < -6,5 < -2 < 0 < 2$
    }
}

\exo{87p200}{
    \imgp{mm-c4/exo-87p200}[6cm]

    \begin{enumerate}
        \item On peut trouver l'altitude du fond du lac Ontario en enlevant sa profondeur à l'altitude de sa surface.\\
        74,2 - 244 = -169,8\\
        Le fond du lac Ontario se trouve à une altitude de -169,8 \meter.
        \item On peut trouver l'altitude du sommet des Chutes du Niagara en ajoutant leurs hauteur à l'altitude de la surface du lac Ontario.\\
        74,2 + 52 = 126.2\\
        Le sommet des Chutes du Niagara se trouve à une altitude de 126,2 \meter.
    \end{enumerate}
}