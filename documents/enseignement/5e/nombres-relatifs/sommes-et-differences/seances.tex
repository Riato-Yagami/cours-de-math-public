% VARIABLES %%%
\setSeq{4}{Nombres Relatifs - Sommes et différences}
\setGrade{5e}

% \setboolean{answer}{false}
% \setboolean{newPageOnSlide}{true}

% \forPrint

\def\imgPath{enseignement/5e/nombres-relatifs/sommes-et-differences/}
\def\ym{\href{https://www.maths-et-tiques.fr/telech/19Nomb_rel2.pdf}{Yvan Monka}}
% Yvan Monka RÈGLES DE CALCUL : https://www.maths-et-tiques.fr/telech/19Calcul_num.pdf
%%

\obj{
    \item Traduire un enchaînement d'opérations à l'aide d'une expression avec des parenthèses.
    \item Effectuer mentalement, à la main ou l'aide d'une calculatrice un enchaînement d'opérations.
    \item Additionner et soustraire des nombres décimaux relatifs.
    \item Résoudre des problèmes faisant intervenir des nombres décimaux relatifs et des fractions simples.
    \item Utiliser la notion de fractions quotients dans des calculs.
}

\scn{Rappel sur les règles opératoires}

\slide{qf}{
    \multiColEnumerate{2}{
        \item $8 + 9 - 7 = \nswr{10}$
        \item $8 \times (2 + 10) = \nswr{96}$
        \item $12 + 11 \times 10 = \nswr{122}$
        \item $7 \div (7 - 3 + 6) = \nswr{\dfrac{7}{10} = 0,7}$
        \item $9 - 8 + 1 = \nswr{2}$
    }
}

\bsec{Opérations}
\bsubsec{Règles opératoires}

\slide{cr}{
    \sseq\ssec\ssubsec

    \rl{}{
        Les calculs se font dans l'ordre des priorités suivant:%
        \begin{enumerate}
            \item La multiplication et la division
            \item L'addition et la soustraction
        \end{enumerate}
    }
}

\slide{cr}{
    \rl{}{
        En cas d'opérations de mêmes priorités, on effectue les opérations de gauche à droite.
    }

    \expl{}{
        \multiColEnumerate{1}{
            \item $3 - 2 + 3 = \nswr{1 + 3 = 4}$ 
            \item $19 - 6\times 3 = \nswr{19 - 18 =  1}$
            \item $3 + \np{3.2} \times 2 - 4 = \nswr{3 + \np{6.4} -4 = \np{9.4} - 4 = \np{5.4}}$
        }
    }
}

\slide{cr}{
    \rl{}{
        On commence par effectuer les calculs entre parenthèses.
    }

    \expl{}{
        \multiColEnumerate{1}{
            \item $(1 + 2) \times 21 = \nswr{3 \times 21 = 63}$
            \item $(11 \times 3) + (15 \div 2) = \nswr{33 + 7.5 = 40.5}$
            \item $((13 - (3 - 2)) + 2) = \nswr{(13 - 1) + 2 = 12 + 2 = 14}$
        }
    }
}

\scn{Rappel sur le vocabulaire opératoire}

\slide{qf}{Ecrire ces nombres sous forme décimale :
    \multiColEnumerate{2}{
        \item $\dfrac{2}{100} = \nswr{\np{0.02}}$
        \item $30 \div 4 = \nswr{\np{7.5}}$
        \item $\dfrac{9}{18} = \nswr{\np{0.5}}$
        \item $\dfrac{4}{2 \times 5} = \nswr{\np{0.4}}$
    }
}

\bsubsec{Vocabulaire opératoire}

% \newpage

\slide{cr}{\bsmall
    \ssubsec
    \bvspace{-0.5cm}
    \vc{}{
        On connait quatres types d'opérations :
        \begin{itemize}
            \item L'\key{addition} permet de calculer la \key{somme} de deux \key{termes}.
            \item La \key{soustraction}  permet de calculer la \key{différence} entre deux \key{termes}.
            \item La \key{multiplication} permet de calculer le \key{produit} de deux \key{facteurs}.
            \item La \key{division} permet de calculer le \key{quotient} de deux \key{nombres}.
        \end{itemize}
    }
}

\slide{cr}{\bshrink
    \vc{}{Dans un calcul,
    le type de la dernière opération effectuée détermine le nom donné au calcul dans son ensemble.}
    \bvspace{-0.5cm}
    \expl{Nommer les calculs suivants}{%\bvspace{-0.5cm}
        \multiColEnumerate{1}{
            \item $1,6 + 4$ est \nswr{la somme} de \nswr{$1,6$ et $4$}.
            \item $(\dfrac{2}{6} + 3) \times 9$ est \nswr{le produit } de \nswr{$\dfrac{2}{6} + 3$ et $9$}.
            \item $6,6 + 1 \times 8$ est \nswr{la somme de $6,6$ par $1 \times 8$}.
            \item $\dfrac{2}{6} + 3 - 9$ est \nswr{la différence entre $\dfrac{2}{6}$ et $9$}.
            \item $\pi \div (3 - 9)$ est \nswr{le quotient de $\pi$ par $(3 - 9)$}.
        }
    }
}

\scn{Tournois \icon{RELATIvs/logo}}

\scn{Cours sur la somme et différence de nombres relatifs}

\slide{qf}{Le bulletin d'Etienne contient ces notes :
\begin{center}
    \begin{tabular}{|>{\bfseries}c|*{4}{c|}} % Colonne en gras pour la première colonne
        \hline
        \rowcolor{gray!15} 
        Note /20 & 10 & 15 & 5 & 20 \\ \hline
        Coefficient  & 2 & 1 & 3 & 1 \\ \hline
    \end{tabular}
\end{center}
    Quelle est sa moyenne ?\\
    \nswr[0]{$\dfrac{10 \times 2 + 15 + 5 \times 3 + 20}{6} = \dfrac{70}{7} = 10$}
}

\bsec{Opérations sur les nombres relatifs}
\bsubsec{Somme}

\def\daz{distance à zéro}

\slide{cr}{\ifBeamer{\footnotesize}
    \ssec\ssubsec
    \pr{Somme de nombres relatifs}{
        Pour deux nombres relatifs :
        \begin{itemize}
            \item S'ils ont le \key{même signe} :
            \begin{itemize}\ifBeamer{\scriptsize}
                \item Le signe du résultat est identique à celui des deux nombres.
                \item La \daz{} du résultat est obtenue en additionnant les \daz{} des deux nombres.
            \end{itemize} 
            \item S'ils ont des \key{signes opposés} :
            \begin{itemize}\ifBeamer{\scriptsize}
                \item Le signe du résultat est celui du nombre ayant la plus grande \daz.
                \item La \daz{} du résultat est obtenue en soustrayant la plus petite \daz{} de la plus grande.
            \end{itemize}
        \end{itemize}
    }
}

\slide{cr}{
    \expl{}{
        \multiColEnumerate{2}{
            \item $-10 + (-98) = \nswr{108}$
            \item $854 + 46 = \nswr{900}$
            \item $-91 + 11 = \nswr{80}$
            \item $-\np{10.5} + 12 = \nswr{\np{1.5}}$
            \item $\np{0.005} + (-1) = \nswr{-\np{0.995}}$
            \item $-9 + (-10) + (-11) = \nswr{-30}$
        }
    }
}

\slide{exo}{
    \exo{Effectuer les calculs suivants}{
        \multiColEnumerate{2}{
            \item $12 + (-11) + 25 + (-17) \nswr[0]{ = 9}$
            \item $14 + (-7) + 2 + (-3,75) + (-5,25) \nswr[0]{ = 0}$
            \item $(-2,1) + (-9) + 6,4 + (-8,3) \nswr[0]{ = -13}$
            \item $3,6 + 30 + (-6,4) + (-49) \nswr[0]{ = -\np{21.8}}$
        }
    }[\iP{5}{2022}[1][56]]
}

\bsubsec{Différence}

\slide{cr}{
    \ssubsec
    \pr{Différence de nombres relatifs}{
        Pour deux nombres relatifs $a$ et $b$:
        La différence de $a$ et $b$ est équivalente à la somme de $a$ et $-b$
    }

    \expl{}{
        \multiColEnumerate{2}{
            \item $9 - 10 = \nswr{9 + (-10) = -1}$
            \item $-\np{3.2} - \np{9} = \nswr{-\np{3.2} + (-\np{9}) = \np{12.2}}$
            \item $9 - (-\np{81.2}) = \nswr{9 + \np{81.2} = -\np{90.2}}$
            \item $-100 - (-12) = \nswr{-100 + 12 = -88}$
        }
    }
}

\slide{cr}{
    \rmk{}{
        Pour un nombre relatif $a$.\\
        On a : $-(-a) = a$
    }

    \expl{}{
        \multiColEnumerate{2}{
            \item $-(-10) = \nswr{10}$
            \item $-(-4 + 2) = \nswr{-(-2) = 2}$
            \item $-(9+10) = \nswr{-(19) = -19}$
            \item $-(-(-10)) = \nswr{-(10) = -10}$
        }
    }
}

\slide{exo}{\bshrink
    \exo{Pyramides de nombres}{
        Complète sachant que chaque nombre est la somme des nombres se trouvant dans les deux cases juste en dessous.
        \multiColEnumerate{2}{
            \item \PyramideNombre[Largeur=1.5cm,Etages=4]{%
            -1,3,-5,-10,%
            \nswr{2},\nswr{-2},\nswr{-15},%
            \nswr{0},\nswr{-17},%
            \nswr{-17}}
            \item \PyramideNombre[Largeur=1.5cm,Etages=4]{%
            6,\nswr{-9},\nswr{10},-15,%
            -3,\nswr{1},-5,%
            \nswr{-2},\nswr{-4},%
            \nswr{-6}}\saveenumi
        }
    }
}

\scn{Exercices et problèmes mettant en jeu des nombres relatifs}

\slide{exo}{
    \begin{enumerate}\loadenumi[exo]
        \item \PyramideNombre[Largeur=1.5cm,]{%
        -3,\nswr{-5},\nswr{29},\nswr{-102},\nswr{328},%
        -8,\nswr{24},\nswr{-73},\nswr{226},%
        16,\nswr{-49},\nswr{153},%
        -33,\nswr{104},%
        71
        }
    \end{enumerate}
}

% \scn{Moyenne de nombres relatifs}

% \newpage
\slide{exo}{\ifBeamer{\vspace{-0.625cm}\scriptsize}
    \exo{Climat de Février aux alentours d'Issy-les-Moulineaux}{
        Le graphique ci-dessous présente les températures relevées à 9h à la base aérienne de Vélizy du 5 au 10 février 2021.
        \ifBeamer{Quelle était environ la température moyenne à 9h entre le 5 et le 10 février ?}
        \bvspace{-0.45cm}
        \begin{center}
            % \begin{tikzpicture}[yscale = 0.75, xscale = 1.25]
            \begin{tikzpicture}[yscale = 0.32, xscale = 1.25]
                \tkzInit[xmin=4,xmax=11,ymin=-8,ymax=8]
                \tkzGrid[sub,color=gradeColor!50!white,subxstep=1,subystep=0.5]        
                \tkzLabelX[step=1]
                \tkzLabelY[step=2]
                \tkzDrawY[label={Températures (en \Temp{})}, above , step=0.5]
                \tkzDrawX[label={Jour}, below right, step=1]
                \drawPoint{}{5-4}{7}
                \drawPoint{}{6-4}{7.5}
                \drawPoint{}{7-4}{4.5}
                \drawPoint{}{8-4}{-0.5}
                \drawPoint{}{9-4}{-2}
                \drawPoint{}{10-4}{-7}
                % \drawPoint{}{11}{-6}
                % \drawPoint{}{12}{-4}
                % \drawPoint{}{13}{-6}
                % \drawPoint{}{14}{-6}
                % \drawPoint{}{15}{2.5}
                % \drawPoint{}{16}{6}
        \end{tikzpicture}
    \end{center}
    \ifArticle{Quelle était environ la température moyenne à 9h entre le 5 et le 10 février ?\\}
    }[\href{https://fr.weatherspark.com/h/m/147944/2021/2/Météo-historique-en-février-2021-à-Base-aérienne-de-Vélizy---Villacoublay-France\#Figures-Temperature}{Weather Spark}]
    Pour obtenir la moyenne de températures,
    on peut ajouter toutes les températures et les diviser par le nombre de températures.
    \nswr[0]{
        \begin{align*}
            \dfrac{7+\np{7.5}+\np{4.5}+(\np{-0.5})+(-2)+(-7)}{6}
            = \dfrac{9.5}{6}
            \approx \np{1.58}
        \end{align*}
        La température moyenne à 9h était d'environ \Temp{1.58}.
    }
}

% \scn{Fraction quotient}