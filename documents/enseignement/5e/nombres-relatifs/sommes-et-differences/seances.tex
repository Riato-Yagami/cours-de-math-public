% VARIABLES %%%
\setSeq{4}{Nombres Relatifs - Sommes et différences}
\setGrade{5e}

% \setboolean{answer}{false}
% \setboolean{newPageOnSlide}{true}

% \forPrint

\def\imgPath{enseignement/5e/nombres-relatifs/sommes-et-differences/}
\def\ym{\href{https://www.maths-et-tiques.fr/telech/19Nomb_rel2.pdf}{Yvan Monka}}
% Yvan Monka RÈGLES DE CALCUL : https://www.maths-et-tiques.fr/telech/19Calcul_num.pdf
%%

\obj{
    \item Traduire un enchaînement d'opérations à l'aide d'une expression avec des parenthèses.
    \item Effectuer mentalement, à la main ou l'aide d'une calculatrice un enchaînement.
    \item Additionner et soustraire des nombres décimaux relatifs.
    \item Résoudre des problèmes faisant intervenir des nombres décimaux relatifs et des fractions simples.
}

\slide{qf}{
    \multiColEnumerate{2}{
        \item $8 + 9 - 7 = \bawsr{10}$
        \item $8 \times (2 + 10) = \bawsr{96}$
        \item $12 + 11 \times 10 = \bawsr{122}$
        \item $7 \div (7 - 3 + 6) = \bawsr{\frac{7}{10} = 0,7}$
        \item $9 - 8 + 1 = \bawsr{2}$
    }
}

\bsec{Opérations}
\bsubsec{Règles opératoires}

\slide{cr}{
    \sseq\ssec\ssubsec

    \rl{}{
        Les calculs se font dans l'ordre des priorités suivant:%
        \begin{enumerate}
            \item La multiplication et la division
            \item L'addition et la soustraction
        \end{enumerate}
    }
}

\slide{cr}{
    \rl{}{
        En cas d'opérations de mêmes priorités, on effectue les opérations de gauche à droite.
    }

    \expl{}{
        \multiColEnumerate{1}{
            \item $3 - 2 + 3 = \bawsr{1 + 3 = 4}$ 
            \item $19 - 6\times 3 = \bawsr{19 - 18 =  1}$
            \item $3 + \num{3.2} \times 2 - 4 = \bawsr{3 + \num{6.2} -4 = \num{9.4} - 4 = \num{5.4}}$
        }
    }
}

\slide{cr}{
    \rl{}{
        On commence par effectuer les calculs entre parenthèses.
    }

    \expl{}{
        \multiColEnumerate{1}{
            \item $(1 + 2) \times 21 = \bawsr{3 \times 21 = 63}$
            \item $(11 \times 3) + (15 \div 2) = \bawsr{33 + 7.5 = 40.5}$
            \item $((13 - (3 - 2)) + 2) = \bawsr{(13 - 1) + 2 = 12 + 2 = 14}$
        }
    }
}

\bsubsec{Vocabulaire opératoires}

% \newpage

\slide{cr}{
    \ssubsec
    \bvspace{-0.5cm}
    \vc{}{
        On connait quatres types d'opérations :
        \begin{itemize}
            \item L'\key{addition} permet de calculer la \key{somme} de deux \key{termes}.
            \item La \key{soustraction}  permet de calculer la \key{différence} entre deux \key{termes}.
            \item La \key{multiplication} permet de calculer la \key{produit} de deux \key{facteurs}.
            \item La \key{division} permet de calculer la \key{quotient} de deux \key{nombres}.
        \end{itemize}
    }
}

\slide{cr}{
    \vc{}{Dans un calcul,
    le type de la dernière opération effectuée détermine le nom donné au calcul dans son ensemble.}
    \bvspace{-0.5cm}
    \expl{Nommer les calculs suivants}{
        \multiColEnumerate{1}{
            \item $1,6 + 4$ est \bawsr{la somme} de \bawsr{$1,6$ et $4$}.
            \item $(\frac{2}{6} + 3) \times 9$ est \bawsr{le produit } de \bawsr{$\frac{2}{6} + 3$ et $9$}.
            \item $6,6 + 1 \times 8$ est \bawsr{la somme de $6,6$ par $1 \times 8$}.
            \item $\frac{2}{6} + 3 - 9$ est \bawsr{la différence entre $\frac{2}{6}$ et $9$}..
            \item $\pi \div (3 - 9)$ est \bawsr{le quotient de $\pi$ par $(3 - 9)$}.
        }
    }
}

% \slide{cr}{

% }

\bsec{Sommes et différences de nombres relatifs}

% \slide{exo}{
%     \act{}{

%     }
% }