% VARIABLES %%%
% \def\authors{\jules \ et \href{http://www.cellulegeometrie.eu/documents/pub/pub_14.pdf}{la Haute École en Hainaut}}
\setGrade{5e}
\def\assignmentNameWidth{6cm}
\tp{Scratch - Triangles}
\def\imgPath{enseignement/6e/geometrie-plane/frises/}
%%

% DOC https://ctan.math.illinois.edu/macros/latex/contrib/scratch3/scratch3-fr.pdf

\setscratch{scale=.75}

\def\block{{\setscratch{scale=.5}\begin{scratch}\blockmove{\Large bloc}\end{scratch} }}

\newcommand{\scr}[1]{\begin{scratch}#1\end{scratch}}

\definecolor{smotion}{HTML}{4C97FF} % #4C97FF
\definecolor{slooks}{HTML}{9966FF} % #9966FF
\definecolor{ssound}{HTML}{D65CD6} % #D65CD6
\definecolor{sevents}{HTML}{FFD500} % #FFD500
\definecolor{scontrol}{HTML}{FFAB19} % #FFAB19
\definecolor{ssensing}{HTML}{4CBFE6} % #4CBFE6
\definecolor{soperators}{HTML}{6DB26E} % #6DB26E
\definecolor{svariables}{HTML}{F28011} % #F28011
\definecolor{smyblocks}{HTML}{FF6680} % #FF6680

\def\smotion{\textcolor{smotion}{\faCircle\,Mouvement}} % Déplacement du lutin
\def\slooks{\textcolor{slooks}{\faCircle\,Apparence}} % Modifier l'apparence du lutin ou de la scène
\def\ssound{\textcolor{ssound}{\faCircle\,Son}} % Jouer des sons ou de la musique
\def\sevents{\textcolor{sevents}{\faCircle\,Événement}} % Déclencher des scripts en réponse à des actions
\def\scontrol{\textcolor{scontrol}{\faCircle\,Contrôle}} % Boucles, conditions, et contrôle du flux
\def\ssensing{\textcolor{ssensing}{\faCircle\,Capteur}} % Réagir à des informations extérieures ou internes
\def\soperators{\textcolor{soperators}{\faCircle\,Opérateur}} % Calculs mathématiques et logiques
\def\svariables{\textcolor{svariables}{\faCircle\,Variable}} % Stockage et manipulation de données
\def\smyblocks{\textcolor{smyblocks}{\faCircle\,Mes blocs}} % Création de blocs personnalisés

\def\spen{{\icon{scratch/pen} Stylo}}
\def\spenExtension{{\icon{scratch/pen-extension} Stylo}}
\def\sextensions{{\icon{scratch/extensions} $\lbrack$ Ajouter une extensions $\rbrack$}}
\def\sflag{{\icon{scratch/flag}%
%  Drapeau
}}

% \setscratch{scale=.75}
% \setscratch{print=true}
% \setscratch{fill blocks=true}

\hint{
    \begin{itemize}
        \item Bien lire les indications.
        \item Répondre aux questions sur son cahier d'exercices.
        \item Appeler M. Pesin à la fin de chaque partie.
        \item Enregistre tes productions avec le nom : "NOM.S-tp-frises-partie-X.ggb"
    \end{itemize}
}

\section{Tracer un triangle équilatéral à l'aide de Scratch} 

\begin{enumerate}
    \item Dessine un triangle équilatéral à main levée.
    \item Note les propriétés d'un triangle équilatéral.
    \item Trace un triangle équilatéral à l'aide de ta règle et ton compas.
    \item Mesure ses angles avec un rapporteur et vérifie si tes mesures confirment les propriétés d'un triangle équilatéral.
    \item Assemble un script dans \Scratch{} pour que le lutin dessine un triangle équilatéral.
    \item Assemble un script dans \Scratch{} pour que le lutin dessine un triangle équilatéral. 
    \hint{\begin{enumerate} 
        \item Utilise une boucle pour répéter les instructions nécessaires au tracé. 
        \item Réfléchis attentivement à l'angle de rotation. Imagine que tu es à la place du lutin et que tu suis les instructions de ton programme : comment devrais-tu te déplacer et tourner pour revenir à ta position de départ ? 
    \end{enumerate}} 
    
\end{enumerate}

\section{Frises}

\begin{enumerate}
    \item Ecrit un programme permettant de dessiner cette frise :
    \ctikz[0.85]{
        % Données des triangles
        \def\side{2} % Longueur du côté du triangle
        \def\spacing{1} % Espacement entre les triangles
        % Calcul des coordonnées
        \foreach \i in {0, 1, 2, 3} {
            % Position de la base gauche du triangle
            \pgfmathsetmacro{\xBase}{\i * (\side + \spacing)}
            \pgfmathsetmacro{\yBase}{0}
            % Dessin du triangle équilatéral
            \draw (\xBase, \yBase) -- 
                ({\xBase + \side}, \yBase) -- 
                ({\xBase + \side / 2}, {\yBase + \side * sqrt(3) / 2}) -- 
                cycle;
        }
    }
    \item Et cette frise :
    \ctikz[0.55]{
        \draw[] (0,0) grid (5,1);
    }
    \item Et ce pavage :
    \ctikz[0.5]{
        \draw[] (0,0) grid (5,8);
    }
\end{enumerate}

