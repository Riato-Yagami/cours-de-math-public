\setSeq{8}{Triangles}
\setGrade{5e}
\def\imgPath{enseignement/5e/ecriture-fractionnaire/comparaison/}

\dym{https://www.maths-et-tiques.fr/telech/19Triangles1.pdf}
% https://www.maths-et-tiques.fr/telech/19Triangles2.pdf

% \forStudents
% \forPrint

\def\caPrefix{5e-juin-2023-}

\obj{
    \item Mettre en œuvre et écrire des protocoles de construction de triangles à partir de leurs propriétés.
    \item Connaître l'inégalité triangulaire.
    \item Construire les médiatrices d'un triangle.
    \item Construire les hauteurs d'un triangle.
    \item Calculer les aires de triangles.
    \item Utiliser la propriété de la somme des angles d'un triangle pour raisonner.
}

\scn{Construire des triangles}
\caSlide{4-5-6}

\slide{exo}{
    \exo{Constructions de triangles}{
    Tracer les triangles :
    \begin{enumerate}
        \item $ABC$ tel que $AB = \Lg{5}$, $AC = \Lg{4}$ et $BC = \Lg{6}$.
        \item $RST$ tel que : $RT = \Lg{6}$, $ST = \Lg{4}$ et $\widehat{RTS} = \ang{70}$.
        \item $EFG$ tel que : $EF = \Lg{7}$, $\widehat{FEG} = \ang{110}$ et $\widehat{EFG} = \ang{40}$.
    \end{enumerate}
}[\ym]
}

\scn{Découvrir l'inégalité triangulaire}
\caSlide{7-8-9}

\slide{exo}{
    \def\figScale{0.7}
\act{Solid Snake}{
    L'agent Solid Snake se trouve au point $S$ et doit rejoindre les escaliers de sortie situés au point $E$ le plus rapidement possible.
    Aide-le en traçant sur les plans de la base militaire le chemin le plus court entre les points $S$ et $E$.
    \hint{Attention, Solid Snake ne peut pas traverser les murs, mais il peut les longer !}
    \begin{center}
        \ctikz{
    % \boundingBox[16][12][0.5pt][1][(-8,-3)]
    \fill[thick,color=gradeColor,fill=gradeColor,fill opacity=0.10] (-2,3) -- (0,3) -- (0,-3) -- (-8,-3) -- (-8,9) -- (4,9) -- (4,7) -- (-6,7) -- (-6,-1) -- (-2,-1) -- cycle;
    \fill[thick,color=gradeColor,fill=gradeColor,fill opacity=0.10] (3,-3) -- (3,-1) -- (6,-1) -- (6,2) -- (8,2) -- (8,-3) -- cycle;
    \drawPoint{E}{-4}{0}
    \drawPoint{S}{4}{5}
    \draw [thick] (-8,9) -- (8,9) -- (8,-3) -- (-8,-3) -- (-8,9);
    \draw [thick,gradeColor] (-2,3) -- (0,3);
    \draw [thick,gradeColor] (0,3) -- (0,-3);
    \draw [thick,gradeColor] (0,-3) -- (-8,-3);
    \draw [thick,gradeColor] (-8,9) -- (4,9);
    \draw [thick,gradeColor] (4,9) -- (4,7);
    \draw [thick,gradeColor] (4,7) -- (-6,7);
    \draw [thick,gradeColor] (-6,7) -- (-6,-1);
    \draw [thick,gradeColor] (-6,-1) -- (-2,-1);
    \draw [thick,gradeColor] (-2,-1) -- (-2,3);
    \draw [thick,gradeColor] (3,-3) -- (3,-1);
    \draw [thick,gradeColor] (3,-1) -- (6,-1);
    \draw [thick,gradeColor] (6,-1) -- (6,2);
    \draw [thick,gradeColor] (6,2) -- (8,2);
    \draw [thick,gradeColor] (8,2) -- (8,-3);
    \draw [thick,gradeColor] (8,-3) -- (3,-3);
    \nswr[0]{%
        \draw [thick] (4,5)-- (-2,3)-- (-4,0);
        \draw [](-1.56,4.35) node[anchor=north west] {9.93};
    }
    \draw [thick] (-8,9) -- (8,9) -- (8,-3) -- (-8,-3) -- (-8,9);
}
        \ctikz{
    % \boundingBox[16][12][0.5pt][1][(-8,-3)]
    \fill[thick,color=gradeColor,fill=gradeColor,fill opacity=0.10] (-8,9) -- (-7,9) -- (-7,-3) -- (-8,-3) -- cycle;
    \fill[thick,color=gradeColor,fill=gradeColor,fill opacity=0.10] (-4,7) -- (-4,1) -- (-2,1) -- (-2,3) -- (1,3) -- (1,0) -- (3,0) -- (3,5) -- (-2,5) -- (-2,7) -- cycle;
    \fill[thick,color=gradeColor,fill=gradeColor,fill opacity=0.10] (2,9) -- (2,8) -- (6,8) -- (6,4) -- (8,4) -- (8,9) -- cycle;
    \fill[thick,color=gradeColor,fill=gradeColor,fill opacity=0.10] (8,0) -- (6,0) -- (6,-2) -- (5,-2) -- (5,-3) -- (8,-3) -- cycle;
    \fill[thick,color=gradeColor,fill=gradeColor,fill opacity=0.10] (-4,-1) -- (-4,-3) -- (-2,-3) -- (-2,-1) -- cycle;
    \drawPoint{E}{-6}{5}
    \drawPoint{S}{6}{2}
    \draw [thick,gradeColor] (-8,9) -- (-7,9);
    \draw [thick,gradeColor] (-7,9) -- (-7,-3);
    \draw [thick,gradeColor] (-7,-3) -- (-8,-3);
    \draw [thick,gradeColor] (-4,7) -- (-4,1);
    \draw [thick,gradeColor] (-4,1) -- (-2,1);
    \draw [thick,gradeColor] (-2,1) -- (-2,3);
    \draw [thick,gradeColor] (-2,3) -- (1,3);
    \draw [thick,gradeColor] (1,3) -- (1,0);
    \draw [thick,gradeColor] (1,0) -- (3,0);
    \draw [thick,gradeColor] (3,0) -- (3,5);
    \draw [thick,gradeColor] (3,5) -- (-2,5);
    \draw [thick,gradeColor] (-2,5) -- (-2,7);
    \draw [thick,gradeColor] (-2,7) -- (-4,7);
    \draw [thick,gradeColor] (2,9) -- (2,8);
    \draw [thick,gradeColor] (2,8) -- (6,8);
    \draw [thick,gradeColor] (6,8) -- (6,4);
    \draw [thick,gradeColor] (6,4) -- (8,4);
    \draw [thick,gradeColor] (8,4) -- (8,9);
    \draw [thick,gradeColor] (8,9) -- (2,9);
    \draw [thick,gradeColor] (8,0) -- (6,0);
    \draw [thick,gradeColor] (6,0) -- (6,-2);
    \draw [thick,gradeColor] (6,-2) -- (5,-2);
    \draw [thick,gradeColor] (5,-2) -- (5,-3);
    \draw [thick,gradeColor] (5,-3) -- (8,-3);
    \draw [thick,gradeColor] (8,-3) -- (8,0);
    \draw [thick,gradeColor] (-4,-1) -- (-4,-3);
    \draw [thick,gradeColor] (-4,-3) -- (-2,-3);
    \draw [thick,gradeColor] (-2,-3) -- (-2,-1);
    \draw [thick,gradeColor] (-2,-1) -- (-4,-1);
    \nswr[0]{%
        \draw [thick] (6,2)-- (3,5)-- (-2,7)-- (-4,7)-- (-6,5);
        \draw [thick] (6,2)-- (3,0)-- (1,0)-- (-4,1)-- (-6,5);
        \draw [](0.68,6.48) node[anchor=north west] {\np{14.46}};
        \draw [](-1.18,-0.15) node[anchor=north west] {\np{15.18}};
    }
    \draw [thick] (-8,9) -- (8,9) -- (8,-3) -- (-8,-3) -- (-8,9);
}
        \ctikz{
    % \boundingBox[16][12][0.5pt][1][(-8,-3)]
    \fill[thick,color=gradeColor,fill=gradeColor,fill opacity=0.10] (1,3) -- (1,6) -- (4,6) -- (4,4) -- (3,4) -- (3,0) -- (2,0) -- cycle;
    \fill[thick,color=gradeColor,fill=gradeColor,fill opacity=0.10] (-8,-2) -- (-2,-2) -- (-2,-3) -- (-8,-3) -- cycle;
    \fill[thick,color=gradeColor,fill=gradeColor,fill opacity=0.10] (8,5) -- (7,5) -- (7,8) -- (-7,8) -- (-7,2) -- (-8,2) -- (-8,9) -- (8,9) -- cycle;
    \fill[thick,color=gradeColor,fill=gradeColor,fill opacity=0.10] (2,-2) -- (5,-2) -- (5,0) -- (8,0) -- (8,-3) -- (2,-3) -- cycle;
    \fill[thick,color=gradeColor,fill=gradeColor,fill opacity=0.10] (-3,5) -- (-1,5) -- (-1,3) -- (-2,0) -- (-3,0) -- cycle;
    \drawPoint{E}{-6}{7}
    \drawPoint{S}{6}{1}
    \draw [thick,gradeColor] (1,3) -- (1,6);
    \draw [thick,gradeColor] (1,6) -- (4,6);
    \draw [thick,gradeColor] (4,6) -- (4,4);
    \draw [thick,gradeColor] (4,4) -- (3,4);
    \draw [thick,gradeColor] (3,4) -- (3,0);
    \draw [thick,gradeColor] (3,0) -- (2,0);
    \draw [thick,gradeColor] (2,0) -- (1,3);
    \draw [thick,gradeColor] (-8,-2) -- (-2,-2);
    \draw [thick,gradeColor] (-2,-2) -- (-2,-3);
    \draw [thick,gradeColor] (-2,-3) -- (-8,-3);
    \draw [thick,gradeColor] (-8,-3) -- (-8,-2);
    \draw [thick,gradeColor] (8,5) -- (7,5);
    \draw [thick,gradeColor] (7,5) -- (7,8);
    \draw [thick,gradeColor] (7,8) -- (-7,8);
    \draw [thick,gradeColor] (-7,8) -- (-7,2);
    \draw [thick,gradeColor] (-7,2) -- (-8,2);
    \draw [thick,gradeColor] (-8,2) -- (-8,9);
    \draw [thick,gradeColor] (8,9) -- (8,5);
    \draw [thick,gradeColor] (2,-2) -- (5,-2);
    \draw [thick,gradeColor] (5,-2) -- (5,0);
    \draw [thick,gradeColor] (5,0) -- (8,0);
    \draw [thick,gradeColor] (8,0) -- (8,-3);
    \draw [thick,gradeColor] (8,-3) -- (2,-3);
    \draw [thick,gradeColor] (2,-3) -- (2,-2);
    \draw [thick,gradeColor] (-3,5) -- (-1,5);
    \draw [thick,gradeColor] (-1,5) -- (-1,3);
    \draw [thick,gradeColor] (-1,3) -- (-2,0);
    \draw [thick,gradeColor] (-2,0) -- (-3,0);
    \draw [thick,gradeColor] (-3,0) -- (-3,5);
    \nswr[0]{%
        \draw [thick] (6,1)-- (4,6)-- (-6,7);
        \draw [thick] (6,1)-- (3,0)-- (2,0)-- (-1,5)-- (-6,7);
        \draw [thick] (6,1)-- (3,0)-- (-3,0)-- (-6,7);
        \draw [](-0.40,7.13) node[anchor=north west] {\np{15.44}};
        \draw [](-0.72,2.45) node[anchor=north west] {\np{15.38}};
        \draw [](-0.85,-0.24) node[anchor=north west] {\np{16.78}};
    }
    \draw[thick] (-8,9) -- (8,9) -- (8,-3) -- (-8,-3) -- cycle;
}
    \end{center}
}
}

\slide{cr}{
    \sseq
    \section{Inégalité triangulaire}
    \pr{Inégalité triangulaire}{
    Dans un triangle,
    la longueur d'un côté est inférieure à la somme des longueurs des deux autres côtés.
}[\wiki{Inégalité_triangulaire}]
    \rmk{}{
    Dans un triangle $ABC$, on peut écrire 3 inégalités triangulaires :
    \multiColItemize{3}{
        \item $\nswr{AB} < \nswr{AC} + \nswr{BC}$
        \item $\nswr{AC} < \nswr{AB} + \nswr{BC}$
        \item $\nswr{BC} < \nswr{AB} + \nswr{AC}$
    }
}[\ym]
    \rmk{Égalité «triangulaire»}{
    Lorsque l'égalité est atteinte dans l'inégalité triangulaire, on obtient un «triangle plat».
    En réalité, dans ce cas, les trois points sont alignés et il ne s'agit pas d'un véritable triangle.
}
}

\scn{Constructibilité de triangles}
\caSlide{10-11-12}

\slide{exo}{
    \act{Triangles constructibles}{
    \begin{enumerate}
        \item Tracez les triangles $ABC$ si possible avec les longueurs suivantes :
        \begin{enumerate}
            \item $AB = \Lg{6}$, $AC = \Lg{4}$ et $BC = \Lg{3}$.
            \item $AB = \Lg{6}$, $AC = \Lg{3}$ et $BC = \Lg{2}$.
            \item $AB = \Lg{6}$, $AC = \Lg{4}$ et $BC = \Lg{2}$.
        \end{enumerate}
        \item Pour chaque tentative, vérifiez si l'inégalité triangulaire $AB < AC + BC$ est respectée.
    \end{enumerate}
}[\href{http://blog.ac-versailles.fr/mathsleray/public/2019-2020/Cours_5eme/Chapitre_2/decouverte_ineg_triangulaire.pdf}
{Mathsleray}]
}

\slide{cr}{
    \cor{Triangle constructible}{
    Un triangle est \key{constructible}
    \ssi la longueur de son côté le plus long est inférieure à la somme des longueurs des deux autres côtés.
}[\wiki{Triangle_constructible}]
}

\scn{Problèmes sur la somme des angles d'un triangle}
\caSlide{13-14-15}

\slide{exo}{
    \exo{Les angles d'un triangle et leurs ratios}{
    \begin{enumerate}
        \item Quel est le ratio des angles d'un triangle équilatéral ?
        \item Quelle est la nature d'un triangle dont les angles sont dans le ratio 1:2:3 ?
        \item Un triangle isocèle peut-il avoir des angles dans le ratio 2:2:7 ?
    \end{enumerate}
}[\rpmc[66]]
}

\scn{Découvrir les médiatrices}
\caSlide{16-17-18}

\slide{exo}{
    \act{Solid Snake - Sortie Delta}{
    L'agent Solid Snake se trouve dans une base ennemie et doit rejoindre les escaliers de secours le plus rapidement possible.
    Aide-le en coloriant en \textcolor{Red}{rouge} la zone la plus proche des escaliers \textcolor{Red}{$A$ (Alpha)},
    en \textcolor{Blue}{bleu} celle des escaliers \textcolor{Blue}{$B$ (Beta)},
    en \textcolor{Green}{vert} celle des escaliers \textcolor{Green}{$\Gamma$ (Gamma)},
    et en \textcolor{Gold}{jaune} celle des escaliers \textcolor{Gold}{$\Delta$ (Delta)}.
    \ctikz{
    % \boundingBox[16][12][0.5pt][1][(-8,-3)]
    \draw[thick] (-8,9) -- (8,9) -- (8,-3) -- (-8,-3) -- cycle;
    \drawPoint{B}{-6}{5}
    \drawPoint{A}{6}{2}
    % \draw [thick,gradeColor] (-1.63,-3) -- (8,-3);
    % \draw [thick,gradeColor] (8,9) -- (1.38,9);
    % \draw [thick,gradeColor] (1.38,9) -- (-1.63,-3);
    % \draw [thick,gradeColor] (-8,-3) -- (-1.63,-3);
    % \draw [thick,gradeColor] (1.38,9) -- (-8,9);
    \nswr[0]{%
        \fill[thick,color=gradeColor,fill=Red,fill opacity=0.10] (-1.63,-3) -- (8,-3) -- (8,9) -- (1.38,9) -- cycle;
        \fill[thick,color=gradeColor,fill=Blue,fill opacity=0.10] (-8,-3) -- (-1.63,-3) -- (1.38,9) -- (-8,9) -- cycle;
        \draw[thick,answer,fill opacity=0.10] (0.55,3.36) -- (0.68,3.91) -- (0.14,4.05) -- (0,3.50) -- cycle; 
        \draw [thick,dashed] (-6,5) -- (6,2);
    }
}
    \ctikz{
    % \boundingBox[16][12][0.5pt][1][(-8,-3)]
    \draw[thick] (-8,9) -- (8,9) -- (8,-3) -- (-8,-3) -- cycle;
    \drawPoint{B}{-4}{6}
    \drawPoint{A}{6}{1}
    \drawPoint{$\Gamma$}{-5.15}{-1.05}
    % \draw [thick,gradeColor] (-8,3.03) -- (-8,9);
    % \draw [thick,gradeColor] (-8,9) -- (3.75,9);
    % \draw [thick,gradeColor] (3.75,9) -- (0.11,1.71);
    % \draw [thick,gradeColor] (-8,3.03) -- (-8,-3);
    % \draw [thick,gradeColor] (-8,-3) -- (0.97,-3);
    % \draw [thick,gradeColor] (0.97,-3) -- (0.11,1.71);
    % \draw [thick,gradeColor] (0.97,-3) -- (8,-3);
    % \draw [thick,gradeColor] (8,9) -- (3.75,9);
    \nswr[0]{%
        \fill[thick,color=gradeColor,fill=Red,fill opacity=0.10] (3.75,9) -- (0.11,1.71) -- (0.97,-3) -- (8,-3) -- (8,9) -- cycle;
        \fill[thick,color=gradeColor,fill=Blue,fill opacity=0.10] (-8,3.03) -- (-8,9) -- (3.75,9) -- (0.11,1.71) -- cycle;
        \fill[thick,color=gradeColor,fill=Green,fill opacity=0.10] (-8,3.03) -- (-8,-3) -- (0.97,-3) -- (0.11,1.71) -- cycle;
        \draw[thick,answer,fill opacity=0.10] (1.51,3.25) -- (1.76,3.75) -- (1.25,4.01) -- (1,3.50) -- cycle; 
        \draw[thick,answer,fill opacity=0.10] (-4.67,1.92) -- (-4.11,1.83) -- (-4.02,2.38) -- (-4.58,2.47) -- cycle; 
        \draw[thick,answer,fill opacity=0.10] (0.98,0.08) -- (0.88,0.63) -- (0.32,0.53) -- (0.42,-0.03) -- cycle; 
        \draw [thick,dashed] (-4,6) -- (6,1);
        \draw [thick,dashed] (-4,6) -- (-5.15,-1.05);
        \draw [thick,dashed] (-5.15,-1.05) -- (6,1);
        \draw [thick] (0.11,1.71) -- (-8,3.03);
    }
}
    \ctikz{
    % \boundingBox[16][12][0.5pt][1][(-8,-3)]
    \draw[thick] (-8,9) -- (8,9) -- (8,-3) -- (-8,-3) -- cycle;
    \drawPoint{A}{2}{3}
    \drawPoint{B}{2}{-2}
    \drawPoint{$\Gamma$}{-7}{0}
    \drawPoint{$\Delta$}{-5}{7}
    % \draw [thick,gradeColor] (8,0.50) -- (8,-3);
    % \draw [thick,gradeColor] (8,-3) -- (-2.94,-3);
    % \draw [thick,gradeColor] (-2.94,-3) -- (-2.17,0.50);
    % \draw [thick,gradeColor] (-2.17,0.50) -- (8,0.50);
    % \draw [thick,gradeColor] (-8,9) -- (0.79,9);
    % \draw [thick,gradeColor] (0.79,9) -- (-2.87,2.61);
    % \draw [thick,gradeColor] (-2.87,2.61) -- (-8,4.07);
    % \draw [thick,gradeColor] (-8,4.07) -- (-8,9);
    % \draw [thick,gradeColor] (-8,-3) -- (-8,4.07);
    % \draw [thick,gradeColor] (-2.87,2.61) -- (-2.17,0.50);
    % \draw [thick,gradeColor] (-2.94,-3) -- (-8,-3);
    % \draw [thick,gradeColor] (8,9) -- (0.79,9);
    % \draw [thick,gradeColor] (8,0.50) -- (8,9);
    \nswr[0]{%
        \fill[thick,color=gradeColor,fill=Blue,fill opacity=0.10] (8,0.50) -- (8,-3) -- (-2.94,-3) -- (-2.17,0.50) -- cycle;
        \fill[thick,color=gradeColor,fill=Gold,fill opacity=0.10] (-8,9) -- (0.79,9) -- (-2.87,2.61) -- (-8,4.07) -- cycle;
        \fill[thick,color=gradeColor,fill=Red,fill opacity=0.10] (-8,-3) -- (-8,4.07) -- (-2.87,2.61) -- (-2.17,0.50) -- (-2.94,-3) -- cycle;
        \fill[thick,color=gradeColor,fill=Green,fill opacity=0.10] (8,9) -- (0.79,9) -- (-2.87,2.61) -- (-2.17,0.50) -- (8,0.50) -- cycle;
        \draw[thick,answer,fill opacity=0.10] (-1.96,1.68) -- (-2.14,2.21) -- (-2.68,2.04) -- (-2.50,1.50) -- cycle; 
        \draw[thick,answer,fill opacity=0.10] (2,-0.06) -- (2.56,-0.06) -- (2.56,0.50) -- (2,0.50) -- cycle; 
        \draw[thick,answer,fill opacity=0.10] (-5.84,4.04) -- (-6.39,4.20) -- (-6.54,3.66) -- (-6,3.50) -- cycle; 
        \draw[thick,answer,fill opacity=0.10] (-1.95,-1.12) -- (-1.83,-0.57) -- (-2.38,-0.45) -- (-2.50,-1) -- cycle; 
        \draw[thick,answer,fill opacity=0.10] (-1.78,4.51) -- (-1.29,4.23) -- (-1.01,4.72) -- (-1.50,5) -- cycle; 
        \draw [thick,dashed] (-5,7) -- (2,3);
        \draw [thick,dashed] (2,3) -- (2,-2);
        \draw [thick,dashed] (-7,0) -- (2,-2);
        \draw [thick,dashed] (-7,0) -- (-5,7);
        \draw [thick,dashed] (-5,7) -- (2,-2);
        \draw [thick,dashed] (-7,0) -- (2,3);
    }
}
}
}

\slide{cr}{
    \section{Médiatrices}
    \df{}{
    On appelle \key{médiatrice} d'un segment, l'ensemble des points situés à égale distance des deux extrémités de ce segment.
}[\wiki{Médiatrice}]
    \pr{}{
    La médiatrice d'un segment coupe perpendiculairement ce segment en son milieu.
}
}

\scn{Découvrir les médiatrices d'un triangle}

% \slide{exo}{
%     \act{Trois médiatrices}{
    Pour chacun de ces triangles $ABC$, tracer les médiatrices.
    Pour lequel obtient-on le triangle $JKL$ le plus petit ?
}[\rpmc[139]]
% }

\slide{cr}{
    \rmk{}{
    Un triangle possède \nswr{trois} médiatrices qui sont concourantes en un point appelé le \key{centre du cercle circonscrit} au triangle.
}[][\cmdGeoGebra[fjehbrn3]]
}

\caSlide{19-20}



\scn{Tracer les hauteurs d'un triangle}
\caSlide{21-22-23}

\slide{cr}{
    \section{Aire du triangle}
    \df{}{
    On appelle \key{hauteur} d'un triangle,
    \key{une droite} passant par un sommet et \key{coupant perpendiculairement} la droite porté par le côté opposé à ce sommet.
}[\wiki{Hauteur_d'un_triangle}]
    \rmk{}{
    Un triangle possède \nswr{trois} hauteurs qui sont concourantes en un point appelé \key{orthocentre}.
}[][\cmdGeoGebra[fjehbrn3]]
}

\slide{exo}{
    \act{Aire du triangle}{
    \begin{enumerate}
        \item Construisez un triangle $ABC$ tel que :
        \begin{itemize}
            \item $AB = \Lg{6}$.
            \item La hauteur issue de $C$ coupe le segment $[AB]$ en un point $H$.
            \item $CH = \Lg{4}$.
        \end{itemize}
        \item Placez les points $E$ et $F$ de sorte que $AHCE$ et $BHCF$ soient des rectangles.
        \item \begin{enumerate}
            \item Trouvez l'aire du rectangle $AEFB$.
            \item Donnez une formule pour calculer l'aire de $AEFB$ en fonction des longueurs des segments $AB$ et $HC$.
        \end{enumerate}
        \item \begin{enumerate} 
            \item Comparez l'aire du rectangle $AHCE$ avec celle du triangle $AHC$.
            \item Faites de même pour $BHCF$ et $BHC$.
            \item En déduisez une comparaison des aires du rectangle $AEFB$ et du triangle $ABC$.
        \end{enumerate}
        \item \begin{enumerate}
            \item Quelle est l'aire du triangle $ABC$ ?
            \item Donnez une formule pour calculer l'aire de $ABC$ en fonction des longueurs des segments $AB$ et $HC$.
        \end{enumerate}
    \end{enumerate}
}
}

\slide{cr}{
    \pr{}{
    \dividePage{\ctikz{
    % \boundingBox[7.04][5.08][0.5pt][1][(-3.79,-1.98)]
    \draw[thick,color=gradeColor,fill=gradeColor,fill opacity=0.10] (1.01,-0.84) -- (0.64,-0.63) -- (0.43,-1.00) -- (0.80,-1.21) -- cycle; 
    \draw[color=Red] (-1.11,-0.71) node {base};
    \draw[color=Green] (1.0,1.15) node {hauteur};
    \draw [thick,Red] (-3.79,1.40) -- (2.15,-1.98);
    \draw [thick] (2.15,-1.98) -- (3.25,3.10) -- (-3.79,1.40);
    \draw [thick,Green] (0.80,-1.21) -- (3.25,3.10);
    \nswr[0]{%
    }
}}{
        \begin{align*}
            \textrm{Aire du triangle} = \dfrac{\textrm{\textcolor{Red}{base}} \times \textrm{\textcolor{Green}{hauteur}}}{2}
        \end{align*}
    }
}
}

\slide{qf}{
    \exo{QCM - Angles d'un triangle dans un ratio}{
    Les angles d'un triangle sont dans le ratio 1:1:2.
    Parmi les affirmations suivantes, laquelle (lesquelles) est (sont) fausse(s) ?
    \begin{enumerate}
        \item  C'est un triangle rectangle.
        \item  C'est un triangle isocèle.
        \item  C'est un triangle quelconque.
        \item  C'est un triangle équilatéral.
    \end{enumerate}
}[\rpmc[66]]
}

\scn{Calculer l'aire d'un triangle}
\caSlide{24-25-26}

\slide{exo}{
    \exo{Aires et médianes}{
    Considérons un triangle $ABC$ et notons $M$ le milieu du segment $[AB]$.
    \begin{enumerate}
        \item Dessinez cette figure à main levée.
        \item Comparez les aires des triangles $ABC$ et $AMC$.
    \end{enumerate}    
}[\rpmc[142]]
}

\caSlide{27-28}
\caSlide{29-30}